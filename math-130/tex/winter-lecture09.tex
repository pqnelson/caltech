%%
%% winter-lecture09.tex
%% 
%% Made by Alex Nelson <pqnelson@gmail.com>
%% Login   <alex@lisp>
%% 
%% Started on  2026-01-27T11:11:15-0800
%% Last update 2026-01-27T11:11:15-0800
%% 

\lecture{}

\begin{theorem}
Let $X$ be a quasiseparated quasicompact scheme.
Let $\sheaf{F}$ be a quasicoherent sheaf on $X$. Then for any
$f\in\sheaf{O}_{X}(X)$, the natural map $\sheaf{F}(X)_{f}\to\sheaf{F}(X_{f})$
is an isomorphism.
\end{theorem}
(We ended last lecture with this claim.)

\begin{corollary}
For any scheme $X$, for any affine open subset $U\subset X$ with $U\iso\Spec(A)$,
if $\sheaf{F}$ is a quasicoherent sheaf on $X$, then
$\sheaf{F}|_{U}\iso\widetilde{M}$ for some $A$-module $M$.
\end{corollary}

\begin{proposition}
Let $\varphi\colon X\to Y$ be a scheme morphism. The following holds:
\begin{enumerate}
\item If $\sheaf{G}$ is a quasi-coherent sheaf on $Y$, then
  $\varphi^{-1}\sheaf{G}$ is a quasi-coherent sheaf on $X$.
\item If $\varphi$ is quasi-compact and quasi-separated and if
  $\sheaf{F}$ is a quasi-coherent sheaf on $X$, then
  $\varphi_{*}\sheaf{F}$ is quasi-coherent on $Y$.
\end{enumerate}
\end{proposition}

\begin{warning}
Although pushforwards of quasi-coherent sheafs are quasi-coherent, it
is not true for coherent sheafs. The example to bear in minx:
$X=\AA^{1}_{k}\xrightarrow{\pi}\Spec(k)$, then
$\pi_{*}\sheaf{O}_{X}=k[x]$ since $\sheaf{O}_{X}=\widetilde{k[x]}$.
\end{warning}

\subsection{Closed Subschemes}

\begin{node}
The idea is if we look at $R=k[x_{1},\dots,x_{n}]$ and $\AA^{n}_{k}\iso\Spec(R)$,
then for any ideal $I\ideal R$ we have $\Spec(R/I)\into\Spec(R)$ be a
homeomorphism onto a closed subset. We can identify closed subschemes
with this definition, modulo some subtle issues.
\end{node}

\begin{definition}
A morphism of schemes $\varphi\colon Z\to X$ is called a
\define{Closed Embedding} (or a \define{Closed Immersion}) if
$\varphi$ is a homeomorphism onto a closed subset (its image) and
$\varphi^{\sharp}\colon\sheaf{O}_{X}\to\varphi_{*}\sheaf{O}_{Z}$ is surjective.
\end{definition}

\begin{definition}
Let $X$ be a scheme. A  \define{Closed Subscheme} of $X$ is an
equivalence class of closed embeddings
\begin{equation}
\vcenter{\xymatrix{Z\ar[rd]^{\varphi}\ar[dd]_{\iso} &\\ 
& X \\
Z'\ar[ru]_{\varphi'} &}}
\end{equation}
\end{definition}

\begin{observation}
If $\varphi\colon Z\to X$ is a closed embedding an $U\subset X$ is an
affine open, then $\varphi^{-1}(U)\to U$ is a closed embedding. (This
leads us to believe we can study closed subschemes locally.)
\end{observation}

\begin{proposition}
If $Z\subset X\iso\Spec(R)$ is a closed subscheme, then
$Z=Z(I)\iso\Spec(R/I)$ for some ideal $I\ideal R$.
\end{proposition}

So there are no ``sneaky'' closed subschemes of affine schemes not
captured by the motivating construction.

\begin{proposition}
For any $X$, closed subschemes of $X$ are ``the same as'' (bijective with)
quasi-coherent sheaves of ideals.
\end{proposition}

\begin{definition}
Let $X$ be a scheme. A \define{Open Immersion} (or \define{Open Embedding})
is a scheme morphism $i\colon U\to X$ which is an isomorphism onto an
open subset.
\end{definition}

\begin{definition}
Let $X$ be a scheme, let $W\subset\underlyingspace(X)$ be a closed subset of the
underlying topological space of $X$. Then there is a ``smallest''
closed subscheme supported on $W$ called the \define{Induced Reduced
  Closed Subscheme Structure} on $W$.
\end{definition}