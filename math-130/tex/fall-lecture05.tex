%%
%% fall-lecture05.tex
%% 
%% Made by Alex Nelson <pqnelson@gmail.com>
%% Login   <alex@lisp>
%% 
%% Started on  2025-10-09T07:51:58-0700
%% Last update 2025-10-09T07:51:58-0700
%% 

\lecture{}

\begin{lemma}[Invariance of dimension under integral extensions]% Gathmann, Commutative Algebra, Lemma 11.8
Let $R\subset R'$ be an integral ring extension.
Then they have the same Krull dimensions (\S\ref{defn:fall-lecture03:krull-dimension}) $\dim(R)=\dim(R')$.
\end{lemma}

\begin{proof}
We will prove $\dim(R)\leq\dim(R')$ and $\dim(R')\leq\dim(R)$.

\textsc{Claim 1} $\dim(R)\leq\dim(R')$: if we have some chain of prime
ideals in $R$, $P_{0}\propersubset P_{1}\propersubset\cdots\propersubset P_{n}$.
Then using Lying Over (\S\ref{prop:fall-lec04:lying-over}) and Going Up (\S\ref{prop:fall-lec04:going-up}) gives us a chain of prime
ideals in $R'$, $P'_{0}\propersubset P'_{1}\propersubset\cdots\propersubset P'_{r}$
which proves the claim.

\textsc{Claim 2} $\dim(R')\leq\dim(R)$; if we have a chain of prime
ideals in $R'$, $P'_{0}\propersubset P'_{1}\propersubset\cdots\propersubset P'_{r}$,
then intersecting with $R$ gives us
$P_{0}\propersubset P_{1}\propersubset\cdots\propersubset P_{r}$ (the
proper inclusion is due to Incompatibility (\S\ref{prop:fall-lec04:incompatibility})) where $P_{j}=R\cap P'_{j}$.
Hence the claim.
\end{proof}

\begin{proposition}[Dimension of polynomial rings]% Gathmann, Commutative Algebra, Proposition 11.9
Let $\kk$ be a field, let $n\in\NN_{0}$ be a natural number.
\begin{enumerate}
\item $\dim(\kk[x_{1},\dots,x_{n}])=n$
\item All maximal chains of prime ideals in $\kk[x_{1},\dots,x_{n}]$
  have length $n$.
\end{enumerate}
\end{proposition}

\begin{proof}
We will prove both statements by induction on $n$.

\textsc{Base case} $(n=0)$: obvious.

\textsc{Inductive case} $(n\geq1)$: let $P_{0}\propersubset\cdots\propersubset P_{m}$
be a chain of prime ideal in $\kk[x_{1},\dots,x_{n}]$. We will show
$m\leq n$, and $m=n$ always for maximal chains of prime ideals.

We may assume without loss of generality that $P_{0}=0$, $P_{1}$ is a
minimal non-zero prime ideal, and that $P_{m}$ is a maximal
ideal. (This may require extending the chain of ideals.) Then $P_{1}=(f)$
for some monic non-zero polynomial $f$, since $\kk[x_{1},\dots,x_{n}]$
is a unique factorization domain.

Then $\kk[x_{1},\dots,x_{n}]/P_{1}$ is integral over $\kk[x_{1},\dots,x_{n-1}]$.
We have
\begin{equation*}
\begin{array}{lccccc}
\kk[x_{1},\dots,x_{n}]\qquad      & P_{1} & \propersubset & \cdots & \propersubset & P_{m}\\
                                 & \downarrow \hbox to0pt{extension} & &       &               & \downarrow \hbox to0pt{extension} \\
\kk[x_{1},\dots,x_{n}]/P_{1}\qquad& P_{1}/P_{1} & \propersubset & \cdots & \propersubset & P_{m}/P_{1}\\
                                 & \downarrow \hbox to0pt{contraction} & &       &               & \downarrow \hbox to0pt{contraction} \\
\kk[x_{1},\dots,x_{n-1}]\qquad    & (P_{1}/P_{1})\cap\kk[x_{1},\dots,x_{n-1}]& \propersubset & \cdots & \propersubset & (P_{m}/P_{1})\cap\kk[x_{1},\dots,x_{n-1}]
\end{array}
\end{equation*}
We claim both these steps preserve prime ideals and their strict
inclusions, and transfer maximal chains to maximal chains.

The claim ``the chain $Q_{1}\propersubset\cdots\propersubset Q_{m}$ where $Q_{j}=(P_{j}/P_{1})\cap\kk[x_{1},\dots,x_{n-1}]$
is maximal'' will suffice for concluding the proof thanks to the
inductive hypothesis.

In the bottom row, if it were a maximal chain of ideals, then $Q_{1}=0$ and $Q_{m}$ is a maximal ideal, and if we
could insert another prime ideal at any step then we could do so in
the middle chain as well.

Now by the induction hypothesis, the length $m-1$ of the bottom chain
is at most $n-1$ and equal to $n-1$ if the chain is maximal. Hence we
always have $m=n$ if the original chain was maximal.
\end{proof}

\begin{remark}
More generally, if $R\subset R'$ is an integral extension and $R$ is a
finitely-generated $\kk$-algebra, and we have a chain of prime ideals
$P_{1}\propersubset P_{2}\propersubset P_{3}$ of $R$ and we have
prime ideals of $R'$ 
($P'_{1}$ lying over $P_{1}$ and $P'_{3}$ lying over $P_{3}$) then
there exists some ideal $P'2$
\begin{equation*}
\begin{array}{lccccc}
R'\qquad & P'_{1} & {\color{red}\propersubset} & {\color{red}P'_{2}} & {\color{red}\propersubset} & P'_{3}\\
\;\rotatebox{90}{$\into$}\;\quad & \downarrow &   &   &   & \downarrow\\
R\qquad & P_{1} & \propersubset & P_{2} & \propersubset & P_{3}
\end{array}
\end{equation*}
Using Noether normalization (\S\ref{thm:fall-lec04:noether-normalization}) and Going Down (\S\ref{prop:fall-lec04:going-down}) gives us the
situations thus.
\end{remark}

\begin{corollary}
Let $R$ be any finitely-generated $\kk$-algebra. If $R$ is an integral
domain, then every maximal chain of prime ideals have the same length
(namely $\dim(R)$).
\end{corollary}

\begin{proposition}[Krull's Principal Ideal Theorem]\label{thm:pit}
Let $R$ be a Noetherian ring, let $a\in R$. Then every minimal prime
ideal $P\ideal R$ containing $a\in P$ has $\codim(R)\leq 1$.
\end{proposition}

\noindent(Geometrically: every connected component of a hypersurface has
codimension 1.)

\begin{proposition}
Let $X$ be a nonempty irreducible affine variety.
Let $0\neq f\in A(X)$. Then every irreducible component of $V(f)$ has
codimension 1.
\end{proposition}

\begin{definition} % Gathmann, Algebraic Geometry, 2.32
A Noetherian topological space $X$ is said to have \define{Pure Dimension}
$n$ if every irreducible component of $X$ is dimension $n$.
\end{definition}

\begin{question}
Is every irreducible hypersurface of a given irreducible variety $X$
the zero locus of a single polynomial?
\end{question}

\begin{proposition} % Gathmann, Algebraic Geometry, Prop 2.37
Let $R$ be a Noetherian integral domain. Then the following are equivalent:
\begin{enumerate}
\item Every prime ideal of codimension 1 in $R$ is irreducible.
\item $R$ is a unique factorization domain.
\end{enumerate}
\end{proposition}

\begin{proof}
$(\Longrightarrow)$ $R$ is Noetherian implies nonzero non-unit $f$ may
  be decomposed as the product of irreducible elements (otherwise we'd
  get an infinite chain). Suffices to show every irreducible element
  is prime. Let $f\in R$ be an irreducible element. Choose a minimal
  prime ideal $P\ideal R$ containing $f\in P$. Then applying Krull's
  principal ideal theorem~\ref{thm:pit}, $P=(g)$ for some prime element $g$. Then
  $g\divides f$ which implies $f$ is prime.
  
$(\Longleftarrow)$ Let $P\ideal R$ be a prime ideal of codimension 1.
Then we can choose a nonzero $f\in P$. Since $R$ is a unique
factorization domain, we can decompose $f$ into a product of prime elements
$f=f_{1}(\cdots)f_{k}$ where $f_{j}$ are all prime. Then $f_{j}\in P$
for some $j$, which implies $0\propersubset(f_{j})\subset P$, which
implies $(f_{j})=P$.
\end{proof}

\begin{remark}
Let $X$ be an irreducible hypersurface in $\AA^{n}$. Then
$I(X)=\langle f\rangle$ for some irreducible $f\in\kk[x_{1},\dots,x_{n}]$.
\end{remark}

\begin{definition} % Gathmann, Algebraic Geometry, Definition 2.39
The \define{Degree} of a hypersurface defined by $I(X)=\langle f\rangle$
is defined to be the degree of $f$.
\end{definition}

\subsection{Sheaf of Regular Functions}

What are the functions between affine varieties? We should aim for a
definition which makes sense on the affine variety $X$ which is its
domain, but also on open subsets $U\subset X$.

What are the functions on an open subset of an affine variety? Trying
to define functions on local covers won't work. We will instead have
functions which are locally fractions of polynomials. This is more
general than ``functions which are locally polynomials''.

\begin{definition} % Gathmann, Algebraic Geometry, 3.1
Let $X\subset\AA^{n}_{\kk}$ be an affine variety, let $U\subset X$ be an open subset.
A \define{Regular Function} on an open subset $U$ is a map
$\varphi\colon U\to\kk$ such that: For every $a\in U$ there are
polynomial functions $f,g\in A(X)$ such that
\begin{equation}
\varphi(x)=\frac{g(x)}{f(x)}
\end{equation}
for all $x\in U_{a}$ with $f(x)\neq0$ and where $U_{a}\subset U$ is an open neighborhood of
$a\in U_{a}$.

The set of all such regular functions on $U$ will be denoted by
$\RegularFuns_{X}(U)$. They form a ring.
\end{definition}

\begin{remark}
Why are these functions called ``regular''? Well, it's because there
are no singularities or singular behaviour: the function $\varphi$ is
regular if it is locally described in an open subset $U$ as a rational
function $g/f$ where $f(x)\neq0$ on $U$.
\end{remark}

\begin{example}\label{ex:fall-lecture05:non-extendible-fn}
Let $X=V(xw-yz)\subset\AA^{4}$. Consider the open subset $U=X\setminus V(y,w)\subset X$.
We define the function $\varphi\colon U\to\kk$ by
\begin{equation}
\varphi(x,y,z,w)=\begin{cases}\frac{x}{y} & \mbox{if }y\neq0\\
\frac{z}{w} & \mbox{if }w\neq0.
\end{cases}
\end{equation}
As Homework: prove there is no global expression as a quotient.
\end{example}

\begin{example}[{Osserman~\cite[3.1.2]{osserman2021concise}}]
If $f\in\kk[x_{1},\dots,x_{n}]$, then $f$ is a regular function on $Y$
for any algebraic set $Y=V(I)$. This example suggests we can inject
$A(Y)$ into $\RegularFuns(Y)$.
\end{example}

\begin{lemma}[Zero loci of regular functions are closed] % Gathmann, 3.4
Let $\varphi\in\RegularFuns_{X}(U)$. Then $V(\varphi)=\{x\in U\mid\varphi(x)=0\}\subset U$
is closed.
\end{lemma}

\begin{remark}[Identity theorem for regular functions]% Gathmann, 2.5
If $U\propersubset V$ are both open, nonempty subsets of an
irreducible affine variety $X$, and if
$\varphi_{1},\varphi_{2}\in\RegularFuns_{X}(V)$, if
$\varphi_{1}|_{U}=\varphi_{2}|_{U}$, then $\varphi_{1}=\varphi_{2}$.
\end{remark}

\begin{lemma}[{Osserman~\cite[3.1.4]{osserman2021concise}}]
A regular function $\varphi\colon U\to\kk$ defines a continuous function $U\to\AA^{1}$.
\end{lemma}

\begin{proof}[Proof sketch]
It suffices to show that the preimage of a point in $\AA^{1}$ is
closed. Check $\varphi^{-1}(c)$ is closed using a cover of $P\in U$, let
$V\subset U$, and so $\varphi=g/f$ on $V$. Then on $V$, we see that
$\varphi^{-1}(c)$ is the zero set of
\begin{equation}
\frac{g}{f}-c=\frac{g-cf}{f}
\end{equation}
which is just $Z(g-cf)$ since $f$ is nonvanishing on $V$. Then
$\varphi^{-1}(c)$ is closed by the Zariski topology. Since the $V$
cover $U$, we conclude that the preimage of a point is closed.
\end{proof}