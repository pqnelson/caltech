%%
%% winter-lecture11.tex
%% 
%% Made by Alex Nelson <pqnelson@gmail.com>
%% Login   <alex@lisp>
%% 
%% Started on  2026-01-29T11:05:36-0800
%% Last update 2026-01-29T11:05:36-0800
%% 

\lecture{}

\begin{recall}
Let $(X,\sheaf{O}_{X})$ be a ringed space. A \define{Sheaf of Ideals}
on $X$ is a sheaf $\sheaf{I}$ of $\sheaf{O}_{X}$-modules which is also
a subsheaf of $\sheaf{O}_{X}$. In other words, for each open subset
$U\subset X$, $\sheaf{I}(U)$ is an ideal in $\sheaf{O}_{X}(U)$.
\end{recall}

\begin{definition}% Hatcher, Algebraic Geometry, II\S5, pg115
Let $Y$ be a closed subscheme of $X$. Let $i\colon Y\into X$ be its
inclusion morphism. We define the \define{Ideal Sheaf} of $Y$, denoted
$\sheaf{I}_{Y}$, to be the kernel of the morphism
$i^{\sharp}\colon\sheaf{O}_{X}\to i_{*}\sheaf{O}_{Y}$.

Note: Vakil~\cite[\S9.1.2]{vakil2025rising} uses the notation $\sheaf{I}_{Y/X}$
to stress this is using the closed subscheme $Y$ of $X$.
\end{definition}

\begin{proposition}
Let $Y$ be a closed subscheme of $X$, let $i\colon Y\into X$ be the
inclusion morphism. Then we have a short exact sequence of sheaves
\begin{equation}
0\to\sheaf{I}_{Y}\to\sheaf{O}_{X}\to i_{*}\sheaf{O}_{Y}\to0.
\end{equation}
\end{proposition}
This is loosely analogous to the situation where $R$ is a ring and
$I\ideal R$ is an ideal, we have $0\to I\to R\to R/I\to 0$ is an exact
sequence. 

\begin{proposition}
Let $Y$ be a closed subscheme of $X$, let $i\colon Y\into X$ be the
inclusion morphism. If $\sheaf{F}$ is a quasi-coherent sheaf on $Y$, then
$i_{*}\sheaf{F}$ is a quasi-coherent sheaf on $X$.
\end{proposition}

\begin{remark}
We can identify the category of quasi-coherent sheaves on $Y$ with the
sheaves on $X$ annihilated by $\sheaf{I}_{Y}$.
\end{remark}

\subsection{Projective Stuff}

\begin{convention}
Let $S$ be a graded ring $S=\bigoplus_{n\in\NN_{0}}S_{n}$, let $M$ be
a graded $S$-module, so
\begin{equation}
M=\bigoplus_{d\in\ZZ}M_{d},
\end{equation}
where $S_{n}M_{d}\subset M_{d+n}$. 
\end{convention}

\begin{node}
If $S$ is a graded ring and $M$ is a graded $S$-module, then we get a
quasi-coherent sheaf $\widetilde{M}$ on $\Proj(S)$ where $\Proj(S_)$
has an open cover $D_{+}(f)$ where each $f$ is a homoeneous element of
positive degree and $D_{+}(f)\iso\Spec(S_{(f)})$, then we see
$\widetilde{M}(D_{+}(f))=M_{(f)}$ is the degree-zero piece of the
usual localization $M_{f}$.
\end{node}

\begin{node}
If we take $S=A[x_{0},\dots,x_{n}]$, then $\Proj(S)=\PP^{n}_{A}$ and
we can look at $M=S(d)$ where we slide the grading by $d$, where the
conventions are $M(\ell)_{d}=M_{d+\ell}$.
\end{node}

\begin{node}
Associated to $S(d)$ is a sheaf
\begin{equation}
\widetilde{S(d)}:=\sheaf{O}_{X}(d).
\end{equation}
In this particular case,
\begin{equation}
\Gamma(X,\sheaf{O}_{X}(d))=A[x_{0},\dots,x_{n}]_{(d)}.
\end{equation}
Why? Well, for any $S$ and for any $M$, we get a natural map
\begin{equation}
M_{0}\to\Gamma(\Proj(S),\widetilde{M}),
\end{equation}
given by looking at $\widetilde{M}(D_{+}(f))=M_{(f)}$ is the degree
zero part. The special thing for $M=S(d)$, this natural map becomes an
isomorphism. 
\end{node}

\begin{node}
In classical algebraic geometry, it's more natural to work on
projective varieties instead of affine varieties. Then homogeneous
polynomials correspond to sections of a sheaf. We would define
subvarieties using homogeneous polynomials.

What do we do here? The idea is that $\sheaf{O}(d)$ is a
\emph{locally-free sheaf of rank 1}---so $X\iso\bigcup_{i}U_{i}$ where
$\sheaf{O}(d)|_{U_{i}}\iso\sheaf{O}_{X}|_{U_{i}}$---we also call
$\sheaf{O}(d)$ a \emph{line bundle} or \emph{invertible sheaf} (these
are all synonyms).
\end{node}

\begin{remark}
These $\sheaf{O}(m)$ are the only invertible sheaves on $\PP^{n}_{k}$.
(Hartshorne~\cite[Corollary 6.17]{hartshorne1977algebraic}, Vakil~\cite[\S15.1.3]{vakil2025rising})
\end{remark}

\begin{definition}
Let $\sheaf{L}$ be an invertible sheaf on a scheme $X$, let
$s\in\sheaf{L}(X)$ be a global section. This has an associated
\define{Zero Scheme} $Z(s)\subset X$ which is a closed subscheme.

(See also: \stackscite{02OQ}.)
\end{definition}

\begin{remark}
To check this notion of a ``zero scheme'' is well-defined, we need to
check it does not depend on the choice of isomorphisms
$\sheaf{O}(d)|_{U_{i}}\iso\sheaf{O}_{X}|_{U_{i}}$. 
\end{remark}

\begin{theorem}[{\cite[Prop~5.12(b)]{hartshorne1977algebraic}}]
Let $S$ be a graded ring, let $X=\Proj(S)$. Then $\sheaf{O}_{X}(n)\otimes\sheaf{O}_{X}(m)\iso\sheaf{O}_{X}(m+n)$.
\end{theorem}