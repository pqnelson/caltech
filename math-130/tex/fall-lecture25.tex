%%
%% fall-lecture25.tex
%% 
%% Made by Alex Nelson <pqnelson@gmail.com>
%% Login   <alex@lisp>
%% 
%% Started on  2025-11-25T11:19:14-0800
%% Last update 2025-11-25T11:19:14-0800
%% 

\lecture{}

\begin{convention}
In this lecture, we will work with $X$ being a normal variety.
\end{convention}

\begin{definition}
We define the \define{Divisor Group} of $X$, denoted $\Div(X)$, for the free Abelian group gneerated by prime
divisors, its elements are called \define{Weil Divisors} written
\begin{equation}
D=\sum_{\text{finite}}n_{i}[Y_{i}].
\end{equation}
\end{definition}

\begin{definition}
The \define{Principal Divisor} of $f\in\UnitGroup{\FunField[\kk]{X}}$
is defined as
\begin{equation}
(f):=\sum_{Y}v_{Y}(f)[Y]\in\Div(X).
\end{equation}
Note: $(f^{-1})=-(f)$ and $(fg)=(f)+(g)$, so $(-)\colon\UnitGroup{\FunField[\kk]{X}}\to\Div(X)$
is a group morphism.
\end{definition}

\begin{definition}
The \define{Divisor Class Group} is
\begin{equation*}
\Cl(X):=\Div(X)/\{(f)\mid f\in\UnitGroup{\FunField[\kk]{X}}\}.
\end{equation*}
\end{definition}

\begin{example}
We see $\Cl(\AA^{n})=0$ and $\Div(\AA^{n})=0$.
\end{example}

\begin{remark}
If $X$ is not normal, we can still define its class group $\Cl(X)$ by
using
\begin{equation}
v_{Y}(f/g)=\length_{\StructureSheaf_{X,Y}}(\StructureSheaf_{X,Y}/(f))-\length_{\StructureSheaf_{X,Y}}(\StructureSheaf_{X,Y}/(g)),
\end{equation}
for $f,g\in\StructureSheaf_{X,Y}$. This recovers the usual definition
for normal varieties.
\end{remark}

\subsection{Divisors on $\PP^{n}$}

\begin{definition}
We may define $\deg\colon\Div(\PP^{n})\to\ZZ$ by
\begin{equation}
\deg(\sum m_{i}[Y_{i}])=\sum m_{i}\deg(Y_{i}).
\end{equation}
\end{definition}

\begin{node}
If $f\in\in\UnitGroup{\FunField[\kk]{\PP^{n}}}$, then
$f=\prod^{r}_{i=0}h_{i}^{m_{i}}$ where
$h_{i}\in\kk[x_{0},\dots,x_{n}]$ irreducible where $\sum m_{i}\deg(h_{i})=0$. 
Then $Y_{i}:=V_{p}(h_{i})\subset\PP^{n}$ prime divisor $V_{Y_{i}}(f)=m_{i}$.
We have $(f)=\sum_{i}m_{i}[Y_{i}]$ implies $\deg(f)=0$. Hence
\begin{equation}
\deg\colon\Cl(\PP^{n})\to\ZZ.
\end{equation}
\end{node}

\begin{proposition}
$\Cl(\PP^{n})\iso\ZZ$.
\end{proposition}

\begin{proof}
Let $H\subset\PP^{n}$. Then $\deg([H])=1$, so $\deg\colon\Cl(\PP^{n})\to\ZZ$
is surjective.

To see this injective, let $D=\sum m_{i}[Y_{i}]$ with $\deg(D)=0$.
Then $Y_{i}=V_{p}(h_{i})$ where $h_{i}$ is irreducible. We see
\begin{equation}
\sum m_{i}\deg(h_{i})=\deg(D)=0,
\end{equation}
which implies
\begin{equation}
f=\prod h_{i}^{m_{i}}\in\UnitGroup{\FunField[\kk]{X}}.
\end{equation}
Then $D=(f)$, which implies $D=0\in\Cl(\PP^{n})$.
\end{proof}

\begin{remark}
Later we will see: if $X$ is nonsingular, then $\Cl(X)\iso\Pic(X)$.
\end{remark}

\begin{fact}[Commutative algebra]
If $R$ is a normal Noetherian domain, then $R=\bigcap_{\dim(R_{P})=1}R_{P}$.
\end{fact}

\begin{corollary}
If $X$ is normal and $f\in\UnitGroup{\FunField[\kk]{X}}$, then
$f\in\StructureSheaf_{X}(X)$ iff $v_{Y}(f)\geq0$ for every prime
divisor $Y\subset X$; i.e., the closed subset where $f$ is not defined
is either empty or has codimension 1.
\end{corollary}

\begin{lemma}\label{lemma:math130a:fall2025:lec25:lemma}
Let $R$ be a Noetherian domain.
Then $R$ is a UFD $\iff$ every height 1 prime is principal.
\end{lemma}

\begin{proposition}
Let $X$ be an irreducible affine variety.
Then $\CoordRing{\kk}{X}$ is a UFD $\iff$ $X$ is normal and $\Cl(X)=0$.
\end{proposition}

\begin{proof}
\forwardproof\ We proved any UFD is normal, so we want to show
$\Cl(X)=0$. If $Y\subset X$ is a prime divisor, then $\height(I(Y))=1$
implies $I(Y)=(f)$ is principal. Then by Lemma~\ref{lemma:math130a:fall2025:lec25:lemma},
$[Y]=(f)=0\in\Cl(X)$.

\backwardproof\ Let $P\in\Spec(\CoordRing{\kk}{X})$ be $\height(P)=1$
prime ideal. We want to show $P$ is principal. Let $Y=V(P)\subset X$
be a prime divisor. Then $[Y]=0\in\Cl(X)$ implies $[Y]=(f)$ for some
$f\in\UnitGroup{\FunField[\kk]{X}}$. Then $v_{Z}(f)\geq0$ for every
prime divisor $Z\subset X$. Then $f\in\CoordRing{\kk}{X}$ is regular.

We claim this $P=I(f)=\langle f\rangle\ideal\CoordRing{\kk}{X}$. Since
$f$ vanishes on $Y$, we have $P\supset\langle f\rangle$. To
demonstrate $P\subset\langle f\rangle$, let $g\in P$ be
arbitrary. Then $v_{Y}(g)\geq 1$ implies $v_{Z}(g/f)\geq0$ for any
prime divisor $Z\subset X$. Then $g/f=a\in\CoordRing{\kk}{X}$, which
implies $g=af$ and therefore $g\in\langle f\rangle$.
\end{proof}

\begin{proposition}
Let $X$ be normal, $Z\propersubset X$ a proper closet subset, and
$U=X\setminus Z$ its complement. Then:
\begin{enumerate}
\item $\Cl(X)\onto\Cl(U)$ is surjective defined by sending
  $[Y]\mapsto\begin{cases}[Y\cap U]&\mbox{if }Y\cap U\neq\emptyset\\0&\mbox{otherwise}\end{cases}$
\item If $\codim(Z;X)\geq2$, then $\Cl(X)=\Cl(U)$ are equal to each other.
\item If $Z$ is a prime divisor, then $\ZZ\to\Cl(X)\to\Cl(U)\to0$ is
  exact (where $\ZZ\to\Cl(X)$ is given by $m\mapsto m[Z]$).
\end{enumerate}
\end{proposition}

\subsection{Picard Group}\label{subsec:fall2025:lec25:picard-group}

We discussed the Picard group earlier
(\S\S\ref{defn:fall2025:lec22:picard-group} \textit{et seq.}).

\begin{node}
Let $X$ be any variety, let $\sheaf{L}_{1}$ and $\sheaf{L}_{2}$ be two
invertible $\StructureSheaf_{X}$-modules. Then
$\sheaf{L}_{1}\otimes_{\StructureSheaf_{X}}\sheaf{L}_{2}$ is an
invertible $\StructureSheaf_{X}$-module.

We also see that $\sheaf{L}^{-1}=[U\mapsto\hom_{\StructureSheaf_{U}}(\sheaf{L}|_{U},\StructureSheaf_{U})]$.
\end{node}

\begin{xca}
Prove $\sheaf{L}^{-1}$ is an invertible $\StructureSheaf_{X}$-module
and that $\sheaf{L}\otimes\sheaf{L}^{-1}\iso\StructureSheaf_{X}$.
\end{xca}

\begin{definition}
The \define{Picard Group} $\Pic(X)$ is the group consisting of isomorphism classes of
invertible $\StructureSheaf_{X}$-modules, and $\otimes$ is the binary operator.
\end{definition}

\begin{notation}
Assume $X$ is an irreducible variety, $\sheaf{L}$ is an invertible
$\StructureSheaf_{X}$-module. Let $s\in\sheaf{L}(U)$,
$t\in\sheaf{L}(V)$ be nonzero sections for open subsets $U,V\subset X$.
Take $W\subset U\cap V$ open affine such that $\sheaf|_{W}\iso\StructureSheaf_{W}$
generated by $u\in\sheaf{L}(W)$. Then $s|_{W}=f\cdot u$,
$t|_{W}=g\cdot u$ where $f,g\in\CoordRing{\kk}{W}$.
\end{notation}

\begin{definition}
Using the notation from the previous node, we define $s/t := f/g\in\FunField[\kk]{X}$,
and it is independent of the choices for $W$ and $u$.
\end{definition}

\begin{example}
Let $s_{0},\dots,s_{n}\in\Gamma(X,\sheaf{L})$ be nonzero sections,
then we can define a rational map $f\colon X\RationalTo\AA^{n}\subset\PP^{n}$
by
\begin{equation}
f(x):=\left(\frac{s_{1}}{s_{0}}(x),\dots,\frac{s_{n}}{s_{0}}(x)\right).
\end{equation}
If $s_{0}$, \dots, $s_{n}$ generate $\sheaf{L}$, then $f$ can be
extended to $f\colon X\to\PP^{n}$.
\end{example}

\begin{notation}
Let $X$ be normal, $s\in\sheaf{L}(U)$ a nonzero section, $Y\subset X$ be
some prime divisor.

We can now take some open $V\subset X$ such that $V\cap Y\neq\emptyset$
and $\sheaf{L}|_{V}\iso\StructureSheaf_{V}$ generated by $t\in\sheaf{L}(V)$.
\end{notation}

\begin{definition}
We define $v_{Y}(s):=v_{Y}(s/t)$.

We can check this is well-defined: it's independent of choice of $t$,
if $V'\cap Y\neq\emptyset$ and $\sheaf{L}|_{V'}$ generated by $t'\in\sheaf{L}(V')$,
then $t/t'$ is nowhere zero regular function on $V\cap V'$. Then
$t/t'$ is a unit in $\StructureSheaf_{X,Y}$. Then
\begin{equation}
v_{Y}(s/t') = v_{Y}\left(\frac{s}{t'}\frac{t'}{t}\right)=v_{Y}(s/t).
\end{equation}
\end{definition}

\begin{definition}
We define $(s)=\sum_{Y}v_{Y}(s)[Y]\in\Div(X)$.
\end{definition}

\begin{note}
If $s'\in\sheaf{L}(U')$ is a nonzero section, then the divisor
associated to $s'$ is
\begin{equation}
(s')=(s)+(s/s'),
\end{equation}
which implies that $(s')=(s)\in\Cl(X)$. So we have a well-defined map
\begin{equation}
\begin{split}
&\Pic(X)\to\Cl(X)\\
&\sheaf{L}\mapsto(s),
\end{split}
\end{equation}
which is a group morphism sending a sheaf to the divisor associated
with a nonzero section of the shef, and
\begin{equation}
(s_{1}\otimes s_{2})=(s_{1})+(s_{2})\in\Div(X)
\end{equation}
for any $s_{1}\in\sheaf{L}_{1}(U)$ and $s_{2}\in\sheaf{L}_{2}(U)$ and
$s_{1}\otimes s_{2}\in(\sheaf{L}_{1}\otimes\sheaf{L}_{2})(U)$.
\end{note}