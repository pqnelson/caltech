%%
%% fall-lecture14.tex
%% 
%% Made by Alex Nelson <pqnelson@gmail.com>
%% Login   <alex@lisp>
%% 
%% Started on  2025-10-30T12:10:21-0700
%% Last update 2025-10-30T12:10:21-0700
%% 

\lecture{}

\begin{proposition}
Let $R$ be a finitely-generated algebra over a field $\kk$, and assume
$R$ is an integral domain.
Then $\dim(R)=\trdeg_{\kk}(R_{0})$ is the transcendence degree of the
field of fractions $R_{0}$ (of $R$) over $\kk$.
\end{proposition}

\begin{proof}
By Noether normalization, $\kk[x_{1},\dots,x_{n}]\into R$ integral.
Then $\kk(x_{1},\dots,x_{n})\into R_{0}$ algebraic field extension.
Then $\dim(R)=n=\trdeg_{\kk}(\kk(x_{1},\dots,x_{n}))=\trdeg_{\kk}(R_{0})$.
\end{proof}

\begin{corollary}
If $X$ is a variety and $a\in X$ is a point, then $\codim_{X}\{a\}=\dim\RegularFuns_{X,a}$
(where $\RegularFuns_{X,a}\subset\FunField{X}$ is contained in the
functional field of $X$). Furthermore, if $X$ is irreducible, then
this is the dimension of $X$.
\end{corollary}

\begin{nakayama}
Let $A$ be a commutative ring, let $I\ideal A$ be an ideal,
let $M$ be a finitely-generated $A$-module such that $M=IM$.
Then there exists an $a\in A$ such that $a\equiv1\bmod{I}$ and $aM=0$.
(Equivalently, $\exists i\in I\ldotp\forall m\in M\ldotp im=m$ ---
take $i=1-a$ to get this equivalence.)
\end{nakayama}

\begin{proof}
Let $M$ be generrated by $m_{1}$, \dots, $m_{n}$.
Then each
\begin{equation}
m_{i}=\sum_{j}a_{ij}m_{j}
\end{equation}
where $a_{ij}\in I$, so
\begin{equation}
(I_{n}-Z)\begin{bmatrix}m_{1}\\\vdots\\m_{n}
  \end{bmatrix} = 0
\end{equation}
with $I_{n}$ is the $n\times n$ identity matrix and $Z=(a_{ij})$ as a matrix. Then
\begin{equation}
\det(I_{n}-Z)\begin{bmatrix}m_{1}\\\vdots\\m_{n}
  \end{bmatrix} = 0
\end{equation}
and we see $\det(I_{n}-Z)\equiv1\bmod{I}$, and take $a=\det(I_{n}-Z)$.
\end{proof}

\begin{xca}
Let $(A,\mathfrak{m})$ be a local ring. Suppose $M$ is a
finitely-generated $A$-module, and $f_{1},\dots,f_{n}\in M$ generates $M/\mathfrak{m}M$.
Then $f_{1}$, \dots, $f_{n}$ generate $M$.
\end{xca}

\begin{corollary}
Let $(R,\mathfrak{m})$ be a Noetherian local ring, let $\FF=R/\mathfrak{m}$.
Then the minimum number of generators for $\mathfrak{m}$ equals $\dim_{\FF}(\mathfrak{m}/\mathfrak{m}^{2})\geq\codim(\mathfrak{m})=\dim(R)$.
\end{corollary}

(The Principal Ideal Theorem~\ref{thm:pit} (Matsumara Theorem~12.I) gives the
inequality, then apply the exercise using $M=\mathfrak{m}$, lift the
generators.)

\begin{definition}
We say a local ring $(R,\mathfrak{m})$ is \define{Regular} when
$\dim(R)=\dim_{\FF}(\mathfrak{m}/\mathfrak{m}^{2})$. 
\end{definition}

\subsection{Smooth Varieties} % 78

We are going to study smooth varieties, then we will go to the later
part of Hartshorne~\cite{hartshorne1977algebraic} Chapter~1.

\begin{node}[Linear term of a polynomial]
Let $f\in R[x_{1},x_{2}]$ be a polynomial. 
Then we can write it as
\begin{equation*}
f(x,y) = \sum_{m,n}a_{m,n}x^{m}y^{n}.
\end{equation*}
Then its linear term looks like
\begin{equation}
f_{1}(x,y) = a_{0,0} + a_{1,0}x + a_{0,1}y.
\end{equation}
We just discard the terms of order greater than 1.

More generally and formally: given $f\in\kk[x_{1},\dots,x_{n}]$ and a
point $p\in\AA^{n}$, we can define a linear functional corresponding
to the lienar part of $f$ at $p$ to be
\begin{equation}
f_{p,1}(v) = \sum^{n}_{i=1}v_{i}\frac{\partial f}{\partial x_{i}}(p).
\end{equation}
Then for an irreducible affine variety $X$, we can define the tangent
space at $p\in X$ as $T_{p}X = \{v\in\kk^{n}\mid\forall f\in I\ldotp f_{1,p}(v)=0\}$.
Well, we let $I_{1,p}=\{f_{1,p}\mid f\in I\}$, then $T_{p}(X)=V(I_{1,p}^{*})$.
\end{node}

\begin{definition}% Gathmann, 10.1
Let $X$ be a variety. Let $a\in X$.
We define the \define{Tangent Space} of $X$ at $a$ to be
$T_{a}X = V(f_{1}\mid f\in I(X))\subset\AA^{n}$, where $f_{1}$ is the
linear term of $f$.
\end{definition}

\begin{remark}
We want to establish the dimension of the tangent space does not
depend on a choice of affine open subset or the coordinates of the point.
We need an alternate description of the tangent space which does not
depend on any such choices.
\end{remark}

\begin{lemma}% Gathmann, 10.4
Let $X\subset\AA^{n}$ be affine and $a=0\in X$ with $I(a)=\langle\bar{x}_{1},\dots,\bar{x}_{n}\rangle\ideal\CoordRing{\kk}{X}$
where $\bar{x}_{i}\in\kk[x_{1},\dots,x_{n}]/I(X)=\CoordRing{\kk}{X}$.
Then $I(a)/I(a)^{2}\iso\hom_{\kk}(T_{a}X,\kk)$.
\end{lemma}

\begin{proof}
Let 
\begin{equation}
\begin{split}
\varphi\colon I(a)&\hom_{\kk}(T_{a}X,\kk)\\
\bar{f}&\mapsto f_{1}|_{T_{a}X}
\end{split}
\end{equation}
We see that $\varphi$ is well-defined and surjective. We want to prove $\ker(\varphi)=I(a)^{2}$.

\textsc{Claim 1} ($\ker(\varphi)\subset I(a)^{2}$).
Consider the vector space $W=\{g_{1}\mid g\in I(X)\}$ of $\kk[x_{1},\dots,x_{n}]$
and let $k$ be its dimension. Then from linear algebra, we know $W$ is
the space of linear forms vanishing on $T_{a}X$. Let $\bar{f}\in\ker(\varphi)$
[equivalence class of $f$]. Then there exists a $g\in I(X)$ such that
$g_{1}=f_{1}$ implies $\overline{f-g}\in I(a)^{2}$.
Hence the claim $\ker(\varphi)\subset I(a)^{2}$.

\textsc{Claim 2} ($\ker(\varphi)\supset I(a)^{2}$).
Let $\bar{f}$, $\bar{g}\in I(a)$. Then
\begin{subequations}
\begin{align}
(fg)_{1} &= f(0)g_{1} + f_{1}g(0)\\
  &= 0\cdot g_{1}+f_{1}\cdot0\\
  &= 0,
\end{align}
\end{subequations}
which implies $\varphi(\bar{f}\bar{g})=0$, hence $\bar{f}\bar{g}\in\ker(\varphi)$.
\end{proof}

\begin{corollary}% Gathmann, 10.5
Let $S=\CoordRing{\kk}{X}\setminus I(a)$. Then
\begin{equation*}
I(a)/I(a)^{2}\iso\bigl(S^{-1}I(a)\bigr)/\bigl(S^{-1}I(a)\bigr)^{2}.
\end{equation*}
In particular, this means if $a\in X$ is a point on any variety $X$,
and $I_{a}$ is the unique maximal ideal in $\RegularFuns_{X,a}$, then
$T_{a}X$ is naturally isomorphic to the vector space dual to
$I_{a}/I_{a}^{2}$ and therefore independent of any choices.
\end{corollary}

\begin{definition}[Smooth and singular varieties]% Gathmann, 10.7
Let $X$ be a variety.
\begin{enumerate}
\item A point $a\in X$ is \define{Smooth} (or \emph{Regular} or \emph{Nonsingular})
  if $T_{a}X=C_{a}X$; otherwise it is \define{Singular}.
\item We say $X$ is \define{Smooth} if it is smooth everywhere,
  otherwise it is \emph{singular}.
\end{enumerate}
\end{definition}

\begin{fact}
A regular local ring is an integral domain.
(This implies if $a$ lies in the intersection of 2 irreducible
components, then $a$ is a singular point.)
\end{fact}

\begin{proposition}[Affine Jacobi criterion]% Gathmann, 10.11
Let $a\in X$ be a point in an affine variety $X\subset\AA^{n}$.
Suppose $I(X)=\langle f_{1},\dots,f_{r}\rangle$.
Then $X$ is smooth at $a$ if and only if the Jacobian matrix has rank
\begin{equation}
\rank(J)=\rank\left(\frac{\partial f_{i}}{\partial x_{j}}(a)\right)
\end{equation}
is at least $n-\codim_{X}\{a\}$. (``At least'' may be replaced by ``equal''.)
\end{proposition}

This is the same as saying $\dim(T_{a}X)\leq\codim_{X}\{a\}$ because
the Jacobian matrix is just picking out linear terms.

\begin{corollary}% Gathmann, 10.14 (b)
If $\rank(J)=r$, then $X$ is smooth at a local dimension $n-r$.
\end{corollary}

\begin{proof}
We see $\codim_{X}\{a\}\geq n-r$ implies $r\geq n-\codim_{X}\{a\}$.
Hence $X$ is smooth at $a$ by Jacobi criterion, and it must be an equality.
\end{proof}

\begin{remark}
This is really similar to the implicit function theorem.
\end{remark}

\begin{example}
Let $X=V(z^{3}-x^{2}y^{2})\subset\AA^{3}$, $\Char(\kk)\neq2,3$.
Let $p\in X$. Then $p$ is nonsingular if and only if $\rank[-2xy^{2},-2x^{2}y,3z^{2}]=3-\dim(X)=1$.
Then
\begin{equation}
X_{\text{sing}}=V(z^{3}-x^{2}y^{2},2xy^{2},2x^{2}y,3z^{2})=V(xy,z),
\end{equation}
so the singular locus is just the union of the $x$, $y$ coordinates.
\end{example}

\begin{corollary}% Gathmann, 10.19
The singular locus is always closed, i.e., the smooth locus is always open.
\end{corollary}

