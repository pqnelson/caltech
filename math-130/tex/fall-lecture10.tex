%%
%% fall-lecture10.tex
%% 
%% Made by Alex Nelson <pqnelson@gmail.com>
%% Login   <alex@lisp>
%% 
%% Started on  2025-10-23T08:36:48-0700
%% Last update 2025-10-23T08:36:48-0700
%% 

\lecture{}

\begin{node}
The image of a projective variety under morphisms is always
closed. This is always true, but we have a stronger statement.
\end{node}

\begin{proposition}
Let $X$ be a projective variety. Let $Y$ be a variety. Then the
projection $X\times Y\to Y$ is closed.
\end{proposition}

(The slogan is either ``$X$ is complete" or ``The morphism $X\to\mathrm{pt}$
is proper''.)

\begin{remark}
Recall from topology, we call a map $f\colon X\to Y$ between
topological spaces \define{Closed} if $f(A)\subset Y$ is closed for
every closed subset $A\subset X$.
\end{remark}

\begin{proposition}
Let $f\colon X\to Y$ be any morphism. We know its graph $\Gamma_{f}$
is closed. We have
\begin{equation}
X\xrightarrow{(\id,f)}X\times Y\xrightarrow{\pi}Y
\end{equation}
and so we see
\begin{equation}
f(X) = \pi(\Gamma_{f}).
\end{equation}
\end{proposition}

\begin{proof}
Suffices to check this for $X=\PP^{n}$ since
\begin{equation}
\vcenter{\xymatrix{%
\raisebox{0pt}[0.9\height][0.3\height]{$X\times Y$}\ar@{^{(}->}[d]\ar[dr] & \\
\PP^{n}\times Y\ar[r] & Y}}
\end{equation}
commutes.
\end{proof}

\begin{example}
This is not true for $\AA^{1}$ --- the projection
$\AA^{1}\times\AA^{1}\to\AA^{1}$ is not closed. Take the parbola
$xy-1$ as $\AA^{1}$, then we see its projection morphism maps this to
$\AA^{1}\setminus\{0\}\neq\AA^{1}$. 
\end{example}

\begin{proposition}% Gathmann, 7.17
The projection morphism $\pi\colon\PP^{n}\times\PP^{m}\to\PP^{m}$ is closed.
\end{proposition}

\begin{proof}
Let $Z\subset\PP^{n}\times\PP^{m}$ be a closed subset. Then using
Segre embedding, we can write
\begin{equation}
Z = V(f_{1},\dots,f_{r}),
\end{equation}
for homogeneous polynomials $f_{1}$, \dots, $f_{r}$. Without loss of
generality, we may assume $f_{1}$, \dots, $f_{r}$ are of the same
degree --- i.e., $\deg(f_{i})=d$ for each $i=1,\dots,r$. Then $f_{i}$
is a degree $d$ bi-homogeneous polynomial in $x_{0}$, \dots, $x_{n}$ and
$y_{0}$, \dots, $y_{m}$.

Let $a\in\PP^{m}$. Define $g_{i}(\cdot)=f_{i}(\cdot, a)$ which is a
homogeneous in the $x$ variables. Then we see
\begin{subequations}
\begin{align}
a\notin\pi(Z) &\iff \neg(\exists x\in\PP^{n}\ldotp (x,a)\in Z)\\
&\iff V_{p}(g_{1},\dots,g_{r})=\emptyset\\
&\iff \Radical{\langle g_{1},\dots, g_{r}\rangle}\mbox{ is irrelevant
  by Projective Nullstellensatz}\\
&\iff\exists i\ldotp\exists k_{i}\in\NN\ldotp x_{i}^{k_{i}}\in\langle g_{1},\dots,g_{r}\rangle\\
&\iff\exists k\in\NN\ldotp\kk[x_{0},\dots,x_{n}]_{k}\subset\langle g_{1},\dots, g_{r}\rangle\\
\intertext{which implies $k\geq d$ and}
&\iff K[x_{0},\dots,x_{n}]_{k}=\langle g_{1},\dots,g_{r}\rangle_{k}.
\end{align}
\end{subequations}
If we look at the $\kk$-linear map:
\begin{equation}
\begin{split}
  \varphi_{k}\colon&(\kk[x_{0},\dots,x_{n}]_{k-d})^{r}\to \kk[x_{0},\dots,x_{n}]_{k}\\
  &(h_{1},\dots,h_{r})\mapsto h_{1}g_{1}+\cdots+h_{r}g_{r},
\end{split}
\end{equation}
then it must be surjective. Therefore we have
\begin{subequations}
\begin{align}
  \rank_{\kk}(\varphi_{k}) &= \dim(\kk[x_{0},\dots,x_{r}]_{k})\\
  &=\binom{n+k}{k}.
\end{align}
\end{subequations}
In other words, at least one minor of this size is nonzero. The
entries of the minors are coefficients of the $g_{i}$ and therefore in
the coordinates of $a$. Then $\{a\in\PP^{m}\mid a\notin\pi(Z)\}$ is open.
Hence its complement is closed.
\end{proof}

\begin{corollary}% Gathmann, 7.19
For any variety $Y$, $\pi\colon\PP^{n}\times Y\to Y$ is closed.
\end{corollary}

\begin{proof}
Suffices to prove the claim for any affine variety
$Y\subset\AA^{m}$. Let $Z\subset\PP^{n}\times Y$ be a closed
subset. Then we can take its closure $\closure{Z}\subset\PP^{n}\times\PP^{m}$
(since $\AA^{m}\subset\PP^{m}$), then we know $\pi(\closure{Z})\subset\PP^{m}$
is closed. But
\begin{equation}
\pi(Z) = \pi(\closure{Z}\cap(\PP^{n}\times Y))=\pi(\closure{Z})\cap Y,
\end{equation}
which is closed in $Y$.
\end{proof}

\begin{definition}[Complete varieties]% Gathmann 7.20
Let $X$ be a variety. We call $X$ \define{Complete} if for any variety
$Y$ the projection $\pi\colon X\times Y\to Y$ is closed. (This is
well-defined: $\PP^{n}$ is complete.)
\end{definition}

\begin{corollary} % Gathmann, 7.23
Let $f\colon X\to Y$ be any morphism of varieties.
If $X$ is complete, then its image $f(X)$ is a complete and closed
subvariety in $Y$.
\end{corollary}

\begin{proof}
We know $f(X)$ is closed. We need to prove completeness.
Let $Y'$ be an arbitrary variety. Consider the projection morphism
\begin{equation}
\pi'\colon f(X)\times Y'\to Y',
\end{equation}
and we want to show that it is closed. Let
\begin{equation}
\begin{split}
\psi\colon& X\times Y'\to f(X)\times Y'\\
&(x,y) \mapsto (f(x),y)
\end{split}
\end{equation}
Then
\begin{subequations}
\begin{align}
\pi'(Z) &= \pi'\bigl(\psi(\psi^{-1}(Z))\bigr)\\
&=\pi(\psi^{-1}(Z))\subset Y,
\end{align}
\end{subequations}
which is closed since $\psi^{-1}(Z)$ is closed and $\pi$ is closed.
\end{proof}

\begin{corollary}[Liouville's theorem for varieties]% Gathmann, 7.24
Let $X$ be a connected, complete variety.
Then $\RegularFuns_{X}(X)=\kk$, i.e., the global regular functions are
just constants. 
\end{corollary}

\begin{proof}
Let $\varphi\in\RegularFuns_{X}(X)$. Now think of it as a morphism
\begin{equation}
\varphi\colon X\to\AA^{1}.
\end{equation}
We know $\varphi(X)$ is closed, connected, complete. The only
possibility is $\varphi$ is a constant function.
\end{proof}

\begin{remark}
This should remind the reader of Liouville's theorem, which says a
bounded holomorphic function on $\CC$ is constant. We should also
recall that functions on the compact space $\CC\PP^{1}$ must be bounded.
\end{remark}

\begin{definition}[Veronese embedding]
Let $V$ be a vector space. Consider $V\to\Sym(V^{\otimes d})$ the
symmetric part of a $d$-fold tensor product. This acts by sending
$v\in V$ to $v^{\otimes d}$.

For $V\iso\kk^{2}$, if $e_{1}$ and $e_{2}$ are a basis for $V$, then
\begin{equation}
\{e_{1}\otimes e_{1},e_{1}\otimes e_{2}+e_{2}\otimes e_{1},
e_{2}\otimes e_{2}\}
\end{equation}
is a basis for $V\otimes V$, and we send $e_{1}\mapsto e_{1}\otimes e_{1}$
and $e_{2}\mapsto e_{2}\otimes e_{2}$.

Then this gives us a mapping $\PP(V)\to\PP(\Sym(V^{\otimes d}))$ sending
$[x_{0}:\dots:x_{n}]$ to degre $d$ monomials.

For $d=2$, this sends
\begin{equation}
\begin{split}
  [x:y]\mapsto&[x^{2}:xy:y^{2}]\\
  &[a:b:c],
\end{split}
\end{equation}
where we demand
\begin{equation}
\det\begin{bmatrix}a & b\\b & c
\end{bmatrix}=0.
\end{equation}
This is an embedding, so it is an isomorphism onto its image. It's
usually reserved for projective varieties.

We call this the \define{Veronese Embedding}. The good thing about
this is it turns degree $d$ equations into degree one equations.
\end{definition}

\begin{remark}
I actually have not seen the Veronese embedding presented at this
level of generality. Usually, the presentation is restricted to the
$d=2$ case.
\end{remark}

\begin{example}
Consider $\AA^{1}\to\AA^{3}$ sending $t\mapsto(t,t^{2},t^{3})$. Let
$C$ be its image. We know $I(C)=\langle y-x^{2},z-x^{3}\rangle$.
What is $\closure{C}\subset\PP^{3}$?

If we na\"{\i}vely homogenize the generators for the ideal, we would
end up with
\begin{subequations}
\begin{equation}
\frac{y}{w}-\left(\frac{x}{w}\right)^{2},
\frac{z}{w}-\left(\frac{x}{w}\right)^{3},
\end{equation}
but that would not quite be right, we would need to clear the
denominators and obtain:
\begin{equation}
wy-x^{2}, w^{2}z-x^{3},
\end{equation}
\end{subequations}
which is still not quite right. We'd need to do quite a bit more work.

On the other hand, we can extend the morphism to
\begin{equation}
\begin{split}
  \PP^{1}\to\PP^{3}\\
  [s:t]\mapsto[s^{2}t:st^{2}:t^{3}:s^{3}].
\end{split}
\end{equation}
If we identify $[x:y:z:w]=[s^{2}t:st^{2}:t^{3}:s^{3}]$, then this is
the Veronese embedding. Then requiring
\begin{equation}
\rank\begin{bmatrix}x & y\\y & z\\w & x
\end{bmatrix}=1,
\end{equation}
we obtain the generators for the ideal.
\end{example}

\begin{corollary}% Gathmann, 7.29
Let $X\subset\PP^{n}$ be a projective variety. Let $f\in S(X)$ be a
homogeneous and non-constant function. Then $X\setminus V(f)$ is an
affine variety.
\end{corollary}

\begin{proof}
  Obvious if $f$ is linear.

  If $\deg(f)>1$, then apply the Veronese embedding and this
  transforms us into the $\deg(f)=1$ situation.
\end{proof}

\subsection{Grassmannians}

\begin{definition}
Let $V$ be a [finite-dimensional] vector space over $\kk$. We define
the \define{Grassmannian} to consist of the set,
\begin{subequations}
\begin{align}
\Gr(r,V) &= \{r-\mbox{dimensional subspaces of }V\}\\
&=FR(n,r)/\GL(\kk^{r})
\end{align}
\end{subequations}
where $FR(n,R)$ is the set of full rank $n\times r$ matrices, and we
mod out by $\GL(\kk^{r})$ to forget about the choice of basis. We
equip this quotient with the quotient topology.

\textbf{Sheaf:} $f$ is a regular function on $U\subset\Gr(r,V)$, then
$\pi\colon FR(n,r)\to\Gr(r,V)$ precomposes $f\circ\pi$ is regular on
$\pi^{-1}(U)$.

\textbf{Observe:}
\begin{enumerate}
\item $\pi$ is a morphism of ringed spaces
\item Let $X$ be a ringed space and $\phi\colon\Gr(r,V)\to X$ be a
  map. Then $\phi$ is a morphism iff $\phi\circ\pi$ is a morphism.
\end{enumerate}

\textbf{Charts:}
We see that elements of $\GL(\kk^{r})$ sends a nonzero $r\times r$
minor to a reduced canonical echelon form matrix with free variables
ranging over $\AA^{(n-r)r}$.

We can think of it in a coordinate-free way: Fix $\Omega\in\Gr(r,V)$
and $\Gamma\in\Gr(n-r,V)$ such that $\Omega\oplus\Gamma=V$. Then the
charts
\begin{equation}
U_{\Gamma}=\{\Sigma\in\Gr(r,V)\mid \Sigma\cap\Gamma=\emptyset\}\iso\hom_{\kk}(\Omega,\Gamma)\iso\AA^{(n-r)r}.
\end{equation}
To see the isomorphisms, let $\Sigma\subset V$, then
\begin{equation}
\vcenter{\xymatrix{
                     & \Sigma\oplus\Gamma\ar@{=}[d] &   &       \\
{\color{red}\ar@[red][d]_{\iso}}
\Sigma\ar@{^{(}->}[r] & V\ar[d]\ar[r] & V/\Omega\ar[r]^{\iso} & \Gamma\\
\Omega\ar[r]^{\iso}\ar@/_4pc/[urrr]^{\iso}   & V/\Gamma      &                      &\\
& & &}}
\end{equation}
If we start on the right with a morphism $\Omega\to\Gamma$, we can
look at its graph as a subset of $\Omega\oplus\Gamma=V$ which is a
subspace of $\Sigma\in\Gr(r,V)$ such that $\Sigma\cap\Gamma=\emptyset$.
This establishes a bijection. We can check
$U_{\Gamma}\iso\AA^{(n-r)r}$ using observation (2).

Without loss of generality, we can assume first $r\times r$ minor is
nonzero (i.e., $\Gamma$ is spanned by the last $(n-r)$ vectors), then
we have the morphism
\begin{equation}
\begin{split}
\Mat(n-r,r)&\to U_{\Gamma}\\
B&\mapsto\pi\left(\begin{bmatrix}I_{r}\\B
\end{bmatrix}\right)
\end{split}
\end{equation}
This gives us a morphism in one direction. The other direction
\begin{equation}
\begin{split}
U_{\Gamma}&\to\Mat(n-r,r)\\
\pi\left(\begin{bmatrix}A\\B
\end{bmatrix}\right)&\mapsto BA^{-1}
\end{split}
\end{equation}
where $A$ is an invertible $r\times r$ matrix. This is well-defined
and gives us our isomorphism.
\end{definition}