%%
%% winter-lecture13.tex
%% 
%% Made by Alex Nelson <pqnelson@gmail.com>
%% Login   <alex@lisp>
%% 
%% Started on  2026-02-03T11:23:37-0800
%% Last update 2026-02-03T11:23:37-0800
%% 

\lecture{}

\begin{node}
This lecture was highly idiosyncratic. For different presentations of
the same material, but done differently, see either the last chapter
of Eisenbud and Harris's \textit{The Geometry of Schemes}, or Vakil~\cite[\S\S10.1.7~\textit{et~seq}.]{vakil2025rising}.

The whole idea is to extend some notions from schemes to functors.
\end{node}

\begin{definition}
Let $\cat{C}$ be a category, $F\colon\cat{C}\to\Set$ be a
contravariant functor. We say $F$ is \define{Representable} if there
exists an $X\in\cat{C}$ such that $\hom(-,X)\iso F$.
\end{definition}

\begin{remark}
One of the key insights, which I don't think is adequately expressed,
is that if $X$ is a scheme---so $X\in\Sch$---then any open subset
$U\subset X$ is also a scheme (thanks to the open immersion $i\colon U\into X$).
But this means the collection of all such open subsets and inclusions
forms a subcategory of $\Sch$. That is to say, $\Open(X)$ is a
subcategory of $\Sch$.
\end{remark}

\begin{definition}
Let $F$ be a contravariant functor $F\colon\Sch\to\Set$. We say that
$F$ is a \define{Sheaf in Zariski Topology} if for any scheme $Z$, the
presheaf on $Z$ induced by $F$ is a sheaf (we see $F|_{\Open(Z)}$ is
obviously a pre-sheaf, by the previous remark).
\end{definition}

\begin{proposition}
If $X$ is a scheme, then its functor of points is a sheaf.
\end{proposition}

\begin{theorem}\label{thm:math130b:winter2026:thm1}
A functor $F\colon\Sch\to\Set$ is representable iff both $F$ is a
sheaf and $F$ admits an open cover by a representable functor.
\end{theorem}

\begin{definition}
Let $F\colon\Sch\to\Set$ be a contravariant functor.
We define an \define{Open Subset} $U$ of $F$ is an association to each
element $\xi\in F(T)$ (where $T$ is a scheme) an open set $U_{\xi}\subset T$ which is
compatible with pullbacks, i.e., if $f\colon S\to T$ is a morphism of
schemes, then
\begin{equation}
f^{-1}(U_{\xi})=U_{F(f)(\xi)}.
\end{equation}
\end{definition}

\begin{node}
The idea is that we write $T\xrightarrow{\xi}F$ for $\xi\in F(T)$. So if
we have a scheme morphism $f\colon S\to T$, then
\begin{equation}
S\xrightarrow{f}T\xrightarrow{\xi}F
\end{equation}
pullsback stuff correctly. After all, we just indexed all open subsets
of a scheme $S$ by elements of the set $F(S)$. We want to prove that
$F(S)$ is the set of scheme morphisms from $S$ to some [fixed] scheme,
which is the intuition underlying this notation (which seems bizarre
at first glance).
\end{node}

\begin{remark}
This definition is highly nonstandard, there seems to be no literature
on it besides those cited earlier in the lecture. If I understand the
``heritage'' of the notions, it may be traced back to Grothendieck. So
this is all \emph{his} fault.

I \emph{think} the reason this seems strange is because we can avoid
it using sites instead, and the construction offered ``works'' because
``secretely'' we're doing something with sites.
\end{remark}

\begin{lemma}
If $F_{X}=\hom(-,X)\colon\Sch\to\Set$ is a contravariant functor, then
open subsets of $X$ are in natural bijection with open subsets of $F_{X}$.
\end{lemma}

\begin{definition}
Let $F\colon\Sch\to\Set$ be a contravariant functor.
An \define{Open Covering} of $F$ is a collection of open subsets
$U_{i}$ of $F$ such that for every scheme $S$ and for every element
$\xi\in F(S)$, the collection $\{(U_{i})_{\xi}\}$ is an open covering of $S$
in the usual sense $S=\bigcup_{i}(U_{i})_{\xi}$.
\end{definition}

\begin{definition}
Let $F\colon\Sch\to\Set$ be a contravariant functor, and also
let $U\subset F$ be an open subset of the functor $F$.
We define the \define{Open Subfunctor} $F_{U}(S)=\{\xi\in F(S)\mid U_{\xi}=S\}$.
\end{definition}

\begin{remark}
Vakil offers a fancier definition for an ``open subfunctor'' which is
equivalent to what we have just defined.
\end{remark}

\begin{proof}[{Proof sketch of Theorem~\ref{thm:math130b:winter2026:thm1}}]
Let $X_{i}$ be the collection of schemes associated to the
representable open subfunctors which cover $F$. Now, we can show that
we can construct the gluing data in Hartshorne~\cite{hartshorne1977algebraic}
(exercise II.2.12).
\end{proof}

\begin{node}
Now, we are in a position to prove
Theorem~\ref{thm:math130b:winter2026:thm1}, but it is really not
enlightening. (Hence we offered only a proof sketch.)
But we can use it to prove the immediate corollary:
\end{node}

\begin{corollary}
Fiber products exist in the cateory of schemes.
\end{corollary}

\begin{node}
We can use all this machinery to construct the Grassmannian $G(r,n)$
by showing the following functor
\begin{equation}
F(X)=\{\sheaf{O}_{X}^{\oplus n}\onto\sheaf{E}\}
\end{equation}
where $\sheaf{E}$ is a rank $r$ locally-free sheaf on $X$.
\end{node}