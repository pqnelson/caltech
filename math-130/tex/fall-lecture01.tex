%%
%% fall-lecture01.tex
%% 
%% Made by Alex Nelson <pqnelson@gmail.com>
%% Login   <alex@lisp>
%% 
%% Started on  2025-09-30T07:30:50-0700
%% Last update 2025-09-30T07:30:50-0700
%% 

\lecture{}

\begin{node}
Most of this quarter will be focused on varieties. Next quarter will
start on schemes. Andreas Gathmann's lecture notes (\url{https://agag-gathmann.math.rptu.de/de/alggeom.php}) will roughly be the primary
source.
\end{node}

\begin{convention}
\begin{enumerate}
\item We will write $K$ or $\kk$ for algebraically closed fields. If
  we need the field to be ``not necessarily'' algebraically closed
  (i.e., it may or may not be algebraically closed), then we will make
  note of it.
\item All rings are unital and commutative.
\item If $J\ideal R$ is an ideal, then its radical is denoted
  $$\sqrt{J} = \{x\in R\mid x^{n}\in J\mbox{ for some }n\in\NN\}.$$
  As usual, if $S\subset R$ is any old set, then we will write the ideal
  generated by this set as $\langle S\rangle$.
\end{enumerate}
\end{convention}

\begin{definition}[Affine spaces]
We define the \define{Affine Space} over $K$ to be
$$\AA^{n}:=\AA^{n}_{K}=\{(a_{1},\dots,a_{n})\mid a_{i}\in K\}.$$
This is a \textit{bona fide} affine space. We can transform
$\AA^{n}_{K}$ into a vector space $K^{n}$ by picking an
origin. Dually, we can transform a vector space $K^{n}$ into an affine
space $\AA^{n}_{K}$ by ``forgetting'' the origin.

Are these \emph{really} that different? Examine its automorphisms. A
generic automorphism for $\AA^{n}$ looks like $\vec{x}\mapsto A\vec{x}+\vec{b}$
where $A\in\GL(K^{n})$ and $\vec{b}\in K^{n}$. On the other hand, a
generic automorphism for $K^{n}$ looks like $\vec{x}\mapsto A\vec{x}$.

The automorphisms for affine space are $K$-algebra automorphisms
sending points to points, lines to lines, conics to conics, and
polynomials to polynomials. But they will not preserve angles,
lengths, circles, etc.
\end{definition}

\begin{definition}[Zero sets and affine varieties]
Let $S\subset K[x_{1},\dots,x_{n}]$, we define the (affine) \define{Zero Locus}
of $S$ to be the subset of $\AA^{n}$ equal to
$$V(S) = \{x\in\AA^{n}\mid f(x)=0\mbox{ for all }f\in S\}.$$
Subsets of this form are called \define{Affine Varieties} (usually
given some extra structure, like a topology --- specifically the
Zariski topology). We will also write
$$V(f_{1},\dots,f_{m}) = V(\{f_{1},\dots,f_{m}\}).$$
Some people require varieties to be irreducible, others do not. We
will not.
\end{definition}

\begin{lemma}[Zero sets resemble closed sets]\label{lemma:fall:lecture01:zero-sets-resemble-closed-sets}
\begin{enumerate}
\item If $S_{1}\subset S_{2}\subset K[x_{1},\dots,x_{n}]$, then
  $V(S_{1})\supset V(S_{2})$.
\item $V(S_{1})\cap V(S_{2})=V(S_{1}S_{2})$ where $S_{1}S_{2}=\{fg\mid f\in S_{1},g\in S_{2}\}$.
\item $\bigcap_{i\in I}V(S_{i})=V\left(\bigcup_{i\in I}S_{i}\right)$.
\end{enumerate}
\end{lemma}

\begin{remark}[Topology formed from zero sets]
In particular, this means that arbitrary intersections and finite
unions of zero sets are zero sets, which makes them resemble closed
sets of a topological space. If we form a topology from these $V(S)$
as closed sets, then we obtain the \define{Zariski topology}.
\end{remark}

\begin{example}[Trivial affine varieties]
We see that $\AA^{n}$ is an affine variety since $\AA^{n}=V(0)$. The
empty set is also an affine variety $\emptyset=V(1)$ where $1$ is the
``one ideal''.
\end{example}

\begin{example}[Points are affine varieties]
Any point is an affine variety. Let $\vec{a}=(a_{1},\dots,a_{n})\in K^{n}$.
Then
$$\{\vec{a}\}=V(x_{1}-a_{1},\dots,x_{n}-a_{n}).$$
Finite subsets are affine varieties. Observe then that affine
varieties in $\AA^{1}$ are $\AA^{1}$ itself and finite subsets.

Also note that affine varieties are not Hausdorff.
\end{example}

\begin{example}[Products of affine varieties]
If $X\subset\AA^{n}$ and $Y\subset\AA^{m}$ are affine varieties, then
we may form $X\times Y\subset\AA^{m+n}$ as an affine variety. The
trick is to view $X=V(S_{1})$ where
\begin{equation}
S_{1}\subset K[x_{1},\dots,x_{n}]\subset K[x_{1},\dots,x_{n},y_{1},\dots,y_{m}]
\end{equation}
and
\begin{equation}
S_{2}\subset K[y_{1},\dots,y_{m}]\subset K[x_{1},\dots,x_{n},y_{1},\dots,y_{m}].
\end{equation}
The Zariski topology on the product \emph{is not the same} as the
product topology.
\end{example}

\begin{example}[Determinantal variety]
Now, let us consider $\AA^{mn}$ as parametrizing linear maps from
$K^{m}$ to $K^{n}$. We can consider matrices of rank less than or
equal to $i$, and these form affine varieties. Moreover, we see that the
subset of [surjective] linear maps $K^{m}\onto K^{n}$ is open in $\AA^{mn}$.
\end{example}

\begin{remark}[Zero sets of generated ideals are the same as zero sets of generators]
We can always think of affine varieties as the zero loci of ideals
$V(S)=V(\langle S\rangle)$. This allows us to connect to linear algebra.
\end{remark}

\begin{lemma}
Let $J,J_{1},J_{2}\ideal K[x_{1},\dots,x_{n}]$.
\begin{enumerate}
\item $V(\Radical{J})=V(J)$
\item $V(J_{1})\cup V(J_{2})=V(J_{1}J_{2})=V(J_{1}\cap J_{2})$
\item $V(J_{1})\cap V(J_{2})=V(J_{1}+J_{2})$.
\end{enumerate}
(Note: the first two results enumerated are not the same as for schemes.)
\end{lemma}

\begin{proof}
\begin{enumerate}
\item Recall $\Radical{J}\supset J$. Therefore $V(\Radical{J})\subset V(J)$.

  We will prove $V(\Radical{J})\supset V(J)$. Let $x\in V(J)$ and let
  $f\in\Radical{J}$. Then $f^{k}\in J$ for some $k\in\NN$ by definition of the radical.
  Then $f^{k}(x)=0$ by definition of the zero locus of $J$.
  Then $f(x)=0$, which implies $f\in V(\Radical{J})$.
  Hence $V(\Radical{J})\supset V(J)$.
\item We see $V(J_{1})\cup V(J_{2})=V(J_{1}J_{2})$ by Lemma~\ref{lemma:fall:lecture01:zero-sets-resemble-closed-sets}
  (which proves the left equality in the claim).
  We will now prove that $\Radical{J_{1}J_{2}}=\Radical{J_{1}\cap J_{2}}$,
  then use (1) of this Lemma to conclude the result.

  We know $\Radical{J_{1}J_{2}}\subset\Radical{J_{1}\cap J_{2}}\subset\Radical{J_{1}}\cap\Radical{J_{2}}$.
  Let $a\in\Radical{J_{1}}\cap\Radical{J_{2}}$.
  We see $a^{m}\in J_{1}$ and $a^{n}\in J_{2}$ for some $m\in\NN$ and $n\in\NN$.
  Then $a^{m+n}\in J_{1}\cap J_{2}$.
  Then $a\in\Radical{J_{1}J_{2}}$.
\item $J_{1}+J_{2}=\langle J_{1}\cup J_{2}\rangle$. \qedhere
\end{enumerate}
\end{proof}

\begin{theorem}[Hilbert's basis theorem]
Any ideal in a polynomial ring $K[x_{1},\dots,x_{n}]$ is finitely generated.
In other words, $K[x_{1},\dots,x_{n}]$ is Noetherian.
\end{theorem}

\begin{proposition}
If $R$ is a Noetherian ring, then the polynomial ring $R[x]$ is also Noetherian.
\end{proposition}

\begin{definition}
Let $X\subset\AA^{n}$ be any subset. We may form \define{the Ideal of $X$}
\begin{equation}
I(X) := \{f\in K[x_{1},\dots,x_{n}]\mid \forall x\in X\ldotp f(x)=0\}.
\end{equation}
\end{definition}

\begin{proposition}
\begin{enumerate}
\item For any subsets $X_{1}\subset X_{2}\subset\AA^{n}$, we have
  $I(X_{1})\supset I(X_{2})$.
\item We have $I(X)=\Radical{I(X)}$.
\end{enumerate}
\end{proposition}

\begin{remark}
We have this correspondence between subsets $X$ of $\AA^{n}$ and
ideals of $K[x_{1},\dots,x_{n}]$. Namely, $X\mapsto I(X)$, and then in
the other direction $I\mapsto V(I)$. (This is an inclusion-reversing bijection.)

What happens with $V\bigl(I(X)\bigr)$? Well, since $V(-)$ produces
closed sets, it is reasonable to believe that
\begin{equation}
V\bigl(I(X)\bigr) = \closure{X}
\end{equation}
where $\closure{X}$ is the closure of $X$ in the Zariski topology.

What about $I\bigl(V(I)\bigr)$? If the field is algebraically closed
$\AlgebraicClosure{K}=K$, then $I\bigl(V(I)\bigr)=\Radical{I}$. This
result is sometimes known as the \define{Strong Nullstellensatz} theorem.
\end{remark}

\begin{remark}
A related problem: what are the maximal ideals in the polynomial ring?

The Weak Nullstellensatz: they are of the form $\langle x_{1}-a_{1},\dots,x_{n}-a_{n}\rangle$
when the field is algebraically closed. The Weak Nullstellensatz
theorem follows from the next lemma.
\end{remark}

\begin{lemma}
If $K$ is a field which is a finitely generated algebra over $\kk$,
then $K$ is a finite-dimensional vector space over $\kk$.
\end{lemma}