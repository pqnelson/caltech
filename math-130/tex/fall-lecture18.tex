%%
%% fall-lecture18.tex
%% 
%% Made by Alex Nelson <pqnelson@gmail.com>
%% Login   <alex@lisp>
%% 
%% Started on  2025-11-08T09:38:26-0800
%% Last update 2025-11-08T09:38:26-0800
%% 

\lecture{}

\begin{proposition}
Let $C$ be an irreducible curve, let $p\in C$ be a nonsingular point,
and let $Y$ be a projective variety.
Any morphism $\varphi\colon C\setminus\{p\}\to Y$
can be uniquely extended to $\varphi\colon C\to Y$.
\end{proposition}

\begin{proof}
Without loss of generality, $Y=\PP^{n}$ and
$\varphi(C\setminus\{p\})\nsubset V_{p}(x_{i})$ for all $i$.
Suffices to prove existence, since uniqueness follows immediately.

Write $U := D_{p}(x_{0}x_{1}(\cdots)x_{n})\subset\PP^{n}$ and suppose
\begin{equation}
\varphi(C\setminus\{p\})\cap U\neq\emptyset.
\end{equation}
Write
\begin{equation}
f_{i,j} = \frac{\partial x_{i}}{\partial x_{j}}\circ\varphi,
\end{equation}
and we see $f_{i,j}\in\FunField[\kk]{C}$, let
\begin{equation}
v_{p}\colon\FunField[\kk]{C}^{*}\to\ZZ,
\end{equation}
the valuation given by $\RegularFuns_{C,p}$ (the stalks of the
structure sheaf of $C$ at $p$). Define
\begin{equation}
r_{i}:=v_{p}(f_{i,0}),
\end{equation}
choose $j$ such that $r_{j}$ is minimal. Then
\begin{equation}
v_{p}(f_{i,j})=r_{i}-r_{j}\geq0.
\end{equation}
Since
\begin{equation}
\frac{x_{i}}{x_{j}} = 
\frac{x_{i}/x_{0}}{x_{j}/x_{0}},
\end{equation}
it follows that
\begin{equation}
f_{i,j} = \frac{f_{i,0}}{f_{j,0}}.
\end{equation}
Then $f_{i,j}\in\RegularFuns_{C,p}$ for all $i$ and $j$.

Note: if $Q\in\varphi^{-1}(U)$, then
\begin{equation}
\varphi(Q):=(f_{0,j}(Q):\cdots:f_{j,j}(Q):\cdots:f_{n,j}(Q)).
\end{equation}
Extend $\varphi$ by defining $\varphi(P)$ by this expression.
\end{proof}

\begin{theorem}
$C_{K}$ is projective.
\end{theorem}

\begin{proof}
We know we have an open affine cover $C_{K}=U\cap V$ where
$U=\Spec(\closure{\kk[f]})$ and $V=\Spec(\closure{\kk[f^{-1}]})$
(where we are taking the spectrum [prime ideals] of the \emph{integral closures}
of the rings). We know $U\subset\AA^{N}$ is closed. We can take the
projective closure $\closure{U}$ of $U$, and $\closure{V}$ of $V$,
then the inclusion $U\into\closure{U}$ (where $\closure{U}\subset\PP^{N}$)
extends to a morphism $\varphi_{1}\colon C_{K}\to\closure{U}$.
Similarly, we have a morphism $\varphi_{2}\colon C_{K}\to\closure{V}$.

Define $\varphi\colon C_{K}\to\closure{U}\times\closure{V}$ by
$\varphi(p)=(\varphi_{1}(p),\varphi_{2}(p))$. Also define
$Y=\closure{\varphi(C_{K})}\subset\closure{U}\times\closure{V}$.

We claim $\varphi\colon C_{K}\to Y$ is an isomorphism. Note that
$\varphi(U)$ is closed in $U\times\closure{V}$, 
observe
\begin{equation}
\psi:=(\varphi_{2},\id)\colon U\times\closure{V}\to\closure{V}\times\closure{V},
\end{equation}
and
\begin{equation}
\varphi(U)=\{(u,v)\in U\times\closure{V}\mid\varphi_{2}(u)=v\}=\psi^{-1}(\Delta_{\closure{V}}).
\end{equation}
This means $\varphi(U)=Y\cap(U\times\closure{V})$, so
$\varphi|_{U}\colon U\to Y\cap(U\times\closure{V})$ is an isomorphism
whose inverse is $\pi_{U}\colon(Y\cap U\times\closure{V})\to U$ is
just the projection.

Why is $\varphi$ injective? Well, for any $p\in C_{K}$, we see
$\RegularFuns_{Y,\varphi(p)}=R_{p}\subset\kk$ and $\RegularFuns_{Y,\varphi(p)}$
is one-to-one[?] with $R_{p}$.

Why is $\varphi$ surjective? Well, Let $y\in Y$. Then
$\RegularFuns_{Y,y}\subset R_{p}$ for some $p\in C_{K}$. But
$R_{p}=\RegularFuns_{Y,\varphi(p)}$. Then $y=\varphi(p)$.
\end{proof}

\begin{lemma}
Let $X$ be an irreducible variety. Let $x,y\in X$. If
$\RegularFuns_{X,x}\subset\RegularFuns_{X,y}$, then $x=y$.
\end{lemma}

\begin{proof}
Take open affine neighborhoods $x\in U\subset X$ and $y\in V\subset X$.
Since $X$ is separated, $U\cap V$ is affine. There exists a surjection
$\CoordRing{\kk}{U}\otimes_{\kk}\CoordRing{\kk}{V}\to\CoordRing{\kk}{U\cap V}$,
$\CoordRing{\kk}{U}\subset\RegularFuns_{X,x}\subset\RegularFuns_{X,y}$
and $\CoordRing{\kk}{V}\subset\RegularFuns_{X,y}$. Then
$\CoordRing{\kk}{U\cap V}\subset\RegularFuns_{X,y}$.

\textsc{Claim:} $\CoordRing{\kk}{V}\subset\CoordRing{\kk}{U\cap V}$.
Take the maximal ideal $\mathfrak{m}_{y}\ideal\RegularFuns_{X,y}$.
Then $\CoordRing{\kk}{U\cap V}\cap\mathfrak{m}_{y}$ is a maximal ideal
in $\CoordRing{\kk}{U\cap V}$ which corresponds to a point in $U\cap V$
mapped to $y$ under the inclusion $U\cap V\into V$. Hence $y\in U$.
Now we can reduce to affine case for $U$.
Let $A:=\CoordRing{\kk}{U}$, $P\ideal A$ be the maximal ideal
corresponding to $x$, and $Q\ideal A$ be the maximal ideal
corresponding to $y$. Then $A_{P}\subset A_{Q}$, so
\begin{equation}
QA_{Q}\cap A_{P}\subset PA_{P}
\end{equation}
which implies
\begin{equation}
Q - QA_{Q}\cap A\subset PA_{P}\cap A=P.
\end{equation}
Since $Q\subset P$ and both are maximal ideals, it follows
$Q=P$. Hence the result.
\end{proof}

\begin{corollary}
Any curve is birational to a nonsingular projective curve. If $X$ is
any curve, then it is birational to $C_{K}$ where $K=\FunField[\kk]{X}$.
\end{corollary}

\begin{corollary}
If $X$ is any nonsingular curve, then $X$ is birational to some open
subset of $C_{K}$ where $K=\FunField[\kk]{X}$.
\end{corollary}

\begin{proof}
Define the morphism
\begin{equation}
\begin{split}
\varphi\colon & X\to C_{K},\\
& x\mapsto p,
\end{split}
\end{equation}
where $\RegularFuns_{X,x}=R_{p}\subset\kk$.

We claim $\varphi(X)\subset C_{K}$ is open. We see $\varphi$ is
clearly injective. Take $U\subset X$ open and affine, then
its coordinate ring $\CoordRing{\kk}{U}$ is finitely-generated by
$f_{1}$, \dots, $f_{n}$. Then
\begin{subequations}
\begin{align}
  \varphi(U) & := \{p\in C_{K}\mid\CoordRing{\kk}{U}\subset R_{p}\}\\
  &= \{p\in C_{k}\mid f_{1},\dots,f_{n}\in R_{p}\}\\
  &= \bigcap^{n}_{i=1}\Spec(\closure{\CoordRing{\kk}{f_{i}}})
\end{align}
\end{subequations}
where $\Spec(\closure{\CoordRing{\kk}{f_{i}}})$ in that last equation
describes the open affine subsets. Hence $\varphi(X)$ is open.

This is an isomorphism because
\begin{subequations}
\begin{align}
  \CoordRing{\kk}{U} &:=\bigcap_{x\in U}\RegularFuns_{X,x}\\
  &=\bigcap_{p\in\varphi(U)}R_{p}\\
  &=\CoordRing{\kk}{\varphi(U)}
\end{align}
\end{subequations}
locally, hence globally.
\end{proof}

\begin{xca}
We have 2 nonsingular projective curves are isomorphic if and only if
they have the same function field.
\end{xca}

\subsection{Degrees of Projective Varieties in $\PP^{n}$}

\begin{node}[Goal]
We want to compute the number of points in the intersection of
``general'' subvarieties of projective spaces.
\end{node}

\begin{definition}[Classical Version]
Let $X\subset\PP^{n}$ be a projective variety.
Then the \define{Degree} of $X$ is the number $\deg(X) := \#(X\cap V)$
where $V\subset\PP^{n}$ is a general linear subspace such that
$\dim(X)+\dim(V)=n$.
\end{definition}

\begin{example}
Let $f\in S=\kk[x_{0},\dots,x_{n}]$ be square-free. Then
\begin{equation}
\#(V_{p}(f)\cap(\mbox{general line}))=\deg(f).
\end{equation}
We repeatedly use the fundamental theorem of algebra, and we find the
number of roots is the degree of $f$.
\end{example}

\begin{caution}
$V_{p}(yz-x^{2})\subset\PP^{2}$ and $V_{p}(x)\subset\PP^{2}$ are both
isomorphic to $\PP^{1}$ but they have different degrees. So the degree
really depends on the embedding.
\end{caution}

\begin{example}
Let $f\in\HomogeneousCoordRing{}{\PP^{n}}$ be a square-free degree $d$
homogeneous polynomial. Let $X = V_{p}(f)\subset\PP^{n}$, and
$I(X)=\langle f\rangle$ its ideal. Then $R := S/\langle f\rangle$.
Recall $S_{m}:=\HomogeneousCoordRing{}{\PP^{n}}_{m}$ is the set of degree $m$ homogeneous polynomials for $\PP^{n}$.
We see
\begin{equation}
\dim_{\kk}(S_{m})=\binom{n+m}{n}=\frac{1}{n!}(m+n)(\cdots)(m+1),
\end{equation}
which itself is a polynomial in the variable $m$ of degree $n$ with
leading coefficient $1/n!$.

To do this for $R$, we just use the short exact sequence
\begin{equation}
0\to S_{m-d}\xrightarrow{f}S_{m}\to R_{m}\to 0,
\end{equation}
then we can compute
\begin{subequations}
\begin{align}
\deg_{\kk}(R_{m}) &= \deg_{\kk}(S_{m}) - \deg_{\kk}(S_{m-d})\\
&= \binom{n+m}{n} - \binom{n+m-d}{n}\\
&= \frac{1}{n!}(m+n)(\cdots)(m+1)-\frac{1}{n!}(m+n-d)(\cdots)(m+1-d),
\end{align}
\end{subequations}
which is a polynomial of degree $n-1$ with leading coefficient
\begin{subequations}
\begin{align}
\begin{pmatrix}\mbox{leading}\\\mbox{coefficient}
\end{pmatrix}&=\frac{1}{n!}\left(\sum^{n}_{i=1}i-\sum^{n}_{i=1}(i-d)\right)\\
&=\frac{d}{(n-1)!}.
\end{align}
\end{subequations}
Then we can recover $\deg(f)$ from $m\mapsto\dim_{\kk}(R_{m})$, and we
call this mapping the \define{Hilbert Function}.
\end{example}

\begin{node}
We need to recall a few notions from abstract algebra about graded stuff.
\end{node}

\begin{definition}
A \define{Graded $S$-Module} is a module $M$ with a decomposition
\emph{as an Abelian group}
\begin{equation}
M = \bigoplus_{d\in\ZZ}M_{d},
\end{equation}
such that $S_{m}M_{d}\subset M_{d+m}$.
\end{definition}

\begin{definition}
When $M$ is a graded $S$-module, we have
\begin{enumerate}
\item Its annihilator $\Annihilator(M)=\{f\in S\mid fM=0\}\subset S$
  which is a homogeneous ideal;
\item Its support $\Support(M)=V_{p}(\Annihilator(M))$ vanishing of
  $\Annihilator(M)$ in $\PP^{n}$.
\end{enumerate}
Why do we use this name ``the support of $M$''? Let $x\in\PP^{n}$,
then consider the corresponding ideal $P=I(\{x\})\subset S$. Then
$M_{p}\neq0$ if and only if $P\supset\Annihilator(M)$ if and only if $x\in\Support(M)$;
i.e., localizing at a point in the support is nonzero.
\end{definition}

\begin{example}
Let $X\subset\PP^{n}$ be closed. Then
\begin{equation}
\Annihilator(S/I(X))=I(X),
\end{equation}
which implies
\begin{equation}
\Support(S/I(X))=V_{p}(I(X))=X.
\end{equation}
\end{example}

\begin{note}
The $M_{d}$ is a $\kk$-vector space, since $S_{0}=\kk$.
If $M$ is a finitely-generated $S$-module, then each $M_{d}$ has
finite dimension.
\end{note}

\begin{definition}
Let $M$ be a graded $S$-module.
We define the \define{Hilbert Function} of $M$ to be the function
$\HilbertFun{M}(d):=\dim_{\kk}(M_{d})$.
\end{definition}