%%
%% winter-lecture06.tex
%% 
%% Made by Alex Nelson <pqnelson@gmail.com>
%% Login   <alex@lisp>
%% 
%% Started on  2026-01-27T10:29:44-0800
%% Last update 2026-01-27T10:29:44-0800
%% 

\lecture{}

\begin{definition}
Let $X$ be a scheme. We say $X$ is \define{Quasi-Separated} if for any
pair $U,V\subset X$ of affine open subsets, $U\cap V$ is quasi-compact.
\end{definition}

\begin{proposition}
Affine schemes are quasi-separated.
\end{proposition}

\begin{definition}
Let $f\colon X\to Y$ be a morphism of schemes. We say $f$ is
\define{Quasi-Separated} if $f^{-1}(V)$ is quasi-separated for all
affine open subsets $V\subset Y$.
\end{definition}

\begin{definition}
Let $f\colon X\to Y$ be a morphism of schemes. We say $f$ is
\define{Quasi-Compact} if $f^{-1}(V)$ is quasi-compact for all affine
open subsets $V\subset Y$.
\end{definition}

\begin{definition}
Let $f\colon X\to Y$ be a morphism of schemes. We say $f$ is
\define{Locally of Finite Type} if $Y$ has an affine cover
$Y=\bigcup_{i}V_{i}$ where each $V_{i}\iso\Spec(A_{i})$ and
$f^{-1}(V_{i})=\bigcup_{j}U_{ij}$ with $U_{ij}\iso\Spec(B_{ij})$ such
that $A_{i}\to B_{ij}$ makes $B_{ij}$ into a finitely-generated
$A_{i}$-algebra.
\end{definition}

\begin{definition}
Let $f\colon X\to Y$ be a morphism of schemes. We say $f$ is
\define{Finite Type} if $f$ is locally of finite type and $f$ is quasi-compact
(i.e., each open affine may be covered with finitely many affine opens).
\end{definition}

\begin{definition}
Let $f\colon X\to Y$ be a morphism of schemes. We say $f$ is
\define{Finite} if there exists an open cover of $Y=\bigcup_{i}V_{i}$
where each $V_{i}\iso\Spec(A_{i})$ such that 
\begin{enumerate}
\item $f^{-1}(V_{i})$ is affine --- so $f^{-1}(V_{i})\iso\Spec(B_{i})$
  for some $B_{i}$;
\item and (???) $A_{i}$-module.
\end{enumerate}
\end{definition}

\subsection{Quasi-coherent sheaves}

\begin{definition}
Let $(X,\sheaf{O}_{X})$ be a ringed space. Then a \define{Sheaf of $\sheaf{O}_{X}$-modules}
is a sheaf $\sheaf{F}$ of Abelian groups such that for any open subset
$U\subset X$, the Abelian group $\sheaf{F}(U)$ has an
$\sheaf{O}_{X}(U)$-module structure for which if $V\subset U$ and
$f\in\sheaf{O}_{X}(U)$, $s\in\sheaf{F}(U)$, then $(f\cdot
s)|_{V}=(f|_{V})\cdot(s|_{V})$. 
\end{definition}

\begin{node}[Vector bundles]
In topology, $\pi\colon E\to X$ is a bundle such that for each $x\in X$
the fiber $\pi^{-1}(x)$ is a real vector space. We can define an
associated sheaf (of sections):
\begin{equation}
\sheaf{E}(U)=\{s\colon U\to E\mid s\circ\pi=\id\}.
\end{equation}
This is an example of a sheaf of $\sheaf{O}_{X}$-modules.

Moreover, $X$ has a cover $\bigcup_{i}U_{i}=X$ where
$\pi^{-1}(U_{i})\iso U_{i}\times\RR^{n}$. The sheaf version of this is
\begin{equation}
\sheaf{E}|_{U_{i}}\iso\sheaf{O}_{U_{i}}^{\oplus n}.
\end{equation}
We say this is a \define{Locally-Free} sheaf.
\end{node}

\begin{construction}
Let $R$ be any ring and $M$ be an $R$-module. Let $X=\Spec(R)$. We
define a sheaf of $\sheaf{O}_{X}$-modules $\widetilde{M}$ by
$\widetilde{M}(D(f))=M_{f}$ where $M_{f}$ is an
$\sheaf{O}_{X}(D(f))$-module since $\sheaf{O}_{X}(D(f))=R_{f}$
\end{construction}

\begin{lemma}
$M_{f}$ gives a sheaf on the base.
\end{lemma}

\begin{definition}
Let $X$ be a scheme and $\sheaf{F}$ a sheaf of
$\sheaf{O}_{X}$-modules. We say $\sheaf{F}$ is \define{Quasi-Coherent}
if $X$ has an open cover $X=\bigcup_{i}U_{i}$ with
$U_{i}\iso\Spec(R_{i})$ and for all $i$ we have
$\sheaf{F}|_{U_{i}}\iso\widetilde{M}_{i}$ for some $R_{i}$-module $M_{i}$.
\end{definition}

\begin{proposition}[Quasi-coherent is a local property]
$\sheaf{F}$ is quasi-coherent if and only if for all affine open
  subsets $U\subset X$ we have $\sheaf{F}|_{U}\iso\widetilde{M}$ for
  some $\sheaf{O}_{X}(U)$-module $M$.
\end{proposition}