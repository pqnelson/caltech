%%
%% fall-lecture09.tex
%% 
%% Made by Alex Nelson <pqnelson@gmail.com>
%% Login   <alex@lisp>
%% 
%% Started on  2025-10-19T11:35:20-0700
%% Last update 2025-10-19T11:35:20-0700
%% 

\lecture{}

\begin{definition}[Projective varieties]
We define a \define{Projective Variety} $Y=V_{p}(J)\subset\PP^{n}$.
\end{definition}

\begin{remark}
  We have some familiar results holds for the projective setting.
\begin{enumerate}
\item $V_{p}(J_{1})\cup V_{p}(J_{2})=V_{p}(J_{1}J_{2})=V_{p}(J_{1}\cap J_{2})$
\item $V_{p}(J_{1}\cap V_{p}(J_{2})=V_{p}(J_{1}+J_{2})$
\item $I_{p}(X_{1}\cap X_{2})=\Radical{I_{p}(X_{1})+I_{p}(X_{2})}$
  provided the right-hand side is not the irrelevant ideal
\item $I_{p}(X_{1}\cup X_{2})=I_{p}(X_{1})\cap I_{p}(X_{2})$.
\end{enumerate}
\end{remark}

\begin{definition}[Homogeneous coordinate ring]% 6.22 of Gathmann
Let $Y\subset\PP^{n}$ be a projective variety. We define the 
\define{Homogeneous Coordinate Ring} of $Y$ to be the [graded] ring
$\kk[x_{0},x_{1},\dots,x_{n}]/I(Y)$ but these are not sets of
well-defined functions. Regardless, we can still speak meaningfully of
their zero locus.
\end{definition}

\begin{node}
Closed subsets given by the vanishing of a homogeneous ideal. We have
schematically something like (for $\PP^{2}$):
\begin{equation*}
\includegraphics{img/img.2}
\end{equation*}
Recall from gluing, if $X\subset\PP^{2}$ is closed, then $X\cap\AA^{2}_{x}$
and $X\cap\AA^{2}_{y}$ and $X\cap\AA^{2}_{z}$ are all closed. Then
$X\cap\AA^{2}_{\alpha}=V_{a}(f_{\alpha}(y/x,z/x),\alpha\in I)$ implies
$x^{\deg(f_{\alpha})}f_{\alpha}(y/x,z/x)=0$ on $X$ is a homogeneous polynomial.
\end{node}

\begin{definition}[Homogeneization of polynomial]% 6.25 of Gathmann
Let
\begin{equation}
f = \sum_{i_{1},\dots,i_{n}\in\NN}a_{i_{1},\dots,i_{n}}x^{i_{1}}(\cdots)x^{i_{n}},
\end{equation}
then we define its \define{Homogeneization} by introducing a new
variable $x_{0}$ and
\begin{equation}
f^{h} = x_{0}^{\deg(f)}f\left(\frac{x_{1}}{x_{0}},\dots,\frac{x_{n}}{x_{0}}\right),
\end{equation}
which is a homogeneous polynomial of degree $d$.
\end{definition}

\begin{remark}
  Note:
\begin{enumerate}
\item $(f+g)^{h}\neq f^{h}+g^{h}$, but $(fg)^{h}=f^{h}g^{h}$.
\item If $J\ideal\kk[x_{1},\dots,x_{n}]$ is an ideal, then its
  homogeneization is generated by the homogeneization of all its
  elements $J^{h}=\langle f^{h}\mid f\in J\rangle$.
\end{enumerate}
\end{remark}

\begin{proposition}[Computing the projective closure]% 6.32 of Gathmann
Let $J\ideal\kk[x_{1},\dots,x_{n}]$ and $X=V_{\text{affine}}(J)\subset\AA^{n}$.
We can consider $\closure{X}\subset\PP^{n}$ since $\AA^{n}\subset\PP^{n}$.
Then we have:
\begin{enumerate}
\item $\closure{X}=V_{p}(J^{h})$
\item If $J=\langle f\rangle$ is a hypersurface, then $\closure{X}=V_{p}(f^{h})$.
\end{enumerate}
\end{proposition}

\begin{remark}[$\PP^{n}$ is irreducible of dimension $n$]% 6.28 of Gathmann
$\PP^{n}$ is irreducible of dimension $n$.
Moreover, $\PP^{n}$ is separated: $\Delta\subset\PP^{n}\times\PP^{n}$
consisting of points
$\bigl((x_{0}:\dots:x_{n}),(y_{0}:\dots:y_{n})\bigr)$ and
cut out by $x_{i}y_{j}=x_{j}y_{i}$. We check this on affine charts,
which implies $\Delta$ is closed.
\end{remark}

\begin{definition}
A \define{Quasiprojective Variety} is an open subset of projective
variety $U\subset X\subset\PP^{n}$ where $U$ is a variety.
\end{definition}

\begin{node}[Structure sheaf]
We want to understand the structure sheaf. From the gluing
perspective, $\StructureSheaf_{\PP^{n}}(U)$ but the only regular
functions which make sense look like $\varphi\colon U\to\kk$ where
$\varphi$ locally is rational whose numerators and denominators are
homogeneous polynomials.
\begin{equation}
\StructureSheaf_{\PP^{n}}(U)=\{\varphi\colon U\to\kk\mid\forall a\in
U\exists d\in\NN, f,g,\in\kk[x_{0},\dots,x_{n}]_{d},
f(x)\neq0\land\varphi(x)=\frac{g(x)}{f(x)}\forall x\in U_{a}\subset
U\mbox{ open}\}.
\end{equation}
If $X\subset\PP^{n}$ closed, allow by quotienting by homogeneous ideal
$I_{p}(X)$, so now we get
\begin{equation}
\StructureSheaf_{\PP^{n}}(U)=\{\varphi\colon U\to\kk\mid\forall a\in
U\exists d\in\NN, f,g,\in S(X)_{d},
f(x)\neq0\land\varphi(x)=\frac{g(x)}{f(x)}\forall x\in U_{a}\subset
U\mbox{ open}\}.
\end{equation}
\end{node}

\begin{lemma}[Morphisms of projective varieties]% 7.4, Gathmann
Let $X\subset\PP^{n}$ be a closed projective variety, and $f_{0}$,
\dots, $f_{m}\in S(X)$ be homogeneous polynomials of the same degree.
Then we may define a morphism $f\colon U\to\PP^{m}$, where
$U:=X\setminus V_{p}(f_{0},\dots,f_{m})$, sending $x\mapsto(f_{0}(x):\dots:f_{m}(x))$.

Observe $\lambda x\mapsto\lambda^{d}f(x)$. We can check this defines a
morphism on each affine chart.
\end{lemma}

\begin{example}
Let $A\subset\GL(n+1,\kk)$ be an invertible matrix. Then
$f\colon\PP^{n}\to\PP^{n}$ defined by $f(x)=Ax$ is an automorphism of $\PP^{n}$.
\end{example}

\begin{example}
Let $\varphi\colon V\to W$ be a linear transformation of vector spaces
$V$ and $W$. Then $\PP(V)\setminus\PP(\ker(\varphi))\to\PP(W)$ is a
morphism. If $\dim(\ker(\varphi))=1$, then this is a
\define{Projection from a Point}.
\end{example}

\begin{example}
Let $X=V_{p}(xz-y^{2})\subset\PP^{2}$. Observe this is the same thing
as the zeroes of
\begin{equation}
g(x,y,z)=\det\begin{bmatrix}x & y\\y & z
\end{bmatrix}.
\end{equation}
That's equivalent to saying this matrix is rank 1. Then
$f\colon X\to\PP^{1}$ defined by
\begin{equation}
(x:y:z)\mapsto\begin{cases}(x:y) & \mbox{if }(x:y:z)\neq(0:0:1)\\
(y:z) & \mbox{if }(x:y:z)\neq(1:0:0)
  \end{cases}
\end{equation}
which may be viewed as gluing together two projections from a
point. This $f$ is an isomorphism. We have $f^{-1}\colon\PP^{1}\to X$
defined by $f(s,t)=(s^{2}:st:t^{2})$ since 
\begin{equation}
(s:t)\mapsto(s:t:\frac{t^{2}}{s})
\end{equation}
almost works but we need to clear the denominators.
\end{example}

\begin{example}[Segre embedding]\label{ex:fall-lec09:segre-embedding}% 7.9 of Gathmann
Let $V$ and $W$ be vector spaces. Consider the mapping
\begin{equation}
\begin{split}
\PP(V)\times\PP(W)\to\PP(V\otimes W)\\
(\langle v\rangle,\langle w\rangle)\mapsto\langle v\otimes w\rangle.
\end{split}
\end{equation}
If we pick a basis $e_{i}$ on $V$ and $h_{j}$ on $W$, then this
extends linearly to
\begin{equation}
(\sum_{i}a_{i}e_{i},\sum_{j}b_{j}h_{j})\mapsto\sum_{i,j}a_{i}b_{j}\;e_{i}\otimes h_{j}.
\end{equation}
If we pick coordinates in the image $z_{ij}$, they must obey
\begin{equation}
z_{ij}z_{k\ell}=z_{ik}z_{j\ell}.
\end{equation}
On the other hand, if we have points in the image satisfying these
relations, then we an recover quotients of $a_{i}/a_{i'}=z_{ij}/z_{i'j}=z_{ik}/z_{i'k}$.
This is called the \define{Segre Embedding}. These coordinates
$z_{ij}$ are called \define{Segre Coordinates}.
\end{example}

\begin{proposition}% 7.10 of Gathmann
Let $f\colon\PP^{n}\times\PP^{m}\to\PP^{N}$ be the Segre embedding, where $N=(n+1)(m+1)-1$.
Then
\begin{enumerate}
\item The image $X=f(\PP^{n}\times\PP^{m})$ is $X=V_{p}(z_{ij}z_{k\ell}-z_{i\ell}z_{kj})$.
\item The map $f\colon\PP^{n}\times\PP^{m}\to X$ is an isomorphism
  onto its image.
\end{enumerate}
\end{proposition}