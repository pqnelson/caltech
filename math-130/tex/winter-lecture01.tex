%%
%% winter-lecture01.tex
%% 
%% Made by Alex Nelson <pqnelson@gmail.com>
%% Login   <alex@lisp>
%% 
%% Started on  2026-01-15T12:59:36-0800
%% Last update 2026-01-15T12:59:36-0800
%% 

\lecture[Defining Spec(R)]{}

\begin{node}
I missed this lecture, but it was about affine schemes and schemes.
The textbooks for this quarter are:
\begin{enumerate}
\item Hartshorne's \textit{Algebraic Geometry}~\cite{hartshorne1977algebraic},
  chapter 2
\item Vakil's \textit{Rising Sea}~\cite{vakil2025rising}.
\end{enumerate}
I am also relying upon G{\"o}rtz and Wedhorn~\cite{gortz2020algebraic}.
\end{node}

\begin{aside}
Truthfully, the lecture experience seems to be\dots well, something to
be \emph{experienced} and impossible to describe (or capture in text).
The notes I transcribed are nowhere nearly as useful as last
quarter (the transcribed notes are more subjectively useful).
Consequently, I think I am going to try to summarize my
reading notes to complement the lectures. Each ``lecture'' will
probably be larger than the actual contents of the lecture.
\end{aside}

\begin{node}[Big idea]
The big idea is that we will be turning $\Spec(R)$ into a ringed
space, and gluing a bunch of them together to form schemes. Just as
varieties were glued together from affine varieties, we glue ``affine schemes''
together to form a scheme. For this reason, people will call
$\Spec(R)$ an \define{Affine Scheme}.

We will need to develop analogous infrastructure we developed for
algebraic varieties: zero sets for $\Spec(R)$, ideals of $X\subset\Spec(R)$,
a Nullstellensatz for relating these guys, as well as the Zariski
topology for $\Spec(R)$ and so on.
\end{node}

%% \begin{proposition}\label{d-f-forms-basis-for-zariski-topology}% Gortz, proposition 2.5
%% Let $R$ be a [commutative unital associative] ring.
%% Then the collection $\{D(f)\mid f\in R\}$ forms a basis for the
%% Zariski topology on $\Spec(R)$.
%% \end{proposition}

\begin{fact}
Let $R$ be a commutative ring.
A nonzero principal ideal of $R$ is prime if and only if it is
generated by a prime element of $R$.
\end{fact}

\begin{notation}
Recall, if $R$ is a commutative ring and $f\in R$, then we often
denote by $R_{f}$ the localization using $S=\{f^{n}\in R\mid n\in\NN_{0}\}$,
i.e., $R_{f}:=S^{-1}R$.
\end{notation}

\begin{node}[``Functions'' of $\Spec(R)$]% Gathmann 12.4, Vakil 3.2.1
Let $R$ be a commutative unital ring. We can view an element $f\in R$
as a function on $\Spec(R)$ in the following sense: for any $P\in\Spec(R)$,
we have the residue field $K(P)$ of $\Spec(R)$ at $P$ (i.e., $K(P)$ is
the quotient field of the integral domain $R/P$), and then the
canonical mappings $R\to R/P\to K(P)$ applied to $f\in R$ is how we
define $f(P)$. In particular, $f(P)=0$ if and only if $f\in P$.

This gives us the analogue of polynomial functions on $\Spec(R)$, just
as we studied the Spec of the coordinate ring for a variety.
\end{node}

\begin{definition}% Gathmann, 12.6
Let $R$ be a commutative ring. Let $S\subset R$.
We define the \define{Zero Locus} of $S$ to be the subset
\begin{equation}
V(S):=\{P\in\Spec(R)\mid \forall f\in S\ldotp f(P)=0\}\subset\Spec(R).
\end{equation}
Equivalently, since $f(P)=0$ if and only if $f\in P$, we could use the
definition
\begin{equation}
V(S)=\{P\in\Spec(R)\mid S\subset P\}.
\end{equation}
As usual, if $S=\{f_{1},\dots,f_{n}\}$, then we'll just write
$V(f_{1},\dots,f_{n})$ instead of $V(\{f_{1},\dots,f_{n}\})$.
\end{definition}

\begin{definition}
The \define{Zariski Topology} on $\Spec(R)$ is the topology whose
closed sets are of the form $V(S)=\{P\in\Spec(R)\mid S\subset P\}$
for some $S\subset R$.
\end{definition}

\begin{proposition}
Let $R$ be a commutative ring. Let $S\subset R$.
Then the zero set of $S$ is the same as the zero set of the ideal it
generates, i.e., $V(S)=V(\langle S\rangle)$.
\end{proposition}

\begin{proof}
This is because any $P\in\Spec(R)$ has $S\subset P$ if and only if
$\langle S\rangle\subset P$. The forward direction is obvious, the
backward direction uses the fact
\begin{equation}
\langle S\rangle = \bigcap_{S\subset I\ideal R}I,
\end{equation}
which means for \emph{any} ideal $I\ideal R$ with $S\subset I$ we have
$\langle S\rangle\subset I$. Hence the backwards direction.
\end{proof}

\begin{definition}% Gathmann 12.6
Let $R$ be a commutative unital ring, let $X\subset R$.
We define the \define{Ideal} of $X$ to be the ideal of $R$ equal to
\begin{equation}
I(X):=\{f\in R\mid\forall P\in X\ldotp f(P)=0\}\ideal R.
\end{equation}
Equivalently, since $f(P)=0$ if and only if $f\in P$, we could write
this as
\begin{equation}
I(X) = \bigcap_{P\in X}P.
\end{equation}
\end{definition}

\begin{proposition}[Scheme nullstellensatz]% Gathmann, 12.10
Let $R$ be a commutative unital ring.
\begin{enumerate}
\item For any closed subset $X\subset\Spec(R)$, we have $V(I(X))=X$.
\item For any ideal $J\ideal R$ we have $I(V(J))=\Radical{J}$.
\end{enumerate}
\end{proposition}

\begin{proposition}
For any ideals $J_{1},J_{2}\ideal R$, we have
\begin{subequations}
\begin{equation}
V(J_{1})\cup V(J_{2})=V(J_{1}J_{2})=V(J_{1}\cap J_{2}),
\end{equation}
and
\begin{equation}
V(J_{1})\cap V(J_{2})=V(J_{1}+J_{2})
\end{equation}
\end{subequations}
in $\Spec(R)$.
\end{proposition}

\begin{proposition}
For any two closed subsets $X_{1},X_{2}\subset\Spec(R)$, we have
\begin{subequations}
\begin{equation}
I(X_{1}\cup X_{2})=I(X_{1})\cap I(X_{2}),
\end{equation}
and
\begin{equation}
I(X_{1}\cap X_{2})=\Radical{I(X_{1})+I(X_{2})}.
\end{equation}
\end{subequations}
\end{proposition}

The punchline (from these three previous propositions) is that $V(-)$
and $I(-)$ behave ``as we expect''.

\begin{definition}
Let $R$ be a [unital commutative] Ring, let $f\in R$.
Then the \define{Distinguished Open Subset} of $f$ in $\Spec(R)$ is
the open subset
\begin{equation}
D(f):=\Spec(R)\setminus V(f)=\{P\in\Spec(R)\mid f\notin P\}.
\end{equation}
(This really is open, since $V(f)$ is closed, and $D(f)$ is just the
complement of $V(f)$.)
\end{definition}

\begin{proposition}\label{d-f-forms-basis-for-zariski-topology}% Gortz, proposition 2.5
The collection of distinguished open subsets
\begin{equation*}
\mathcal{B}=\{D(f)\subset\Spec(R)\mid f\in R\}
\end{equation*}
forms a basis for the Zariski topology of $\Spec(R)$.
\end{proposition}

(This means that distinguished open subsets for affine schemes are
like those for affine varieties.)

\begin{proof}
For any open subset $U\subset\Spec(R)$, we know $U=\Spec(R)\setminus V(S)$
for some $S\subset R$. We see then that
\begin{equation}
U=\Spec(R)\setminus\left(\bigcap_{f\in S}V(f)\right)=\bigcup_{f\in S}\bigl(\Spec(R)\setminus V(f)\bigr)=\bigcup_{f\in S}D(f),
\end{equation}
which proves the claim.
\end{proof}

\endinput