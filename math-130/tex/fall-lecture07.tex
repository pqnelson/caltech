%%
%% fall-lecture07.tex
%% 
%% Made by Alex Nelson <pqnelson@gmail.com>
%% Login   <alex@lisp>
%% 
%% Started on  2025-10-14T10:37:35-0700
%% Last update 2025-10-14T10:37:35-0700
%% 

\lecture{}

\subsection{Morphisms}

We can think of regularr functions we previously introduced as a
special case of the more general ``morphism'', specifically as a
morphism from an open subset $U\subset X$ of an affine variety $X$ to
the affine line $\AA^{1}=\kk$.

\begin{definition}[Ringed spaces]
A \define{Ringed Space} consists of a topological space $X$ equipped
with a \define{Structure Sheaf} $\StructureSheaf_{X}$ which is a sheaf
of rings.
\end{definition}

\begin{example}
Let $X$ be an affine variety, let $\RegularFuns_{X}$ be the sheaf of
regular functions. If $X$ is a ringed space, then any open subset
$U\subset X$ is a ringed space (where we restrict $\RegularFuns_{X}|_{U}=\RegularFuns_{U}$).
\end{example}

\begin{remark}
For varieties, and only for varieties, is a sheaf of rings is a sheaf
of functions to $\kk$. Affine schemes do not enjoy this property (nor,
do I think, schemes in general).
\end{remark}

\begin{definition}
\begin{enumerate}
\item Let $f\colon X\to Y$ be a function of ringed spaces. If $U\subset Y$
is an open subset and $\varphi\colon U\to\kk$ is any function, we denote
by $f^{*}\varphi=\varphi\circ f\colon f^{-1}(U)\to\kk$ the usual pullback.
\item We say $f$ is a \define{Morphism} of ringed spaces if
  \begin{enumerate}[label=(\roman*)]
  \item $f$ is continuous, and
  \item for all open subsets $U\subset Y$ and any
    $\varphi\in\StructureSheaf_{Y}(U)$, we have $f^{*}\varphi\in\StructureSheaf_{X}(f^{-1}(U))$.
  \end{enumerate}
\item Furthermore, a morphism $f$ is called an \define{Isomorphism}
  (of ringed spaces) if it has a two-sided inverse.
\end{enumerate}
\end{definition}

\begin{remark}
Composition of morphisms (of ringed spaces) are morphisms (of ringed spaces).
Restrictions of morphisms (of ringed spaces) are morphisms (of ringed spaces).
\end{remark}

\begin{lemma}[Gluing]% Gathmann 4.6
Let $f\colon X\to Y$ be a map of ringed spaces, let $\{U_{i}\}_{i\in I}$
be an open cover of $X$.
If $f|_{U_{i}}\colon U_{i}\to Y$ is a morphism (of ringed spaces) for
each $i\in I$, then $f$ is a morphism.
\end{lemma}

This lemma tells us that ``morphism of ringed spaces'' is a local
thing.

\begin{proof}[Proof outline]
Step 1: Continuity is a local property.

Step 2: use gluing property for sheaves.
\end{proof}

\begin{proposition}[Morphisms between affine varieties]% Gathmann 4.7
Let $U\subset X$ be an open subset of an affine variety $X$,
let $Y\subset\AA^{n}$ be another affine variety.
Then the morphisms $f\colon U\to Y$ are maps of the form
$f=(\varphi_{1},\dots,\varphi_{n})$ --- i.e., $n$-tuples of $n$
regular functions over $U$ which are each of the $\varphi_{i}\in\RegularFuns_{X}(U)$ for $i=1,\dots,n$.
\end{proposition}

\begin{node}
We have an equivalence of categories, namely (1) the category of
affine varieties over $\kk$, and (2) reduced finitely-generated
$\kk$-algebras (recall: ``reduced'' means there are no nilpotent elements).

Recall we say categories $\cat{C}$ and $\cat{D}$ are equivalent if
there are functors $F\colon\cat{C}\to\cat{D}$ and $G\colon\cat{D}\to\cat{C}$
such that there are natural isomorphisms with the identity functors $\cat{G}\circ\cat{F}\iso\id_{\cat{C}}$
and $\cat{F}\circ\cat{G}\iso\id_{\cat{D}}$.
\end{node}

\begin{example}[Bijective morphism which is not an isomorphism]% Gathmann, 4.9 
Let $X=V(x^{2}-y^{3})\subset\AA^{2}$ be an affine variety.
Consider the morphism
\begin{equation}
f\colon\AA^{1}\to X
\end{equation}
defined by $f(t)=(t^{3},t^{2})$. Then $f$ is bijective, but its
inverse is not a morphism since
\begin{equation}
f^{-1}\colon X\to\AA^{1}
\end{equation}
is defined by
\begin{equation}
f^{-1}(x,y) = \begin{cases}\frac{x}{y} & \mbox{if }y\neq0\\
0 &\mbox{otherwise}
\end{cases}
\end{equation}
This is not a morphism because $X\not\iso\AA^{1}$, $\CC[x,y]/\langle x^{2}-y^{3}\rangle\not\iso\CC[t]$.
\end{example}

\begin{proposition}[Universal property of product of affine varieties]% Gathmann, 4.10
Let $X$ and $Y$ be affine varieties.
Then for any affine variety $Z$ and morphisms $f\colon Z\to X$ and
$g\colon Z\to Y$, then there exists a unique morphism $Z\to X\times Y$
such that the following diagram commutes:
\begin{equation}
\xymatrix{Z\ar@{-->}[dr]\ar@/^/[drr]^{g}\ar@/_/[ddr]_{f} & &\\
  & X\times Y\ar[r]^{\pi_{2}}\ar[d]^{\pi_{1}} & Y\\
  & X & }
\end{equation}
\end{proposition}

\begin{proof}[Proof sketch]
We see
\begin{equation*}
  \xymatrix{
 A(X\times Y)\ar[r]^{\iso} & A(X)\otimes_{\kk}A(Y) & \ar[l]A(Y)\\
 & A(X)\ar[u] & \kk\ar[l]\ar[u]}
\end{equation*}
which gives the result as a consequence of the equivalence of
categories.
\end{proof}

\begin{proposition}% Gathmann, 4.17
Let $X$ ba an affine variety, let $f\in A(X)$.
Then $D(f)$ is an affine variety and $A(D(f))\iso A(X)_{f}$.
\end{proposition}

\subsection{Varieties}

\begin{definition}% Gathmann, 5.1
A \define{Prevariety} is a ringed space $X$ such that has a finite open
cover by affine varieties.
\end{definition}

\begin{remark}[Prevariety using atlases]
Brian Osserman points out, in Chapter 5 of his book on varieties~\cite{osserman2021concise}, that
a prevariety is a ringed space equipped with an atlas whose charts are
locally affine varieties.

Using the Baez--Dolan ``Stuff, Structure, Properties'' template for
definitions, Osserman's definition of prevarieties would add an Atlas
as part of the ``stuff'' or ``structure''. But Gathmann simply makes a
prevariety a ringed space with an extra ``property''.
This would make a difference for the definition of morphisms between
prevarieties. 
\end{remark}

\begin{example}% Gathmann, 5.3
Let $X$ be an affine variety, let $U\subset X$ be an open set. Then
$U$ is a prevariety.
\end{example}

\begin{construction}[Gluing]% Gathmann, 5.4
Let $X_{1}$, $X_{2}$ be prevarieties.
Let $U_{1,2}\subset X_{1}$ and $U_{2,1}\subset X_{2}$ be open subsets.
Let $f\colon U_{1,2}\to U_{2,1}$ be an isomorphism. Schematically, we
have the following diagram:
\begin{equation*}
\includegraphics{img/img.0}
\end{equation*}
Then we can glue them together and $U\subset X$ is open if
$i_{1}^{-1}(U)\subset X_{1}$ and $i_{2}^{-1}(U)\subset X_{2}$ are both open.
We have
\begin{equation}
\RegularFuns_{X}(U)=\{\varphi\colon U\to\kk\mid i_{1}^{*}\varphi\in\RegularFuns_{X}(i_{1}^{-1}(U)),i_{2}^{*}\varphi\in\RegularFuns_{X}(i_{2}^{-1}(U))\}
\end{equation}
\end{construction}
