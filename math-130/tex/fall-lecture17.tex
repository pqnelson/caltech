%%
%% fall-lecture17.tex
%% 
%% Made by Alex Nelson <pqnelson@gmail.com>
%% Login   <alex@lisp>
%% 
%% Started on  2025-11-06T12:24:36-0800
%% Last update 2025-11-06T12:24:36-0800
%% 

\lecture{}

We continue with the classification of smooth projective
curves. Recall the key Construction~\ref{construction:fall-lec15:key}:

\begin{construction}
Let $f\in K\setminus\kk$. Then $\kk(f)\subset K$ must be an algebraic extension.
Set $B=\closure{\kk[f]}$ be the integral closure of $\kk[f]$ in $K$.
Then $B$ must be a finitely-generated $\kk$-algebra (hence Noetherian),
and moreover $B$ must be a Dedekind domain.
Its field of fractions is $B_{0}=K$.
\end{construction}

\begin{proposition}
Let $X=\MSpec(B)$ be a nonsingular curve. We have two bijective correspondences:
\begin{enumerate}
\item Points in $X$ $\stackrel{1:1}{\longleftrightarrow}$ Discrete
  Valuation Rings $R$ of $K/\kk$ such that $f\in R$ --- the bijective
  correspondence is given by $p\mapsto\RegularFuns_{X,p}$
\item Points in $V(f)$ $\stackrel{1:1}{\longleftrightarrow}$ Discrete
  valuation rings $R$ of $K/\kk$ such that $f\in\mathfrak{m}_{R}$.
\end{enumerate}
\end{proposition}

\begin{proof}
\begin{enumerate}
\item This map $p\mapsto\RegularFuns_{X,p}$ is clearly bijective.
  We want to show it is injective.
  Let $R$ be a discrete valuation ring such that $f\in R$.
  Then $\kk[f]\subset R$ which implies $B\subset R$ since $R$ is integrally closed.
  Taking $M:= B\cap\mathfrak{m}_{R}$ gives us a prime ideal in $B$.
  Localizing $B_{M}\subset R$.
  Then $M\neq0$, so $B_{M}$ is a discrete valuation ring of $K/\kk$.
  We know $B_{M}\cap\mathfrak{m}_{R}$ is the maximal ideal of $B_{M}$
  (so $B_{M}$ is dominated by $R$).
  This implies $B_{M}=R$.
\item This follows from (1).\qedhere
\end{enumerate}
\end{proof}

\begin{corollary}
Every discrete valuation ring $R$ of $K/\kk$ is the local ring of a
nonsingular curve at a point. In particular, the residue field of $R/\mathfrak{m}_{R}\iso\kk$.
\end{corollary}

\begin{proof}
Let $f\in R\setminus\kk$, $B=\closure{\kk[f]}\subset K$ be the
integral closure of $\kk[f]$. Then $R=B_{P}$ for some maximal ideal $P\in\MSpec(B)$.
That is just the local ring of $\MSpec(B)$ at $P$.
\end{proof}

\begin{corollary}\label{fall-lec17:cor2}
Given $f\in K^{*}$, there exists finitely many discrete valuation
rings $R$ of $K/\kk$ such that $f\in\mathfrak{m}_{R}$. Also there are
finitely many $f\notin R$.
\end{corollary}

\begin{proof}
Without loss of generality, we may stipulate $f\notin\kk$. Taking
$B:=\closure{\kk[f]}$ (integral closure of $\kk[f]$). Then we know
there is a bijection
$\{R\mid f\in\mathfrak{m}_{R}\}\leftrightarrow V(f)\subset\MSpec(B)$.
By Principal Ideal Theorem, this implies $\dim(V(f))=0$, which then
implies $V(f)$ is a finite set. (This proves the first claim.)

Now, to prove there are finitely many discrete valuation rings $R$ of
$K/\kk$ such that $f\notin\mathfrak{m}_{R}$. Taking $1/f\in\mathfrak{m}_{R}$
iff $f\notin R$, then apply the first claim.
\end{proof}

\begin{definition}
We define
\begin{equation}
C_{K}:=\{\mbox{discrete valuation rings of }K/\kk\}.
\end{equation}
We think of its elements as points. So every point $p$ in $C_{K}$ is
in a bijective correspondence with discrete valuation rings
$R_{p}\subset C_{K}$ with maximal ideal $\mathfrak{m}_{p}$.
\end{definition}

\begin{proposition}
The $C_{K}$ is a ringed space.
\end{proposition}

\begin{proof}
\textsc{Topology}: closed proper subsets are finite subsets of
$C_{K}$. This gives us the topology.

\textsc{Sheaf}: Let  $f\in K$, $p\in C_{K}$. If $f\in R_{p}$
(``$f$ is defined at $p$''), then set $f(p)$ to be the image of $f$
under the canonical morphism $R_{p}\to R_{p}/\mathfrak{m}_{p}$. So we
have $f(p):=f\bmod{p}$.

\textsc{Regular functions}: if we have $\emptyset\neq U\subset C_{K}$ open,
then
\begin{equation*}
\RegularFuns_{X}(U):=\bigcap_{p\in U}R_{p}\subset K.\qedhere
\end{equation*}
\end{proof}

\begin{remark}
If $U\subset C_{K}$ open and a regular function $f\in\kk[U]$, then
\begin{equation}
D(f)=\{p\in U\mid f(p)\neq0\}=\{p\in U\mid f\notin\mathfrak{m}_{p}\}
\end{equation}
is the complement of a finite set, hence $D(f)$ is open.
\end{remark}

\begin{example}
Let $K=\kk(t)$, then Discrete valuation rings of $K/\kk$ look like either
\begin{enumerate}
\item $\kk[t]_{(t-a)}$ for $a\in\kk$ (which is bijective with $a\in\AA^{1}$), or
\item $\kk[t^{-1}]_{(t^{-1})}$ (which is bijective with the point $\infty\in\PP^{1}$).
\end{enumerate}
This gives us a bijection with $C_{K}\stackrel{1:1}{\longleftrightarrow}\PP^{1}_{\kk}$,
which means $f(\kk[t]_{(t-a)})=f(a)$ as we expect.
\end{example}

\begin{theorem}
The curve $C_{K}$ is a nonsingular curve whose function field is $\FunField[\kk]{C_{K}}=K$.
\end{theorem}

\begin{proof}
\textsc{Claim 1: Prevariety.} That is to say, $C_{K}$ has a finite affine cover.
Take $f\in K\setminus\kk$, with $B=\closure{\kk[f]}$ the integral
closure in $K$. Set
\begin{equation}
U:=\{p\in C_{K}\mid f\in R_{p}\}\subset C_{K}.
\end{equation}
We see $U$ is open by Corollary~\ref{fall-lec17:cor2}. Define
$\varphi\colon\MSpec(B)\to U$ by $\varphi(M)=B_{M}$. We proved this is
bijective and a homeomorphism (since proper closed sets on both sides
are just finite sets).

For $V\subset\MSpec(B)$ open, then
\begin{equation}
\kk[V] = \bigcap_{M\in V}B_{M} = \bigcap_{p\in\varphi(V)}R_{p}=\kk[\varphi(V)].
\end{equation}
Hence $\varphi$ is an isomorphism. We found an open affine for $C_{K}$
we only need 2 of them to cover $C_{K}$ --- this cover is given by the
maximal ideals of the integral closures $\MSpec(\closure{\kk[f]})\cup\MSpec(\closure{\kk[f^{-1}]})$
which gives us our open affine cover. Hence $C_{K}$ is a prevariety.

\textsc{Claim 2: Separatedness.}
If $P,Q\in C_{K}$, it suffices (by homework) to show there is an open
affine $U\subset C_{K}$ such that $P\in U$ and $Q\in U$. Take any
$f\in R_{P}\setminus\mathfrak{m}_{P}$ such that $f\notin\kk$. Then
$f,f^{-1}\in R_{P}$. If $f\in R_{Q}$, then $P,Q\in\MSpec(\closure{\kk[f]})$.
Otherwise $f^{-1}\in R_{Q}$, so $P,Q\in\MSpec(\closure{\kk[f^{-1}]})$.
Hence $C_{K}$ is separated (i.e., $C_{K}$ is a variety.

\textsc{Claim 3: Smoothness.} Obvious.

\textsc{Claim 4: $C_{K}$ is projective.}
\end{proof}