%%
%% fall-lecture28.tex
%% 
%% Made by Alex Nelson <pqnelson@gmail.com>
%% Login   <alex@lisp>
%% 
%% Started on  2025-12-04T11:44:03-0800
%% Last update 2025-12-04T11:44:03-0800
%% 

\lecture{}

(Student presentation by Justin Lee.)

\begin{node}
The goal for today: Quillen--Suslin theorem, which states every vector
bundle over $\AA^{n}$ is trivial. This was posed by Serre in FAC 1955,
and resolved by Quillen and Suslin in 1975. It won Quillen the Fields
Medal in 1978.
\end{node}

\endinput

\begin{node}
There are four [equivalent] ways to look at a vector bundles:
\begin{enumerate}
\item Vector bundles over $X$ is some variety $E\xrightarrow{\pi}X$
  such that $\pi^{-1}(\{x\})\iso\AA^{r}$ and there exists some open
  $U\subset X$ such that $x\in U$ and $\pi^{-1}(U)\iso U\times\AA^{r}$
\item We can look at vector bundles as a locally free sheaf,
  specifically an $\StructureSheaf_{X}$-module $M$ such that
  $M|_{U}\iso\StructureSheaf_{U}^{r}$ for some open covering $\{U\}$
\item (???) Let $\sheaf{M}$ be an $\StructureSheaf_{X}$ module which is coherent. If
  $x\in X$, restrict to an open affine $U\ni x$,
  $\sheaf{M}|_{U}\iso\widetilde{M}$ where $\widetilde{M}$ is the
  ``twiddle-ification'' of a $\FunField[\kk]{U}$-module $M$. We look at
  $\CoordRing{\kk}{U}/\mmm$ where $\mmm$ is the maximal ideal corresponding to the
  point $x$, then $\dim_{\CoordRing{\kk}{U}/\mmm}(M/\mmm M)$ is the dimension of
  $M$ at $x$. Then a constant-dimensional $\StructureSheaf_{X}$-module
  is a locally free sheaf.
\item If $X$ is an affine variety, then we can look at a
  $\CoordRing{\kk}{X}$-module $M$ and an $\StructureSheaf_{X}$-module
  $\widetilde{M}$ is locally free iff $M$ is a projective module.
\end{enumerate}
\end{node}

\begin{node}[``Data for a vector bundle''\footnote{The speaker kept
    calling random things ``data needed for a vector bundle'', it was
    unfortunately rather opaque what he hoped to require.}]
If we have some open cover $\{U_{\alpha}\}$ of $X$ which is a local
trivialization $\pi^{-1}(U_{\alpha})\iso U_{\alpha}\times\AA^{r}$, or
equivalently that the following diagram commutes (describing the local trivialization):
\begin{equation}
\vcenter{\xymatrix{\pi^{-1}(U_{\alpha})\ar[rr]^{\varphi_{\alpha}}\ar[rd]_{\pi} & & U_{\alpha}\times\AA^{r}\ar[ld]\\
& U_{\alpha} & }}
\end{equation}
If we let
\begin{equation}
U_{\alpha\beta}=U_{\alpha}\cap U_{\beta},
\end{equation}
then
\begin{equation}
U_{\alpha\beta}\times\AA^{r}\xrightarrow{\varphi_{\beta}^{-1}}\pi^{-1}(U_{\alpha\beta})\xrightarrow{\varphi_{\alpha}}U_{\alpha\beta}\times\AA^{r}
\end{equation}
composes to (the identity?).

For each $x\in U$, $E_{x}:=\pi^{-1}(\{x\})$. If we could determine the
coordinate functions on $U\times\AA^{r}$, then this determines the map
$\varphi$ of vector bundles:
\begin{equation}
\vcenter{\xymatrix{U\times\AA^{r}\ar[rr]^{\varphi}\ar[rd] & & U\times\AA^{r}\ar[ld]\\
& U & }}
\end{equation}
Then the pullback of the coordinate functions
$\pullback{\varphi}(x_{i})$ for $i=1,\dots,r$ determines $\varphi$.
Let $e_{1}$, \dots, $e_{r}$ be a basis of $\AA^{r}$. Then
\begin{equation}
U\to U\times\AA^{r}\to U\times\AA^{r}
\end{equation}
determines the morphism $\varphi$ as an $r\times r$ matrix with
entries in the global sections $\StructureSheaf_{U}(U)$. The $r\times r$
matrix lives in $\GL_{r}(\StructureSheaf_{U}(U))$.
\end{node}

\begin{node}[Transition functions]
We see that the transition functions
\begin{equation}
C_{\alpha\beta}=\varphi_{\alpha}\circ\varphi_{\beta}^{-1}\in\GL_{r}(\StructureSheaf_{U}(U))
\end{equation}
must be such that on $U_{\alpha}\cap U_{\beta}\cap U_{\gamma}$,
\begin{equation}
C_{\alpha\gamma}=C_{\alpha\beta}\circ C_{\beta\gamma}.
\end{equation}
\end{node}

\begin{node}
For each vector bundle $E$, we can produce the sheaf of sections
$\sheaf{L}(E)$. It turns out every locally-free sheaf comes from this
construction. Replacing the vector bundles with free sheaves in the
construction. We obtain a transition function which satisfies the data above.
\end{node}

\begin{node}[Claim]
Quillen--Suslin theorem holds if and only if every finitely-generated
projective module over $\kk[x_{1},\dots,x_{d}]$ is free.
\end{node}

\begin{definition}
Let $R$ be a commutative ring. We say an element $f\in R^{\oplus n}$
(of the free module) is \define{Unimodular} if $f=(f_{1},\dots,f_{n})$
are its components, then the ideal generated by $\langle f_{1},\dots,f_{n}\rangle=(1)$
is irrelevant.
\end{definition}

\begin{node}
The exact sequence
\begin{equation}
\begin{array}{ccccc}
0 & \to & R & \to & R^{n}\\
  &     & 1 & \mapsto & f
\end{array}
\end{equation}
there exists surjective $R^{n}\to R$ iff $f$ is unimodular.
\end{node}

\begin{definition}
Let $f$ be unimodular. We say $f$ has \define{the Unimodular Extension Property}
(or ``has UEP'') if $f$ can be extended into a basis for $R^{n}$.

This is logically equivalent to $R^{n}\to R^{n}$ (obtained by this
extension) is an isomorphism.

This is logically equivalent to there exists $M\in\GL_{n}(R)$ such
that the first column of $M$ is $V$ (???).
\end{definition}

\begin{lemma}
If $R$ is local and $A=R[x]$, if $f$ is unimodular over $A$ and $f$
has at least one monic component, then $f$ has UEP (and $f(0)$ is
unimodular with UEP).
\end{lemma}

\begin{node}
If $f$ has UEP and $M\in\GL_{n}(R)$, then $Mf$ has UEP. In particular,
elementary row operations preserve UEP.
\end{node}

\begin{node}
Then we can write $f=(f_{1},\dots)$ where $f_{1}$ is monic and
remaining components are of smaller degree.
\end{node}

\begin{notation}
We will write $f\sim g$ for unimodular $f$ and $g$ if there is an
$M\in\GL_{n}(R)$ such that $g=Mf$. This is an equivalence relation of
unimodular guys.
\end{notation}

\begin{lemma}
Let $f$ be unimodular over $R[x]$ and $f\sim f(0)$ in $S^{-1}R[x]$
(some localization of $R[x]$). Then there exists $c\in S$ such that
$f(x+cy)\sim f(x)$ in $R[x,y]$.
\end{lemma}

\begin{lemma}
Let $R$ be an integral domain. 
If $f$ is unimodular and has at least one monic component, then $f\sim f(0)$.
\end{lemma}

\begin{lemma}
Let $R=\kk[x_{1},\dots,x_{n}]$. Then any unimodular guy has the UEP.
\end{lemma}

\begin{remark}
Truthfully, at this point, there was a lot of frantic scribbling and
muttering, it was difficult to follow along. The speaker clearly was
intelligent, but it was unclear how competent he was on the
material. I think if he organized his talk better (and practiced it a
few times in front of a mirror), he might have had a decent
presentation. But he repeatedly spoke about ``The data needed for a
vector bundle is the following\dots'' which made it unclear (this
should have been an ``easy win''), and that derailed the rest of his
presentation. 
\end{remark}