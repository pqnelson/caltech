%%
%% fall-lecture22.tex
%% 
%% Made by Alex Nelson <pqnelson@gmail.com>
%% Login   <alex@lisp>
%% 
%% Started on  2025-11-18T12:29:49-0800
%% Last update 2025-11-18T12:29:49-0800
%% 

\lecture{}

\begin{node}
Notice: the $\mathcal{O}(m)$ sheafs are invertible (i.e., locally free
of rank 1). On $U_{i}:=D_{p}(x_{i})\subset\PP^{n}$,
\begin{equation}
\begin{split}
&\StructureSheaf_{U_{i}}\xrightarrow{\iso}\mathcal{O}(m)|_{U_{i}}\\
&h\mapsto x_{i}^{m}h
\end{split}
\end{equation}
induces an isomorphism since everything is generated by $x_{i}^{m}$ locally.
If we take $m=0$, then this just gives us the structure sheaf (by
definition).

The global sections are
\begin{equation}
\Gamma(\PP^{n},\mathcal{O}(m))=\begin{cases}0 & m<0\\
S_{m} & m\geq0.
\end{cases}
\end{equation}
\end{node}

\begin{lemma}
$\mathcal{O}(m)\otimes_{\PP^{n}}\mathcal{O}(p)\iso\mathcal{O}(m+p)$
\end{lemma}

\begin{proof}
Let $U\subset\PP^{n}$ be open. We define
\begin{equation}
\begin{split}
\Gamma(U,\mathcal{O}(m))\otimes\Gamma(U,\mathcal{O}(p))&\to\Gamma(U,\mathcal{O}(m+p))\\
f\otimes g&\mapsto fg,
\end{split}
\end{equation}
then we sheafify, it induces a morphism from the left-hand side to the
right handside $\mathcal{O}(m)\otimes\mathcal{O}(p)\to\mathcal{O}(m+p)$.
When we restrict to $U_{i}$, this is an isomorphism
\begin{equation*}
\mathcal{O}(m)|_{U_{i}}\otimes\mathcal{O}(p)|_{U_{i}}\xrightarrow{\iso}\mathcal{O}(m+p)|_{U_{i}}\qedhere
\end{equation*}
\end{proof}

\begin{note}
There is no nonzero map $\mathcal{O}(m)\to\mathcal{O}(p)$ if $m>p$.
To see this, tensor both sides by $\mathcal{O}(-m)$ to get $0\to\mathcal{O}(p-m)$,
so $\mathcal{O}(m)\iso\mathcal{O}(p)\iff m=p$.
\end{note}

\begin{node}\label{defn:fall2025:lec22:picard-group}
Any invertible sheaf on $\PP^{n}$ is isomorphic to $\mathcal{O}(m)$
for some $m\in\ZZ$. They form an Abelian group
$\Pic(\PP^{n})=\{\mathcal{O}(m)\mid m\in\ZZ\}\iso\ZZ$ called the
\define{Picard group} (which we'll discuss later in this course in
Section~\ref{subsec:fall2025:lec25:picard-group}).
\end{node}

\subsection{Coherent sheaves}

\begin{definition}
Let $X$ be an affine variety, $A=\CoordRing{\kk}{X}$ be the coordinate
ring, let $M$ be a module over $A$. We define the sheaf
$\widetilde{M}$ by
\begin{equation}
\widetilde{M} := \sheafify{[U\mapsto M\otimes_{A}\StructureSheaf_{X}(U)]}.
\end{equation}
Hartshorne calls this the \define{Sheaf Associated} to $M$ (see
Chapter~II \S5).

This is an example of a \define{Quasi-Coherent $\StructureSheaf_{X}$-Module} which is a procedure for turning
modules into sheaves.

NB: $M\otimes_{A}\StructureSheaf_{X}(D(f))=M\otimes_{A}A_{f}=M_{f}$
for $f\in A$.
\end{definition}

\begin{example}
When $M=A$ itself, $\widetilde{A}=\StructureSheaf_{X}$.
\end{example}

\begin{example}
Let $Y\subset X$ be a closed subvariety. Then $I=I(Y)\ideal A$ is an ideal,
so the ideal-sheaf $\IdealSheaf{Y}=\widetilde{I}\subset\StructureSheaf_{X}$,
and $\pushforward{i}\StructureSheaf_{Y}=\widetilde{(A/I)}$.
\end{example}

\begin{node}
$\widetilde{M}_{p}=M_{I(p)}$ where $p\in X$ and $I(p)=I(\{p\})\ideal A$.

We have a map
\begin{equation}
\begin{split}
& M_{I(p)}\to \widetilde{M}_{p}\\
&\frac{m}{f}\mapsto(m\otimes\frac{1}{f})_{p}\in\widetilde{M}(D(f)),
\end{split}
\end{equation}
and further this map is surjective. It is also injective, since if
\begin{equation}
\frac{m}{f}\mapsto 0,
\end{equation}
then
\begin{equation}
(m\otimes\frac{1}{f})_{D(h)}=0
\end{equation}
for some $h\in A\setminus I(p)$. But this means
\begin{equation}
m\otimes\frac{1}{f}=0\in M\otimes_{A} A_{h}=M_{h},
\end{equation}
which implies $h^{N}m=0\in M$. Hence $m/f=0$ in $M_{I(p)}$.
\end{node}

\begin{corollary}
$\widetilde{M}(U)$ is the set of functions $s\colon U\to\bigsqcup_{p\in U}M_{I(p)}$
such that for all $p\in U$, there exists $V\subset U$ and $m\in M$
such that $p\in V$ and for all $f\in A$ and $q\in V$ we have
$s(q)=\frac{m}{f}\in M_{I(q)}$.
\end{corollary}

\begin{proposition}
$\widetilde{M}(D(f))\iso M_{f}$ which recovers
  $\StructureSheaf_{X}(D(f))\iso A_{f}$.
\end{proposition}

\begin{proof}
Consider $M_{f}=M\otimes_{A}\StructureSheaf_{X}(D(f))\xrightarrow{\psi}\widetilde{M}(D(f))$.
It exists by sheafification. We claim $\psi$ is injective and surjective.\marginpar{TODO: finish transcribing the proof}
\end{proof}

\begin{note}
$\Gamma(X,\widetilde{M})=M$.
\end{note}

\begin{definition}
Let $X$ be any variety, let $\sheaf{F}$ be an $\StructureSheaf_{X}$-module.
We call $\sheaf{F}$ \define{Quasi-Coherent} if there exists an open
cover $X=\bigcup_{i}U_{i}$ and $\CoordRing{\kk}{U_{i}}$-module $M_{i}$
such that $\sheaf{F}|_{U_{i}}=\widetilde{M}_{i}$ for each $i$.

If further $M_{i}$ is a finitely-generated
$\CoordRing{\kk}{U_{i}}$-module for each $i$, then $\sheaf{F}$ is
called \define{Coherent}.
\end{definition}

\begin{example}
Locally free $\StructureSheaf_{X}$-modules are coherent.
\end{example}

\begin{example}
Let $Y\subset X$ be a closed subvariety, then the ideal sheaf
$\IdealSheaf{Y}$ and pushforward of the structure sheaf
$\pushforward{i}\StructureSheaf_{Y}$ are coherent $\StructureSheaf_{X}$-modules.
\end{example}

\begin{example}[NON-example]
Let $X=\AA^{1}$,
\begin{equation}
\sheaf{F}(U) = \begin{cases}\StructureSheaf_{X}(U) & \mbox{if }0\notin U\\
0 & \mbox{if }0\in U
\end{cases}
\end{equation}
An ``extension of $\StructureSheaf_{\AA^{1}\setminus0}$ by $0$'' is
clearly a sheaf and an $\StructureSheaf_{X}$-module, but it is not
quasi-coherent. (If it were quasi-coherent, $\sheaf{F}(U)=0$ but
$\sheaf{F}|_{U}\neq0=\widetilde{0}$ for $U\ni0$ open.)
\end{example}

\begin{xca}
Prove $\sheaf{F}$ is quasi-coherent if and only if
$\sheaf{F}|_{U}=\widetilde{\sheaf{F}(U)}$ for all $U\subset X$ open affine.
\end{xca}

\begin{xca}
If $\sheaf{F}$ is coherent, then $\sheaf{F}(U)$ is a
finitely-generated $\CoordRing{\kk}{U}$-module for all $U\subset X$
open affine.
\end{xca}

\begin{example}
Let $M$ be a finitely-generated $A$-module, and let $X=\Spec(A)$.
Then $\widetilde{M}$ is a locally-free $\StructureSheaf_{X}$-module of
rank $r$ if and only if $M$ is a projective $A$-module of constant
rank $r$.
\end{example}

\begin{proof}
\begin{subequations}
  \begin{align*}
\widetilde{M}\mbox{ locally free}
&\iff X=\bigcup D(f_{i})\mbox{ where }\forall i\ldotp\widetilde{M}|_{D(f_{i})}=\StructureSheaf_{D(f_{i})}^{\oplus{r}}\\
&\iff \forall i\ldotp M_{f_{i}}\iso A^{\oplus r}_{f_{i}}\mbox{ as modules}\\
&\iff M\mbox{ projective of constant rank }r.\qedhere
  \end{align*}
\end{subequations}
\end{proof}

\begin{node}
Recall, if $X$ is complete, then $\Gamma(X,\StructureSheaf_{X})=\kk$
are constant. More general fact: if $\sheaf{F}$ is a coherent sheaf on
a complete variety $X$, then $\dim_{\kk}(\Gamma(X,\sheaf{F}))<\infty$
is finite.

Note: this is not true if $X$ is not complete, e.g., $\Gamma(\AA^{1},\StructureSheaf_{\AA^{1}})=\kk[t]$
is infinite-dimensional.
\end{node}