%%
%% fall-lecture12.tex
%% 
%% Made by Alex Nelson <pqnelson@gmail.com>
%% Login   <alex@lisp>
%% 
%% Started on  2025-10-25T12:19:03-0700
%% Last update 2025-10-25T12:19:03-0700
%% 

\lecture{}

We will reserve $X$ and $Y$ for irreducible varieties in this lecture
(unless we state otherwise).

\begin{node}
Recall, a rational map is an equivalence class of $(U,f\colon U\to Y)$
where $f$ is a morphism and $U\subset X$ is open. We usually denote
such a rational map by $f\colon X\dashrightarrow Y$.
\end{node}

\begin{example}% Gathmann, 9.7
Let $Y=\AA^{1}$. Then $f\colon X\dashrightarrow\AA^{1}$ is just a
rational function.
\end{example}

\begin{example}
We see $f\colon\AA^{1}\setminus0\to\AA^{1}$, $x\mapsto 1/x$ gives a
rational map $\AA^{1}\dashrightarrow\AA^{1}$.
\end{example}

\begin{definition}[Dominant rational map]% Gathmann, 9.5
We say $f\colon X\dashrightarrow Y$ is \define{Dominant} if
$f(X)\supset U$ for some nonempty open $\emptyset\neq U\subset Y$,
--- in particular, this means that its image is dense.
\end{definition}

\begin{definition}% Gathmann, 9.5
We say $f\colon X\dashrightarrow Y$ is \define{Birational} if it is
dominant and there is some dominant $g\colon Y\dashrightarrow X$ such
that $g\circ f\sim\id_{X}$ and $f\circ g\sim\id_{Y}$.
\end{definition}

\begin{definition}% Gathmann, 9.5
We say the varieties $X$ and $Y$ are \define{Birational} if there
exists some birational map $f\colon X\dashrightarrow Y$.
\end{definition}

\begin{definition}[Function field]
We define the \define{Function Field} of $X$ to be the set $\FunField{X}$ of
all rational functions on $X$.

We can see that if $U\subset X$ is a nonempty open affine subset, then $\FunField{X}\iso\FunField{U}$.
We also can see the function field is the localization of the
coordinate ring at the ideal $I(X)=\langle0\rangle$, i.e., $\FunField{X}=\CoordRing{\kk}{X}_{0}$.
\end{definition}

\begin{remark}[Function field really is a field]% Gathmann, 9.6
We see that $\FunField{X}$ is a field: for every
$\varphi_{1}\in\RegularFuns_{X}(U_{1})$ and
$\varphi_{2}\in\RegularFuns_{X}(U_{2})$, we can define
$\varphi_{1}+\varphi_{2}$ and $\varphi_{1}\varphi_{2}$ on $U_{1}\cap U_{2}\neq\emptyset$,
as well as the additive inverse $-\varphi_{1}$ on $U_{1}$, and also
for $\varphi_{1}\neq0$ the multiplicative inverse $\varphi_{1}^{-1}$
on $U_{1}\setminus V(\varphi_{1})$. This gives $\FunField{X}$ the obvious
field structure.
\end{remark}

\begin{proposition}
The set $\{f\colon X\dashrightarrow Y\mbox{ dominant}\}$ is bijective
with the set $\{\mbox{field extensions }\kk[X]\into\kk[Y]\mbox{ preserving the base field }\kk\}$,
sending $\varphi\mapsto\varphi^{*}$.
\end{proposition}

\begin{proof}
Without loss of generality, $X$ and $Y$ are affine irreducible varieties.

\textsc{Claim 1:} This mapping is injective. Suppose we have another
dominant $\psi\colon X\dashrightarrow Y$ with
$\varphi^{*}=\psi^{*}$. Take a principal open $D(h)\subset X$ such
that $\varphi$ and $\psi$ are both defined on $D(h)$. We have
\begin{equation}
\vcenter{\xymatrix{
    K(Y)\ar[r]^-{\varphi^{*}=\psi^{*}} & K(X)\\
    \ar@{^{(}->}[u]^{\subset}\CoordRing{\kk}{Y}\ar@{.>}[r] & \CoordRing{\kk}{X}_{h}\ar@{^{(}->}[u]^{\subset}}}
\end{equation}
Then $\varphi=\psi$ on $D(h)$. This establishes injectiveness.

\textsc{Claim 2:} This mapping is surjective. Let
$\alpha\colon\CoordRing{\kk}{Y}\to\CoordRing{\kk}{X}$ preserve $\kk$. Then $\CoordRing{\kk}{Y}$ generated
by $f_{1}$, \dots, $f_{n}$. We see $\alpha(f_{i})=g_{i}/h_{i}$ where $g_{i},h_{i}\in\CoordRing{\kk}{X}$.
Let
\begin{equation}
h := \prod^{n}_{i=1}h_{i},
\end{equation}
then $\alpha(f_{i})\in\CoordRing{\kk}{X}_{h}$ for all $i$. Then
$\alpha\colon\CoordRing{\kk}{Y}\to\CoordRing{\kk}{X}_{h}$ translates
to $\varphi\colon D(h)\to Y$ with $\varphi^{*}=\alpha$. 
\end{proof}

\begin{theorem}
The following are equivalent:
\begin{enumerate}
\item $X\sim Y$ (i.e., $X$ and $Y$ are birational)
\item $\FunField{X}\iso\FunField{Y}$ as $\kk$-algebras
\item There exists open nonempty subsets $U\subset X$ and $V\subset Y$
  such that $U\iso V$ as varieties.
\end{enumerate}
\end{theorem}

\begin{proof}
$(3)\implies(2)$ is obvious, and $(2)\implies(1)$ follows from the
previous proposition.

$(1)\implies(3)$ Assume $f(,U)\colon X\dashrightarrow Y$ is
birational with $(g,V)\colon Y\dashrightarrow X$ its inverse. Then
$U\cap f^{-1}(V)\iso V\cap g^{-1}(U)$ must be isomorphic.
\end{proof}

\begin{definition}
We call $X$ \define{Rational} if it is birational to some
$\AA^{n}$. (We can parametrize some dense open subset of $X$.)
\end{definition}

\begin{example}
Suppose $\Char(\kk)\neq2$. Then any plane curve $C\subset\PP^{2}$ of
degree 2 is rational.
\end{example}

\begin{example}
The nodal curve $y^{2}-x^{3}-x^{2}$. Let
\begin{equation}
y = tx,
\end{equation}
then we have
\begin{equation}
(tx)^{2}-x^{3}-x^{2}=x^{2}(t^{2}-x-1)
\end{equation}
vanish when
\begin{equation}
(x,y) = (t^{2}-1,t^{3}-t).
\end{equation}
Observe if $t\in\QQ$, then both $x\in\QQ$ and $y\in\QQ$, and these
give us the rational points over $\QQ$. This also gives a birational
map with $\AA^{1}$.
\end{example}

\begin{example}
Let $E=V(y^{2}-x^{3}+x)\subset\AA^{2}$. Then $E$ is not rational. But
$E$ is an example of an elliptic curve.
\end{example}

\begin{node}
One thing we want to do: given a variety $X$, we want to finda
``nicer'' or ``simpler'' $Y$ birational with $X$.
\end{node}

\subsection{Blowing Up}

\begin{node}
Consider the set
\begin{equation}
\operatorname{Bl}_{0}(\AA^{2})=\{\bigl((x,y), (X:Y)\bigr)\in\AA^{2}\times\PP^{1}\mid
xY=yX\}\subset\AA^{2}\times\PP^{1}
\end{equation}
as a closed subset. We can project $\operatorname{Bl}_{0}(\AA^{2})\to\AA^{2}$
and pretend this is a fiber bundle. What are the fibers? Well, over
$(x,y)=(0,0)$, it's an entire copy of $\PP^{1}$. But over other
points of $\AA^{2}$, it's just a single point of $\PP^{1}$.
\end{node}

\begin{example}
Consider the nodal curve $y^{2}-x^{3}-x^{2}$. For $X\neq0$, we set
$t:=Y/X$. From $xt=y$, we find
\begin{equation}
y^{2}-x^{3}-x^{2}=(xt)^{2}-x^{3}-x^{2}=x^{2}(t^{2}-x-1).
\end{equation}
The ``strict transform'' of the nodal curve is precisely the variety
$V(t^{2}-x-1)$, which is just a gentle quadratic. Its complement is
called the ``exceptional set'', and it consists of just the single
line $x=0$. 
\end{example}

\begin{example}
For $V(y^{2}-x^{3})$, we see $(xt)^{2}-x^{3}=x^{2}(t^{2}-x)$. The
``strict transform'' is then $V(t^{2}-x)$ and the ``exception set'' is $x=0$.
\end{example}

\begin{example}
For $V(x^{2}+y^{2}-z^{2})$, we introduce for $Z\neq0$
\begin{equation}
s:=\frac{X}{Z},\quad\mbox{and}\quad t:=\frac{Y}{Z}.
\end{equation}
Then we see $x=sz$ and $y=tz$, so we end up with 
\begin{equation}
x^{2}+y^{2}-z^{2}=z^{2}(s^{2}+t^{2}-1),
\end{equation}
which gives us the ``strict transform'' $V(s^{2}+t^{2}-1)$ (which is a cylinder),
and the ``exception set'' is the $V(z)$ plane.
\end{example}

\begin{example}
Sometimes we need to perform multiple blow-ups. Consider $V(x^{8}-y^{5})$.
Let $y=sx$, then
\begin{equation}
x^{8}-y^{5}=x^{8}-(sx)^{5}=x^{5}(x^{3}-s^{5}).
\end{equation}
Then let $x=st$, so we have
\begin{equation}
x^{3}-s^{5}=(st)^{3}-s^{5}=s^{3}(t^{3}-s^{2})
\end{equation}
and letting $s=ut$,
\begin{equation}
t^{3}-s^{2}=t^{3}-(ut)^{2}=t^{2}(t-u^{2}),
\end{equation}
so we end up with the strict transform $V(t-u^{2})$.
\end{example}

\begin{example}
Sometimes the choice of blow-up produces no change. For example,
$V(xy^{2}-z^{2})$, if we have $y=sx$ and $z=tx$, then
\begin{equation}
xy^{2}-z^{2}=x(sx)^{2}-(tx)^{2}=s^{2}x^{3}-t^{2}x^{2}=x^{2}(s^{2}x-t^{2}),
\end{equation}
and $V(s^{2}x-t^{2})$ is the same variety as we started with.

We can instead consider the blow-up at $y=z=0$ in
$\AA^{3}\times\PP^{1}$, which will consist of the points $(x,y,z)$ and $(Y:Z)$
such that for $Y\neq0$ we have $t=Z/Y$ so $z=ty$, and then
\begin{equation}
xy^{2}-z^{2}=Y^{2}(x-t^{2}).
\end{equation}
Then the strict transform gives us $V(x-t^{2})$, which is simpler.
\end{example}

\begin{construction}
Let $X\subset\AA^{n}$ be an affine vareity. Let $f_{1}$, \dots,
$f_{r}\in\CoordRing{\kk}{X}$. Let $U:=X\setminus V(f_{1},\dots,f_{r})$.
We will look at $f\colon U\to\PP^{r-1}$ defined by sending
\begin{equation}
x\mapsto(f_{1}(x):\dots:f_{r}(x))
\end{equation}
which is a morphism. Looking at its graph
\begin{equation}
\Gamma_{f}=\{(x,f(x))\mid x\in U\}\subset U\times\PP^{r-1}
\end{equation}
which is a closed subset of $X\times\PP^{r-1}$.
The the \define{Blow-Up} of $X$ at $f_{1}$, \dots, $f_{r}$ is defined
to be $\widetilde{X}:=\closure{\Gamma_{f}}\subset X\times\PP^{r-1}\xrightarrow{\pi}X$
where $\pi$ is the restriction of the projection to $U\times\PP^{r-1}$.
\end{construction}

\begin{definition}
The complement of a blow-up $\widetilde{X}\setminus U=\pi^{-1}\bigl(V(f_{1},\dots,f_{r})\bigr)$
is defined to be the \define{Exceptional Set} of the blow-up.
\end{definition}

\begin{definition}
If $Y\subset X$ is a closed subvariety, then we can blow-up $Y$ at
$f_{1}$, \dots, $f_{r}$. This gives us the \define{Strict Transform}
of $Y$, denoted $\widetilde{Y}$, which is a subset of the blow-up $\widetilde{X}$
equal to $\widetilde{Y} = \closure{Y\cap U}$.
\end{definition}

\begin{remark}
If $X$ is not irreducible, then $X=X_{1}\cup\cdots\cup X_{m}$ are the
irreducible components. We can still blow up $\widetilde{X}=\widetilde{X}_{1}\cup\cdots\cup\widetilde{X}_{m}$
by blowing-up each component $\widetilde{X}_{i}$ of $X$.
\end{remark}

\begin{lemma}% Gathmann, 9.14
Let $\widetilde{X}$ be the blow-up of an affine variety $X$ at
$f_{1}$, \dots, $f_{r}$. Then $\widetilde{X}\subset\{(x,y)\in X\times P^{r-1}\mid y_{i}f_{j}(x)=y_{j}f_{i}(x)\;\forall i,j=1,\dots,r\}$.
\end{lemma}