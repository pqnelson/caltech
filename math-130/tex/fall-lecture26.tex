%%
%% fall-lecture26.tex
%% 
%% Made by Alex Nelson <pqnelson@gmail.com>
%% Login   <alex@lisp>
%% 
%% Started on  2025-11-29T10:56:39-0800
%% Last update 2025-11-29T10:56:39-0800
%% 

\lecture{}

\begin{convention}
Unless otherwise stated, in this lecture $X$ is a normal variety.
\end{convention}

\begin{definition}
A Weil divisor $D=\sum n_{i}[Y_{i}]\in\Div(X)$ is called a
\define{Cartier Divisor} if it is locally principal, i.e., there
exists an open cover of $X=\bigcup_{j}U_{j}$ such that
$D|_{U_{j}}=0\in\Cl(U_{j})$ for each $j$.
\end{definition}

\begin{definition}
We also have the \define{Cartier Class} of $X$ defined as
\begin{equation*}
\CaCl(X)=\{\mbox{Cartier divisors on }X\}/\{\mbox{principal divisors }(f)\mid
f\in\UnitGroup{\FunField[\kk]{X}}\}.
\end{equation*}
\end{definition}

\begin{node}
Recall, if $\sheaf{L}$ is an invertible $\StructureSheaf_{X}$-module,
and $s\in\sheaf{L}(U)$ is a nonzero section, then
\begin{equation}
(s)=\sum_{Y\subset X}v_{Y}(s)[Y]\in\Div(X),
\end{equation}
where $v_{Y}(s)=v_{Y}(s/t)$ with $t\in\sheaf{L}(V)$ local in an open
subset $V\subset X$ which intersects $U\cap V\neq\emptyset$ such that $\sheaf{L}_{V}\iso\StructureSheaf_{V}$.

We see that $(s)$ is always a Cartier divisor.

So we have a group morphism
\begin{equation}
\begin{split}
  \Pic(X)&\to\CaCl(X)\subset\Cl(X)\\
  \sheaf{L}&\mapsto(s).
\end{split}
\end{equation}
\end{node}

\begin{node}
We want to construct invertible sheaves from divisors. Let's start
with a Weil divisor $D$ on $X$, where $D=\sum n_{Y}[Y]$.
\end{node}

\begin{definition}
Letting $D=\sum n_{Y}[Y]$ be a Weil divisor on a normal variety $X$.
Following Hartshorne (Ch~II, \S6, pg~144), we may define the
\define{Sheaf Associated with $D$} to be the sheaf $\sheaf{L}(D)$ of
$\StructureSheaf_{X}$-modules by the sections
\begin{equation}
\Gamma(U,\sheaf{L}(D)) = \{f\in\UnitGroup{\FunField[\kk]{X}}\mid
v_{Y}(f)\geq n_{Y}\mbox{ for all prime divisors $Y$ intersecting }U\cap Y\neq\emptyset\}\cup\{0\}.
\end{equation}
\textsc{Caution:} This sheaf is sometimes denoted $\StructureSheaf_{X}(D)$.
\end{definition}

\begin{example}
$\sheaf{L}(0)=\StructureSheaf_{X}$.
\end{example}

\begin{example}
If $X=\PP^{1}$ and $Q=(a:b)\in\PP^{1}$, with $D=n[Q]$. We see
$Q=V_{p}(h)$ where $h=bx_{0}-ax_{1}\in\kk[x_{0},x_{1}]$, then
$\sheaf{L}(nQ)\iso\sheaf{O}(n)$ with the isomorphism given by
$f\mapsto h^{n}f$.
\end{example}

\begin{notice}
If we add some principal divisor given by
$h\in\UnitGroup{\FunField[\kk]{X}}$, then we have an isomorphism
$\sheaf{L}(D+(h))\xrightarrow{\iso}\sheaf{L}(D)$, $f\mapsto hf$.

Consequently, if $D$ is a Cartier divisor, then $\sheaf{L}(D)$ is an
invertible sheaf. (If $D|_{U}=(h)\in\Div(U)$, then $\sheaf{L}(D)|_{U}\iso\StructureSheaf_{U}$.)
This means we now have a mapping
\begin{equation}
\begin{split}
\CaCl(X)&\to\Pic(X)\\
D&\mapsto\sheaf{L}(D),
\end{split}
\end{equation}
which is a group morphism.
\end{notice}

\begin{warning}
If $f\in\Gamma(U,\sheaf{L}(D))$, then $f$ is a rational function, so
$(f)$ means two different things depending on if we view $f$ as a
\underline{section} or as a \underline{rational function}.\footnote{I was not
consulted about this notation, since it was decided about a decade or two
before I was even born\dots but I would like to think I would have
tried to pursuade Algebraic Geometers to choose better notation.}
\end{warning}

\begin{proposition}
We have $\Pic(X)\iso\CaCl(X)$ as Abelian groups.
\end{proposition}

\begin{proof}
We will prove the maps we have constructed are inverses of each other.
Let $D=\sum n_{Y}[Y]$ be a Cartier divisor on $X$. Set
\begin{equation}
V=X\setminus\left(\bigcup_{n_{Y}<0}Y\right).
\end{equation}
Then $1\in\UnitGroup{\FunField[\kk]{X}}$ is a section of
$\sheaf{L}(D)$ over $V$.

We claim $(1)=D$ [where $1$ is viewed as a section] in $\Div(X)$. If
\begin{equation}
D|_{U} = (\underbrace{h}_{\text{in }\UnitGroup{\FunField[\kk]{X}}})
\end{equation}
then $\sheaf{L}(D)|_{U}\iso\StructureSheaf_{U}$ generated by $h^{-1}\in\Gamma(U,\sheaf{L}(D))$.

If $Y\cap U\neq\emptyset$, then $v_{Y}(1)$ [with $1$ viewed as a section]
equals
\begin{equation}
v_{Y}(1) = v_{Y}(1/h^{-1}) = v_{Y}(h),
\end{equation}
where $v_{Y}(1/h^{-1})$ has $1/h^{-1}$ converted to a rational
function. This then implies
\begin{equation}
(1)|_{U}=D|_{U}.
\end{equation}
Then $\CaCl(X)\to\Pic(X)\to\CaCl(X)$ composes to give us the identity
morphism on $\CaCl(X)$.

If we start with an invertible sheaf on $X$, $t\in\sheaf{L}(U)$ is a
nonzero secction. Note: if $0\neq s\in\sheaf{L}(V)$ (where $V$ is some
other open subset) and $Y\cap V\neq\emptyset$, then
\begin{equation}
v_{Y}(s/t)=v_{Y}(s)-v_{Y}(t)\geq-v_{Y}(t).
\end{equation}
We claim
\begin{equation}
\begin{split}
\sheaf{L}&\to\sheaf{L}((t))\\
s&\mapsto s/t
\end{split}
\end{equation}
is an isomorphism. If $\sheaf{L}|_{V}\iso\StructureSheaf_{V}$ is
generated by $u\in\sheaf{L}(V)$, then the divisor
\begin{equation}
(t)|_{V}=(t/u)|_{V},
\end{equation}
which implies $\sheaf{L}((t))|_{V}$ is generated by $u/t$. Since
$u\mapsto u/t$, we get a local isomorphism
\begin{equation}
\sheaf{L}|_{V}\iso\sheaf{L}((t))|_{V}.
\end{equation}
This gives us the composition $\Pic(X)\to\CaCl(X)\to\Pic(X)$ is the
identity morphism.
\end{proof}

\begin{example}
$\Pic(\AA^{n})=\CaCl(\AA^{n})=0$ since $\CaCl(\AA^{n})\subset\Cl(\AA^{n})=0$,
so all invertible sheaves on $\AA^{n}$ are isomorphic to the
structure sheaf. (More is true: all locally free
$\StructureSheaf_{\AA^{n}}$-modules of finite rank are trivial, i.e.,
they are direct sums of copies of the structure sheaf.)
\end{example}

\begin{example}
We can write the projective space as the union of
$\PP^{n}=\bigcup^{n}_{i=0}D_{p}(x_{i})$ and then we see $\Cl(D_{p}(x_{i}))=0$ implies
all Weil divisors are locally trivial. Then
\begin{equation}
\Pic(\PP^{n})=\CaCl(\PP^{n})=\Cl(\PP^{n})\iso\ZZ.
\end{equation}
Any invertible sheaf is isomorphic to
$\StructureSheaf_{\PP^{n}}(m[H])$ where $H\subset\PP^{n}$ is a
hyperplane. So $H=V_{p}(h)$ for some $h\in\kk[x_{0},\dots,x_{n}]$
linear, then we have
$\StructureSheaf_{\PP^{n}}(m[H])\to\StructureSheaf_{\PP^{n}}(m)$ given
by sending $f\mapsto h^{m}f$. This proves $\Pic(\PP^{n})=\{\mathcal{O}(m)\}$.
\end{example}

\begin{example}
Let $X=V(xy-z^{2})\subset\AA^{3}$ describe the cone. We see
\begin{equation}
\Cl(X)\iso\ZZ/2\ZZ
\end{equation}
is generated by $[L]$ where $L=V(y)\cap X$. We also find the ideal
corresponding to $L$ is
\begin{equation}
I(L)=\langle y,z\rangle\subset\kk[x,y,z].
\end{equation}
We claim $[L]$ is not a Cartier divisor. Otherwise there exists an
open affine $U\subset X$ such that $P=(0,0,0)\in U$ and
$f\in\UnitGroup{\FunField[\kk]{X}}$ such that
\begin{equation}
[L\cap U]=(f|_{U})\in\Div(U).
\end{equation}
But this implies $f\in\CoordRing{\kk}{X}$ and $I(L\cap U)=\langle f\rangle\subset\CoordRing{\kk}{U}$.
Then $I(L)\StructureSheaf_{X,P}=\langle y,z\rangle\subset\StructureSheaf_{X,P}$
principal. But $P$ is singular. So $\dim_{\kk}(\mathfrak{m}_{P}/\mathfrak{m}_{P}^{2})=3$
has basis $\{x,y,z\}$.
\end{example}

\begin{definition}
Any irreducible variety $X$ is called \define{Locally Factorial} if
$\StructureSheaf_{X,P}$ is a UFD for every $P\in X$. (Slogan: ``All
local rings are UFD''.)
\end{definition}

\begin{remark}
We have a hierarchy: nonsingular varieties are locally factorial. And
locally factorial varieties are normal.
\end{remark}

\begin{proposition}
If $X$ is locally factorial, then $\Pic(X)=\Cl(X)$.
\end{proposition}
