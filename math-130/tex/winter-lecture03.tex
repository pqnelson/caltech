%%
%% winter-lecture03.tex
%% 
%% Made by Alex Nelson <pqnelson@gmail.com>
%% Login   <alex@lisp>
%% 
%% Started on  2026-01-15T15:46:20-0800
%% Last update 2026-01-15T15:46:20-0800
%% 

\lecture[Definition of Schemes]{}

\subsection{Sheaf on a Base}

\begin{node}[Presheaf defined on a base]
Let $X$ be a topological space, let $\mathcal{B}$ be a basis for the
topology of $X$. Suppose we have a functor
$\sheaf{F}\colon\mathcal{B}\to\cat{C}$ which is presheaf-like. How can
we extend it to a presheaf on $X$? Well, what do we mean that
$\sheaf{F}$ is ``presheaf-like''? Let us first make that notion
explicitly clear. 

If $B_{1}\subset B_{2}$ are elements of $\mathcal{B}$, then there is a
restriction morphism
\begin{equation}
\rho_{B_{2},B_{1}}\colon\sheaf{F}(B_{2})\to\sheaf{F}(B_{1})
\end{equation}
such that $\rho_{B_{i},B_{i}}=\id_{B_{i}}$ for all
$B_{i}\in\mathcal{B}$, and also for any elements $B_{i}\in\mathcal{B}$
such that $B_{3}\subset B_{2}\subset B_{1}$ we have 
$\rho_{B_{3},B_{1}}=\rho_{B_{2},B_{1}}\circ\rho_{B_{3},B_{2}}$. This
is what we mean by a ``presheaf-like functor defined on the base $\mathcal{B}$''.
The common terminology in the literature seems to be: $\sheaf{F}$ is a
\define{Presheaf on a Base} $\mathcal{B}$.
\end{node}

\begin{node}[Presheaf defined on a base induces a presheaf on the space]
Can we ``promote'' this into a presheaf on $X$?
One [cryptically] terse way to do this is by introducing the presheaf
$\sheaf{F}'$ on $X$ defined by, for any open subset $V\subset X$, we
have
\begin{equation}
\sheaf{F}'(V)=\varprojlim_{V\supset B}\sheaf{F}(B),
\end{equation}
which\dots does not seem clear to me.

(See Vakil~\cite[\S2.5]{vakil2025rising}, G\"{o}rtz and Wedhorn~\cite[\S2.5]{gortz2020algebraic}.)
\end{node}

\begin{definition}
Let $X$ be a topological space with base $\mathcal{B}$ for its topology.
Let $\sheaf{F}$ be a presheaf for its base $\mathcal{B}$. We say
$\sheaf{F}$ is a \define{Sheaf on a Base} $\mathcal{B}$ if it also
satisfies:
\begin{enumerate}
\item\textsc{Base identity:} Let $B=\bigcup_{i}B_{i}$ be a union of
  elements in the base $B_{i}\in\mathcal{B}$. If $f,g\in\sheaf{F}(B)$
  are such that $\rho_{B,B_{i}}f=\rho_{B,B_{i}}g$ for all $i$, then $f=g$.
\item\textsc{Base gluability:} Let $B=\bigcup_{i}B_{i}$ be a union of
  elements in the base $B_{i}\in\mathcal{B}$. Suppose we have
  $f_{i}\in\sheaf{F}(B_{i})$ for all $i$ such that $f_{i}$ agrees with
  $f_{j}$ whenever there is a basic open set contained in $B_{i}\cap B_{j}$
  (i.e., $\rho_{B_{i},B_{k}}f_{i}=\rho_{B_{j},B_{k}}f_{j}$ for any
  $B_{k}\in\mathcal{B}$ such that $B_{k}\subset B_{i}\cap B_{j}$),
  then there exists an $f\in\sheaf{F}(B)$ such that $\rho_{B,B_{i}}f=f_{i}$
  for all $i$.
\end{enumerate}
\end{definition}

\begin{theorem}[{Vakil~\cite[Th2.5.1]{vakil2025rising}}]
Let $X$ be a topological space. Let $\mathcal{B}$ be a basis for the
topology of $X$. Let $\sheaf{F}$ be a sheaf on $\mathcal{B}$. Then
there is a sheaf $\sheaf{G}$ extending $\sheaf{F}$ (with isomorphisms
$\sheaf{F}(B_{i})\iso\sheaf{G}(B_{i})$ agreeing with the restriction maps).
This sheaf $\sheaf{G}$ is unique up to unique isomorphism.
\end{theorem}

\begin{theorem}[{G{\"o}rtz and Wedhorn~\cite[Prop.2.20]{gortz2020algebraic}}]
Let $X$ be a topological space. Let $\mathcal{B}$ be a basis for the
topology of $X$. Let $\sheaf{F}$ be a sheaf on $\mathcal{B}$. Let
$\sheaf{F}'$ be the associated presheaf on $X$. Then $\sheaf{F}'$ is a
sheaf if and only if for every $U\in\mathcal{B}$ and for every open
covering $\{B_{i}\in\mathcal{B}\}_{i\in I}$ of $U$, and for every open
covering $\{B_{ijk}\in\mathcal{B}\}_{k\in K}$ of $U_{i}\cap U_{j}$,
the diagram
\begin{equation}
  \vcenter{\xymatrix{\sheaf{F}(U)\ar[r]^-{\rho}
& \displaystyle{\prod_{i\in I}\sheaf{F}(B_{i})} \ar@<0.6666666666666ex>[r]^-{\sigma}\ar@<-0.6666666666666ex>[r]_-{\sigma'}&\displaystyle{\prod_{i,j,k}\sheaf{F}(B_{ijk})}}}
\end{equation}
is exact, where $\sigma((s_{i})_{i\in I})=(s_{i}|_{B_{i}\cap B_{j}})_{i\in I}$
\end{theorem}

\begin{xca}
Let $X$ be a topological space, let $U\subset X$ be an open subset.
Let $\sheaf{F}$ be a presheaf on $X$.
\begin{enumerate}
\item Prove or find a counter-example: $\sheaf{F}|_{U}$ is a presheaf on $U$.
\item Prove or find a counter-example: if $\sheaf{F}$ is a sheaf on
  $X$, then $\sheaf{F}|_{U}$ is a sheaf on $U$.
\end{enumerate}
\end{xca}

\subsection{Structure Sheaf}

\begin{definition}
Let $R$ be a ring. We define the \define{Structure Sheaf} of
$\Spec(R)$ to be the presheaf $\sheaf{O}_{\Spec(R)}$ such that for
each $f\in R$ we have
\begin{equation}
\sheaf{O}_{\Spec(R)}(D(f)):=R_{f}.
\end{equation}
(We will prove this is a sheaf soon.)

\begin{proof}[Proof ($\sheaf{O}_{\Spec(R)}$ is a presheaf)]
We should prove this is a presheaf. Since (\S\ref{d-f-forms-basis-for-zariski-topology}) the distinguished opens
$D(f)$ forms a basis $\mathcal{B}$ on the Zariski topology, it suffices to prove
this is a presheaf on $\mathcal{B}$. We just need to check that
$R_{f}=R_{g}$ when $D(f)=D(g)$.

For any $f,g\in R$, we have $D(f)\subset D(g)$ if and only if there
exists an $n\geq1$ such that $f^{n}\in Rg$ (or equivalently $g/1\in\MultGroup{(R_{f})}$).
In this case, we obtain a unique ring morphism
\begin{equation}
\rho_{f,g}\colon R_{g}\to R_{f}
\end{equation}
such that when composed with the canonical ring morphism $i_{f}\colon R\to R_{f}$
sending $x\mapsto x/1$, $\rho_{g,f}\circ i_{g}=i_{f}$.

Furthermore, if $D(f)\subset D(g)\subset D(h)$, we have $\rho_{f,g}\circ\rho_{g,h}=\rho_{f,h}$.

In particular, when $D(f)=D(g)$, the ring morphism $\rho_{r,g}$ is the
identity morphism.
\end{proof}
\end{definition}

\begin{proposition}
Let $R$ be a commutative ring. The stalks of $\sheaf{O}_{\Spec(R)}$
are local rings.
\end{proposition}

\begin{proposition}
Let $R$ be a commutative ring. The presheaf $\sheaf{O}_{\Spec(R)}$ is
a sheaf on the topological space $\Spec(R)$.
\end{proposition} 

\begin{proof}
Let $\mathcal{B}=\{D(f)\mid f\in R\}$ be the basis of the topology
using principal opens. For any $U\in\mathcal{B}$, for every open
covering $\{U_{i}\in\mathcal{B}\}_{i\in I}$ of $U$, we need to prove
\begin{enumerate}
\item For any $s,s'\in\sheaf{O}_{\Spec(R)}(U)$ with
  $s|_{U_{i}}=s'|_{U_{i}}$ for all $i$, we have $s=s'$; and
\item For each $i\in I$ and any $s_{i}\in\sheaf{O}_{\Spec(R)}(U_{i})$ such that
  for all $i,j\in I$ we have $s_{i}|_{U_{i}\cap U_{j}}=s_{j}|_{U_{i}\cap U_{j}}$,
  then there exists an
  $s\in\sheaf{O}_{\Spec(R)}(U)$ such that $s|_{U_{i}}=s_{i}$ for all
  $i\in I$.
\end{enumerate}
Since $D(f)$ is quasi-compact, we can assume that $I$ is finite
\end{proof}

\subsection{Schemes}

\begin{definition}
A \define{Affine Scheme} is a ringed space $(X,\sheaf{O}_{X})$ consisting of a
topological space $X$ and a structure sheaf $\sheaf{O}_{X}$ of rings
such that $(X,\sheaf{O}_{X})\iso(\Spec(R),\sheaf{O}_{\Spec(R)})$. 
\end{definition}

\begin{remark}
We call these ``affine'' in analogy to ``affine varieties'', not
because there's an obvious connection to the affine space (which was
the case for affine varieties).
\end{remark}

\begin{remark}
The nLab defines a notion of a ``spectral topological space'' (or
``coherent space'') as a topological space $X$ such that it is
homeomorphic to $\Spec(R)$ for some [commutative] ring $R$. That is to
say, it is homeomorphic to the topological space of an affine scheme.
``Spectral spaces'' are thus named because they are the spectra of rings.
\end{remark}

\begin{terminology}
Really, since an affine scheme $(X,\sheaf{O}_{X})$ is isomorphic to some $(\Spec(R),\sheaf{O}_{\Spec(R)})$, it
is common to abuse language and just refer to
$(\Spec(R),\sheaf{O}_{\Spec(R)})$ as the affine schemes.
\end{terminology}

\begin{definition}
A \define{Scheme} is a ringed space $(X,\sheaf{O}_{X})$ such that for
each point $x\in X$ there exists an open neighborhood $x\in U\subset X$
such that $(U,\sheaf{O}_{X}|_{U})$ is an affine scheme.
\end{definition}

\begin{remark}
I think that we could introduce charts $(U,\varphi)$ such that
$U\subset X$ is an open subset and
$\varphi\colon U\to\Spec(R)$ is a homeomorphism
(and then define the structure sheaf on $U$
accordingly as isomorphic to $\sheaf{O}_{\Spec(R)}$).
Then an atlas of such charts---possibly modulo some compatibility
conditions---gives us a scheme structure on $X$.
\end{remark}

\begin{example}[Affine space]
Let $\FF$ be an algebraically closed field. We can define the affine
space $\AA^{n}_{\FF}$ as the algebraic variety of points
$(a_{1},\dots,a_{n})$ with coordinates in $\FF$. Its coordinate ring
is the polynomial ring $R=\FF[x_{1},\dots,x_{n}]$. This should be
familiar from last quarter.

We can now define the corresponding scheme $X=\Spec(R)$ with the
Zariski topology. The closed points are the maximal ideals $(x_{1}-a_{1},\dots,x_{n}-a_{n})$.
But there are also non-closed points for each prime ideal of $R$.

The open subsets of the scheme $X$ are given by the complements of
hypersurfaces $D(f)=X\setminus V(f)$ where $f\in R$ is an irreducible polynomial.
The structure sheaf on these principal open subsets are
\begin{equation}
\sheaf{O}_{X}(D(f))=R[f^{-1}]=\{\frac{r}{f^{m}}\mid m\in\NN_{0},r\in R\}.
\end{equation}
The structure sheaf may be constructed from this data alone.
\end{example}

\begin{xca}
Prove or find a counter-example: in the previous example (affine space
as a scheme), the scheme is an affine scheme. [Hint: obvious.]
\end{xca}

\begin{puzzle}
When we view $\Spec(R)$ as the scheme for an affine space
$\AA^{n}_{\FF}$, is the automorphism group $\Aut(\Spec(R))$
(consisting of scheme automorphisms) ``the same as'' [isomorphic to?]
the automorphism group $\Aut(\AA^{n}_{\FF})$ of affine transformations?
\end{puzzle}

\begin{definition}
Let $(X,\sheaf{O}_{X})$ be a scheme. Let $U\subset X$ be an open subset.
An \define{Open Subscheme} of $(X,\sheaf{O}_{X})$ is a scheme
$(U,\sheaf{O}_{X}|_{U})$.
\end{definition}

\begin{remark}
There is a notion of a ``closed subscheme'', but it is far more
complicated. We will revisit it much later.
\end{remark}

\begin{definition}
Let $(X,\sheaf{O}_{X})$ and $(Y,\sheaf{O}_{Y})$ be schemes.
An \define{Isomorphism of Schemes} (or ``Scheme Isomorphism'')
consists of
\begin{enumerate}
\item a homeomorphism $f\colon X\to Y$ of the underlying topological
  spaces, and
\item a isomorphism of sheaves $f^{\sharp}\colon\sheaf{O}_{Y}\to f_{*}\sheaf{O}_{X}$
\end{enumerate}
\end{definition}

\begin{definition}
Some \define{Gluing Data for Schemes} consists of
\begin{enumerate}
\item A family of schemes $\{U_{i}\}_{i\in I}$ where $I$ is a
  (possibly infinite) indexing set,
\item A family of open subsets $U_{ij}\subset U_{i}$ for all $i,j\in I$
\item A family of isomorphisms $\varphi_{ji}\colon U_{i}\to U_{j}$
\end{enumerate}
such that
\begin{enumerate}
\item for each $i\in I$ we have $U_{ii}=U_{i}$, and
\item \textsc{Cocycle condition:} for all $i,j,k\in I$, we have $\varphi_{kj}\circ\varphi_{ji}=\varphi_{ki}$.
\end{enumerate}
Observe that $\varphi_{ii}=\id_{U_{i}}$ and so the cocycle condition
$\varphi_{ij}\circ\varphi_{ji}=\id_{U_{i}}$ implies
$\varphi_{ji}=\varphi_{ij}^{-1}$ for all $i,j\in I$. Furthermore
$\varphi_{ji}|_{U_{ij}\cap U_{ik}}\colon U_{ij}\cap U_{ik}\to U_{ji}\cap U_{jk}$
is an isomorphism.
\end{definition}

\begin{construction}
Let $((U_{i})_{i\in I},(U_{ij})_{i,j\in I},(\varphi_{ji})_{i,j\in I})$
be gluing data for schemes. There exists a scheme $X$ and morphisms
$\psi_{i}\colon U_{i}\to X$ such that
\begin{enumerate}
\item For all $i\in I$, the morphism $\psi_{i}$ is an isomorphism of
  $U_{i}$ with an open subset of $X$;
\item $\psi_{j}\circ\varphi_{ji}=\psi_{i}$ for all $i,j\in I$;
\item $X=\bigcup_{i\in I}\psi_{i}(U_{i})$
\item
  $\psi_{i}(U_{i})\cap\psi_{j}(U_{j})=\psi_{i}(U_{ij})=\psi_{j}(U_{ji})$
  for all $i,j\in I$.
\end{enumerate}
Furthermore, $X$ and the morphisms $\psi_{i}$ are unique up to unique isomorphism.
\end{construction}

\begin{proof}
\textsc{Step 1:} Construct $X$.
To construct $X$, we construct an equivalence relation $\sim$ on the
disjoint union $\coprod_{i\in I}U_{i}$ as follows: for any
$x_{i}\in U_{i}$ and $x_{j}\in U_{j}$ we have $x_{i}\sim x_{j}$
if and only if $x_{i}\in U_{ij}$ and $x_{j}\in U_{ji}$ and $\varphi_{j}(x_{i})=x_{j}$.
The cocycle condition ensures this is an equivalence relation. Then we
form
\begin{equation}
X = \left(\coprod_{i\in I}U_{i}\right)/\sim.
\end{equation}
The maps $\psi_{i}\colon U_{i}\into X$ are the natural maps. We equip
$X$ with the quotient topology (which is the finest topology such that
the $\psi_{i}$ are continuous). Thus we have constructed the
topological space for the scheme we seek.

\textsc{Step 2:} construct the structure sheaf for $X$. It suffices to
define it on those open subsets $U\subset X$ which are contained in
one of the $\psi_{i}(U_{i})$ for some $i\in I$. Specifically we set $\sheaf{O}_{X}(U)=\sheaf{O}_{U_{i}}(\psi_{i}^{-1}(U))$.

If $U\subset\psi_{i}(U_{i})\cap\psi_{j}(U_{j})$, we identify the rings
$\sheaf{O}_{U_{i}}(\psi^{-1}_{i}(U))$ and $\sheaf{O}_{U_{j}}(\psi^{-1}_{j}(U))$
via $\varphi_{ji}$. This gives us the restriction maps for the sheaf.

\textsc{Step 3:} Check that $\sheaf{O}_{X}$ is a ring sheaf and
$(X,\sheaf{O}_{X})$ is a locally ringed space.

\textsc{Step 4:} Check the $\psi_{i}$ are morphisms of locally ringed
spaces. This will imply $X=\bigcup_{i\in I}\psi_{i}(U_{i})$.
\end{proof}