%%
%% fall-lecture02.tex
%% 
%% Made by Alex Nelson <pqnelson@gmail.com>
%% Login   <alex@lisp>
%% 
%% Started on  2025-10-02T13:55:44-0700
%% Last update 2025-10-02T13:55:44-0700
%% 

\lecture{}

\begin{node}[Recap]
We had an interplay between subsets of $\AA^{n}$ and ideals of $\kk[x_{1},\dots,x_{n}]$
from (1) the ideal of subsets $X\mapsto I(X)$, and (2) the zero set of
an ideal $V(I)\mapsfrom I$.
Furthermore, we introduced
\begin{enumerate}
\item Weak Nullstellensatz: if $\kk=\AlgebraicClosure\kk$ is
  algebraically closed, then $I\bigl(V(I)\bigr)=\Radical{I}$
\item String Nullstellensatz: if $\kk=\AlgebraicClosure\kk$ is
  algebraically closed, then all maximal ideals of $\kk[x_{1},\dots,x_{n}]$
  are of the form $(x_{1}-a_{1},\dots,x_{n}-a_{n})$; this means there
  is a bijection between points in $\AA^{n}$ and maximal ideals in the
  polynomial ring.
\end{enumerate}
\end{node}

\begin{proposition}
The Weak Nullstellensatz in $\AA^{n+1}$ implies the Strong
Nullstellensatz in $\AA^{n}$.
\end{proposition}

\begin{proof}[Proof sketch]
Let $I\ideal\kk[x_{1},\dots,x_{n}]$.

\textsc{Claim 1:} $\Radical{I}\subset I\bigl(V(I)\bigr)$.

\textsc{Claim 2:} $\Radical{I}\supset I\bigl(V(I)\bigr)$.
We will prove $\forall f,f\in\Radical{I}\implies f\in I\bigl(V(I)\bigr)$.
Let $f$ be such that $f\in\Radical{I}$. We know
\begin{equation}
I\bigl(V(I)\bigr)=\langle f_{1},\dots,f_{m}\rangle
\end{equation}
so $f$ vanishes if $f_{1}$, \dots, $f_{m}$ vanish.

Then Rbinovitch's trick: $f_{1}$, \dots, $f_{m}$, $1-x_{0}f$ have no
common zeroes in $\AA^{n+1}$.

Then using the Weak Nullstellensatz, they generrate the unit ideal. So
\begin{equation}\label{eq:math130a:fall-lecture02:weak-implies-strong:eq2}
1 = g_{0}\cdot(1-x_{0}f) + g_{1}\cdot f_{1} + \cdots+g_{m}\cdot f_{m}
\end{equation}
for some $g_{i}\in\kk[x_{0},\dots,x_{n}]$. Let
\begin{equation}
x_{0} = \frac{1}{f},\quad\mbox{so}\quad 1-x_{0}f=0.
\end{equation}
This is completely formal. Then we see this makes the first term on
the right-hand side of Equation~\eqref{eq:math130a:fall-lecture02:weak-implies-strong:eq2}
vanish, giving us:
\begin{equation}
1 = g_{1}\cdot f_{1} + \cdots+g_{m}\cdot f_{m}
\end{equation}
with $g_{1}$, \dots, $g_{m}\in\kk[\frac{1}{f},x_{1},\dots,x_{n}]$.

We ``clear the denominators'', so
\begin{equation}
f^{N} = h_{1}f_{1} + \cdots + h_{m}f_{m}
\end{equation}
with $N\in\NN$, $f^{N}\in I$, and $h_{i}\in\kk[x_{1},\dots,x_{n}]$ for
each $i=1,\dots,n$. This means $f\in\Radical{I}$.
\end{proof}

% 1.12
\begin{lemma}
Let $X_{1}$, $X_{2}$ be affine varieties in $\AA^{n}$.
\begin{enumerate}
\item $I(X_{1}\cup X_{2})=I(X_{1})\cap I(X_{2})$, and
\item $I(X_{1}\cap X_{2})=\Radical{I(X_{1})+I(X_{2})}$.
\end{enumerate}
\end{lemma}

\begin{proof}
We will prove the second claim. We see
\begin{subequations}
\begin{align}
I(X_{1}\cap X_{2}) &= I\left(V\bigl(I(X_{1})\bigr)\cap V\bigl(I(X_{2})\bigr)\right)\quad\mbox{since $X_{1}$, $X_{2}$ are closed}\\
&= I\left(V\bigl(I(X_{1})+I(X_{2})\bigr)\right)\\
&= \Radical{I(X_{1})+I(X_{2})}
\end{align}
\end{subequations}
by the Strong Nullstellensatz.
\end{proof}

\begin{remark}
We see there are bijections:
\begin{equation*}
\mbox{Affine Varieties}\longleftrightarrow\mbox{Radical ideals in }\kk[x_{1},\dots,x_{n}].
\end{equation*}
\begin{equation*}
\mbox{Affine Schemes}\longleftrightarrow\mbox{All ideals in }\kk[x_{1},\dots,x_{n}].
\end{equation*}
Don't worry too much about that second bijection, it's just a
``roadmap'' to where we're going.
\end{remark}

\begin{example} % 1.13
Consider $I(X_{1})=\langle y-x^{2}\rangle$ and $I(X_{2})=\langle y\rangle$
where $X_{1},X_{2}\subset\AA^{2}$. We see
\begin{equation}
I(X_{1}\cap X_{2})=\Radical{I(X_{1})+I(X_{2})}=\langle x,y\rangle.
\end{equation}
The ``Scheme-theoretic intersection'' is
\begin{equation}
I(X_{1})+I(X_{2})=\langle y,x^{2}\rangle,
\end{equation}
and that ``$x^{2}$'' component (instead of $x$) encodes $X_{2}$ is
tangent to $X_{1}$.

We should observe that $I(X_{1})+I(X_{2})$ is not a radical ideal in
$\CC[x,y]$. 
\end{example}

\begin{remark} % 1.14
Polynomials in $\kk[x_{1},\dots,x_{n}]$ can be viewed as functions on
$\AA^{n}$. They are polynomial functions. If $f,g\in\kk[x_{1},\dots,x_{n}]$
define the same functions on $\AA^{n}$, then their difference
\begin{equation}
f-g\in I(\AA^{n})=I(V(0))\iso\Radical{0}=0.
\end{equation}
Then $f=g$. So if they define the same function, then they must be the
same formal polynomial.

We can generalize this to any affine variety
$X\subset\AA^{n}$. Specifically, the claim is
\begin{equation}
\biggl(f,g\in\kk[x_{1},\dots,x_{n}]\implies\forall x\in X,f(x)=g(x)\biggr)\iff f-g\in I(X)
\end{equation}
Then the quotient set $\kk[x_{1},\dots,x_{n}]/I(X)$ is the ring of
polynomial functions on $X$. It is called the \define{Coordinate Ring}
of $X$ and various denoted either $\kk[X]$ or $A(X)$.
\end{remark}

\begin{definition}
Let $Y$ be an affine variety, consider zero loci of ideals in $\AA(Y)$.
\begin{enumerate}
\item Let $S\subset A(Y)$. We define an \define{Affine Subvariety} of
  $Y$ to be $V(S):=V_{Y}(S)=\{x\in Y\mid f(y)=0\forall f\in S\}$
\item Let $X\subset Y$ be a subset. We define the \define{Ideal of $X$
  in $Y$} $I(X):=I_{Y}(X)=\{f\in A(Y)\mid f(x)=0\forall x\in X\}$,
which is an ideal in $A(Y)$.
\end{enumerate}
\end{definition}

\begin{remark} % 1.18
\begin{enumerate}
\item $A(X)\iso A(Y)/I(X)$
\item \textsc{Relative Nullstellensatz:} $I_{Y}(V_{Y}(J))=\Radical{J}$
  for any $J\ideal A(Y)$, which gives us a bijection
\begin{equation*}
\{\mbox{Affine Subvarieties of }Y\}\longleftrightarrow\{\mbox{Radical Ideals in }A(Y)\}
\end{equation*}
\end{enumerate}
\end{remark}

\subsection{Zariski Topology}

\begin{definition}
Let $X$ be an affine variety.
The \define{Zariski Topology} on $X$ is the topology whose closed sets
are the affine subvarieties of $X$.
\end{definition}

\begin{definition}
Let $X$ be a topological space. We say $X$ is \define{Reducible} if
$X=X_{1}\cup X_{2}$ wherre $X_{1}$ and $X_{2}$ are proper closed
subsets of $X$. Otherwise, $X$ is called \define{Irreducible}.

(Almost nothing is irreducible in classical topology!)

Every disconnected topological space is reducible.
\end{definition}

\begin{proposition} % 2.7
Let $X$ be a disconnected Affine variety, $X=X_{1}\cup X_{2}$ where
$X_{1}\cap X_{2}=\emptyset$ are closed subsets of $X$.
Then $A(X)\iso A(X_{1})\times A(X_{2})$.
\end{proposition}

\begin{proof}
We know
\begin{equation}
I(X_{1})\cap I(X_{2})=I(X_{1}\cup X_{2})=I(X)=0\ideal A(X).
\end{equation}
On the other hand,
\begin{equation}
\begin{split}
I(X_{1}\cap X_{2}) &= I(\emptyset) = \Radical{I(X_{1})+I(X_{2})}\\
&= \langle1\rangle
\end{split}
\end{equation}
where $\langle1\rangle$ is the unit ideal. Then
\begin{equation}
I(X_{1})+I(X_{2})=\langle1\rangle
\end{equation}
where $I(X_{i})\ideal A(X)$ are ideals. Then by the Chinese Remainder
Theorem, $A(X)\iso(A(X)/I(X_{1}))\times(A(X)/I(X_{2}))$.
\end{proof}

\begin{proposition} % 2.8
A non-empty affine variety $X$ which is irreducible if and only if
$A(X)$ is an integral domain.
\end{proposition}

\begin{proof}
$(\Longrightarrow)$ We will prove the forward direction by contrapositive.
Assume $A(X)$ is not an integral domain. Then there are zero divisors
$f_{1},f_{2}\in A(X)$ which are nonzero such that $f_{1}f_{2}=0$.
Then $X_{1}=V(f_{1})$ and $X_{2}=V(f_{2})$. Then
\begin{subequations}
  \begin{align}
X_{1}\cup X_{2} &= V(f_{1})\cup V(f_{2})\\
&= V(f_{1}f_{2})\\
&= V(0) = X.
  \end{align}
\end{subequations}
Hence $X$ is reducible.

$(\Longleftarrow)$ Again, we will prove the backwards claim by contrapositive.
Assume $X$ is reducible. Then consider $X_{1}$, $X_{2}$ being closed
subsets of $X$ such that $X=X_{1}\cup X_{2}$. Then $I(X_{i})\neq0$.
We can choose $f_{i}\in I(X_{i})$. We see $f_{1}f_{2}$ vanishes on
$X_{1}\cup X_{2}=X$ which implies $f_{1}f_{2}=0$, which then implies
$A(X)$ has a nonzero zero-divisor. Hence $A(X)$ is not an integral domain.
\end{proof}

\begin{remark} % 2.9
We see there is a bijection
\begin{equation*}
\{\mbox{non-empty irreducible affine subvarieties of }Y\}\longleftrightarrow%
\{\mbox{prime ideals in }A(Y)\},
\end{equation*}
where $Y$ is irreducible. This is from restricting the bijection from
the Relative Nullstellensatz.
\end{remark}

\begin{definition} % 2.11
A topological space $X$ is \define{Noetherian} if every chain
$X_{0}\propersupset X_{1}\propersupset X_{2}\propersupset \dots$
of closed subsets stabilizes.
\end{definition}

\begin{lemma} % 2.12
Any affine variety $X$ is Noetherian.
\end{lemma}

\begin{proof}[Proof sketch]
The trick is to see $A(X)\iso\kk[x_{1},\dots,x_{n}]/I$ which is Noetherian.
The relative Nullstellensatz translates the chain of closed subsets
[subvarieties of $X$] into a chain of ideals.
\end{proof}

\begin{proposition}[Irreducible decomposition of Noetherian spaces] %2.14
Every Noetherian topological space can be written as a finite union of
non-empty irreducible closed subsets
\begin{equation}
X = X_{1}\cup\cdots\cup X_{r}.
\end{equation}
If further the $X_{i}$ are maximal (so $X_{i}\nsubset X_{j}$ for
$i\neq j$), then the $X_{1}$, \dots, $X_{r}$ are unique up to
permutation. We call the $X_{i}$ \define{Irreducible Components} of $X$.
\end{proposition}