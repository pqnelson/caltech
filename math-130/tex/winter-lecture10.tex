%%
%% winter-lecture10.tex
%% 
%% Made by Alex Nelson <pqnelson@gmail.com>
%% Login   <alex@lisp>
%% 
%% Started on  2026-01-27T08:55:20-0800
%% Last update 2026-01-27T08:55:20-0800
%% 

\lecture[Fiber Products]{}

% Gues Lecturer: Dr Mantovan

\begin{node}
If $X$ is a scheme and $U\subset X$ is an open subset, then
$(U,\sheaf{O}_{X}|_{U})$ has a natural scheme structure. But for
closed subsets $F\subset X$, we can ask if it has a natural scheme
structure. The problem is that there are \emph{many distinct} natural
scheme structures we could give it.

We have seen one natural scheme structure, the reduced scheme structure.
\end{node}

\begin{definition}[Fiber product of sets]
For sets $X$, $Y$, $S$ and maps $f_{X}\colon X\to S$ and $f_{Y}\colon Y\to S$
we define the \define{Fiber Product} to be the set
\begin{equation}
X\times_{S}Y:=\{(x,y)\in X\times Y\mid f_{X}(x)=f_{Y}(y)\}\subset X\times Y.
\end{equation}
\end{definition}

\begin{remark}
Since $X\times_{S}Y\subset X\times Y$, there are natural maps
$q_{X}\colon X\times_{S}Y\to X$ and $q_{Y}\colon X\times_{S}Y\to Y$
given by the projection maps.
\end{remark}

\begin{example}
If $S=\{*\}$ is a singleton, then $X\times_{S}Y=X\times Y$.
\end{example}

\begin{example}
If $f_{X}(X)\cap f_{Y}(Y)=\emptyset$, then $X\times_{S}Y=\emptyset$.
\end{example}

\begin{example}
If $X\subset S$ and $Y\subset S$, using the inclusions $f_{X}(x)=x$
and $f_{Y}(y)=y$, then the fiber product is isomorphic to the
intersection $X\times_{S}Y\iso X\cap Y\subset S$ with the
isomorphism given by $(z,z)\mapsfrom z$.
\end{example}

\begin{example}
If $X\subset S$ and $Y\subset S$, if we have $f_{X}\colon X\to Y$ and 
and the inclusion $f_{Y}(y)=y$, then the fiber product is isomorphic
to the pre-image of $f$, i.e., $X\times_{S}Y\iso f^{-1}(Y)\subset S$ with the
isomorphism given by $(z,f_{X}(z))\mapsfrom z$.
\end{example}

\begin{proposition}
$\hom(*,X\times_{S}Y)=\hom(*,X)\times_{\hom(*,S)}\hom(*,Y)$
\end{proposition}

(This was surprising to me, because we have not discussed Yoneda's lemma.)

\begin{proof}
  This is because we have natural maps $q_{X}$ and $q_{Y}$ with
  \begin{equation}
f_{X}\circ q_{X}=f_{Y}\circ q_{Y}.
  \end{equation}
  Then any map $\varphi\colon *\to X\times_{S}Y$ satisfies
\begin{equation}
\vcenter{\xymatrix{
  {*}\ar[dr]^{\varphi}\ar@/^/[drr]^{\varphi_{Y}}\ar@/_/[ddr]_{\varphi_{X}}&&\\
&X\times_{S}Y\ar[d]^{f_{X}}\ar[r]^{q_{Y}} & Y\ar[d]^{f_{Y}}\\
&X\ar[r]^{f_{X}} & S}}
\end{equation}
which will give us the universal property of the fiber product.
\end{proof}

\begin{remark}
This will give us a natural way to construct closed subschemes.
\end{remark}

\begin{definition}
Let $X$, $Y$, $Z$ be schemes and $f_{X}\colon X\to S$ and $f_{Y}\colon Y\to S$
be scheme morphisms. The \define{Fiber Product} $W=X\times_{S}Y$ is a
scheme (if it exists) such that for any scheme $Z$ we have
\begin{equation}
\hom(Z,W)=\hom(Z,X)\times_{\hom(Z,S)}\hom(Z,Y).
\end{equation}
\end{definition}

\begin{theorem}
The fiber product of schemes exists.
\end{theorem}

Proving this is done in several steps. The basic idea is to prove the
fiber product of affine schemes exists, then carefully glue things
together. However, the proof is not terribly useful at this stage,
because more powerful machinery will be given in a few lectures (so
why waste time learning something which will quickly become
obsolete?).

For the proof, see (e.g.): Hartshorne~\cite[Theorem II.3.3]{hartshorne1977algebraic},
Vakil~\cite[\S10.1]{vakil2025rising}.

But we can use the fiber product of schemes to define the intersection
of schemes and the fiber for a scheme.

\begin{definition}
Let $Z_{1}$, $Z_{2}$ be closed subschemes of $X$. Then we define their
\define{Intersection} $Z_{1}\cap Z_{2}:=Z_{1}\times_{X} Z_{2}$.
\end{definition}

\begin{definition}
Let $f\colon X\to Y$ be a scheme morphism. Let $y\in Y$. Define the
\define{Fiber of $y$ over $f$} by $X_{y}:=X\times_{Y}\{y\}$ where
$\{y\}\iso\Spec(\kk(y))$. It makes sense when we write the commutative
diagram as
\begin{equation}
\vcenter{\xymatrix{X\ar[r]^{f} & Y\\
X_{y}\ar[u]\ar[r] &\ar[u] y\iso\Spec(\kk(y))}}
\end{equation}
\end{definition}

\begin{example}[Extending the base field]
Let $k\subset K$ be a subfield, let $X$ be a $k$-scheme.
Then $X_{K}=X\times_{\Spec(k)}\Spec(K)$.
\end{example}

\begin{example}
Let $L/k$ be a field, $\closure{k}=K/k$ be its algebraic closure. Let
$X=\Spec(L)$. Then $X\to\Spec(k)$. So $X$ is a point, but $X_{K}$ is
no longer a point since
\begin{equation}
L\otimes_{k}\closure{k}=\prod_{\hom_{k}(L,\closure{k})}\closure{k}
\end{equation}
is just a copy of $\closure{k}$ for each $k$-linear embedding $L\into\closure{k}$.

For concreteness, have $k=\QQ$, $L=\QQ[x]/(f(x))$ for some irreducible
$f(x)\in\QQ[x]$, then by the Chinese remainder theorem
\begin{subequations}
  \begin{align}
L\otimes\closure{k}
&=\prod\closure{k}\\
&=\frac{\closure{k}[x]}{f(x)=\prod(x-\alpha_{i})}
  \end{align}
\end{subequations}
\end{example}

\begin{example}[Intersection of 2 reduced schemes need not be reduced]
  In $\AA^{2}=X$, consider
  \begin{equation}
Z_{1}=\Spec(k[x,y]/(y-x^{2})),
  \end{equation}
  and
  \begin{equation}
Z_{2}=\Spec(k[x,y]/(y)).
  \end{equation}
Then the intuition to have is, just as we have ideals $I,J\ideal A$,
that $\Spec(A)\supset\Spec(A/I)$ and $\Spec(A)\supset\Spec(A/J)$ and
we want to compute
\begin{equation}
A\onto (A/I)\otimes_{A}(A/J)
\end{equation}
and its kernel is $I+J\ideal A$. Then $Z_{1}\cap Z_{2}$ corresponds to
the ideal $(y-x^{2},y)=(y,x^{2})$ which is a ``double point''.
\end{example}

\begin{example}
Fiber of a map (where for concreteness we can pretend $k=\CC$),
\begin{equation}
\begin{split}
k[y]\to k[x,y]\\
y\mapsto x^{2}
\end{split}
\end{equation}
Then $y=a\in\CC$ we consider $X_{y}=\Spec(k[x]/(x^{2}-a))$ and we have
the tensor product $\CC[x,y]\otimes_{\CC[y]}\CC$ where $y\mapsto a$
describes how to do this tensor product.

Now, if we instead take $k=\RR$. Then $\RR[x]/(x^{2}-a)$ is what we
look at. If $a>0$ we get $2$ real points. If $a=0$ we get a double
point. If $a<0$ we get 1 complex point.
\end{example}

\begin{example}[An interesting map]
Let $X$ be a scheme over $S$---these are all $\FF_{p}$ schemes. We
know the Frobenius map
\begin{equation}
\begin{split}
\Frob_{S}\colon&S\to S\\
&x\mapsto x^{p}.
\end{split}
\end{equation}
Moreover for any ring morphism $f$,
\begin{equation}
f(x^{p})=f(x)^{p},
\end{equation}
i.e., the Frobenius map commutes with $f$. So we have
\begin{equation}
\vcenter{\xymatrix{X\ar[d]\ar[r]^{\Frob_{X}} & X\ar[d]\\
S\ar[r] & S}}
\end{equation}
Now, we consider
\begin{equation}
\vcenter{\xymatrix{X^{(p)}\ar[d]\ar[r] & X\ar[d]\\
S\ar[r] & S}}
\end{equation}
where
\begin{equation}
X^{(p)}=X\times_{S}S
\end{equation}
is the fiber product (where $f_{S}=\Frob_{S}$). The universal property
of the fiber product gives us
\begin{equation}
\vcenter{\xymatrix{
X\ar@{.>}[dr]^{\Frob}\ar@/^1pc/[drr]^{\Frob_{X}}\ar@/_1pc/[ddr]_{f_{X}}&&\\
& X^{(p)}\ar[d]^{f_{X}^{(p)}}\ar[r] & X\ar[d]^{f_{X}}\\
& S\ar[r]_{\Frob_{S}} & S}}
\end{equation}
If
\begin{equation}
X=\Spec(\closure{\FF_{p}}[x]/(f(x,y))),
\end{equation}
then
\begin{equation}
X^{(p)}=\Spec(\closure{\FF_{p}}[x]/(f^{(p)}(x,y)))
\end{equation}
which is like looking at Frobenius acting on solutions to the polynomial.
\end{example}