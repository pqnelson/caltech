%%
%% winter-lecture02.tex
%% 
%% Made by Alex Nelson <pqnelson@gmail.com>
%% Login   <alex@lisp>
%% 
%% Started on  2026-01-15T15:50:00-0800
%% Last update 2026-01-15T15:50:00-0800
%% 

\lecture[Examples of Spec(R)]{}

Let us give some examples of $\Spec(R)$.

\begin{example}[Point]
Let $\FF$ be a field. Then $\Spec(\FF)$ consists of a single point,
namely the zero ideal $(0)$. (There is only one other ideal of a
field, namely $\FF$ itself.) This describes a point.
\end{example}

\begin{example}
Consider $\Spec(\CC[x])$. Since $\CC[x]$ is a principal ideal domain,
the irreducible polynomials will be of interest. These all look like
$\langle x-a\rangle$ for some $a\in\CC$, or $(0)$. So we have
\begin{equation}
\Spec(\CC[x])=\{(0)\}\cup\{\langle x-a\rangle\mid a\in\CC\}.
\end{equation}
We typically draw this as a line, with each prime ideal $x-a$ as a
point on the line located at $a\in\CC$. The $(0)$ ideal corresponds to
what Algebraic Geometers call the \define{Generic Point}.
\end{example}

\begin{remark}
There's some apparent disagreement over the definition of a generic
point. Vakil (\S3.6.11) says: let $X$ be a topological space, let
$x\in X$ be a point, let $K\subset X$ be a closed subset of $X$; then $x$ is a ``generic point''
for $K$ if $\closure{\{x\}}=K$.

Mumford's \textit{Red Book of Schemes}, Gortz and Wederhorn (\S2.8), say that: let $X$ be a topological space,
let $x\in X$. Then $x$ is a ``generic point'' if $\closure{\{x\}}=X$.

The Stacks Project~\stackscite{004X} says: let $X$ be a topological
space, let $Z\subset X$ be an irreducible closed subset. A ``generic point''
of $Z$ is a point $\xi\in X$ such that $Z=\closure{\{\xi\}}$.

If a topological space is not irreducible, then for each irreducible
component there exists a point whose closure is the entire irreducible
component. (Not all irreducible topological spaces have a generic
point, so not all irreducible components may have one.)
The question is whether we want schemes to be irreducible or not.
\end{remark}

\begin{example}
Let $\kk=\closure\kk$ be an algebraically closed field. Consider
$\Spec(\kk[x])$. Everything we said about $\Spec(\CC[x])$ carries over
to $\Spec(\kk[x])$: it's drawn as a line, prime ideals are points, and
there's a generic point $(0)$.
\end{example}

\begin{example}
Consider $\Spec(\RR[x])$. If you'd want to study solutions to
polynomials in $\RR[x]$, then you might think you would want to study
this. However, we need to consider the ideals generated by \emph{all}
irreducible polynomials in $\RR[x]$, which are just the linear and
quadratic polynomials,
\begin{equation}
\Spec(\RR[x])=\{(0)\}\cup\{\langle x-a\rangle\mid
a\in\RR\}\cup\{\langle x^{2}+ax+b\rangle\mid a,b\in\RR; x^{2}+ax+b\mbox{ is irreducible}\}
\end{equation}
We identify ideals generated by quadratic polynomials with
\begin{equation}
x^{2}+ax+b=(x+\alpha)(x-\overline{\alpha})
\end{equation}
where $\alpha\in\CC$ is the complex root of the polynomial. So the
geometric picture we should have would be the real line \emph{and}
complex numbers identified by their Galois orbits $\alpha\sim\overline{\alpha}$.
\end{example}

\begin{example}
The space $\Spec(\kk[x])$ when $\kk$ is not algebraically closed, it
looks a lot like the situation with $\Spec(\RR[x])$ except the
discussion needs to be extended to include \emph{all} irreducible
polynomials of $\kk[x]$. (For $\RR[x]$, these were just the linear and
quadratic polynomials, but for generic fields that may not be the
case: we may need more higher-order polynomials.)
\end{example}

\begin{example}
Consider $\Spec(\CC[x,y])$ We see
\begin{equation}
\Spec(\CC[x,y]) =\{\langle x-a,y-b\rangle\mid a,b\in\CC\}\cup\{(0)\}\cup\{\langle f\rangle\mid f\mbox{ irreducible}\}.
\end{equation}
Then each ideal generated by $\langle x-a,y-b\rangle$ is drawn as
corresponding to the point $(a,b)\in\CC^{2}$

We also have some ideals which are not of this form, for example when
$f(x,y)=y-x^{2}$ or $y^{2}-x^{3}$. So $\Spec(\CC[x,y])$ is not merely
the whole plane $\CC^{2}$, but also a bunch more. 
\end{example}

\begin{example}
Consider $\Spec(\ZZ)$. This consists of $(p)$ where $p\in\ZZ$ is a
prime number, and $(0)$. We usually draw this as the number line, with
the points $(2)$, $(3)$, $(5)$, \dots, $(p)$, \dots, all highlighted,
and then an extra point $(0)$.
\end{example}

\begin{example}
Consider $\AA^{1}_{\ZZ}=\Spec(\ZZ[t])$. We have the canonical map
$\ZZ\to\ZZ[t]$ (sending $1\mapsto1$), so we have all the prime ideals
from $\ZZ$ in $\ZZ[t]$. But at $p=2$, we have an ideal of $\ZZ[t]$
containing $2$ be $\ZZ[t]/(2)\iso\FF_{2}[t]$. So over each prime
$p\in\ZZ$, we have a fiber $\FF_{p}[t]\iso\AA^{1}_{\FF_{p}}$.
\end{example}

\begin{example}
Let $R$ be a discrete valuation ring. Then there are two prime ideals
of $R$: $(0)$ and a maximal ideal $\mmm$. So $\Spec(R)=\{(0),\mmm\}$
consists of two points. This isn't true for local rings.
\end{example}


%%%
%%% PROPERTIES OF SPEC(R)
%%%

\begin{xca}
Prove or find a counter-example: If $R$ is an integral domain, then
$\Spec(R)$ is a Noetherian topological space.
\end{xca}

\begin{proposition}
For any commutative unital ring $R$, $\Spec(R)$ is quasicompact.
\end{proposition}

The proof is basically the same everywhere (e.g., \stackscite{00E8}).

\begin{proof}
Let $\{U_{i}\}_{i\in I}$ be an open cover for $\Spec(R)$, so
\begin{equation}
\Spec(R)=\bigcup_{i\in I}U_{i}.
\end{equation}
Then $U_{i}=\Spec(R)\setminus V(S_{i})$ for some $S_{i}\subset\Spec(R)$.
This is equivalent to
\begin{equation}
\bigcap_{i\in I}V(S_{i})=\emptyset.
\end{equation}
However, we see
\begin{equation}
\bigcap_{i\in I}V(S_{i})=V\left(\bigcup_{i\in I}S_{i}\right),
\end{equation}
so
\begin{equation}
\bigcap_{i\in I}V(\langle S_{i}\rangle)=V\left(\langle\bigcup_{i\in I}S_{i}\rangle\right)=\emptyset,
\end{equation}
which implies
\begin{equation}
1\in\langle\bigcup_{i\in I}S_{i}\rangle\implies 1\in\langle\sum_{i\in I}S_{i}\rangle.
\end{equation}
But this means there must be some finite subcollection of the
$\{S_{i}\}_{i\in J}$ with $J$ finite whose ideal contains $1\in\langle\sum_{j\in J}S_{j}\rangle$.
This gives us the finite subcover, hence $\Spec(R)$ is quasi-compact.
\end{proof}