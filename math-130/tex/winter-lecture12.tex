%%
%% winter-lecture12.tex
%% 
%% Made by Alex Nelson <pqnelson@gmail.com>
%% Login   <alex@lisp>
%% 
%% Started on  2026-01-31T11:36:43-0800
%% Last update 2026-01-31T11:36:43-0800
%% 

\lecture{}

(This seems to be based on Chapter 15 of
Vakil~\cite{vakil2025rising}. But it may be useful to consult
Hartshorne~\cite[Ch~II~\S7]{hartshorne1977algebraic}.)

\subsection{Key Construction}

\begin{theorem}
The set $\Mor(X,\PP^{n}_{\ZZ})$ is \emph{naturally} in bjection with
the set of tuples $(\sheaf{L},s_{0},\dots,s_{n})$ where $\sheaf{L}$ is
an invertible sheaf on $X$ and the $s_{i}\in\sheaf{L}(X)$ are global
sections of $\sheaf{L}$ on $X$ such that $\bigcap_{i}Z(s_{i})=\emptyset$.
This data is unique up to isomorphism.
\end{theorem}

This is Hartshorne~\cite{hartshorne1977algebraic} Ch~II Theorem~7.1

\begin{proof}
\forwardproof\ Given $\varphi\colon X\to\PP^{n}$, we can set
$\sheaf{L}=\varphi^{*}(\sheaf{O}(1))$ and
$s_{i}=\varphi^{*}(x_{i})$. It is true that $\bigcap_{i}Z(s_{i})=\emptyset$
because $\bigcap_{i}Z(x_{i})=\emptyset$.

\backwardproof\ Given $(\sheaf{L},s_{0},\dots,s_{n})$ such that
\begin{equation}
\bigcap_{i}Z(s_{i})=\emptyset.
\end{equation}
We define $V_{i}\subset X$ equal to
\begin{equation}
V_{i}:=\{s_{i}\neq0\}.
\end{equation}
Then by hypothesis, the $V_{i}$ cover $X=\bigcup_{i}V_{i}$. Then we
will map $V_{i}$ to $U_{i}\subset\PP^{n}$ where
\begin{equation}
U_{i}=\Spec(\ZZ[x_{0}/x_{i},\dots,x_{i-1}/x_{i},x_{i+1}/x_{i},\dots,x_{n}/x_{i}]),
\end{equation}
given by $n$ functions
\begin{equation}
f_{j}=\frac{s_{j}}{s_{i}}.
\end{equation}
Why does this make sense?/ Well, on $V_{i}$ we have $s_{i}\in\sheaf{L}(V_{i})$,
so
\begin{equation}
s_{i}\colon\sheaf{O}_{V_{i}}\to\sheaf{L}|_{V_{i}}
\end{equation}
is nowhere vanishing. Then $s_{i}$ is an isomorphism of
sheaves. Since this \emph{function} $s_{i}$ is multiplication by the
element $s_{i}$, it's perfectly reasonable to denote the inverse
morphism as division by $s_{i}$. All we have left to check is
agreement on overlaps which is left as an exercise (i.e., check gluing).
\end{proof}

\begin{note}
If $\sheaf{L}=\sheaf{O}$, then $f_{0},\dots,f_{n}\in\sheaf{O}_{X}(X)$
gives us the map
\begin{equation}
\begin{split}
\varphi\colon & X\to\PP^{n}\\
& x\mapsto[f_{0}(x):\cdots:f_{n}(x)],
\end{split}
\end{equation}
but multiplying by a nonzero scalar gives us an isomorphism $\varphi$.
\end{note}

\begin{definition}[{Vakil~\cite[\S7.3.10]{vakil2025rising}}]
Given a scheme $W$, the functor 
\begin{equation}
\begin{split}
F_{W}\colon & \Sch_{S}\to\Set\\
& X\mapsto\hom_{S}(X,W)
\end{split}
\end{equation}
is called the \define{Functor of Points of $W$}. If $W$ is a
$k$-scheme, then
\begin{equation}
F_{W}(\Spec(k))=\hom_{k}(\Spec(k),W)
\end{equation}
is like the collection of points.
\end{definition}

\begin{remark}
The Yoneda lemma says that $W$ is determined by this functor.
\end{remark}

\begin{aside}[Yoneda's Lemma]
If $\cat{C}$ is a locally small category, and $A\in\cat{C}$ is some
object, we define $h_{A}(-):=\hom(A,-)$ to be the covariant
hom-functor which sends a morphism $f\colon X\to Y$ to compose on the
left by $f$, i.e., for any $g\in\hom(A,X)$ (so $g\colon A\to X$) we
have $h_{A}(f)(g)=f\circ g\colon A\to Y$.

Now, Yoneda's lemma says for any functor $F\colon\cat{C}\to\Set$, for
each object $A\in\cat{C}$ the natural transformations $\Nat(h_{A},F)$
from $h_{A}$ to $F$ are bijective with elements of $F(A)$. This means
\begin{equation}
\Nat(h_{A},F)\iso F(A).
\end{equation}
Moreover, this isomorphism is natural in $A$ and $F$ when both sides
are regarded as functors $\cat{C}\times\Set^{\cat{C}}\to\Set$.

There is a contravariant version of Yoneda's lemma, working with a
contravariant hom-functor $h^{A}(-):=\hom(-,A)$. For any contravariant
functor $G\colon\cat{C}\to\Set$, Yoneda's contravariant lemma gives
$\Nat(h^{A},G)\iso G(A)$.
\end{aside}

\begin{node}
We are usually introduced to projective space as $\PP^{n}_{\CC}\mathrel{\mbox{``=''}}$ 1-dimensional subspaces of $\CC^{n+1}$.
So how does this all connect?

Well, we should think of $X\to\PP^{n}$ as a family of lines through
the origin. So $X\to\PP^{n}$ is the same as a line bundle and
$(n+1)$-sections. 
\end{node}