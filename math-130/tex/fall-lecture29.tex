%%
%% fall-lecture29.tex
%% 
%% Made by Alex Nelson <pqnelson@gmail.com>
%% Login   <alex@lisp>
%% 
%% Started on  2025-12-08T10:26:18-0800
%% Last update 2025-12-08T10:26:18-0800
%% 

\lecture{}

\begin{proposition}
If $X$ is locally factorial, then $\Pic(X)=\Cl(X)$.
\end{proposition}

\begin{proof}
Suffices to prove every prime divisor $[Y]$ is Cartier.

First: if $U=X\setminus Y$, then $[Y]|_{U}=0\in\Div(U)$ is principal.
Let $p\in Y$, then $I(Y)\StructureSheaf_{Y,p}\subset\StructureSheaf_{X,p}$ is a height 1 prime.
Then $\StructureSheaf_{X,p}$ is a unique factorization domain.
Then $I(Y)\StructureSheaf_{X,p}=(f)$ for some $f\in\StructureSheaf_{X,p}\subset\FunField[\kk]{X}$.
Note: $v_{Y}(f)=1$.
If we look at another prime divisor $Z\neq Y$, and $p\in Z$, then
$f\in\StructureSheaf_{X,Z}$ and $f\notin I(Z)\StructureSheaf_{X,p}$.
This implies $v_{Z}(f)=0$. So 
\begin{equation}
(f)=[Y] + \sum n_{i}[Z_{i}]
\end{equation}
where $p\notin Z_{i}$ for any $i$. Looking at $U:=X\setminus(\bigcup_{i}Z_{i})$,
then $p\in U$ and $[Y]|_{U}=(f)|_{U}\in\Div(U)$ principal. This
implies $[Y]$ is Cartier.
\end{proof}

\begin{example}
Consider the cone $X=V(xy-z^{2})\subset\AA^{3}$, let $X_{0}=X\setminus\{(0,0,0)\}$.
Then $X_{0}$ is nonsingular, so
\begin{equation}
\Pic(X_{0})=\Cl(X_{0})=\Cl(X)\iso\ZZ/2\ZZ,
\end{equation}
and
\begin{equation}
\Pic(X)=0.
\end{equation}
There exists a unique nontrivial line bundle on $X_{0}$ which is not
the restriction of a line bundle on $X$.
\end{example}

\subsection{Divisors on Curves}

\begin{definition}[{Hartshorne~\cite{hartshorne1977algebraic} Ch.II~\S3.2, pg.94}]
A morphism $f\colon X\to Y$ is called \define{Finite} if there exists
an open covering of $Y$ by open affine subsets $V_{i}=\Spec(B_{i})$
such that for each $i$, $f^{-1}(V_{i})$ is affine and
$f^{-1}(V_{i})=\Spec(A_{i})$ where $A_{i}$ is a $B_{i}$-algebra which
is a finitely-generated $B_{i}$-module.
\end{definition}

\begin{fact}
Every closed embedding is finite.
\end{fact}

\begin{recall}
If $X$ is a complete nonsingular curve, then $X$ is projective.
\end{recall}

\begin{lemma}
If $X$ is a complete nonsingular curve, then any non-constant morphism
$\varphi\colon X\to Y$ is finite.
\end{lemma}

\begin{proof}
Without loss of generality, we can replace $Y$ with $\varphi(X)$, or
equivalently that $Y$ is a curve. Then
$\pullback{f}\colon\FunField{Y}\into\FunField{X}$ is a finite field
extension. Let $V\subset Y$ be open affine. Its coordinate ring
$\CoordRing{\kk}{V}\subset\FunField{Y}$ and we may consider its
integral closure in $\FunField{X}$, call it
\begin{equation}
A := \closure{\FunField{V}}\subset\FunField{X}.
\end{equation}
This is a finitely-generated $\CoordRing{\kk}{V}$-module.
Setting $U:=\MSpec(A)$ nonsingular curve, we have $\FunField{U}=\FunField{X}$.
We have the commutative diagram
\begin{equation}
\vcenter{\xymatrix{X\ar[r]^{f}&Y\\
U\ar[u]^{\subset}\ar[r]&V\ar[u]^{\subset}}}
\end{equation}
\textsc{Claim:} $U=f^{-1}(V)$.
Let $x\in f^{-1}(V)$. Then $\CoordRing{\kk}{V}\subset\StructureSheaf_{X,x}$
which implies $A\subset\StructureSheaf_{X,x}$ and this implies
\begin{equation}
\StructureSheaf_{X,x}=A_{P},
\end{equation}
where $P\ideal A$ is a maximal ideal. But this is only possible if
\begin{equation}
x=P\in\Spec(A)
\end{equation}
by dominance. (To prove $\StructureSheaf_{X,x}=A_{P}$, it suffices to
prove [easier] $\StructureSheaf_{X,x}\supset A_{P}$ or [harder] $\StructureSheaf_{X,x}\subset A_{P}$.)
\end{proof}

\begin{definition}
For any finite morphism (of, say, curves) $\varphi\colon X\to Y$, we
define the \define{Degree} of $\varphi$ to be 
\begin{equation}
\deg(\varphi)=[\FunField{X}:\FunField{Y}],
\end{equation}
the degree of field extensions.
\end{definition}

\subsection{Pullbacks of Divisors on Curves}

\begin{definition}
Let $f\colon X\to Y$ be a finite morphism of nonsingular curves,
let $Q\in Y$, $m_{Q}=(t)\subset\StructureSheaf_{Y,Q}$.
If $f(P)=Q$, then $\pullback{f}\colon\StructureSheaf_{Y,Q}\to\StructureSheaf_{X,P}$,
and $\pullback{f}(t)\in m_{P}\subset\StructureSheaf_{X,P}$. We write $v_{P}(t):=v_{P}(\pullback{f}(t))$.

Then the \define{Pullback} of divisor groups is a morphism $\pullback{f}\colon\Div(Y)\to\Div(X)$
sending
\begin{equation}
[Q]\mapsto\sum_{P\in f^{-1}(Q)}v_{P}(t)[Q].
\end{equation}
\end{definition}

\begin{node}[Alternative definition]
If $D\in\Div(Y)$, set $V:=Y\setminus\Support(D)$, and looking at the
invertible sheaf associated with $D$, $\sheaf{L}(D)$, and
$1\in\Gamma(V,\sheaf{L}(D))$ is a section which is commonly denoted $s=1$.
Then we can just pullback this section $\pullback{f}s=\Gamma(f^{-1}(V),\pullback{f}\sheaf{L}(D))$.
\end{node}

\begin{xca}
Show $\pullback{f}D=(\pullback{f}s)\in\Div(X)$ is the divisor
associated with the pullback of the section.
\end{xca}

\begin{definition}
Let $R$ be a domain, let $M$ be an $R$-module.
We say $M$ is \define{Torsion-Free} if for all $a\in R$, for all $m\in M$,
we have $am=0$ implies $a=0$ or $m=0$.
\end{definition}

\begin{fact}
A Finitely-generated torsion-free module over a principal ideal domain
is free. (Example: $R=\ZZ$.)
\end{fact}

\begin{definition}
Let $X$ be a nonsingular curve,
let $D=\sum n_{i}P_{i}\in\Div(X)$. We define the \define{Degree} of
$D$ to be $\deg(D):=\sum n_{i}$ the sum of the coefficients.
\end{definition}

\begin{caution}
The degree of a divisor is not well-defined (when $X$ is not complete)
on the class group.
\end{caution}

\begin{proposition}[{Hartshorne~\cite[Prop.~II~6.9]{hartshorne1977algebraic}}]
If $f\colon X\to Y$ is a finite morphism of nonsingular curves and
$D\in\Div(Y)$, then the $\deg(\pullback{f}D)=\deg(f)\deg(D)$.
\end{proposition}

\begin{proof}
Suffices to prove this for a principal divisor $Q\in Y$ such that $\deg(\pullback{f}[Q])=\deg(f)$.
Let $V\subset Y$ open affine containing $Q\in V$, then
$f^{-1}(V)=\MSpec(A)\subset X$ where
$A=\closure{\CoordRing{\kk}{V}}\subset\FunField{X}$ is the integral
closure of the coordinate ring of $V$. Then abusing notation we have
$Q\ideal\CoordRing{\kk}{V}$ is a maximal ideal,
$B:=A_{Q}=(\CoordRing{\kk}{V}\setminus Q)^{-1}A$. Then $A$ is a
finitely-generated $\CoordRing{\kk}{V}$-module which implies $B$ is a
finitely-generated $\CoordRing{\kk}{V}_{Q}=\StructureSheaf_{Y,Q}$-module.
Then $B$ is torsion-free and $\StructureSheaf_{Y,Q}$ is a discrete
valuation ring.
Then $B$ is a free $\StructureSheaf_{Y,Q}$-module.
Then
\begin{subequations}
  \begin{align}
\rank_{\StructureSheaf_{Y,Q}}(B) &= \dim_{\FunField{Y}}\FunField{X}\\
&=\deg(f),
  \end{align}
\end{subequations}
where this first equation is justified by the fact $B\otimes\FunField{Y}=\FunField{X}$.
If we look at the maximal ideal at $Q$,
\begin{equation}
\mmm_{Q} :=(t)\ideal\StructureSheaf_{Y,Q}
\end{equation}
is generated by $t$, so
\begin{equation}
\StructureSheaf_{Y,Q}/\mmm_{Q}=\kk,
\end{equation}
which implies
\begin{equation}
\dim_{\kk}(B/tB)=\deg(f).
\end{equation}
To see this, just tensor product the whole thing by $\StructureSheaf_{Y,Q}/\mmm_{Q}=\kk$.

Recall, points in $f^{-1}(Q)$ are in a bijective correspondence with
maximal ideals $P\ideal A$ such that $P\cap\CoordRing{\kk}{V}=Q$
(which is logically equivalent to maximal ideals in $A_{Q}=B$).

Let $f^{-1}(Q)=\{P_{1},\dots,P_{s}\}$ where each $P_{i}\ideal A$ is a
maximal ideal. Since $B$ is a normal Noetherian domain,
\begin{equation}
B=\bigcap^{s}_{i=1}B_{P_{i}},
\end{equation}
and so
\begin{equation}\label{eq:math130a:fall2025:lec29:eq-star}
tB=\bigcap^{s}_{i=1}(tB_{P_{i}}\cap B).
\end{equation}
\textsc{Claim} (Chinese remainder theorem):
\begin{equation}\label{eq:math130a:fall2025:lec29:eq-star-star}
B/tB\xrightarrow{\iso}\bigoplus^{s}_{i=1}B/(tB_{P_{i}}\cap B)
\end{equation}
We see this is injective by Equation~\eqref{eq:math130a:fall2025:lec29:eq-star}.
To see surjectivity: We know $t\in P_{i}$ for each $i$, and
$B_{P_{i}}$ is a discrete valuation ring. Then $tB_{P_{i}}=(P_{i}B_{P_{i}})^{n_{i}}$,
which implies
\begin{equation}
tB_{P_{i}}\cap B\subset P_{i}B,\quad\mbox{and}\quad tB_{P_{i}}\cap
B\nsubset P_{j}B
\end{equation}
for $j\neq i$. Then Equation~\eqref{eq:math130a:fall2025:lec29:eq-star-star}
describes an isomorphism after tensoring by $\otimes B_{P_{i}}$.

Now,
\begin{subequations}
  \begin{align}
B/(tB_{P_{i}}\cap B) &= (B/(tB_{P_{i}}\cap B))_{P_{i}}\\
&= (B/tB)_{P_{i}}\\
&= B_{P_{i}}/tB_{P_{i}}\\
&=\StructureSheaf_{X,P_{i}}/t\StructureSheaf_{X,P_{i}},
  \end{align}
\end{subequations}
which implies $\dim_{\kk}(B/(tB_{P_{i}}\cap B))=v_{P_{i}}(t)$,
and this gives us
\begin{subequations}
  \begin{align}
\deg(\pullback{f}[Q]) &= \sum\dim_{\kk}(B/(tB_{P_{i}}\cap B)\\
&=\dim_{\kk}(B/tB)\\
&=\deg(f).
  \end{align}
\end{subequations}
Hence the result.
\end{proof}