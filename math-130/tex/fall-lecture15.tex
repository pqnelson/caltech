%%
%% fall-lecture15.tex
%% 
%% Made by Alex Nelson <pqnelson@gmail.com>
%% Login   <alex@lisp>
%% 
%% Started on  2025-11-01T11:48:27-0700
%% Last update 2025-11-01T11:48:27-0700
%% 

\lecture{}

\begin{remark}
A few words about Blow-Ups: the Blow-up really depends on the ideal
$I$ and not just $V(I)$. For example, the cusp $C$ described by the
curve $y^{2}-x^{3}$ in the plane, we have
\begin{equation}
\CoordRing{\kk}{C} = \kk[x,y]/\langle y^{2}-x^{3}\rangle
\end{equation}
we see there are ideals
\begin{equation}
I = \langle\bar{x},\bar{y}\rangle,\quad\mbox{and}\quad J=\langle\bar{x}\rangle
\end{equation}
such that $I=\Radical{I}=\Radical{J}\neq J$ beause $\bar{y}\notin J$.
Then $V(I)=V(J)=\{(0,0)\}$. But $\Bl_{I}(C)\neq\Bl_{J}(C)=C$. When we
say B lowing-Up at a subvariety $Y$, we are talking about the radical
ideal $I(Y)$.
\end{remark}

\begin{proposition}[Generic smoothness]
$X_{\text{sing}}\propersubset X$ closed proper.
\end{proposition}

\begin{proof}
For $X=V(f)\subset\AA^{n}$ is a hypersurface, where $f$ is a
non-constant irreducible polynomial $f\in\kk[x_{1},\dots,x_{n}]$.
Assume $X$ is singular everwhere. Then the Jacobian matrix $J$ must
have rank zero everwhere (by Jacobian criterion),
\begin{equation}
\rank(J)=\rank(\partial f/\partial x_{i})=0.
\end{equation}
Then $\partial f/\partial x_{i}\in I(X)=\langle f\rangle$ by Nullstellensatz.
But $\partial f/\partial x_{i}$ must have a smaller degree than $f$,
so these must vanish for all $i$, $\partial f/\partial x_{i}=0$.

If $\Char(\kk)=0$, then we have our contradiction.

If $\Char(\kk)\neq0$, then
\begin{equation}
\begin{split}
f &= \sum_{i_{1},\dots,i_{n}}a_{i_{1},\dots,i_{n}}x_{1}^{pi_{1}}(\cdots)x_{n}^{pi_{n}}\\
&= \left(\sum\dots\right)^{p}
\end{split}
\end{equation}
by Freshmen's dream. Hence we contradict $f$ being irreducible.
\end{proof}

\begin{theorem}
Every irreducible variety $X$ of dimension $n$ is birational to a
surface in $\AA^{n+1}$.
\end{theorem}

\begin{proof}
\textsc{Stipulation 1:} $\trdeg\FunField{X}=n$ implies $\FunField{X}$
can be realized as a finite and separable extension of $\kk[x_{1},\dots,x_{n}]$.

\textsc{Stipulation 2:} If we have such a field extension, then using
the primitive element theorem implies $\FunField{X}$ can be generated
b y a single element $y$ over $\kk[x_{1},\dots,x_{n}]$.

Then $\FunField{X}\iso\kk(x_{1},\dots,x_{n})[x_{n+1}]/(f)$ where $f$
is an irreducible polynomial with coefficients in $\kk(x_{1},\dots,x_{n})$.
Clear the denominators and then consider $f\in\kk[x_{1},\dots,x_{n},x_{n+1}]$
with coefficients in $\kk$. We see that
\begin{equation}
\kk(x_{1},\dots,x_{n})[x_{n+1}]/(f)\iso(\kk[x_{1},\dots,x_{n},x_{n+1}]/(f))_{0}
\end{equation}
where the right-hand side is the field of fractions of the ring $\kk[\dots]/(f)$.
Hence the result.
\end{proof}

\subsection{Nonsingular Curves over Algebraically Closed Fields}

\begin{node}[Motivating Question]
If $K$ is some function field of dimension 1 over $\kk$ (where
$\kk\subset K$ ios a finitely-generated field extension of $\trdeg_{\kk}(K)=1$),
can we always find a non-singular curve $C$ such that $K=\FunField{C}$?
\end{node}

First, a review of a few topics in commutative algebra.

\begin{definition}
Let $R$ be an integral domain, let $K$ be its field of fractions.
We say $R$ is a  \define{Valuation Ring} of $K$ if for all $x\in K\setminus\{0\}$,
$x\in R$ or $x^{-1}\in R$.
\end{definition}

\begin{proposition}
\begin{enumerate}
\item A valuation ring $R$ is a local ring;
\item Valuation rings are integrally closed.
\end{enumerate}
\end{proposition}

\begin{proof}
\begin{enumerate}
\item Consider all non-units of $R$ which is an ideal of $R$, hence
  closed under scalar multiplication and addition. Let $I=\{x\in R\mid
  x\mbox{ is not unit}\}$.
  Then for all $x\in I$, either $x=0$ or $x^{-1}\notin R$.
  Then for all $a\in R$ and $x\in I$, we have $ax\in \mathfrak{m}$ or
  ($(ax)^{-1}\in R$ which implies $x^{-1}\in R$ which implies
  contradiction).
  And for all non-zero $x,y\in I\setminus0$, we have
  \begin{equation}
x+y=x(1+x^{-1}y)=y(y^{-1}x+1),
  \end{equation}
  so either $x^{-1}y\in R$ or $y^{-1}x\in R$, and both implies $x+y\in I$.
\item Let $x^{n}+b_{1}x^{n-1}+\cdots+b_{0}=0$ where $x\neq0$ and
  $b_{i}\in R$ for each $i$. Then either $x\in R$ or $x^{-1}\in R$,
  but if $x^{-1}\in R$ we have:
  \begin{equation}
x=-b_{1}x^{-1}-\cdots-b_{n}x^{-n-1}.
  \end{equation}
  Hence $x\in R$.
\end{enumerate}
\end{proof}

\begin{definition}
Let $(A,\mathfrak{m}_{A})$, $(B,\mathfrak{m}_{B})$ be local rings which are subrings of the same field $\kk$.
We say that $B$ \define{Dominates} $A$ if $A\subset B$ and $\mathfrak{m}_{A}=A\cap\mathfrak{m}_{B}$.
\end{definition}

\begin{theorem}
Let $(R,\mathfrak{m})\subset\kk$ be a local ring.
Then $R$ is a valuation ring with respect to $\kk$ if and only if $R$
is maximal with respect to domination.
\end{theorem}

\begin{proof}
\forwardproof\ Suppose $R\subset S\subset\kk$ and $S$ dominates $R$.
Then $S$ is local and $\mathfrak{m}_{S}\cap R=\mathfrak{m}_{R}$. We
want to show $R=S$.

For any $y\in S$, we have either $s\in R$ or $s^{-1}\in R$.
If $y^{-1}\in R$, then $y\notin\mathfrak{m}_{S}$. Then $y\notin\mathfrak{m}_{R}$.
Since $y$ is not a non-unit in $R$, we have $y$ must be in $R$.
\end{proof}

\begin{node}
If we have a curve $C$, it is a variety of dimension 1 over $\kk$,
and if $p\in C$ is a smooth point, then its stalk $\RegularFuns_{C,p}$
is a regular local ring of dimension 1.
\end{node}

\begin{theorem}\label{thm:fall-lec15:tfae}
Let $(R,\mathfrak{m})$ be a Noetherian local ring of dimension 1,
$\kk:=R/\mathfrak{m}$, and its field of fractions $K=R_{0}$. The
following are equivalent:
\begin{enumerate}
\item $R$ is a discrete evaluation ring of $K/\kk$;
\item $R$ is integrally closed and $R\subset K$;
\item $R$ is regular;
\item $\mathfrak{m}$ is principal.
\end{enumerate}
(Nakayama lemma proves these last two are equivalent.)
\end{theorem}

\begin{definition}
Let $R$ be an integral domain, let $K=R_{0}$ its field of fractions.
Suppose $\kk\subset K$.
We call $R$ a \define{Discrete Valuation Ring} of $K$ over $\kk$
if
\begin{enumerate}
\item there exists a \define{Valuation} $v\colon K\setminus0\to\ZZ$
  such that $v(xy)=v(x)+v(y)$ and $v(x+y)=\min(v(x),v(y))$ and for any
  $x\in\kk\setminus0$ we have $v(x)=0$;
\item $R=\{x\in K\mid v(x)\geq0\}\cup\{0\}$.
\end{enumerate}
\end{definition}

\begin{remark}
\begin{enumerate}
\item Note that property (1) for a discrete valuation ring of $K$ over
  $\kk$ implies $R$ is a valuation ring. Further, $R$ is a local
ring whose maximal ideal $\mathfrak{m}=\{x\in K\mid v(x)>0\}\cup\{0\}$.
\item Then the theorem implies $\RegularFuns_{C,p}$ is a discrete
  valuation ring of $K$ over $\kk$ where $K=\FunField{C}$ is the
  function field for the curve.
\end{enumerate}
\end{remark}

\begin{example}
Let 
\begin{equation}
\kk\subset\underbrace{\kk[t]_{\langle t\rangle}}_{\RegularFuns_{C,0}}\subset\kk(t),
\end{equation}
let $f\in\kk(t)$ be equal to $f=p/q$ where $p=t^{n}p_{0}$ and $q=t^{m}q_{0}$
where $p_{0}(0)\neq0$ and $q_{0}(0)\neq0$. Then $f=u\cdot t^{n-m}$
where $u\in\RegularFuns_{C,0}\setminus\mathfrak{m}_{0}$. Write
$v(f)=n-m$ for the order of vanishing.
\end{example}

\begin{proposition}\label{prop:fall-lec15:domain-int-closed-iff-every-localization-closed}
A domain $A$ is integrally closed if and only if every maximal ideal
$P\in\MSpec(A)$ has $A_{P}$ be integrally closed.
\end{proposition}

\begin{proof}
\forwardproof\ Trivial.
\backwardproof\ Since $A=\bigcap A_{P}\subset\FunField{X}$.
\end{proof}

\begin{definition}
A \define{Dedekind Domain} is an integrally closed Noetherian domain
of dimension 1.
\end{definition}

\begin{node}
If $X$ is an irreducible affine variety, then $X$ is a nonsingular
curve if and only if $\CoordRing{\kk}{X}$ is a Dedekind domain. The
proof requires using Theorem~\ref{thm:fall-lec15:tfae} and Proposition~\ref{prop:fall-lec15:domain-int-closed-iff-every-localization-closed}.
\end{node}

\begin{theorem}[Finiteness of integral closure]
Let $R$ be a finitely-generated domain over $\kk$, let $K=R_{0}$ be
the field of fractions. Suppose $K\subset L$ is a finite field extension.
Let $\closure{R}$ be the integral closure of $R$ in $L$.
[Note $(\closure{R})_{0}=L$ the field of fractions of $\closure{R}$ is
  precisely $L$. To see this: let $f\in L$, then
  $f^{n}+a_{1}f^{n-1}+\cdots+a_{n}=0$ where $a_{i}\in K$.
  Take $b\in R$ such that for all $i$, $ba_{i}\in R$ --- simply clear
  the denominators. Then $(bf)^{n}+b^{1}a_{1}(bf)^{n-1}+\dots+b^{n}a_{n}=0$
  implies $bf\in\closure{R}$ which then implies $f\in R_{0}$.]

Then $\closure{R}$ is finitely-generated as an $R$-module.
In particular, this means $\closure{R}$ is a finitely-generated $\kk$-algebra.
\end{theorem}

\begin{construction}[Key in answering motivation question]\label{construction:fall-lec15:key}
Let $f\in K\setminus\kk$. Then $\kk(f)\subset K$ must be an algebraic extension.
Set $B=\closure{\kk[f]}$ be the integral closure of $\kk[f]$ in $K$.
Then $B$ must be a finitely-generated $\kk$-algebra (hence Noetherian),
and moreover $B$ must be a Dedekind domain.
Its field of fractions is $B_{0}=K$.
\end{construction}

\begin{proposition}
Using the setup from the previous Construction~\ref{construction:fall-lec15:key},
let $X=\MSpec(B)$ be a nonsingular curve.
\begin{enumerate}
\item The points in $X$ are in a bijective correspondence with
  discrete valuation rings $R$ of $K/\kk$ with $f\in R$ [this
    bijection is $p\mapsto\RegularFuns_{X,p}$];
\item Points in vanishing of $f=V(f)$ are in bijective correspondence
  with the discrete valuation rings $R$ of $K/\kk$ such that $f\in\mathfrak{m}_{R}$.
\end{enumerate}
\end{proposition}