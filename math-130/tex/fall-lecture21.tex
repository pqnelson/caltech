%%
%% fall-lecture21.tex
%% 
%% Made by Alex Nelson <pqnelson@gmail.com>
%% Login   <alex@lisp>
%% 
%% Started on  2025-11-15T11:57:44-0800
%% Last update 2025-11-15T11:57:44-0800
%% 

\lecture[Sheaves]{}

\begin{node}[Review of Sheafification]
Let $\sheaf{F}$ be a presheaf on $X$. We will denote the
\define{Sheafification} of $\sheaf{F}$ to be the sheaf
$\sheaf{F}^{+}$ as follows:
if $U\subset X$ open, then $\sheaf{F}^{+}(U)$ is the set of all
functions $s\colon U\to\bigsqcup_{p\in U}\sheaf{F}_{p}$ such that
\begin{enumerate}
\item $s(p)\in\sheaf{F}_{p}$ for all $p\in U$, and
\item $\forall p\in U\ldotp\exists V\subset U\ldotp V\mbox{ is open}\land p\in V\land\exists t\in\sheaf{F}(V)\ldotp\forall q\in V\ldotp s(q)=t_{q}$.
\end{enumerate}
Moreover, we always have a morphism (of presheaves) $\theta\colon\sheaf{F}\to\sheaf{F}^{+}$
defined by if $t\in\sheaf{F}(U)$, then $\theta_{U}(t)=[p\mapsto t_{p}]\in\sheaf{F}^{+}(U)$.
This induces an isomorphism \underline{on stalks} $\theta_{p}\colon\sheaf{F}_{p}\to\sheaf{F}^{+}_{p}$
which is an isomorphism for each $p\in X$.
\end{node}

\begin{node}[Universal property of sheafification]
If $\sheaf{F}$ is a presheaf, $\sheaf{G}$ is a sheaf, and
$\varphi\colon\sheaf{F}\to\sheaf{G}$ is a morphism, there is a
unique morphism which makes the following diagram commute
\begin{equation}
\vcenter{\xymatrix{
\sheaf{F}\ar[rr]^{\varphi}\ar[dr]_{\theta} & & \sheaf{G}\\
&\sheaf{F}^{+}\ar@{..>}[ur]&}}
\end{equation}
This determines sheafification up to (unique?) isomorphism.
\end{node}

\begin{node}
If a presheaf $\sheaf{F}$ is a subsheaf of $\sheaf{G}$, then
$\sheaf{F}^{+}$ is a subsheaf of $\sheaf{G}$.
\end{node}

\begin{definition}
Let $\varphi\colon\sheaf{F}\to\sheaf{G}$ be a morphism of sheaves
of Abelian groups.
\begin{enumerate}
\item The \define{Kernel} of $\varphi$ is the sheaf $\ker(\varphi)=[U\mapsto\ker(\varphi_{U})\subset\sheaf{F}(U)]$
\item The \define{Image} of $\varphi$ is the sheaf obtained from the sheafification
  $\Im(\varphi)=[\mbox{Presheaf }U\mapsto\Im(\varphi_{U})\subset\sheaf{G}(U)]^{+}$
\end{enumerate}
Observe $\ker(\varphi)$ is a subsheaf of $\sheaf{F}$, and
$\Im(\varphi)$ is a subsheaf of $\sheaf{G}$.
\end{definition}

\begin{node}
\begin{enumerate}
\item $\varphi$ is injective if and only if $\ker(\varphi)=0$
\item $\varphi$ is surjective if and only if
  $\Im(\varphi)=\sheaf{G}$ (which does not imply $\varphi_{U}$ is surjective!)
\item $\varphi$ is surjective if and only if
  $\varphi_{p}\colon\sheaf{F}_{p}\to\sheaf{G}_{p}$ is surjective
  for each $p\in X$.
\end{enumerate}
\end{node}

\begin{definition}
Let $\sheaf{F}'\subset\sheaf{F}$ be sheaves. We define the
\define{Quotient Sheaf} $\sheaf{F}/\sheaf{F}':=[U\mapsto\sheaf{F}(U)/\sheaf{F}'(U)]^{+}$
which is just the sheafification of the presheaf sending $U$ to the
quotient group $\sheaf{F}(U)/\sheaf{F}'(U)$.

Note there is a natural map $\sheaf{F}\onto\sheaf{F}/\sheaf{F}'$ and
its kernel is precisely $\sheaf{F}'$.
\end{definition}

\begin{notation}
Suppose we have a sequence of sheaves
\begin{equation}
\dots\to\sheaf{F}^{i}\xrightarrow{\varphi_{i}}\sheaf{F}^{i+1}\xrightarrow{\varphi_{i+1}}\sheaf{F}^{i+2}\to\dots
\end{equation}
We say it's a \define{Complex} iff $\varphi_{i+1}\circ\varphi_{i}=0$
for all $i$.

We say it is \define{Exact} iff $\Im(\varphi_{i})=\ker(\varphi_{i+1})$
for all $i$.

Equivalently, the sequence is a complex (resp., exact) iff at the level of stalks
\begin{equation}
\dots\to\sheaf{F}^{i}_{p}\to\sheaf{F}^{i+1}_{p}\to\dots
\end{equation}
is a complex (resp., exact) for all $p\in X$.
\end{notation}

\begin{example}
$0\to\sheaf{F}'\to\sheaf{F}\to\sheaf{F}''\to0$ is exact if and only if
  $\sheaf{F}'\subset\sheaf{F}$ and $\sheaf{F}/\sheaf{F}'\iso\sheaf{F}''$.
\end{example}

\begin{definition}
Let $f\colon X\to Y$ be a continuous function of topological spaces.
If $\sheaf{F}$ is a sheaf on $X$, then its \define{Pushforward} by $f$
is the sheaf $\pushforward{f}\sheaf{F}$ on $Y$ defined by
$(\pushforward{f}\sheaf{F})(V)=\sheaf{F}(f^{-1}(V))$ for any open
subset $V\subset Y$.
\end{definition}

\begin{example}[Ideal sheaf]
Let $X$ be a variety and $Y$ be a closed subset with the inclusion
morphism $i\colon Y\into X$. For $U\subset X$ open, set
\begin{equation}
\IdealSheaf{Y}(U):=\{f\in\StructureSheaf_{X}(U)\mid \forall y\in U\cap Y\ldotp f(y)=0\},
\end{equation}
then $\IdealSheaf{Y}(U)\ideal\StructureSheaf_{X}(U)$ is a subsheaf of ideals.

The quotient
\begin{equation}
\StructureSheaf_{X}(U)/\IdealSheaf{Y}(U)=\{\mbox{regular functions }U\cap Y\to\kk\mbox{ which can be extended to all of }U\}.
\end{equation}
We have
$\StructureSheaf_{X}/\IdealSheaf{Y}\iso\pushforward{i}\StructureSheaf_{Y}$
where $\pushforward{i}\StructureSheaf_{Y}(U)=\StructureSheaf_{Y}(U\cap Y)$.
Why are these sheaves isomorphic? Every regular function on $Y$ can be
extended to $X$ locally. This would imply the sheaves are isomorphic
\emph{\underline{on} \underline{stalks}}, which is sufficient for proving two
sheaves are isomorphic. (Note: this generically does not work for
proving two presheaves are isomorphic.)

Then
\begin{equation}
0\to\IdealSheaf{Y}\to\StructureSheaf_{X}\to\pushforward{i}\StructureSheaf_{Y}\to0
\end{equation}
is a short exact sequence which is \emph{analogous} to the sequence
\begin{equation}
0\to I\to A\to A/I\to 0
\end{equation}
where $I\ideal A$ is an ideal of a [commutative] ring.

\textsc{Caution:} Some people drop the $\pushforward{i}$ when writing
the short exact sequence $0\to\IdealSheaf{Y}\to\StructureSheaf_{X}\to\StructureSheaf_{Y}\to0$.
\end{example}

\begin{example}
Let $f\colon X\to Y$ be a morphism of ringed spaces. This gives us a
morphism of structure sheafs
\begin{equation}
f^{*}\colon\StructureSheaf_{Y}\to\pushforward{f}\StructureSheaf_{X}
\end{equation}
which acts on sections by
\begin{equation}
\StructureSheaf_{Y}(V)\mapsto\StructureSheaf_{X}(f^{-1}(V)),
\end{equation}
given by $h\mapsto h\circ f=f^{*}h$.
\end{example}

\begin{xca}
Find a morphism of varieties $f\colon X\to Y$ such that
\begin{equation*}
f^{*}\colon\StructureSheaf_{Y}\to\pushforward{f}\StructureSheaf_{X}
\end{equation*}
is an isomorphism but $f$ is not an isomorphism.
\end{xca}
% Inclusion $\AA^{2}\setminus0\into\AA^{2}$

\subsection{Inverse Image Sheaves}

\begin{definition}
Let $f\colon X\to Y$ be a continuous map of topological spaces.
Let $\sheaf{G}$ be a sheaf on $Y$.
We consider a presheaf on $X$ sending
\begin{equation}
U\mapsto\colim_{V\supset f(U)}\sheaf{G}(V).
\end{equation}
Since this is defined by colimits, it is born with maps
\begin{equation}
\begin{split}
&\sheaf{G}(V)\to\mbox{pre-}f^{-1}\sheaf{G}(U)\quad\forall f(U)\subset V\\
&s\mapsto f^{-1}s.
\end{split}
\end{equation}
The sections of this presheaf $\mbox{pre-}f^{-1}\sheaf{G}$ are
\begin{equation}
\mbox{pre-}f^{-1}\sheaf{G}(U)=\{f^{-1}s\mid s\in\sheaf{G}(V),V\supset f(U)\}.
\end{equation}
We see $f^{-1}s=0$ if and only if $s|_{W}=0\in\sheaf{G}(W)$ where
$f(U)\subset W\subset V$.
We define the \define{Inverse Image of $\sheaf{G}$ by $f$} to be the
sheafification of
\begin{equation}
f^{-1}\sheaf{G} := (\mbox{pre-}f^{-1}\sheaf{G})^{+}.
\end{equation}
\end{definition}

\begin{node}[Special case]
When $i\colon X\into Y$ is an inclusion, then
$\sheaf{G}|_{X}:=i^{-1}\sheaf{G}$ is the notation for the inverse
image of $\sheaf{G}$ under inclusion (also called \define{Restriction}).
\end{node}

\begin{example}
Let $X\subset Y$ be an open subset.
Then $\mbox{pre-}i^{-1}\sheaf{G}(U)=\sheaf{G}(U)$ is already a sheaf.
\end{example}

\begin{xca}
Show the inverse image sheaf has the same stalk
\begin{equation}
(f^{-1}\sheaf{G})_{p}=\sheaf{G}_{f(p)}\quad\forall p\in X.
\end{equation}
\end{xca}

\begin{node}[Adjoint property]
Let $f\colon X\to Y$ be continuous, let $\sheaf{F}$ be a sheaf on $X$,
and $\sheaf{G}$ be a sheaf on $Y$. Let
$\varphi\colon\sheaf{G}\to\pushforward{f}\sheaf{F}$ be a morphism of
sheaves on $Y$. If $U\subset X$ is open, and $V\supset f(U)$, then
\begin{equation}
\sheaf{G}(V)\xrightarrow{\varphi_{V}}\sheaf{F}(f^{-1}(V))\to\sheaf{F}(U)
\end{equation}
is compatible with restriction maps on $\sheaf{G}$, and so it induces
\begin{equation}
\psi_{U}\colon\mbox{pre-}f^{-1}\sheaf{G}(U)\to\sheaf{F}(U),
\end{equation}
and $\varphi^{+}\colon f^{-1}\sheaf{G}\to\sheaf{F}$.
\end{node}

\begin{xca}
Prove or find a counter-example: Let
\begin{equation}
\begin{split}
\hom(\sheaf{G},\pushforward{f}\sheaf{F})&\xrightarrow{\sim}\hom(f^{-1}\sheaf{G},\sheaf{F})\\
\varphi&\mapsto\varphi^{+}
\end{split}
\end{equation}
be isomorphisms of Abelian groups. Then $f^{-1}$ is left adjoint to the pushforward
$\pushforward{f}$, and $\pushforward{f}$ is right adjoint to $f^{-1}$.
\end{xca}

\begin{definition}
Let $X$ be a ringed space.
An \define{$\StructureSheaf_{X}$-Module} is a sheaf $\sheaf{F}$ on $X$
such that its sections $\sheaf{F}(U)$ is an $\StructureSheaf_{X}(U)$-module:
for any $U\subset X$ and for any $V\subset U$,
$\sheaf{F}(U)\to\sheaf{F}(V)$ is an $\StructureSheaf_{X}(U)$-morphism
(i.e., if $f\in\StructureSheaf_{X}(U)$ and $\sigma\in\sheaf{F}(U)$,
then $(f\cdot\sigma)|_{V}=f|_{V}\sigma|_{V}$).
\end{definition}

\begin{definition}
A \define{$\StructureSheaf_{X}$-Homomorphism} (or
a ``morphism of $\StructureSheaf_{X}$-modules'')
$\varphi\colon\sheaf{F}\to\sheaf{G}$ is a morphism of sheaves such
that for all $U\subset X$,
$\varphi_{U}\colon\sheaf{F}(U)\to\sheaf{G}(U)$ is an $\StructureSheaf_{X}(U)$-morphism.

Note:
\begin{enumerate}
\item $\ker(\varphi)$ and $\Im(\varphi)$ are $\StructureSheaf_{X}$-submodules
\item If $\sheaf{F}$ is an $\StructureSheaf_{X}$-moduile, then
  $\sheaf{F}_{p}$ is an $\StructureSheaf_{X,p}$-module.
\end{enumerate}
\end{definition}

\begin{definition}
If $\sheaf{F}$ and $\sheaf{G}$ are $\StructureSheaf_{X}$-modules, we
can define their \define{Tensor Product} to be the sheaf
\begin{equation*}
\sheaf{F}\otimes_{\StructureSheaf_{X}}\sheaf{G} := [U\mapsto \sheaf{F}(U)\otimes_{\StructureSheaf_{X}(U)}\sheaf{G}(U)]^{+}.
\end{equation*}
\end{definition}

\begin{xca}
Prove or find a counter-example: for any $\StructureSheaf{O}_{X}$
modules $\sheaf{F}$ and $\sheaf{G}$, we have $(\sheaf{F}\otimes_{\StructureSheaf_{X}}\sheaf{G})_{p}=\sheaf{F}_{p}\otimes_{\StructureSheaf_{X,p}}\sheaf{G}_{p}$
\end{xca}

\begin{definition}
An $\StructureSheaf_{X}$-module $\sheaf{F}$ is said to be
\define{Locally Free of rank $r$} if there exists an open covering
$\bigcup_{\alpha}U_{\alpha}=X$ such that
for each $\alpha$ we have $\sheaf{F}|_{U_{\alpha}}\iso\StructureSheaf_{U_{\alpha}}^{\oplus r}$.

A locally free sheaf of rank 1 is called \define{Invertible}.
\end{definition}

\begin{remark}
These will be related to vector bundles.
\end{remark}

\begin{notation}
If $\sheaf{F}$ is a sheaf on $X$, and $U\subset X$ open,
then it is common to write the section of $\sheaf{F}$ at $U$ as
\begin{equation}
\Gamma(U,\sheaf{F}) := \sheaf{F}(U).
\end{equation}
This notation is either inspired by differential geometry, or inspired
differential geometry to adopt this notation.
\end{notation}

\begin{example}
Consider some invertible sheaves on projective space. Recall we had
\begin{equation}
\pi\colon\AA^{n+1}\setminus0\to\PP^{n}.
\end{equation}
Let $m\in\ZZ$. We can define a sheaf
\begin{equation}
\mathcal{O}(m) = \StructureSheaf_{\PP^{n}}(m)
\end{equation}
on $\PP^{n}$ by
\begin{equation}
\Gamma(U,\mathcal{O}(m))=\{h\in\kk[\pi^{-1}(U)]\mid h(\lambda x)=\lambda^{m}h(x)\forall x\in\pi^{-1}(U),\lambda\in\kk^{*}\},
\end{equation}
where $\kk[\pi^{-1}(U)]$ are the regular functions on $\pi^{-1}(U)$.

We can check this is a sheaf. Note that if $f\in S=\kk[x_{0},\dots,x_{n}]$ is
a form of degree greater zero, then $\kk[\pi^{-1}(D_{p}(f))]=\kk[D(f)]=S_{f}$
is a graded ring.
Then
\begin{equation}
\Gamma(D_{p}(f),\mathcal{O}(m))=(S_{f})_{m}
\end{equation}
is the degree $m$ part of the graded ring. This sheaf $\mathcal{O}(m)$
is an $\StructureSheaf_{\PP^{n}}$-module: if $f\in\StructureSheaf_{\PP^{n}}(U)$
and $h\in\Gamma(U,\mathcal{O}(m))$, then $f\cdot h=(f\circ \pi)\circ h\in\kk[\pi^{-1}(U)]$.
\end{example}