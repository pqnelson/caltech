%%
%% fall-lecture11.tex
%% 
%% Made by Alex Nelson <pqnelson@gmail.com>
%% Login   <alex@lisp>
%% 
%% Started on  2025-10-23T10:05:11-0700
%% Last update 2025-10-23T10:05:11-0700
%% 

\lecture{}

\begin{node}[Recap]
Last time, we induced a topology and structure sheaf using the
projection map $\pi\colon Fr(n,r)\to\Gr(r,n)$. We also saw $\Gr(r,n)$
can be covered using an open cover $\{U_{\Gamma}\}$ of affine
spaces. So in other words, we have proven it is a prevariety.

We see that any two $U_{\Gamma}$ have a nonempty intersection, each
$U_{\Gamma}$ is irreducible. This implies
$\closure{U_{\Gamma}}=\Gr(r,n)$ is irreducible.

Observe: $Gr(1,n)\iso\PP^{n}$ and $Gr(1,V)\iso\PP(V)$ as prevarieties.

Today we will prove $\Gr(r,n)$ is a variety.
\end{node}

\begin{definition}[Pl\"{u}cker coordinates]
We call the injective mapping
\begin{equation}
\begin{split}
f\colon&\Gr(r,V)\into\PP(\extp^{r}\Sigma)\\
&\Sigma\mapsto\extp^{r}\Sigma
\end{split}
\end{equation}
the \define{Pl\"{u}cker Embedding}.
\end{definition}

\begin{proposition}
The Pl\"{u}cker embedding is a morphism of prevarieties.
\end{proposition}

\begin{proof}
We can see this by precomposing it with $\pi$ to get a morphism. In
other words, the following diagram
\begin{equation}
\vcenter{\xymatrix{
FR(r,n)\ar[d]^{\pi}\ar[r]^{\det_{r\times r}} & \AA^{\binom{n}{r}}\setminus0\ar[d]\\
\Gr(r,V)\ar[r]^{f}&\PP(\extp^{r}V)}}
\end{equation}
commutes. Hence $f$ is a morphism.
\end{proof}

\begin{node}[Image of Pl\"{u}cker embedding]% Gathmann, 8.14
So what's the image of the Pl\"{u}cker embedding?

Let $w\in\extp^{r}V$. Suppose $\dim(V)=n$ (if it comes up).
Then let us consider the mapping
\begin{equation}
\begin{split}
f_{w}\colon &V\to\extp^{r+1}V\\
&v\mapsto v\wedge w.
\end{split}
\end{equation}
We claim if $w\neq0$, then $\rank(f_{w})\geq n-r$ and moreover we have
$\rank(f_{w})=n-r$ iff $w=v_{1}\wedge\cdots\wedge v_{r}$ for $r$
linearly independent nonzero vectors $v_{1}$, \dots, $v_{r}$.
\end{node}


\begin{proof}
Let $\ker(f_{w})=\langle v_{1},\dots,v_{s}\rangle$. We can extend this
to a basis $v_{1}$, \dots, $v_{n}$ of $V$. Then we can write
\begin{equation}
w = \sum_{|J|=r}a_{J}\,\extp^{r}\langle v_{j}\mid j\in J\rangle,
\end{equation}
then we see
\begin{subequations}
\begin{align}
v_{i}\wedge w &= \sum_{J}a_{J}v_{i}\wedge\left(\langle v_{j}\mid j\in J\rangle\right)\\
&= \sum_{\stackrel{J}{i\notin J}}a_{J}v_{i}\wedge\left(\langle v_{j}\mid j\in J\rangle\right),
\end{align}
\end{subequations}
then
\begin{equation}
v_{i}\wedge(\langle v_{j}\mid j\in J\rangle)=0\iff i\in J.
\end{equation}
We see $v_{i}\wedge w=0$ iff $i\in\{1,\dots,s\}$ by definition of the
kernel for $f_{w}$.

If $\{1,\dots,s\}\nsubset J$, then $a_{J}=0$. The $a_{J}$ are not all
zero. Then $s\leq|J|=r$ and so $s=r$ iff exactrly one $a_{J}\neq0$.
\end{proof}

\begin{corollary}
The image of the Pl\"{u}cker embedding is $\{\langle w\rangle\mid\rank(f_{w})=n-r\}$,
obtained by the vanishing of minors.
\end{corollary}

\subsection{Detour on Fiber Bundles}

\begin{node}
Let $E$, $B$, and $F$ be spaces.
A fiber bundle over $B$ is given by a surjection $\pi\colon E\to B$
such that for every $U\subset B$ open, $\pi^{-1}(U)\iso U\times F$.
We call $F$ the \emph{fiber}.
\end{node}

\begin{example}
Affine budnles: $E$ and $B$ are varieties over $\kk$ and
$F\iso\AA^{n}$ is the fiber.
\end{example}

\begin{example}
Vector bundles = Affine bundles + the transition functions are $\kk$-linear.
\end{example}

\begin{example}
The \define{Tautological Bundle} over the Grassmannian space
$\pi\colon\TautologicalBundle\to\Gr(r,n)$ where
$\TautologicalBundle\into\Gr(r,n)\times\AA^{n}$ may be given as the
set
\begin{equation}
\TautologicalBundle=\{(\Sigma,v)\mid v\in\Sigma,\Sigma\in\Gr(r,n)\}.
\end{equation}
We can check it hasa local trivialization. Let 
\begin{equation}
\pi^{-1}(U_{\Gamma})=\{(\Sigma,v)\mid\Sigma\cap\Gamma=\emptyset,v\in\Sigma\}.
\end{equation}
Now we want to check if this is isomorphic to $U_{\Gamma}\times\Omega$
where $\Omega\oplus U_{\Gamma}=V$ defines $\Omega$. Observe
\begin{equation}
\Sigma\into V\to V/\Gamma\iso\Omega,
\end{equation}
so this composition is an isomorphism since $\Sigma\cap\Gamma=\emptyset$.
Let us call this isomorphism $\varphi$. Then
\begin{equation}
(\Sigma,v)\mapsto(\Sigma,\varphi(v))
\end{equation}
is an isomorphism. So we have proven it is a fiber bndle. We just need
to prove its transition functions are linear, but it's easy. We can
get a short exact sequence
\begin{equation}
0\to\TautologicalBundle\to\Gr(r,n)\times\AA^{n}\to Q\to 0.
\end{equation}
We can embed
\begin{equation}
\TautologicalBundle\into\Gr(r,n)\times\AA^{n}\into\PP(\extp^{r}\kk^{n})\times\AA^{n}
\end{equation}
using Pl\"{u}cker embedding, which will produce a pair $(\omega,v)$
such that $\omega\wedge v=0$.
\end{example}

\begin{node}[Constructing morphisms from tautological bundle]
Let $Y$ be any variety and suppose we have a rank $m$ subbundle of the
trivial bundle $\mathcal{A}\subset Y\times V$. To give a morphism
$f\colon Y\to\Gr(m,n)$ is equivalent to give a rank $m$ sub-bundle
$\mathcal{A}$ of the trivial bundle $Y\times V$ given by
$f^{*}(\TautologicalBundle)$ (the preimage of the tautological
bundle).

Conversely, given any $\mathcal{A}\subset Y\times V$, we obtain a morphism
\begin{equation}
Y\to\Gr(m,n)
\end{equation}
given by $y\mapsto \mathcal{A}_{y}$. We see
\begin{equation}
f^{*}(\TautologicalBundle) = \{(x,s)\in X\times\TautologicalBundle\mid f(x)=\pi(s)\}
\end{equation}
gives the preimage (as the pullback, in the category theoretic sense).
\end{node}

\begin{remark}
We can upgrade the previous construction to produce an isomorphism of
functors, since we are looking at the functors
\begin{enumerate}
\item $\hom(-,\Gr(m,n))$ in the category of prevarieties, and
\item $X\mapsto\{\mbox{rank $m$ sub-bundles of }X\times\AA^{n}\}$.
\end{enumerate}
We say the second functor is ``represented by $\Gr(m,n)$''. This is
related to the so-called \define{Fine Moduli Space}.
\end{remark}

\begin{lemma}[Yoneda]
$\hom(-,A)$ determines $A$ up to unique isomorphism.
\end{lemma}

We won't prove this, we just remind the reader of the Yoneda lemma. We
should understand it more as a ``change of viewpoint'' (i.e., the
morphisms are the important thing, not the object).

\begin{definition}[Projective bundles]
Let $\pi\colon E\to X$ be a vector bundle. We can take  the complement
of the zero section $E_{0}\into E\to X$ which is a subset of $E$, then
we can quotient out by nonzero scalars $E_{0}/\kk^{\times}$ and this
is precisely a \define{Projective Bundle}.
\end{definition}

\begin{example}
Consider $\PP(\TautologicalBundle)=\{(\Sigma,\ell)\in\Gr(r,V)\times\PP(V)\mid\ell\subset\Sigma\}$
and we can map it onto $\Gr(r,V)$. This is an example of a projective bundle.
\end{example}

\begin{node}
Observe that $\GL(V)\times\Gr(r,V)\to\Gr(r,V)$ gives us a transitive
action in the obvious way. Then by the orbit--stabilizer Theorem, we
see
\begin{equation}
\GL(V)/P\iso\Gr(r,V)
\end{equation}
where $P$ looks like a block upper-triangular matrix in $\GL(V)$. This
can be viewed as a group scheme, and can be generalized to other group
schemes of Lie type.
\end{node}

\subsection{Birational Maps and Blowing Up}

\begin{example}
\begin{enumerate}
\item $\AA^{n}$ and $\PP^{n}$ share the same open subset $\AA^{n}$ and
  that $\PP^{n}\times\PP^{m}$ and $\PP^{m+n}$ share the same open
  subsets $\AA^{m}\times\AA^{n}\iso\AA^{m+n}$ (even though
  $\PP^{m}\times\PP^{n}$ is \underline{\emph{\textbf{not}}} isomorphic
  to $\PP^{m+n}$).
\item $\Gr(k,n)\supset\AA^{k(n-k)}$
\item The nodal curve and the affine line share the same open subset,
  which looks like $\AA\setminus\mbox{pt}$.
\end{enumerate}
In each example, we consider several \emph{different} varieties which
share \emph{the same} open subset which is dense in the varieties
presented in each example.
These all share the same function field.
\end{example}

\begin{definition}
Let $X$, $Y$ be irreducible varieties. A \define{Rational Map}
$f\colon X\dashrightarrow Y$ (the dashed arrow is the notation
indicating we have a rational map) is a morphism $f\colon U\to Y$ for some
nonempty open subset $U\subset X$.

We say rational maps $f_{1},f_{2}\colon X\dashrightarrow Y$ (defined
on $U_{1}$, resp. $U_{2}$) are the same $f_{1}\sim f_{2}$ if $U_{1}\cap U_{2}\neq\emptyset$
and for some open subset $V\subset U_{1}\cap U_{2}$ the two functions agree $f_{1}|_{V}=f_{2}|_{V}$.
It's not hard to prove that $f_{1}\sim f_{2}$ is an equivalence relation.
\end{definition}

\begin{remark}
See the discussion in the Stacks Project \href{https://stacks.math.columbia.edu/tag/01RR}{\texttt{[01RR]}}.
\end{remark}