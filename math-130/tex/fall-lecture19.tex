%%
%% fall-lecture19.tex
%% 
%% Made by Alex Nelson <pqnelson@gmail.com>
%% Login   <alex@lisp>
%% 
%% Started on  2025-11-11T09:12:59-0800
%% Last update 2025-11-11T09:12:59-0800
%% 

\lecture{}

\begin{node}
Recall, $S=\kk[x_{0},\dots,x_{n}]$ and $M$ is a graded $S$-module. The
Hilbert function for $M$ is $\mathcal{H}_{M}(d)=\dim_{\kk}(M_{d})$.
\end{node}

\begin{theorem}
If $M$ is a finitely-generated $S$-module, then there exists a unique
polynomial $P_{M}(z)\in\QQ[z]$ such that $\dim_{\kk}(M_{d})=P_{M}(d)$
for all $d\gg0$. Furthermore, $\deg(P_{M})=\dim\Support(M)$.
\end{theorem}

We will prove this at the end of the lecture.

\begin{definition}
Let $P(z)\in\QQ[z]$. We call it a \define{Numerical Polynomial} (or
just say ``$P$ is \emph{Numerical}'') if $P(d)\in\ZZ$ for all
$d\in\ZZ$ where $d\gg0$.
\end{definition}

\begin{example}
The binomial coefficients
\begin{equation}
p(z) = \binom{z}{m} = \frac{1}{m!}z(z-1)(\cdots)(z-m+1),
\end{equation}
for sufficiently large $z\in\ZZ$ (specifically when $z\geq m$), this
will turn out to be numerical.

Note: $\{\binom{z}{m}\mid m\in\NN_{0}\}$ forms a bassis for
$\QQ$-vector space $\QQ[z]$.
\end{example}

\begin{lemma}
Let 
\begin{equation}
p(z) = c_{0} + c_{1}\binom{z}{1} + \dots + c_{r}\binom{z}{r},
\end{equation}
where $c_{i}\in\QQ$ and $p(z)\in\QQ[z]$, then the following are equivalent:
\begin{enumerate}
\item $p(z)$ is numerical;
\item $p(d)\in\ZZ$ for all $d\in\ZZ$;
\item $c_{i}\in\ZZ$.
\end{enumerate}
\end{lemma}

\begin{proof}
We see that $(3)\implies(2)\implies(1)$. We will prove $(1)\implies(3)$.

Observe
\begin{equation}
\binom{z+1}{m}-\binom{z}{m}=\binom{z}{m-1},
\end{equation}
which implies
\begin{equation}
p(z+1)-p(z) = c_{1}+c_{2}\binom{z}{1}+\dots+c_{r}\binom{z}{r-1}.
\end{equation}
If $p(z)$ is numerical, then $p(z+1)-p(z)$ is numerical and
$\deg(p(z))>\deg(p(z+1)-p(z))$. By induction on $r$ we see
$c_{1},\dots,c_{r}\in\ZZ$. Then we must have $c_{0}\in\ZZ$.
\end{proof}

\begin{note}
Now, we will make $S=\bigoplus_{m\geq0}S_{m}$ a graded ring, and $M$
is a graded $S$-module.
\end{note}

\begin{definition}
Let $\ell\in\ZZ$. The \define{Twisted Module} $M(\ell)$ (called ``the
twist of $M$ by $\ell$'') is defined as
the same module as $M$ but with the grading $M(\ell)_{d}=M_{\ell+d}$.
\end{definition}

\begin{definition}
An $S$-morphism $\varphi\colon M\to N$ (of graded $S$-modules $M$ and $N$)
is called \define{Homogeneous} if $\varphi(M_{d})\subset N_{d}$ for
all $d$.
\end{definition}

\begin{definition}
Let $M$ be a graded $S$-module.
A submodule $N\subset M$ is called \define{Graded}
% (I have written in my notes \define{Homogeneous}, erroneously?)
if $N=\oplus_{d\in\ZZ}(N\cap M_{d})$. This implies $N$ and
$M/N=\bigoplus_{d}M_{d}/(N\cap M_{d})$ are both graded. We have a
short exact sequence
\begin{equation*}
0\to N\to M\to M/N\to 0
\end{equation*}
with homogeneous maps.
\end{definition}

\begin{node}
If $m\in M$ is homogeneous of degree $\ell$ (i.e., $m\in M_{\ell}$),
then its annihilator $\Annihilator(m)$ is a homogeneous ideal. If we
set $N:=Sm$, then we have the short exact sequence
\begin{equation}
0\to I(-\ell)\to S(-\ell)\xrightarrow{m}N\to 0,
\end{equation}
where $I(-\ell)$ is the twisted $\Annihilator(m)$. Then
$(S/I)(-\ell)\iso N\subset M$.
\end{node}

\begin{xca}
If $M$ is a finitely-generated module over a Noetherian ring,
then $M$ is a Noetherian module (i.e., every submodule is finitely-generated).
\end{xca}

\begin{lemma}
Let $M$ be a finitely-generated module over a Noetherian ring $S$.
Then there exists a filtration $0=M^{0}\propersubset M^{1}\propersubset\cdots\propersubset M^{r}=M$
(of nested submodules of $M$) such that each $M^{i}$ is homogeneous
and $M^{i}/M^{i-1}\iso(S/P_{i})(\ell_{i})$ for some homogeneous prime
ideal $P_{i}\in\Spec(S)$ (and $\ell_{i}\in\ZZ$).
\end{lemma}

\begin{proof}
By Noetherian induction. Consider $N\subset M$ a maximal submodule
such that the claim holds for $N$. We want to show $N=M$. If not, we
can take the quotient $M''=M/N\neq0$. Choose nonzero $0\neq m\in M''$
such that its annihilator $\Annihilator(m)\subset S$ is as large as
possible.

\textsc{Claim:} $P=\Annihilator(m)\in\Spec(S)$. Let $f,g\in S\setminus P$
be homogeneous. Suffices to show $fg\notin P$. Then $gm\neq0$ implies
$\Annihilator(gm)\supset\Annihilator(m)$, but since $\Annihilator(m)$
is as large as possible, we see
$\Annihilator(gm)=\Annihilator(m)$.
Since $f\notin\Annihilator(m)$,
we see $f\notin\Annihilator(gm)$, which implies $fg\notin\Annihilator(m)$.
This proves $P$ is prime.

Then $Sm\iso(S/P)(-\ell)$ where $\ell=\deg(m)$. But $M\onto M/N=M''\supset Sm$.
If we take the inverse image $\widetilde{N}\subset M$ of $Sm$, then
$\widetilde{N}\propersupset N$ and we get a contradiction. Hence the result.
\end{proof}

\begin{definition}
Let $f\colon\ZZ\to\ZZ$, we say $f$ is \define{Eventually Polynomial}
if there is a polynomial $p(z)\in\QQ[z]$ such that $f(n)=p(n)$ for all
integers $n\gg0$.
\end{definition}

\begin{note}
Write $\Delta f$ for the function $(\Delta f)(n)=\Delta f(n)=f(n+1)-f(n)$.
\end{note}

\begin{lemma}
We have $f$ is eventually polynomial of degree $r$ if and only if
$\Delta f$ is eventually polynomial of degree $r-1$.
\end{lemma}

\begin{proof}
\forwardproof\ Obvious.

\backwardproof\ Assume $\Delta f(n)=q(n)$ for $n\gg0$ where
\begin{equation}
q(z) = c_{1}+c_{2}\binom{z}{1}+\dots+c_{r}\binom{z}{r-1},
\end{equation}
and $q(z)\in\QQ[z]$. Set
\begin{equation}
P(z) = c_{1}\binom{z}{1}+c_{2}\binom{z}{2}+\dots+c_{r}\binom{z}{r}.
\end{equation}
Clearly $\Delta p=q$.
Then $\Delta(f-P)(n)=0$ for $n\gg0$. Then $f(n)-P(n)=c_{0}$ for all $n\gg0$.
So $f$ is eventually polynomial (since $f$ is eventually $c_{0}+P$).
\end{proof}

\begin{notation}
We let $S=\kk[x_{0},\dots,x_{n}]$ again.
\end{notation}

\begin{proposition}
If we have a short exact sequence $0\to M'\to M\to M''\to0$
of graded $S$-modules, then $\Support(M)=\Support(M')\cup\Support(M'')$.
\end{proposition}

\begin{proof}
$(\supset)$ $\Annihilator(M)\subset\Annihilator(M')\cap\Annihilator(M'')$.

$(\subset)$ Let $x\in\PP^{n}$.
If $x\notin\Support(M')\cup\Support(M'')$, then there is some
$f\in\Annihilator(M')$ and $g\in\Annihilator(M'')$ such that
$f(x)\neq0$ and $g(x)\neq0$. But then $fg\in\Annihilator(M)$ and
$(fg)(x)\neq0$. This shows $x\notin\Support(M)$.
\end{proof}

Now, we can prove the Theorem we initially presented at the start of
the lecture, recall;

\begin{theorem}
If $M$ is a finitely-generated $S$-module, then there exists a unique
polynomial $P_{M}(z)\in\QQ[z]$ such that $\dim_{\kk}(M_{d})=P_{M}(d)$
for all $d\gg0$. Furthermore, $\deg(P_{M})=\dim\Support(M)$.
\end{theorem}

\begin{proof}
There exists a filtration $0=M^{0}\propersubset M^{1}\propersubset\cdots\propersubset M^{r}=M$
such that $M^{i}/M^{i-1}=(S/P_{i})(\ell_{i})$.
Then we can write the Hilbert function as
\begin{equation}
\mathcal{H}_{M}(d) = \sum^{r}_{i=1}\mathcal{H}_{M^{i}/M^{i-1}}(d)=\sum^{r}_{i=1}\mathcal{H}_{S/P_{i}}(d+\ell_{i}),
\end{equation}
and
\begin{equation}
\Support(M) = \bigcup^{r}_{i=1}\Support(M^{i}/M^{i-1})=\bigcup^{r}_{i=1}V_{p}(P_{i}).
\end{equation}
Without loss of generality, $M=S/P$ where $P\in\Spec(S)$ is a
homogeneous prime ideal. By induction on $\dim(V_{p}(P))$:
\begin{enumerate}
\item If $P=(x_{0},x_{1},\dots,x_{n})\subset S$ is the trivial ideal,
  then $V_{p}(P)=\emptyset$ and $\mathcal{H}_{S/P}(d)=0$ for all
  $d\gg1$. The Theorem is true if we set $\deg(0)=\dim(\emptyset)=-1$.
\item Otherwise, there is some $x_{i}\notin P$. We can consider the
  ideal
  \begin{equation}
I:=(x_{i})+P\subset S.
  \end{equation}
  Then $V_{p}(I)\propersubset V_{p}(P)$. By projective dimension
  theorem [which is a homework problem],
  \begin{equation}
\dim(V_{p}(I))=\dim(V_{p}(P))-1.
  \end{equation}
  So now we can apply the inductive hypothesis, $\mathcal{H}_{S/I}(d)$
  is eventually polynomial and $\deg(P_{S/I})=\dim(V_{p}(P))-1$ (ugh,
  confusing notation: the $P$ on the left-hand side is a polynomial,
  the $P$ on the right-hand side is a prime ideal). We have the short
  exact sequence
  \begin{equation}
0\to(S/P)(-1)\xrightarrow{x_{i}}S/P\to S/I\to 0,
  \end{equation}
  which implies $\Delta\mathcal{H}_{S/P}(d-1)=\mathcal{H}_{S/P}(d)-\mathcal{H}_{S/P}(d-1)=\mathcal{H}_{S/I}(d)$.
  Then $\mathcal{H}_{S/P}$ is eventually polynomial of degree equal to
  $\dim(V_{p}(P))$. \qedhere
\end{enumerate}
\end{proof}