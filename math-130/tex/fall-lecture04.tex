%%
%% fall-lecture04.tex
%% 
%% Made by Alex Nelson <pqnelson@gmail.com>
%% Login   <alex@lisp>
%% 
%% Started on  2025-10-07T10:59:44-0700
%% Last update 2025-10-07T10:59:44-0700
%% 

\lecture[More about integral ring extensions]{}

\textsc{TODO:} There are a lot of examples in this lecture which I
should type up, and draw diagrams for.

\begin{example}
Consider the variety
\begin{equation}
X = V(y^{2}-x^{2}-x^{3})\subset\AA^{2}_{\CC},
\end{equation}
with $R:=A(X)=\CC[x,y]/\langle y^{2}-x^{2}-x^{3}\rangle$.
Is this ring normal?

No, it is not normal. Let $\varphi=\frac{y}{x}\in R_{0}$ in the field
of fractions. Then it satisfies the monic polynomial
$\varphi^{2}-x^{2}-1$, but $\varphi\notin R$. So $R$ is not normal.
\end{example}

\begin{proposition}[Lying over]\label{prop:fall-lec04:lying-over}
Let $R\subset R'$ be a ring extension, let $P\ideal R$ be a prime
ideal. Is there a corresponding prime ideal $P'\ideal R'$ such that
$P'\cap R=P$? This is true if and only if $(PR')\cap R\subset P$ which
is always the case if $R\subset R'$ is an integral extension.
\end{proposition}

\begin{example}
Let $R=\CC[x]$.
\begin{enumerate}
\item Let $R'=\CC[x,y]/\langle y^{2}-x^{2}\rangle$. This corresponds
  to the lines $y=x$ and $y=-x$. For any $P'\ideal R'$ prime, we see
  that $R\cap P'=P$ is a prime ideal.
\item Let $R'=\CC[x,y]/\langle xy-1\rangle$ which corresponds to
  hyperbolic curves. There is no $P'$ corresponding to $P$ because $y$
  does not satisfy a monic polynomial in $x$.
\end{enumerate}
\end{example}

\begin{proposition}[Incompatibility]\label{prop:fall-lec04:incompatibility}\index{Incompatibility}%
Let $R\subset R'$ be an integral ring extension, let $P'\ideal R'$
and $Q'\ideal R'$ be distinct prime ideals. Suppose they both lie over
some prime ideal $P\ideal R$. Then $P'\nsubset Q'$ and $Q'\nsubset P'$. 
\end{proposition}

\begin{corollary}
If $R\subset R'$ is an integral ring extension, then a prime ideal
$P'\ideal R'$ is maximal if and only if $P'\cap R$ is maximal.
\end{corollary}

\begin{equation*}
\xymatrix{
R' \ar@{}[r]|-*[@]{\supset} & P' \ar@{<->}[r] & \mbox{irreducible
  subvariety} & Y'\ar[d]^{f}\ar@{}[r]|-*[@]{\subset} & X'\ar[d]^{f}\\
R\ar@{^{(}->}[u]\ar@{}[r]|-*[@]{\supset} & R\cap P \ar@{<->}[r] & \mbox{irreducible
  subvariety} & f(Y')\ar@{}[r]|-*[@]{\subset} & f(X')
}
\end{equation*}

\begin{remark}
We will see ``relative versions'' of lying over. This looks like
\begin{equation*}
\begin{array}{lccc}
R' \qquad & {P'}^{(?)} & \subset & {Q'}^{(?)}\\
          & \downarrow &         & \downarrow^{(?)}\\
R \qquad  & P         & \subset & Q
\end{array}
\end{equation*}
where some of the quantities with superscript ${}^{(?)}$ are being sought.
\end{remark}

\begin{proposition}[Going up]\label{prop:fall-lec04:going-up} % Gathmann, Commutative Algebra, 9.24
Let $R\subset R'$ be an integral ring extension.
Given $P\subset Q$ prime ideals in $R$, and $P'\ideal R$' lying over
$P$, then you can always find $Q'\ideal R'$ lying over $Q$.
\end{proposition}

In pictures,
\begin{equation}
\begin{array}{ccc}
P' & {\color{red}\subset} & {\color{red}Q'}\\
\downarrow &   & {\color{red}\downarrow}\\
P & \subset & Q
\end{array}
\end{equation}
where the red font are asserted to exist.

\begin{proposition}[Going down]\label{prop:fall-lec04:going-down} % Gathmann, Commutative Algebra, 9.27
Let $R\subset R'$ be an integral ring extension.
Assume $R$ is normal and $R'$ is an integral domain. Let $P\subset Q$
be prime ideals of $R$, let $Q'\ideal R'$ be a prime ideal with
$Q'\cap R=Q$.
Then there exists a prime ideal $P'\ideal R'$ with $P'\subset Q'$
and $P'\cap R=P$.
\begin{equation}
\begin{array}{lccc}
R'\qquad & {\color{red} P'}  & {\color{red}\subset} & Q'\\
         & {\color{red}\downarrow} &                & \downarrow\\
R\qquad  & P                & \subset & Q
\end{array}
\end{equation}
\end{proposition}

\begin{theorem}[Noether Normalization]\label{thm:fall-lec04:noether-normalization}\index{Noether normalization}% Gathmann, Comm. Alg., 10.5
Let $R$ be a finitely generated algebra over $\kk$ with generators
$x_{1}$, \dots, $x_{n}$. Then there is an injective $\kk$-algebra morphism
$\kk[z_{1},\dots,z_{n}]\into R$ that is a finite extension. Moreover,
if $\kk$ is infinite, then the images of $z_{i}$ may be chosen to be
$\kk$-linear combinations of $x_{1}$, \dots, $x_{n}$.
\end{theorem}

We need a lemma before we can prove Noether normalization.

\begin{lemma}
Let $f\in\kk[x_{1},\dots,x_{n}]$ be a nonzero polynomial and $\kk$ is
infinite, then there exists a $\lambda\in\kk$ and $a_{1}$, \dots,
$a_{n-1}\in\kk$ such that
\begin{equation*}
\lambda f(y_{1}+a_{1}y_{n}, y_{2}+a_{2}y_{n},\dots, y_{n-1}+a_{n-1}y_{n})\in\kk[y_{1},\dots,y_{n}]
\end{equation*}
is monic in $y_{n}$ (i.e., as an element of $R[y_{n}]$ with $R=\kk[y_{1},\dots,y_{n-1}]$).
\end{lemma}

\begin{proof}[Proof (of Noether Normalization)]
Proof by induction on $n$.

\textsc{Base case} ($n=0$): trivial, since $r=0$ works.

\textsc{Inductive case} ($n>0$): there is a ``per cases'' argument
here.

\textsc{Subcase 1:} There is no algebraic relation among the $x_{1}$,
\dots, $x_{n}\in R$ (i.e., there is no non-zero polynomial $f$ over
$\kk$ such that $f(x_{1},\dots,x_{n})=0$ in $R$). Then we can choose
$r=n$ and the map $\kk[z_{1},\dots,z_{n}]\to R$ sending $z_{i}\mapsto x_{i}$
for all $i$, which is even an isomorphism in this situation. The
result follows for this subcase.

\textsc{Subcase 2:} There is a non-zero polynomial $f$ over $\kk$ such
that $f(x_{1},\dots,x_{n})=0$ in $R$. Then we choose $\lambda$ and
$a_{1}$, \dots, $a_{n-1}$ as we did in our Lemma and set
\begin{equation}
y_{1}:=x_{1}-a_{1}x_{n},\quad\dots\quad,y_{n-1}:=x_{n-1}-a_{n-1}x_{n},\quad y_{n}:=x_{n}.
\end{equation}
Then by our Lemma, $\lambda f(y_{1}+a_{1}y_{n},\dots,y_{n-1}+a_{n-1}y_{n},y_{n})$
is monic in $y_{n}$ so $\lambda f(x_{1},\dots,x_{n})=0$.

(``Projection $\AA^{n}\to\AA^{n-1}$'') Then $R=\kk[y_{1},\dots,y_{n}]$
is finite over $\kk[y_{1},\dots,y_{n-1}]$.

The inductive hypothesis: $\kk[y_{1},\dots,y_{n-1}]$ is finite over
$\kk[z_{1},\dots,z_{r}]$ so $R$ is finite over $\kk[z_{1},\dots,z_{r}]$.
\end{proof}