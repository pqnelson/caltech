%%
%% fall-lecture23.tex
%% 
%% Made by Alex Nelson <pqnelson@gmail.com>
%% Login   <alex@lisp>
%% 
%% Started on  2025-11-20T12:20:05-0800
%% Last update 2025-11-20T12:20:05-0800
%% 

\lecture[Pushforward, Pullback]{}

\begin{node}[Pushforward]
Let $f\colon X\to Y$ be a morphism of ringed spaces.
When $\sheaf{F}$ is an $\StructureSheaf_{X}$-module, then
$\pushforward{f}\sheaf{F}$ is an $\pushforward{f}\StructureSheaf_{X}$-module.
Hartshorne calls this the \define{Direct Image} of $\sheaf{F}$ under
$f$ (see~\cite[II~\S5, p109]{hartshorne1977algebraic}).
\end{node}

\begin{definition}[Pullback]
Similarly, we have a ring morphism
\begin{equation}
\pullback{f}\colon\StructureSheaf_{Y}\to\pushforward{f}\StructureSheaf_{X},
\end{equation}
which means $\pushforward{f}\sheaf{F}$ is also a $\StructureSheaf_{Y}$-module.

Let $\sheaf{G}$ be a $\StructureSheaf_{Y}$-module. Then
$f^{-1}\sheaf{G}$ is an $f^{-1}\StructureSheaf_{Y}$-module
(with $f^{-1}h\cdot f^{-1}s=f^{-1}(hs)$). We see
$\StructureSheaf_{Y}\to\pushforward{f}\StructureSheaf_{X}$ corresponds
to $f^{-1}\StructureSheaf_{Y}\to\StructureSheaf_{X}$.
The pullback gives us a sheaf on $X$
\begin{equation}
\pullback{f}\sheaf{G}:=f^{-1}\sheaf{G}\otimes_{f^{-1}\StructureSheaf_{Y}}\StructureSheaf_{X}
\end{equation}
which is called the \define{Pullback of $\sheaf{G}$ to $X$ by $f$}.

Hartshorne calls this the \define{Inverse Image} of $\sheaf{G}$ under
$f$ (see~\cite[II~\S5, p110]{hartshorne1977algebraic}).
\end{definition}

\begin{caution}
Notation is all over the place. The nLab uses extraordinarily
confusing notation, in particular when discussing the inverse image of
a sheaf under a continuous map of ringed spaces. Just be careful when
reading sources.

We will be using the intuition that $f^{-1}$ uses superscripts for
preimages and inverse functions, and the notation $\pullback{f}$
appeals to this intuition (of using superscripts to remind us its
``inverse'' or ``preimage''-like). 
\end{caution}

\begin{example}
The pullback of structure sheaves are structure sheaves,
$$\pullback{f}\StructureSheaf_{Y}=\StructureSheaf_{X}.$$
\end{example}

\begin{example}[Stalks]
The stalks of pullbacks:
$$(\pullback{f}\sheaf{G})_{p}=(f^{-1}\sheaf{G})_{p}\otimes_{(f^{-1}\StructureSheaf_{Y})_{p}}\StructureSheaf_{X,p}=\sheaf{G}_{f(p)}\otimes_{\StructureSheaf_{Y,f(p)}}\StructureSheaf_{X,p}$$
\end{example}

\begin{example}
Let $U\subset Y$ be open, let $i\colon U\into Y$ be the inclusion map.
Then the pullback of $\sheaf{G}$ by the inclusion $\pullback{i}\sheaf{G}=\sheaf{G}|_{U}\otimes_{\StructureSheaf_{Y}|_{U}}\StructureSheaf_{U}=\sheaf{G}|_{U}$
is just the restriction of $\sheaf{G}$ to $U$.
\end{example}

\begin{node}[Adjoint property]
Let $f\colon X\to Y$ be a morphism, let $\sheaf{G}$ be an
$\StructureSheaf_{Y}$-module. We have
$f^{-1}\StructureSheaf_{Y}$-morphism
$f^{-1}\sheaf{G}\to\pullback{f}\sheaf{G}$ sending $s\mapsto s\otimes1$.
This gives us a $\StructureSheaf_{Y}$-morphism $\alpha\colon\sheaf{G}\to\pushforward{f}\pullback{f}\sheaf{G}$.
\end{node}

\begin{lemma}
Let $\sheaf{F}$ be an $\StructureSheaf_{X}$-module, let $\sheaf{G}$ be
an $\StructureSheaf_{Y}$-module. Then
\begin{equation}
\hom_{\StructureSheaf_{X}}(\pullback{f}\sheaf{G},\sheaf{F})\iso\hom_{\StructureSheaf_{Y}}(\sheaf{G},\pushforward{f}\sheaf{F}).
\end{equation}
\end{lemma}

\begin{proof}
Given a $\psi\colon\pullback{f}\sheaf{G}\to\sheaf{F}$, we may obtain
\begin{equation}
\sheaf{G}\xrightarrow{\alpha}\pushforward{f}\pullback{f}\sheaf{G}\xrightarrow{\pushforward{f}\psi}\pushforward{f}\sheaf{F},
\end{equation}
which compose to
\begin{equation}
(\pushforward{f}\psi)\circ\alpha\in\hom_{\StructureSheaf_{Y}}(\sheaf{G},\pushforward{f}\sheaf{F}).
\end{equation}
This is one direction.

Given $\varphi\colon\sheaf{G}\to\pushforward{f}\sheaf{F}$, we obtain
$\widetilde{\varphi}\colon f^{-1}\sheaf{G}\to\sheaf{F}$ an $f^{-1}\StructureSheaf_{Y}$-morphism;
take
\begin{equation}
\psi\colon\pullback{f}\sheaf{G}=f^{-1}\sheaf{G}\otimes_{f^{-1}\StructureSheaf_{Y}}\StructureSheaf_{X}\to\sheaf{F}
\end{equation}
defined by $\psi(s\otimes h)=h\circ\widetilde{\varphi}(s)$. It's not
hard to see these form an isomorphism.
\end{proof}

\begin{node}[Functoriality]
Let $f\colon X\to Y$ and $g\colon Y\to Z$ be continuous maps of ringed
spaces. Let $\sheaf{F}$ be a sheaf on $X$, then
\begin{equation}
\pushforward{(g\circ f)}\sheaf{F}=\pushforward{g}\pushforward{f}\sheaf{F}.
\end{equation}
\end{node}

\begin{proposition}
\begin{enumerate}
\item If $\sheaf{G}$ is a sheaf on $Z$, then $(g\circ f)^{-1}\sheaf{G}=f^{-1}g^{-1}\sheaf{G}$
\item If $\sheaf{G}$ is an $\StructureSheaf_{Z}$-module, then
  $\pullback{(g\circ f)}\sheaf{G}=\pullback{f}\pullback{g}\sheaf{G}$
\end{enumerate}
\end{proposition}

\begin{proof}
(2) Consider $\id\colon\pullback{(gf)}\sheaf{G}\to\pullback{(gf)}\sheaf{G}$,
which gives us a morphism $\sheaf{G}\to\pushforward{(gf)}\pullback{(gf)}\sheaf{G}=\pushforward{g}\pushforward{f}\pullback{(gf)}\sheaf{G}$
by the adjoint property, then we have a morphism
$\pullback{g}\sheaf{G}\to\pushforward{f}\pullback{(gf)}\sheaf{G}$
(again by adjoint property) and this gives us
$\pullback{f}\pullback{g}\sheaf{G}\to\pullback{(gf)}\sheaf{G}$ by the
adjoint property. 

We check this works by examining it on stalks
\begin{subequations}
  \begin{align}
\pullback{f}(\pullback{g}\sheaf{G})_{p}
&=(\pullback{g}\sheaf{G})_{f(p)}\otimes_{\StructureSheaf_{Y,f(p)}}\StructureSheaf_{X,p}\\
&=(\sheaf{G}_{gf(p)}\otimes_{\StructureSheaf_{Z,gf(p)}}\StructureSheaf_{Y,f(p)})\otimes_{\StructureSheaf_{Y,f(p)}}\StructureSheaf_{X,p}\\
&=\sheaf{G}_{gf(p)}\otimes_{\StructureSheaf_{Z,gf(p)}}\StructureSheaf_{X,p}\\
&=(\pullback{(gf)}\sheaf{G})_{p}.
  \end{align}
\end{subequations}
Hence the result.
\end{proof}

\begin{corollary}
If $\sheaf{G}$ is a locally-free $\StructureSheaf_{Y}$-module,
then $\pullback{f}\sheaf{G}$ is a locally-free $\StructureSheaf_{X}$-module.
\end{corollary}

\begin{lemma}
Let $f\colon X\to Y$ be a morphism, suppose we have a short exact
sequence of $\StructureSheaf_{Y}$-modules:
\begin{equation}
0\to\sheaf{G}'\to\sheaf{G}\to\sheaf{G}''\to0.
\end{equation}
Then $\pullback{f}\sheaf{G}'\to\pullback{f}\sheaf{G}\to\pullback{f}\sheaf{G}''\to0$
is right exact on $X$.

If further $\sheaf{G}''$ is locally-free, then we have
$0\to\pullback{f}\sheaf{G}'\to\pullback{f}\sheaf{G}\to\pullback{f}\sheaf{G}''\to0$
be a short exact sequence.
\end{lemma}

\begin{proof}
(1) Check on stalks. Recall pullback on stalks is tensor product
$(\pullback{f}\sheaf{G})_{p}=\sheaf{G}_{f(p)}\otimes_{\StructureSheaf_{Y,f(p)}}\StructureSheaf_{X,p}$.
Then the short exact sequence
\begin{equation}
\sheaf{G}'_{f(p)}\otimes_{\StructureSheaf_{Y,f(p)}}\StructureSheaf_{X,p} \to  \sheaf{G}_{f(p)}\otimes_{\StructureSheaf_{Y,f(p)}}\StructureSheaf_{X,p}  \to  \sheaf{G}''_{f(p)}\otimes_{\StructureSheaf_{Y,f(p)}}\StructureSheaf_{X,p} \to0
\end{equation}
Since $\otimes_{\StructureSheaf_{Y,f(p)}}\StructureSheaf_{X,p}$ is
right exact, we have the first claim.

(2) If $\sheaf{G}''$ is locally-free, then $\sheaf{G}''_{f(p)}$ is a
free $\StructureSheaf_{Y,f(p)}$-module, hence the sequence is split exact.
\end{proof}

\begin{definition}
The $\StructureSheaf_{X}$-module $\sheaf{F}$ is \define{generated by
  finitely many global sections} if there exists a short exact
sequence $\StructureSheaf_{X}^{\oplus m}\to\sheaf{F}\to0$, which is
equivalent to:
There exists global sections $s_{1}$, \dots, $s_{m}\in\Gamma(X,\sheaf{F})$
such that for all $p\in X$, the stalk $\sheaf{F}_{p}$ is generated by
$(s_{1})_{p}$, \dots, $(s_{m})_{p}$ as $\StructureSheaf_{X}$-modules.
\end{definition}

\begin{example}
Any coherent sheaf on an affine variety is generated by finitely many
global sections.
\end{example}

\begin{example}
$\mathcal{O}_{\PP^{n}}(1)$ is globally generated by $x_{0}$, \dots, $x_{n}\in\Gamma(\PP^{n},\mathcal{O}(1))$.
\end{example}

\begin{example}
Suppose $f\colon X\to\PP^{n}$ is a morphism.
If $\StructureSheaf_{\PP^{n}}^{\oplus m}\to\mathcal{O}(1)\to0$ is a
short exact sequence,
then
$\StructureSheaf_{X}^{\oplus(n+1)}\to\pullback{f}\mathcal{O}(1)\to0$
is exact. Hence $\pullback{f}\mathcal{O}(1)$ generated by
$\pullback{f}(x_{0})$, \dots, $\pullback{f}(x_{n})\in\Gamma(X,\pullback{f}\mathcal{O}(1))$.
\end{example}

\begin{proposition}
Let $X$ be a variety, $\mathcal{L}$ an invertible sheaf generated by
$s_{0}$, \dots, $s_{n}\in\Gamma(X,\mathcal{L})$. Then there exists a
unique $f\colon X\to\PP^{n}$ such that
$\mathcal{L}\iso\pullback{f}\mathcal{O}(1)$ and $s_{i}=\pullback{f}(x_{i})$
for all $i$.
\end{proposition}

