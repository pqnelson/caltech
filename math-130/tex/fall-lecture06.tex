%%
%% fall-lecture06.tex
%% 
%% Made by Alex Nelson <pqnelson@gmail.com>
%% Login   <alex@lisp>
%% 
%% Started on  2025-10-11T11:55:08-0700
%% Last update 2025-10-11T11:55:08-0700
%% 

\lecture{}

\begin{definition}[Distinguished open sets]% Gathmann, 3.6
Let $X$ be an affine variety, let $f\in A(X)$. We define
the \define{Distinguished Open Set} of $f$ to be the subset
$D(f)=X\setminus V(f)$. We see it is, in fact, an open subset of $X$
in the Zariski topology (since $V(f)$ is a closed subset in that topology).

Some references call these $D(f)$ ``Basic open subsets'' or
``Principal open subsets''.
\end{definition}

\begin{remark}% Gathmann 3.7
The distinguished open subsets behave as we expect under intersections
and unions.
\begin{enumerate}
\item $D(f)\cap D(g)=D(fg)$ for any $f\in A(X)$ and $g\in A(X)$
\item $D(f_{1})\cup\cdots\cup D(f_{n})=X\setminus\bigl(V(f_{1})\cap\cdots\cap V(f_{n})\bigr)=X\setminus V(f_{1},\dots,f_{n})$.
\end{enumerate}
Thus they form a basis for the Zariski topology.
\end{remark}

\begin{proposition}[Regular functions on distinguished open subsets]% Gathmann 3.8
We find $\RegularFuns_{X}(D(f))\iso A(X)_{f}$. In particular, if
$f=1$, then $\RegularFuns_{X}(X)=A(X)$.
\end{proposition}

\begin{proof}
Let $D(h)$ be a principal open set. Take
$f\in\RegularFuns_{V}(D(h))$. We know we can cover $D(h)=\bigcup V_{i}$
such that $\exists g_{i},h_{i}\in A(V)$ such that $h_{i}$ is nowhere
zero on $V_{i}$ and 
\begin{equation}
\left.f\right|_{V_{i}} = \frac{g_{i}}{h_{i}}.
\end{equation}
We may further assume $V_{i}$ are principal open (since they form a basis).
Then $V_{i}=D(a_{i})\subset D(h_{i})$.
We will see $D(a_{i})\supset D(h_{i})$.
So $a_{i}\in I(V(a_{i}))\subset I(V(h_{i}))$.

Then the Nullstellensatz implies $a_{i}\in\Radical{\langle h_{i}\rangle}$,
so $a_{i}^{N} = h_{i}g'_{i}$ for some $N\in\NN$.
\end{proof}

\begin{remark}
When the field $\kk$ is not algebraically closed, the previous
proposition is false. For example
\begin{equation}
f = \frac{1}{x^{2}}
\end{equation}
on $\RR$ is defined, but $f\notin\RR[x]$.
\end{remark}

\begin{example}
The affine line with coordinates $t$. Then we see
$\frac{1}{t}\in\RegularFuns_{X}(D(t))$ is not extendible to all $\AA^{1}$.
\end{example}

\begin{example}
The cusp $y^{2}=x^{3}$, we can look at the function
\begin{equation}
\varphi = \frac{y}{x}
\end{equation}
which is regular away from the origin, but it is not extendible to the origin.
\end{example}

\begin{example}
From Example~\ref{ex:fall-lecture05:non-extendible-fn} last time, we had a function which is not a global fraction. Then
$U$ is not a distinguished open set.
\end{example}

\begin{example}% 3.11
On $\AA^{2}\setminus 0$, the affine plane without its origin, we find
\begin{equation}
\RegularFuns_{\AA^{2}}(\AA^{2}\setminus0)=\kk[x,y].
\end{equation}
Let $U:=\AA^{2}\setminus0$. Then
\begin{equation}
D(x)\cup D(y)=U.
\end{equation}
Let $\varphi\in\RegularFuns_{X}(U)$.
Then there exists $f,g\in\kk[x,y]$ and $m,n\in\NN$ such that
\begin{equation}
\varphi=\frac{f}{x^{m}}\qquad\mbox{on }D(x),
\end{equation}
and
\begin{equation}
\varphi=\frac{g}{y^{n}}\qquad\mbox{on }D(y).
\end{equation}
Without loss of generality, $x$ does not divide $f$, and $y$ does not
divide $g$.

On $D(x)\cap D(y)$ we have $fy^{n}=gx^{m}$, hence
\begin{equation}
\overline{D(x)\cap D(y)}=\AA^{2}.
\end{equation}
This holds, so the equality holds everywhere, so $fy^{n}gx^{m}$ in $\kk[x,y]$.
\end{example}

\begin{definition}
Let $X$ be a topological space.
A \define{Presheaf} $\mathcal{F}$ (of rings) on $X$ consists of
\begin{enumerate}
\item for each open subset $U\subset X$, a ring $\mathcal{F}(U)$
\item for each inclusion $U\subset V$ of open sets in $X$, a ring
  morphism $\rho_{V,U}\colon\mathcal{F}(V)\to\mathcal{F}(U)$ called
  the \define{Restriction Map}
\end{enumerate}
such that
\begin{enumerate}
\item $\mathcal{F}(\emptyset)=0$;
\item for all open subsets $U\subset X$, we have $\rho_{U,U}$ is the
  identity map on $\mathcal{F}(U)$;
\item for any inclusions $U\subset V\subset W$ of open subsets of $X$
  we have $\rho_{V,U}\circ\rho_{W,V}=\rho_{W,U}$.
\end{enumerate}
We call the elements of $\mathcal{F}(U)$ \define{Sections} of
$\mathcal{F}$ over $U$. We often write the restriction
$\rho_{V,U}(\varphi)$ as $\varphi|_{V}$.
\end{definition}

\begin{remark}
Observe that a presheaf is a [contravariant] functor.
When $X$ is a topological space, we form a category $\mathrm{Op}(X)$
of its open sets and morphisms are inclusions. A presheaf is then a
function $\mathcal{F}\colon\mathrm{Op}(X)^{\text{op}}\to\mathbf{C}$ to
some category $\mathbf{C}$ (usually the category of rings or Abelian
groups or $\kk$-algebras or\dots).
\end{remark}

\begin{definition}
A presheaf $\mathcal{F}$ of rings on $X$ is called a \define{Sheaf}
(of rings) if it satisfies:
\begin{enumerate}
\item \textsc{Gluing Property:} If $U\subset X$ is open, $\{U_{i}\mid i\in I\}$
is an arbitrary open cover of $U$, and $\varphi_{i}\in\mathcal{F}(U_{i})$
are sections for all $i$ such that
$\varphi_{i}|_{U_{i}\cap U_{j}}=\varphi_{j}|_{U_{i}\cap U_{j}}$ for
all $i,j\in I$, then there exists a unique $\varphi\in\mathcal{F}(U)$
such that $\varphi|_{U_{i}}=\varphi_{i}$ for all $i\in I$.
\end{enumerate}
\end{definition}

\begin{example}
If $X$ is an affine variety, then $\RegularFuns_{X}$ is a sheaf of
$\kk$-algebras on $X$.
\end{example}

\begin{example}
Let $X=\RR^{n}$ with the standard topology. Then
\begin{equation}
\mathcal{F}(U)=\{f\colon U\to\RR\mbox{ continuous}\}
\end{equation}
are the local sections of $\mathcal{F}$. This is a sheaf.
\end{example}

\begin{example}
Let $X=\RR^{n}$. Suppose for any $U\subset X$ open we define
\begin{equation}
\mathcal{F}(U)=\{f\colon U\to\RR\mbox{ constant}\}
\end{equation}
is a presheaf but it is not a sheaf. It fails to satisfy the gluing property.
\end{example}

\begin{definition}[Restriction of presheaves]% Gathmann 3.16
Let $\mathcal{F}$ be a presheaf on $X$. Let $U\subset X$ be open.
Then we may define the \define{Restriction} of $\mathcal{F}$ to $U$ to
be the presheaf $\mathcal{F}|_{U}$ given by $\mathcal{F}|_{U}(V)=\mathcal{F}(V)$
for any $V\subset U$ open. The restriction map is induced from $\mathcal{F}$
\end{definition}

\begin{definition}[Germs of presheaves]
Let $X$ be a topological space. Let $\mathcal{F}$ be a presheaf on
$X$. Let $a\in X$ be a point. We define the \define{Stalk} of
$\mathcal{F}$ at $a$ to be defined as
\begin{equation}
\mathcal{F}_{a}:=\{(U,\varphi)\mid U\subset X\mbox{ open with }a\in U,\varphi\in\mathcal{F}(U)\}/\sim
\end{equation}
where $(U,\varphi)\sim(U',\varphi')$ iff there exists an open subset
$V$ with $a\in V\subset U\cap U'$ and
$\varphi|_{V}=\varphi'|_{V}$. This is an equivalence relation, and
$\mathcal{F}_{a}$ inherits the structure of a ring (or Abelian group
or $\kk$-algebra or\dots).
\end{definition}

\begin{lemma}
The stalk $\RegularFuns_{X,a}\iso A(X)_{I(a)}$ as $\kk$-algebras.
\end{lemma}

\begin{definition}% 3.17
The stalk $\RegularFuns_{X,a}$ is a local ring, and its maximal ideal
corresponds to the germs which vanish at
\begin{equation}
I_{a}:=\{\overline{(U,\varphi)}\in\RegularFuns_{X,a}\mid\varphi(a)=0\}\iso\{\frac{g}{f}\mid
g,f\in A(x), g(a)=0, f(a)\neq0\}.
\end{equation}
This $A(X)_{I(a)}$ is called the \define{Local Ring of $X$ at $a$}.
\end{definition}