%%
%% fall-lecture13.tex
%% 
%% Made by Alex Nelson <pqnelson@gmail.com>
%% Login   <alex@lisp>
%% 
%% Started on  2025-10-28T10:53:36-0700
%% Last update 2025-10-28T10:53:36-0700
%% 

\lecture{}

\begin{node}
Recall the blow-up $\widetilde{X}$ of an affine variety $X$ at
$f_{1}$, \dots, $f_{r}\in\CoordRing{\kk}{X}$ satisfies
\begin{equation}
\widetilde{X}\subset\{(x,y)\in X\times\PP^{r-1}\mid y_{i}f_{j}(x)=y_{j}f_{i}(x)\forall i,j\},
\end{equation}
with equality at coordinate functions.
\end{node}

\begin{example}
Blow-up $X=\AA^{n}$, then
\begin{equation}
Y:=\widetilde{A}^{n}=\{(x,y)\in\AA^{n}\times\PP^{n-1}\mid x_{i}y_{j}=x_{j}y_{i}\forall i,j\},
\end{equation}
at $x_{1}$, \dots, $x_{n}$. We claim $\widetilde{A}^{n}\subset Y$.
Note that $Y$ can be covered by $\AA^{n}$ and all charts meet at
$((1,\dots,1),(1:\dots:1))$.
This implies $Y$ is irreducible and of dimension $n$.
Then $\widetilde{\AA}^{n}=Y$, since $\widetilde{\AA}^{n}$ cannot be a
proper subset of $Y$.

Also observe $\pi^{-1}(0)=\{(0,y)\in\AA^{n}\times\PP^{n-1}\}\iso\PP^{n-1}$.
\end{example}

\begin{lemma}% Gathmann, 9.16
The blow-up $\widetilde{X}$ of an affine variety $X$ at $f_{1}$, \dots,
$f_{r}\in\CoordRing{\kk}{X}$ depends only the ideal $\langle f_{1},\dots,f_{r}\rangle\ideal\CoordRing{\kk}{X}$.

More precisely, if $f'_{1}$, \dots, $f'_{s}\in\CoordRing{\kk}{X}$
generate the same ideal
$\langle f_{1},\dots,f_{r}\rangle=\langle f'_{1},\dots,f'_{s}\rangle\ideal\CoordRing{\kk}{X}$,
and $\pi\colon\widetilde{X}\to X$ and $\pi'\colon\widetilde{X}'\to X$
are the corresponding blow-ups, there is an isomorphism
$\colon\widetilde{X}\to\widetilde{X}'$ such that $\pi'\circ F=\pi$.
\end{lemma}

\begin{proof}
We can write
\begin{equation}
f_{i}=\sum^{s}_{j=1}g_{ij}f'_{j},\quad\mbox{and}\quad f'_{j}=\sum^{r}_{k=1}h_{jk}f_{k}
\end{equation}
for some suitable $g_{ij}$, $h_{jk}\in\CoordRing{\kk}{X}$. We claim
\begin{equation}
F(x,y) = (x,y') = (x, (\sum^{r}_{k=1}h_{1k}(x)y_{k}:\dots:\sum^{r}_{k=1}h_{s,k}(x)y_{k}))
\end{equation}
is the desired isomorphism.

\textsc{Claim 1:} The $y'$ components cannot all be zero simultaneously.
By construction we have $(y_{1}:\dots:y_{r})=(f_{1}:\dots:f_{r})$ on
$U=X\setminus V(f_{1},\dots,f_{r})\subset\widetilde{X}\subset X\times\PP^{r-1}$.
That is to say, these two vectors are linearly dependent and non-zero
at each point in this set. If for all $j$,
\begin{equation}
\sum_{k}h_{j,k}y_{k}=0,
\end{equation}
then for all $i$,
\begin{equation}
\underbrace{\sum_{j}\sum_{k}g_{i,j}h_{j,k}y_{k}}_{=y_{i}}=0.
\end{equation}
Hence $y_{i}=0$ for all $i$ on $U$ and hence on its closure $\closure{U}=\widetilde{X}$. 
This means if $y'_{j}=\sum_{k}h_{j,k}y_{k}=0$ for all $j$, then
$y_{i}=0$ for all $i$, which is a contradiction.

\textsc{Claim 2:} The image of $F$ lies in $\widetilde{X}'$.

\textsc{Claim 3:} $F$ is an isomorphism.

\textsc{Claim 4:} Obviously $\pi'\circ F=\pi$.
\end{proof}

\begin{construction}% Gathmann, 9.17
If we blow-up an arbitrary variety (not necessarily an affine variety)
$X$ at $Y\subset X$ where $Y$ is a closed subvariety, we could pick an
affine open cover $\{U_{i}\}_{i\in I}$. We blow-up $U_{i}$ at
$U_{i}\cap Y$ to obtain $\widetilde{U}_{i}$. Then we glue the
$\widetilde{U}_{i}$ together to form $\widetilde{X}$. We call this the
\define{Blow-Up} of $X$ at $Y$.
\end{construction}

\begin{remark}% Gathmann, 9.17c
We can blow-up along an ``ideal sheaf'' (which assigns ideals to open sets).
This will turn out to be critical when we get to schemes next quarter.
\end{remark}

\begin{example}
If $Y=\{a\}$ is some point, we can pick an open affine neighborhood
$U\subset X$ containing $a\in U$, then glue $\widetilde{U}$ with
$X\setminus\{a\}$.
\end{example}

\begin{example}
If $X$ is a projective variety at $f_{1}$, \dots, $f_{r}\in\HomogeneousCoordRing{\kk}{X}$
are homogeneous and of the same degree\footnote{If they are not of the
same degree, we can transform them into homogeneous polynomials of the
same degree.}. We can define the blow-up in the same way, taking the
closure of the graph
\begin{equation}
\begin{split}
\Gamma &=\{(x,(f_{1}(x):\dots:f_{r}(x)))\mid x\in X\setminus V_{p}(f_{1},\dots,f_{r})\}\\
&\subset U\times\PP^{r-1}.
\end{split}
\end{equation}
We obtain $\widetilde{X}=\closure{\Gamma}\subset X\times\PP^{r-1}$ is
a projective variety.
\end{example}

\begin{remark}
Intuitively, if we blow-up at a point, then the exceptional set
parametrizes the tangent directions at that point.
\end{remark}

\begin{definition}% Gathmann, 9.20
Let $a\in X$, $\pi\colon\widetilde{X}\to X$ where $\widetilde{X}$ is
the blow-up of $X$ at $a$, its exceptional set $\pi^{-1}(\{a\})$ is a
projective variety. The cone over this exceptional set is called the
\define{Tangent Cone} of $X$ at $a$, denoted
$\TangentCone{a}{X}$. It's defined as $\TangentCone{a}{X}:=\pi^{-1}(\{a\})\subset\{a\}\times\PP^{n-1}$
where $n=\dim(X)$.
\end{definition}

\begin{remark}
Why is this called a ``cone''? Well, in a vector space $V$, a cone $C$
contains the origin $0\in C$ and $\lambda x$ for all $x\in C$ and
$\lambda\in\FF$. This can be generalized to the settings of affine
varieties $X$ instead of vector spaces, with exactly the same
definition, and also to projective varieties.
\end{remark}

\begin{proposition}[Dimension of the exceptional set]% 9.23 of Gathmann
Let $X$ be an irreducible variety. Then every irreducible component of
the exceptional set in $\widetilde{X}$ has codimension 1 in $\widetilde{X}$.
\end{proposition}

\begin{proof}
Since dimension is a local property, we take a nonempty open affine
$U_{i}\subset\widetilde{X}$ where $y_{i}\neq0$. Then notice what are
the conditions on the exceptional set. Supposing $\widetilde{X}$ is a
blow-up at $f_{1}$, \dots, $f_{r}$. If $a\in U_{i}$, then $f_{i}(a)=0$
which implies $f_{j}(a)=0$ for all $j=1,\dots,r$ because $y_{i}f_{j}=y_{j}f_{i}$.
So we only need one equation on $U_{i}$.
If $\widetilde{X}\neq\emptyset$, then $f_{i}\neq0$ on $U_{i}$, and
moreover this implies the codimension of the exceptional set in
$\widetilde{X}$ must be 1.
\end{proof}

\begin{corollary}[Dimension of tangent cones]% Gathmann, 9.24
Let $a$ be a point in a variety $X$. Then the dimension of the tangent
cone of $X$ at $a$ is the codimension of $\{a\}$ in $X$:
\begin{equation}
\dim\TangentCone{a}{X}=\codim_{X}(\{a\}).
\end{equation}
\end{corollary}

\begin{remark}
We can use blow-up to extend morphisms. If $X$ is some affine variety,
and $f_{1}$, \dots, $f_{r}\in\CoordRing{\kk}{X}$, if we can define
$X\dashrightarrow \PP^{r-1}$ by $x\mapsto(f_{1}(x):\dots:f_{r}(x))$.
If we consider the blow-up $\widetilde{X}\subset X\times\PP^{r-1}\to\PP^{r-1}$.
\end{remark}

\begin{lemma}% 9.27 of Gathmann
The $\PP^{1}\times\PP^{1}$ blow-up at one point is isomorphic to the
blow-up of $\PP^{2}$ at two points.
\end{lemma}

\begin{proof}
We know $\PP^{1}\times\PP^{1}$ is isomorphic to the quadric surface
\begin{equation}
X = \{(x_{0}:x_{1}:x_{2}:x_{3})\mid x_{0}x_{3}=x_{1}x_{2}\}\subset\PP^{3}.
\end{equation}
This is just the Segre embedding (\S\S\ref{ex:fall-lec09:segre-embedding} \emph{et seq.}).
We have a rational map $X\RationalTo\PP^{2}$ projection from $(0:0:0:1)=a$
sending
\begin{equation}
(x_{0}:x_{1}:x_{2}:x_{3})\mapsto(x_{0}:x_{1}:x_{2}).
\end{equation}
We want to extend this morphism to all of $\PP^{1}\times\PP^{1}$.

So $\widetilde{X}=\Bl_{a}X$ we have replaced $a$ by all the tangent
directions to $a$, so this gives us a morphism
$\widetilde{X}\to\PP^{2}$. We want an inverse morphism. Let
$y\in\PP^{2}$. Consider the line connecting $\overline{ay}$.
\end{proof}