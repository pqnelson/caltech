%%
%% fall-lecture23.tex
%% 
%% Made by Alex Nelson <pqnelson@gmail.com>
%% Login   <alex@lisp>
%% 
%% Started on  2025-11-29T09:29:55-0800
%% Last update 2025-11-29T09:29:55-0800
%% 

\lecture[Riemann Integration]

\subsection*{Review of One-Dimensional Case}

\begin{definition}
Let $[a,b]$ be a closed interval of the real line. We define a
\define{Partition} of $[a,b]$, denoted $P$, to be a finite sequence of numbers $a=c_{0}<c_{1}<\dots<c_{n}=b$.
We call each interval $I_{i}=[c_{i},c_{i+1}]$ a \define{Sub-interval} of the partition
(and they cover $[a,b]$ in the sense that $[a,b]=\bigcup_{i}[c_{i},c_{i+1}]$).

Also, we use the notation $|I_{i}|=c_{i+1}-c_{i}\geq0$ for the
``width'' of the subinterval $I_{i}=[c_{i},c_{i+1}]$.
\end{definition}

\begin{definition}
Let $f\colon[a,b]\to\RR$ be a [piecewise continuous]\footnote{A
function is ``piecewise continuous'' if it is continuous with at most
countably many discontinuities.} function. Let $P$ be a partition of $[a,b]$.
We define the \define{Lower Sum} of $f$ on this partition $P$ to be
\begin{equation}
L(f,P):=\sum^{n-1}_{i=0}|I_{i}|\inf_{x\in I_{i}}f(x).
\end{equation}
We define the \define{Upper Sum} of $f$ on $P$ to be
\begin{equation}
U(f,P):=\sum^{n-1}_{i=0}|I_{i}|\sup_{x\in I_{i}}f(x).
\end{equation}
\end{definition}

\begin{definition}
Let $f\colon[a,b]\to\RR$ be a [piecewise continuous] function.
We may define the \define{Upper Riemann Integral} to be the number
\begin{equation}
\overline{\int^{b}_{a}}f\,\D x:=\inf_{P}U(f,P)
\end{equation}
where the infinimum is over all partitions $P$ of the interval $[a,b]$.
Similarly, we define the \define{Lower Riemann Integral} to be the number
\begin{equation}
\underline{\int^{b}_{a}}f\,\D x:=\sup_{P}U(f,P)
\end{equation}
where, again, the supremum is over all partitions $P$ of the interval $[a,b]$.
\end{definition}

\begin{remark}
Technically, these define Darboux integration, which is formally the
same as Riemann integration. The difference is that Riemann integrals
pick ``some point'' $t_{i}\in I_{i}$ and work with the sums
\begin{equation}
S(f,P) = \sum^{n-1}_{i=0}f(t_{i})|I_{i}|.
\end{equation}
A partition $P$ with these $t_{i}\in I_{i}$ are called \define{Tagged Partitions}.

Riemann integration relies on the notion of the \define{Mesh} of the partition
which is just $\max\{|I_{i}|\}$. The integral of $f$ is the value $s$
defined as: For each $\varepsilon>0$, there
exists a $\delta>0$ such that for any tagged partition $P$ whose mesh
is less than $\delta$, we have
\begin{equation}
\left|\left(\sum^{n-1}_{i=0}f(t_{i})|I_{i}|\right)-s\right|<\varepsilon.
\end{equation}
That is to say, the sequence of Riemann sums converges to the integral $S(f,P)\to s$.

This is actually more tedious to work with than Darboux's version
(which is logically equivalent to Riemann's version). So we'll just
work with what's convenient, because life is hard enough already.
\end{remark}

\begin{definition}
We call a function $f\colon[a,b]\to\RR$ \define{Riemann Integrable} if
\begin{equation}
\overline{\int^{b}_{a}}f\,\D x=\underline{\int^{b}_{a}}f\,\D x.
\end{equation}
Since we will not consider other forms of integration in this class,
we will just call $f$ ``integrable''.

When $f$ is (Riemann) integrable, we just write $\int^{b}_{a}f\,\D x$ for this
common value which we call the \define{Riemann Integral} of $f$.
\end{definition}

\subsection*{Generalization}

\begin{definition}
The \define{Content} of a (closed) cell $\closure{\Delta}=\{\vec{x}\in\RR^{n}\mid a_{i}\leq x_{i}\leq b_{i}\mbox{ for }i=1,\dots,n\}$
is given by
\begin{equation}
|\closure{\Delta}|=\prod^{n}_{i=1}(b_{i}-a_{i}).
\end{equation}
We can define the content of an open cell to be equal to the content
of its closure, i.e., $|\Delta|:=|\closure{\Delta}|$.
\end{definition}

\begin{definition}
Let $\closure{\Delta}=\{\vec{x}\in\RR^{n}\mid a_{i}\leq x_{i}\leq b_{i}\mbox{ for }i=1,\dots,n\}$
be a closed cell. We can define a \define{Partition} $P$ of $\closure{\Delta}$
to be a sequence of points of $\closure{\Delta}$ which project to
components which partitions each dimension: 
$a_{i}=c_{i,0}<c_{i,1}<\dots<c_{i,n_{i}}=b_{i}$ for each
dimension $i=1,\dots,n$.

It is useful to discuss \define{Subcells} of a partition $P$ which are sets of the form
\begin{equation}
\Delta_{k_{1},\dots,k_{n}}=\{\vec{x}\in\RR^{n}\mid c_{i,k_{i}-1}\leq x_{i}\leq c_{i,k_{i}}, i=1,\dots,n\}
\end{equation}
where $0\leq k_{i}-1<n_{i}$ for each $i=1,\dots,n$.

We can reindex these guys to be $\Delta_{i}$ where $i=1,\dots,N$ with $N=\prod^{n}_{i}n_{i}+1$
so that
\begin{subequations}
  \begin{align}
\closure{\Delta} &=\bigcup_{i}\Delta_{i}\\\intertext{and}
|\closure{\Delta}| &=\sum_{i}|\Delta_{i}|.
  \end{align}
\end{subequations}
\end{definition}

\begin{definition}
Let $\closure{\Delta}=\{\vec{x}\in\RR^{n}\mid a_{i}\leq x_{i}\leq b_{i}\mbox{ for }i=1,\dots,n\}$
be a closed cell.
Let $f\colon\closure{\Delta}\to\RR$ be a bounded function.
Let $P$ be a partition of $\closure{\Delta}$.

We define the \define{Lower Sum} of $f$ with respect to $P$ to be
\begin{equation}
S_{*}(f,P):=\sum_{i}\left(\inf_{\vec{x}\in\closure{\Delta_{i}}}f(\vec{x})\right)|\closure{\Delta}_{i}|
\end{equation}
Similarly, the \define{Upper Sum} of $f$ with respect to $P$ is
\begin{equation}
S^{*}(f,P):=\sum_{i}\left(\sup_{\vec{x}\in\closure{\Delta_{i}}}f(\vec{x})\right)|\closure{\Delta}_{i}|
\end{equation}
Clearly for fixed $f$ and $P$, $S_{*}(f,P)\leq S^{*}(f,P)$.
\end{definition}

\begin{definition}
Let $\closure{\Delta}=\{\vec{x}\in\RR^{n}\mid a_{i}\leq x_{i}\leq b_{i}\mbox{ for }i=1,\dots,n\}$
be a closed cell. Let $P_{1}$ and $P_{2}$ be partitions of $\closure{\Delta}$.
We say $P_{2}$ is a \define{Refinement} of $P_{1}$ if $P_{1}\subset P_{2}$.
\end{definition}

\begin{proposition}
Let $\closure{\Delta}=\{\vec{x}\in\RR^{n}\mid a_{i}\leq x_{i}\leq b_{i}\mbox{ for }i=1,\dots,n\}$
be a closed cell. Let $P_{1}$ and $P_{2}$ be partitions of $\closure{\Delta}$.
Let $f\colon\closure{\Delta}\to\RR$ be a bounded function.
if $P_{2}$ is a refinement of $P_{1}$, then
$S^{*}(f,P_{2})\leq S^{*}(f,P_{1})$ and also $S_{*}(f,P_{2})\geq S_{*}(f,P_{1})$.
\end{proposition}

\begin{proof}
We can write each subcell $\closure{\Delta}_{i}$ of $P_{1}$ as a union $\closure{\Delta}_{i}=\bigcup_{j}\closure{\Delta}_{i,j}$
of subcells $\closure{\Delta}_{i,j}$ of $P_{2}$. Then
\begin{equation}
\sup_{\vec{x}\in\closure{\Delta}_{i,j}}f(\vec{x})\leq\sup_{\vec{x}\in\closure{\Delta}_{i}}f(\vec{x})
\end{equation}
and
\begin{equation}
\sum_{j}|\closure{\Delta}_{i,j}|=|\closure{\Delta}_{i}|,
\end{equation}
which implies
\begin{subequations}
  \begin{align}
S^{*}(f,P_{2}) &=\sum_{i,j}|\closure{\Delta}_{i,j}|\sup_{\vec{x}\in\closure{\Delta}_{i,j}}f(\vec{x})\\
&\leq\sum_{i,j}|\closure{\Delta}_{i,j}|\sup_{\vec{x}\in\closure{\Delta}_{i}}f(\vec{x})\\
&\leq\sum_{i}|\closure{\Delta}_{i}|\sup_{\vec{x}\in\closure{\Delta}_{i}}f(\vec{x})=S^{*}(f,P_{1}).
  \end{align}
\end{subequations}
The argument for the lower sums is analogous.
\end{proof}

\begin{proposition}
Let $\closure{\Delta}=\{\vec{x}\in\RR^{n}\mid a_{i}\leq x_{i}\leq b_{i}\mbox{ for }i=1,\dots,n\}$
be a closed cell.
Let $f\colon\closure{\Delta}\to\RR$ be a bounded function.
For any two arbitrary partitions $P_{1}$ and $P_{2}$ of $\closure{\Delta}$,
we have $S_{*}(f,P_{1})\leq S^{*}(f,P_{2})$.
\end{proposition}

\begin{proof}
We may form a common refinement $P'$ for both $P_{1}$ and $P_{2}$,
then by the previous proposition we have the chain of inequalities
\begin{equation}
S_{*}(f,P_{1})\leq S_{*}(f,P')\leq S^{*}(f,P')\leq S^{*}(f,P_{2}).
\end{equation}
Hence the result.
\end{proof}

\begin{definition}
Let $\closure{\Delta}=\{\vec{x}\in\RR^{n}\mid a_{i}\leq x_{i}\leq b_{i}\mbox{ for }i=1,\dots,n\}$
be a closed cell.
Let $f\colon\closure{\Delta}\to\RR$ be a bounded function.
We may define the \define{Lower Integral} of $f$ on $\closure{\Delta}$
to be
\begin{equation}
\int_{*\closure{\Delta}}f:=\sup_{P}S_{*}(f,P).
\end{equation}
Similarly, the \define{Upper Integral} of $f$ on $\closure{\Delta}$ to be
\begin{equation}
\int^{*}_{\closure{\Delta}}f:=\inf_{P}S^{*}(f,P).
\end{equation}
These infinimum and supremum range over all partitions of $\closure{\Delta}$.
\end{definition}

\begin{proposition}
Let $\closure{\Delta}=\{\vec{x}\in\RR^{n}\mid a_{i}\leq x_{i}\leq b_{i}\mbox{ for }i=1,\dots,n\}$
be a closed cell.
Let $f\colon\closure{\Delta}\to\RR$ be a bounded function. Then
\begin{equation}
\int_{*\closure{\Delta}}f\leq\int^{*}_{\closure{\Delta}}f.
\end{equation}
\end{proposition}

\begin{definition}
We say a subset $S\subset\RR^{n}$ has \define{Content Zero} if for any
$\varepsilon>0$ there are \underline{finitely}-many open cells $\Delta_{i}$ such
that $S\subset\bigcup_{i}\Delta_{i}$ and $\sum_{i}|\Delta_{i}|<\varepsilon$.
\end{definition}

\begin{definition}
We say a subset $S\subset\RR^{n}$ has \define{Measure Zero} if for any
$\varepsilon>0$ there are \underline{countably}-many open cells $\Delta_{i}$ such
that $S\subset\bigcup_{i}\Delta_{i}$ and $\sum_{i}|\Delta_{i}|<\varepsilon$.
\end{definition}

\begin{proposition}
A set of content zero is also a set of measure zero.
\end{proposition}

\begin{proposition}
A compact set of measure zero is also a set of content zero.
\end{proposition}

\begin{proof}[Proof sketch]
Since every open cover has a finite subcover, every countably open
cells covering the set has a finite subcover.
\end{proof}

\begin{lemma}
If countably many subsets $E_{i}\subset\RR^{n}$ each has measure zero,
then $E=\bigcup_{i}E_{i}$ has measure zero.
\end{lemma}

\begin{proof}
Let $\varepsilon>0$ be arbitrary. For each $i$, $E_{i}$ has measure
zero, so there exists countably many $\Delta_{ij}$ (with $j\in\NN$)
such that $E_{i}\subset\bigcup_{j\in\NN}\Delta_{ij}$ and $\sum_{j}|\closure{\Delta}_{ij}|<\varepsilon/2^{i}$.
Then $E=\bigcup_{i}E_{i}\subset\bigcup_{i,j}\Delta_{ij}$ and $\sum_{ij}|\closure{\Delta}_{ij}|<\sum_{i}\varepsilon/2^{i}=\varepsilon$.
\end{proof}

\begin{example}
$\QQ\subset\RR$: We see $\QQ$ does not have content zero because there
  is no finite collection of open intervals which can cover $\QQ$. But
  $\QQ$ has measure zero because of the previous Lemma.
\end{example}

\begin{example}
$\RR\times\{0\}\subset\RR^{2}$: This has measure zero. Take
\begin{equation}
\Delta_{i}=(i-\frac{\varepsilon}{2},i+\frac{\varepsilon}{2})\times(\frac{-\varepsilon}{2},\frac{\varepsilon}{2}).
\end{equation}
Each $\Delta_{i}$ has content zero, and we have countably many $\Delta_{i}$.
Then $\bigcup_{i}\Delta_{i}$ has measure zero by the previous Lemma.
\end{example}

\begin{example}
Both $(0,1)\times\{0\}\subset\RR^{2}$ and $[0,1]\times\{0\}\subset\RR^{2}$
have content zero, but $[0,1]\times\{0\}$ is compact, whereas $(0,1)\times\{0\}$
is noncompact.
\end{example}

\begin{lemma}
If $E\subset\RR^{n}$ has content zero, then $\closure{E}$ has content zero.
But if $EE\subset\RR^{n}$ has measure zero, then $\closure{E}$ may or
may not have measure zero.
\end{lemma}

\begin{proof}
(1) If $E$ has content zero, then for any $\varepsilon>0$ there are
finitely many $\Delta_{i}$ (where $i=1,\dots,N$) such that
\begin{equation}
E\subset\bigcup^{N}_{i=1}\Delta_{i}
\end{equation}
and
\begin{equation}
\sum^{N}_{i=1}|\closure{\Delta}_{i}|<\varepsilon.
\end{equation}
Then for each $\closure{\Delta}_{i}$, there is an open cell
$\Delta_{i,+}$ such that $\closure{\Delta}_{i}\propersubset\Delta_{i,+}$
and
\begin{equation}
|\Delta_{i,+}|-|\closure{\Delta}_{i}|<\frac{\varepsilon}{N}.
\end{equation}
Then
\begin{equation}
E\propersubset\bigcup^{N}_{i=1}\closure{\Delta}_{i}\propersubset\bigcup^{N}_{i=1}\Delta_{i,+},
\end{equation}
and
\begin{equation}
\sum^{N}_{i=1}|\closure{\Delta}_{i}|<\sum^{N}_{i=1}|\Delta_{i,+}|=\varepsilon+\sum^{N}_{i=1}|\closure{\Delta}_{i}|<2\varepsilon,
\end{equation}
as desired.

(2) Consider $\QQ\subset\RR$ which has measure zero. But $\closure{\QQ}=\RR$.
This obviously does not have measure zero.
\end{proof}