%%
%% winter-lecture08.tex
%% 
%% Made by Alex Nelson <pqnelson@gmail.com>
%% Login   <alex@lisp>
%% 
%% Started on  2026-01-24T09:56:54-0800
%% Last update 2026-01-24T09:56:54-0800
%% 

\lecture{}

\begin{recall}
A simple function is of the form
\begin{equation}
f = \sum^{N}_{k=1}a_{k}\chi_{E_{k}}
\end{equation}
where the $E_{k}$ are measurable sets, and this is a finite sum.
\end{recall}

\begin{theorem}% Stein, Shakarchi, Real Analysis, Theorem 4.1
Let $f\colon\RR^{d}\to[0,\infty]$ be a measurable function.
Then there exists a seuqence $(\varphi_{k})_{k\in\NN}$ of positive
$\varphi_{k}\geq0$ simple functions such that
\begin{enumerate}
\item\textsc{Monotonic:} $\varphi_{k}(x)\leq\varphi_{k+1}(x)$ for all
  $k\in\NN$ and $x\in\RR^{d}$, and
\item\textsc{Convergence:} $\lim_{k\to\infty}\varphi_{k}(x)=f(x)$ for
  all $x$.
\end{enumerate}
\end{theorem}

\begin{proof}
\begin{enumerate}
\item Truncate the range of the function to $2^{k}$
\item Divide the range into intervals of size $2^{-k}$ (for $2^{2k}-1$
  intervals). We define
  \begin{equation}
E_{k,j}=f^{-1}\bigl([2^{-k}j,2^{-k}(j+1))\bigr),
  \end{equation}
  where $j=0,1,\dots,2^{2k}-1$. We also define
\begin{equation}
\varphi_{k}(x)=\left(\sum^{2^{2k}-1}_{j=1}\frac{j}{2^{k}}\chi_{E_{k,j}}(x)\right)+2^{k}\chi_{\{f\geq2^{k}\}}(x)
\end{equation}
where $j/2^{k}$ is the ``floor value''.
(Observe: If $|f(x)|<\infty$ for all $x$, then eventually we will find a
$k\in\NN$ such that $2^{k}>f(x)$ for all $x$.)
\end{enumerate}
This gives us the sequence of functions, and we assert they have the
desired properties. Monotonicity is obvious, as is convergence.
\end{proof}

\begin{proposition}
Let $f\colon\RR^{d}\to\RR$. We can decompose this as $f=f^{+}-f^{-}$ where
\begin{equation}
f^{+}(x)=\max(f(x),0)\geq0
\end{equation}
and
\begin{equation}
f^{-}(x)=\max(-f(x),0)\geq0.
\end{equation}
Then we can apply the previous theorem to get sequences
$\varphi^{(1)}_{k}\nearrow f^{+}$ and $\varphi^{(2)}_{k}\nearrow f^{-}$.
Then $\varphi_{k}(x)=\varphi_{k}^{(1)}(x)-\varphi_{k}^{(2)}(x)$ would
give us a sequence of measurable functions. (But $\varphi_{k}$ is not
monotonic, they just converge.)
\end{proposition}

\begin{definition}
  Let $f\colon A\to\ExtRR$ be any function.
\begin{enumerate}
\item We say $f$ is \define{Finite-Valued} if $f(x)\in\RR$ for all $x\in A$.
\item We say $f$ is \define{Finite-Valued Almost Everywhere} (or
  ``Finite-Valued a.e.'') if $f(x)\in\RR$ for all $x\in A\setminus N$
  where $N$ is a null set.
\end{enumerate}
\end{definition}

\begin{recall}
A \define{Step Function} is a simple function where the $E_{k}$ are
all rectangles.
\end{recall}

\begin{theorem}
Let $f\colon\RR^{d}\to\RR$ be finite-valued almost everywhere. If $f$
is measurable, then there exists a sequence of step functions
$\psi_{k}$ such that $\psi_{k}(x)\to f(x)$ at almost every $x$.
\end{theorem}

\begin{proof}
The first step is to use earlier results from this lecture to form a
sequence $(\varphi_{k})_{k\in\NN}$ which converges to $f$ almost everywhere.

Then we take the $E_{k,j}$ measurable sets used to define the simple
functions $\varphi_{k}$. We extend them to form rectangles
$\widetilde{R}_{k,j}$ which are almost disjoint. Then we extend them
to $\widetilde{R}'_{k,j}$ such that if
$m(E\symdiff\bigcup_{j}\widetilde{R}_{k,j})<\varepsilon$
then $m(E\symdiff\bigcup_{j}\widetilde{R}'_{k,j})<2\varepsilon$.
Then we take $R_{k,j}=\Interior(\widetilde{R}'_{k,j})$ as the rectangles to
form the basis of a new sequence of simple functions.
\end{proof}