%%
%% fall-lecture15.tex
%% 
%% Made by Alex Nelson <pqnelson@gmail.com>
%% Login   <alex@lisp>
%% 
%% Started on  2025-11-08T08:56:46-0800
%% Last update 2025-11-08T08:56:46-0800
%% 

\lecture[Arzel\'{a}--Ascoli Theorem]

\begin{definition}
Let $(X,d)$ be a metric space. We define the \define{Space of
  Continuous Real-Valued Functions} on $X$ to be the vector space
$C(X)$ whose underlying set consists of all $f\colon X\to\RR$
continuous, addition is defined pointwise $(f+g)(x)=f(x)+g(x)$ as is
scalar multiplication $(\alpha f)(x)=\alpha f(x)$.

Note: if $X$ is compact, then every element of $C(X)$ has a minimum
and maximum value (by the extreme value theorem).
\end{definition}

\begin{definition}
We may define the \define{Uniform Metric} on $C(X)$ by
\begin{equation}
d(f,g)=\|f-g\|_{\infty}=\sup_{x\in X}|f(x)-g(x)|,
\end{equation}
for any $f,g\in C(X)$.

This is called the ``uniform'' metric because a sequence converges
with respect to this metric if and only if it is uniformly convergent.
\end{definition}

\begin{proposition}
Let $(X,d)$ be compact. Then $C(X)$ is complete.
\end{proposition}

\begin{proof}
Let $\{f_{n}\}$ be a Cauchy sequence in $C(X)$. We want to show it converges.
For each $\varepsilon>0$ there is an $N\in\NN$ such that for all
$m,n\geq N$ we have $\|f_{m}-f_{n}\|_{\infty}<\varepsilon$ by
definition of Cauchy. But then for all $x\in X$, we have
\begin{equation}
|f_{m}(x)-f_{n}(x)|\leq\|f_{m}-f_{n}\|_{\infty}.
\end{equation}
So for each $x\in X$, $\{f_{n}(x)\}$ is a Cauchy sequence in $\RR$,
and since $\RR$ is complete we see $\{f_{n}(x)\}$ converges. For each
$x$, we may define
\begin{equation}
f(x) := \lim_{n\to\infty}f_{n}(x),
\end{equation}
which is a real-valued function on $X$. If we can show $f$ is
continuous, then we will have $f_{n}\to f$ which will prove the
theorem.

We use the Triangle inequality, for some $n$,
\begin{equation}\label{pf:c-x-is-complete:step-zero}
|f(x)-f(y)|\leq|f(x)-f_{n}(x)|+|f_{n}(x)-f_{n}(y)|+|f_{n}(y)-f(y)|.
\end{equation}
Given $\varepsilon>0$, there exists $N_{1}\in\NN$ and $N_{2}\in\NN$
such that
\begin{equation}\label{pf:c-x-is-complete:step-one}
|f_{n}(x)-f(x)|<\frac{\varepsilon}{3}\forall n\geq N_{1}\quad\mbox{and}\quad
|f_{n}(y)-f(y)|<\frac{\varepsilon}{3}\forall n\geq N_{2}.
\end{equation}
Set $N=\max(N_{1},N_{2})$. Since $X$ is compact, by Heine--Cantor
$f_{N}$ is uniformly continuous. Then there exists $\delta>0$ such
that
\begin{equation}\label{pf:c-x-is-complete:step-two}
d(x,y)<\delta\implies|f_{N}(x)-f_{N}(y)|<\frac{\varepsilon}{3}.
\end{equation}
Now using Equations~\eqref{pf:c-x-is-complete:step-one} and~\eqref{pf:c-x-is-complete:step-two},
we can rewrite Equation~\eqref{pf:c-x-is-complete:step-zero} as
\begin{equation}
|f(x)-f(y)|<\varepsilon.
\end{equation}
This shows $f\in C(X)$. Since $\RR$ is complete, uniformly Cauchy
implies uniform convergence.
\end{proof}

\begin{lemma}\label{lemma:math-108a:fall-lec15:compact-metric-space-is-separable}
A compact metric space is separable (i.e., has a countable dense set).
\end{lemma}

\begin{proof}
Let $(X,d)$ be compact. For each $n\in\NN$, $\{B_{1/n}(x)\}_{x\in X}$ is an
open cover. By compactness, it has a finite subcover
$\{B_{1/n}(x_{i_{n}})\mid 1\leq i_{n}\leq M_{n}\}$. Then take the union of
\begin{equation}
D:=\bigcup_{n\in\NN}\{x_{i_{n}}\mid 1\leq i_{n}\leq M_{n}\},
\end{equation}
which is the countable union of finite sets (hence $D$ is a countable set).
We claim $D$ is dense, hence the result.
\end{proof}

\begin{definition}
A sequence $f_{n}\in C(X)$ is called \define{Point-wise Bounded} if
for each $x\in X$ there exists some $M_{x}\geq0$ such that
$|f_{n}(x)|\leq M_{x}$ for all $n$.

We say $f_{n}$ is \define{Uniformly Bounded} if there exists some
$M\geq0$ such that for all $x\in X$, $|f_{n}(x)|\leq M$ for any $n$.
\end{definition}

\begin{theorem}[Arzel\'{a}--Ascoli]
Let $(X,d)$ be a compact metric space. Let $\{f_{n}\}$ be a sequence of
functions in $C(X)$. If $\{f_{n}\}$ is uniformly bounded and uniformly
continuous, then it has a subsequence that converges uniformly on $X$
to a continuous function on $X$.
\end{theorem}

\begin{proof}
By Lemma~\ref{lemma:math-108a:fall-lec15:compact-metric-space-is-separable},
there exists a countably dense set $\{x_{i}\in X\}$. We claim there is
a convergent subsequence of $\{f_{n}(x_{1})\}$.

\textsc{Case 1:} If $\{f_{n}(x_{1})\}\subset\RR$ is an infinite set,
then $\{f_{n}(x_{1})\}$ is pointwise bounded (since $\{f_{n}\}$ is
uniformly bounded and $\{f_{n}(x_{1})\}$ is closed). If $\{f_{n}(x)\}$
is closed, then the Bolzano--Weierstrass Theorem implies the sequence
has a limit point. Then there is a convergent subsequence which we
write as $f_{1,n}(x_{1})$.

\textsc{Case 2:} If $\{f_{n}(x_{1})\}\subset\RR$ is a finite set.
Then we can enumerate the elements and there will eventually be
infinitely many copies of the same element. Hence the sequence is convergent.

Now, inductively, we can find $f_{2,n}(x_{2})$ where $f_{2,n}$ is a
subsequence of $f_{1,n}$, and $f_{2,n}(x_{2})$ is convergent. Then
$\{f_{2,n}(x_{2})\}$ and $\{f_{2,n}(x_{1})\}$ are both convergent (the
latter because it's a subsequence of a convergent sequence). Now we
continue constructing $f_{3,n}$ as a subsequence of $f_{2,n}$ such
which has $\{f_{3,n}(x_{3})\}$ be a convergent sequence, and so on. We
end up with a family of subsequences, which we can write out
\begin{equation}
\begin{array}{lcccc}
\{f_{1,n}\}\quad&\underline{f_{1,1}} & f_{1,2} & f_{1,3} & \dots\\
\{f_{2,n}\}\quad&f_{2,1} & \underline{f_{2,2}} & f_{2,3} & \dots\\
\{f_{3,n}\}\quad&f_{3,1} & f_{3,2} & \underline{f_{3,3}} & \dots\\
\vdots\quad&\vdots & \vdots & \vdots & \ddots
\end{array}
\end{equation}
We then construct $g_{n}=f_{n,n}$ which is the underlined diagonal of
the previous table. This $g_{n}$ is a convergent sequence when
evaluated at any $x_{i}$ --- i.e., $\{g_{n}(x_{i})\}$ is a convergent
sequence for any $x_{i}$ --- since $\{f_{j,j}(x_{i})\}_{j>i}$
converges.

Since $\RR$ is complete, it suffices to show $g_{n}$ is uniformly
Cauchy in $X$. Since $\{f_{n}\}$ is uniformly equicontinuous, we have
\begin{equation}
\forall\varepsilon>0\exists\delta>0\forall x,y\in X\forall n\ldotp d(x,y)<\delta\implies|f_{n}(x)-f_{n}(y)|<\frac{\varepsilon}{3}.
\end{equation}
In particular, this is true for each $g_{n}$.

If we consider $\{B_{\delta}(x_{i})\}$, then this is an open cover
(since $\{x_{i}\}$ is dense in $X$). Then by compactness, there is a
finite open subcover --- re-indexing the $x_{i}$, we can write this
out as $\{B_{\delta}(x_{1}),\dots,B_{\delta}(x_{M})\}$. For each
$1\leq j\leq M$, $\{g_{n}(x_{j})\}$ converges in $\RR$. Then there
exists $N_{i}\in\NN$ such that for all $m,n\geq N_{i}$,
\begin{equation}
|g_{n}(x_{i})-g_{m}(x_{i})|<\frac{\varepsilon}{3}.
\end{equation}
Taking $N=\max(N_{1},\dots,N_{M})$. Every $x$ is contained in a
$B_{\delta}(x_{i})$ (since it's a cover). Let us call the ball
containing $x\in B_{\delta}(x_{i_{0}})$. Then for all $m,n\geq N$, the
Triangle inequality gives us
\begin{equation}
\begin{split}
|g_{m}(x)-g_{n}(x)| & \leq|g_{n}(x)-g_{n}(x_{i_{0}})|+|g_{n}(x_{i_{0}})-g_{m}(x_{i_{0}})|+|g_{m}(x_{i_{0}})-g_{m}(x)|\\
&<\varepsilon,
\end{split}
\end{equation}
which shows $g_{n}$ is uniformly Cauchy. Hence the result.
\end{proof}