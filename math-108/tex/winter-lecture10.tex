%%
%% winter-lecture10.tex
%% 
%% Made by Alex Nelson <pqnelson@gmail.com>
%% Login   <alex@lisp>
%% 
%% Started on  2026-01-27T08:27:04-0800
%% Last update 2026-01-27T08:27:04-0800
%% 

\lecture{}

\begin{node}[Littlewood's 2nd principle]
Every measurable function is ``nearly'' continuous.
\end{node}

\begin{example}
The characteristic function $\chi_{\QQ}\colon\RR\to[0,1]$ is measurable.
Then $\chi_{\QQ}|_{\QQ}$ is continuous and defined on $\RR\setminus N$
where $N$ is a null set.
\end{example}

\begin{theorem}% Stein, Shakarchi, Real Analysis, Chapter I \S4 theorem 4.5
Suppose $f\colon E\to\RR$ is measurable, and $m(E)<\infty$. Then for
each $\varepsilon>0$ there exists a closed set $F_{\varepsilon}\subset E$
such that
\begin{enumerate}
\item $m(E\setminus F_{\varepsilon})<\varepsilon$, and
\item $f|_{F_{\varepsilon}}$ is continuous.
\end{enumerate}
\end{theorem}

The proof strategy consitsts of 4 steps:
\begin{enumerate}
\item Step functions are nearly continuous
\item Measurable functions are {a.e.} limits of step functions
\item Egorov's theorem makes the convergence from step 2 into
  uniformly convergent sequences on a large measurable set
\item Uniform limits of continuous functions are continuous (which is
  a result from last quarter, so we won't prove it, we'll just use it)
\end{enumerate}

\begin{proof}
\textssc{Step 1:} ``Any $\psi$ becomes a $g$''. Let
\begin{equation}
\psi=\sum^{N}_{j=1}c_{j}\chi R_{j}
\end{equation}
be a step function, where the $R_{j}$ are rectangles. In general,
$\psi$ is discontinuous on $\boundary R_{j}$. For any $\delta>0$ we
can construct a continuous $g$ such that $g(x)=\psi(x)$ except on a
set of measure less than $\delta$.

\textsc{Step 2:} By a previous result, there exists a sequence
$\psi_{n}$ of step functions such that $\psi_{n}\to f$ almost
everywhere on $E$. By step 1, we can choose continuous $g_{n}$ such
that the set of disagreements,
\begin{equation}
A_{n}=\{x\in E\mid \psi_{n}(x)\neq g_{n}(x)\},
\end{equation}
has sufficiently small measure,
\begin{equation}
m(A_{n})<2^{-n}.
\end{equation}
Then
\begin{equation}
\sum_{n}m(A_{n})\leq\sum_{n}2^{-n}<\infty.
\end{equation}
Now we can use Borel--Cantelli Theorem,
\begin{equation}
\limsup_{n\to\infty}A_{n}=0.
\end{equation}
This means that $\psi_{n}=g_{n}$ almost everywhere eventually (i.e.,
for sufficiently large $n$). This gives us $g_{n}\to f$ almost
everywhere. (This $f$ may not be continuous, and in general it is
not.)

\textsc{Step 3:} By Egorov, there exists a closed measurable set
$F_{\varepsilon}\subset E$ such that
\begin{equation}
m(E\setminus F_{\varepsilon})\leq\varepsilon,
\end{equation}
and $F_{\varepsilon}$ is closed, and $g_{n}\to f$ uniformly on $F_{\varepsilon}$.

\textsc{Step 4:} Each $g_{n}$ is continuous on $\RR^{d}$. So
$g_{n}|_{F_{\varepsilon}}$ is cotninuous on $F_{\varepsilon}$. Since
$g_{n}\to f$ uniformly on $F_{\varepsilon}$, the limit is continuous.
\end{proof}

\subsection{Integration Theory}

\begin{node}
Why does Riemann integration fail? Well, remember, the strategy for
Riemann integration is to compute something like
\begin{equation}
\int f(x)\approx\sum_{k}f(x^{*}_{k})\,\Delta x_{k}.
\end{equation}
But when $f(x)$ is highly oscillatory, this approach fails.

We will introduce Lebesgue integration on simple functions, then on
integrable functions.
\end{node}

\begin{definition}
A simple function $\varphi$ on $\RR^{d}$ is said to be in
\define{Canonical Form} if it is
\begin{equation}
\varphi(x)=\sum^{N}_{j=1}a_{j}\chi_{E_{j}}(x)
\end{equation}
where $a_{j}\in\RR$ are finite real numbers, and the $E_{j}$ are
measurable, finite measure, and pairwise disjoint.
\end{definition}

\begin{remark}
We should prove that every simple function may be written in canonical
form. It's obvious, but it's also true.
\end{remark}

\begin{definition}
Let $\varphi$ be a simple function in canonical form,
\begin{equation}
\varphi(x)=\sum^{N}_{j=1}a_{j}\chi_{E_{j}}(x).
\end{equation}
Then its \define{Integral} is
\begin{equation}
\int\varphi(x)\,\D x=\sum^{N}_{j=1}a_{j}m(E_{j}).
\end{equation}
\end{definition}

\begin{caution}
We may have well-definedness issues: if $\varphi$ can be defined in
many different ways (e.g., $\chi_{[0,1]}+\chi_{[1,2]}$ and $\chi_{[0,2]}$)
then our definition should not depend on the representation used.
\end{caution}

\begin{lemma}
Let $\varphi\geq0$ be a simple function defined on $\RR^{d}$.
Suppose we have two representations for it:
\begin{equation}
\varphi=\sum^{M}_{j=1}a_{j}\chi_{E_{j}}=\sum^{N}_{k=1}b_{k}\chi_{F_{k}}
\end{equation}
where $a_{j}\in\RR$, $b_{k}\in\RR$, and the $E_{j}$ and $F_{k}$ are
measurable sets (the $E_{j}$ are disjoint with themselves, the $F_{k}$
are disjoint with themselves, but the $E_{j}$ is not necessarily
disjoint with the $F_{k}$). Then
\begin{equation}
\sum^{M}_{j=1}a_{j}m(E_{j})=\sum^{N}_{k=1}b_{k}m(F_{k})
\end{equation}
(i.e., the integral of $\varphi$ is well-defined).
\end{lemma}

\begin{proof}[Proof sketch]
Define a common refinement
\begin{equation}
G_{jk}=E_{j}\cap F_{k},
\end{equation}
for $1\leq j\leq M$ and $1\leq k\leq N$. Then the $G_{jk}$ are all
pairwise disjoint, and they're all measurable. We have
\begin{equation}
E_{j}=\bigcup_{k=1}G_{jk}
\end{equation}
and
\begin{equation}
F_{k}=\bigcup_{j=1}G_{jk}.
\end{equation}
On each $G_{jk}$, we have $\varphi=a_{j}=b_{k}$ whenever
$m(G_{jk})>0$. (When $m(G_{jk})=0$, we just ignore it.) Since both are
equal to $\varphi$ pointwise, we have
\begin{subequations}
  \begin{align}
\sum^{M}_{j=1}a_{j}m(E_{j})
&=\sum^{M}_{j=1}a_{j}\left(\sum^{N}_{k=1}m(G_{jk})\right)\quad\mbox{by finite additivity}\\
&=\sum_{j}\sum_{k}(a_{j}m(G_{jk}))\quad\mbox{distributivity}\\
&=\sum_{k}\sum_{j}(a_{j}m(G_{jk}))\quad\mbox{distributivity}\\
&=\sum_{k}\sum_{j}(b_{k}m(G_{jk}))\quad\mbox{since $a_{j}=b_{k}$ on $G_{jk}$}\\
&=\sum_{k}b_{k}\sum_{j}(m(G_{jk}))\quad\mbox{distributivity}\\
&=\sum_{k}b_{k}m(F_{k})\quad\mbox{by finite additivity}
  \end{align}
\end{subequations}
Hence the claim.
\end{proof}