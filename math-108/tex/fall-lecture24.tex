%%
%% fall-lecture24.tex
%% 
%% Made by Alex Nelson <pqnelson@gmail.com>
%% Login   <alex@lisp>
%% 
%% Started on  2025-12-02T10:32:55-0800
%% Last update 2025-12-02T10:32:55-0800
%% 

\lecture

\begin{definition}
A function $f$ is \define{Continuous Almost Everywhere} (or
``\emph{Piecewise Continuous}'') in a set $S$ if the set of
discontinuities of $f$ is a set of measure zero. We'll write
``continuous a.e.'' as the abbreviation for this term.
\end{definition}

\begin{example}
The function
\begin{equation}
f(x) = \begin{cases}0 & x\in\RR\setminus\QQ\\
\frac{1}{q} & x=p/q\in\QQ, \gcd(p,q)=1
\end{cases}
\end{equation}
Then $f$ is continuous almost everywhere. This is because $f=0$ on
irrational numbers and discontinuous on rational numbers (which has
measure zero).
\end{example}

\begin{definition}
Let $f$ be a bounded function on a set $S$, we define its
\define{Oscillation} by $o_{f}(S)=(\sup_{x\in S}f(x))-(\inf_{x\in S}f(x))$.
\end{definition}

We see the difference between the upper and lower sums of $f$ on a
partition is just the sum of oscillations $S^{*}(f,P)-S_{*}(f,P)=\sum_{i}o_{f}(\Delta_{i})|\closure{\Delta}_{i}|$.

\begin{definition}
Let $f$ be a bounded function on a set $S$, let $a\in S$, we define its
\define{Oscillation at a point} by $o_{f}(a)=\lim_{r\to0}o_{f}(B_{r}(a))$.
\end{definition}

\begin{lemma}\label{lemma:math108a:fall2025:lec24:needed-for-riemann-lebesgue}
Let $f$ be a real-valued function on a metric space. Suppose $f$ is
bounded in an open neighborhood of $x$. Then $f$ is continuous at $x$
iff $o_{f}(x)=0$.
\end{lemma}

\begin{proof}
\forwardproof\ If $f$ is continuous at $x$, then for all
$\varepsilon>0$ there exists a $\delta>0$ such that
\begin{equation}
|f(x)-f(a)|<\frac{\varepsilon}{2}
\end{equation}
for every $a\in B_{\delta}(x)$. Then for every $a\in B_{\delta}(x)$,
\begin{equation}
f(x)-\frac{\varepsilon}{2}<f(a)<f(x)+\frac{\varepsilon}{2},
\end{equation}
which implies
\begin{equation}
o_{f}(B_{\delta}(x))<\varepsilon.
\end{equation}
Since $\varepsilon>0$ arbitrary, $o_{f}(x)=0$.

\backwardproof\ If $o_{f}(x)=0$, then for each $\varepsilon>0$ there
is a corresponding $\delta>0$ such that $o_{f}(B_{\delta}(x))<\varepsilon$.
Then whenever $a\in B_{\delta}(x)$, we have
\begin{equation}
|f(a)-f(x)|\leq o_{f}(B_{\delta}(x))<\varepsilon.
\end{equation}
Hence $f$ is continuous at $x$.
\end{proof}

\begin{theorem}[Riemann--Lebesgue]
Let $f$ be a bounded function in a cell $\closure{\Delta}\subset\RR^{n}$.
Then $\int^{*}_{\closure{\Delta}}f=\int_{*\closure{\Delta}}f$ if and
only if $f$ is continuous almost everywhere on $\closure{\Delta}$.
\end{theorem}

The proof requires us doing some preliminary work to get an equivalent
condition to ``$f$ is continuous almost everywhere on $\closure{\Delta}$''
which will be easier to prove. Then it's a standard iff proof
(consisting of a forward and backward proof).

\begin{proof}
Let $\varepsilon>0$, let
\begin{equation}
E_{\varepsilon}=\{x\in\Delta\mid o_{f}(x)<\varepsilon\},
\end{equation}
and its complement is denoted
\begin{equation}
F_{\varepsilon}=\{x\in\Delta\mid o_{f}(x)\geq\varepsilon\}.
\end{equation}
If we can show $E_{\varepsilon}$ is open in $\closure{\Delta}$, then
we'd obtain $F_{\varepsilon}$ is compact. For any $\vec{x}\in E_{\varepsilon}$,
there exists $r>0$ such that $o_{f}(B_{r}(\vec{x}))<\varepsilon$.
If $\vec{x}'\in B_{r}(\vec{x})$ and $r_{0}>0$ is sufficiently small
such that $B_{r_{0}}(\vec{x}')\propersubset B_{r}(\vec{x})$,
then $o_{f}(B_{r_{0}}(\vec{x}'))\leq o_{f}(B_{r}(\vec{x}))<\varepsilon$.
This implies $\vec{x}'\in E_{\varepsilon}$, which implies
$B_{r}(\vec{x})\propersubset E_{\varepsilon}$. Then $E_{\varepsilon}$
is open.

By the previous Lemma~\ref{lemma:math108a:fall2025:lec24:needed-for-riemann-lebesgue},
$f$ is continuous at $\vec{x}$ iff $o_{f}(\vec{x})=0$, which implies
the set of discontinuities of $f$ is
\begin{equation}
F=\{\vec{x}\in\closure{\Delta}\mid o_{f}(\vec{x})>0\},
\end{equation}
and this $F$ may be written as a union
\begin{equation}
F=\bigcup^{\infty}_{n=1}F_{1/n}.
\end{equation}
For the rest of this proof, suffices to show
$\int^{*}_{\closure{\Delta}}f=\int_{*\closure{\Delta}}f$ if and
only if $F$ has measure zero.

\backwardproof\ Assume $F$ has measure zero. Then each of $F_{1/n}$
has measure zero. Since $F_{\varepsilon}$ is compact, we have
$F_{1/n}$ has content zero. Then for each $\varepsilon>0$, there
exists finitely many open cells $\Delta'_{j}\propersubset\RR^{n}$
which cover $F_{\varepsilon}\subset\bigcup_{j}\Delta'_{j}$ and have
content zero $\sum_{j}|\closure{\Delta}_{j}|<\varepsilon$.

Its complement $K=\closure{\Delta}\setminus(\bigcup_{j}\Delta'_{j})$
is compact.

If $\vec{x}\in K$, then $\vec{x}\notin F_{\varepsilon}$, and so
$o_{f}(\vec{x})<\varepsilon$. So for all $\vec{x}\in K$, there exists
an open cell $\Delta''_{x}$ such that $\vec{x}\in\Delta''_{x}$ and
$o_{f}(\Delta''_{x})<\varepsilon$. Then $\{\Delta''_{x}\}$ is an open
cover of $K$. By compactness, there is a finite open subcover
$\Delta''_{x_{1}}$, \dots, $\Delta''_{x_{N}}$.

Let $P$ be the partition determined by the union of points of
subdivisions of the finitely many $\Delta'_{j}$ and $\Delta''_{x_{k}}$.
We write
\begin{equation}
\closure{\Delta}=\bigcup_{i}\closure{\Delta}_{i}
\end{equation}
where $\Delta_{i}\in P$.

Let $A$ be the set of indices $i$ such that
$\closure{\Delta}_{i}\subset\closure{\Delta}'_{j}$ for some $j$, so
\begin{equation}
\closure{\bigcup_{j}\Delta'_{j}}=\bigcup_{i\in A}\closure{\Delta}_{i}.
\end{equation}
Let $B$ be the set of remaining indices. We have
\begin{equation}
F_{\varepsilon}\propersubset\bigcup_{i\in A}\closure{\Delta}_{i},
\end{equation}
and
\begin{equation}
\sum_{i\in A}|\closure{\Delta}_{i}|<\varepsilon.
\end{equation}
If $i\in B$, then there is a $k$ such that
$\closure{\Delta}_{i}\subset\closure{\Delta}''_{x_{k}}$, so $o_{f}(\Delta_{i})<\varepsilon$.
Since $f$ is bounded, there exists $M$ such that $|f(\vec{x})|<M$ for
every $\vec{x}\in\closure{\Delta}$, so $o_{f}(D)<2M$ for any $D\subset\closure{\Delta}$.
Then
\begin{subequations}
  \begin{align}
S^{*}(f,P)-S_{*}(f,P) &= \sum_{i}o_{f}(\closure{\Delta}_{i})|\closure{\Delta}_{i}|\\
&=\sum_{i\in
  A}o_{f}(\closure{\Delta}_{i})|\closure{\Delta}_{i}|+\sum_{i\in B}o_{f}(\closure{\Delta}_{i})|\closure{\Delta}_{i}|\\
&<2M\sum_{i\in A}|\closure{\Delta}_{i}|+\varepsilon\sum_{i\in B}|\closure{\Delta}_{i}|\\
&<2M\varepsilon+\varepsilon|\closure{\Delta}|=\varepsilon(2M+|\closure{\Delta}|).
  \end{align}
\end{subequations}
Since $\varepsilon$ was arbitrary, this difference goes to zero, which
then implies $\int^{*}_{\closure{\Delta}}f=\int_{*\closure{\Delta}}f$.

\forwardproof\ Assume $\int^{*}_{\closure{\Delta}}f=\int_{*\closure{\Delta}}f$.
We want to prove $F$ has measure zero. For each $\varepsilon>0$ and
for every integer $n>0$, there exists a partition $P_{i}$ of
$\closure{\Delta}$ such that
\begin{equation}
S^{*}(f,P)-S_{*}(f,P)= \sum_{i}o_{f}(\closure{\Delta}_{i})|\closure{\Delta}_{i}|<\frac{\varepsilon}{n}.
\end{equation}
Let $A$ be the set of indices such that $\Delta_{i}\cap F_{1/n}\neq\emptyset$,
and let $B$ be the remaining indices. Then
\begin{equation}
F_{1/n}\subset\left(\bigcup_{i\in A}\closure{\Delta}_{i}\right)\cup\left(\bigcup_{i\in B}\boundary\closure{\Delta}_{i}\right).
\end{equation}
If $i\in A$, then there exists a point $\vec{x}_{i}\in\Delta_{i}$ such
that $o_{f}(\vec{x}_{i})\geq1/n$.
Then $o_{f}(\closure{\Delta}_{i})\geq1/n$.
Then
\begin{equation}
\frac{\varepsilon}{n}>\sum_{i\in A}o_{f}(\closure{\Delta}_{i})|\closure{\Delta}_{i}|\geq\frac{1}{n}\sum_{i\in A}|\closure{\Delta}_{i}|.
\end{equation}
Then
\begin{equation}
\varepsilon>\sum_{i\in A}|\closure{\Delta}_{i}|.
\end{equation}
So for each $F_{1/n}$, we saw we wrote this as
\begin{equation}
F_{1/n}\subset\left(\bigcup_{i\in A}\closure{\Delta}_{i}\right)\cup\left(\bigcup_{i\in B}\boundary\closure{\Delta}_{i}\right),
\end{equation}
and each $\boundary\closure{\Delta}_{i}$ is measure zero and $B$ is
countable, so $\bigcup_{i\in B}\boundary\closure{\Delta}_{i}$ is
measure zero. We just proved $\bigcup_{i\in A}\closure{\Delta}_{i}$
has measure zero. Together, this implies $F_{1/n}$ has measure
zero. Since we defined
\begin{equation}
F=\bigcup^{\infty}_{n=1}F_{1/n},
\end{equation}
this means $F$ has measure zero. Hence the result.
\end{proof}