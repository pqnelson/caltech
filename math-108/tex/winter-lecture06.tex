%%
%% winter-lecture06.tex
%% 
%% Made by Alex Nelson <pqnelson@gmail.com>
%% Login   <alex@lisp>
%% 
%% Started on  2026-01-17T08:40:22-0800
%% Last update 2026-01-17T08:40:22-0800
%% 

\lecture{}

\begin{node}
Recall, if $f\colon X\to Y$ is a continuous function, then we have
\begin{equation}
f^{-1}(\mbox{open})=\mbox{open}
\end{equation}
(the preimage of an open subset of $Y$ is an open subset of $X$). We
want something similar for the defining property of ``measurable'' functions.
\end{node}

\begin{definition}
Let $E$ be a measurable set, let $f\colon E\to\ExtRR$ be a function
from $E$ to the extended reals $\ExtRR=\RR\cup\{-\infty,+\infty\}$. We
say $f$ is \define{Measurable} if for every $a\in\RR$, the preimage
\begin{equation}
f^{-1}\bigl([-\infty,a)\bigr)=\{x\in E\mid f(x)<a\}
\end{equation}
is measurable.
\end{definition}

\begin{remark}
The rays $[-\infty,a)$ generate the Borel $\sigma$-algebra, which is
  why we want to use them in the definition.
\end{remark}

\begin{notation}
It is common to see measure theorists write something like $\{f<a\}$
as an abbreviation for
\begin{equation}
\{f<a\}:=\{x\in E\mid f(x)<a\}.
\end{equation}
Similarly, $\{f>a\}$, $\{f\leq a\}$, etc., are all abbreviations for
analogously defined sets.
\end{notation}

\begin{definition}
We say a function $f\colon E\to\ExtRR$ is \define{Finite-Valued} if
for every $x\in E$ we have $f(x)\in\RR$. That is to say, $f(E)\subset\RR$
(the image does not include any infinities).
\end{definition}

\begin{proposition}
The following conditions are equivalent:
\begin{enumerate}
\item $\{x\in E\mid f(x)<a\}$ is measurable for all $a\in\RR$
\item $\{x\in E\mid f(x)\leq a\}$ is measurable for all $a\in\RR$
\item $\{x\in E\mid f(x)>a\}$ is measurable for all $a\in\RR$
\item $\{x\in E\mid f(x)\geq a\}$ is measurable for all $a\in\RR$
\end{enumerate}
\end{proposition}

\begin{proof}
$(1)\iff(4)$ The complement of a measurable set is measurable, and
  these are complementary sets.

$(2)\iff(3)$ The complement of a measurable set is measurable, and
  these are complementary sets.

$(1)\implies(2)$ We can assume that $\{f<a + \frac{1}{k}\}$ is
  measurable for all $k\in\NN$. We claim
\begin{equation}
\{f\leq a\}=\bigcap_{k\in\NN}\{f<a+\frac{1}{k}\}.
\end{equation}
The result follows since the countable intersection of measurable sets
is measurable, so we just need to prove the claim.

$(\subset)$ If $f(x)\leq a$, then obviously for any $k\geq1$ we have $f(x)<a+\frac{1}{k}$.

$(\supset)$ If $\beta<a+\frac{1}{k}$ for all $k\in\NN$, then $\beta<a$.

$(2)\implies(1)$ We can assume that $\{f\leq a-\frac{1}{k}\}$ is
measurable for all $k\in\NN$. Then we have the equality
\begin{equation}
\{f<a\}=\bigcup_{k\in\NN}\{f\leq a-\frac{1}{k}\}
\end{equation}
and the countable union of measurable sets is measurable. Proving
these two sets are the same follows similar arguments as the previous case.
\end{proof}

\begin{corollary}
If $f$ is a measurable function, then $(-f)(x)=-(f(x))$ is a
measurable function.
\end{corollary}

\begin{proposition}[Algebraic operations]
If $f$ and $g$ are measurable functions, then so are [their pointwise sum] $f+g$ and [their
  pointwise product] $fg$
and [their pointwise absolute product] $|f|$ all measurable functions.
\end{proposition}

\begin{proof}[Proof ($f+g$ is measurable)]
Suppose $f$, $g$ are measurable. If $f(x)+g(x)<a$, then $f(x)>a-g(x)$
(and conversely). There must exist an $r\in\QQ$ such that $f(x)>r>a-g(x)$.
We want to prove $\{f+g>a\}$ is measurable. Then we see
\begin{equation}
\{f+g>a\}=\bigcup_{r\in\QQ}(\{f>r\}\cap\{g>a-r\}),
\end{equation}
and both $\{f>r\}$ and $\{g>a-r\}$ are measurable sets, so is their
intersection, and so is the countable union of such sets.
\end{proof}

\begin{proposition}[Limits]
Let $(f_{n})$ be a sequence of measurable functions. Then
$\sup_{n}f_{n}(x)$, $\inf_{n}f_{n}(x)$, $\liminf_{n}f_{n}(x)$, and
$\limsup_{n}f_{n}(x)$ are all measurable functions.
\end{proposition}
(As an immediate consequence, $\lim_{n}f_{n}(x)$ is measurable when it
exists.)

\begin{proposition}
If $f$ is measurable and $\Phi\colon\RR\to\RR$ is continuous, then
$\Phi\circ f$ is measurable.
\end{proposition}

\begin{proposition}[Almost everywhere property]
If $f$ is measurable and $f=g$ almost everywhere (i.e., $f(x)=g(x)$
for all $x\in E\setminus N$ where $N$ is some null set), then $g$ is measurable.
\end{proposition}

\begin{definition}
A \define{Simple Function} $\varphi$ is a finite sum of indicator
functions of measurable sets of finite measure,
\begin{equation}
\varphi(x)=\sum^{N}_{k=1}a_{k}\chi_{E_{k}}(x)
\end{equation}
where each $E_{k}$ is a measurable set of finite measure, and
$\chi_{E_{k}}$ is the indicator function (``characteristic function'')
of $E_{k}$. And $a_{k}\in\RR$ are some real numbers.
\end{definition}

\begin{definition}
Let $\varphi$ be a simple function. The \define{Canonical Representation}
of $\varphi$ has the $E_{k}$ be almost-disjoint (i.e., for all $i\neq j$
we have $E_{i}\cap E_{j}$ be a null set).

Intuitively, this means that there ``is no overlap'' with the summands.
\end{definition}

\begin{theorem}
If $f\colon\RR^{d}\to[0,\infty]$ is a measurable function, then there
exists a sequence of simple functions $\{\varphi_{k}\}_{k\in\NN}$ such that
\begin{enumerate}
\item\textsc{Monotonicity:} $0\leq\varphi_{1}\leq\varphi_{2}\leq\dots\leq f$
\item\textsc{Pointwise convergence:} for all $x\in\RR^{d}$, we have $\lim_{k\to\infty}\varphi_{k}(x)=f(x)$.
\end{enumerate}
\end{theorem}