%%
%% winter-lecture05.tex
%% 
%% Made by Alex Nelson <pqnelson@gmail.com>
%% Login   <alex@lisp>
%% 
%% Started on  2026-01-15T11:19:52-0800
%% Last update 2026-01-15T11:19:52-0800
%% 

\lecture{}

% stated but not proven at end of last lecture
\begin{corollary}
If $E_{k}\nearrow E$ (i.e., $E_{k}\subset E_{k+1}$ for every $k$, and
$E=\bigcup_{k\in\NN}E_{k}$), then
\begin{equation}
m(E)=\lim_{k\to\infty}m(E_{k}).
\end{equation}
\end{corollary}

% proof given at beginning of this lecture
\begin{proof}
Let $G_{1}=E_{1}$, $G_{2}=E_{2}\setminus E_{1}$, \dots,
$G_{k+1}=E_{k+1}\setminus E_{k}$ for all $k\in\NN$. Then $G_{k}$ are
disjoint and measurable sets, and
\begin{equation}
E = \bigcup^{\infty}_{k=1}G_{k}.
\end{equation}
Also observe, for any $N\in\NN$, we have:
\begin{equation}
E_{N} = \bigcup^{N}_{k=1}G_{k}.
\end{equation}
So these are ``telescoping unions'' (analogous to ``telescoping sums'').
Then
\begin{subequations}
  \begin{align}
    m(E) &= m\left(\bigcup_{k=1}^{\infty}G_{k}\right)\\
&=\sum^{\infty}_{k=1}m(G_{k})\quad\mbox{by countable additivity}\\
&=\lim_{N\to\infty}\sum^{N}_{k=1}m(G_{k})\\
&=\lim_{N\to\infty}m(E_{N})\quad\mbox{by finite additivity},
  \end{align}
\end{subequations}
which gives us the result.
\end{proof}

% stated but not proven at end of last lecture
\begin{corollary}
If $E_{k}\searrow E$ (i.e., $E_{k}\supset E_{k+1}$ for every $k$, and $E=\bigcap_{k\in\NN}E_{k}$)
and if $m(E_{1})<\infty$ is finite,
then
\begin{equation}
m(E)=\lim_{k\to\infty}m(E_{k}).
\end{equation}
\end{corollary}

\begin{proof}
Write $D_{k}=E_{1}\setminus E_{k}$ for each $k\in\NN$. Then
$D_{k}\subset D_{k+1}$. Then we see
\begin{equation}
\bigcup^{\infty}_{k=1}D_{k}=E_{1}\setminus\left(\bigcap^{\infty}_{k=1}E_{k}\right)=E_{1}\setminus E.
\end{equation}
Then applying the previous corollary to $D_{k}$ gives us
\begin{subequations}
  \begin{align}
m(E_{1}\setminus E) &= \lim_{k\to\infty}m(D_{k})\\
&=\lim_{k\to\infty}m(E_{1}\setminus E_{k})\\
&=\lim_{k\to\infty}(m(E_{1})-m(E_{k})),
  \end{align}
\end{subequations}
and using $m(E_{1}\setminus E)=m(E_{1})-m(E)$, we have
\begin{equation}
m(E_{1})-m(E)=\lim_{k\to\infty}(m(E_{1})-m(E_{k})).
\end{equation}
Then using $m(E_{1})<\infty$ is finite, we can subtract it from both
sides to conclude
\begin{equation}
m(E)=\lim_{k\to\infty}m(E_{k}),
\end{equation}
as desired.
\end{proof}

\begin{node}[Littlewood's three principles]
Littlewood's \textit{Lectures on the Theory of Functions} (1944)
describes three principles of real analysis. We are now in a position
to describe the first principle: every measurable subset of $\RR$ is
almost always a union of finitely many intervals. (More generally,
every measurable subset of $\RR^{d}$ is almost the union of finitely
many cubes.)
\end{node}

\begin{theorem}[Littlewood's first principle]
Let $E\subset\RR^{d}$ be a measurable set. If $m(E)<\infty$ has finite
measure, then for each $\varepsilon>0$ there exists a finite family of
cubes $\{Q_{j}\}_{j=1}^{N}$ such that $F=\bigcup^{N}_{j=1}Q_{j}$ and
$m(E\symdiff F)<\varepsilon$.
\end{theorem}

Note that $E\symdiff F:=(E\setminus F)\cup(F\setminus E)$ is the
symmetric difference of the two sets. (Exercise: prove the symmetric
difference of two measurable sets is measurable. In particular,
$E\symdiff F$ is a measurable set!)

\begin{proof}
Exercise.
\end{proof}

\subsection{Unmeasurable sets}

\begin{node}
We've made a big fuss over measurable sets. Are we worrying about
nothing? Is \emph{every} set secretly measurable, and this is just a
practice in self-torture?

Suppose we want a translation-invariant measure $m(E+h)=m(E)$ for any
measurable set $E\subset\RR^{d}$ and any vector $h\in\RR^{d}$, and
suppose we also want countable additivity --- for any disjoint family
of measurable sets $\{E_{j}\}_{j\in\NN}$,
\begin{equation}
m\left(\bigcup_{j=1}^{\infty}E_{j}\right)=\sum_{j=1}^{\infty}m(E_{j}).
\end{equation}
Then there must be some non-measurable sets.
\end{node}

\begin{example}
Let us work in $\RR$ (but the idea generalizes easily to other spaces).
We define an equivalence relation on the closed unit interval $[0,1]$
by $x\sim y$ if and only if $x-y\in\QQ$. We should check that this is
reflexive (since $x-x=0\in\QQ$, we see $x\sim x$ for all $x\in[0,1]$),
symmetric (since $x-y\in\QQ$ implies $y-x\in\QQ$, we see $x\sim y$
implies $y\sim x$), and transitive. So it's really an equivalence
relation.

Now, let $\mathcal{N}$ be the set constructed by using the axiom of
choice to pick one representative from each equivalence class of
$[0,1]/\sim$. There may be uncountably many equivalence classes since
$\sqrt{p}/p$ is in a different equivalence class for each prime number
$p\in\NN$, and there are so many transcendental numbers it's just too
much. So $\mathcal{N}\subset[0,1]$ is an infinite subset.

Now, let $\{r_{k}\}_{k\in\NN}$ be an enumeration of $\QQ\cap[-1,1]$
all rational numbers between $-1$ to $1$. We will consider the
translates
\begin{equation}
\mathcal{N}_{k}:=\mathcal{N}+r_{k}.
\end{equation}
What do we know about these guys?
\begin{enumerate}
\item\textsc{Disjoint:} Assume
  $\mathcal{N}_{\alpha}\neq\mathcal{N}_{\beta}$. If we assume for contradiction
  $\mathcal{N}_{\alpha}\cap\mathcal{N}_{\beta}\neq\emptyset$, then
  there exists $x+r_{\alpha}\in\mathcal{N}_{\alpha}\cap\mathcal{N}_{\beta}$ and
  $y+r_{\beta}\in\mathcal{N}_{\alpha}\cap\mathcal{N}_{\beta}$ such that 
  $x+r_{\alpha}=y+r_{\beta}$. But then
  $x-y=r_{\beta}-r_{\alpha}\in\QQ$, so $x\sim y$. By construction,
  this means $x=y$ since they are in the same equivalence class. But
  this means $r_{\alpha}=r_{\beta}$. So we have a contradiction, and
  we must have
  $\mathcal{N}_{\alpha}\cap\mathcal{N}_{\beta}=\emptyset$.
\item\textsc{Covers unit interval.} Any $x\in[0,1]$ differs from an
  element in $\mathcal{N}$ by a rational $r\in[-1,1]$. This means $[0,1]\subset\bigcup_{k=1}^{\infty}\mathcal{N}_{k}$.
\item\textsc{Boundedness.} Since $\mathcal{N}\subset[0,1]$ and
  $r_{k}\in[-1,1]$, we find $\mathcal{N}_{k}\subset[-1,3]$ for all $k\in\NN$.
\end{enumerate}
Now, we have enough to prove our big claim.

\textsc{Claim:} $\mathcal{N}$ is unmeasurable.
\begin{proof}
Assume for contradiction $\mathcal{N}$ is measurable. Well, by
translation invariance
\begin{equation}
m(\mathcal{N}_{k})=
m(\mathcal{N}+r_{k})=
m(\mathcal{N}).
\end{equation}
By countable additivity,
\begin{subequations}
  \begin{align}
m\left(\bigcup^{\infty}_{k=1}\mathcal{N}_{k}\right) &= \sum^{\infty}_{k=1}\mathcal{N}_{k}\\
&=\sum^{\infty}_{k=1}\mathcal{N}.
  \end{align}
\end{subequations}
But by the monotonicity of measures,
\begin{equation}
1=m([0,1])\leq m\left(\bigcup^{\infty}_{k=1}\mathcal{N}_{k}\right).
\end{equation}
Similarly, we see
\begin{equation}
m\left(\bigcup^{\infty}_{k=1}\mathcal{N}_{k}\right)\leq m([-1,2])=3.
\end{equation}
We know $m(\mathcal{N})$ must be finite. There are two cases:
\begin{enumerate}
\item $\mathcal{N}$ is a null set, so $m(\mathcal{N})=0\geq1$ which is
  a contradiction;
\item $m(\mathcal{N})>0$ then $m\left(\bigcup^{\infty}_{k=1}\mathcal{N}_{k}\right)=\infty\leq3$,
  which is another contradiction.
\end{enumerate}
In either event, we are forced to reject our assumption that
$\mathcal{N}$ is measurable.
\end{proof}
\end{example}

\begin{remark}
Note we discussed $\mathcal{M}$ briefly as the collection of all
measurable subsets of $\RR^{d}$. We could equally have constructed the
Borel $\sigma$-algebra $\mathcal{B}_{\RR^{d}}$ which is a strictly
smaller $\sigma$ algebra than $\mathcal{M}$ as follows: form the
family $G_{\delta}$ consisting of all countable intersections of open
subsets of $\RR^{d}$ (or, form the family $F_{\sigma}$ consisting of all
countable unions of closed subsets of $\RR^{d}$), then construct
$G_{\delta,\sigma}=(G_{\delta})_{\sigma}$ consisting of all countable unions of elements of
$G_{\delta}$ (resp., $F_{\sigma,\delta}=(F_{\sigma})_{\delta}$ consisting of all countable
intersections of elements of $F_{\sigma}$), and so on. You just keep
alternating between $\sigma$ and $\delta$ constructions (all countable
unions [resp., intersections] of elements from the previous step).

Then the limit
of this process gives you the $\mathcal{B}_{\RR^{d}}$ where
starting with $G_{\delta}$ corresponds to using ``the smallest open set $O$
containing $E\subset O$'',
and starting with $F_{\sigma}$ corresponds to using ``the biggest closed set $F$
contained in $E\supset F$'' to define measurable sets.

In practice, almost all the measurable sets you'd ever need would be
found in the Borel $\sigma$-algebra. But what's more important is that
this construction we just described works for \emph{any arbitrary}
topological space (not just $\RR^{d}$!). 
\end{remark}