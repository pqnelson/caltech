%%
%% fall-lecture08.tex
%% 
%% Made by Alex Nelson <pqnelson@gmail.com>
%% Login   <alex@lisp>
%% 
%% Started on  2025-10-16T08:44:16-0700
%% Last update 2025-10-16T08:44:16-0700
%% 

\lecture{}

We were in the middle of proving Cantor's intersection theorem. We
just need to prove the backwords direction.

\begin{proof}[Proof (Cantor intersection theorem)]
\backwardproof{} Assume every such sequence $\{E_{n}\}$ there is a
unique $x\in X$ such that $\{x\}=\bigcap^{\infty}_{n=0}E_{n}$. We want
to prove that $X$ is complete.

Let $(x_{n})$ be a Cauchy sequence in $X$. We will prove $(x_{n})$
converges. For each $n$, define
\begin{equation}
E_{n}=\closure{\{x_{k}\mid k\geq n\}},
\end{equation}
which is the closure of the tail of the sequence. We see that
$E_{n}\neq\emptyset$ and $E_{n+1}\subset E_{n}$ and $E_{n}$ is closed for each $n$.
What happens to $\diam(E_{n})$? We see that for each $\varepsilon>0$
there is a corresponding $N\in\NN$ such that
\begin{subequations}
\begin{align}
\diam(E_{N}) &= \sup\{d(x,y)\mid x,y\in E_{N}\}\\
&\leq\sup\{d(x_{m},x_{n})\mid m,n\geq N\}\\
&<\varepsilon.
\end{align}
\end{subequations}
Then we see $E_{n}$ is indeed a contracting sequence. Now we can use
the assumption: there exists an $x\in X$ such that
\begin{equation}
\{x\} = \bigcap^{\infty}_{n=0}E_{n}.
\end{equation}
For each $n$, $x\in\closure{\{x_{k}\mid k\geq n\}}$ is an element of
the closure of the tail of the sequence. So $x$ is a limit point of
the tails $\{x_{k}\mid k\geq n\}$. Then we inductively select a
subsequence $\{n_{k}\}$ such that for all $k\in\NN$,
\begin{equation}
d(x,x_{n_{k}})<\frac{1}{k}.
\end{equation}
This subsequence converges to $x$. But what we want to show is that
$x_{n}\to x$, i.e., the \emph{whole sequence} converges to $x$. We see
that the triangle inequality gives us
\begin{equation}
  \begin{split}
    d(x,x_{m})&\leq d(x,x_{n_{k}}) + d(x_{n_{k}},x_{m})\\
    &<\frac{\varepsilon}{2} + \frac{\varepsilon}{2},
  \end{split}
\end{equation}
by picking sufficiently large $m$. Then $x_{n}\to x$ converges.

Since we begin with an arbitrary Cauchy sequence, this proves $(X,d)$
is complete.
\end{proof}

\begin{definition}[Continuous at a point]
Let $(X,d)$ and $(Y,\rho)$ be metric spaces. Let $x\in X$.
We call a function $f\colon X\to Y$ \define{Continuous at $x$} if for
each $\varepsilon>0$ there exists a $\delta>0$ such that for all $a\in X$,
$d(x,a)<\delta$ implies $\rho(f(x),f(a))<\varepsilon$.

We can think of this in terms of open balls, we would have
\begin{equation}
f\bigl(B_{\delta}(x)\bigr)\subset B_{\varepsilon}\bigl(f(x)\bigr).
\end{equation}
This will be more useful when generalizing the notion in topology.
\end{definition}

\begin{definition}[Continuous function]
Let $(X,d)$ and $(Y,\rho)$ be metric spaces. 
We call a function $f\colon X\to Y$ \define{Continuous}
if $f$ is continuous at every point of $X$.
\end{definition}

\begin{remark}
We can also define a notion of ``continuous on a subset $U\subset X$''
by saying the function is continuous at each point of $U$.
\end{remark}

\begin{proposition}
Let $f\colon X\to Y$, $x\in X$, $(X,d)$ and $(Y,\rho)$ be metric spaces.
Then $f$ is continuous at $x$ if and only if for every sequence $x_{n}\to x$
in $X$ we have $f(x_{n})\to f(x)$ in $Y$.
\end{proposition}

\begin{proof}
\forwardproof\ Assume $f$ is continuous at $x\in X$.
Let $(x_{n})$ be a sequence in $X$. Assume $x_{n}\to x$ converges in $X$.
Let $\varepsilon>0$, consider $\delta>0$ such that for all $a\in X$ we have
\begin{equation}
d(x,a)<\delta\implies\rho\bigl(f(x),f(a)\bigr)<\varepsilon.
\end{equation}
Since $x_{n}\to x$ converges, there exists an $N_{\delta}\in\NN$ such that
\begin{equation}
d(x,x_{n})<\varepsilon
\end{equation}
for all $n\geq N_{\delta}$. When we substitute this into continuity,
we have
\begin{equation}
\rho\bigl(f(x),f(x_{n})\bigr)<\varepsilon
\end{equation}
for all $n\geq N_{\delta}$. So $f(x_{n})\to f(x)$ converges in $Y$.

\backwardproof\ We will do a proof by contradiction. Let $x_{n}\to x$
converges in $X$, and we have $f(x_{n})\to f(x)$ in $Y$. Assume for
contradiction that $f$ is not continuous at $x$.

We see this means
\begin{multline}
\neg\Bigl(\forall\varepsilon>0\ldotp\exists\delta>0\ldotp\forall
a\in X\ldotp \bigl(d(x,a)<\delta\implies\rho(f(x),f(a))<\varepsilon\bigr)\Bigr)\iff\\
\exists\varepsilon>0\ldotp\forall\delta>0\ldotp\exists a\in X\ldotp
d(x,a)<\delta\land\rho(f(x),f(a))\geq\varepsilon.
\end{multline}
Given such an $\varepsilon_{0}>0$, for each $n\in\NN$, there exists
$x_{n}\in B_{1/n}(x)$ such that $\rho(f(x),f(x_{n}))\geq\varepsilon_{0}$.
This means we have $x_{n}\to x$ converges in $X$, but
$f(x_{n})$ does not converge to $f(x)$ in $Y$. This contradicts our
hypothesis that $f(x_{n})\to f(x)$ converges in $Y$. Hence we reject
our assumption (that $f$ is not continuous at $x$) and conclude that
$f$ is continuous at $x$.
\end{proof}

\begin{remark}
If we want to prove $A\implies B$, then what's the difference between
``proof by contrapositive'' and ``proof by contradiction''?

Proof by contrapositive uses the fact that $A\implies B$ is logically
equivalent to $\neg B\implies\neg A$ and proves that statement.

Proof by contradiction uses the fact that $\neg\neg P$ is logically
equivalent to $P$ (and that $\neg Q$ is logically equivalent to
$Q\implies\falsum$ where $\falsum$ is the canonical contradiction), so
we have $\neg P\implies\falsum$. In our case, since $A\implies B$ is
$(\neg A)\lor B$ --- i.e., $\neg(A\land\neg B)$ --- a proof by
contradiction would prove $(A\land\neg B)\implies\falsum$.

Some useful facts in logic:
\begin{subequations}
\begin{equation}
\neg(\forall x\ldotp P[x]\implies Q[x])\iff\exists x\ldotp P[x]\land\neg Q[x]
\end{equation}
and
\begin{equation}
\neg(\exists x\ldotp P[x]\land Q[x])\iff\forall x\ldotp P[x]\implies\neg Q[x].
\end{equation}
\end{subequations}
Also recall that bounded quantifiers like $\forall x\in X\ldotp P[x]$
(respectively, $\exists y>0\ldotp Q[y]$) are shorthand for $\forall x\ldotp x\in X\implies P[x]$
(resp., $\exists y\ldotp y>0\land Q[y]$).
\end{remark}

\begin{proposition}
Let $f\colon X\to Y$. Then $f$ is continuous if and only if every open
subset $\mathcal{O}\subset Y$ we have the preimage under $f$ of
$\mathcal{O}$, $f^{-1}(\mathcal{O})$, is open in $X$.
\end{proposition}

Remember: $f^{-1}(\mathcal{O})=\{x\in X\mid f(x)\in\mathcal{O}\}$.
This horrible notation is grandfathered-in, and there's nothing we can
do about it. Very old texts on set theory use worse notation, which we
will only mention to give thanks for the alternative we use.

\begin{proof}
\forwardproof\ Assume $f$ is continuous. Let $\mathcal{O}\subset Y$ be
an open subset. (Recall, $\mathcal{O}$ is open if for all
$y\in\mathcal{O}$ there exists an open ball $B_{\varepsilon}(y)\subset\mathcal{O}$.)
Let $x\in f^{-1}(\mathcal{O})$. Then $f(x)\in\mathcal{O}$.
Since $f$ is continuous, for all $r>0$ there exists $\delta>0$ such
that every $a\in B_{\delta}(x)$ implies $f(a)\in B_{r}\bigl(f(x)\bigr)$.
This means $f\bigl(B_{\delta}(x)\bigr)\subset B_{r}\bigl(f(x)\bigr)\subset\mathcal{O}$.
Hence $B_{\delta}(x)\subset f^{-1}(\mathcal{O})$. Since this was for
arbitrary $x$, it implies $f^{-1}(\mathcal{O})$ is open.

\backwardproof\ Assume every open set $\mathcal{O}\subset Y$ has its
preimage $f^{-1}(\mathcal{O})\subset X$ be an open subset.
Let $x\in X$. Let $\varepsilon.0$. We see
$B_{\varepsilon}\bigl(f(x)\bigr)$ is open in $Y$. Then by assumption,
$f^{-1}\bigl(B_{\varepsilon}(f(x))\bigr)$ is open in $X$. We see
$x\in f^{-1}\bigl(B_{\varepsilon}(f(x))\bigr)$. There exists a
$\delta>0$ such that $B_{\delta}(x)\subset f^{-1}\bigl(B_{\varepsilon}(f(x))\bigr)$
by definition of open sets. In other words,
\begin{equation}
f\bigl(B_{\delta}(x)\bigr)\subset B_{\varepsilon}\bigl(f(x)\bigr),
\end{equation}
which is the definition of $f$ being continuous at $x$. Since we let
$x\in X$ be arbitrary, this means $f$ is continuous.
\end{proof}

\begin{corollary}
We see $f\colon X\to Y$ is continuous if and only if the inverse image
of a closed subset of $Y$ is closed in $X$.
\end{corollary}

\begin{example}
The mapping $\RR^{n}\times\RR^{n}\to\RR^{n}$ sending $(x,y)\mapsto x+y$
is continuous.
\end{example}

\begin{example}\label{ex:fall:lec08:metric-is-continuous}
Let $(X,d)$ be a metric space.
Let $x_{0}\in X$, define $f\colon X\to\RR$ by $f(x)=d(x,x_{0})$.
Then this $f$ is continuous. Look at the pre-image of the open
interval $(a,b)$ under $f$:
\begin{equation}
f^{-1}\bigl((a,b)\bigr)=\{x\in X\mid a<d(x,x_{0})<b\},
\end{equation}
then $x'\in f^{-1}\bigl((a,b)\bigr)$ when there is an open ball around
$x'$ containing (etc.\ etc.\ etc.). Take
\begin{equation}
r=\frac{1}{2}\min\{|d(x',x_{0})-a|,|d(x',x_{0})-b|\}.
\end{equation}
Then $B_{r}(x')\subset f^{-1}\bigl((a,b)\bigr)$, as desired.
\end{example}

\begin{example}
Let $(X,d)$ be a discrete metric space. Then the singleton sets $\{x\}\subset X$
are both open and closed. The only continuous function
\begin{equation}
f\colon[0,1]\to(X,d)
\end{equation}
is the constant function $r\mapsto x$. If we tried a piecewise
function like
\begin{equation}
f(a) = \begin{cases}x_{1} & \mbox{if }0\leq a\leq\frac{1}{2}\\
x_{2} & \mbox{if }\frac{1}{2}<a\leq 1,
\end{cases}
\end{equation}
well, we recall that $(1/2, 1]$ is an open subset of $[0,1]$, and it
is not a closed subset of $(1/2,1]$. This
cannot be a continuous function. Then $\{x_{2}\}$ is closed. 
But the only subsets of $[0,1]$ which are both open and closed are the
empty set $\emptyset$ and the entire unit interval itself
$[0,1]$. THIS IS IMPOSSIBLE, I TELL YOU: IMPOSSIBLE!
\end{example}