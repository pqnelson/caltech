%%
%% fall-lecture13.tex
%% 
%% Made by Alex Nelson <pqnelson@gmail.com>
%% Login   <alex@lisp>
%% 
%% Started on  2025-10-28T10:10:38-0700
%% Last update 2025-10-28T10:10:38-0700
%% 

\lecture{}

\subsection{Banach Contraction Principle}

\begin{definition}[Fixed-point]
Let $X$ be a generic set, let $T\colon X\to X$ be a function, let
$x\in X$ be a point of $X$. We say $x$ is a \define{Fixed-Point} of $T$
if $T(x)=x$.
\end{definition}

\begin{example}
The map $T\colon\RR\to\RR$, $T(x)=x+1$ has no fixed-point.
\end{example}

\begin{definition}[Lipschitz function]
Let $(X,d)$ and $(Y,\rho)$ be metric spaces. Let $f\colon X\to Y$ be a function.
We call $f$ \define{$c$-Lipschitz} if there exists a constant $c\geq0$
such that $\forall x,y\in X\ldotp\rho\bigl(f(x),f(y)\bigr)\leq c\,d(x,y)$.
(Recall, Lipschitz functions are uniformly continuous.)
\end{definition}

\begin{definition}[Contraction]
Let $(X,d)$ be a metric space.
Let $T\colon X\to X$ be a function.
We call $T$ a \define{Contraction} if it is $c$-Lipschitz for some
$0\leq c<1$.
\end{definition}

\begin{theorem}[Banach contraction principle]
Let $(X,d)$ be a complete metric space, and let $T\colon X\to X$
be a contraction.
Then $T$ has a unique fixed point.
\end{theorem}

(There are 2 claims being made here: (1) there exists a fixed-point of
$T$, and (2) the fixed-point of $T$ is unique.)

\begin{proof}
Let $c\in[0,1)$, let $T\colon X\to X$ be $c$-Lipschitz. Let $x_{0}\in X$
be any arbitrary point. We define inductively $x_{n+1}=T(x_{n})$ for
all $n\in\NN_{0}$. This sequence $(x_{n})$ is well-defined.

\textsc{Claim 1: Existence of fixed-point.} First, we see that
$d(x_{2},x_{1})=d(T(x_{1}),T(x_{0}))$ by definition of the sequence,
but then using $c$-Lipschitz this implies
$d(x_{2},x_{1})\leq c\,d(x_{1},x_{0})$.

If we look at
\begin{subequations}
\begin{align}
d(x_{k+1},x_{k}) &= d(T(x_{k}),T(x_{k-1}))\mbox{ by definition of }(x_{n})\\
&\leq c\,d(x_{k},x_{k-1})\mbox{ by $c$-Lipschitz}\\
&\leq c^{k}d(x_{1},x_{0})\mbox{ by induction}\\
&\leq c^{k}d(T(x_{0}),x_{0}).
\end{align}
\end{subequations}
For any $m,n\in\ZZ$ with $m>n$, we have (using the triangle inequality),
\begin{subequations}
\begin{align}
d(x_{m},x_{n})
&\leq d(x_{n},x_{n+1})+d(x_{n+1},x_{n+2})+\cdots+d(x_{m-1},x_{m})\\
&\leq(c^{n}+c^{n+1}+\cdots+c^{m-1})d(T(x_{0}),x_{0})\\
&\leq c^{n}\left(\frac{1-c^{m-n}}{1-c}\right)d(T(x_{0}),x_{0})\\
&\leq\frac{c^{n}}{1-c}d(T(x_{0}),x_{0})
\end{align}
\end{subequations}
since $0\leq c<1$, the numerator is bounded by $1-c^{m-n}\leq1$. Now
taking $n\to\infty$, we see $c^{n}\to 0$, so $d(x_{m},x_{n})\to0$. This
proves the sequence $(x_{n})$ is Cauchy. By hypothesis, $(X,d)$ is complete.
This implies $(x_{n})$ converges in $X$. We call the point it
converges to $x_{*}$, so $x_{n}\to x_{*}$ in $X$.

We want to prove this $x_{*}$ is a fixed-point. Since $T$ is
Lipschitz, it's continuous. So
\begin{equation}
T(x_{*}) = \lim_{n\to\infty}T(x_{n})=\lim_{n\to\infty}x_{n+1}=x_{*},
\end{equation}
hence $x_{*}$ is a fixed-point. This establishes existence of a fixed-point.

\textsc{Claim 2: uniqueness of fixed-point.} Whenever we have to prove
uniqueness, ``proof by contradiction'' simplifies things. Assume we
have two fixed-points $u=T(u)$ and $v=T(v)$. Then consider the
distance
\begin{subequations}
\begin{align}
d(u,v) &= d(T(u),T(v))\mbox{ by being fixed-points}\\
&\leq c\,d(u,v)\mbox{ by Lipschitz}
\end{align}
\end{subequations}
If $u\neq v$, then
\begin{equation}
0<d(u,v)\leq c\,d(u,v)\implies 1\leq c,
\end{equation}
but $0\leq c<1$ since $T$ is a contraction. This is a contradiction.
Hence we must conclude $u=v$.
\end{proof}

\begin{remark}
Observe that $x_{0}$ was chosen arbitrarily.
The proof gave us an algorithm for finding the fixed-point, and a
bound on where the fixed-point could be. After $m$ iterations, the
error is bounded by $c^{m}\,d(x_{1},x_{0})$.
\end{remark}

\subsection{Returning to Hausdorff Spaces}

\begin{definition}
A topological space $(X,\mathcal{T})$ is called \define{Hausdorff}
if for any distinct points $x,y\in X$, $x\neq y$, there exists
disjoint open neighborhoods $x\in U\subset X$ and $y\in V\subset X$
(and $U\cap V=\emptyset$).

If $(X,\mathcal{T})$ is Hausdorff, we will simply call it a
``Hausdorff space''.
\end{definition}

\begin{remark}
Metric spaces are Hausdorff. In metric spaces, sequences converge to
at most one point.
\end{remark}

\begin{proposition}
In a Hausdorff space $(X,\mathcal{T})$, a convergent sequence
$(x_{n})$ has a unique limit.
\end{proposition}

\begin{proof}
Assume for contradiction $(x_{n})$ converges to two distinct points
$x_{n}\to x$ and $x_{n}\to y$ and $x\neq y$. Then by Hausdorff and
since $x\neq y$, there exists two disjoint open subsets $U\subset X$
and $V\subset X$ such that $x\in U$ and $y\in V$ and $U\cap V=\emptyset$.
Then by definition of convergence, there exists positive integers
$N_{1}$ and $N_{2}$ such that $x_{n}\in U$ for all $n\geq N_{1}$ and
$x_{m}\in V$ for all $m\geq N_{2}$. Set $N=\max(N_{1},N_{2})$. Then
$x_{n}\in U\cap V$ for all $n\geq N$. But this means
$x_{n}\in\emptyset$ for all $n\geq N$, which is a contradiction. Hence
the result.
\end{proof}

\begin{proposition}
Singleton sets are closed in a Hausdorff space.
\end{proposition}

\begin{proof}
Let $(X,\mathcal{T})$ be a Hasudorff space. Let $x\in X$. We claim
$\{x\}$ is a closed subset of $X$. For each $y\neq x$, $y\in X$, there
exists an open $U_{y}\subset X$ such that $y\in U_{y}$ and $x\notin U_{y}$.
Then
\begin{equation}
U := \bigcup_{\substack{y\in X\\y\neq x}}U_{y}
\end{equation}
is an open subset of $X$ which contains every $y\in X$ distinct from
$x\neq y$. Moreover, $x\notin U$. Hence $U=X\setminus\{x\}$.
\end{proof}

\begin{definition}
Let $(X,\mathcal{T})$ be a topological space.
The notion of a \define{Cover} of $X$ refers to any family of subsets
$\{E_{\alpha}\subset X\}_{\alpha\in A}$ where $A$ is an arbitrary
indexing set, such that $\bigcup_{\alpha\in A}E_{\alpha}\supset X$.

Furthermore, an \define{Open Cover} of $X$ consists of
$E_{\alpha}\subset X$ open for each $\alpha\in A$.

A \define{Subcover} of $\{E_{\alpha}\}_{\alpha\in A}$ is a subset
$\{E_{\beta}\}_{\beta\in B}$ where $B\subset A$ and it covers
$X\subset\bigcup_{\beta\in B}E_{\beta}$.
\end{definition}

\begin{definition}
Let $(X,\mathcal{T})$ be a topological space.
Let $K\subset X$ be a subset.
We say $K$ is \define{Compact} if every open cover of $K$ has a finite subcover.
\end{definition}

\begin{theorem}[Characterization of compactness]
Let $(X,\mathcal{T})$ be a topological space.
Let $K\subset X$ be a subset. The following are true:
\begin{enumerate}
\item\label{item:fall-lec13:characterizations-of-compactness:item-1}%
  If $K$ is compact and $X$ is Hausdorff, then $K$ is closed.
\item\label{item:fall-lec13:characterizations-of-compactness:item-2}%
  If $K$ is compact and $F\subset K$ is closed, then $F$ is compact.
\item\label{item:fall-lec13:characterizations-of-compactness:item-3}%
  If $f\colon(X,\mathcal{T})\to(Y,\mathcal{T}')$ is continuous and
  $K$ is compact, then $f(K)$ is compact.
\end{enumerate}
\end{theorem}

\begin{corollary}[Extreme value theorem]
A continuous real-valued function on a compact topological spaces
takes a maximum and minimum value.
\end{corollary}

(This is a consequence of item
\ref{item:fall-lec13:characterizations-of-compactness:item-3}).

\begin{definition}
Let $(X,\mathcal{T})$ and $(Y,\mathcal{T}')$ be topological spaces.
A continuous function $f\colon X\to Y$ is called a
\define{Homeomorphism} if it is bijective and its inverse function
$f^{-1}\colon Y\to X$ is continuous.
\end{definition}

\begin{definition}
Let $(X,\mathcal{T})$ and $(Y,\mathcal{T}')$ be topological spaces.
We say $X$ and $Y$ \define{are Homeomorphic} if there exists a
homeomorphism between them.
\end{definition}

\begin{remark}
\begin{enumerate}
\item Homeomorphisms preserve topologies. So they just ``relabel points''.
\item Homeomorphisms are analogous to isometries of metric spaces,
  except isometries preserve distance. There is no notion of
  ``distance'' for topological spaces.
\end{enumerate}
\end{remark}

\begin{proposition}
Let $(X,\mathcal{T})$ and $(Y,\mathcal{T}')$ be topological spaces.
If $X$ is compact and $Y$ is Hausdorff, then
any continuous bijection $f\colon X\to Y$ is a homeomorphism.
\end{proposition}

\begin{proof}
We only need to prove $f^{-1}\colon Y\to X$ is continuous.
Suffices to show $f^{-1}$ maps closed sets to closed sets.
Let $E\subset X$ be closed.
Then $E$ is compact by~\ref{item:fall-lec13:characterizations-of-compactness:item-2}.
Then by~\ref{item:fall-lec13:characterizations-of-compactness:item-3},
$f(E)\subset Y$ is compact.
Then by~\ref{item:fall-lec13:characterizations-of-compactness:item-1}, $f(E)$
is closed. Hence $f^{-1}$ is continuous.
\end{proof}

\begin{remark}
This was not mentioned in the lectures, but it is worth mentioning: if
$(X,\mathcal{T})$ is a topology, if $W$ is a set, and $f\colon W\to X$
is a function, then we may form the \define{Induced Topology} on $W$
induced by $f$ defined by $\{f^{-1}(U)\subset W\mid U\in\mathcal{T}\}$.
This is the coarsest topology on $W$ such that $f$ is continuous.
\end{remark}