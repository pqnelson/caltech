%%
%% fall-lecture10.tex
%% 
%% Made by Alex Nelson <pqnelson@gmail.com>
%% Login   <alex@lisp>
%% 
%% Started on  2025-10-21T12:36:54-0700
%% Last update 2025-10-21T12:36:54-0700
%% 

\lecture{}

\begin{definition}
Let $(x_{n})$ be a sequence, let $(n_{k})$ be a sequence of positive
integers such that $n_{1}<n_{2}<\cdots$ (it is strictly
increasing). Then $(x_{n_{k}})$ is a \define{Subsequence} of $(x_{n})$.

If $(x_{n_{k}})$ converges, then its limit is called a
\define{Subsequential Limit} of $(x_{n})$.
\end{definition}

\begin{definition}
We call $E\subset X$ \define{Subsequentially Compact} if every
sequence in $E$ has a subsequence that converges to a point in $E$.
\end{definition}

\begin{theorem}
Let $(X,d)$ be a metric space. Then the following are equivalent:
\begin{enumerate}
\item $X$ is complete and totally bounded;
\item $X$ is compact;
\item $X$ is sequentially compact.
\end{enumerate}
\end{theorem}

We'll prove $(1)\implies(2)\implies(3)\implies(1)$. These are three
proofs, so we will present them separately.

\begin{proof}
$(1)\implies(2)$. Assume for contradiction that $X$ is not compact.
Then consider $\{U_{\alpha}\}$ an open cover of $X$ such that it has
no finite subcover. We will repeatedly use the following construction
(described in the next paragraph).

(i) Since $X$ is totally bounded, we can consider a finite collection
of open balls of radius $r<1/2$ which cover $X$. Then at least one of
these open balls $B_{1}$ that can not be covered by finitely many
$U_{\alpha}$. We call its closure $A_{1}=\closure{B_{1}}$. Then
$A_{1}$ is closed and $\diam(A_{1})\leq1$.

Using totally boundedness again, repeating $(i)$ but with balls of
radius $r<1/4$, we find another open ball $B_{2}$ whose intersection
with $A_{1}$ cannot be covered by finitely many $U_{\alpha}$. We call
the closure of this open ball intersected with $A_{1}$,
\begin{equation*}
A_{2} =\closure{A_{1}\cap B_{2}}.
\end{equation*}
We have $B_{2}\cap A_{1}$ cannot be covered by finitely many
$U_{\alpha}$. Then $A_{2}$ is closed and $A_{2}\subset A_{1}$ and
$\diam(A_{2})\leq1/2$.

We repeat this process inductively and obtain a contracting sequence
of nonempty closed subsets $(A_{n})$. We see that
$\diam(A_{n})\leq2^{-n}$ and $A_{n+1}\subset A_{n}$. Since $X$ is
complete, we use Cantor's intersection Theorem. (Each $A_{n}$ cannot
be covered by finitely many $U_{\alpha}$.) There exists a point
$x_{0}\in X$ which is the unique intersection point
\begin{equation}
\{x_{0}\}=\bigcap^{\infty}_{n=1}A_{n}.
\end{equation}
There exists a particular $U_{\alpha_{0}}$ in our open cover such that
$x_{0}\in U_{\alpha_{0}}$. (There may be several, but we only need one.)
Since $U_{\alpha_{0}}$ is open, there exists an $r_{0}>0$ such that
$B_{r_{0}}(x_{0})\subset U_{\alpha_{0}}$. We know $\diam(A_{n})\to0$,
so there exists some index $N\in\NN$ such that for all $n\geq N$ we
have
\begin{equation}
A_{n}\subset B_{r_{0}}(x_{0})\subset U_{\alpha_{0}}.
\end{equation}
This is a contradiction, because we chose these $A_{n}$ to be
impossible to cover by finitely many $U_{\alpha}$ (and we just covered
them with a single one). Hence the contradiction. Hence $X$ is compact.
\end{proof}

\begin{proof}
$(2)\implies(3)$ Assume $X$ is compact. Let $(x_{n})$ be a sequence in
$X$. We want to show $(x_{n})$ has a convergent subsequence in $X$.
Define the set
\begin{equation}
E=\{x_{n}\in X\mid n\in\NN\}
\end{equation}
to be the set of terms appearing in the sequence. Then either $E$ is a
finite set or not. So we have a proof by cases.

\textsc{Suppose} $E$ is a finite set. Then we can use the pidgeonhole
principle: there exists an $x\in E$ repeated infinitely many times,
and a sequence $(n_{k})$ where $n_{1}<n_{2}<\dots$ of positive
integers such that the subsequence $(x_{n_{k}})$ is identically
$x$. Then $x_{n_{k}}\to x$ converges.

\textsc{Suppose} $E$ is an infinite set. There are infinitely many
distinct terms in $(x_{n})$. We assume for contradiction no point in
$X$ is a limit point of $E$. Then for each $z\in X$, there is a
neighborhood $z\in U_{z}$ which intersects $E\cap U_{z}\subset\{z\}$
at most trivially.
Then $\{U_{z}\}_{z\in X}$ is an open cover of $X$ and an open cover of
$E$ and has no finite subcover. (Since $E$ has infinitely many points
and intersects $U_{z}$ in at most 1 point.) Then $\{U_{z}\}$ has no
finite subcover of $X$. This contradicts the hypothesis that $X$ is
compact. So we reject the assumption that $E$ has no limit point. Let
us call that limit point $x\in X$. Then we can choose the smallest $n$
such that $x_{n_{1}}\in B_{1}(x)$. We can choose $n_{k}$ to be the
smallest integer such that $n_{k}>n_{k-1}$ such that $x_{n_{k}}\in B_{1/k}(x)$.
We have then constructed a subsequence $(x_{n_{k}})\to x$. This proves
the claim.
\end{proof}

\begin{proof}
$(3)\implies(1)$ Assume $X$ is sequentially compact. Assume for
contradiction $X$ is not totally bounded. Then there exists some
$\varepsilon>0$ such that we cannot cover $X$ with finitely many
open balls of radius $\varepsilon$.

We select a point $x_{1}\in X$. Then $X$ cannot be covered by
$B_{\varepsilon}(x_{1})$. Then consider $x_{2}\in X$ such that
$d(x_{1},x_{2})\geq2\varepsilon$. Then $B_{\varepsilon}(x_{1})\cap B_{\varepsilon}(x_{2})=\emptyset$
are disjoint, and $B_{\varepsilon}(x_{1})\cup B_{\varepsilon}(x_{2})$
does not cover $X$.

Inductively, we select $x_{3}\in X$ such that
$\diam(x_{3},x_{i})\geq2\varepsilon$ for $i=1,2$, so we can continue
in this manner to get $(x_{n})$ a sequence with no convergent
subsequence (because any two terms are at least $\varepsilon$
apart). This means $X$ is not sequentially compact. This is a
contradiction. Hence $X$ is totally bounded.

But we are not finished, we need to prove $X$ is complete. Let
$(x_{n})$ be an arbitrary Cauchy sequence. Then $(x_{n})$ has a
convergent subsequence $(x_{n_{k}})\to x$ by sequential
completeness. Then $(x_{n})\to x$ since it is a Cauchy sequence, we
have (using the triangle inequality):
\begin{equation}
d(x,x_{m})\leq d(x,x_{n_{k}})+d(x_{n_{k}},x_{m}),
\end{equation}
where for sufficiently big $n_{k}\geq N$ we have $d(x,x_{n_{k}})<\varepsilon/2$
and similarly for sufficiently big $m\geq N$ we have $d(x_{n_{k}},x_{m})<\varepsilon/2$
which taken together implies
\begin{equation}
d(x,x_{m})<\varepsilon
\end{equation}
for $m\geq N$. Hence $x_{n}\to x$ as desired.
\end{proof}

\begin{remark}
If $E\subset\RR^{n}$, then the following are equivalent:
\begin{enumerate}
\item $E$ is closed and bounded;
\item $E$ is compact;
\item Every subset of $E$ has a limit point of $E$.
\end{enumerate}
\end{remark}

\begin{proposition}
Let $E\subset\RR^{n}$. Then $E$ is bounded iff $E$ is totally bounded.
\end{proposition}

\begin{proof}
\backwardproof\ Obvious, discussed last time.

\forwardproof\ For simplicity, suppose $n=2$ (the argument generalizes
to other $n$). Let $\varepsilon>0$. Since $E$ is bounded, there is an
open ball which totally contains $E$; there is an $a>0$ sufficiently
large such that $E\subset[-a,a]^{2}$. Then we take a partition $P_{k}$
of $[-a,a]$ such that each interval has length \emph{less than}
$1/k$. Then $P_{k}\times P_{k}$ is a partition of $[-a,a]^{2}$ and
each square has a diameter bounded by $\sqrt{2}/k$. We can choose $k$
sufficiently large such that $\sqrt{2}/k<\varepsilon$. There are
finitely many such open squares which cover $E$. This implies there
are finitely many open balls of radius $\varepsilon$ covering
$E$. Hence the result.
\end{proof}

\begin{theorem}[Heine--Cantor]
Let $f\colon(X,d)\to(Y,\rho)$ be a continuous function, where $(X,d)$ is a
compact metric space and $(Y,\rho)$ is an arbitrary metric space.
Then $f$ is uniformly continuous.
\end{theorem}

\begin{proof}
By contradiction. Assume for contradiction $f$ is not uniformly
continuous. Then there exists an $\varepsilon>0$ such that for every
$\delta>0$ and for every $a,b\in X$ we have
\begin{equation}
d(a,b)<\delta\quad\mbox{and}\quad\rho\bigl(f(a),f(b)\bigr)\geq\varepsilon.
\end{equation}
Since $(X,d)$ is compact, it is sequentially compact.
Then for each $n$, there exists $x_{n}\in X$ and $y_{n}\in X$ such
that
\begin{equation}
d(x_{n},y_{n})<\frac{1}{n}\quad\mbox{and}\quad\rho\bigl(f(x_{n}),f(y_{n})\bigr)\geq\varepsilon.
\end{equation}
Then there exists an $x\in X$ and subsequence $(x_{n_{k}})$ such that
$x_{n_{k}}\to x$ converges.
Then using the triangle inequality, we have
\begin{align}
d(y_{n_{k}},x) &\leq d(y_{n_{k}},x_{n_{k}})+d(x_{n_{k}},x)\\
&\leq\frac{1}{n_{k}}+d(x,x_{n_{k}}),
\end{align}
which goes to zero for sufficiently large $n_{k}$.
Then $(y_{n_{k}})\to x$.
Then by continuity of $f$, their images (using the triangle inequality):
\begin{equation}
\rho\bigl(f(x_{n_{k}}),f(y_{n_{k}})\bigr)\leq
\rho\bigl(f(x_{n_{k}}),f(x)\bigr)+\rho\bigl(f(x),f(y_{n_{k}})\bigr),
\end{equation}
which goes to zero by continuity. But this contradicts
$\rho(f(x_{n_{k}}),f(x))\geq\varepsilon$. hence $f$ must be uniformly continuous.
\end{proof}