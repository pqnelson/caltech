%%
%% fall-lecture12.tex
%% 
%% Made by Alex Nelson <pqnelson@gmail.com>
%% Login   <alex@lisp>
%% 
%% Started on  2025-10-25T11:48:53-0700
%% Last update 2025-10-25T11:48:53-0700
%% 

\lecture{}

\begin{definition}
Let $(X,\mathcal{T})$ be a topological space. Let $E\subset X$.
We call $E$ \define{Closed} if $E^{\complement}=X\setminus E$ is open.

We will generically write $\mathcal{F}$ for the set of all closed
subsets of $(X,\mathcal{T})$.
\end{definition}

\begin{proposition}\label{prop:lec12:properties-of-closed-sets}
Let $(X,\mathcal{T})$ be a topological space. If $\mathcal{F}$ is the
collection of closed subsets of $X$, then:
\begin{enumerate}
\item $X\in\mathcal{F}$ and $\emptyset\in\mathcal{F}$
\item $\bigcap\{F_{\alpha}\in\mathcal{F}\mid \alpha\in I\}\in\mathcal{F}$
for any arbitrary indexing set $I$
\item $F_{1}\cup\cdots\cup F_{n}\in\mathcal{F}$ for $F_{i}\in\mathcal{F}$
\end{enumerate}
\end{proposition}

\begin{definition}
Let $(X,\mathcal{T})$ be a topological space.
Let $E\subset X$ be a subset.
We define the \define{Closure} of $E$ to be the subset $\closure{E}$
of $X$ equal to
\begin{equation}
\closure{E}=\bigcap\{F\in\mathcal{F}\mid E\subset F\},
\end{equation}
the intersection of all closed sets containing $E$.
\end{definition}

\begin{proposition}
The closure of $E$ is a closed subset of $(X,\mathcal{T})$.
\end{proposition}

\begin{proposition}
$E$ is closed if and only if $E=\closure{E}$.
\end{proposition}

\begin{proof}
Observe $E\subset\closure{E}$ always. So we only need to check if
$\closure{E}\subset E$ or not.

\forwardproof\ Assume $E$ is closed. Then there is an $F\in\mathcal{F}$,
$F=E$, so
\begin{subequations}
\begin{align}
\closure{E} &= \bigcap\{F\subset X\mid F\mbox{ is closed}, E\subset F\}\\
&= E\cap\left(\bigcap\{F\subset X\mid F\mbox{ is closed}, E\subset F\}\right)\\
&= E.
\end{align}
\end{subequations}

\backwardproof\ If $\closure{E}\subset E$, then $\closure{E}=E$ and
$\closure{E}$ is the intersection of closed sets. Therefore
$\closure{E}$ is closed by Definition of closure and Proposition~\ref{prop:lec12:properties-of-closed-sets}.
\end{proof}

\begin{proposition}
\begin{enumerate}
\item $\closure{E}$ is closed
\item $\closure{E}$ is the smallest closed subset of $X$ containing
  $E\subset\closure{E}$ (in the sense that: for any $F\subset X$
  closed, if $E\subset F$, then $\closure{E}\subset F$).
\end{enumerate}
\end{proposition}

\begin{definition}
Let $(X,\mathcal{T})$ be a topological space.
Let $E\subset X$.
Let $x\in X$. We call $x$ a \define{Limit Point} of $E$ if every
neighborhood $N_{x}\in\mathcal{T}$ of $x\in N_{x}$ contains a $y\in E$ such that $x\neq y$.
\end{definition}

\begin{definition}
We call $E\subset X$ \define{Dense} if $\closure{E}=X$. 
\end{definition}

\begin{definition}
We call $(X,\mathcal{T})$ \define{Separable} if it has a countable
dense subset.
\end{definition}

\begin{example}[Compact metric spaces are separable]
Let $(X,d)$ be a compact metric space.
We claim $X$ is totally bounded: For each $n\in\NN$, we can cover $X$
by a finite collection of $(1/n)$-balls. We take the centerpoints for
all such balls for all $n\in\NN$. This defines the subset
\begin{equation}
D=\{\mbox{centerpoints of }(1/n)\mbox{-balls covering }X\mid n\in\NN\}.
\end{equation}
Then $D$ is countable and dense.
\end{example}

\begin{proposition}
Let $E\subset X$.
\begin{enumerate}
\item $\closure{E}=E\subset\{x\in X\mid x\mbox{ is a limit point of }E\}$
\item $E$ is closed if and only if it contains all its limit points.
\item $E$ is dense if and only if every point $x\in X$ is a limit
  point of $E$.
\end{enumerate}
\end{proposition}

\begin{definition}
Let $(x_{n})$ be a sequence in a topological space $(X,\mathcal{T})$.
We say $(x_{n})$ \define{Converges} to $x\in X$ if for every
neighborhood $U$ of $x$, there exists an $N\in\NN$ such that $x_{n}\in U$ for every $n\geq N$.
\end{definition}

\begin{definition}
Let $(X,\mathcal{T}_{1})$ and $(Y,\mathcal{T}_{2})$ be topological spaces.
Let $x\in X$.
Let $f\colon X\to Y$ be a function.
We say $f$ is \define{Continuous} at $x$ if for each $V\in\mathcal{T}_{2}$
containing $f(x)\in V$, there exists a $U\in\mathcal{T}_{1}$ such that
$x\in U$ and $f(U)\subset V$.

We say $f$ is \define{Continuous} if it is continuous at every point
of $X$.
\end{definition}

\begin{proposition}
The function $f\colon(X,\mathcal{T}_{1})\to(Y,\mathcal{T}_{2})$.
is continuous if and only if every open subset $V\subset Y$ has an
open preimage $f^{-1}(V)\in\mathcal{T}_{1}$.
\end{proposition}

\begin{proposition}
Let $f\colon(X,\mathcal{T}_{1})\to(Y,\mathcal{T}_{2})$, $g\colon(Y,\mathcal{T}_{2})\to(Z,\mathcal{T}_{3})$
be continuous functions.
Then $g\circ f\colon(X,\mathcal{T}_{1})\to(Z,\mathcal{T}_{3})$ is continuous.
\end{proposition}

\begin{definition}
Let $U_{}$ and $U_{2}$ be proper open subsets of $(X,\mathcal{T})$.
We say $U_{1}$, $U_{2}$ \define{Separate} $X$ if they're disjoint
$U_{1}\cap U_{2}=\emptyset$, and they cover $X=U_{1}\cup U_{2}$.
\end{definition}

\begin{definition}
We call the topological space $(X,\mathcal{T})$ \define{Connected} if
there is no $U_{1}$, $U_{2}\propersubset X$ which separate $X$.
\end{definition}

\begin{remark}
We can also define connected by: for any clopen subsets $U_{1}$, $U_{2}\subset X$
[i.e., subsets which are both open and closed] such that $U_{1}$ and $U_{2}$
are disjoint and cover $X$,
we have $\{\emptyset,X\}=\{U_{1},U_{2}\}$ iff $X$ is connected.
\end{remark}

\begin{proposition}\label{prop:fall-lec12:continuous-image-of-connected-set-is-connected}
The continuous image of a connected set is connected.
\end{proposition}

\begin{definition}
Let $(X,\mathcal{T})$ be a topological space.
We say $(X,\mathcal{T})$ has the \define{Intermediate Value Property} (IVP)
if the image of any continuous real-valued function is an interval:
for all $f\colon X\to\RR$ continuous, $f(X)$ is an interval.
\end{definition}

\begin{proposition}
The topological space $(X,\mathcal{T})$ has IVP if and only if it is connected.
\end{proposition}

\begin{proof}
[Exercise: a subset of $\RR$ is connected iff it is an interval.]

\backwardproof\ Obvious by Proposition~\ref{prop:fall-lec12:continuous-image-of-connected-set-is-connected}.

\forwardproof\ Need to show if $X$ has IVP, then $X$ is connected. We
will prove the contrapositive: if $X$ is not connected, then $X$ does
not have IVP. Assume there exists a separating pair $U_{1}$,
$U_{2}\subset X$. We need to construct $f\colon X\to\RR$ which is not
an interval. Take $f|_{U_{1}}\colon U_{1}\to\{0\}$ and
$f|_{U_{2}}\colon U_{2}\to\{1\}$. We claim $f$ is continuous.

Let $A$ be an open subset in $\RR$. Then $f^{-1}(A)$ is open in $(X,\mathcal{T})$,
which can be verified by examining the four cases:
\begin{enumerate}
\item If $0\in A$ and $1\in A$, then $f^{-1}(A)=X$
\item If $0\in A$ and $1\notin A$, then $f^{-1}(A)=U_{1}$
\item If $0\notin A$ and $1\in A$, then $f^{-1}(A)=U_{2}$
\item If $0\notin A$ and $1\notin A$, then $f^{-1}(A)=\emptyset$.
\end{enumerate}
But $f(X)=\{0,1\}$ is not connected in $\RR$. Hence $X$ does not have
the IVP.
\end{proof}

\begin{definition}
Let $(X,\mathcal{T})$ be a topological space.
We call $(X,\mathcal{T})$ \define{Hausdorff} if for any distinct
points $x\in X$ and $y\in X$ (where $x\neq y$) there exists disjoint
open subsets $U,V\subset X$ with $x\in U$ and $y\in V$ and $U\cap V=\emptyset$.

We will call such an $(X,\mathcal{T})$ a \emph{Hausdorff Space}.
\end{definition}

\begin{example}[Trivial topology is not Hausdorff]
The trivial topological space --- the trivial topology on $X$ --- is
not Hausdorff. The only open sets containing any two distinct points
$x\in X$, $y\in X$ is the entire set $X$ itself.
\end{example}

\begin{example}[Metric spaces generally Hausdorff]
Metric spaces induce a topology which is generally Hausdorff.
Take $0<r<d(x,y)/2$ and then we see $B_{r}(x)$ and $B_{r}(y)$ are
disjoint open sets containing $x$ and $y$, respectively.
\end{example}



% \mathcal{T}