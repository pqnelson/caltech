%%
%% fall-lecture11.tex
%% 
%% Made by Alex Nelson <pqnelson@gmail.com>
%% Login   <alex@lisp>
%% 
%% Started on  2025-10-23T12:39:18-0700
%% Last update 2025-10-23T12:39:18-0700
%% 

\lecture{}

\begin{theorem}[Baire Category]
Let $(X,d)$ be a complete metric space.
Let $\{\mathcal{O}_{n}\}$ be a sequence of dense open subsets of $X$.
Then $\bigcap_{n}\mathcal{O}_{n}$ is dense in $X$.
\end{theorem}

\begin{proof}
Let $x_{0}\in X$ and $r_{0}>0$. We want to show $x_{0}$ is a limit
point of $\bigcap_{n}\mathcal{O}_{n}$, i.e., 
\begin{equation}
B_{r_{0}}(x_{0})\cap\left(\bigcap_{n}\mathcal{O}_{n}\right)\neq\emptyset.
\end{equation}
Since $\mathcal{O}_{1}$ is dense in $X$, we know
$B_{r_{0}}(x_{0})\cap\mathcal{O}_{1}$ is nonempty. Take $x_{1}\in B_{r_{0}}(x_{0})\cap\mathcal{O}_{1}$
and since this intersection is an open set, we know there exists an
$0<r_{1}<1$ such that $B_{1}=B_{r_{1}}(x_{1})$ satisfying
$\closure{B_{1}}\subset B_{r_{0}}(x_{0})\cap\mathcal{O}_{1}$.

Inductively, we have chosen this, we have a sequence of open balls
$B_{1}$, $B_{2}$, \dots, where $B_{k}$ has radius $r_{k}<1/k$ and each
$\closure{B_{k}}\subset B_{k-1}\cap\mathcal{O}_{k}$. This makes
$(B_{k})$ a contracting sequence of nonempty sets, so
$(\closure{B_{k}})$ is a contracting sequence of closed nonempty
sets. Since $X$ is complete, we can use the Cantor intersection
Theorem, then there exists an $x^{*}\in X$ such that
\begin{equation}
\{x^{*}\}=\bigcap^{\infty}_{k}\closure{B_{k}}.
\end{equation}
Then
\begin{equation}
x^{*}\in\bigcap^{\infty}_{k}\mathcal{O}_{k}
\end{equation}
since $B_{k}\subset\mathcal{O}_{k}$.

On the other hand, $x^{*}\in B_{0}=B_{r_{0}}(x_{0})$. This implies
\begin{equation}
B_{0}\cap\left(\bigcap^{\infty}_{k=0}\mathcal{O}_{k}\right)\neq\emptyset.
\end{equation}
Hence the result.
\end{proof}

\subsection{Topology}

\begin{definition}
Let $X$ be a set. A \define{Topology} of $X$ is a collection
$\mathcal{T}$ of subsets of $X$ such that
\begin{enumerate}
\item $\emptyset\in\mathcal{T}$ and $X\in\mathcal{T}$
\item Closed under finite intersections: if $U_{1}$, \dots, $U_{n}\in\mathcal{T}$
for any $n\in\NN$, then $U_{1}\cap\cdots\cap U_{n}\in\mathcal{T}$.
\item Closed under arbitrary unions: if
  $\{U_{\alpha}\in\mathcal{T}\mid\alpha\in I\}$ where $I$ is an
  arbitrary indexing set (possibly infinite), then $\bigcup_{\alpha\in I}U_{\alpha}\in\mathcal{T}$.
\end{enumerate}
\end{definition}

\begin{definition}
A set equipped with a topology is called a \define{Topological Space}.
The elements of the topology are called \define{Open Sets}.
\end{definition}

\begin{definition}
Let $(X,\mathcal{T})$ be a topological space.
Let $x\in X$.
A \define{Neighborhood} of $x$ is an open set $U$ of $X$ which
contains $x\in U$.
\end{definition}

\begin{example}
Let $(X,d)$ be a metric space.
We can define the \define{Metric Topology} for $X$ consisting of all
open subsets (defined using the metric) which are the only open sets
of the topology.
\end{example}

\begin{example}
The \define{Discrete Topology} of $X$ is the powerset of $X$. It's
also induced by the metric topology using the discrete metric on $X$.
\end{example}

\begin{example}
The \define{Trivial Topology} for $X$ is just $\mathcal{T}=\{\emptyset,X\}$.
\end{example}

\begin{example}
The \define{Subspace Topology}: if $(X,\mathcal{T})$ is a topological
space and $E\subset X$ is a nonempty subset $E\neq\emptyset$, then the
\define{Inherited Topology} on $E$ is given by the set
\begin{equation}
\mathcal{S}=\{U\cap E\mid U\in\mathcal{T}\}.
\end{equation}
We call $(E,\mathcal{S})$ a \define{Subspace} of $(X,\mathcal{T})$.
\end{example}

\begin{definition}
Let $(X,\mathcal{T})$ be a topological space.
A \define{Base} for $\mathcal{T}$ is a collection $\mathcal{B}$ of
open sets such that:
\begin{itemize}
\item For every $G\in\mathcal{T}$, there is a collection of elements
  $B_{\alpha}\in\mathcal{B}$ such that $\bigcup_{\alpha}B_{\alpha}=G$.
\end{itemize}
\end{definition}

\begin{example}
The base for a metric topology is the set of open balls.
\end{example}

\begin{example}
The base for the discrete topology is the set of all singleton sets.
\end{example}

\begin{example}
The base for the trivial topology on a set $X$ is just $\{X\}$.
\end{example}

\begin{example}
The base for the subspace topology on $E$: if $\mathcal{B}$ is a base
for $(X,\mathcal{T})$, then
\begin{equation}
\mathcal{B}'=\{E\cap B\mid B\in\mathcal{B}\}
\end{equation}
is a base for the subspace topology.
\end{example}

\begin{remark}
Observe if $(X,\mathcal{T})$ is a topological space, then
$\mathcal{T}$ is a base for itself. So there always exists at least
one base for a given topology (so this is a well-defined notion).
\end{remark}

\begin{definition}
Let $X$ be a set, let $\mathcal{B}$ be an arbitrary family of subsets
of $X$ such that $\bigcup\mathcal{B}=X$ (they cover $X$). Then we may
define the \define{Topology Generated} by $\mathcal{B}$ to be the
collection $\mathcal{T}$ of subsets of $X$ such that:
\begin{itemize}
\item A subset $U\subset X$ is an element of $\mathcal{T}$ if for each
  $x\in U$ there exists a $B\in\mathcal{B}$ such that $x\in B$ and
  $B\subset U$.
\end{itemize}
\end{definition}

\begin{proposition}
Let $(X,\mathcal{T})$ be a topological space.
Let $\mathcal{B}$ be a collection of subsets for $X$.
Then $\mathcal{B}$ is a base for $\mathcal{T}$ iff 
\begin{enumerate}
\item $\mathcal{B}$ covers $X$: $X=\bigcup\mathcal{B}$, and also
\item for all $B_{1},B_{2}\in\mathcal{B}$, for all $x\in X$,
  if $x\in B_{1}\cap B_{2}$, then there exists a $B\in\mathcal{B}$
  such that $x\in B\subset B_{1}\cap B_{2}$.
\end{enumerate}
\end{proposition}

\begin{proof}
\forwardproof\ Assume $\mathcal{B}$ is a base. Then $X\in\mathcal{T}$,
so $X$ must be the union of elements of $\mathcal{B}$ by definition of
the base. So (1) is satisfied.

(2) $\mathcal{B}\subset\mathcal{T}$, so when $B_{1}\in\mathcal{B}$ and
$B_{2}\in\mathcal{B}$, then they are elements of the topology. So
$B_{1}\cap B_{2}\in\mathcal{T}$. By Definition of a base, $B_{1}\cap B_{2}\in\mathcal{T}$
is the union of a family of elements $\{B_{k}\}\subset\mathcal{B}$. In
particular, any $x\in B_{1}\cap B_{2}$ must belong to one of the
$B_{k}\subset B_{1}\cap B_{2}$, which proves (2).

\backwardproof\ Assume $\mathcal{B}$ satisfies the two conditions. We
will build a topology $\mathcal{T}$ from $\mathcal{B}$, prove
$\mathcal{T}$ is a topology, and $\mathcal{B}$ is a base for that
topology. We define $\mathcal{T}$ to be the family of unions of
subcollections of $\mathcal{B}$ adjoined with $\emptyset$.

We claim $\mathcal{T}$ is a topology, let us verify the condition:
\begin{enumerate}
\item $\emptyset\in\mathcal{T}$ since we adjoined $\emptyset$ to
  $\mathcal{T}$ explicitly; and $X=\bigcup\mathcal{B}$, so $X\in\mathcal{T}$.
\item Finite intersections: Let $U_{1}$, \dots,
  $U_{n}\in\mathcal{T}$. Then $U_{1}\cap\cdots\cap
  U_{n}\in\mathcal{T}$.
  This follows from property (ii). Let $U_{1}=\bigcup\mathcal{B}_{1}$
  and $U_{2}=\bigcup\mathcal{B}_{2}$ for some
  $\mathcal{B}_{i}\subset\mathcal{B}$ and $i=1,2$. Then
  \begin{equation}
U_{1}\cap U_{2}=\bigcup\{B_{1}\cap B_{2}\mid B_{i}\in\mathcal{B}_{i},i=1,2\},
  \end{equation}
  but this is just the union of
  \begin{equation}
\mathcal{B}_{12}=\{B_{12}\subset B_{1}\cap B_{2}\mid
B_{12}\in\mathcal{B}, B_{1}\in\mathcal{B}_{1}, B_{2}\in\mathcal{B}_{2}\},
  \end{equation}
  which proves the closure under finite intersections.
\item Closure under arbitrary unions: Any $\{G_{\alpha}\mid\exists\mathcal{S}_{\alpha}\subset\mathcal{B},G_{\alpha}=\bigcup\mathcal{S}_{\alpha}\}$
  has
  \begin{equation}
\bigcup_{\alpha}G_{\alpha}=\bigcup_{\alpha}\left(\bigcup\mathcal{S}_{\alpha}\right)
=\bigcup\left(\bigcup_{\alpha}\mathcal{S}_{\alpha}\right)
  \end{equation}
  and $\bigcup_{\alpha}\mathcal{S}_{\alpha}\subset\mathcal{B}$. Hence
  (3) holds.
\end{enumerate}
Hence $\mathcal{T}$ is a topology.
\end{proof}

% \mathcal{T}

\begin{remark}
A topology may be generated by many distinct bases. So the base is not
unique. 
\end{remark}

\begin{example}
Let $(X,d)$ be a metric space.
\begin{enumerate}
\item $\mathcal{B}=\{B_{r}(x)\mid x\in X,r>0\}$ is a basis for the
  metric topology.
\item $\mathcal{B}=\{B_{r}(x)\mid x\in X,r>0,r\in\QQ\}$ is another basis for the
  metric topology.
\end{enumerate}
\end{example}

\begin{definition}
Let $(X,\mathcal{T})$ be a topological space.
A \define{Sub-base} of $\mathcal{T}$ is a subcollection
$\mathcal{S}\subset\mathcal{T}$ such that
\begin{equation}
\mathcal{B}=\{\bigcap\mathcal{C}\mid\mathcal{C}\in\Fin(\mathcal{S})\},
\end{equation}
where $\Fin(X)$ is the subset of the power set of $X$ consisting of
finite subsets of $X$ only. We call $\mathcal{B}$ the \define{Base
  Generated by} $\mathcal{S}$.
\end{definition}

\begin{remark}
\begin{enumerate}
\item The topology generated by $\mathcal{S}$ is unique
\item A set which covers $X$ is a sub-base for some topology. In
  particular, this means an open cover of $X$ gives us a sub-base for
  some topology (the smallest topology where those sets are open).
\end{enumerate}
\end{remark}

\begin{example}
Every base is also a sub-base.
\end{example}

\begin{example}
For $X=\RR$, $\mathcal{S}=\{(-\infty,a)\mid a\in\RR\}\cup\{(a,\infty)\mid a\in\RR\}$
is a sub-base but not a base for the standard topology on $\RR$.
\end{example}

\begin{example}
Let $\mathcal{S}=\{\emptyset,(-\infty,1),(-1,\infty)\}$. This generates the base
\begin{equation}
\mathcal{B}=\{\emptyset,(-\infty,1),(-1,1),(-1,\infty)\}
\end{equation}
and
\begin{equation}
\mathcal{T}=\{\emptyset,(-\infty,1),(-1,\infty),(-1,1),\RR\}.
\end{equation}
It is a topology, but is obviously not the standard topology.
\end{example}