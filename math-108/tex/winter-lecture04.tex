%%
%% winter-lecture04.tex
%% 
%% Made by Alex Nelson <pqnelson@gmail.com>
%% Login   <alex@lisp>
%% 
%% Started on  2026-01-13T10:47:00-0800
%% Last update 2026-01-13T10:47:00-0800
%% 

\lecture{}

\begin{node}
We showed we can wrap a measurable set by open sets with small
error. We can also fill a measurable set by closed sets with small error.
\end{node}

\begin{corollary}
If $E$ is measurable, then for any $\varepsilon>0$ there is a
corresponding closed set $F\subset E$ such that $m(E\setminus F)<\varepsilon$.
\end{corollary}

\begin{proof}
Since $E$ is measurable, we have $E^{\complement}$ is measurable.
Then by definition of measurability applied to $E^{\complement}$,
there is an open set $O$ such that $E^{\complement}\subset O$ and
$m(O\setminus E^{\complement})<\varepsilon$. Let $F=O^{\complement}$.
Then $F$ is closed. Since $E^{\complement}\subset O$, we see
$O^{\complement}\subset(E^{\complement})^{\complement}=E$. So
$F\subset E$. Now we can compare the errors. We know
\begin{equation}
E\setminus F=E\cap F^{\complement}=E\cap O=O\cap(E^{\complement})^{\complement}=O\setminus E^{\complement}.
\end{equation}
Then $m(E\setminus F)=m(O\setminus E^{\complement})<\varepsilon$.
\end{proof}

\begin{proposition}[Countable intersections of measurable sets is measurable]
If $\{E_{j}\}_{j\in\NN}$ is a countable family of measurable sets,
then $\bigcap_{j\in\NN}E_{j}$ is measurable.
\end{proposition}

\begin{proof}
We see
\begin{equation}
\left(\bigcap_{j\in\NN}E_{j}\right)^{\complement}=\bigcup_{j\in\NN}(E_{j}^{\complement}).
\end{equation}
Since each $E_{j}$ is measurable, each $E_{j}^{\complement}$ is measurable.
We also know the countable union of measurable sets is measurable---so
$\bigcup_{j\in\NN}(E_{j}^{\complement})$ is measurable. Then its
complement
\begin{equation}
\left(\bigcup_{j\in\NN}(E_{j}^{\complement})\right)^{\complement}=\bigcap_{j\in\NN}E_{j}
\end{equation}
is measurable.
\end{proof}

\begin{node}
Let $\mathcal{M}\subset\powerset{\RR^{d}}$ be the set of all measurable subsets of $\RR^{d}$.
Then it has the following properties:
\begin{enumerate}
\item $\emptyset\in\mathcal{M}$ and $\RR^{d}\in\mathcal{M}$
\item Closed under complements: if $X\in\mathcal{M}$, then $X^{\complement}\in\mathcal{M}$
\item Closed under countable unions: if $E_{j}\in\mathcal{M}$ for each $j\in\NN$,
  then we have their union $\bigcup_{j=1}^{\infty}E_{j}\in\mathcal{M}$.
\end{enumerate}
If we replace $\RR^{d}$ with ``some set'' $X$, then the analogous
collection $\mathcal{M}\subset\powerset{X}$ is called a
\define{$\sigma$-Algebra} of $X$.
\end{node}

\begin{node}
Recall, $m_{*}(A\cup B)\leq m_{*}(A)+m_{*}(B)$ for exterior
measures. We will prove $m(A\cup B)=m(A)+m(B)$ for disjoint $A,B\in\mathcal{M}$,
and also the exterior measure of a measurable set agrees with the
measure of that set $m_{*}|_{\mathcal{M}}=m$.
\end{node}

\begin{example}
Let $\chi_{\QQ}$ be the indicator function for the rational numbers.
Well, $\chi_{\QQ\cap[0,1]}$ is not Riemann integrable. The reason is
that for any partition $P$ of $\QQ\cap[0,1]$, due to the density of
the rational numbers in the set of reals, the upper Riemann sum
will always be 1, but the lower Riemann sum will always be 0.
\end{example}

\begin{theorem}[Countable additivity]
If $\{E_{j}\}_{j\in\NN}$ is a family of disjoint measurable sets and $E=\bigcup_{j\in\NN}E_{j}$,
then $m(E)=\sum_{j\in\NN}m(E_{j})$.
\end{theorem}

\begin{proof}
We will prove $m(E)\geq\sum_{j}m(E_{j})$ because subadditivity already
gives us $m(E)\leq\sum_{j}m(E_{j})$.

\textsc{Case 1:} every $E_{j}$ is bounded. Let $\varepsilon>0$. For
each $j$ we can find a closed set $F_{j}\subset E_{j}$ such that
\begin{equation}
m(E_{j}\setminus F_{j})\leq\frac{\varepsilon}{2^{j}}.
\end{equation}
Since each $E_{j}$ is bounded, each $F_{j}$ are compact and
disjoint. For any $N\in\NN$, we see $F_{1}$, \dots, $F_{N}$ are
compact and disjoint. Then
\begin{equation}
d(F_{j},F_{k})>0.
\end{equation}
This implies
\begin{equation}
m\left(\bigcup^{N}_{j=1}F_{j}\right)=\sum^{N}_{j=1}m(F_{j}).
\end{equation}
Since $F_{j}\subset E_{j}$, we have
\begin{equation}
m(F_{j}\geq m(E_{j}) - \frac{\varepsilon}{2^{j}}.
\end{equation}
Then we see
\begin{equation}
m(E)\geq m\left(\bigcup^{N}_{j=1}F_{j}\right)=\sum^{N}_{j=1}m(F_{j})\geq\sum^{N}_{j=1}(m(E_{j})-\frac{\varepsilon}{2^{j}})\geq\left(\sum^{N}_{j=1}m(E_{j})\right)-\varepsilon,
\end{equation}
for arbitrary $N$. Then we can send $N\to\infty$ and conclude
\begin{equation}
m(E)\geq\left(\sum^{\infty}_{j=1}m(E_{j})\right)-\varepsilon.
\end{equation}
Since $\varepsilon>0$ was arbitrary, we conclude
\begin{equation}
m(E)\geq\sum^{\infty}_{j=1}m(E_{j}),
\end{equation}
hence $m(E)=\sum_{j\in\NN}m(E_{j})$ as desired.

\textsc{Case 2:} $E_{j}$ are unbounded (e.g., $E_{j}=[j,j+1)$).
The trick: chop the space into bounded rings. Let $Q_{k}=[-k,k]^{d}$
be the cube centered at the origin. Then we can define the ``annuli''
$S_{k}$ by
\begin{equation}
S_{1}=Q_{1}\quad\mbox{and}\quad S_{k+1}=Q_{k+1}\setminus Q_{k}
\end{equation}
for all $k$. The $\{S_{k}\}_{k\in\NN}$ are disjoint, measurable, and
\begin{equation}
\bigcup_{k\in\NN}S_{k}=\RR^{d}.
\end{equation}
For any set $A$, we can always write
\begin{equation}
A = \bigcup^{\infty}_{k=1}(A\cap S_{k}),
\end{equation}
as the union of bounded disjoint sets. (If furthermore $A$ is measurable, then we
can write $A$ as the union of bounded, disjoint, measurable sets.) So
given any $E_{j}$ we have
\begin{equation}
E_{j}=\bigcup_{k\in\NN}(E_{j}\cap S_{k}),
\end{equation}
that is to say, we have $E_{j}$ as a countable union of bounded, disjoint,
measurable sets. Let us write $E_{j,k}=E_{j}\cap S_{k}$.
Now we use bounded additivity (i.e., the result of case 1) twice
\begin{subequations}
  \begin{align}
m(E) &= \sum^{\infty}_{k=1}m(E\cap S_{k})\\
&=\sum^{\infty}_{k=1}m\left(\bigcup^{\infty}_{j=1}E_{j,k}\right)\\
&=\sum^{\infty}_{k=1}\sum^{\infty}_{j=1}m(E_{j,k})\quad\mbox{by bounded additivity}\\
&=\sum_{j}\sum_{k}m(E_{j,k})\quad\mbox{since each summand is non-negative}\\
&=\sum^{\infty}_{j=1}m(E_{j})
  \end{align}
\end{subequations}
by bounded additivity again.
\end{proof}

\begin{corollary}
If $E_{k}\nearrow E$ (i.e., $E_{k}\subset E_{k+1}$ for every $k$, and
$E=\bigcup_{k\in\NN}E_{k}$), then
\begin{equation}
m(E)=\lim_{k\to\infty}m(E_{k}).
\end{equation}
\end{corollary}

\begin{corollary}
If $E_{k}\searrow E$ (i.e., $E_{k}\supset E_{k+1}$ for every $k$, and $E=\bigcap_{k\in\NN}E_{k}$)
and if $m(E_{1})<\infty$ is finite,
then
\begin{equation}
m(E)=\lim_{k\to\infty}m(E_{k}).
\end{equation}
\end{corollary}

\begin{remark}
We could have started with family of subsets
$\mathcal{M}\subset\powerset{X}$ and some function
$m\colon\mathcal{M}\to[0,\infty]$ such that
\begin{enumerate}
\item $m(\emptyset)=0$, and
\item $m(\bigcup_{j\in\NN}E_{j})=\sum_{j\in\NN}m(E_{j})$ for disjoint $E_{j}\in\mathcal{M}$.
\end{enumerate}
But that's too abstract to learn measure theory.
\end{remark}