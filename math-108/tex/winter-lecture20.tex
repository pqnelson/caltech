%%
%% winter-lecture20.tex
%% 
%% Made by Alex Nelson <pqnelson@gmail.com>
%% Login   <alex@lisp>
%% 
%% Started on  2026-02-19T11:16:23-0800
%% Last update 2026-02-19T11:16:23-0800
%% 

\lecture{}

\begin{recall}
We defined and introduced Hilbert spaces $L^{2}(\RR)$, $L^{2}([0,1])$, and
$\ell^{2}(\ZZ,\CC)$. We also defined a notion of isomorphism for
Hilbert spaces as inner product preserving bijective linear
transformations. 
\end{recall}

\begin{node}[Polarization identity]
Let $T\colon\mathcal{H}\to\mathcal{H}'$ be a linear transformation
between Hilbert spaces. If $T$ preserves the norm, then $T$ preserves
the inner product.
\end{node}

\subsection{Closed subspaces and Orthogonal projection}

\begin{definition}
Let $\mathcal{H}$ be a Hilbert space, let $\mathcal{S}\subset\mathcal{H}$
be a subspace. We say $\mathcal{S}$ is \define{Closed} if every
sequence $(f_{n})\subset\mathcal{S}$ which converges $f_{n}\to f$ to
some $f\in\mathcal{H}$ (converging in norm $\|f_{n}-f\|_{\mathcal{H}}\to0$)
contains its limit $f\in\mathcal{S}$.
\end{definition}

\begin{non-example}
The epitome of a non-example: continuous functions on the interval as
a subspace of $L^{2}$ functions $C([-1,1])\subset L^{2}([-1,1])$. We
know continuous functions $f\in C^([-1,1])$ attain maximum values in
the interval
\begin{equation}
M=\max_{I}f.
\end{equation}
We then have $|f|\leq M$ which implies $|f|^{2}\leq M^{2}$, so we get:
\begin{equation}
\int^{1}_{-1}|f|^{2}\leq2M^{2}.
\end{equation}
However, we could have a sequence of continuous functions converge to
a step function, which is not continuous.
\end{non-example}

\begin{problem}[Approximation problem]
Let $\mathcal{H}$ be a Hilbert space. Let
$\mathcal{S}\subset\mathcal{H}$ be a subspace.

\textsc{Given:} $f\in\mathcal{H}$.

\textsc{Find:} the closest ``approximation'' to $f$ in $\mathcal{S}$.
\end{problem}

\begin{theorem}[Projection]\label{thm:4.1}
Let $\mathcal{H}$ be a separable Hilbert space, let
$\mathcal{S}\subset\mathcal{H}$ be a \emph{closed} subspace of
$\mathcal{H}$. Let $f\in\mathcal{H}$.
\begin{enumerate}
\item\textsc{Existence and uniqueness of minimizer:} There exists a
  unique $g_{0}\in\mathcal{S}$ closest to $f$ in the sense that
  \begin{equation}
\|f-g_{0}\|_{\mathcal{H}}=\inf_{g\in\mathcal{S}}\|f-g\|_{\mathcal{H}}=d.
  \end{equation}
\item\textsc{Orthogonality to the subspace:} For any $g\in\mathcal{S}$,
\begin{equation}
(f-g_{0},g)_{\mathcal{H}}=0;
\end{equation}
i.e., $f-g_{0}\perp\mathcal{S}$.
\end{enumerate}
\end{theorem}

\begin{proof}
\begin{enumerate}
\item\textsc{Existence:} Let
\begin{equation}
d=\inf_{g\in\mathcal{S}}\|f-g\|_{\mathcal{H}}.
\end{equation}
We can choose a sequence $(g_{n})_{n\in\NN}\subset\mathcal{S}$ such
that $\|f-g_{n}\|_{\mathcal{H}}\to d$ (we know such a sequence exists
by definition of the infinimum).

\textsc{Claim 1:} $(g_{n})$ is Cauchy. We apply the parallelogram law
to $f-g_{n}$ and $f-g_{m}$:
\begin{subequations}
  \begin{align}
2(\|f-g_{n}\|^{2}+\|f-g_{m}\|^{2})
&=\|(f-g_{m})+(f-g_{n})\|^{2}+\|(f-g_{n})-(f-g_{m})\|^{2}\\
&=\|2f-g_{m}-g_{n}\|^{2}+\|g_{m}-g_{n}\|^{2}.
  \end{align}
\end{subequations}
We observe that
\begin{equation}
\left\|f-\frac{g_{m}+g_{n}}{2}\right\|\geq d,
\end{equation}
and that $(g_{m}+g_{n})/2\in\mathcal{S}$ since $\mathcal{S}$ is a
vector space. Then we rearrange terms to find
\begin{equation}
\|g_{n}-g_{m}\|^{2}\leq2\bigl(\|f-g_{n}\|^{2}+\|f-g_{m}\|^{2})-4d^{2}
\end{equation}
since
\begin{subequations}
  \begin{align}
\|2f-g_{m}-g_{n}\|^{2}
&=4\left\|f-\frac{g_{m}+g_{n}}{2}\right\|^{2}\\
&\leq4d^{2}.
  \end{align}
\end{subequations}
Taken together, this implies $\|g_{n}-g_{m}\|^{2}\leq0$, which implies
$(g_{n})$ is Cauchy.

\textsc{Claim 2:} Since $\mathcal{H}$ is complete, $g_{n}\to g_{0}\in\mathcal{H}$.
Since $\mathcal{S}$ is closed, $g_{0}\in\mathcal{S}$. Hence $g_{n}\to g_{0}$
in $\mathcal{S}$.
\item\textsc{Orthogonality:} We will use a variational argument. Let $h=f-g_{0}$.
Assume for contradiction there is a \emph{nonzero} $g\in\mathcal{S}$ such that
\begin{equation}
(h,g)_{\mathcal{H}}\neq0.
\end{equation}
Consider the perturbation $g_{0}+\varepsilon g$ for some ``small''
$-1<\varepsilon<1$ with ``correct sign''. We see then that
\begin{subequations}
  \begin{align}
\|f-(g_{0}+\varepsilon g)\|^{2}
&=\|h-\varepsilon g\|^{2}\\
&=\|h\|^{2}-2\Re\bigl(\varepsilon\cdot(h,g)\bigr)+|\varepsilon|^{2}\|g\|^{2}\\
&<\|h\|^{2}
  \end{align}
\end{subequations}
where we pick $\varepsilon$ such that $2\Re(\varepsilon(h,g))>|\varepsilon|^{2}\|g\|^{2}$.
This contradicts the result that $g_{0}$ is the minimizer. The faulty
assumption is that $(h,g)\neq0$. Hence we conclude $(h,g)=0$ as desired.\qedhere
\end{enumerate}
\end{proof}

\begin{definition}
Let $\mathcal{S}\subset\mathcal{H}$ be a subspace of a Hilbert space
$\mathcal{H}$. We define its \define{Orthogonal Complement} to be the
subspace $\mathcal{S}^{\perp}\subset\mathcal{H}$ consisting of
$f\in\mathcal{S}^{\perp}$ such that for all $g\in\mathcal{S}$ we have
$(f,g)=0$. 
\end{definition}

% This is from lecture 21, but seems more suitably placed here
\begin{proposition}
We can write $\mathcal{H}=\mathcal{S}\oplus\mathcal{S}^{\perp}$.
\end{proposition}

\begin{proof}
\begin{enumerate}
\item\textsc{Existence}: By the Projection Theorem~\ref{thm:4.1},
let $g_{0}\in\mathcal{S}$ be the closest element to $f$, so we can
write
\begin{equation}
f=g_{0}+(f-g_{0}).
\end{equation}
We see $f-g_{0}\in\mathcal{S}^{\perp}$, so we denote it by
\begin{equation}
h:=f-g_{0}.
\end{equation}
Then $f=g_{0}+h$ is a valid decomposition, which gives us existence.
\item\textsc{Uniqueness}: Suppose we had two such decompositions
\begin{equation}
f=g+h=\widetilde{g}+\widetilde{h}.
\end{equation}
Then
\begin{equation}
g-\widetilde{g}=\widetilde{h}-h,
\end{equation}
and the left-hand side belongs to $\mathcal{S}$, but the right hand
side belongs to $\mathcal{S}^{\perp}$. The only element belonging to
both subspaces is the zero vector. This means
\begin{equation}
0=g-\widetilde{g}=\widetilde{h}-h,
\end{equation}
and therefore
\begin{equation}
g=\widetilde{g}\quad\mbox{and}\quad\widetilde{h}=h.
\end{equation}
Hence the decomposition is unique.\qedhere
\end{enumerate}
\end{proof}

\begin{definition}
Let $\mathcal{H}$ be a Hilbert space.
Let $\mathcal{S}$ be a closed subspace of $\mathcal{H}$.
The \define{Orthogonal Projection} onto $\mathcal{S}$ is a linear
transformation $P_{S}\colon\mathcal{H}\to\mathcal{S}$ such that for
all $f\in\mathcal{H}$ we have $P_{S}(f)=g$ where $f=g+h$ with
$g\in\mathcal{S}$ and $h\in\mathcal{S}^{\perp}$. This is well-defined
due to the previous propositions.
\end{definition}

\begin{proposition}
The orthogonal projection $P_{S}$ of $\mathcal{H}$ onto $\mathcal{S}$
satisfies the following properties:
\begin{enumerate}
\item $P_{S}$ is linear
\item $P_{S}(f)=f$ whenever $f\in\mathcal{S}$
\item $P_{S}(f)=0$ whenever $f\in\mathcal{S}^{\perp}$
\item $\|P_{S}(f)\|_{\mathcal{H}}\leq\|f\|_{\mathcal{H}}$ for all $f\in\mathcal{H}$
\end{enumerate}
\end{proposition}

\begin{observation}
  We can abstractly define:
\begin{enumerate}
\item A \emph{projection} is a linear transformation
  $P\colon\mathcal{H}\to\mathcal{H}$ such that $P\circ P=P$.
\item An \emph{orthogonal projection} as a linear
transformation $P\colon\mathcal{H}\to\mathcal{H}$ such that $P\circ P=P$
and it is self-adjoint (i.e., equals its complex conjugate of its transpose) $P=P^{*}$.
\item An \emph{oblique projection} is a projection which is not an
  orthogonal projection.
\end{enumerate}
\end{observation}