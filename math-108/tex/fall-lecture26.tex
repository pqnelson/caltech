%%
%% fall-lecture26.tex
%% 
%% Made by Alex Nelson <pqnelson@gmail.com>
%% Login   <alex@lisp>
%% 
%% Started on  2025-12-06T11:00:16-0800
%% Last update 2025-12-06T11:00:16-0800
%% 

\lecture[Fubini's Theorem]

\begin{notation}
Let $\closure{\Delta}\subset\RR^{n+m}=\RR^{n}\times\RR^{m}$.
We will write this closed cell out as a product
$\closure{\Delta}=\closure{\Delta}'\times\closure{\Delta}''$ where
$\closure{\Delta}'\subset\RR^{n}$ and $\closure{\Delta}''\subset\RR^{m}$.
Then $f\colon\closure{\Delta}\to\RR$ can be written as
$f(\vec{x}',\vec{x}'')$ where $\vec{x}'\in\closure{\Delta}'$ and
$\vec{x}''\in\closure{\Delta}''$. 
\end{notation}

\begin{definition}
Let $f\colon\closure{\Delta}\to\RR$ be bounded
on $\closure{\Delta}\subset\RR^{n+m}=\RR^{n}\times\RR^{m}$, and let
$\closure{\Delta}=\closure{\Delta}'\times\closure{\Delta}''$ where
$\closure{\Delta}'\subset\RR^{n}$ and
$\closure{\Delta}''\subset\RR^{m}$.
Then:
\begin{enumerate}
\item If we fix $\vec{x}''\in\closure{\Delta}''$, we can write
  $\int^{*}_{\closure{\Delta}''}\left(\int^{*}_{\closure{\Delta}'}f\right)$
  for the \define{Iterated Upper Integral}
\item If we fix $\vec{x}''\in\closure{\Delta}''$, we can write
  $\int_{*,\closure{\Delta}''}\left(\int_{*,\closure{\Delta}'}f\right)$
  for the \define{Iterated Lower Integral}
\end{enumerate}
\end{definition}

\begin{theorem}
If $f$ is Riemann integrable on $\closure{\Delta}=\closure{\Delta}'\times\closure{\Delta}''$
where $\closure{\Delta}'\subset\RR^{n}$, $\closure{\Delta}''\subset\RR^{m}$,
then the integrals
\begin{equation}
\int^{*}_{\closure{\Delta}'}f(\vec{x}',\vec{x}''),\quad\mbox{and}\quad\int_{*,\closure{\Delta}'}f(\vec{x}',\vec{x}'')
\end{equation}
are Riemann integrable functions of $\vec{x}''\in\closure{\Delta}''$.
We can rewrite the integral
\begin{subequations}
  \begin{align}
\int_{\closure{\Delta}}f &= \int_{\closure{\Delta}''}\left(\int^{*}_{\closure{\Delta}'}f(\vec{x}',\vec{x}'')\right)\\
&=\int_{\closure{\Delta}''}\left(\int_{*,\closure{\Delta}'}f(\vec{x}',\vec{x}'')\right)
  \end{align}
\end{subequations}
\end{theorem}

\begin{proof}
Let $P'$ be a partition of $\closure{\Delta}'=\bigcup_{i}\closure{\Delta}'_{i}$,
and $P''$ be a partition of $\closure{\Delta}''=\bigcup_{j}\closure{\Delta}''_{j}$.
Then we can form a partition $P$ of $\closure{\Delta}$ which is
$P'\times P''$. Now, let us consider the upper sum
\begin{subequations}
  \begin{align}
S^{*}(f,P) &= \sum_{i,j}\left(\sup_{\substack{\vec{x}'\in\closure{\Delta}'_{i}\\\vec{x}''\in\closure{\Delta}''_{j}}}f(\vec{x}',\vec{x}'')\right)|\closure{\Delta}'_{i}\times\closure{\Delta}''_{j}|\\
&= \sum_{j}\left(\sum_{i}\left(\sup_{\substack{\vec{x}'\in\closure{\Delta}'_{i}\\\vec{x}''\in\closure{\Delta}''_{j}}}f(\vec{x}',\vec{x}'')\right)|\closure{\Delta}'_{i}|\right)|\closure{\Delta}''_{j}|
  \end{align}
\end{subequations}
For each fixed $\vec{x}''_{0}\in\closure{\Delta}''_{j}$, define
\begin{equation}
F(\vec{x}''_{0}) := \int_{\vec{x'}\in\closure{\Delta}'}f(\vec{x}',\vec{x}''_{0}),
\end{equation}
then we see
\begin{subequations}
  \begin{align}
\sum_{i}\left(\sup_{\substack{\vec{x}'\in\closure{\Delta}'_{i}\\\vec{x}''\in\closure{\Delta}''_{j}}}f(\vec{x}',\vec{x}'')\right)|\closure{\Delta}'_{i}|
&\geq\sum_{i}\left(\sup_{\vec{x}'\in\closure{\Delta}'_{i}}f(\vec{x}',\vec{x}''_{0})\right)|\closure{\Delta}'_{i}|\\
&\geq S^{*}(f(\vec{x}',\vec{x}''_{0}),P')\\
&\geq\int^{*}_{\vec{x}'\in\closure{\Delta}'}f(\vec{x}',\vec{x}''_{0})=F(\vec{x}''_{0}).
  \end{align}
\end{subequations}
Since the choice for $\vec{x}''_{0}\in\closure{\Delta}''_{0}$ was
arbitrary, we can return to the upper sum to find
\begin{equation}
S^{*}(f,P)\geq\sum_{j}(\sup_{\vec{x}''\in\closure{\Delta}''}F(\vec{x}''))|\closure{\Delta}''_{j}|=S^{*}(F,P'')\geq\int^{*}_{\closure{\Delta}''}F.
\end{equation}
Then
\begin{equation}
\int^{*}_{\Delta}f=\inf_{P}S^{*}(f,P)\geq\int^{*}_{\vec{x}''\in\closure{\Delta}''}\left(\int^{*}_{\vec{x}'\in\closure{\Delta}'}f(\vec{x}',\vec{x}'')\right).
\end{equation}
Similarly, we have an upper bound for the lower integral of
$f$. Putting them together, we find
\begin{equation}
\int^{*}_{\closure{\Delta}}f
\geq\int^{*}_{\vec{x}''}\int^{*}_{\vec{x}'}f(\vec{x}',\vec{x}'')
\geq\int_{*,\vec{x}''}\int^{*}_{\vec{x}'}f
\geq\int_{*,\vec{x}'}\int_{*,\vec{x}'}f
\geq\int_{*,\closure{\Delta}}f,
\end{equation}
Since $f$ is Riemann integrable,
\begin{equation}
\int^{*}_{\closure{\Delta}}f=\int_{*,\closure{\Delta}}f,
\end{equation}
and so these inequalities are all equalities. In particular, we have
\begin{equation}
\int^{*}_{\vec{x}''}\int^{*}_{\vec{x}'}f(\vec{x}',\vec{x}'')=\int_{*,\vec{x}''}\int^{*}_{\vec{x}'}f=\int_{\closure{\Delta}}f.
\end{equation}
This means
\begin{equation}\label{eq:math108a:fall2025:lec26:penultimate-step}
\int_{\vec{x}''}\int^{*}_{\vec{x}'}f=\int_{\closure{\Delta}}f.
\end{equation}
Similar reasoning gives us a chain of inequalities
\begin{equation}
\dots\geq\int^{*}_{\vec{x}''}\int_{*,\vec{x}'}f\geq\int_{*,\vec{x}''}\int_{*,\vec{x}'}f\geq\dots,
\end{equation}
which gives us
\begin{equation}
\int_{\vec{x}''\in\closure{\Delta}''}\int_{*,\vec{x}'}f=\int_{\closure{\Delta}}f.
\end{equation}
The result follows immediately by combining this with Equation~\eqref{eq:math108a:fall2025:lec26:penultimate-step}.
\end{proof}

\begin{theorem}[Fubini]
If $f$ is continuous on $\closure{\Delta}=\closure{\Delta}'\times\closure{\Delta}''$,
then
\begin{subequations}
  \begin{align}
\int_{\closure{\Delta}}f &= \int_{\closure{\Delta}'}\int_{\closure{\Delta}''}f(\vec{x}',\vec{x}'')\\
&= \int_{\closure{\Delta}''}\int_{\closure{\Delta}'}f(\vec{x}',\vec{x}'')
  \end{align}
\end{subequations}
\end{theorem}

\begin{proof}
If $f$ is continuous on $\closure{\Delta}$, then it is bounded and $f$
is continuous with respect to $\vec{x}'$ when fixing $\vec{x}''$. Then
$f$ is Riemann integrable with respect to $\vec{x}''$. So by the
previous Theorem,
\begin{equation}
\int_{\closure{\Delta}}f=\int_{\closure{\Delta}''}\int_{\closure{\Delta}'}f(\vec{x}',\vec{x}''),
\end{equation}
and the same for the other order of integration.
\end{proof}

\begin{example}
Consider the sequence of functions defined on the unit interval $[0,1]$:
\begin{equation}
f_{n}(x) = \begin{cases}2n & \mbox{if }x\in[\frac{1}{2n},\frac{1}{n}]\\
0 & \mbox{otherwise}
\end{cases}
\end{equation}
Then we see that $\int^{1}_{0}f_{n}=1$, but $\lim_{n\to\infty}f_{n}=0$
is the constant function and $\int^{1}_{0}0=0$. This means
\begin{equation}
\lim_{n\to\infty}\int^{1}_{0}f_{n}\neq\int^{1}_{0}\lim_{n\to\infty}f_{n},
\end{equation}
which is\dots weird. When will we obtain an equality (i.e., when will
the limit of the integral be the integral of the limit)?
\end{example}

\begin{theorem}
If $f_{n}$ is a uniformly convergent sequence of Riemann integrable
functions in $\closure{\Delta}\subset\RR^{n}$, then the limit of $f$
is Riemann integrable and
\begin{equation}
\lim_{n\to\infty}\int_{\closure{\Delta}}f_{n}=\int_{\closure{\Delta}}\lim_{n\to\infty}f_{n}=\int_{\closure{\Delta}}f.
\end{equation}
\end{theorem}

\begin{proof}
\textsc{Claim 1:} $f$ is bounded.

\textsc{Claim 2:} $f$ is continuous almost everywhere.

\textsc{Claim 3:} By the first two claims, we see $f$ is integrable.
We see, by linearity,
\begin{subequations}
  \begin{align}
|\int_{\closure{\Delta}}f-\int_{\closure{\Delta}}f_{n}|&=|\int_{\closure{\Delta}}(f-f_{n})|\\
&\leq\int_{\closure{\Delta}}|f-f_{n}|.
  \end{align}
\end{subequations}
For any $\varepsilon>0$,
\begin{subequations}
  \begin{align}
\int_{\closure{\Delta}}|f-f_{n}| &
\leq\int_{\closure{\Delta}}\varepsilon \mbox{ for any }n\geq N\\
&\leq\varepsilon|\closure{\Delta}|.
  \end{align}
\end{subequations}
As $\varepsilon\to0$, we see $\int_{\closure{\Delta}}|f-f_{n}|\to0$
which proves the result.
\end{proof}