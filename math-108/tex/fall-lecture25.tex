%%
%% fall-lecture25.tex
%% 
%% Made by Alex Nelson <pqnelson@gmail.com>
%% Login   <alex@lisp>
%% 
%% Started on  2025-12-04T11:04:19-0800
%% Last update 2025-12-04T11:04:19-0800
%% 

\lecture{}

\begin{definition}
We say $f$ is \define{Riemann integrable} on a generic set
$S\subset\RR^{n}$ if $f$ is bounded and continuous almost everywhere
on $S$.
\end{definition}

\begin{definition}
If $f$ is Riemann integrable over $\closure{\Delta}$, then we define
the \define{Riemann integral} of $f$ is just the common value of the
upper and lower integrals, which we denote as
$\int_{\closure{\Delta}}f$ or something similar (if we need the
variable of integration, then $\int_{\closure{\Delta}}f(\vec{x})\,\D\vec{x}$).
\end{definition}

\begin{proposition}
Let $f_{1}$, $f_{2}$ be Riemann integrable functions on $S\propersubset\RR^{n}$.
Then $f_{1}+f_{2}$ and $f_{1}f_{2}$ are Riemann integrable on $S$.
\end{proposition}

\begin{proposition}
If $f_{1}$ and $f_{2}$ are Riemann integrable in $\closure{\Delta}\subset\RR^{n}$,
and if $f_{1}(\vec{x})>f_{2}(\vec{x})$ for all $\vec{x}\in\closure{\Delta}$,
then $\int_{\closure{\Delta}}f_{1}>\int_{\closure{\Delta}}f_{2}$.
\end{proposition}

\begin{definition}
A bounded set $D\subset\RR^{n}$ is called \define{Jordan measurable}
if its boundary $\boundary D=\closure{D}\setminus\Interior(D)$ is a
set of measure zero. 
\end{definition}

\begin{note}
The boundary $\boundary D=\closure{D}\cap(\Interior(D))^{\complement}$
is always a closed set. This means $\boundary D$ is closed and
bounded, hence $\boundary D$ is compact. Then $\boundary D$ has
measure zero iff it is content zero.
\end{note}

\begin{remark}
Most ``nice'' spaces which spring to mind are Jordan measurable. This
is a feature, not a defect, of the definition: we want to describe
domains of integration as Jordan measurable sets. The place to look
for non-examples of Jordan measurable sets are bizarre fractal-like
constructions or sets like $\QQ\cap(0,1)$.
\end{remark}

\begin{xca}
The Cantor set is defined recursively with $C_{0}=[0,1]$.
We will abuse notation to write $[a,b]/x = [a/x,b/x]$ and $k+[a,b]=[k+a,k+b]$.
Then
\begin{equation}
C_{n} = \frac{C_{n-1}}{3}\cup\left(\frac{2}{3}+\frac{C_{n-1}}{3}\right).
\end{equation}
The Cantor set is $C:=\lim_{n\to\infty}C_{n}=\bigcap^{\infty}_{n=0}C_{n}$.

Is the Cantor set Jordan measurable or not? (Hint: does the boundary
of the Cantor set $\boundary C$ have measure zero or not?)
\end{xca}

\begin{xca}
Consider the set $S=\QQ\cap[0,1]$. Is $S$ Jordan measurable or not?
(Hint: is $\boundary S=[0,1]$?)
\end{xca}

\begin{lemma}
If $D_{1}$ and $D_{2}$ are Jordan measurable sets,
then both $D_{1}\cap D_{2}$ and $D_{1}\cup D_{2}$ are Jordan measurable.
\end{lemma}

\begin{node}
For each Jordan measurable set $D\subset\RR^{n}$, there is a closed
cell $\closure{\Delta}$ such that $D\subset\closure{\Delta}$. A
Riemann integrable function
\begin{equation}
f\colon D\to\RR
\end{equation}
may be extended to a Riemann integrable function
\begin{equation}
\widetilde{f}\colon\closure{\Delta}\to\RR,
\end{equation}
defined by
\begin{equation}
\widetilde{f}(\vec{x}) =\begin{cases}f(\vec{x}) & \mbox{if }\vec{x}\in D\\
0 & \mbox{if }\vec{x}\in\closure{D}\setminus D.
\end{cases}
\end{equation}
We claim that $\widetilde{f}$ is Riemann integrable in
$\closure{\Delta}$. We see that $\widetilde{f}$ is bounded, but is it
continuous almost everywhere? Well, $\widetilde{f}$ is continuous in
$D$ and in $\closure{\Delta}\setminus D$, so the only possible
discontinuities would be in $\boundary D$ which has measure zero (by
Definition of Jordan measurable). Since
\begin{equation}
\closure{\Delta}=\Interior(D)\cup(\closure{\Delta}\setminus\closure{D})\cup\boundary D
\end{equation}
and $\widetilde{f}$ is continuous on the first two factors (and the
third factor is measure zero), this implies $\widetilde{f}$ is
continuous almost everywhere on $\closure{\Delta}$.

Now, we can define the \define{Riemann Integral} of $f$ over a Jordan
measurable subset $D\subset\RR^{n}$ by
\begin{equation}
\int_{D}f := \int_{\closure{\Delta}}\widetilde{f}.
\end{equation}
This is why we wanted to introduce Jordan measurable sets: we can
define Riemann integration over them.
\end{node}

\begin{definition}
Let $D$ be any set. We define its \define{Characteristic Function} to be
\begin{equation}
\chi_{D}(x) = \begin{cases}1 & \mbox{if }x\in D\\
0 & \mbox{otherwise}
\end{cases}
\end{equation}
\end{definition}

\begin{remark}
Not to be a stick in the mud, but technically we need $D\subset S$ and
the characteristic function is defined on $\chi_{D}\colon S\to\{0,1\}$.
\end{remark}

\begin{definition}
If $f$ is a Riemann integrable function on $\closure{\Delta}$, and if
$D\subset\closure{\Delta}$ is a Jordan measurable set, then we may
define the \define{Riemann integral} of $f$ over $D$ by
\begin{equation}
\int_{D}f=\int_{\closure{\Delta}}\chi_{D}f.
\end{equation}
\end{definition}

\begin{definition}
We may extend the notion of the ``content'' of a set to a Jordan
measurable set $D$, defining the \define{Content} of $D\subset\closure{\Delta}$ to be
\begin{equation}
|D|=\int_{\closure{\Delta}}\chi_{D}.
\end{equation}
When $D=\closure{\Delta}$, we recover the usual notion of the content
of a cell.
\end{definition}

\begin{proposition}[Linearity]
If $f_{1}$ and $f_{2}$ are Riemann integrable functions in a Jordan
measurable set $D$, then for any real constants $c_{1},c_{2}\in\RR$ we
have
\begin{equation}
\int_{D}(c_{1}f_{1}+c_{2}f_{2})=(c_{1}\int_{D}f_{1})+(c_{2}\int_{D}f_{2}),
\end{equation}
and $c_{1}f_{1}+c_{2}f_{2}$ is Riemann integrable on $D$.
\end{proposition}

\begin{proposition}[Positivity]
Let $f$ and $g$ be Riemann integrable functions on a Jordan measurable
set $D$. Then the following hold:
\begin{enumerate}
\item If for each $\vec{x}\in D$ we have $f(\vec{x})\geq0$, then $\int_{D}f\geq0$;
\item If for each $\vec{x}\in D$ we have $f(\vec{x})\geq g(\vec{x})$,
  then $\int_{D}f\geq\int_{D}g$;
\item $|\int_{D}f|\leq\int_{D}|f|$.
\end{enumerate}
\end{proposition}

\begin{lemma}
If $f$ is bounded in a Jordan measurable set $D\subset\RR^{n}$ and if
$f(\vec{x})=0$ for all $\vec{x}\in D\setminus E$ where $E\subset D$
has \emph{content} zero, then $f$ is Riemann integrable on $D$ and
\begin{equation}
\int_{D}f=0.
\end{equation}
\end{lemma}

\begin{proof}
We place $D$ in a closed cell $\closure{\Delta}\supset D$ since $D$ is
bounded. Define
\begin{equation}
\widetilde{f}=\chi_{D}f\colon\closure{\Delta}\to\RR,
\end{equation}
we see $\widetilde{f}=0$ on $\closure{\Delta}\setminus E$.
Then $\widetilde{f}$ is Riemann integrable in $\closure{\Delta}$ since
$E$ has content zero.
Byt definition of content zero, for each $\varepsilon>0$ there exists
finitely many $\Delta'_{i}$ such that
\begin{subequations}
\begin{equation}
E\subset\bigcup^{n}_{i=1}\Delta'_{i}
\end{equation}
and
\begin{equation}
\sum_{i}|\closure{\Delta}'_{i}|<\varepsilon.
\end{equation}
\end{subequations}
We want to find a partition $P$ that contains these $\Delta'_{i}$, and
we'll denote by $\Delta''_{j}$ the remaining cells of the partition.
Since $f$ is bounded on $E$, there exists an $M>0$ such that for all
$\vec{x}\in E$,
\begin{equation}
|f(\vec{x})|\leq M.
\end{equation}
Then we can start to compute the integral by looking at the upper
Riemann sum,
\begin{subequations}
  \begin{align}
S^{*}(\widetilde{f},P)
&=\sum_{i}\left(\sup_{\vec{x}\in\Delta'_{i}}f(\vec{x})\right)|\closure{\Delta}'_{i}|+\sum_{j}\left(\sup_{\vec{x}\in\Delta''_{j}}f(\vec{x})\right)|\closure{\Delta}''_{j}|\\
&\leq M\sum_{i}|\closure{\Delta}'_{i}| + 0 \mbox{ since $f(\vec{x})=0$ on $\Delta''_{j}$}\\
&<M\varepsilon.
  \end{align}
\end{subequations}
Since $\varepsilon$ is arbitrary, $S^{*}(\widetilde{f},P)$ is an upper
bound of the upper integral, so
\begin{equation}
\int_{D}\widetilde{f}\leq\int^{*}_{\closure{\Delta}}\widetilde{f}\leq0.
\end{equation}
A similar argument gives the lower bound
\begin{equation}
0\leq\int_{*\closure{\Delta}}\widetilde{f}\leq\int_{D}\widetilde{f},
\end{equation}
hence the result.
\end{proof}

\begin{theorem}
If $f$ is Riemann integrable over a Jordan measurable set $D$,
and if $g$ is a bounded function on $D$, and if
$g(\vec{x})=f(\vec{x})$ at all points $\vec{x}\in D\setminus E$ where
$E\subset D$ is content zero, then $g$ is Riemann integrable on $D$
and
\begin{equation}
\int_{D}g=\int_{D}f.
\end{equation}
\end{theorem}

\begin{proof}
Apply the previous lemma to $f-g$.
\end{proof}

\begin{theorem}
Let $D_{1}$, $D_{2}$ be Jordan measurable sets, and $E=D_{1}\cap D_{2}$
have content zero. If $f$ is Riemann integrable on $D_{1}\cup D_{2}$,
then
\begin{equation}
\int_{D_{1}\cup D_{2}}f=\int_{D_{1}}f+\int_{D_{2}}f.
\end{equation}
In particular, if $f=1$, then $|D_{1}\cup D_{2}|=|D_{1}|+|D_{2}|$.
\end{theorem}

\begin{proof}[Proof sketch]
$\chi_{D_{1}\cup D_{2}}=\chi_{D_{1}}+\chi_{D_{2}}-\chi_{E}$ but
  $\int_{D_{1}\cup D_{2}}\chi_{E}=0$.
\end{proof}