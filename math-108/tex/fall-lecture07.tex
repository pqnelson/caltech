%%
%% fall-lecture07.tex
%% 
%% Made by Alex Nelson <pqnelson@gmail.com>
%% Login   <alex@lisp>
%% 
%% Started on  2025-10-14T14:16:26-0700
%% Last update 2025-10-14T14:16:26-0700
%% 

\lecture[Convergence of sequences]{}

\begin{definition}
A sequence $\{x_{n}\}$ in a metric space $(X,d)$ \define{Converges} to
$x$ if for each $\varepsilon>0$ there is an $N\in\NN$ such that
$d(x,x_{n})<\varepsilon$ for all $n\geq N$. We write $x_{n}\to x$ or
$\lim_{n\to\infty}x_{n}=x$ and we call $x$ the \define{Limit} of $x$.
\end{definition}

\begin{remark}
Check this against the familiar notion of convergence of sequences of
real numbers. When we take $(X,d)=(\RR,|\cdot|)$ we find this recovers
what we learned in calculus.
\end{remark}

\begin{question}
What happens if a sequence converges to, say, 2 points --- is it even possible?
\end{question}

\begin{definition}[Solution]
If $x_{n}\to u$ and $x_{n}\to v$, then $d(u,v)>0$ if $u\neq v$.
Set $r=\frac{1}{2}d(u,v)$. Then every $\{x_{n}\mid n\geq N\}\subset B_{r}(u)$
--- which we know to be true by the definition of convergence.

Similarly, we would find $\{x_{n}\mid n\geq N\}\subset B_{r}(v)$. This means
\begin{equation}
\{x_{n}\mid n\geq N\}\subset B_{r}(u)\cap B_{r}(v)=\emptyset,
\end{equation}
which means we arrive a contradiction. Hence if a sequence converges
at all, then it converges to at most 1 value.
\end{definition}

\begin{definition}[Cartesian product of metric spaces]
Let $(X_{1},d_{1})$ and $(X_{2},d_{2})$ be two metric spaces.
We define the \define{Product metric space} $(X,d)$ to consist of
$X=X_{1}\times X_{2}$, with
\begin{equation}
d\bigl((x_{1},x_{2}),(y_{1},y_{2})\bigr)=d_{1}(x_{1},y_{1})+d_{2}(x_{2},y_{2}).
\end{equation}
It's trivial to check $d$ is, in fact, a metric.
\end{definition}

\begin{remark}
If $(x^{(1)}_{n})$ is a sequence in $(X_{1},d_{1})$ which converges to
$x^{(1)}_{n}\to x^{(1)}$, and if 
$(x^{(2)}_{n})$ is a sequence in $(X_{2},d_{2})$ which converges to
$x^{(2)}_{n}\to x^{(2)}$, then
$(x^{(1)}_{n},x^{(2)}_{n})\to(x^{(1)},x^{(2)})$ converges in the
product metric space $(X_{1}\times X_{2},d)$.
\end{remark}

\begin{proposition}\label{prop:fall-lec07:subset-closed-iff-contains-all-its-limit-points}
A subset $E$ is closed in $X$ iff for any sequence $\{x_{n}\}\subset E$
and $x_{n}\to x$ in $X$ we see that the limit $x\in E$.
\end{proposition}

\begin{proof}
Recall, $E$ is closed if and only if it contains all its limit points.
It suffices to show that a point $x\in X\setminus E$ is a limit point of $E$ iff
it is the limit of a sequence in $E$.

$(\Longrightarrow)$ Assume $x\in X$ is a limit point of $E$.
Then every neighborhood of $x\in N_{x}$ intersects $E$
nontrivially. We can take a sequence of neighborhoods. For each $n$,
we take $B_{1/n}(x)\cap E$ at a point $x_{n}\neq x$ (by definition of
a limit point). Then $x_{n}\to x$ in $E$, since
\begin{equation}
d(x,x_{n})<\frac{1}{n},
\end{equation}
so then we pick $N\geq1/\varepsilon$.

$(\Longleftarrow)$ Given a sequence $\{x_{n}\}\subset E$ which
converges to $x_{n}\to x\in E$, then the sequence eventually enters an
$\varepsilon$-ball of $x$ for any $\varepsilon>0$. Then $B_{\varepsilon}(x)$
contains infinitely many terms of $\{x_{n}\}$ and so
$B_{\varepsilon}(x)\cap E$ is nontrivial. Hence $x$ is a limit point
of $E$.
\end{proof}

\begin{remark}
If we had instead said, ``It suffices to show that a point $x\in X$ is a limit point of $E$ iff it is the limit of a sequence in $E$'',
then the $(\Longleftarrow)$ direction proof may be ``buggy'': if $x_{n}=x$
for all $n$ (i.e., we have ``the constant sequence''), then the
argument breaks down.

It seems the remedy is to prove that $B_{\varepsilon}(x)$ contains
infinitely many points of $E$, but it suffices to prove
$B_{\varepsilon}(x)\cap E$ contains at least two distinct points. This
is a fact topologically for $T_{1}$ spaces (like metric spaces).
\end{remark}

\begin{definition}
We call a sequence $(x_{n})$ in $(X,d)$ a \define{Cauchy} sequence if
for each $\varepsilon>0$ there exists an $N\in\NN$ such that for all
$m\geq N$ and $n\geq N$ we have $d(x_{m},x_{n})<\varepsilon$.
\end{definition}

\begin{proposition}
Any convergent sequence is Cauchy.
\end{proposition}

\begin{proof}
Let $x_{n}\to x$ be a convergent sequence in $(X,d)$.
Then for $\varepsilon>0$ there exists a $N\in\NN$ such that 
(1) for all $n\geq N$ we have $d(x,x_{n})<\varepsilon/2$ and
(2) for all $m\geq N$ we have $d(x,x_{m})<\varepsilon/2$.
Then by the triangle inequality,
\begin{equation}
d(x_{m},x_{n})\leq d(x_{m},x)+d(x,x_{n})<\frac{\varepsilon}{2}+\frac{\varepsilon}{2}=\varepsilon,
\end{equation}
hence the result.
\end{proof}

\begin{example}
Consider $\bigl((0,1], |\cdot|\bigr)$. The sequence $x_{n}=1/n$ is a
Cauchy sequence in $(0,1]$ but it does not converge to any element
of $(0,1]$.
\end{example}

\begin{example}
Consider $(\QQ,|\cdot|)$. The sequence $x_{n}$ given by the decimal
expansion of $\sqrt{2}$ (truncated to $n+1$ digits) converges to
$x_{n}\to\sqrt{2}$ but famously $\sqrt{2}\notin\QQ$.
\end{example}

\begin{remark}
These examples show there are some ``gaps'' in these metric spaces.
When ``there are no gaps'', Cauchy sequences are convergent. Let us
make this vague intuition more precise.
\end{remark}

\begin{definition}
We call a metric space $(X,d)$ \define{Complete} if every Cauchy
sequence in $X$ converges (in $X$).
\end{definition}

\begin{example}
We see $(\RR^{n}, \|\cdot\|)$ is a complete metric space.
\end{example}

\begin{proposition}\label{prop:fall-lec07:subspace-complete-iff-closed}
Let $(X,d)$ be complete, and $E\subset X$.
Then the subspace $(E,d)$ is complete if and only if $E$ is closed in $X$.
\end{proposition}

\begin{proof}
$(\Longrightarrow)$ Assume $(E,d)$ is complete.
We claim: every convergent sequence $(x_{n})\subset E$ converges to
$x_{n}\to x\in E$.
\begin{proof}[Subproof]
Let $(x_{n})\subset E$ be a convergent sequence of terms in $E$.
Then $x_{n}\to x$ converges to a value $x\in X$.
But $(x_{n})$ is Cauchy (as a sequence in $E$) and $(E,d)$ is complete
by assumption. Hence $x\in E$ by definition of ``complete'' metric spaces.
\end{proof}
Then $E$ contains all its limit points. Hence $E$ is a closed subset
of $X$.

$(\Longleftarrow)$ Assume $E$ is closed in $X$. Let $(x_{n})$ be a
Cauchy sequence in $E$. Then $(x_{n})$ is a Cauchy sequence in $X$.
Then $x_{n}\to x$ where $x\in X$ since $(X,d)$ is complete (and
definition of ``complete'' metric spaces). Then, since $E$ is closed
in $X$ by assumption, we see every convergent sequence's limit is in
$E$. So then we see $x\in E$ by Proposition~\ref{prop:fall-lec07:subset-closed-iff-contains-all-its-limit-points}.
Then every Cauchy sequence converges in $E$. Hence $E$ is complete by
definition of ``complete'' metric spaces.
\end{proof}

\begin{definition}
Let $E\subset X$. We define the \define{Diameter} of $E$ to be the number
\begin{equation}
\diam(E):=\sup\{d(x,y)\mid x,y\in E\}.
\end{equation}
\end{definition}

\begin{definition}
Let $\{E_{n}\}$ be a descending sequence of nonempty subsets of
$(X,d)$ --- so $E_{m}\subset E_{n}$ if $m\geq n$. We say $\{E_{n}\}$
is a \define{Contracting Sequence} if $\diam(E_{n})\to 0$ as $n\to\infty$.
\end{definition}

\begin{theorem}[Cantor's Intersection Theorem]
Let $(X,d)$ be a metric space.
Then $(X,d)$ is complete iff for any contracting sequence $\{E_{n}\}$
of nonempty closed subsets of $X$, there exists a point $x\in X$ such
that
\begin{equation*}
\bigcap^{\infty}_{n=0}E_{n}=\{x\}.
\end{equation*}
\end{theorem}
We see that there cannot be two points in the intersection, because
$\diam(E_{n})\to 0$ means these two points must be the same.

\begin{proof}
$(\Longrightarrow)$ Assume $(X,d)$ is complete. (We should immediately
think of ``Cauchy sequences''.)
We want to show every contracting sequence $\{E_{n}\}$ has a singleton
be its intersection.
Let $\{E_{n}\}$ be a contracting sequence of closed nonempty subsets
of $X$.
For each $n$, take $x_{n}\in E_{n}$. We claim $x_{n}$ is Cauchy (which
implies $x_{n}\to x$ by completeness).
Let $\varepsilon>0$, there exists an $N\in\NN$ such that
$d(\diam(E_{n}),0)<\varepsilon$ for all $n\geq N$. Since $E_{n}$ is
descending, we see $x_{m}\in E_{N}$ and $x_{n}\in E_{N}$ so
$d(x_{m},x_{n})\leq\diam(E_{N})<\varepsilon$. Then $x_{n}$ is Cauchy,
hence $x_{n}\to x$ by completeness.

Since each $E_{n}$ is closed and the sequence we have selected has
$x_{k}\in E_{n}$ for all $k\geq n$, so $\{x_{k}\mid k\geq n\}\subset E_{n}$.
Then by Proposition~\ref{prop:fall-lec07:subspace-complete-iff-closed}, we have $x\in E_{n}$. Therefore
\begin{equation}
x\in\bigcap^{\infty}_{n=0}E_{n},
\end{equation}
and it is the only point there. Hence
\begin{equation}
\bigcap^{\infty}_{n=0}E_{n}=\{x\}.
\end{equation}
This is the forward direction.
\end{proof}