%%
%% winter-lecture11.tex
%% 
%% Made by Alex Nelson <pqnelson@gmail.com>
%% Login   <alex@lisp>
%% 
%% Started on  2026-01-29T10:13:45-0800
%% Last update 2026-01-29T10:13:45-0800
%% 

\lecture{}

\begin{recall}
A function $f$ is a measurable function if for each $a\in\RR$, $\{f<a\}$
is a measurable set.
\end{recall}

\begin{definition}[Lebesgue integral for non-negative function]
Let $f\colon\RR^{d}\to[0,+\infty]$ be a measurable function. We define
its \define{Integral} to be
\begin{equation}
\int f(x)\,\D x=\sup\{\int\varphi(x)\,\D x\mid \varphi\mbox{ is simple},0\leq\varphi\leq f\}.
\end{equation}
\end{definition}

\begin{remark}
This is the ``inner measure'' philosophy approximating the area of the
graphy by stacking simple blocks below it.
\end{remark}

\begin{proposition}[Basic properties]
Let $f,g\colon\RR^{d}\to[0,+\infty]$ be measurable. The following all hold:
\begin{enumerate}
\item\textsc{Monotonicity:} If $f\leq g$, then $\int f(x)\,\D x\leq\int g(x)\,\D x$
\item\textsc{Homogeneity:} If $c\in\RR$ and $c>0$, then $\int cf(x)\,\D x=c\int f(x)\,\D x$
\item\textsc{Chebyshev's Inequality:} For any $\alpha>0$,
$m(\{f\geq\alpha\})\leq\frac{1}{\alpha}\int f(x)\,\D x$.
\end{enumerate}
\end{proposition}

\begin{proof}
\begin{enumerate}
\item Let 
\begin{subequations}
  \begin{align}
\Phi_{f}&=\{\mbox{simple functions }\varphi\mid0\leq\varphi\leq f\}\\
\intertext{and}
\Phi_{g}&=\{\mbox{simple functions }\varphi\mid0\leq\varphi\leq g\}.
  \end{align}
\end{subequations}
If $f(x)\leq g(x)$ for all $x$, then any $\varphi\leq f$ also
satisfies $\varphi\leq g$. So $\Phi_{f}\subset\Phi_{g}$. Then
\begin{equation}
\sup\Phi_{f}\leq\sup\Phi_{g},
\end{equation}
but
\begin{equation}
\int f(x)\,\D x=\sup_{\varphi\in\Phi_{f}}\int\varphi\leq\sup_{\varphi\in\Phi_{g}}\int\varphi=\int g(x)\,\D x.
\end{equation}
Hence the first claim.
\item Let $c>0$.

\textsc{Claim 1:} $\int cf(x)\,\D x\geq c\int f(x)\,\D x$.
Let $\varphi$ be simple and $0\leq\varphi\leq f$. Then $c\varphi$ is
simple and moreover $0\leq c\varphi\leq cf$. By linearity we can
conclude $\int(c\varphi)=c\int\varphi$. Hence $\int(c\varphi)\geq c\int\varphi$.
Then by taking the supremum over all such $\varphi$ gives us $\int cf\geq c\int f$.

\textsc{Claim 2:} $\int cf(x)\,\D x\leq c\int f(x)\,\D x$.
Let $\psi$ be simple and $0\leq\psi\leq cf$.
Dividing by $c$, $0\leq c^{-1}\psi\leq f$, and again $c^{-1}\psi$ is simple.
Denote by $\varphi=c^{-1}\psi$, we have $0\leq\varphi\leq f$. Then we have
\begin{equation}
\int\psi=c\int\varphi\leq c\int f(x)\,\D x.
\end{equation}
Taking the supremum on both sides gives us
\begin{equation}
\sup_{\psi}\int\psi=\int(cf(x))\,\D x\leq c\int f(x)\,\D x.
\end{equation}
Hence the claim.
\item Let $E_{\alpha}=\{f\geq\alpha\}$. This is measurable, since $f$
  is measurable. We define $\varphi=\alpha\chi_{E_{\alpha}}$. Then $0\leq\varphi$
since $\alpha>0$ and $\chi_{E_{\alpha}}\geq0$. We also have
$\varphi\leq f$ by definition of $\varphi$. So $0\leq\varphi\leq f$.
Then by monotonicity $\int\varphi\leq\int f$ and by definition
$\int\varphi=\alpha m(E_{\alpha})$, which gives us the desired result.
\end{enumerate}
\end{proof}

\begin{recall}
We had some difficulty with taking integrals of sequences of functions.
It's not always true that limit of the integral is the integral of
limits. A couple examples should spring to mind:
\begin{enumerate}
\item $f_{n}=\chi_{[n,n+1]}$. Then $f_{n}(x)\to0$ for all $x$, but
  $\int f_{n}(x)\,\D x=1$ for all $n$. So clearly we have problems
  since
  \begin{equation}
1=\lim_{n\to\infty}\int f_{n}(x)\,\D
x\neq\int\lim_{n\to\infty}f_{n}(x)\,\D x=0,
  \end{equation}
  which is problematic.
\item Consider
  \begin{equation}
f_{n}(x) = \begin{cases}n\chi_{(0,1/n]}(x) & \mbox{if }x\neq0\\
0 & \mbox{if }x=0
\end{cases}
  \end{equation}
  Then $f_{n}(x)\to f(x)$ where $f(x)=0$. But $\int f_{n}(x)\,\D x=1$,
  despite $\int f(x)\,\D x=0$.
\end{enumerate}
\end{recall}

\begin{monotone-convergence-theorem}
Let $(f_{n})$ be a sequence of measurable functions (for simplicity, $f_{n}\colon\RR^{d}\to[0,+\infty]$)
such that
\begin{enumerate}
\item\textsc{Monotone:} $0\leq f_{n}(x)\leq f_{n+1}(x)$ for almost
  every $x$, and
\item\textsc{Convergence:} $f_{n}(x)\to f(x)$ pointwise almost everywhere.
\end{enumerate}
Then $\lim_{n\to\infty}\int f_{n}(x)\,\D x=\int f(x)\,\D x$.
\end{monotone-convergence-theorem}

The monotone convergence theorem holds more generally than just for
functions on $\RR^{d}$.

\begin{proof}
\textsc{Claim 1:} $\lim_{n\to\infty}\int f_{n}(x)\,\D x\leq\int f(x)\,\D x$.
Since $\int f_{n}\leq\int f$ for each $n$, the result follows.

\textsc{Claim 1:} $\lim_{n\to\infty}\int f_{n}(x)\,\D x\geq\int f(x)\,\D x$.
Let $\varphi$ be any simple function such that $0\leq\varphi\leq f$, let
$\alpha\in\RR$ be such that $0<\alpha<1$. Consider
\begin{equation}
E_{n}=\{x\in\RR^{d}\mid f_{n}(x)\geq\alpha\varphi(x)\}.
\end{equation}
Then
\begin{equation}
E_{n}\subset E_{n+1},
\end{equation}
since $f_{n}\leq f_{n+1}$ almost everywhere. Also Since $f_{n}\to f$
pointwise and $\alpha<1$,
\begin{equation}
\bigcup^{\infty}_{n=1}E_{n}=\{x\in\RR^{d}\mid\varphi(x)>0\}.
\end{equation}
If $\varphi(x)>0$ then $\alpha\varphi(x)<\varphi(x)\leq f(x)$, and
eventually $f_{n}(x)$ surpasses $\alpha\varphi(x)$.

Now, we integrate over $E_{n}$:
\begin{equation}
\int_{\RR^{d}}f_{n}\geq\int_{E_{n}}f_{n}\geq\int_{E_{n}}\alpha\varphi=\alpha\int_{E_{n}}\varphi
\end{equation}
where the last equality is by linearity of integrating simple
functions. Writing
\begin{equation}
\varphi=\sum_{j}c_{j}\chi_{A_{j}},
\end{equation}
we have
\begin{equation}
\int_{E_{n}}\varphi=\sum_{j}c_{j}m(A_{j}\cap E_{n}).
\end{equation}
By continuity of measures, $m(A_{j}\cap E_{n})\to m(A_{j})$. Thus
\begin{equation}
\lim_{n\to\infty}\int_{\RR^{d}}f_{n}\geq\alpha\int\varphi
\end{equation}
for any $\alpha<1$. Take the limit $\alpha\to 1$, we get
\begin{equation}
\lim_{n\to\infty}\int f_{n}\geq\int\varphi.
\end{equation}
This holds for \emph{any} simple $\varphi\leq f_{n}$. Then
\begin{equation}
\lim_{n\to\infty}\int f_{n}\geq\sup\{\int\varphi\mid\varphi\leq f\}=\int f(x)\,\D x.
\end{equation}
Hence the second claim, and the result follows immediately.
\end{proof}

\begin{xca}
Review this proof. Where is monotonicity used?
\end{xca}