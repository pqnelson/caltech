%%
%% winter-lecture13.tex
%% 
%% Made by Alex Nelson <pqnelson@gmail.com>
%% Login   <alex@lisp>
%% 
%% Started on  2026-02-03T10:38:52-0800
%% Last update 2026-02-03T10:38:52-0800
%% 

\lecture[Completeness of L1]{}

% Stein, Sharkarchi, Ch. 2, \S2, Theorem 2.2

\begin{node}
We had asserted that the function $\|-\|\colon L^{1}(\RR^{d})\to\RR$,
defined as
\begin{equation}
\|f\|=\int|f|,
\end{equation}
was a norm. But we never proved it. We should prove it now. What needs
to be done?
\begin{enumerate}[start=0]
\item $L^{1}(\RR^{d})$ is a vector space over $\RR$
\item Positive-definiteness: for any $f\in L^{1}$, $\|f\|\geq0$ and $\|f\|=0$ iff $f=0$
\item Absolute homogeneity: for any $f\in L^{1}$ and $c\in\RR$, $\|cf\|=|c|\cdot\|f\|$
\item Triangle inequality: for any $f,g\in L^{1}$, we have $\|f+g\|\leq\|f\|+\|g\|$.
\end{enumerate}
\end{node}

\begin{remark}
We're going to get bogged down $\|f\|=0$ iff $f=0$, because it's not
true for any $f\in L^{1}$. Well, $f=0$ implies $\|f\|=0$ is true. But
$\|f\|=0$ is equivalent to $\int|f|=0$, but that can happen for $f$
which is nonzero on a null set (e.g., the indicator function for a
singleton). For such $f$, they are not identically zero, merely ``zero
almost everywhere''. What do we do?

The remedy is to define $L^{1}$ as the equivalence classes of
functions $f\sim g$ iff $f=g$ almost everywhere. Then writing
$[f]_{\sim}$ for the equivalence class of functions $g\sim f$, we see
that $L^{1}$ consists of all such equivalence classes of functions $f$
such that $\int|f|<\infty$:
\begin{equation}
L^{1}(X)=\{[f]_{\sim}\mid f\colon X\to\RR, \int|f|<\infty\}.
\end{equation}
As per usual mathematical practice, we abuse notation and write $f\in L^{1}(X)$
to mean $[f]_{\sim}\in L^{1}(X)$ and we use $f$ as its representative element.
\end{remark}

\begin{theorem}
The space $L^{1}(\RR^{d})$ is complete in its metric: if $(f_{n})$ is
a Cauchy sequence of elements of $L^{1}(\RR^{d})$, then
\begin{enumerate}
\item there exists an $f\in L^{1}(\RR^{d})$ such that $f_{n}\to f$
  converges in norm (i.e., $\|f_{n}-f\|\to0$); and
\item furthermore, there exists a subsequence of $(f_{n})$ which
  converges pointwise almost everywhere.
\end{enumerate}
\end{theorem}

Remember: by ``subsequence'', we mean there is a strictly increasing
sequence of integers $n_{1}<n_{2}<\dots$ which index the subsequence
$(f_{n_{k}})$ of $(f_{n})$. (We do not mean: another sequence with
values in $\rng(f_{n})=\{g\mid\exists n\ldotp f_{n}=g\}$.)

\begin{proof}
\textsc{Step 1}. Since $f_{n}$ is Cauchy, we choose indices $n_{1}<n_{2}<\dots$
such that
\begin{equation}
\|f_{n_{k+1}}-f_{n_{k}}\|<\frac{1}{2^{k}}.
\end{equation}
This follows by definition of Cauchy sequences. We define
\begin{equation}
f(x):=f_{n_{1}}(x)+\sum^{\infty}_{k=1}\bigl(f_{n_{k+1}}(x)-f_{n_{k}}(x)\bigr).
\end{equation}
The partial sums are $s_{k}=f_{n_{k}}$, so we just need to prove
convergence.

\textsc{Step 2: Convergence}.
Define
\begin{equation}
g(x) := |f_{n_{1}}(x)| + \sum^{\infty}_{k=1}\bigl|f_{n_{k+1}}(x)-f_{n_{k}}(x)\bigr|.
\end{equation}
We see since the partial sums $s_{N}$ are monotone,
\begin{subequations}
  \begin{align}
\int g &=\int|f_{n_{1}}| + \int\sum^{\infty}_{k=1}\bigl|f_{n_{k+1}}(x)-f_{n_{k}}(x)\bigr|\\
&=\int|f_{n_{1}}| + \sum^{\infty}_{k=1}\int\bigl|f_{n_{k+1}}(x)-f_{n_{k}}(x)\bigr|\\
&\leq(\int|f_{n_{1}}|) + \sum^{\infty}_{k=1}2^{-k}=(\int|f_{n_{1}}|)+1\\
&<\infty
  \end{align}
\end{subequations}
Then the series
$\sum^{\infty}_{k=1}\bigl|f_{n_{k+1}}(x)-f_{n_{k}}(x)\bigr|$ is finite
almost everywhere, so therefore $f(x)$ converges absolutely almost everywhere.

\textsc{Step 3: Norm convergence}. We see $f_{n_{k}}\to f$ pointwise
almost everywhere. We see that $|f|\leq g$ and also $g\in L^{1}$, so
we find $f\in L^{1}$ as well. We claim
\begin{equation}
\lim_{k\to\infty}\|f_{n_{k}}-f\|=0.
\end{equation}
We see, by the dominated convergence theorem (and that
$|f_{n_{k}}|\leq g$),
\begin{equation}
\lim_{k\to\infty}\|f_{n_{k}}-f\|=\lim_{k\to\infty}\int|f_{n_{k}}-f|=0
\end{equation}
We see by the triangle inequality
\begin{subequations}
  \begin{align}
\|f_{n}-f\| &\leq\|f_{n}-f_{n_{k}}\|+\|f_{n_{k}}-f\|\\
&\leq \|f_{n}-f_{n_{k}}\|+0\quad\mbox{since we just proved this}\\
&\leq 0+0\quad\mbox{since $(f_{n})$ Cauchy},
  \end{align}
\end{subequations}
which proves the claim.
\end{proof}

\begin{xca}
Is there a Cauchy sequence ``in norm'' $(f_{n})$ in $L^{1}$ such that
$\int|f_{n}-f|\to0$ but not all subsequences of $(f_{n})$ converges
pointwise to $f$? (Specifically, there exists at least one subsequence
of $(f_{n})$ which does not converge pointwise to $f$?)
\end{xca}

% One example which springs to mind: $f_{n}=g_{n}+f$ where $g_{n}=0$
% for composite $n$ and $g_{p}\neq0$ is measure zero for prime
% $p$. Then $(f_{n_{k}})=(g_{p})$ for all primes $p$. Clearly
% $g_{p}\not\to f$ pointwise.

\begin{node}
Not all $L^{1}$ functions are nice and continuous. They can be pretty
wild and ornery. However, there is a dense subset of $L^{1}$
consisting of ``nice continuous functions''. Let us make this claim
more precise in the following theorem.
\end{node}

\begin{theorem}[Density]
The set of continuous functions with compact support is dense in
$L^{1}$. That is to say, for any $f\in L^{1}$ and for any
$\varepsilon>0$, there exists a $g\in L^{1}$ continuous and with
compact support such that $\|f-g\|_{L^{1}}<\varepsilon$.
\end{theorem}