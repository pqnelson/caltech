%%
%% winter-lecture12.tex
%% 
%% Made by Alex Nelson <pqnelson@gmail.com>
%% Login   <alex@lisp>
%% 
%% Started on  2026-01-31T11:04:22-0800
%% Last update 2026-01-31T11:04:22-0800
%% 

\lecture{}

\begin{corollary}
If $f\geq0$ and $g\geq0$, then $\int(f+g)=(\int f)+(\int g)$.
\end{corollary}

\begin{proof}
We had a sequence of simple functions $\varphi_{n}\nearrow f$ (recall,
which means for all $n$ we have $0\leq\varphi_{n}\leq\varphi_{n+1}$
and pointwise {a.e.} $\varphi_{n}\to f$) and $\psi_{n}\nearrow g$.
But also $(\varphi_{n}+\psi_{n})$ is an increasing sequence of simple functions 
converging to $f+g$. Then we can apply the Monotone Convergence
Theorem to $(\varphi_{n}+\psi_{n})$, but it is well known linearity
holds for simple functions, so we have
\begin{subequations}
  \begin{align}
\lim_{n\to\infty}\int(\varphi_{n}+\psi_{n}) &= \int\lim(\varphi_{n}+\psi_{n})=\int(f+g)\\
&=\lim\bigl((\int\varphi_{n})+(\int\psi_{n})\bigr)=(\lim\int\varphi_{n})+(\lim\int\psi_{n})=(\int f) + (\int g).
  \end{align}
\end{subequations}
Hence the result.
\end{proof}

\subsection{Non-Monotone Sequences}

\begin{recall}
We will use notions of ``lim inf'' and ``lim sup'' for sequences $(a_{n})$. Let us recall
their definitions
\begin{equation}
\liminf_{n\to\infty}a_{n}=\lim_{n\to\infty}\inf_{k\geq n}a_{k}
\end{equation}
and
\begin{equation}
\limsup_{n\to\infty}a_{n}=\lim_{n\to\infty}\sup_{k\geq n}a_{k}.
\end{equation}
This works on any sort of sequence (e.g., a sequence of measurable functions).

Also recall, the lim-inf and lim-sup are always well-defined. When
they are equal, the limit exists and equals the common value. And
conversely, when the limit exists, both the lim-inf and lim-sup are
equal to the limit.
\end{recall}

\begin{node}
Recall we had $f_{n}=\chi_{[n,n+1]}$ which converged pointwise to
$f_{n}\to0$, but $\int f_{n}=1$. So ``mass escapes to infinity''. This
is a general phenomena. Fatou's lemma says mass can be lost but not created.
\end{node}

\begin{lemma}[Fatou]
If $(f_{n})$ is a sequence of non-negative measurable functions, then
\begin{equation*}
\int\liminf_{n\to\infty}f_{n}\leq\liminf_{n\to\infty}\int f_{n}.
\end{equation*}
\end{lemma}

\begin{proof}
We define a function
\begin{equation}
g_{n}(x):=\inf_{k\geq n}f_{k}(x).
\end{equation}
Then clearly $g_{n}(x)\leq f_{n}(x)$ for all $n$, and $g_{n}$ is an
increasing sequence, so $g_{n}\nearrow\liminf_{n}f_{n}$. By the
Monotone Convergence Theorem,
\begin{subequations}
  \begin{align}
\int\liminf_{n\to\infty}f_{n} &= \int\lim_{n\to\infty}g_{n}\\
&= \lim\int g_{n} \mbox{by MCT}
  \end{align}
\end{subequations}
Since $g_{n}\leq f_{n}$, we have
\begin{equation}\label{eq:proof:fatou-lemma:star}
\int g_{n}\leq\int f_{n}
\end{equation}
by monotonicity of integral for non-negative functions. Then (because
the limit exists),
\begin{equation}
\lim\int g_{n}=\liminf\int g_{n},
\end{equation}
and then Equation~\eqref{eq:proof:fatou-lemma:star} implies
\begin{equation}
\liminf\int g_{n}\leq\liminf\int f_{n}.
\end{equation}
Hence the result.
\end{proof}

% Starsky and Shenarki, Chapter 2, \S1.4

\begin{definition}
A measurable function $f\colon\RR^{d}\to\RR$ is \define{Integrable} if $\int|f|<+\infty$.
\end{definition}

\begin{remark}
Recall, $|x|$ is a continuous {a.e.} function, and the composition of
a measurable function followed by a continuous {a.e.} function is a
measurable function. So this ``makes sense'', i.e., $|f|$ is measurable.
\end{remark}

\begin{definition}
If $f\colon\RR^{d}\to\RR$ is integrable, then we can write it as
$f=f^{+}-f^{-}$ where $f^{+}(x)=\max(f(x),0)$ and $f^{-}(x)=\max(-f(x),0)$.
Then the \define{Integral} of $f$ is defined to be
\begin{equation}
\int f(x)\,\D x:=\int f^{+}(x)\,\D x-\int f^{-}(x)\,\D x.
\end{equation}
Since $|f|=f^{+}+f^{-}$, by hypothesis this is finite-valued, both $f^{+}$
and $f^{-}$ are finite-valued.
\end{definition}

\begin{notation}
The space of integrable functions is denoted $L^{1}(\RR^{d})$. More
generally, if $X$ is the domain of integrable functions, then
$L^{1}(X)$ is the space of integrable functions defined on $X$.
\end{notation}

\begin{proposition}
$L^{1}(\RR^{d})$ is a vector space, and for any $f,g\in L^{1}(\RR^{d})$
and for any $\alpha,\beta\in\RR$ we have
\begin{equation}
\int(\alpha f+\beta g)\,\D x =\alpha\int f(x)\,\D x+\beta\int g(x)\,\D x.
\end{equation}
\end{proposition}

\begin{proof}
Proving homogeneity of integration is left as an exercise. We will
prove additivity,
\begin{equation}
\int(f+g)\,\D x=\int f(x)\,\D x +\int g(x)\,\D x.
\end{equation}
We see
\begin{subequations}
\begin{equation}
(f+g)^{+}-(f+g)^{-}=f+g,
\end{equation}
and
\begin{equation}
(f+g)=(f)+(g)=(f^{+}-f^{-})+(g^{+}-g^{-}).
\end{equation}
Then by adding $(f+g)^{-}+f^{-}+g^{-}$ to both of these equations. We
can do this because we can replace $f$ and $g$ with finite-valued
functions since the set of points where they are infinite-valued forms
a null set because they are integral (Proof: by contradiction,
$\int|f|<\infty$ implies $|f|$ finite {a.e.}). Then we rearrange terms
to get
\begin{equation}
(f+g)^{+}+f^{-}+g^{-}=f^{+}+g^{+}+(f+g)^{-}.
\end{equation}
\end{subequations}
Then by linearity of the integral of non-negative functions, we have
\begin{subequations}
  \begin{align}
\int\bigl((f+g)^{+}+f^{-}+g^{-}\bigr)&=\int(f+g)^{+}\,\D x+\int f^{-}\,\D x+\int g^{-}\,\D x\\
&=\int\bigl((f+g)^{-}+f^{+}+g^{+}\bigr)\\
&=\int(f+g)^{-}\,\D x+\int f^{+}\,\D x+\int g^{+}\,\D x
  \end{align}
\end{subequations}
Then rearranging terms, we obtain the result.
\end{proof}

\subsection{Dominated Convergence Theorem}

\begin{theorem}
Let $(f_{n})$ be a sequence of \emph{measurable} functions such that
$f_{n}\to f$ pointwise almost everywhere. Suppose there exists a
function $g\in L^{1}(\RR^{d})$ such that for all $n$,
\begin{equation}
|f_{n}(x)|\leq g(x)\quad\mbox{a.e.}
\end{equation}
Then $f\in L^{1}(\RR^{d})$ and
\begin{equation}
\lim_{n\to\infty}\int|f_{n}-f|=0,
\end{equation}
which implies (by the triangle inequality) $\lim\int f_{n}=\int f$.
\end{theorem}

\begin{proof}
Since $|f|\leq g$, we know $f$ is integrable.

\textsc{Step 1: Lower bound.} Consider
\begin{equation}
h_{n}=g+f_{n}.
\end{equation}
So $h_{n}\geq0$ since $|f_{n}|\leq g$. By Fatou's lemma,
\begin{equation}
\underbrace{\int\liminf_{n\to\infty}(g+f_{n})}_{=\int(g+f)}\leq\liminf(g+f_{n}).
\end{equation}
For the right-hand side, since $g,f_{n}\in L^{1}(\RR^{d})$, we can
split the sum
\begin{equation}
\int(g+f_{n})=(\int g)+(\int f_{n})\leq(\int g)+\liminf\int f_{n}.
\end{equation}
Subtracting $\int g$ from both sides gives us
\begin{equation}
\int f\leq\liminf_{n\to\infty}\int f_{n}.
\end{equation}
(We cannot use Fatou's lemma directly because $f_{n}$ may be
negative.)

\textsc{Step 2: Upper bound}. Let
\begin{equation}
k_{n}=g-f_{n}.
\end{equation}
Then $k_{n}\geq0$. Again, using Fatou's Lemma,
\begin{equation}
\int\liminf(g-f_{n})\leq\liminf\int(g-f_{n}).
\end{equation}
Then by linearity,
\begin{equation}
\int(g-f_{n})=(\int g)-(\int f_{n}),
\end{equation}
we have
\begin{equation}
\int(g-f)=(\int g)+(\int-f)\leq(\int
g)+\liminf_{n\to\infty}\int-f_{n}=\int g-\limsup_{n\to\infty}\int f_{n}.
\end{equation}
Then subtracting $\int g\,\D x$ from both sides gives us
\begin{equation}
\int-f\leq-\limsup_{n\to\infty}\int f_{n}
\end{equation}
and then multiplying both sides by $-1$ gives us
\begin{equation}
\int f\geq\limsup_{n\to\infty}\int f_{n}.
\end{equation}

\textsc{Step 3: Sandwich}. From the previous two steps, we see
\begin{equation}
\limsup_{n\to\infty}\int f_{n}\leq\int f\leq\liminf_{n\to\infty}\int f_{n},
\end{equation}
which implies $\lim\int f_{n}$ exists and equals
\begin{equation}
\lim_{n\to\infty}\int f_{n}=\int f.
\end{equation}
(Since $|a|-|b|\leq|a-b|$, we see $|f|-|f_{n}|\leq|f-f_{n}|$, then
$0\leq\int\bigl||f|-|f_{n}|\bigr|\leq\int|f_{n}-f|=0$, we see why
$\int|f-f_{n}|=0$ is a stronger result.)

% This last part was proven at the start of the next lecture
\textsc{Step 4: Triangle}. We see by the triangle inequality,
\begin{subequations}
  \begin{align}
|f_{n}-f| &\leq |f_{n}|+|f|\\
&\leq g+g=2g.
  \end{align}
\end{subequations}
Define $\psi_{n}=2g-|f_{n}-f|$. Observe $\psi_{n}\geq0$. We also see
$\psi_{n}\to 2g$ (since $f_{n}\to f$). Then using Fatou's lemma
\begin{subequations}
  \begin{align}
\int 2g &= \int\liminf_{n\to\infty}(2g-|f_{n}-f|)\\
&\leq\liminf_{n\to\infty}\int(2g-|f_{n}-f|)\quad\mbox{by Fatou's lemma}\\
&=(\int2g)+\liminf_{n\to\infty}\int-|f_{n}-f|\\
&=(\int2g)-\limsup_{n\to\infty}\int|f_{n}-f|.
  \end{align}
\end{subequations}
Then we have $0\leq-\limsup_{n\to\infty}\int|f_{n}-f|$ which means
\begin{equation}
\limsup_{n\to\infty}\int|f_{n}-f|\leq0.
\end{equation}
But we also see the integrand is non-negative, so we have
\begin{equation}
0\leq\limsup_{n\to\infty}\int|f_{n}-f|\leq0,
\end{equation}
which sandwiches together. So then
\begin{equation}
\int|f_{n}-f|=0,
\end{equation}
as desired.
\end{proof}

\begin{remark}
We have a notion of ``Banach spaces''. Roughly,
\begin{equation}
\mbox{Banach space} = \mbox{normed space} + \mbox{metric space} +
\mbox{complete space}.
\end{equation}
A normed space has a metric induced by the norm $d(x,y)=\|x-y\|$. A
Banach space has this metric space structure be ``complete''.
We will see $L^{1}$ is a Banach space.
\end{remark}