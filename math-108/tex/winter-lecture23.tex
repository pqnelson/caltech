%%
%% winter-lecture23.tex
%% 
%% Made by Alex Nelson <pqnelson@gmail.com>
%% Login   <alex@lisp>
%% 
%% Started on  2026-02-26T09:51:06-0800
%% Last update 2026-02-26T09:51:06-0800
%% 

\lecture{}

\begin{node}
We showed every $T\in\mathcal{B}(\mathcal{H})$ there exists a unique
$T^{*}\in\mathcal{B}(\mathcal{H})$ such that for all $\xi,\eta\in\mathcal{H}$
\begin{equation}
\langle T\xi,\eta\rangle=\langle\xi,T^{*}\eta\rangle.
\end{equation}
This gives us a map
\begin{equation}
\begin{split}
\mathcal{B}(\mathcal{H})\to\mathcal{B}(\mathcal{H})\\
T\mapsto T^{*}
\end{split}
\end{equation}
which is involutive $T^{**}=T$, conjugate-linear $(\lambda T)^{*}=\overline{\lambda}T^{*}$,
anti-multiplicative $(TS)^{*}=S^{*}T^{*}$, and it is an isometry, and
it obeys the ``$C*$ identity'':
\begin{equation}
\|T^{*}T\|=\|T\|^{2}.
\end{equation}
\end{node}

\begin{node}
If $T\in\mathcal{B}(\mathcal{H})$, then $T^{**}=T$.
\end{node}

\begin{proof}
For any $\xi,\eta\in\mathcal{H}$, we have
\begin{subequations}
  \begin{align}
\langle\xi,T^{**}\eta\rangle
&=\langle T^{*}\xi,\eta\rangle\\
&=\overline{\langle\eta,T^{*}\xi\rangle}\\
&=\overline{\langle T\eta,\xi\rangle}\\
&=\langle\xi,T\eta\rangle.
  \end{align}
\end{subequations}
Hence the claim.
\end{proof}

\begin{node}
For any $S,T\in\mathcal{B}(\mathcal{H})$, we have $(ST)^{*}=T^{*}S^{*}$.
\end{node}

\begin{proof}
Let $\xi,\eta\in\mathcal{H}$ be arbitrary. Then
\begin{subequations}
  \begin{align}
\langle\xi,(ST)^{*}\eta\rangle
&=\langle(ST)\xi,\eta\rangle\\
&=\langle S(T\xi),\eta\rangle\\
&=\langle T\xi,S^{*}\eta\rangle\\
&=\langle\xi,T^{*}S^{*}\eta\rangle.
  \end{align}
\end{subequations}
Since $\xi$ and $\eta$ were arbitrary, we must have $(ST)^{*}=T^{*}S^{*}$,
as desired.
\end{proof}

\begin{node}[$C*$ norm identity]
For any $T\in\mathcal{B}(\mathcal{H})$, we have $\|T^{*}T\|=\|T\|^{2}$.
\end{node}

\begin{proof}
We know that $\|TS\|\leq\|T\|\cdot\|S\|$, so we have automatically
$\|T^{*}T\|\leq\|T^{*}\|\cdot\|T\|$. Then for any $\xi\in\mathcal{H}$,
we have
\begin{subequations}
  \begin{align}
\|T\xi\|_{\mathcal{H}}
&=\langle T\xi,T\xi\rangle\\
&=\langle T^{*}T\xi,\xi\rangle\\
&\leq\|T^{*}T\|_{\text{op}}\|\xi\|_{\mathcal{H}}^{2},
  \end{align}
\end{subequations}
where we used the Chebyshev-inequality to get the last line, since we
can restrict attention to $\xi$ such that $\|\xi\|\leq1$, this means
$\|T\|^{2}\leq\|T^{*}T\|$.

We can use the fact $\|T^{*}T\|\leq\|T\|\cdot\|T^{*}\|$ now to get the
fact that $\|T\|\leq\|T^{*}\|$. Then we use the fact $T^{**}=T$ to
obtain $\|T^{*}\|\leq\|T^{**}\|=\|T\|$. Hence $\|T^{*}\|=\|T\|$. Then
it follows immediately that $\|T^{*}T\|\geq\|T\|\cdot\|T^{*}\|$.
\end{proof}

\begin{proposition}
For any $T\in\mathcal{B}(\mathcal{H})$, $\ker(T^{*})=(T\mathcal{H})^{\perp}$.
\end{proposition}

\begin{proof}
\textsc{Claim 1:} $\ker(T^{*})\subset(T\mathcal{H})^{\perp}$.
We see for any $\eta\in\ker(T^{*})$ and for any $\xi\in\mathcal{H}$
we have
\begin{subequations}
  \begin{align}
\langle\eta,T\xi\rangle
&=\langle T^{*}\eta,\xi\rangle\\
&=\langle0,\xi\rangle\\
&=0.
  \end{align}
\end{subequations}
Hence $\eta\in(T\mathcal{H})^{\perp}$.

\textsc{Claim 2}: $\ker(T^{*})\supset(T\mathcal{H})^{\perp}$.
The same reasoning read backwards.
\end{proof}

\begin{proposition}
For $T\in\mathcal{B}(\mathcal{H})$, the following are equivalent:
\begin{enumerate}
\item $T$ is invertible
\item $T^{*}$ is invertible
\item $T$ is bijective
\item $T$ and $T^{*}$ are both surjective
\item $T$ and $T^{*}$ are both injective and the image of $T$ is closed
\end{enumerate}
\end{proposition}

\subsection{Types of Operators}

\begin{definition}
We call $T$ \define{Self-Adjoint} if $T=T^{*}$ (so if $F=\CC$, $T$ is
self-adjoint iff  for all $\xi\in\mathcal{H}$ we have $\langle T\xi,\xi\rangle\in\RR$).
\end{definition}

\begin{notation}
We denote the set of all self-adjoint bounded linear operators on $\mathcal{H}$
as $\mathcal{B}(\mathcal{H})_{sa}$
\end{notation}

\begin{definition}
We say $T\in\mathcal{B}(\mathcal{H})$ is \define{Unitary} if
$T^{*}T=TT^{*}=\id$. Equivalently, if $T$ is an isometric isomorphism.
\end{definition}

\begin{notation}
We denote the set of unitary bounded linear operators on $\mathcal{H}$
as $\mathcal{U}(\mathcal{H})$ and the intuition is that they are
analogous to the unit circle in the complex plane $\CC$.

They form a group under multiplication since $(UV)^{*}UV=V^{*}U^{*}UV=\id$
and $UV(UV)^{*}=\id$ (and multiplication is associative, and the
identity operator is unitary).
\end{notation}

\begin{definition}
We say $T,S\in\mathcal{B}(\mathcal{H})$ are \define{Unitarily Equivalent}
if there exists a unitary $U\in\mathcal{B}(\mathcal{H})$ such that $U^{*}TU=S$.
\end{definition}

\begin{definition}
We say $T\in\mathcal{B}(\mathcal{H})$ is \define{Normal} if it
commutes with its adjoint $T^{*}T=TT^{*}$.

Equivalently, if for all $\xi\in\mathcal{H}$ we have $\|T\xi\|_{\mathcal{H}}=\|T^{*}\xi\|_{\mathcal{H}}$.
\end{definition}

\begin{definition}
An \define{Isometry} $T\in\mathcal{B}(\mathcal{H})$ is such that $TT^{*}=\id$.
\end{definition}

\begin{example}[Isometry but not unitary]
Consider $\ell^{2}$ with the standard canonical orthonormal basis 
$e_{1}=(1,0,0,\dots)$, $e_{2}=(0,1,0,\dots)$, and so on. Now consider
the linear operator $T\colon\ell^{2}\to\ell^{2}$ such that $T(e_{n})=e_{n+1}$.
This is not a unitary operator, but it is isometric.
\end{example}

\begin{xca}
Determine $T^{*}$ from the previous example.
\end{xca}

\begin{definition}
We say $T\in\mathcal{B}(\mathcal{H})$ is \define{Positive} if for all
$\xi\in\mathcal{H}$ we have $\langle T\xi,\xi\rangle\geq0$ and (if $F=\RR$)
$T=T^{*}$.
\end{definition}

\begin{node}
Let $T\in\mathcal{B}(\mathcal{H})$. Observe the following are equivalent:
\begin{enumerate}
\item $T$ is positive
\item There exists an $S\in\mathcal{B}(\mathcal{H})$ such that $T=S^{*}S$
\item There exists a self-adjoint $W\in\mathcal{B}(\mathcal{H})_{sa}$
  such that $T=W^{2}$ (i.e., ``$T$ has a square root'').
\end{enumerate}
\end{node}

\begin{node}[Functional calculus]
We can take advantage of the density of polynomials in the ring of
continuous real-valued functions to do stuff with operators. Observe
that if $p(x)\in\RR[x]$ (or $p(z)\in\CC[z]$), and if
$T\in\mathcal{B}(\mathcal{H})$ is a linear operator, then $p(T)$ is a
bounded linear operator on $\mathcal{H}$ since it is determined by
finitely many additions of finitely many scalar multiples of operator products.

If $f\in C_{c}(\RR)$ or $f\in C_{c}(\CC)$ (depending on if $F=\RR$ or
$F=\CC$, respectively), then we can try the same trick to find a
bounded linear operator $f(T)\in\mathcal{B}(\mathcal{H})$.

We first need to introduce the notion of the spectrum of a linear
operator $T$ as
\begin{equation}
\sigma(T):=\{\lambda\in\FF\mid\lambda\id-T\mbox{ is not invertible}\}.
\end{equation}
We can consider $f\in C_{c}(\sigma(T))$, then $f(T)\in\mathcal{B}(\mathcal{H})$
makes sense.

For measurable functions $f$, it is tricky to consider $f(T)$. We need
something like $f\in L^{\infty}$.
\end{node}

\begin{node}
Using functional calculus, we can show $T$ is positive implies
$\sigma(T)\subset[0,\infty)$ and the square root of $T$ then ``make sense''.
\end{node}

\begin{definition}
We say $P\in\mathcal{B}(\mathcal{H})$ is a \define{Projection} if
$P=P^{2}$ it is idempotent, and if it is self-adjoint $P=P^{*}$.
\end{definition}

\begin{remark}
Projections among operators are analogous to characteristic functions
of subsets.
\end{remark}

\begin{remark}
Projections are building blocks for all of the theory. Any operator
can be written as a weighted sum of projections (or an integral of projections).
\end{remark}

\begin{xca}
Suppose $T\colon L^{2}([0,1])\to L^{2}([0,1])$ is defined by $(Tf)(t)=tf(t)$.
\begin{enumerate}
\item Prove this is a bounded linear operator.
\item Prove this has no eigenvectors.
\end{enumerate}
\end{xca}

\begin{definition}
An operator $T$ is \define{Finite Rank} if $\dim(\Im(T))<\infty$.
We denote $\mathcal{B}_{f}(\mathcal{H})$ for the subalgebra of finite
rank operators.
\end{definition}

\begin{proposition}
If $T\in\mathcal{B}_{f}(\mathcal{H})$, then $\mathcal{H}=T\mathcal{H}\oplus\ker(T^{*})$.
Moreover, $T^{*}\in\mathcal{B}_{f}(\mathcal{H})$.
\end{proposition}

\begin{proof}
We see $T^{*}\mathcal{H}=T^{*}T\mathcal{H}\oplus0=T^{*}T\mathcal{H}$.
Then $T^{*}\in\mathcal{B}_{f}(\mathcal{H})$.
\end{proof}

\begin{remark}
The analogy: $\mathcal{B}_{f}(\mathcal{H})\subset\mathcal{B}(\mathcal{H})$
is kind of like $C_{c}(X)\subset C_{b}(X)$ compact continuous
real-valued functions of a compact Hausdorff space $X$ as contained in
the bounded continuous real-valued functions on $X$.
\end{remark}

\begin{definition}
Let $B=\{f\in\mathcal{H}\mid \|f\|\leq1\}$ be the unit ball in $\mathcal{H}$.
  We say $T$ is \define{Compact} if the [topological] closure of the
  image of the unit ball $\closure{T(B)}$ is compact.
\end{definition}

\begin{fact}
We see $T$ is compact if and only if $T\in\closure{\mathcal{B}_{f}(\mathcal{H})}$
in the normed topology.
\end{fact}

\begin{notation}
We write $K(\mathcal{H})$ for the set of all compact operators on $\mathcal{H}$.
\end{notation}

\begin{fact}
$K(\mathcal{H})=\closure{\mathcal{B}_{f}(\mathcal{H})}$.
\end{fact}

\begin{fact}
If $T\in K(\mathcal{H})$ and $S\in\mathcal{B}(\mathcal{H})$,
then $ST\in K(\mathcal{H})$.
\end{fact}

\begin{theorem}[Spectral theorem for compact operators]
For convenience, let $\mathcal{H}$ be a separable Hilbert space.
Then if $T\in K(\mathcal{H})$ is compact and self-adjoint, then there
exists an orthonormal basis $\{e_{i}\}$ of $\mathcal{H}$ such that
each $e_{i}$ is an eigenvector of $T$.
\end{theorem}