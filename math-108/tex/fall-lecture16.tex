%%
%% fall-lecture16.tex
%% 
%% Made by Alex Nelson <pqnelson@gmail.com>
%% Login   <alex@lisp>
%% 
%% Started on  2025-11-11T08:26:10-0800
%% Last update 2025-11-11T08:26:10-0800
%% 

\lecture[Stone--Weierstrass]{}

\begin{note}
For today's lecture, $X$ is a compact Hausdorff space unless otherwise stated.
\end{note}

\begin{theorem}
Let $f\in C[a,b]$. Then for each $\varepsilon>0$, there exists a
polynomial $p$ such that for all $x\in[a,b]$ we have $|f(x)-p(x)|<\varepsilon$.
\end{theorem}

\begin{remark}
The preceding theorem is the ``classic'' version of the
Stone--Weierstrass \emph{approximation} theorem.
There is a constructive proof using Bernstein polynomials
$b_{k,n}(x)=\binom{n}{k}x^{k}(1-x)^{n-k}$ for $k=0,\dots,n$ which are
defined on the interval $[0,1]$.
\end{remark}

\begin{lemma}\label{lemma:fall-lec16:abs}
For each $\varepsilon>0$ there is a polynomial $p(x)$ such that
$|p(x)-|x||<\varepsilon$ for all $x\in[-M,M]$.
\end{lemma}

\begin{proof}
Consider the Binomial series $f(x)$ for
\begin{equation}
f(x)=(1-x)^{1/2}=\sum^{\infty}_{n=0}c_{n}x^{n},
\end{equation}
which converges uniformly in $[-1,1]$. For any $\varepsilon>0$, there
is a sum $p(x)$ such that $|p(x)-(1-x)^{1/2}|<\varepsilon$ for any $|x|\leq1$
(which may be found by truncating the binomial series $f(x)$ to some
sufficiently high degree polynomial).

We do several changes of variables. First, let $x=1-t^{2}$, then
\begin{equation}
\left|p(1-t^{2})-|t|\right|<\varepsilon
\end{equation}
whenever $|t|\leq1$. We set $u=t/M$ and $\varepsilon'=M\varepsilon$,
then
\begin{equation}
\left|p(1-M^{2}u^{2})-|Mu|\right|<M\varepsilon',
\end{equation}
and dividing through by $M$ gives us
\begin{equation}
\left|\frac{p(1-M^{2}u^{2})}{M}-|u|\right|<\varepsilon',
\end{equation}
which concludes the proof for the lemma.
\end{proof}

\begin{definition}
We call a vector subspace $\mathcal{A}$ of $C(X)$ a \define{Function Algebra}
if (1)~it contains the constant functions, and (2)~for any $f,g\in\mathcal{A}$,
their pointwise product $fg\in\mathcal{A}$.
\end{definition}

\begin{example}
The set of real polynomials is a function algebra on $C[a,b]$.
\end{example}

\begin{proposition}
The closure of a function algebra $\mathcal{A}$ of $C(X)$ in the
metric topology (induced using the uniform metric),
$\closure{\mathcal{A}}$, is also a function algebra of $C(X)$.
\end{proposition}

The proof is in the next homework.

\begin{definition}
A collection $\mathcal{A}$ of real-valued functions on $X$ is said to
\define{Separate Points} in $X$ if for any $u,v\in X$ distinct $u\neq v$,
there exists an $f\in\mathcal{A}$ such that $f(u)\neq f(v)$.
\end{definition}

\begin{theorem}[Stone--Weierstrass]
Let $X$ be a compact Hausdorff space.
Let $\mathcal{A}$ be a function algebra of $C(X)$ that separates
points in $X$.
Then $\mathcal{A}$ is dense in $C(X)$.
\end{theorem}

(See also Baby Rudin, Theorem~7.32.)

\begin{proof}
\textsc{Step 1:} \emph{If $f\in\closure{\mathcal{A}}$, then $|f|\in\closure{\mathcal{A}}$.}
Let $M=\|f\|_{\infty}$. By Lemma~\ref{lemma:fall-lec16:abs}, there is
a polynomial $p(t)$ such that it approximates $\left|p(t)-|t|\right|<\varepsilon$
for any $\varepsilon>0$ and $|t|\leq M$. Since $|f(x)|\leq M$ for all
$x\in X$, we have
\begin{equation}
\left|p(f(x))-|f(x)|\right|<\varepsilon,
\end{equation}
i.e.,
\begin{equation}
\left|p\circ f-|f|\right|<\varepsilon.
\end{equation}
Then by definition of a function algebra, $p\circ f\in\mathcal{A}$ for
any $f\in\mathcal{A}$. (This is by the second property of the
definition --- think about it, $cx^{n}\in\mathcal{A}$ if $x\in\mathcal{A}$.) Then $|f|$ is a limit point of $\closure{\mathcal{A}}$,
but by virtue of the closure containing all its limit points we have $|f|\in\closure{\mathcal{A}}$.

\textsc{Step 2:} \emph{If $f\in\closure{\mathcal{A}}$ and $g\in\closure{\mathcal{A}}$, then $\max(f,g)\in\closure{\mathcal{A}}$
and $\min(f,g)\in\closure{\mathcal{A}}$.}
We see that $f\pm g\in\closure{\mathcal{A}}$ and $|f\pm g|\in\closure{\mathcal{A}}$.
Then using
\begin{equation}
\max(f,g)=\frac{1}{2}(f+g)+\frac{1}{2}|f-g|,
\end{equation}
implies $\max(f,g)\in\closure{\mathcal{A}}$. Similarly, from
\begin{equation}
\min(f,g)=\frac{1}{2}(f+g)-\frac{1}{2}|f-g|,
\end{equation}
we find $\min(f,g)\in\closure{\mathcal{A}}$.

\textsc{Step 3:} Since $\mathcal{A}$ separates points in $X$, take any
distinct $u,v\in X$, there exists a function $g\in\mathcal{A}$ such
that $g(u)\neq g(v)$ (by definition of separate points). For any fixed
distinct $a,b\in\RR$, we may define a function
\begin{equation}
f(x) = \frac{g(x)-g(v)}{g(u)-g(v)}a+\frac{g(x)-g(u)}{g(v)-g(u)}b.
\end{equation}
Then $f\in\mathcal{A}$ and $f(u)=a$ and $f(v)=b$.

\textsc{Step 4:} \emph{Given a real-valued function $h$ continuous on $X$
and $\varepsilon>0$, there exists an $f\in\closure{\mathcal{A}}$ such
that $|h(x)-f(x)|<\varepsilon$ for all $x\in X$.}
For any $\varepsilon>0$ and distinct $u,v\in X$, there exists a
function $f_{u,v}\in\closure{\mathcal{A}}$ such that $f_{u,v}(u)=h(u)$
and $f_{u,v}(v)=h(v)$ by step 3. If we fix $u$, then $f_{u,v}(v)=h(v)>h(v)-\varepsilon$.
Then we use continuity, there exists a neighborhood $\mathcal{O}_{v}$
such that $f_{u,v}(x)>h(x)-\varepsilon$ for all $x\in\mathcal{O}_{v}$.
We do this for every $v\in X$ and $\{\mathcal{O}_{v}\}$ is an open
cover for $X$. By compactness, there is a finite subcover
$\mathcal{O}_{v_{1}}$, \dots, $\mathcal{O}_{v_{n}}$. We define a
function
\begin{equation}
f_{u}(x) := \max\{f_{u,v_{1}}(x),\dots,f_{u,v_{n}}(x)\}
\end{equation}
for all $x\in X$. This approximates $h$ from below.

Similarly, since $f_{u}(u)=h(u)<h(u)+\varepsilon$, there is a
neighborhood $U_{u}$ such that $f_{u}(x)<h(x)+\varepsilon$. We do this
for every $u\in X$ and obtain an open cover $\{U_{u}\}$. By
compactness, there is a finite subcover $U_{u_{1}}$, \dots, $U_{u_{m}}$.
Then we define
\begin{equation}
f(x) = \min\{f_{u_{1}}(x),\dots,f_{u_{m}}(x)\}.
\end{equation}
Then we get $f\in\closure{\mathcal{A}}$ and $|f(x)-h(x)|<\varepsilon$
for all $x\in X$ --- or, equivalently, $\|f-h\|_{\infty}<\varepsilon$.
Since $\varepsilon$ was arbitrary, $h$ is a limit point of
$\closure{\mathcal{A}}$. Hence the result.
\end{proof}

\subsection{Differentiation}

\begin{note}
We will work in $\RR^{n}$ from now on.
\end{note}

\begin{node}[Recall calculus]
Let $x\in(a,b)$ and $f\colon(a,b)\to\RR$.
Then we have the derivative of $f$ at $x$ be defined as the limit
\begin{equation}
f'(x) := \lim_{h\to0}\frac{f(x+h)-f(x)}{h},
\end{equation}
when it exists. This is equivalent to something like
\begin{equation}
f(x+h)-f(x)=f'(x)\cdot h + \varepsilon(h),
\end{equation}
where $\varepsilon(h)$ is called the \define{Remainder} and is such
that
\begin{equation}
\lim_{h\to0}\frac{\varepsilon(h)}{h}=0.
\end{equation}
We then view $f'(x)$ as a linear operator acting on $h$.
\end{node}

\begin{note}
We will write vectors as $\vec{x}$ instead of $\vector{x}$.
\end{note}

\begin{definition}
Let $\vec{f}\colon U\to\RR^{m}$ where $U\subset\RR^{n}$ is open.
We say $\vec{f}$ is \define{Differentiable} at $\vec{x}\in U$ if there
exists a linear transformation $A\colon\RR^{n}\to\RR^{m}$ denoted by
the $m\times n$ matrix $A$ such that for all $\vec{h}$ in a
neighborhood of the origin of $\RR^{n}$,
\begin{equation}
\vec{f}(\vec{x}+\vec{h})-\vec{f}(\vec{x})=A\vec{h}+\varepsilon(\vec{h}),
\end{equation}
where
\begin{equation}
\lim_{\vec{h}\to\vec{0}}\frac{\|\varepsilon(\vec{h})\|}{\|\vec{h}\|}=0,
\end{equation}
where $\|-\|$ is the $\ell_{\infty}$-norm in $\RR^{n}$ or $\RR^{m}$
(depending on where the vector lives).

We say $\vec{f}$ is \define{Differentiable} if it is differentiable at
every point.
\end{definition}