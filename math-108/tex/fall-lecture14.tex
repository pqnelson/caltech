%%
%% fall-lecture14.tex
%% 
%% Made by Alex Nelson <pqnelson@gmail.com>
%% Login   <alex@lisp>
%% 
%% Started on  2025-11-06T10:59:42-0800
%% Last update 2025-11-06T10:59:42-0800
%% 

\lecture{}

(The professor was out of town, there was a quiz/miderm, and so this
was lecture was 10 days after the previous lecture.)

Recall the Heine--Cantor Theorem~\ref{thm:fall-2025:heine-cantor}:

\begin{theorem}[Heine--Cantor]
Let $f\colon(X,d)\to(Y,\rho)$ be a continuous function, where $(X,d)$ is a
compact metric space and $(Y,\rho)$ is an arbitrary metric space.
Then $f$ is uniformly continuous.
\end{theorem}

\begin{node}
The intuition: ``uniform'' stuff has $\delta$ (or $N$ or\dots) depend
only on $\varepsilon$, but not on $x$ or anything else.
\end{node}

\begin{definition}
Let $f_{n}\colon X\to Y$ be a sequence of functions on metric
spaces. We say $(f_{n})$ \define{Converges Uniformly} to $f\colon X\to Y$ 
if:
\begin{equation*}
\forall\varepsilon>0\exists N\in\NN\forall x\in X\forall n\in\NN\ldotp n\geq N\implies\rho(f_{n}(x),f(x))<\varepsilon.
\end{equation*}
(The threshold $N$ \emph{does not} depend on the choice of $x\in X$.)
\end{definition}

\begin{example}
Consider $f_{n}(x)=x^{1/n}$ for $x\in[0,1]$. Then pointwise
\begin{equation}
f_{n}\to f(x)=\begin{cases}0 & \mbox{if }x=0\\
1 & \mbox{otherwise}
\end{cases}
\end{equation}
But this is not uniformly convergent. Why/ For any $x\in(0,1)$, we see
\begin{equation}
|f_{n}(x)-f(x)|=1-x^{1/n}\geq\varepsilon
\end{equation}
we can always pick an $n$ large enough for $1-x^{1/n}\geq\varepsilon$.
\end{example}

\begin{definition}
Let $f_{n}\colon X\to Y$ be a sequence of functions between metric spaces.
We say $f_{n}$ is \define{Uniformly Cauchy} if
\begin{equation*}
\forall\varepsilon>0\exists N\in\NN\forall x\in X\forall m,n\in\NN\ldotp%
m\geq N\land n\geq N\implies\rho(f_{m}(x),f_{n}(x))<\varepsilon.
\end{equation*}
\end{definition}

\begin{theorem}
Let $(Y,\rho)$ be a omplete metric space. Then $f_{n}\colon X\to Y$
is uniformly convergent if and only if it is uniformly Cauchy.
\end{theorem}

\begin{proof}
\forwardproof\ Triangle inequality.

\backwardproof\ Assume $f_{n}$ is uniformly Cauchy. Then for all
$\varepsilon>0$ there exists an $N\in\NN$ such that for all $x\in X$,
for all $m,n\geq N$ we have:
\begin{equation}
\rho(f_{m}(x),f_{n}(x))<\varepsilon/2.
\end{equation}
In particular, fixing $x\in X$, we see $\{f_{m}(x)\}$ is a Cauchy
sequence. Since $Y$ is complete, $f_{m}(x)\to f(x)$ in $Y$. There
exists some index $M_{x}\in\NN$ such that $\rho(f(x),f_{M_{x}}(x))<\varepsilon/2$.
Then by the Triangle Inequality, we can write
\begin{subequations}
\begin{align}
\rho(f_{n}(x),f(x)) & \leq \rho(f_{n}(x),f_{M_{x}}(x)) + \rho(f_{M_{x}}(x),f(x))\\
&<\frac{\varepsilon}{2} + \rho(f_{M_{x}}(x),f(x))\quad\mbox{provided }M_{x},n\geq N\\
&<\frac{\varepsilon}{2} + \frac{\varepsilon}{2}\quad\mbox{provided }M_{x}\geq N
\end{align}
\end{subequations}
This was for any choice of $N$ (i.e., $N$ does not depend on the
choice of $x$). Hence it converges uniformly.
\end{proof}

\begin{theorem}
Let $\{f_{n}\}$ be a sequence of continuous functions and $f_{n}\to f$
uniformly. Then $f$ is continuous.
\end{theorem}

\begin{proof}
By uniform convergence, for each $\varepsilon>0$ there is an $N\in\NN$
such that
\begin{equation}
\rho(f_{N}(x),f(x))<\varepsilon,
\end{equation}
for every $x\in X$. In particular, fix a point $a\in X$. Then $f_{N}$
is continuous at $a\in X$. There exists a $\delta>0$ such that
\begin{equation}
f_{N}(B_{\delta}(a))\subset B_{\varepsilon}(f_{N}(a)).
\end{equation}
Then for any $x\in B_{\delta}(a)$, we have by the Triangle Inequality:
\begin{subequations}
\begin{align}
\rho(f(x),f(a)) & \leq \rho(f(x),f_{N}(x)) + \rho(f_{N}(x),f_{N}(a))+\rho(f_{N}(a),f(a))\\
&< \varepsilon + \rho(f_{N}(x),f_{N}(a))+\varepsilon\quad\mbox{by $f_{n}\to f$ uniformly}\\
&< 3\varepsilon\quad\mbox{by continuity of $f_{n}$ at $a$}
\end{align}
\end{subequations}
Then $f$ is continuous at $a\in X$. Since $a\in X$ was arbitrary, we
see $f$ is continuous everywhere.
\end{proof}

\begin{theorem}\label{thm:fall-lec14:complete-metric-spaces-have-super-nice-convvergent-sequences}
Let $(Y,\rho)$ be a complete metric space. Let $f_{n}\colon X\to Y$
converge uniformly to $f\colon X\to Y$.
If the limits $\lim_{x\to a}f_{n}(x)=b_{n}$ exists in $Y$, then
$b_{n}\to b$ in $Y$ with $\lim_{x\to a}f(x)=b$.
Equivalently,
\begin{equation}
\lim_{n\to\infty}\lim_{x\to a}f_{n}(x)=\lim_{x\to a}\lim_{n\to\infty}f_{n}(x).
\end{equation}
\end{theorem}

\begin{proof}
We see $f_{n}$ is uniformly Cauchy: for each $\varepsilon>0$ there is
an $N\in\NN$ such that for all $x\in X$ we have
\begin{equation}\label{fall-lec14:step-1}
\rho(f_{m}(x),f_{n}(x))<\varepsilon\quad\forall m,n\geq N.
\end{equation}
Then $b_{n}=\lim_{x\to a}f_{n}(x)$. Take $x\to a$, we have
\begin{equation}\label{fall-lec14:step-2}
\rho(b_{m},b_{n})<\varepsilon,
\end{equation}
which implies $(b_{n})$ is Cauchy in $Y$. By completeness, it
converges to $b_{n}\to b\in Y$. Taking $n\to\infty$ and $m\to N$, we
see that Equation~\eqref{fall-lec14:step-1} becomes
\begin{equation}\label{fall-lec14:step-1-prime}
\rho(f(x),f_{N}(x))<\varepsilon
\end{equation}
and Equation~\eqref{fall-lec14:step-2} becomes
\begin{equation}\label{fall-lec14:step-2-prime}
\rho(b,b_{N})<\varepsilon.
\end{equation}
Using $b_{N}=\lim_{x\to a}f_{N}(x)$, we can choose a neighborhood
$U\ni a$ such that
\begin{equation}\label{fall-lec14:step-3}
\rho(b_{N},f_{N}(x))<\varepsilon\quad\mbox{for all }x\in U.
\end{equation}
The triangle inequality gives us
\begin{subequations}
\begin{align}
\rho(f(x),b) &\leq \rho(f(x),f_{N}(x)) + \rho(f_{N}(x),b_{N}) + \rho(b_{N},b)\\
&<\varepsilon + \rho(f_{N}(x),b_{N}) + \rho(b_{N},b)\quad\mbox{by Eq~\eqref{fall-lec14:step-1-prime}}\\
&<\varepsilon + \varepsilon + \rho(b_{N},b)\quad\mbox{by Eq~\eqref{fall-lec14:step-3}}\\
&<\varepsilon + \varepsilon + \varepsilon\quad\mbox{by Eq~\eqref{fall-lec14:step-2-prime}}.
\end{align}
\end{subequations}
Hence $\lim_{x\to a}f(x)=b$.
\end{proof}

\begin{definition}
A collection $\mathcal{F}$ of functions $X\to Y$ is called
\define{Uniformly Equicontinuous} if for each $\varepsilon>0$ there is
a $\delta>0$ such that $\rho(f(x),f(y))<\varepsilon$ whenever
$d(x,y)<\delta$ and $f\in\mathcal{F}$.

Observe the choice of $\delta$ does not depend on $f\in\mathcal{F}$.
(This implies every $f\in\mathcal{F}$ is continuous, but the converse
is not true.)
\end{definition}

\begin{example}
Consider $f_{n}(x)=x^{n}$ for $x\in[0,1]$. Then by Heine--Cantor Theorem~\ref{thm:fall-2025:heine-cantor}, each
$f_{n}$ is uniformly continuous. Fix $\varepsilon>0$ and $x\in[0,1)$,
then $|f_{n}(x)-f_{n}(1)|=1-x^{n}$. But $x^{n}\to 0$ as $n\to\infty$
for all $0\leq x<1$. We can choose $n$ large enough such that $1-x^{n}>\varepsilon$.
\end{example}

\begin{theorem}
Let $(X,d)$ be compact. If $f_{n}\colon X\to Y$ is a uniformly
convergent sequence of continuous functions, then $\{f_{n}\}$ is
uniformly equicontinuous.
\end{theorem}

The proof consists of two moments: prove $\rho(f_{n}(x),f_{n}(y))<\varepsilon$,
then prove it is continuous.

\begin{proof}
The limiting function $f\colon X\to Y$ is continuous (and by
Heine--Cantor, uniformly continuous). So for each $\varepsilon>0$,
there is a $\delta_{0}>0$ such that
\begin{equation}\label{fall-lec14:last-thm:eq-1}
\forall x,y\in X\ldotp d(x,y)<\delta_{0}\implies \rho(f(x),f(y))<\varepsilon/3
\end{equation}
Since $f_{n}\to f$ uniformly, there is an $N\in\NN$ such that
\begin{equation}\label{fall-lec14:last-thm:eq-2}
\forall x\in X\forall n\geq N\ldotp \rho(f_{n}(x),f(x))<\varepsilon/3.
\end{equation}
Then by the Triangle Inequality
\begin{subequations}
\begin{align}
\rho(f_{n}(x),f_{n}(y)) &\leq \rho(f_{n}(x),f(y)) + \rho(f(x),f(y))+\rho(f(y),f_{n}(y))\\
&<\frac{\varepsilon}{3} + \rho(f(x),f(y)) + \frac{\varepsilon}{3}\quad\mbox{by Eq~\eqref{fall-lec14:last-thm:eq-2}}\\
&<\frac{\varepsilon}{3} + \frac{\varepsilon}{3} + \frac{\varepsilon}{3}=\varepsilon\quad\mbox{by Eq~\eqref{fall-lec14:last-thm:eq-1}}.
\end{align}
\end{subequations}
This concludes the first moment of the proof.

Now, each function $f_{n}$ is also uniformly continuous by
Heine--Cantor. So for each $n$, there is a $\delta_{n}>0$ such that
\begin{equation}
\rho(f_{n}(x),f_{n}(y))<\varepsilon\quad\mbox{whenever }d(x,y)<\delta_{n}.
\end{equation}
Take
\begin{equation}
\delta=\min(\delta_{0},\delta_{1},\dots,\delta_{n}).
\end{equation}
Then $\delta>0$ and so for all $n\in\NN$, we have:
\begin{equation}
d(x,y)<\delta\implies\rho(f_{n}(x),f_{n}(y))<\varepsilon,
\end{equation}
as desired.
\end{proof}