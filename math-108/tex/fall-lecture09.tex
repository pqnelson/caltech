%%
%% fall-lecture09.tex
%% 
%% Made by Alex Nelson <pqnelson@gmail.com>
%% Login   <alex@lisp>
%% 
%% Started on  2025-10-19T11:01:49-0700
%% Last update 2025-10-19T11:01:49-0700
%% 

\lecture{}

\begin{node}
Recall, a function $f\colon X\to Y$ is \emph{continuous} at $x\in X$ if
\begin{equation*}
\forall\varepsilon>0\ldotp\exists\delta>0\ldotp\forall a\in X\ldotp
d(x,a)<\delta\implies\rho\bigl(f(x),f(a)\bigr)<\varepsilon.
\end{equation*}
We stress that $\delta$ depends on the choice of $x$ and $\varepsilon$.
\end{node}

\begin{definition}
We call $f\colon X\to Y$ \define{Uniformly Continuous} if
\begin{equation}
\forall\varepsilon>0\ldotp\exists\delta>0\ldotp\forall u,v\in X\ldotp
d(u,v)<\delta\implies\rho\bigl(f(u),f(v)\bigr)<\varepsilon.
\end{equation}
\end{definition}

\begin{example}
A function $f\colon X\to Y$ is called \define{Lipschitz} if there is a
constant $c\geq0$ such that
\begin{equation}
\forall u,v\in X\ldotp\rho\bigl(f(u),f(v)\bigr)\leq cd(u,v).
\end{equation}
We claim that a Lipschitz function is uniformly continuous. Let
$\varepsilon>0$ be arbitrary, then pick $\delta=\varepsilon/c$ when $c\neq0$.
When $c=0$, then we have the constant function, which is uniformly continuous.
\end{example}

\begin{example}
Consider $f\colon\RR\to\RR$, $f(x)=x^{2}$. This is continuous on
$\RR$, but not uniformly continuous.

Let $x\in\RR$ be arbitrary. Let us look at
\begin{subequations}
\begin{align}
|f(x)-f(x+\delta)| &=(x+\delta)^{2}-x^{2}\\
&= \delta^{2}+2x\delta > 2x\delta.
\end{align}
\end{subequations}
But there is no way to pick $\varepsilon>0$ which does not depend on
$x$ in this situation, so it cannot be uniformly continuous.
\end{example}

\subsection{Compactness}

\begin{definition}
Let $E\subset X$. Let $\{U_{\alpha}\}_{\alpha\in A}$ be a collection
of subsets. We call $\{U_{\alpha}\}_{\alpha\in A}$ a \define{Cover} of
$E$ if
\begin{equation}
E\subset\bigcup_{\alpha\in A}U_{\alpha}.
\end{equation}
\end{definition}

\begin{definition}
We define a \define{Subcover} of $E$ to be a subset $S$ of a cover $C$ such
that the subset $S$ is also a cover of $E$.
\end{definition}

\begin{definition}
We call the cover $\{U_{\alpha}\}$ of $E$ \define{Open} if each
$U_{\alpha}$ is an open set.
\end{definition}

\begin{definition}
Let $K\subset X$. We call $K$ \define{Compact} if every open cover of
$K$ has a finite subcover.
\end{definition}

\begin{example}
Any finite set is compact. Why? Well, pick any open cover, then only
finitely many open subsets are needed (at most one for each of the
finitely many points, which would yield a finite subcover).
\end{example}

\begin{example}
The open interval $(0,1)\subset\RR$ is not compact. We take $U_{n}=(1/n,1)$,
and these form an open cover of $(0,1)$. But this particular covering
has no finite subcover.
\end{example}

\begin{example}
We see $\RR\subset\RR$ is not compact. Observe $\{(-n,n)\mid n\in\NN\}$
is an open cover. But again, it has no finite subcover.
\end{example}

\begin{definition}
Let $E\subset X$. We call $E$ \define{Bounded} if there exists an
$x\in X$ and $r.0$ such that $E\subset B_{r}(x)$.
\end{definition}

\begin{theorem}
\begin{enumerate}
\item Compact subset of metric spaces are closed and bounded.
\item Closed subsets of compact sets are compact.
\item The continuous image of a compact set is compact (i.e., if
  $f\colon X\to Y$ is continuous and $K\subset X$ is compact, then
  $f(K)$ is compact).
\end{enumerate}
\end{theorem}

\begin{proof}
\begin{enumerate}
\item Let $K$ be a compact subset of $X$. Let $x\in X\setminus K$.
  Then for each $z\in K$, let $r_{z}>0$ and $s_{z}>0$ be such that
\begin{equation}
B_{r_{z}}(z)\cap B_{s_{z}}(x)=\emptyset.
\end{equation}
We can do this for every $z\in K$. Then $\{B_{r_{z}}(z)\}$ forms an
open cover of $K$. Since $K$ is compact, it has a finite subcover:
$B_{r_{1}}(z_{1})$, \dots, $B_{r_{k}}(z_{k})$ which covers $K$. Let
\begin{equation}
s = \min\{s_{z_{j}}\mid j=1,\dots,k\}.
\end{equation}
Then $B_{s}(x)\cap B_{z_{j}}(z_{j})=\emptyset$ for each $j=1,\dots,k$.
Then
\begin{equation}
B_{s}(x)\cap\bigl(B_{z_{1}}(z_{1})\cup\cdots\cup B_{z_{k}}(z_{k})\bigr)=\emptyset,
\end{equation}
so we must have
\begin{equation}
B_{s}(x)\subset X\setminus K.
\end{equation}
Then $X\setminus K$ is open if and only if $K$ is closed.

To prove boundedness: For any $x_{0}\in K$, pick
$\{B_{n}(x_{0})\mid n\in\NN\}$ as an open cover of $K$. Then there
exists a finite subcover. Since $B_{m}(x_{0})\subset B_{n}(x_{0})$ for
any $m\leq n$, this means there exists an $N\in\NN$ such that
$K\subset B_{N}(x_{0})$. This implies $K$ is bounded.
\item Let $F$ be a closed subset of a compact set $K$. Let
  $\{U_{\alpha}\}$ be an open cover of $F$. Since $F$ is closed,
  $K\setminus F$ is open, so $\{K\setminus F\}\cup\{U_{\alpha}\}$ is
  an open cover of $K$. Then there exists a finite subcover of $K$ ---
  there are finitely many $U_{\alpha_{1}}$, \dots, $U_{\alpha_{n}}$
  such that $U_{\alpha_{1}}\cup\cdots\cup U_{\alpha_{n}}=F$.
\item Let $f\colon X\to Y$ be a continuous map of metric spaces. Let
  $K\subset X$ be compact. We want to show $f(K)\subset Y$ is compact.
  Let $\{U_{\alpha}\}$ be an arbitrary open cover of $f(K)$ in
  $Y$. For each $x\in K$, there is an $\alpha$ such that $f(x)\in U_{\alpha}$.
  So $x\in f^{-1}(U_{\alpha})$. This means $\{f^{-1}(U_{\alpha})\}$ is
  an open cover of $K$ --- $f^{-1}(U_{\alpha})$ is open since $f$ is continuous.
  This means since $K$ is compact, there exists a finite subcover
  $\{f^{-1}(U_{\alpha_{i}})\mid i=1,\dots,n\}$ of $K$. So
  \begin{equation}
K\subset\bigcup^{n}_{i=1}f^{-1}(U_{\alpha_{i}}).
  \end{equation}
  Then $f(K)\subset\bigcup^{n}_{i=1}f(f^{-1}(U_{\alpha_{i}}))\subset\bigcup^{n}_{i=1}U_{\alpha_{i}}$.
  Hence $f(K)$ is compact.\qedhere
\end{enumerate}
\end{proof}

\begin{theorem}[Extreme value]
If $(X,d)$ is a compact metric space, and $f\colon X\to\RR$ is
continuous, then there are points $a\in X$ and $b\in X$ such that for
all $x\in X$ we have $f(a)\leq f(x)\leq f(b)$.
\end{theorem}

\begin{proof}
By (3) of the previous theorem, $f(X)$ is compact in $\RR$, and by (1)
of the same theorem, $f(X)$ is closed and bounded in $\RR$. Let
\begin{equation}
\alpha=\inf(f(X)),
\end{equation}
and
\begin{equation}
\beta=\sup(f(X)),
\end{equation}
by the least-upper bound property of $\RR$. Since $f(X)$ is closed,
the limit points of $f(X)$ are elements of $f(X)$. The result follows
immediately. 
\end{proof}

\begin{theorem}
Let $E\subset\RR^{n}$. The following are equivalent:
\begin{enumerate}
\item $E$ is closed and bounded;
\item $E$ is compact;
\item Every infinite subset of $E$ has a limit point in $E$.
\end{enumerate}
\end{theorem}

\begin{remark}
We won't prove it, but some remarks.
\begin{enumerate}
  \item We have shown $(1)\implies(2)$ already.
  \item The equivalence $(1)\iff(2)$ is the Heine--Borel Theorem.
  \item The equivalence $(1)\iff(3)$ is the Bolzano--Weierstrass Theorem.
\end{enumerate}
\end{remark}

\begin{example}
Consider $(\QQ,|\cdot|)$. The set $E=\{q\in\QQ\mid\sqrt{2}<q<\pi\}$,
we see $E=\QQ\cap(\sqrt{2},\pi)$ is closed and bouunded in $\QQ$, but
it is not compact.
\end{example}

\begin{definition}
Let $E\subset X$. We call $E$ \define{Totally Bounded} if for each
$\varepsilon>0$, $E$ can be covered by a finite number of open balls
of radius $\varepsilon$.

Observe: totally bounded sets are bounded (but the converse is not true).
\end{definition}

\begin{example}
Consider $(\ell^{2},\|\cdot\|)$, which consists of the set of
sequences $(x_{n})$ such that $x_{n}^{2}$ is summable:
\begin{equation}
\sum^{\infty}_{n=0}x_{n}^{2}<\infty.
\end{equation}
Then the closed unit ball,
\begin{equation}
B=\{(x_{n})\in\ell^{2}\mid\|(x_{n})\|\leq1\},
\end{equation}
is bounded. But it is not totally bounded. For each $n\in\NN$, look at
the sequence
\begin{equation}
e_{(n),k} = \begin{cases}1 & \mbox{if }k=n\\
0 & \mbox{otherwise},
\end{cases}
\end{equation}
where $(n)$ is notation tracking which entry is 1, and $k$ is the
index for the entry in the sequence.Then the distance
\begin{equation}
\|e_{(n)}-e_{(m)}\|=\sqrt{2}.
\end{equation}
If $\varepsilon<1$, then each ball of radius $\varepsilon$ would
contain only 1 of the $e_{(n)}$. But there are infinitely many
$e_{(n)}$ (one for each natural number) which live in the unit
ball. So there is no finite number of $\varepsilon$-balls which could
even cover the unit ball of $\ell^{2}$.
\end{example}