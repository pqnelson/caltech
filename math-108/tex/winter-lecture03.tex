%%
%% winter-lecture03.tex
%% 
%% Made by Alex Nelson <pqnelson@gmail.com>
%% Login   <alex@lisp>
%% 
%% Started on  2026-01-10T12:31:09-0800
%% Last update 2026-01-10T12:31:09-0800
%% 

\lecture{}

We want to develop the infrastructure to introduce $\sigma$-algebras,
which are ``safe spaces'' for performing integration.

\begin{proposition}
If $E$ is any set with $m_{*}(E)=0$, then it is measurable.
\end{proposition}

\begin{proof}
Let $E$ be a set with $m_{*}(E)=0$. Let $\varepsilon>0$, there exists
a covering of cubes $\{Q_{j}\}$ such that $E\subset\bigcup_{j}Q_{j}$ and
$\sum_{j}|Q_{j}|<\varepsilon$. We want to find an open set $O$ such
that $E\subset O$ and $m_{*}(O\setminus E)\leq\varepsilon$.

We could try
\begin{equation}
O\stackrel{?}{=}\bigcup_{j}\Interior(Q_{j}),
\end{equation}
which is the union of open sets and therefore open. Well, this won't
always work: consider the situation where $E$ is in the boundary of
one of the cubes, we run into problems. So we need to ``enlarge'' the
cubes using our $\varepsilon$-trick, and use the interiors of the
``enlarged'' covering. Specifically, we pick cubes $Q'_{j}$ such that
$Q_{j}\subset\Interior(Q'_{j})$ and $|Q'_{j}|\leq(1+\delta)|Q_{j}|$
for some sufficiently small $0<\delta\ll1$. Then
\begin{equation}
O = \bigcup_{j}\Interior(Q'_{j}),
\end{equation}
and we see $m_{*}(O)\leq2\varepsilon$ for sufficiently small $\delta$.

We see that $E\subset O$, so we just need to check
\begin{equation}
m_{*}(O\setminus E)\leq m_{*}(O)\leq 2\varepsilon,
\end{equation}
which satisfies the definition of Lebesgue measurable by rescaling
$\varepsilon$. 
\end{proof}

\begin{proposition}[Countable union of measurable sets is measurable]\label{prop:winter26:lecture3:3}
Let $\{E_{j}\}$ be a countable family of measurable sets. Then
$E=\bigcup_{j}E_{j}$ is measurable.
\end{proposition}

\begin{proof}
Let $\varepsilon>0$. We can find an open set $O_{j}\supset E_{j}$ such that
$m_{*}(O_{j}\setminus E_{j})\leq \varepsilon/2^{j}$. Then
\begin{equation}
O=\bigcup_{j}O_{j}.
\end{equation}
We claim this is the open set we're seeking.

\textsc{Claim 1:} $E\subset O$. This is obvious since each component
$E_{j}\subset O_{j}$.

\textsc{Claim 2:} $m_{*}(O\setminus E)\leq\varepsilon$. We see
\begin{equation}
O\setminus E=\left(\bigcup_{j}O_{j}\right)\setminus\left(\bigcup_{j}E_{j}\right)\subset\bigcup_{j}(O_{j}\setminus E_{j}).
\end{equation}
Then by subadditivity
\begin{equation}
m_{*}(O\setminus E)\leq\sum^{\infty}_{j=1}m_{*}(O_{j}\setminus E_{j})=\sum_{j}\frac{\varepsilon}{2^{j}}=\varepsilon.
\end{equation}
Hence the result.
\end{proof}

\begin{lemma}\label{lemma:winter:lec03:positive-distance}
Let $A\subset\RR^{d}$ and $B\subset\RR^{d}$. If $d(A,B)>0$, then
$m_{*}(A\cup B)=m_{*}(A)+m_{*}(B)$.
\end{lemma}

\begin{proof}
Recall $d(A,B)=\inf\{d(a,b)\mid a\in A,b\in B\}$. Then we can pick a
covering of $A$ which is disjoint from the covering of $B$.
\end{proof}

\begin{definition}
Let $A$, $B\subset\RR^{d}$. We say $A$ and $B$ are \define{Almost Disjoint}
if they overlap at most only on boundaries $A\cap B\subset\boundary A$ and $A\cap B\subset\boundary B$.
(Observe: disjoint sets are almost disjoint.)
\end{definition}

\begin{proposition}
Closed sets are measurable.
\end{proposition}

\begin{proof}
  Let $F\subset\RR^{d}$ be a closed set.
  
\textsc{Case 1:} $F$ is unbounded. Then
\begin{equation}
F = \bigcup^{\infty}_{n=1}(F\cap[-n,n]^{d}).
\end{equation}
So we have written $F$ as a countable union of compact closed sets,
which we will prove is measurable. Suffices to prove compact closed
sets are measurable thanks to Proposition~\ref{prop:winter26:lecture3:3}.

\textsc{Case 2:} $F$ is compact. Let $\varepsilon>0$.
Pick an open set $O\supset F$ such
that $m_{*}(O)\leq m_{*}(F)+\varepsilon$.

We see $O\setminus F=O\cap F^{\complement}$ is open, and we can write
\begin{equation}
O\setminus F=\bigcup^{\infty}_{j=1}Q_{j},
\end{equation}
where $Q_{j}$ are almost disjoint closed cubes by our structure theorem.
For any $N\in\NN$, we write
\begin{equation}
K_{N}:=\bigcup^{N}_{j=1}Q_{j},
\end{equation}
which is closed and bounded (hence $K_{N}$ is compact for any
$N\in\NN$). We see that $K_{N}\cap F=\emptyset$ are disjoint since
$K_{N}\subset O\setminus F$.

Since $F$ and $K_{N}$ are disjoint and both are compact, the distance
$d(K_{N},F)$ between $K_{N}$ and $F$ is positive. Define
\begin{equation}
g(x)=d(x,F).
\end{equation}
Then (from lecture 8 of Fall quarter,
Example~\ref{ex:fall:lec08:metric-is-continuous}) this function is
continuous. Therefore $g(x)$ attains both its minima and maxima by the
extreme value theorem. We write $x^{*}\in K_{N}$ for the point where
$g(x^{*})$ is its minimum. Since $K_{N}$ is disjoint from $F$,
$x^{*}\notin F$. But since $F$ is closed, $d(x^{*},F)\neq0$ ---
otherwise $x^{*}$ would be a limit point of $F$, and since $F$ is
closed this would mean $x^{*}\in F$ (contradicting $F$ being disjoint
from $K_{N}$). ence $\min(g)\geq0$, which establishes the key fact
that the distance between $K_{N}$ and $F$ is positive.

Applying Lemma~\ref{lemma:winter:lec03:positive-distance} to $K_{N}$
and $F$,
\begin{equation}
m_{*}(F\cup K_{N})=m_{*}(F)+m_{*}(K_{N}).
\end{equation}
Since $K_{N}\cup F\subset O$, this means (by monotonicity)
\begin{equation}
m_{*}(F\cup K_{N})\leq m_{*}(O),
\end{equation}
and so
\begin{equation}
m_{*}(F)+m_{*}(K_{N})\leq m_{*}(O)\leq m_{*}(F)+\varepsilon.
\end{equation}
Since $F$ is compact, $m_{*}(F)$ is finite, so we can subtract through
by it to find
\begin{equation}
m_{*}(K_{N})\leq\varepsilon.
\end{equation}
Then
\begin{equation}
\sum^{\infty}_{j=1}|Q_{j}|\leq\varepsilon,
\end{equation}
since we did not depend on $N$ anywhere.

We have
\begin{equation}
O\setminus F=\bigcup^{\infty}_{j=1}Q_{j},
\end{equation}
so taking the exterior measure for both sides
\begin{equation}
m_{*}(O\setminus F)\leq\sum^{\infty}_{j=1}|Q_{j}|\leq\varepsilon,
\end{equation}
by countable subadditivity and the preceding results. Hence the claim.
\end{proof}

\begin{proposition}
The complement of a measurable set is measurable.
\end{proposition}

\begin{proof}
Let $E$ be a measurable set. For each $n\in\NN$, choose an open set
$O_{n}$ such that $O_{n}\supset E$ and $m_{*}(O_{n}\setminus E)\leq 1/n$.

Let $F_{n}=O_{n}^{\complement}$. Then $F_{n}$ is measurable and
$F_{n}\subset E^{\complement}$.

Let
\begin{equation}
S=\bigcup^{\infty}_{n=1}F_{n}.
\end{equation}
Then $F_{n}$ is measurable implies $S$ is measurable. We see $S\subset E^{\complement}$.
For any $n$, note that $E^{\complement}\setminus S\subset E^{\complement}F_{n}=O_{n}\setminus E$,
so by monotonicity
\begin{equation}
m_{*}(E^{\complement}\setminus S)\leq m_{*}(O_{n}\setminus E)\leq \frac{1}{n}.
\end{equation}
Letting $n\to\infty$, $m_{*}(E^{\complement}\setminus S)=0$. We can
conclude
\begin{equation}
E^{\complement}=S\cup(E^{\complement}\setminus S),
\end{equation}
and $S$ is measurable and $E^{\complement}\setminus S$ is measurable,
so $E^{\complement}$ is measurable.
\end{proof}

\begin{fact}
Null sets---i.e., sets $N$ where $m_{*}(N)=0$---are measurable.
\end{fact}

\begin{node}
We will get to $\sigma$-algebras, collections of sets which are closed
under complements and countable unions. Specifically, we will want
$\sigma$-algebras of measurable sets, which describes where we can do
integration. 
\end{node}

\begin{remark}
We are describing measures by looking at open sets ``bigger'' than the
set we want to measure, and then looking for smaller and smaller such
open sets. There is another way to approach measure theory, where we
consider closed subsets of $E$ and we look for ``bigger and bigger''
closed subsets of $E$. These two methods of defining measures are
equivalent.
\end{remark}