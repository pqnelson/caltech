%%
%% winter-lecture18.tex
%% 
%% Made by Alex Nelson <pqnelson@gmail.com>
%% Login   <alex@lisp>
%% 
%% Started on  2026-02-14T08:33:42-0800
%% Last update 2026-02-14T08:33:42-0800
%% 

\lecture{}

\begin{definition}
A \define{Hilbert Space} over a field $\FF$ (usually taken to be $\CC$
or $\RR$ for our purposes) consists of a set $H$ such that:
\begin{enumerate}
\item $H$ is a vector space over $\FF$
\item $H$ has an inner product $(-,-)\colon H\times H\to\FF$ such that
\begin{enumerate}
\item $f\mapsto(f,g)$ is linear in its first entry
\item conjugate symmetry: $(f,g)=\overline{(g,f)}$
\item Positive-definiteness: $(f,f)\geq0$, and $(f,f)=0$ iff $f=0$
\end{enumerate}
\item $H$ is complete in the sense of a metric space induced by the
  norm $\|f\|:=\sqrt{(f,f)}$
\item $H$ is separable---i.e., it has a countable dense subset.
\end{enumerate}
\end{definition}

\begin{remark}
If we drop the completeness criterion from the definition of a Hilbert
space, we end up with something called a \define{Pre-Hilbert Space}.
But we can always take the completion of a pre-Hilbert space to
construct a Hilbert space.
\end{remark}

\subsection{Geometry}

\begin{definition}
We say two vectors $f,g\in H$ in a Hilbert space $H$ are
\define{Orthogonal} $f\perp g$ if $(f,g)=0$.
\end{definition}

\begin{theorem}[Pythagorean]
For any orthogonal $f,g\in H$, we have $\|f+g\|^{2}=\|f\|^{2}+\|g\|^{2}$.
\end{theorem}

\begin{theorem}[Parallelogram law]
For any $f,g\in H$, we have
\begin{equation*}
\|f+g\|^{2}+\|f-g\|^{2}=2(\|f\|^{2}+\|g\|^{2}).
\end{equation*}
\end{theorem}

\begin{remark}
If we have a norm which satisfies the Parallelogram law, then the norm
is induced from an inner product. Moreover, this can be used to test
if we are working with a Hilbert space ``in disguise''.
\end{remark}

\subsection{Bases}

\begin{node}
Recall, a basis is a set which spans the vector space.
\end{node}

\begin{theorem}
Let $\{e_{k}\}_{k\in\NN}$ be an orthonormal set for a Hilbert space
$H$ (so $(e_{i},e_{j})=\delta_{i,j}$ for all $i,j\in\NN$). Then the
following are equivalent:
\begin{enumerate}
\item Density: finite linear combinations of $\{e_{k}\}_{k\in\NN}$ are
  dense in $H$;
\item Complete as a basis: If $(f,e_{k})=0$ for all $k\in\NN$, then $f=0$;
\item For all $f\in H$, if $s_{N}(f)=\sum^{N}_{k=1}(f,e_{k})e_{k}$,
  then $s_{N}(f)\to f$ in norm;
\item Parsevel's identity: for all $f\in H$, $\|f\|^{2}=\sum^{\infty}_{k=1}|(f,e_{k})|^{2}$
\end{enumerate}
\end{theorem}

\subsection{Isomorphism Theorem}

\begin{definition}
Let $H$ and $H'$ be Hilbert spaces (over the same field). We say $H$
and $H'$ are \define{(Isometrically) Isomorphic} (or \emph{Unitarily Equivalent})
if there exists a linear bijection $U\colon H\to H'$ such that
for all $f,g\in H$ we have
\begin{equation}
(U(f),U(g))_{H'}=(f,g)_{H}.
\end{equation}
That is to say, it preserves the inner product.
\end{definition}

\begin{theorem}[Classification]
Any separable infinite-dimensional Hilbert space over $\FF$ is isomorphic to $\ell^{2}(\ZZ,\FF)$.
\end{theorem}

The idea is to look at a basis of a Hilbert space, and write any
vector $f=\sum^{\infty}_{k=1}(f,e_{k})e_{k}$. But then it corresponds
to a sequence $(c_{k})\in\ell^{2}(\NN,\FF)$ [$\FF$-valued sequences
  indexed by $\NN$] with $c_{k}=(f,e_{k})$ the
coefficients of the basis vector for $f$. This gives us an isomorphism
of $H$ with $\ell^{2}(\NN,\FF)$. We can use the bijection $\NN\to\ZZ$ to
reindex the sequences giving us $\ell^{2}(\ZZ,\FF)$ as desired.
The inner product on $\ell^{2}(\ZZ,\FF)$ is given by the ``dot product''.