%%
%% fall-lecture22.tex
%% 
%% Made by Alex Nelson <pqnelson@gmail.com>
%% Login   <alex@lisp>
%% 
%% Started on  2025-11-25T10:25:11-0800
%% Last update 2025-11-25T10:25:11-0800
%% 

\lecture{}

\begin{node}[Review of 1-dimensional situation]
We learned in calculus if we have a function of two variables $f(x,y)$
and we have $f(x,y)=0$, then we can write one variable in terms of the
other if the partial derivative of $f(x,y)$ with respect to the one
variable is nonzero.
\end{node}

\begin{theorem}[Linear implicit function theorem]
Let $A\in\mathcal{L}(\RR^{n+m},\RR^{n})$ and suppose $A_{x}$ is invertible.
Then for each $\vec{k}\in\RR^{m}$ there exists a unique
$\vec{h}\in\RR^{n}$ such that $A(\vec{h},\vec{k})=\vec{0}$ and $\vec{h}=-(A_{x})^{-1}A_{y}\vec{k}$.
\end{theorem}

\begin{proof}
By direct calculation,
\begin{align*}
A(\vec{h},\vec{k})=\vec{0}
&\iff A_{x}\vec{h}+A_{y}\vec{k}=0\\
&\iff A_{x}\vec{h}=-A_{y}\vec{k}\\
&\iff \vec{h}=-(A_{x})^{-1}A_{y}\vec{k}
\end{align*}
Hence the result.
\end{proof}

\begin{theorem}[Implicit function theorem]
Let $\vec{f}\colon E\to\RR^{n}$ be a continuously differentiable
function in an open set $E\subset\RR^{n+m}$ such that
$\vec{f}(\vec{a},\vec{b})=\vec{0}$ for some $(\vec{a},\vec{b})\in E$.
Let $A=\vec{f}'(\vec{a},\vec{b})$, and assume $A_{x}$ is invertible.
Then
\begin{enumerate}
\item There exists some open subset $U\subset\RR^{n+m}$ and $W\subset\RR^{m}$
  where $(\vec{a},\vec{b})\in U$ and $\vec{b}\in W$ and for each
  $\vec{y}\in W$ there exists a unique $(\vec{x},\vec{y})\in U$
  satisfying $\vec{f}(\vec{x},\vec{y})=\vec{0}$; and
\item If this $\vec{x}$ is defined to be $\vec{g}(\vec{y})$, then
  $\vec{g}\colon W\to\RR^{n}$ is continuously differentiable,
  satisfies $\vec{g}(\vec{b})=\vec{a}$, and for all $\vec{y}\in W$ we
  have $\vec{f}(\vec{g}(\vec{y}),\vec{y})=\vec{0}$ and $\vec{g}'(\vec{b})=-(A_{x})^{-1}A_{y}$.
\end{enumerate}
\end{theorem}

We just explicitly draw attention to the two claims being made.

\begin{proof}
Define $\vec{F}\colon E\to\RR^{n+m}$ by
\begin{equation}
\vec{F}(\vec{x},\vec{y})=\bigl(\vec{f}(\vec{x},\vec{y}),\vec{y}\bigr)
\end{equation}
for $(\vec{x},\vec{y})\in E$. Then $\vec{F}$ is continuously
differentiable in $E$.

\textsc{Step 1: $\vec{F}'(\vec{a},\vec{b})$ is invertible.} Since $\vec{f}(\vec{a},\vec{b})=\vec{0}$,
\begin{subequations}
  \begin{align}
\vec{f}(\vec{a}+\vec{h},\vec{b}+\vec{k})-\vec{f}(\vec{a},\vec{b})
&=\vec{f}'(\vec{a},\vec{b})(\vec{h},\vec{k})+\vec{r}(\vec{h},\vec{k})\\
&=A(\vec{h},\vec{k})+\vec{r}(\vec{h},\vec{k})
  \end{align}
\end{subequations}
by definition of the differential. Then
\begin{subequations}
  \begin{align}
\vec{F}(\vec{a}+\vec{h},\vec{b}+\vec{k})-\vec{F}(\vec{a},\vec{b})
&=\bigl(\vec{f}(\vec{a}+\vec{h},\vec{b}+\vec{k}),\vec{b}+\vec{k}\bigr)-(\vec{0},\vec{b})\\
&=\bigl(\vec{f}(\vec{a}+\vec{h},\vec{b}+\vec{k}),\vec{k}\bigr)\\
&=\bigl(A(\vec{h},\vec{k})+\vec{r}(\vec{h},\vec{k}),\vec{k}\bigr)\\
&=\bigl(A(\vec{h},\vec{k}),\vec{k}\bigr)+\bigl(\vec{r}(\vec{h},\vec{k}),\vec{0}\bigr)
  \end{align}
\end{subequations}
which is written as the sum of a linear part and the remainder term.
Then $\vec{F}'(\vec{a},\vec{b})$ is the linear map
\begin{equation}
(\vec{h},\vec{k})\mapsto\bigl(A(\vec{h},\vec{k}),\vec{k}\bigr).
\end{equation}
If $\bigl(A(\vec{h},\vec{k}),\vec{k}\bigr)=\vec{0}$, then
$\vec{k}=\vec{0}$ and also $A(\vec{h},\vec{0})=\vec{0}$. This implies
$\vec{F}'(\vec{a},\vec{b})$ is injective (since its kernel is zero).

The domain and codomain of $\vec{F}'(\vec{a},\vec{b})$ have the same
dimension. This implies $\vec{F}'(\vec{a},\vec{b})$ is
surjective. Hence $\vec{F}'(\vec{a},\vec{b})$ is surjective. Hence $A$
is a bijection.

\textsc{Step 2: find $g$.} We apply the inverse function theorem to
$\vec{F}$. There exists open subsets $U\subset\RR^{n+m}$ and
$V\subset\RR^{n+m}$ such that $(\vec{a},\vec{b})\in U$ and
$(\vec{0},\vec{k})\in V$ and $\vec{F}\colon U\to V$ bijective. Let
\begin{equation}
W :=\{\vec{y}\in\RR^{m}\mid(\vec{0},\vec{y})\in V\}.
\end{equation}
Since $V$ is open, $W$ is open. (Why? ``Continuity''. Think about it!)

\textsc{Step 3.}
We define $\vec{g}(\vec{y})$ such that $(\vec{g}(\vec{y}),\vec{y})\in U$
and $\vec{f}(\vec{g}(\vec{y}),\vec{y})=\vec{0}$. Then
\begin{equation}
\vec{F}(\vec{g}(\vec{y}),\vec{y})=(\vec{0},\vec{y}).
\end{equation}
The inverse function
\begin{equation}
\vec{G}\colon V\to U,
\end{equation}
by the inverse mapping theorem, $\vec{G}$ is continuously
differentiable and for all $\vec{y}\in W$,
\begin{equation}
\vec{G}(\vec{0},\vec{y})=(\vec{g}(\vec{y}),\vec{y}).
\end{equation}
Then $\vec{g}$ is continuously differentiable.

\textsc{Step 4.} Define $\Phi(\vec{y}):=(\vec{g}(\vec{y}),\vec{y})$. Then
\begin{subequations}
  \begin{align}
\Phi(\vec{y}+\vec{k})-\Phi(\vec{y})
&=(\vec{g}(\vec{y}+\vec{k}),\vec{y}+\vec{k})-(\vec{g}(\vec{y}),\vec{y})\\
&=(\vec{g}(\vec{y}+\vec{k})-\vec{g}(\vec{y}),\vec{k})\\
&=\bigl(\vec{g}'(\vec{y})\vec{k},\vec{k}\bigr)+(\vec{s}(\vec{k}),\vec{0}),
  \end{align}
\end{subequations}
where $(\vec{s}(\vec{k}),\vec{0})$ is the remainder term. Then for all
$\vec{y}\in W$ and $\vec{k}\in\RR^{m}$,
\begin{equation}
\Phi'(\vec{y})\vec{k}=\bigl(\vec{g}'(\vec{y})\vec{k},\vec{k}\bigr).
\end{equation}
Then for all $\vec{y}\in W$, we see
$\vec{f}(\Phi(\vec{y}))=\vec{0}$. Then we may apply the chain rule
\begin{equation}\label{eq:fall2025-lec22:math108a:eq-star}
\vec{f}'(\Phi(\vec{y}))\Phi'(\vec{y}).
\end{equation}
We substitute $\phi(\vec{b})=(\vec{a},\vec{b})$ since $\vec{g}(\vec{b})=\vec{a}$,
and we find Eq~\eqref{eq:fall2025-lec22:math108a:eq-star} becomes
\begin{equation}
A\Phi'(\vec{b})=\vec{0}.
\end{equation}
Then
\begin{subequations}
  \begin{align}
A\Phi'(\vec{b})\vec{k}
&=A(\vec{g}'(\vec{b})\vec{k},\vec{k})\\
&=A_{x}\vec{g}'(\vec{b})\vec{k} + A_{y}\vec{k}\\
&=\vec{0}.
  \end{align}
\end{subequations}
Then $A_{x}\vec{g}'(\vec{b})\vec{k} + A_{y}\vec{k}=\vec{0}$ and
$A_{x}$ is invertible (by hypothesis) gives us
\begin{equation}
\vec{g}'(\vec{b})=-(A_{x})^{-1}A_{y},
\end{equation}
as desired.
\end{proof}