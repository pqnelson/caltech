%%
%% winter-lecture17.tex
%% 
%% Made by Alex Nelson <pqnelson@gmail.com>
%% Login   <alex@lisp>
%% 
%% Started on  2026-02-12T10:13:58-0800
%% Last update 2026-02-12T10:13:58-0800
%% 

\lecture{}

\begin{definition}[$L^{2}$]
The space $L^{2}(\RR^{d})$ consists of all complex-valued measurable functions
$f\colon\RR^{d}\to\CC$ such that
\begin{equation}
\int_{\RR^{d}}|f(x)|^{2}\,\D x<\infty.
\end{equation}
Furthermore, it has a norm
\begin{equation}
\|f\|_{L^{2}}=\left(\int_{\RR^{d}}|f|^{2}\D x\right)^{1/2}
\end{equation}
and a pairing
\begin{equation}
(f,g)=\int_{\RR^{d}}f(x)\overline{g(x)}\,\D x.
\end{equation}
Note: the elements of $L^{2}$ are equivalence classes of functions
which are the same almost everywhere $[f]=\{g\colon\RR^{d}\to\CC\mid f=g~\mbox{a.e.}\}$.
\end{definition}

\begin{proposition}
\begin{enumerate}
\item $L^{2}$ is a vector space
\item\textsc{Cauchy-Schwarz inequality:} $|(f,g)|\leq\|f\|_{L^{2}}\|g\|_{L^{2}}$
\item\textsc{Triangle inequality:} $\|f+g\|\leq\|f\|+\|g\|$.
\end{enumerate}
\end{proposition}

\begin{proof}
We will prove the Cauchy-Schwarz inequality.

\textsc{Case 1:} $\|f\|=0$ or $\|g\|=0$. Then $f=0$ almost everywhere
(resp., $g=0$ almost everywhere), so $(f,g)=0$.

\textsc{Case 2:} $\|f\|=\|g\|=1$. Then using the fact
\begin{equation}
2|AB|\leq|A|^{2}+|B|^{2},
\end{equation}
we get
\begin{equation}
2|f(x)g(x)|\leq|f(x)|^{2}+|g(x)|^{2}.
\end{equation}
Integrating this through (and observing $\|f\|^{2}=\|f\|$ and
$\|g\|^{2}=\|g\|$) gives us
\begin{equation}
|(f,g)|\leq\int|f(x)g(x)|\,\D x\leq\int2|f(x)g(x)|\,\D
x\leq\int|f(x)|^{2}+|g(x)|^{2}\,\D x=2.
\end{equation}
Hence $|(f,g)|\leq\|f\|\cdot\|g\|$.

\textsc{Case 3:} For any nonzero $f$ and $g$, we introduce
\begin{equation}
\widetilde{f}=\frac{f}{\|f\|},\quad\mbox{and}\quad\widetilde{g}=\frac{g}{\|g\|}.
\end{equation}
Then we see $|(\widetilde{f},\widetilde{g})|\leq1$ by the previous
case, which implies
\begin{equation}
\frac{|(f,g)|}{\|f\|\cdot\|g\|}\leq1.
\end{equation}
Hence the claim.
\end{proof}

\begin{theorem}
$L^{2}$ is complete in its metric.
\end{theorem}

The proof is remarkably similar to the proof that $L^{1}$ is complete.

\begin{theorem}[Separability]
$L^{2}(\RR^{d})$ is separable (i.e., there exists a countable dense
  subset of $L^{2}(\RR^{d})$).
\end{theorem}

\begin{proof}[Proof idea]
The idea is to take $g_{n}=f\chi_{B(r=n,0)\cap\{|f|\leq n\}}$ as a
sequence of bounded compactly-supported functions, show
$\|f-g_{n}\|\to0$, and then approximate $g_{n}$ by step functions
$\psi_{n}$ where $\|g_{n}-\psi_{n}\|<\varepsilon$.
We then approximate the $\psi_{n}$ by a ``rational
approximation'' $\widetilde{\psi}_{n}$ consisting of coefficients
$c_{n}\in\QQ+\I\QQ$ and using rectangles whose vertices lie in $(\QQ+\I\QQ)^{d}$.
These $\widetilde{\psi}_{n}$ enjoy the property that $\|\psi_{n}-\widetilde{\psi}_{n}\|<\varepsilon$,
and the triangle inequality gives us
\begin{equation}
\|f-\widetilde{\psi}_{n}\|\leq\|f-g_{n}\|+\|g_{n}-\psi_{n}\|+\|\psi_{n}-\widetilde{\psi}_{n}\|<3\varepsilon.
\end{equation}
Redefining $\varepsilon'=\varepsilon/3$ and using $\varepsilon'$
everywhere gives us the claim.
\end{proof}
