%%
%% winter-lecture21.tex
%% 
%% Made by Alex Nelson <pqnelson@gmail.com>
%% Login   <alex@lisp>
%% 
%% Started on  2026-02-22T09:24:36-0800
%% Last update 2026-02-22T09:24:36-0800
%% 

\lecture[Bounded linear transformations]{}

\begin{definition}
A \define{Linear Transformation} $T\colon\mathcal{H}\to\mathcal{H}'$
is a linear map preserving the vector space structure.

A \define{Linear Operator} is a linear transformation whose domain is
its codomain $T\colon\mathcal{H}\to\mathcal{H}$.
\end{definition}

\begin{remark}
For finite-dimensional Hilbert spaces $\mathcal{H}$ and $\mathcal{H}'$,
linear transformations $T\colon\mathcal{H}\to\mathcal{H}'$ are
automatically continuous. This is \emph{not} necessarily true for
infinite-dimensional Hilbert spaces.
\end{remark}

\begin{example}
Consider $L^{2}(\RR)$. Then monomials $\{e_{n}:=x^{n}\mid n\in\NN_{0}\}$ is a basis
for the space. The derivative is a linear operator
\begin{subequations}
  \begin{align}
\frac{\D}{\D x}e_{n+1}&=(n+1)\cdot e_{n}\\
\intertext{and}
\frac{\D}{\D x}e_{0}&=0
  \end{align}
\end{subequations}
But as $n\to\infty$, we see
\begin{equation}
\lim_{n\to\infty}\frac{\D}{\D x}e_{n}\mbox{ becomes unbounded}.
\end{equation}
We need a condition to impose some stability.
\end{example}

\begin{definition}
\begin{enumerate}
\item A linear transformation $T\colon\mathcal{H}\to\mathcal{H}'$ is
  called \define{Bounded} if there exists a positive constant $M>0$
  such that for all $f\in\mathcal{H}$ we have
  $\|T(f)\|_{\mathcal{H}'}\leq M\cdot\|f\|_{\mathcal{H}}$
\item The \define{Operator Norm} of a [bounded] linear operator $T\colon\mathcal{H}\to\mathcal{H}'$
  is the non-negative number
  \begin{equation}
\|T\|_{\text{op}}=\sup_{\substack{f\in\mathcal{H}\\f\neq0}}\frac{\|T(f)\|_{\mathcal{H}'}}{\|f\|_{\mathcal{H}}}=\sup_{\substack{f\in\mathcal{H}\\\|f\|_{\mathcal{H}}=1}}\|T(f)\|_{\mathcal{H}'}.
  \end{equation}
  [Strictly speaking, unbounded linear operators do not have a
    well-defined operator norm.]
\end{enumerate}
\end{definition}

\begin{remark}
When $T\colon\mathcal{H}\to\mathcal{H}'$ is a linear transformation,
we will write $\|T\|$ instead of $\|T\|_{\text{op}}$ for the operator
norm when it is clear.
\end{remark}

\begin{proposition}
A linear transformation $T\colon\mathcal{H}\to\mathcal{H}'$ is bounded
if and only if $T$ is continuous.
\end{proposition}

\begin{proof}
\forwardproof\ Assume $T$ is bounded. Then for any $f,g\in\mathcal{H}$,
\begin{subequations}
  \begin{align}
\|T(f)-T(g)\|_{\mathcal{H}} &= \|T(f-g)\|_{\mathcal{H}}\quad\mbox{by linearity}\\
&\leq M\cdot\|f-g\|_{\mathcal{H}}\quad\mbox{by boundedness},
  \end{align}
\end{subequations}
If $f\to g$, then $\|f-g\|\to0$, and so we have $T(f)\to T(g)$.

\backwardproof\ Assume $T$ is continuous. Assume for contradiction
that $T$ is not bounded. Then for every $n\in\NN$, there exists
$f_{n}\in\mathcal{H}$ such that
\begin{equation}
\|T(f)\|_{\mathcal{H}'}>n\|f_{n}\|_{\mathcal{H}}.
\end{equation}
Let us now define
\begin{equation}
g_{n}:=\frac{f_{n}}{n\|f_{n}\|_{\mathcal{H}}}.
\end{equation}
Then
\begin{equation}
\|g_{n}\|=\frac{1}{n}.
\end{equation}
Then
\begin{equation}
\|T(g_{n})\|=\frac{1}{n\|f_{n}\|}\|T(f_{n})\|>1.
\end{equation}
So $\|g_{n}\|\to0$ which means $g_{n}\to0$, but $\|T(g_{n})\|$ does
not converge, which contradicts continuity. Hence $T$ must be
considered bounded.
\end{proof}

\begin{remark}
I am unhappy with the \backwardproof, and I think a direct proof may
be given:

Assume $T$ is continuous. Then for $0\in\mathcal{H}$, choose
$\varepsilon=1$, so there should exist a $\delta_{1}>0$ such that for
all $g\in\mathcal{H}$ we have $\|g\|<\delta_{1}$ implies $\|T(g)\|<1$.
Without loss of generality, assume $g\neq0$. Then multiplying through
by $1/\|g\|$ gives us
\begin{equation}
\left\|\frac{T(g)}{\|g\|_{\mathcal{H}}}\right\|_{\mathcal{H}'}<\frac{1}{\|g\|_{\mathcal{H}}}<\frac{1}{\delta_{1}}.
\end{equation}
This means for all $g\in\mathcal{H}$, we have:
\begin{equation}
\|T(g)\|_{\mathcal{H}'}\leq\frac{1}{\delta_{1}}\|g\|_{\mathcal{H}}.
\end{equation}
(We checked the case for $g\neq0$, which satisfies this property by
continuity; for $g=0$, the condition trivially holds.)
Hence $T$ must be bounded with $M=1/\delta_{1}$ as one possible
bounding constant.
\end{remark}

\begin{remark}
Boundedness is often a more natural condition than continuity in many
applications. So in practice, it's easier to check and work with.
\end{remark}

\begin{definition}
\begin{enumerate}
\item A \define{Linear Functional} is just a (linear) map
  $\ell\colon\mathcal{H}\to\CC$ (or more generally,
  $\ell\colon\mathcal{H}\to\FF$ when $\mathcal{H}$ is a Hilbert space
  over $\FF$)
\item A linear functional $\ell$ is \define{Bounded} if there exists
  an $M>0$ such that for all $f\in\mathcal{H}$ we have $|\ell(f)|\leq M\cdot\|f\|_{\mathcal{H}}$.
\end{enumerate}
\end{definition}

\begin{example}
The canonical example: fix $g\in\mathcal{H}$, we define
$\ell\colon\mathcal{H}\to\CC$ by
\begin{equation}
\ell(f):=(f,g)_{\mathcal{H}}.
\end{equation}
Is it bounded?

Well, if it were, then we would have
\begin{equation}
|\ell(f)|=|(f,g)|\leq\|g\|\cdot\|f\|.
\end{equation}
Our $M=\|g\|_{\mathcal{H}}>0$ for nonzero $g\neq0$.

Further, its operator norm is bounded
\begin{equation}
\|\ell\|_{\text{op}}\leq\|g\|_{\mathcal{H}},
\end{equation}
since $f\in\mathcal{H}$ was arbitrary. We can also prove
\begin{equation}
\|\ell\|_{\text{op}}\geq\|g\|_{\mathcal{H}}
\end{equation}
by taking $f=g$. Hence
$\|\ell\|_{\text{op}}=\|g\|_{\mathcal{H}}$. This suggests there is
some kind of duality going on here. We can ask: if I have an arbitrary
continuous linear functional $\ell$, is there a $g\in\mathcal{H}$ such
that $\ell(f)=(f,g)$?
\end{example}

\begin{theorem}[Riesz representation]
If $\ell$ is a continuous linear functional on $\mathcal{H}$, then
there exists a unique $g\in\mathcal{H}$ such that for all $f\in\mathcal{H}$,
$\ell(f)=(f,g)$. Moreover, $\|\ell\|_{\text{op}}=\|g\|_{\mathcal{H}}$.
\end{theorem}

\begin{proof}
\begin{enumerate}
\item\textsc{Existence:} Define
\begin{equation}
N=\{f\in\mathcal{H}\mid\ell(f)=0\}=\ker(\ell).
\end{equation}
Since $\ell$ is continuous, $N$ is a closed subspace [Check it: for
  any sequence $(f_{n})\subset N$ which converges to $f_{n}\to f$,
  if $\ell(f)=0$, then $f\in N$; hence $N$ is closed.]

\textsc{Subcase 1:} $N=\mathcal{H}$. Then $\ell(f)=0$ is the zero
map. Choose $g=0$, and the result follows immediately.

\textsc{Subcase 2:} $N\neq\mathcal{H}$. Then $\mathcal{H}=N\oplus N^{\perp}$
and $N^{\perp}\neq\{0\}$. We can pick an $h\in N^{\perp}$ such that $\|h\|=1$
and $\ell(h)\neq0$. For any $f\in\mathcal{H}$, pick
\begin{equation}
u:=\ell(f)h-\ell(h)f.
\end{equation}
Then, by linearity,
\begin{subequations}
  \begin{align}
\ell(u) &= \ell\bigl(\ell(f)h-\ell(h)f\bigr)\\
&=\ell(f)\ell(h)-\ell(h)\ell(f)\\
&=0,
  \end{align}
\end{subequations}
so $u\in N$. Then $(u,h)=0$. We see
\begin{subequations}
  \begin{align}
0 &= (u,h) = (\ell(f)h-\ell(h)f,h)\\
&=\ell(f)(h,h)-\ell(h)(f,h)\quad\mbox{by linearity in first slot}\\
&=\ell(f)\cdot1-\ell(h)(f,h)\quad\mbox{by }\|h\|^{2}=1.
  \end{align}
\end{subequations}
Then
\begin{subequations}
  \begin{align}
\ell(f) &= \ell(h)(f,h)\\
&=(f,\overline{\ell(h)}h)
  \end{align}
\end{subequations}
by antilinearity in the second slot. Define
\begin{equation}
g:=\overline{\ell(h)}h,
\end{equation}
then for all $f\in\mathcal{H}$,
\begin{equation}
\ell(f)=(f,g).
\end{equation}
Hence existence is proven.

\item\textsc{Uniqueness:} Suppose we have $g_{1}$, $g_{2}$ such that
  for all $f\in\mathcal{H}$ we have
\begin{equation}
\ell(f)=(f,g_{1})=(f,g_{2}).
\end{equation}
Then
\begin{equation}
0=(f,g_{1})-(f,g_{2})=(f,g_{1}-g_{2}).
\end{equation}
But this implies
\begin{equation}
0=(g_{1}-g_{2},g_{1}-g_{2})=\|g_{1}-g_{2}\|^{2}.
\end{equation}
This implies $g_{1}-g_{2}=0$ by positive-definiteness of the norm (or
inner product). Hence $g_{1}=g_{2}$, which is uniqueness.
\end{enumerate}
\end{proof}