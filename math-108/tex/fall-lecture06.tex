%%
%% fall-lecture06.tex
%% 
%% Made by Alex Nelson <pqnelson@gmail.com>
%% Login   <alex@lisp>
%% 
%% Started on  2025-10-12T10:04:46-0700
%% Last update 2025-10-12T10:04:46-0700
%% 

\lecture{}

\begin{node}[Review from last time]
Let $E\subset X$, let $d$ be a metric on $X$. Some definitions introduced:
\begin{enumerate}
\item Open ball $B_{r}(x)=\{y\in X\mid d(x,y)<r\}$.
\item $E$ is \define{Open} if $\forall x\in E\ldotp\exists r>0\ldotp B_{r}(x)\subset E$
\item A \define{Neighborhood} of $x\in X$ is an open set containing $x$
\item We say $x$ is a \define{Limit Point} of $E$ if every
  neighborhood of $x\in N_{x}$ intersects $N_{x}\cap E\ni y\neq x$
  (the neighborhood intersects $E$ in a distinct point)
\item $E$ is called \define{Closed} if it contains all its limit points.
\end{enumerate}
\end{node}

\begin{example}
Let $(\RR,|\cdot|)$ be the real numbers equipped with the absolute
value norm. Let $E=\{1/n\mid n\in\NN\}$ has 1 limit point at zero, but
no $x\in E$ is a limit point. Then $E$ is not closed since $0\notin E$.
But $E$ is not open either. Look at $B_{r}(1/2)=(\frac{1}{2}-r,\frac{1}{2}+r)$
which is not contained in $E$ for any $r>0$.
\end{example}

\begin{example}
Any finite subset of $X$ is closed. Let $E=\{x_{1},\dots,x_{n}\}$. We
will show $X\setminus E$ is open. So for any $x\in X\setminus E$ we
find $r>0$, and $r<\max\{d(x,x_{i})\mid x_{i}\in E\}$. So
$B_{r}(x)\subset E^{\complement}=X\setminus E$. This means
$E^{\complement}$ is open. Hence $E$ is closed.
\end{example}

\begin{remark}
Singleton sets $\{x_{i}\}$ are always closed.
\end{remark}

\begin{puzzle}
Is any countable set of points closed in $X$?
\end{puzzle}

The first example suggests no; the argument breaks down because $r$ is
no longer positive.

\begin{example}
The rational numbers $\QQ$ is neither open nor closed.
\end{example}

\begin{example}
In a discrete metric space $(X,d)$ where
\begin{equation}
d(x,y)=\begin{cases}1 & \mbox{if }x\neq y\\
0 & \mbox{otherwise}
\end{cases}
\end{equation}
Then every singleton is both open and closed. Let $x\in X$, pick
$0<r<1$, then $B_{r}(x)=\{x\}$.
\end{example}

\begin{proposition}
The following are all true:
\begin{enumerate}
\item $X$ and $\emptyset$ are both open and closed
\item $\bigcap^{n}_{i=0}\mathcal{O}_{i}$ the intersection of
  finitely-many open sets is open
\item $\bigcup_{\alpha\in A}\mathcal{O}_{\alpha}$ the union of an
  arbitrary family of open sets is open
\item $\bigcup^{n}_{i=1}C_{i}$ the union of finitely-many closed
  subsets is closed
\item $\bigcap_{\alpha\in A}C_{\alpha}$ the intersection of an
  arbitrary-family of closed sets is closed.
\end{enumerate}
\end{proposition}

\begin{definition}
Let $E\subset X$. We define the \define{Closure} of $E$ to be the set
\begin{equation}
\closure{E}:=E\cup\{x\in X\mid x\mbox{ is a limit point of }E\}.
\end{equation}  
\end{definition}

\begin{proposition}
\begin{enumerate}
\item $\closure{E}$ is closed
\item $\closure{E}$ is the smallest closed subset of $X$ containing
  $E$ (i.e., if $F$ is closed and $E\subset F$, then
  $\closure{E}\subset F$).
\end{enumerate}
\end{proposition}

\begin{proof}
\begin{enumerate}
\item By definition. We do not know \emph{a priori} if the limit
  points of $\closure{E}$ coincides with limit points of $E$. We will
  prove every limit point of $\closure{E}$ is a limit point of $E$.

  Let $x$ be a limit point of $\closure{E}$. Then let $N_{x}$ be a neighborhood
  of $x\in N_{x}\subset X$. Then there exists a $x'\in N_{x}\cap E$
  such that $x'\neq x$ by definition of a limit point. Then either
  $x'$ is in $E$ or $x'$ is a limit point of $E$ --- we will prove
  either way, $N_{x}$ contains an element of $E$.
  (Case 1: $x'$ is a point of $E$, so $N_{x}$ contains an element of $E$.
  Case 2: if $x'$ is a limit point of $E$, then $x'\in N_{x}$, so
  there is a point $x''\in N_{x}\cap E$ and $x''\neq x'$. In this
  situation, $N_{x}$ contains an element of $E$.)
  Hence any limit point $x$ of $\closure{E}$ is a limit point of
  $E$. Hence $x\in\closure{E}$.
\item If $A\subset B$, then $\closure{A}\subset\closure{B}$.
  If $E\subset F$, then $\closure{E}\subset\closure{F}$. But
  $F=\closure{F}$ since $F$ is closed. So $\closure{E}\subset F$.\qedhere
\end{enumerate}
\end{proof}

\begin{definition}
We say $E$ is \define{Dense} in $X$ if every point of $X$ is either
(1) a limit point of $E$ or (2) a point of $E$.
\end{definition}

\begin{proposition}
We see $E$ is dense in $X$ if and only if $\closure{E}=X$.
\end{proposition}

\begin{example}
Every metric space is dense in itself.
\end{example}

\begin{example}
We see $\QQ$ is dense in $(\RR,|\cdot|)$.
\end{example}

\begin{example}
Consider the metric space $([0,1], |\cdot|)$ as a metric subspace of
$(\RR,|\cdot|)$. Then $[0,\frac{1}{2})$ is open in $([0,1],|\cdot|)$
but not in $(\RR,|\cdot|)$.

So when do we know a set $U$ being open in one space implies $U$ is
open its parent space?
\end{example}

\begin{proposition}
Let $E\subset Y\subset X$. Then $E$ is open in $(Y,d_{Y})$ iff there
exists an open subset $\mathcal{O}\subset X$ such that $E=Y\cap\mathcal{O}$.
\end{proposition}

\begin{proof}
$(\Longrightarrow)$ Assume $E$ is open in $(Y,d_{Y})$. Then for each
$x\in E$, there exists a positive $r_{x}>0$ radius depending on $x$
such that $B_{r_{x}}(x)\subset E\cap Y$ an open ball in $Y$. Then let
$V_{x}$ be the set of all $y\in X$ such that $d(x,y)<r_{x}$, so
$V_{x}$ is the $r_{x}$-ball of $x$ in $X$. We define
\begin{equation}
\mathcal{O}=\bigcup_{x\in E}V_{x}
\end{equation}
which is open in $X$. We want to show $E=Y\cap\mathcal{O}$.
For $x\in E$, we see $x\in V_{x}$, so $E\subset\mathcal{O}$.
Then $E\cap Y\subset\mathcal{O}\cap Y$.
On the other hand, $V_{x}\cap Y=B_{r_{x}}(x)\subset E$ for each $x$.
Taking their union, $\mathcal{O}\cap Y\subset E$.
Hence $\mathcal{O}\cap Y=E$.
  
$(\Longleftarrow)$ Given an open subset $\mathcal{O}\subset X$ such
that $E=Y\cap\mathcal{O}$.
For any $x\in E$, there exists a neighborhood $N_{x}\subset\mathcal{O}$.
($N_{x}$ was called $V_{x}$ in the first half of this proof.)
Then $N_{x}\cap Y\subset\mathcal{O}\cap Y\subset E$.
This gives us the neighborhood desired. Hence $E$ is open in $Y$.
\end{proof}