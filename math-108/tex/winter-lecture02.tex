%%
%% winter-lecture02.tex
%% 
%% Made by Alex Nelson <pqnelson@gmail.com>
%% Login   <alex@lisp>
%% 
%% Started on  2026-01-10T12:00:09-0800
%% Last update 2026-01-10T12:00:09-0800
%% 

\lecture{}

\begin{node}
Recall $\chi_{\QQ}$ is not Riemann integrable because the set of its
discontinuities is the entire real line. On the other hand, the Cantor
set $C$ has its characteristic function $\chi_{C}$ be Riemann
integrable (since $C$ has measure zero).
\end{node}

\begin{node}[Wish list for a measure]
We have several desired properties for a measure
$m\colon\powerset{\RR^{d}}\to[0,\infty]$:
\begin{enumerate}
\item $m(\mbox{unit cube})=1$
\item translation invariance: $m(E+h)=m(E)$ for any $h\in\RR^{d}$ and $E\subset\RR^{d}$;
\item countable additivity: if $E=\bigcup^{\infty}_{j=1}E_{j}$ and
  $E_{i}\cap E_{j}=\emptyset$ for any $i\neq j$, then $m(E)=\sum^{\infty}_{j=1}m(E_{j})$.
\end{enumerate}
We try to build this measure, we start with a ``candidate measure''.
\end{node}

\begin{definition}
For any $E\subset\RR^{d}$, the \define{Exterior Measure} of $E$ is
\begin{equation}
m_{*}(E)=\inf\{\sum^{\infty}_{j=1}|Q_{j}|\mid E\subset\bigcup^{\infty}_{j=1}Q_{j},
Q_{j}\mbox{ are closed cubes}\}.
\end{equation}
(We could use closed rectangles or closed balls instead of closed
cubes, but closed cubes simplify some of the proofs.)
\end{definition}

\begin{example}
Let $E=\{x\}$ be a point. For any $\varepsilon>0$, we may cover $x$ by
a cube of sidelength $\varepsilon^{1/d}$ with $x$ in its interior:
\begin{equation}
Q_{\varepsilon}=\prod^{d}_{j=1}[\pi_{j}(x)-\frac{1}{2}\varepsilon^{1/d},\pi_{j}(x)+\frac{1}{2}\varepsilon^{1/d}],
\end{equation}
where $\pi_{j}(x)$ projects the $j^{\text{th}}$ component of the point $x$.
This gives us a covering of volume
\begin{equation}
m_{*}(Q_{\varepsilon})=\varepsilon,
\end{equation}
which was chosen arbitrarily, so $m_{*}(\{x\})=0$.
\end{example}

\begin{example}
Let $E=\{x_{1},x_{2},\dots\}$ be a countable set of points indexed by $\NN$.
Then for each $\varepsilon>0$, we can cover $x_{1}$ by a cube of
volume $\varepsilon/2$, $x_{2}$ by a cube of volume
$\varepsilon/2^{2}$, \dots, $x_{n}$ by a cube of volume $\varepsilon/2^{n}$,
and so on. This gives us a covering of volume
\begin{equation}
m_{*}(E)\leq m_{*}(\bigcup^{\infty}_{j=1}Q_{j})=\sum^{\infty}_{j=1}\frac{\varepsilon}{2^{j}}=\varepsilon.
\end{equation}
Since $\varepsilon$ was arbitrary, this means $m_{*}(E)=0$.
\end{example}

\begin{puzzle}
Let $Q\subset\RR^{d}$ be a cube. Is $m_{*}(Q)=|Q|$ or not?
\end{puzzle}

\begin{theorem}
Let $Q\subset\RR^{d}$ be a closed cube. Then $m_{*}(Q)=|Q|$. 
\end{theorem}

\begin{proof}
\textsc{Claim 1:} $m_{*}(Q)\leq|Q|$. Since $Q$ covers itself, $Q\subset Q$, so $\inf\{\dots\}\leq|Q|$.

\textsc{Claim 2:} $|Q|\leq m_{*}(Q)$. Let $\{Q_{j}\}_{j\in\NN}$ is an
arbitrary covering of $Q$, i.e., suppose we have
$Q\subset\bigcup^{\infty}_{j=1}Q_{j}$.
We want to prove $|Q|\leq\sum_{j}|Q_{j}|$. Each $Q_{j}$ is contained
in an open set $O_{j}$ which is the interior of a ``slightly larger''
cube $C_{j}$ where $\pi_{i}(Q_{j})\propersubset\pi_{i}(\Interior(C_{j}))$.
Then the volume of $C_{j}$ is $(1+\varepsilon/2^{j})^{d}|Q_{j}|$. The
rest of the proof is obvious.
\end{proof}

\begin{node}
Now, we will check if the exterior measure satisfies the desired
properties of a measure.
\end{node}

\begin{proposition}
If $E_{1}\subset E_{2}$, then $m_{*}(E_{1})\leq m_{*}(E_{2})$.
\end{proposition}

\begin{proof}[Proof sketch]
Since every covering of $E_{2}$ is a covering of $E_{1}$, we have
\begin{equation}
\inf\{\sum|Q_{j}|\mid Q_{j}\mbox{ covers }E_{1}\}\leq\inf\{\sum|Q_{j}|\mid Q_{j}\mbox{ covers }E_{2}\}.
\end{equation}
The result follows.
\end{proof}

\begin{proposition}
For any $E\subset\RR^{d}$ and $h\in\RR^{d}$, we have $m_{*}(E+h)=m_{*}(E)$.
\end{proposition}

\begin{proof}
Let $\{Q_{j}\}$ cover $E$. Then $\{Q_{j}+h\}$ covers $E+h$ since
\begin{equation}
E+h\subset\left(\bigcup^{\infty}_{j=1}Q_{j}\right)+h=\bigcup^{\infty}_{j=1}(Q_{j}+h).
\end{equation}
From $\sum_{j}|Q_{j}|=\sum_{j}|Q_{j}+h|$, the result follows.
\end{proof}

\begin{proposition}[Countable subadditivity]
For \emph{any} sequence of sets $\{E_{k}\}^{\infty}_{k=1}$, we have
\begin{equation}
m_{*}\left(\bigcup^{\infty}_{k=1}E_{k}\right)\leq\sum^{\infty}_{k=1}m_{*}(E_{k}).
\end{equation}
\end{proposition}

\begin{proof}
Let $\varepsilon>0$. For each $E_{k}$, choose a covering
$\{Q_{k,j}\}_{j}$ for $E_{k}$ such that
\begin{equation}
\sum^{\infty}_{j=1}|Q_{k,j}|\leq m_{*}(E_{k})+\frac{\varepsilon}{2^{k}}.
\end{equation}
Then the collection $\{Q_{k,j}\mid j\in\NN,k\in\NN\}$ covers
\begin{equation}
E=\bigcup^{\infty}_{k=1}E_{k}\subset\bigcup^{\infty}_{k=1}\bigcup^{\infty}_{j=1}Q_{k,j}.
\end{equation}
Then
\begin{equation}
m_{*}(E)\leq\sum_{k=1}\sum_{j=1}|Q_{k,j}|\leq\sum_{k=1}\left(m_{*}(E)+\frac{\varepsilon}{2^{k}}\right),
\end{equation}
and
\begin{equation}
\sum_{k=1}\left(m_{*}(E)+\frac{\varepsilon}{2^{k}}\right)=\left(\sum_{k=1}m_{*}(E)\right)+\varepsilon.
\end{equation}
Since $\varepsilon>0$ is arbitrary, we have the result
\begin{equation}
m_{*}(E)\leq\sum_{k=1}^{\infty}m_{*}(E_{k}),
\end{equation}
as desired.
\end{proof}

\begin{puzzle}
If $E_{1}$ and $E_{2}$ are disjoint, do we always have
$m_{*}(E_{1})+m_{*}(E_{2})=m_{*}(E_{1}\cup E_{2})$?
\end{puzzle}

\begin{proof}[Answer]
No, this is not always true. We can have pathological subsets where
this is not the case. But we always have $m_{*}(E_{1})+m_{*}(E_{2})\geq m_{*}(E_{1}\cup E_{2})$,
if it's any consolation. 
\end{proof}

\begin{definition}
A subset $E\subset\RR^{d}$ is \define{Lebesgue Measurable} (or just
``measurable'') if for each $\varepsilon>0$ there exists a
corresponding open subset $O\subset\RR^{d}$ such that
\begin{enumerate}
\item it covers $E$: $E\subset O$; and
\item arbitrarily well: $m_{*}(O\setminus E)\leq\varepsilon$.
\end{enumerate}
\end{definition}

\begin{proposition}
Any open set is measurable.
\end{proposition}

\begin{proposition}
The empty set is measurable and $m_{*}(\emptyset)=0$.
\end{proposition}

\begin{proof}
Pick the cube $Q$ of volume $\varepsilon$, then $\Interior(Q)$ has
exterior measure $\varepsilon$, and $\emptyset\subset Q$ always.
\end{proof}