%%
%% main.tex
%% 
%% Made by Alex Nelson <pqnelson@gmail.com>
%% Login   <alex@lisp>
%% 
%% Started on  2025-10-12T10:03:27-0700
%% Last update 2025-10-12T10:03:27-0700
%% 

\documentclass{article}

\usepackage{macros,notation}

\title{Classical Analysis}
\author{Alex Nelson}
\date{October 10, 2025}

\begin{document}
\maketitle
\begin{abstract}
These are my notes from auditing Caltech's Mathematics 108 course
series on classical analysis, Fall 2025.
\end{abstract}

\noindent\textsc{Note:} I only started auditing the Fall course at lecture 6.
So let me review a few important definitions.

\begin{definition}
Let $X$ be a vector space over $\FF$. We define a \define{Norm} on $X$
to be a function $\|\cdot\|\colon X\to\RR$ such that
\begin{enumerate}
\item\textsc{Triangle inequality:} $\|x+y\|=\|x\|+\|y\|$ for any $x$,
  $y\in X$
\item\textsc{Homogeneity:} $\|sx\|=|s|\cdot\|x\|$ for any $s\in\FF$
  and $x\in X$
\item\textsc{Positive-definiteness:} For any $x\in X$, if $\|x\|=0$,
  then $x=0$.
\item\textsc{Non-negative:} For any $x\in X$, $\|x\|\geq0$.
\end{enumerate}
\end{definition}

\begin{definition}
Let $X$ be a set, We define a \define{Metric} on $X$ to be a function $d\colon X\times X\to\RR$
such that
\begin{enumerate}
\item for any $x\in X$, $d(x,x)=0$
\item\textsc{Positivity:} for any $x$, $y\in X$, $d(x,y)\geq0$ and
  $d(x,y)$ iff $x=y$.
\item\textsc{Symmetry:} for any $x$, $y\in X$, $d(x,y)=d(y,x)$
\item\textsc{Triangle inequality:} for any $x$, $y$, $z\in X$, $d(x,y)+d(y,z)\geq d(x,z)$
\end{enumerate}
\end{definition}

\begin{example}
On $\RR$, $d(x,y)=|x-y|$ is a metric.
\end{example}

\begin{example}
More generally, if $(X,\|\cdot\|)$ is a normed space over a field
$\FF$, then $d(x,y)=\|x-y\|$ is a metric.
\end{example}

\begin{example}[Discrete metric]
Let $X$ be any set. We define the \define{Discrete Metric} to be the function
\begin{equation}
d(x,y) = \begin{cases}1 &\mbox{if }x\neq y\\
0 & \mbox{otherwise}
\end{cases}
\end{equation}
Then this is an actual metric.
\end{example}

\setcounter{section}{5}
%%
%% fall-lecture06.tex
%% 
%% Made by Alex Nelson <pqnelson@gmail.com>
%% Login   <alex@lisp>
%% 
%% Started on  2025-10-11T13:49:34-0700
%% Last update 2025-10-11T13:49:34-0700
%% 

\lecture{}

The professor is trying a new approach: every Friday is a more relaxed
discussion than a lecture, with emphasis on puzzles and history.

\begin{definition}
Let $A$ be a ring. If $M$ is a finitely-generated $A$-module, then
there exists an $n\in\NN$ such that
\begin{equation}
A^{n}\xrightarrow{p}M\xrightarrow{q}0
\end{equation}
is a short exact sequence. We can do this for \emph{any} $A$-module,
but non-finitely-generated $A$-modules would have their short exact
sequence look like
\begin{equation}
\bigoplus^{\infty}_{i=0}A_{i}\xrightarrow{p}M\xrightarrow{q}0
\end{equation}
where $A_{i}=A$ is just a copy of $A$ indexed by $i\in\NN_{0}$.

The $\ker(p)$ describes the relations among the generators. We can
extend this short exact sequence to
\begin{equation}
\cdots\to A^{m}\xrightarrow{r}A^{n}\xrightarrow{p}M\xrightarrow{q}0
\end{equation}
such that $\ker(p)=\Im(r)$. This is really a \define{Complex}, it's
exact when $\ker(p)=\Im(r)$ and $\ker(q)=\Im(p)$ and\dots
\end{definition}

\begin{node}
There is another variant of the definition for a Noetherian ring:
A ring $R$ is Noetherian iff every ideal is finitely-generated. (If
$R$ is Noetherian, take the union of the chain of ideals, it is
necessarily finitely-generated.)
\end{node}

\begin{proposition}
A submodule of a finitely-generated module is finitely-generated.
\end{proposition}

\begin{example}
Let $\kk[x_{1},x_{2},\dots]$ be a ring of formal polynomials in
infinitely many unknowns $x_{i}$ indexed by $i\in\NN$. This is not
Noetherian, because we can form the chain of ideals
$I_{k}=(x_{1},\dots,x_{k})$ which will not stabilize.
\end{example}

\begin{node}
Emmy Noether, ``Ideal Theory in Rings'' (English translation
\arXiv{1401.2577}) introduced the notion in \S1 as a finiteness
condition on rings.

Algebraic integers refers to all algebraic elements of $\CC$ adjoined
to $\ZZ$. That is to say, if $f\in\ZZ[x]$ is monic, it's roots
$\alpha\in\CC$ satisfying $f(\alpha)=0$ are adjoined to $\ZZ$. If we
write $\mathcal{O}$ for the integral closure of $\ZZ$, we have
\begin{equation}
\xymatrix@R=1pc{\ZZ\;\ar@{^{(}->}[r]\ar@{}[d]|-*[@]{\rotatebox{-90}{$\subset$}} & \raisebox{0pt}[0.9\height][0.3\height]{$\mathcal{O}$}\ar@{^{(}->}[d] \\
\QQ\;\ar@{^{(}->}[r] & \overline{\QQ}\;\ar@{^{(}->}[r] & \CC}
\end{equation}
where $\overline{\QQ}$ is the algebraic closure of $\QQ$. The integral
closure $\mathcal{O}$ of $\ZZ$ is not Noetherian, just consider the
chain of ideals generated by square roots of primes which is not terminating $(\sqrt{2},\sqrt{3},\sqrt{5},\dots)$.
\end{node}

\begin{node}
Let $A$ be Noetherian, let $M$ be a finitely-generated $A$-module. We
have the short-exact sequence
\begin{equation}
A^{n_{1}}\onto M\to 0.
\end{equation}
But we can form a complex of free modules:
\begin{equation}
\underbrace{\cdots\to A^{n_{2}}\to A^{n_{1}}}_{\text{free modules}}\to M\to 0
\end{equation}
We can see
\begin{equation}
\varphi_{i}\colon A^{n_{i+1}}\to A^{n_{i}}
\end{equation}
and
\begin{equation}
\varphi_{0}\colon A^{n_{1}}\to M
\end{equation}
satisfy the expected relations $\Im(\varphi_{i+1})\subset\ker(\varphi_{i})$.
This is called the \define{Free Resolution} of $M$.
\end{node}

\begin{definition}[Projective modules]
A module $P$ is called \define{Projective} if there exists a module
$Q$ such that $P\oplus Q\iso F$ for some free module $F$.
\end{definition}

\begin{example}
More examples of Noetherian rings: $\ZZ$ is Noetherian; any principal
ideal domain is Noetherian; if $R$ is Noetherian, then $R[x]$ is
Noetherian (and so is $R[[x]]$).

If $R$ is Noetherian and $I\ideal R$, then $R/I$ is Noetherian.
\end{example}

\begin{example}
If $B$ is a Noetherian ring, and $A\subset B$ is a subring, then is
$A$ Noetherian?

The algebraic integers $\mathcal{O}\subset\overline{Q}$ are not
Noetherian despite $\overline{Q}$ being a field.
\end{example}

\begin{example}
Let $A\into B$ be an integral ring extension. If $B$ is Noetherian,
then is $A$ Noetherian?
\end{example}

\begin{node}
Any proper ideal $I$ in a Noetherian ring $A$ has a primary
decomposition: $I=Q_{1}\cap\cdots\cap Q_{n}$ where $Q_{i}$ is a
primary ideal of $A$.
\end{node}

\begin{node}[Puzzle]
We know ``If $Q\ideal A$ is primary, then $\Radical{Q}$ is prime''.

But is the converse true, ``If $\Radical{Q}$ is prime, then $Q$ is primary''?
\end{node}

\begin{example}
Consider $Q=(x^{2},xy)\ideal\CC[x,y]$. Is $Q$ primary?

We have $x\cdot x\in Q$ implies $x^{n}\in Q$ or $x\in Q$ (which is
true).

We have $y\cdot x\in Q$ implies $y\in Q$ or $x^{n}\in Q$ (which is
true).

But $x\cdot y\notin Q$ since this would imply $x\in Q$ or $y^{n}\in Q$
which is not true. This means $Q$ is not primary.

However, $\Radical{Q}=(xy,x)=(x)$ which is prime. This gives a
counter-example to the claim ``$\Radical{Q}$ is prime implies $Q$ is primary''.
\end{example}
%%
%% fall-lecture07.tex
%% 
%% Made by Alex Nelson <pqnelson@gmail.com>
%% Login   <alex@lisp>
%% 
%% Started on  2025-10-14T12:12:24-0700
%% Last update 2025-10-14T12:12:24-0700
%% 

\lecture{}

\begin{example}
For $X=\point{x}$, then for each $n$ there is exactly one singular
$n$-simplex $\sigma\colon\Delta^{n}\to X$ which is the cosntant
map. Then
\begin{equation}
C_{n}(X)\iso\ZZ\quad\mbox{for all }n\geq0.
\end{equation}
The boundary map
\begin{equation}
\begin{split}
\boundary_{n}\colon&C_{n}(X)\to C_{n-1}(X)\\
&\sigma\mapsto\sum^{n}_{i=0}(-1)^{i}\sigma|_{[v_{0},\dots,\widehat{v_{i}},\dots,v_{n}]}
\end{split}
\end{equation}
but all these $\sigma|_{\text{faces}}$ are the same map, so this means
$\boundary_{n}\colon\ZZ\to\ZZ$ corresponds to multiplication by
$0=\sum^{n}_{i=0}(-1)^{i}$ when $n$ is odd, or by $1=\sum(-1)^{i}$
when $n$ is even. So we have
\begin{equation}
\cdots\to C_{3}\xrightarrow{0}C_{2}\iso C_{1}\xrightarrow{0}C_{0}\to 0.
\end{equation}
Then $H_{0}(X)\iso C_{0}\iso\ZZ$, $H_{1}(X)=\ker(\boundary_{1})/\Im(\boundary_{2})\iso0$.
We find
\begin{equation}
H_{n}(\point{x})=\begin{cases}\ZZ & \mbox{if }n=0\\
0 & \mbox{otherwise}
\end{cases}
\end{equation}
\end{example}

\begin{proposition}
If $X_{\alpha}$ are path-connected components of $X$,
then we see that $H_{n}(X) = \bigoplus_{\alpha}H_{n}(X_{\alpha})$.
\end{proposition}

Every singular $n$-simplex must be contained in one of the
path-connected components, so we just need to compute the homology of
those path-connected components.

\begin{proposition}
$H_{0}(X)\iso\bigoplus_{\alpha}\ZZ$ where $X_{\alpha}$ are the
  path-connected components of $X$.
\end{proposition}

\begin{proof}
If $X$ is path-connected, then
\begin{equation}
C_{1}(X)\xrightarrow{\boundary_{1}} C_{0}(X)\xrightarrow{0}0
\end{equation}
so $H_{0}(X)=C_{0}(X)/\Im(\boundary_{1})$. Then $C_{0}(X)$ is freely
generated by $\sigma\colon\Delta^{0}\to X$ but $\Delta^{0}$ is just a
point, so $C_{0}(X)$ is just freely generated by points in $X$.

Then signular $1$-simplex is $\sigma\colon\Delta^{1}\to X$, but this
is homeomorphic to the closed unit interval $\Delta^{1}\iso[0,1]$, so
$\sigma$ is just a path in $X$. Then $\boundary(\sigma)=\sigma(1)-\sigma(0)$.
Then $\Im(\boundary_{1})$ is freely generated by $\sigma(1)-\sigma(0)$
for all paths $\sigma$. But $X$ is path-connected, so \emph{every} two
points is connected by a path: $\forall x,y\in X\ldotp x-y\in\Im(\boundary_{1})$.

Taken together, these facts imply $H_{0}(X)\iso\ZZ$ is generated
freely by a single generator. Hence the result.
\end{proof}

We have exhausted the ability of computing singular homology by
definition, or at least the examples become more and more tedious (and
continuing to work on them becomes increasingly de-motivating). So we
need to build some results. This will involve digressions to
homological algebra!

\subsection{Homotopy Invariance}

\begin{definition}
Let $(C,\boundary)$ and $(C',\boundary')$ be chain complexes, and let
$f,g\colon C\to C'$ be chain maps. We define a \define{Chain Homotopy}
between $f$ and $g$ to be a morphism $H\colon C\to C'$ which can be
decomposed as $H=\bigoplus_{n\in\ZZ}H_{n}$ where $H_{n}\colon C_{n}\to C'_{n+1}$
increases the grading by 1, such that $\boundary'\circ H+H\circ\boundary=f-g$,
or $\boundary'_{n+1}\circ H_{n}+H_{n-1}\circ\boundary_{n}=f_{n}-g_{n}$.

We stress, \emph{a chain homotopy \textbf{is not} a chain map!}
\end{definition}

\begin{remark}
We can look at ``higher homotopies'', which increase the grading by $k>1$.
This is useful in a lot of modern mathematics.
\end{remark}

\begin{theorem}
If $f$ and $g$ are chain homotopic, then
\begin{equation}
f_{*}=g_{*}\colon H_{*}(C)\to H_{*}(C')
\end{equation}
(they induce the same morphism on the chain Homology groups).
\end{theorem}

\begin{proof}
We just need to prove for every cycle $\alpha$ in $C$
that $f(\alpha)$ and $g(\alpha)$ are homologous. We just need to apply
the definition of chain homotopy. We find
\begin{subequations}
\begin{align}
f(\alpha) - g(\alpha) &= \boundary'H(\alpha)+H\boundary(\alpha)\\
&=\boundary'H(\alpha) + 0
\end{align}
\end{subequations}
since $\alpha$ is a cycle. Recall two elements are homologous if their
difference is an element of the image of the boundary map, which we
just proved.
\end{proof}

\begin{proposition}
Let $X$ and $Y$ be topological spaces, let $f,g\colon X\to Y$ be homotopic
maps.
Then $f_{\sharp},g_{\sharp}\colon C_{*}(X)\to C_{*}(Y)$ are chain
homotopic, and therefore $f_{*}=g_{*}\colon H_{*}\to H_{*}(Y)$.
\end{proposition}

\begin{proof}
We want to construct a chain homotopy. Let us call this chain homotopy
\begin{equation}
P\colon C_{*}(X)\to C_{*+1}(Y),
\end{equation}
where we emphasize the grading increases by 1. For any singular
$n$-simplex $\sigma\colon\Delta^{n}\to X$, we want to send it to some
$P(\sigma)$ which is an $(n+1)$-\underline{chain} in $Y$.

Now, we use $f$ and $g$ are homotopic. There exists $F\colon X\times[0,1]\to Y$
such that
\begin{equation}
F|_{X\times\{0\}}=f,\quad\mbox{and}\quad F|_{X\times\{1\}}=g.
\end{equation}
We have $\Delta^{n}\xrightarrow{\sigma}X$ induce
\begin{equation}
\Delta^{n}\times[0,1]\xrightarrow{\widetilde{\sigma}}X\times[0,1]\xrightarrow{F}Y,
\end{equation}
the idea is that $P(\sigma)=F\circ\widetilde{\sigma}$. We view
$\Delta^{n}\times[0,1]$ as a prism
\begin{equation}
\vcenter{\hbox{\includegraphics{img/img.29}}}\xrightarrow{\widetilde{\sigma}}X\times[0,1]\xrightarrow{F}Y
\end{equation}
Now, we triangulate the prism so it is the union of
$(n+1)$-simplices. Then restricting $F\circ\widetilde{\sigma}$ to each
$(n+1)$-simplex in the triangulation gives us a singular
$(n+1)$-simplex. We take the prism as the linear combination of the
singular $(n+1)$-simplices.

If we look at the boundary of the prism $\boundary(\Delta^{n}\times[0,1])$,
it consists of three parts:
\begin{equation}
\boundary(\Delta^{n}\times[0,1])=\bigl((\boundary\Delta^{n})\times[0,1]\bigr)\cup(\Delta^{n}\times\{0\})\cup(\Delta^{n}\cup\{1\}).
\end{equation}
We have $\boundary P+P\boundary=f_{\sharp}+g_{\sharp}$.
We look at the restriction of $F\circ\widetilde{\sigma}$ on these
components, we see
$F\circ\widetilde{\sigma}|_{\text{prism}}=P$ so
\begin{subequations}
\begin{equation}
\boundary(F\circ\widetilde{\sigma}|_{\text{prism}})=\boundary P
\end{equation}
and
\begin{align}
F\circ\widetilde{\sigma}|_{\boundary X\times[0,1]} &= P\boundary\\
F\circ\widetilde{\sigma}|_{X\times\{0\}} &= f_{\sharp}\\
F\circ\widetilde{\sigma}|_{X\times\{1\}} &= g_{\sharp}.
\end{align}
\end{subequations}
Now we can write down the chain homotopy. Formally, we have
\begin{equation}
P(\sigma) = \sum^{n}_{i=0}(-1)^{i}F\circ\widetilde{\sigma}|_{[v_{0},\dots,v_{i},w_{i},\dots,w_{n}]}
\end{equation}
where the vertices on the ``left simplex'' are $v_{0}$, \dots,
$v_{n}$, and the vertices on the ``right simplex'' are $w_{0}$, \dots, $w_{n}$:
\begin{equation}
\vcenter{\hbox{\includegraphics{img/img.30}}}
\end{equation}
For the case when $n=2$, we can draw the situations. When $i=0$, we
get the tetrahedron spanned by $[v_{0},w_{0},w_{1},w_{2}]$:
\begin{equation*}
[v_{0},w_{0},w_{1},w_{2}] = \vcenter{\hbox{\includegraphics{img/img.31}}}
\end{equation*}
For $i=1$, the tetrahedron is spanned by $[v_{0},v_{1},w_{1},w_{2}]$
\begin{equation*}
\vcenter{\hbox{\includegraphics{img/img.32}}}
\end{equation*}
For $i=2$, the tetrahedron is spanned by $[v_{0},v_{1},v_{2},w_{2}]$
\begin{equation*}
\vcenter{\hbox{\includegraphics{img/img.33}}}
\end{equation*}
We see the tetrahedron looks like
\begin{equation*}
i=0:\quad\vcenter{\hbox{\includegraphics{img/img.34}}}
\end{equation*}
\begin{equation*}
i=1:\quad\vcenter{\hbox{\includegraphics{img/img.35}}}
\end{equation*}
\begin{equation*}
i=2:\quad\vcenter{\hbox{\includegraphics{img/img.36}}}
\end{equation*}
\end{proof}

\begin{corollary}
A homotopy equivalence $f\colon X\to Y$ induces an isomorphism
$f_{*}\colon H_{*}(X)\to H_{*}(Y)$.
\end{corollary}

\begin{proof}
There exists a $g\colon Y\to X$ such that $g\circ f\iso\id_{X}$ and
$f\circ g\iso\id_{Y}$. Then by the previous theorem,
$f_{*}\circ g_{*}=\id_{H_{*}(Y)}$ and $g_{*}\circ f_{*}=\id_{H_{*}(X)}$,
which implies $f_{*}$ is an isomorphism of homology groups.
\end{proof}
%%
%% fall-lecture08.tex
%% 
%% Made by Alex Nelson <pqnelson@gmail.com>
%% Login   <alex@lisp>
%% 
%% Started on  2025-10-16T07:06:20-0700
%% Last update 2025-10-16T07:06:20-0700
%% 

\lecture{}

\begin{corollary}
A ring $R\neq0$ is Artinian if and only if $R$ is Noetherian and $R$
is of dimension zero.
\end{corollary}

\begin{definition}
We say a ring $A$ \define{has Dimension Zero} (or ``is of dimension zero'')
if all its prime ideals are maximal ideals.
\end{definition}

\begin{remark}
There is a more general notion of a ``dimension of a ring'' (the Krull
dimension). The preceding definition is logically equivalent to when
the Krull dimension is zero.
\end{remark}

\begin{proof}[Proof (of Corollary)]
\forwardproof{} Assume $R$ is Artinian. Let $P$ be a prime ideal of $R$.
We want to prove $P$ is among the finitely-many maximal ideals
$P_{1}$, \dots, $P_{r}$ of $R$. We form from the proof of the previous
proposition from last lecture
$I^{s}=(P_{1}(\cdots)P_{r})^{s}=(0)\subset P$, so therefore there
exists an $i$ such that $P_{i}\subset P$ which implies $P_{i}=P$.

\backwardproof{} Assume $R$ is Noetherian and zero-dimensional. Then
we can take its primary decomposition
\begin{equation}
(0) = (q_{1})\cap\cdots\cap(q_{r})
\end{equation}
where the $(q_{i})$ are primary ideals of $R$. Take $P_{i}=\Radical{(q_{i})}$
for each $i=1,\dots,r$. Then each $P_{i}$ is finitely generated. Then
there exists $n\in\NN$ such that $P_{i}^{n}\subset Q_{i}=(q_{i})$
since $P_{i}$ are finitely-generated. Then take
\begin{equation}
\bigl(P_{1}(\cdots)P_{r}\bigr)^{n}\subset Q_{1}(\cdots)Q_{r}\subset Q_{1}\cap\cdots\cap Q_{r}=(0).
\end{equation}
Recall from last time, when we get to this point, we got
$\length_{R}(R)<\infty$ is finite, which implies $R$ is Artinian.
\end{proof}

\begin{remark}
\begin{enumerate}
\item A ring is Noetherian iff every prime ideal is finitely-generated.
\item ``Recently'' (1968) it has been shown if $R\subset R'$ and $R'$
  is a finitely-generated module over $R$, if $R'$ is Noetherian then
  $R$ is Noetherian.
\end{enumerate}
\end{remark}

\subsection{Flat Modules}

\begin{definition}[Tensoring a sequence by a module]
Let $A$ be a ring. Let $M$ be a module over $A$.
When we have any sequence of the form
\begin{equation}
S\colon\quad\cdots\to N\to N'\to N''\to\cdots
\end{equation}
we denote $S\otimes M$ the sequence
\begin{equation}
S\otimes M\colon\quad \cdots\to N\otimes M\to N'\otimes M\to
N''\otimes M\to\cdots.
\end{equation}
\end{definition}

\begin{definition}[Flat and faithfully flat modules]
Let $A$ be a ring. Let $M$ be a module over $A$.
We call $M$ \define{Flat} if $S\otimes M$ is an exact sequence
whenever $S$ is an exact sequence.

We call $M$ \define{Faithfully Flat} (or ``f.f.'') if $S\otimes M$ is exact
iff $S$ is exact.
\end{definition}

\begin{remark}
We have a hierarchy of attributes for modules. Let us review them here:
\begin{enumerate}
\item Free modules (nicest) are of the form $A^{n}$ or
  $\bigoplus_{i\in I}A$
\item Projective modules $P$ are defined by the property
  \begin{equation}
\xymatrix{P\ar@{-->}[r]\ar[dr] & M\ar@{->>}[d]\\
& N}
  \end{equation}
  which lifts along surjective maps. There are many ways to
  characterize projective modules.

  We see for any sequence
  \begin{equation}
S\colon\quad \cdots\to N\to N'\to N''\to\cdots
  \end{equation}
  thaat the sequence
  \begin{equation}
\cdots\to\hom(P,N)\to\hom(P,N')\to\hom(P,N'')\to\cdots
  \end{equation}
  Recall $\hom(M,-)\colon\Mod[A]\to\Ab$ is a functor and it is
  \textbf{\emph{always}} left exact (for every module $M$), meaning if
  $0\to N\to N'\to N''\to\cdots$ is exact, then
  \begin{equation}
0\to\hom(M,N)\to\hom(M,N')\to\hom(M,N'')\to\cdots
  \end{equation}
  is exact. So $P$ is projective if $\hom(P,-)$ is exact (on left and right).
\item Flat modules $M$ means $M\otimes-$ is an exact functor.
\end{enumerate}
\medbreak
We have in summary
\begin{equation}
\mbox{Free}\subset\mbox{Projective}\subset\mbox{Flat}.
\end{equation}
\end{remark}

\begin{proposition}
Let $A$ be a ring, let $P$ be a module over $A$.
The module $P$ is projective iff there exists a module $Q$ over $A$
such that $P\oplus Q$ is a free module.
\end{proposition}

\begin{question}
What is an example of a projective module which is not free?
\end{question}

\begin{proof}[Answer]
Take $R=\ZZ/6\ZZ$. We see that $M=\ZZ/2\ZZ$ is a module over $R$.
We see that $(\ZZ/2\ZZ)\oplus(\ZZ/3\ZZ)=\ZZ/6\ZZ$. But $M$ is not free.

More generally, if $A=B\times C$ as rings, then $B$ is a module over
$A$ and $B$ is flat over $A$ (but $B$ is not faithfully flat).
\end{proof}

\begin{node}
If we have a functor $\hom(M,-)$ for $M$ being a flat module, or
$M\otimes-\colon\Mod[A]\to\Ab$ but it is not left exact in general. We
can write a short exact sequence
\begin{equation}
0\to N\to N'\to N''\to 0
\end{equation}
the wish is to find a module $M$ such that
\begin{equation}
0\to M\otimes N\to M\otimes N'\to M\otimes N''\to0
\end{equation}
to preserve the injective map.

Recall the projective resolution of $N$,
\begin{equation}
\cdots\to P_{2}\to P_{1}\to P_{0}\to N,
\end{equation}
the idea is to apply the functor $M\otimes-$ to the projective
resolution
\begin{equation}
\cdots M\otimes P_{2}\xrightarrow{f_{2}} M\otimes P_{1}\xrightarrow{f_{1}} M\otimes P_{0}\xrightarrow{f_{0}} M\otimes N.
\end{equation}
Then we can look at the homology of $C=\mbox{(projective resolution)}\otimes M$
which measures th failure of exactness. We call the $i^{\text{th}}$
homology group $H^{i}(C)=\Tor^{i}(N,M)$.

The point: even if $M$ is not flat, there is a way to obtain flat
stuff by tensoring with the projective resolution.
\end{node}

\begin{xca}
Compute $\Tor(\ZZ/2\ZZ,\ZZ)$.
\end{xca}
%%
%% fall-lecture09.tex
%% 
%% Made by Alex Nelson <pqnelson@gmail.com>
%% Login   <alex@lisp>
%% 
%% Started on  2025-10-19T10:03:29-0700
%% Last update 2025-10-19T10:03:29-0700
%% 

\lecture{}

We want to find equivalent criteria for flatness.

\begin{theorem}
The following are equivalent:
\begin{enumerate}
\item $M$ is flat;
\item For any exact sequence of modules $0\to N'\to N$, we see $0\to N'\otimes M\to N\otimes M$ is exact;
\item For any finitely-generated ideal $I\ideal A$, $0\to I\otimes M\to M$
  is exact (the sequence $0\to I\to A$ is always exact), so we can say
  $I\otimes M\iso IM$;
\item $\Tor^{A}_{1}(M,A/I)=0$ for all finitely-generated ideals
  $I\ideal A$;
\item $\Tor^{A}_{1}(M,N)=0$ for any $A$-module $N$;
\item If $a_{i}\in A$ and $x_{i}\in M$ for some $i=1,\dots,r$, and
  \begin{equation}
\sum^{r}_{i=1}a_{i}x_{i}=0,
  \end{equation}
  then there exists some $s\in\NN$ and $b_{ij}\in A$ and $y_{j}\in M$
  for $j=1,\dots,s$ such that
  \begin{enumerate}[label=(\roman*)]
  \item $\sum^{r}_{i=1}a_{i}b_{ij}=0$ for all $j$ --- i.e.,
    $\vec{a}^{T}B=0$; and
  \item $x_{i}=\sum^{s}_{j=1}b_{ij}y_{j}$ for all $i$ --- i.e., $\vec{x}=B\vec{y}$.
  \end{enumerate}
\end{enumerate}
\end{theorem}

\begin{example}
The $\ZZ$-module $\QQ$ is flat but not projective. If we take
$2\cdot\frac{1}{2}+(-3)\cdot\frac{1}{3}=0$, then we can rewrite it as
$2\cdot3\cdot\frac{1}{6}+2\cdot(-3)\cdot\frac{1}{6}=0$. This
satisfies the last criterion. The indexes
here are $i=1,2$ and $j=1$. The $y_{1}=1/6$.
\end{example}

\begin{example}
Consider $\ZZ/5\ZZ$. Let us view $\bar{1}\in\ZZ/5\ZZ$ as an element of a $\ZZ$-module.
Then $5\cdot\bar{1}=0$, but it fails to satisfy the last criterion,
therefore $\ZZ/5\ZZ$ is not a flat $\ZZ$-module.

These two examples should demonstrate the usefulness of the seemingly
random last criterion in our theorem.
\end{example}

\begin{proposition}[Transitivity (3.B)]
If $\phi\colon A\to B$ is a ring morphism and $\phi$ makes $B$ a flat
$A$-module. Then a flat module $N$ over $B$ is also a flat module over $A$.
\end{proposition}

\begin{proposition}[Change of basis (3.C)]
If we have a ring morphism $\phi\colon A\to B$ and a flat module $M$
over $A$, then $M\otimes_{A}B$ is flat (as a module over $B$).
\end{proposition}

\begin{remark}
This is a ``change of basis'' in the sense that we are changing the
base ring of scalars.
\end{remark}

\begin{remark}
We should think of rings as functions on topological spaces (like the
ring of continuous functions, the ring of smooth functions, the ring
of analytic functions, the ring of holomorphic functions, etc.). Then
flat and free modules correspond to vector bundles. So this tells us
pulling back vector bundles gives us vector bundles.
\end{remark}


\begin{proposition}[Localization (3.D)]
Let $S\subset A$ be a multiplicative subset. Then $S^{-1}A$ is a flat
module over $A$.
\end{proposition}


\begin{proposition}[3H]
Let $A\to B$ be a flat map of rings. Let $I_{1},I_{2}\ideal A$ be ideals.
Then
\begin{enumerate}
\item $(I_{1}\cap I_{2})B=(I_{1}B)\cap(I_{2}B)$
\item if $I_{2}$ is finitely-generated, then $(I_{1}:I_{2})B = (I_{1}B:I_{2}B)$.
\end{enumerate}
\end{proposition}

\end{document}