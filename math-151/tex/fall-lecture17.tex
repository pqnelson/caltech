%%
%% fall-lecture17.tex
%% 
%% Made by Alex Nelson <pqnelson@gmail.com>
%% Login   <alex@lisp>
%% 
%% Started on  2025-11-06T10:22:20-0800
%% Last update 2025-11-06T10:22:20-0800
%% 

\lecture{}

\begin{example}[Abelianization]
We have two functors,
\begin{equation}
\vcenter{\xymatrix{%
                                          & \GRP\\
\TOP_{*}\ar[ur]^-{\pi_{1}}\ar[dr]_-{H_{1}} &\\
                                          & \AB}}
\end{equation}
We claim we have a natural transformation $\alpha\colon\pi_{1}\To H_{1}$
called \define{Abelianization}. The Abelianization of a group $G$ is
just $G/[G,G]$. (Intuitively, write the law of composition additively
and treat it as commutative.)
\end{example}

\begin{theorem}[Hurewicz]
Given a path-connected pointed-space $(X,x_{0})$, there is a
surjective group morphism $h\colon\pi_{1}(X,x_{0})\to H_{1}(X)$ whose
kernel is $\ker(h)=[G,G]$ where $G=\pi_{1}(X,x_{0})$.
\end{theorem}

\begin{corollary}
$H_{1}(X)\iso\pi_{1}(X,x_{0})/[\pi_{1}(X,x_{0}),\pi_{1}(X,x_{0})]$
\end{corollary}

\begin{proof}[Proof (Hurewicz's Theorem)]
What is $h$? Well, if we have a loop $\gamma$, then
$[\gamma]\in\pi_{1}(X,x_{0})$. We may think of $\gamma\colon[0,1]\to X$
such that $\gamma(0)=\gamma(1)=x_{0}$, or we may think of
$\gamma\colon\sphere{1}\to X$ where $\gamma(1)=x_{0}$ when we use $\sphere{1}\subset\CC$ is the unit sphere.

Then we have the induced map $\gamma_{*}\colon H_{1}(\sphere{1})\to H_{1}(X)$
where we define $h([\gamma])=\gamma_{*}(1)$ where $1$ is the generaor
of $H_{1}(\sphere{1})$.

Only consider the case where $X$ is a connected CW complex. Then
$X^{0}$ is just a set of points, $X^{1}$ is a connected 1-complex
(which is just a graph). But any graph is homotopy equivalent to the
1-point union of circles, so
\begin{equation}
X^{1}=\bigvee_{\alpha}\sphere{1}_{\alpha},
\end{equation}
which induces a homotopy equivalence $X\homotopic Y$ where $Y$ has
only one 0-cell. (We use the homotopy extension theorem to justify the
existence of $Y$.)

Then $\pi_{1}(Y)=\langle g_{\alpha}\mid r_{\beta}\rangle$ where each
generator $g_{\alpha}$ corresponds to each $\sphere{1}_{\alpha}$ in
the 1-skeleton, and relations $r_{\beta}$ correspond to 2-cells of
$Y$.

Now, how do we compute $H_{1}$? We can write down the chain complex
\begin{equation}
\dots\to C_{2}\xrightarrow{\boundary_{2}}C_{1}\xrightarrow{0}C_{0}\to0.
\end{equation}
Here $C_{1}=\ZZ\langle e^{1}_{\alpha}\rangle$ is freely-generated by
the 1-cells, and $C_{2}=\ZZ\langle e^{2}_{\beta}\rangle$ is
freely-generated by the 2-cells. Then by definition
\begin{subequations}
\begin{align}
H_{1}(Y) &= C_{1}/\Im(\boundary_{2})\\
&=\langle e^{1}_{\alpha}\mid\boundary e^{2}_{\beta}\rangle,
\end{align}
\end{subequations}
but $\boundary e^{2}_{\beta}$ is just the relations $r_{\beta}$
written \emph{additively}. This is precisely the meaning of
Abelianization. We'd need to prove this coincides with the Hurewicz
morphism, but we omit it.
\end{proof}

There is a purely algebraic way to\dots maybe we won't talk about this.

\subsection{Applications of Homology}

\begin{proposition}\label{prop:fall-lec17:application-of-homology}
\begin{enumerate}
\item If $D$ is a subspace of $\sphere{n}$ and $D$ is homeomorphic to
  the $k$-disk $\disk{k}$ for some $k\geq0$, then the reduced homology
  $\widetilde{H}_{*}(\sphere{n}\setminus D)=0$.
\item If $S$ is a subspace of $\sphere{n}$ and $S$ is homeomorphic to
  the sphere $\sphere{k}$ for $0\leq k<n$, then
  \begin{equation}
\widetilde{H}_{m}(\sphere{n}\setminus S)\iso\begin{cases}
\ZZ & \mbox{if }m=n-k-1\\
0 & \mbox{otherwise}
\end{cases}
  \end{equation}
\end{enumerate}
\end{proposition}

\begin{proof}[Proof sketch]
\begin{enumerate}
\item We see $D$ is contractible, so
\begin{equation}
\widetilde{H}_{*}(\sphere{n}\setminus D)\iso
\widetilde{H}_{*}(\sphere{n}\setminus\point{x})\iso
\widetilde{H}_{*}(\RR^{n})=0.
\end{equation}
\item By induction on $k$. The base case ($k=0$), we see
  $S\iso\sphere{0}$ is just 2 points. Then $\sphere{n}\setminus S\iso\RR^{n}\setminus\point{x}$.

For the $k+1$ case, $S$ has $E\iso\sphere{k-1}$ as its equator. Then
$\sphere{n}\setminus S=(\sphere{n}\setminus D_{+})\cap(\sphere{n}\setminus D_{-})$
where $D_{\pm}$ are the hemispheres of $S$, and 
$\sphere{n}\setminus E=(\sphere{n}\setminus D_{+})\cup(\sphere{n}\setminus D_{-})$.
These form M-V pairs, so we use the exact triangle
\begin{equation}
\vcenter{\xymatrix{
\widetilde{H}_{*}(\sphere{n}\setminus S)\ar[rr] &&\ar[dl]\widetilde{H}_{*}(\sphere{n}\setminus D_{+})\oplus \widetilde{H}_{*}(\sphere{n}\setminus D_{-})\\
& \ar[ul]_{[-1]} \widetilde{H}_{*}(S\setminus E) &}}
\end{equation}
But $\widetilde{H}_{*}(\sphere{n}\setminus D_{+})\oplus \widetilde{H}_{*}(\sphere{n}\setminus D_{-})=0$,
which gives us our isomorphism.\qedhere
\end{enumerate}
\end{proof}

\begin{corollary}
When $k=n-1$, $\widetilde{H}_{0}(\sphere{n}\setminus S)\iso\ZZ$,
and $H_{0}(\sphere{n}\setminus S)\iso\ZZ^{2}$ (which implies
$\sphere{n}\setminus S$ has 2 path-connected components).
\end{corollary}

\begin{corollary}[Jordan--Brouwer separation theorem]
If a subspace $S$ of $\sphere{n}$ is homeomorphic to $\sphere{n-1}$,
then $\sphere{n}\setminus S$ has 2 path-connected components. Each
component is open with boundary $S$.
\end{corollary}

When $n=2$, this is the Jordan Curve Theorem:
\begin{corollary}[Jordan Curve Theorem]
If $C\subset\sphere{2}$
is homeomorphic to $\sphere{1}$, then $\sphere{2}\setminus C$ has 2
path-connected components $U_{1}$ and $U_{2}$ such that $(U_{1}\cup C,C)\iso(\disk{2},\boundary\disk{2})$.
This tells you more: each component is an open disk and its union with
$C$ is homeomorphic to the closed disk.
\end{corollary}

\begin{example}[Alexander Horned Sphere]
This is a fractal-like construction. We will consider a finite
sequence of spaces $S_{n}$. Initially $S_{0}=\sphere{2}$ is just the
sphere. Then $S_{1}$ is obtained by attaching two ``horns'' to $S_{0}$.
Then $S_{2}$ attaches two ``sub-horns'' to each horn. Then $S_{3}$
attaches two ``sub-sub-horns'' to each ``sub-horn'', and so on. The
limiting case $S:=\lim_{n\to\infty}S_{n}$ is the \define{Horned Sphere}.
We see $S\iso\sphere{2}$. We have $\sphere{3}\setminus S=U_{1},U_{2}$
--- $U_{2}$ is not simply-connected. We have $(U_{1}\cup S,S)\iso(\disk{3},\boundary\disk{3})$.
In fact, $\pi_{1}(U_{2})$ is infinitely generated.

We can modify this such that there are ``bubbles'' in the Southern
hemisphere ``carved out'' of the sphere, when we attach ``horns'' in
the Northern hemisphere. This process yields the space $\widetilde{S}$
and neither of the components of $\sphere{3}\setminus\widetilde{S}$ is
homeomorphic to the disk.
\end{example}

\begin{corollary}[Invariance of domain]
If a subspace $X\subset\RR^{n}$ is homeomorphic to an open subset of
$\RR^{n}$, then $X$ itself is open in $\RR^{n}$.
\end{corollary}

\begin{proof}
Let $x\in X$. Then there exists a subset $D$ of $X$ containing $x\in D$,
which is homeomorphic to $(D,x)\iso(\disk{n},0)$ an $n$-disk
containing the origin. Moreover,
$\boundary\disk{n}\iso\sphere{n-1}$. Then apply the Jordan--Brouwer
theorem to the boundary $\boundary D$. Then $\RR^{n}\setminus\boundary D$
has 2 path-connected components, both are open in $\RR^{n}$.
By Proposition~\ref{prop:fall-lec17:application-of-homology} result (1),
we know $\RR^{n}\setminus D$ has only one path-connected components,
since
\begin{equation}
\widetilde{H}_{0}(\RR^{n}\setminus D)\iso\widetilde{H}_{0}(\sphere{n}\setminus D)=0.
\end{equation}
This implies $D\setminus\boundary D$ is a path-connected component of
$X\setminus D$. Hence it is open in $\RR^{n}$. Then $x\in D\setminus\boundary D\subset X$
(and $D\setminus\boundary D$ is open in $\RR^{n}$), but since $x\in X$
is arbitrary, it follows that every $x\in X$ is contained in an open
subset $U\subset X$, and hence $X$ is open in $\RR^{n}$ as desired.
\end{proof}