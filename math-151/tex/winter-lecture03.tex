%%
%% winter-lecture03.tex
%% 
%% Made by Alex Nelson <pqnelson@gmail.com>
%% Login   <alex@lisp>
%% 
%% Started on  2026-01-10T10:20:23-0800
%% Last update 2026-01-10T10:20:23-0800
%% 

\lecture{}

\begin{node}
If $X\homotopic Y$, then $\pi_{n}(X)\iso\pi_{n}(Y)$.
\end{node}

\begin{example}
Let $\widetilde{X}=\RR$ and $X=\sphere{1}$, we have a covering
$p\colon\widetilde{X}\to X$ sending $\theta\mapsto\E^{\I\theta}$. Then
for $n\geq2$, $\pi_{n}(\RR)\iso\pi_{n}(\sphere{1})=0$ since
$R\homotopic\point{*}$ is contractible; and for $n=1$,
$\pi_{1}(\sphere{1})\iso\ZZ$. 
\end{example}

\begin{definition}
A space is called \define{Aspherical} if $\pi_{n}(X)=0$ for all $n\geq2$.
\end{definition}

\begin{example}
The $n$-torus $\torus{n}=\sphere{1}\times\dots\times\sphere{1}$ has $\pi_{k}(\torus{n})\iso\pi_{k}(\sphere{1})\times\dots\times\pi_{k}(\sphere{1})=0$
for $k\geq2$.
\end{example}

\subsection{Relative homotopy groups}

\begin{definition}
Suppose $x_{0}\in A\subset X$ where $A$ is a subspace of $X$.
We may define the \define{Relative Homotopy Group} $\pi_{n}(X,A,x_{0})$
as follows: for $I^{n}$ with coordinates $(s_{1},\dots,s_{n})$
consider $I^{n-1}$ sitting in the boundary with $s_{1}=0$. Let
$J^{n-1}=\closure{\boundary{I^{n}}\setminus I^{n-1}}$ (the union of
the rest of the faces in $\boundary I^{n}$) as doodled thus:
\begin{center}
\includegraphics{img/img.52}
\end{center}
We define
\begin{equation}
\pi_{n}(X,A,x_{0})=[(I^{n},\boundary I^{n},J^{n-1}),(X,A,x_{0})]
\end{equation}
where $J^{n-1}$ is mapped to $x_{0}$, $\boundary I^{n}$ is mapped to $A$,
and $I^{n}$ is mapped to $X$.

The group structure is defined as before (for $n\geq2$):
\begin{center}
\includegraphics{img/img.53}
\end{center}
where the thick red portion of the boundary of the square/rectangle is
mapped to $x_{0}$.
\end{definition}

\begin{remark}
When $n=1$, what's going on? Well, we can draw analogous diagrams, but
with intervals instead of squares/cubes. One endpoint (drawn in red)
is mapped to $x_{0}$, the other endpoint (in black) is mapped to some
point in $A$:
\begin{center}
\includegraphics{img/img.54}
\end{center}
But in general this endpoint in $A$ is not equal to $x_{0}$, so we
don't get a group structure since concatenation is not a well-defined
binary operator. Nevertheless, it remains a pointed set.
\end{remark}

\begin{node}
The group $\pi_{n}(X,A,x_{0})$ is Abelian for $n\geq3$.
\end{node}

\begin{remark}
The absolute homotopy group $\pi_{n}(X,x_{0})$ could be viewed as a
special case of the relative homotopy group $\pi_{n}(X,A,x_{0})$ where
$A=\{x_{0}\}$. An equivalent way of defining $\pi_{n}(X,A,x_{0})$
since $J^{n-1}$ is sent to $x_{0}$, we can quotient out $J^{n-1}$ from
the domain, which turns $I^{n}$ into a closed disk $\disk{n}$,
\begin{subequations}
  \begin{align}
\pi_{x}(X,A,x_{0}) &= [(I^{n}, \boundary I^{n}, J^{n-1}), (X,A,x_{0})]\\
&= [(\disk{n},\boundary\disk{n}, \point{*}), (X,A,x_{0})].
  \end{align}
\end{subequations}
In this description, we have the quotient map doodled as:
\begin{center}
\includegraphics{img/img.55}
\end{center}
We also can describe the group operator using discs, where we identify
the ``equator'' $\disk{n-1}$ of the disc $\disk{n}$ and quotient out
by it, sending $\disk{n}\to\disk{n}/\disk{n-1}$ which collapses the
``equator'' to a point giving us $\disk{n}\wedge\disk{n}\iso\disk{n}/\disk{n-1}$.
Then we apply $f$ to one disk, and $g$ to the other disk
\begin{center}
\includegraphics{img/img.56}
\end{center}
\end{remark}

\begin{remark}
All functorial properties work for relative homotopy groups, as well.
For example, if $f\colon(X,A,x_{0})\to(Y,B,y_{0})$, then it induces a
group morphism $f_{*}\colon\pi_{n}(X,A,x_{0})\to\pi_{n}(Y,B,y_{0})$
for $n$ sufficiently large. Also, an action of $\pi_{1}(A,x_{0})$ on
$\pi_{n}(X,A,x_{0})$ may be defined.
\end{remark}

\begin{lemma}[Compression/vanishing criterion]
An element $[f]\in\pi_{n}(X,A,x_{0})$ is zero if and only if
$f\homotopic g\rel{\boundary\disk{n}}$ homotopic such that
$g(\disk{n})\subset A$ (the boundary is fixed throughout).
\end{lemma}

(That is, $H\colon\disk{n}\times I\to X$ is a homotopy from $f$ to $g$
is such a homotopy if $H(x,t)=f(x)=g(x)$ for all
$x\in\boundary\disk{n}$ and all $t\in I$.)

\begin{proof}
\backwardproof\ If $f\homotopic g\rel{\boundary\disk{n}}$ then $[f]=[g]\in\pi_{n}(X,A,x_{0})$.
Suffices to show $[g]=0$ in $\pi_{n}(X,A,x_{0})$. Here there isa
homotopy
\begin{equation}
H\colon(\disk{n},\boundary\disk{n},\point{*})\times I\to(X,A,x_{0})
\end{equation}
from $g$ to the constant function $c_{x_{0}}$ which takes the
following form: for all $x\in\disk{n}$ consider the path $\lambda_{x}$
from $x$ to $p$ by $\lambda_{x}\colon[0,1]\to\disk{n}$. Define
\begin{equation}
H(x,t)=g(\lambda_{x}(t)).
\end{equation}
Then when $t=0$,
\begin{equation}
H(x,0)=g(\lambda_{x}(0))=g(x),
\end{equation}
and when $t=1$ we have
\begin{equation}
H(x,1)=g(\lambda_{x}(1))=g(p)=x_{0}.
\end{equation}
This is valid in the sense that $H(\boundary\disk{n},t)\subset A$
since $g(\disk{n})\subset A$ (if we tried doing this directly for $f$,
we can't do that as for $x\in\boundary\disk{n}$,
$H(x,t)=f(\lambda_{x}(t))$ which does not necessarily lie in $A$).

\forwardproof\ Suppose $[f]=0$ in $\pi_{n}(X, A, x_{0})$. Then there
exists a relative homotopy
\begin{equation}
H\colon(\disk{n},\boundary\disk{n},\point{*})\times I\to(X,A,x_{0})
\end{equation}
such that $H(x,0)=f(x)$ and $H(x,1)=x_{0}$. We can draw the homotopy
as a disc (dashed line corresponds to $\disk{n}$, the red points
correspond to $\boundary\disk{n}$):
\begin{center}
\includegraphics{img/img.57}
\end{center}
But it defines a relative homotopy by changing perspective
\begin{equation}
\disk{n} = \boundary(\disk{n}\times I)\setminus(\disk{n}\times\{0\}),
\end{equation}
which is the entire box except the dashed line, and define $g$ to be
$H$ restricted to $\disk{n}$. Then $H$ could be viewed as a homotopy
from $f$ to $g$ such that $H|_{\boundary\disk{n}}(x)=f(x)$. Then
$g(\disk{n})\subset A$ by definition of $H$.
\end{proof}

\subsection{Long Exact Sequence of Relative Homotopy Groups}

\begin{node}
Suppose $x_{0}\in A\subset X$. Then the following:
\begin{equation}
\vcenter{\xymatrix{\dots & \dots\ar[r] & \pi_{n+1}(X,A,x_{0})\ar@/_/[dll]_-{\boundary}\\
\pi_{n}(A,x_{0})\ar[r]^-{i_{*}} & \pi_{n}(X,x_{0})\ar[r]^-{j_{*}} & \pi_{n}(X,A,x_{0})\ar@/_/[dll]_-{\boundary}\\
\pi_{n}(A,x_{0})\ar[r]^-{i_{*}} & \dots \ar[r]& \pi_{1}(X,A,x_{0})\ar@/_/[dll]_-{\boundary}\\
\pi_{0}(A,x_{0})\ar[r] & \pi_{0}(X,x_{0}) & }}
\end{equation}
where
\begin{enumerate}[start=0]
\item $\pi_{1}(X,A,x_{0})$, $\pi_{0}(A,x_{0})$, and $\pi_{0}(X,x_{0})$
  are all just pointed sets (\textbf{not groups});
\item the inclusion $i\colon(A,x_{0})\into(X,x_{0})$ induces the
  morphism $i_{*}\colon\pi_{n}(A,x_{0})\to\pi_{n}(X,x_{0})$;
\item the inclusion $j\colon(X,x_{0})=(X,\{x_{0}\},x_{0})\into(X,A,x_{0})$
  induces the morphism
  $j_{*}\colon\pi_{n}(X,x_{0})\to\pi_{n}(X,A,x_{0})$;
\item the $\boundary\colon\pi_{n}(X,A,x_{0})\to\pi_{n-1}(A,x_{0})$ is
  induced by the boundary operator which
  restricts the continuous map $f\colon (\disk{n},\sphere{n-1},x_{0})\to(X,A,x_{0})$ 
  to its boundary $\boundary\disk{n}=\sphere{n-1}$ --- so it sends
  $f\mapsto f|_{\boundary\disk{n}}$.
\end{enumerate}
\end{node}

\begin{proposition}
This really is a long exact sequence.
\end{proposition}

\begin{remark}
The end of the long sequence,
\begin{equation}
\pi_{1}(X,x_{0})\xrightarrow{j_{*}}\pi_{1}(X,A,x_{0})\xrightarrow{\boundary}\pi_{0}(A,x_{0})\xrightarrow{i_{*}}\pi_{0}(X,x_{0})\to0,
\end{equation}
except for $\pi_{1}(X,x_{0})$ (and $0$), everything is a pointed
set. So in what sense is this ``exact'', really?

Well, we define the ``kernel'' of $i_{*}\colon\pi_{0}(A,x_{0})\to\pi_{0}(X,x_{0})$
to be the preimage $\ker(i_{*})=i_{*}^{-1}([x_{0}])$. Similarly, $\ker(\boundary)=\boundary^{-1}([x_{0}])$.
\end{remark}