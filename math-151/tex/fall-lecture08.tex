%%
%% fall-lecture08.tex
%% 
%% Made by Alex Nelson <pqnelson@gmail.com>
%% Login   <alex@lisp>
%% 
%% Started on  2025-10-16T07:49:51-0700
%% Last update 2025-10-16T07:49:51-0700
%% 

\lecture[Exact Sequences]

\begin{definition}
Let $A$, $B$, $C$ be Abelian groups. Let $f\colon A\to B$
and $g\colon B\to C$ be morphisms. We will often abbreviate this
information using the notation $A\xrightarrow{f}B\xrightarrow{g}c$.
We call this sequence \define{Exact at $B$} if $\ker(g)=\Im(f)$.
\end{definition}

\begin{definition}
When we have many Abelian groups forming the sequence $A_{1}\xrightarrow{f_{1}}A_{2}\xrightarrow{f_{2}}A_{2}\xrightarrow{f_{3}}\cdots$,
we call it \define{Exact} when it is exact everywhere.
\end{definition}

\begin{example}
We see $0\to A\xrightarrow{f}B$ is exact at $A$ if $\ker(f)=0$. This
is equivalent to stating $f$ is an injective morphism.
\end{example}

\begin{example}
We see $B\xrightarrow{g}C\to0$ is exact at $C$ if $\Im(g)=\ker(0)=C$.
This is equivalent to stating $g$ is a surjective morphism.
\end{example}

\begin{remark}
Observe these two examples are ``dual'' to each other, in the sense
that if we reverse the arrows in one example, then we obtain the other
example. In this sense, ``injective'' is dual to ``surjective''.
\end{remark}

\begin{definition}[Short exact sequence of Abelian groups]
We call an exact sequence $0\to A\xrightarrow{f}B\xrightarrow{g}C\to0$
a \define{Short Exact Sequence}.

In this case, $A$ is isomorphic to the (normal) subgroup
$f(A)\normalsubgroup B$, and also that $C\iso B/f(A)$.
\end{definition}

\begin{definition}
If $A$, $B$, $C$ are chain complexes (\S\ref{defn:fall-lec06:chain-complex}),
and $f$ and $g$ are chain maps (\S\ref{defn:fall-lec06:chain-map}),
then we call $0\to A\xrightarrow{f}B\xrightarrow{g}C\to0$ a
\define{Short Exact Sequence of Chain Complexes}. What does this mean?
At each grading $n$, we have a short exact sequence
\begin{equation}
\xymatrix{
0\ar[r] & A\ar[r]^{f_{n}}\ar[d]^{\boundary^{(A)}_{n}}& B_{n}\ar[r]^{g_{n}}\ar[d]^{\boundary^{(B)}_{n}}&C\ar[d]^{\boundary^{(C)}_{n}}\ar[r]&0\\
0\ar[r] & A_{n-1}\ar[r]^{f_{n-1}} & B_{n-1}\ar[r]^{g_{n-1}} & C_{n-1}\ar[r] & 0}
\end{equation}
which commutes every which way, for every possible $n$.
\end{definition}

\begin{lemma}\label{lem:fall-lec08}
A short exact sequence of chain complexes,
\begin{equation}
0\to A_{*}\xrightarrow{i}B_{*}\xrightarrow{j}C_{*}\to0,
\end{equation}
induces a long exact sequence
\begin{equation}
\xymatrix{H_{n}(A)\ar[r]^{i_{*}} & H_{n}(B)\ar[r]^{j*} & H_{n}(C)\ar@/_/[dll]^{\boundary}\\
H_{n-1}(A)\ar[r]^{i_{*}} & H_{n-1}(B)\ar[r]^{j*} & H_{n-1}(C)\ar@/_/[dll]^{\boundary}\\
\dots& &}
\end{equation}
The $\boundary$ morphisms here are called the \define{Connecting Morphisms}.
\end{lemma}

(A lot of times, short exact sequences give rise to a long exact
sequence of homology groups.)

\begin{proof}
How do we define the connecting morphisms? Well, consider $c$ being a
cycle in $C_{n}$, 
\begin{equation}
\xymatrix{0\ar[r] & A_{n}\ar[d]^{\boundary}\ar[r]^{i} & \in B_{n}\ar[d]^{\boundary}\ar[r]^{j} & {\color{red}c}\in C_{n}\ar[r]\ar[d]^{\boundary}&0\\
0\ar[r]& \in A_{n-1}\ar[d]\ar[r]^{i}&\in B_{n-1}\ar[r]^{j}\ar[d]&C_{n-1}\ar[d]\ar[r]&0\\
&\dots&\dots&\dots&}
\end{equation}
Then $c$ has a preimage under $j$,
\begin{equation}
\exists b\in B_{n}\ldotp j(b)=c.
\end{equation}
So we have
\begin{equation}
\xymatrix{0\ar[r] & A_{n}\ar[d]^{\boundary}\ar[r]^{i} & {\color{red}b}\in B_{n}\ar[d]^{\boundary}\ar[r]^{j} & {\color{red}c}\in C_{n}\ar[r]\ar[d]^{\boundary}&0\\
0\ar[r]&  A_{n-1}\ar[d]\ar[r]^{i}& B_{n-1}\ar[r]^{j}\ar[d]&C_{n-1}\ar[d]\ar[r]&0\\
&\dots&\dots&\dots&}
\end{equation}
We claim $j(\boundary b)=0$. This is by invoking the commutative
diagram
\begin{subequations}
\begin{align}
  j(\boundary b)&=\boundary(j(b))\\
  &=\boundary c\\
  &=0
\end{align}
\end{subequations}
since $c\in\ker(\boundary)$ is a cycle.

Then there exists an $a\in A_{n-1}$ such that $i(a)=\boundary b$. That
is to say, we have constructed the red objects in the diagram:
\begin{equation}
\xymatrix{0\ar[r] & A_{n}\ar[d]^{\boundary}\ar[r]^{i} & {\color{red}b}\in B_{n}\ar[d]^{\boundary}\ar[r]^{j} & {\color{red}c}\in C_{n}\ar[r]\ar[d]^{\boundary}&0\\
0\ar[r]& {\color{red}a}\in A_{n-1}\ar[d]\ar[r]^{i}&{\color{red}\boundary b}\in B_{n-1}\ar[r]^{j}\ar[d]&C_{n-1}\ar[d]\ar[r]&0\\
&\dots&\dots&\dots&}
\end{equation}

We now claim $a$ is a cycle, i.e., $\boundary a=0$. Again, applying
commutativity, we see
\begin{subequations}
\begin{align}
  i(\boundary a) &= \boundary(i(a))\\
  &=\boundary(\boundary b)\\
  &=0
\end{align}
\end{subequations}
But since $i$ is injective, we find $\boundary a=0$ as desired. Thus
we obtain some $a\in A_{n-1}$ from $c\in C_{n}$.

We define the connecting morphism $\boundary_{\text{connecting}}[c]=[a]$.

Does this construction depends on the particular choice of $b\in j^{-1}(\{c\})\subset B_{n}$?
Or the choice of $a\in A_{n-1}$? We should prove that the connecting
morphism \emph{does not} depend on these choices.

Let us begin by proving the construction does not depend on the choice
of $b\in B_{n}$. If we instead choose $b'\in j^{-1}(\{c\})\subset B_{n}$,
then we observe $b-b'\in\ker(j)$. Then by exactness, there exists an
element $e\in A_{n}$ such that
\begin{equation}
i(e)=b-b'.
\end{equation}
Then
\begin{subequations}
\begin{align}
i(\boundary e) &= \boundary(i(e))\\
&=\boundary(b-b'),
\end{align}
\end{subequations}
so then
\begin{subequations}
\begin{align}
i(a - \boundary e) &= \boundary b - \boundary(b - b')\\
&=\boundary b.
\end{align}
\end{subequations}
That is to say, if we picked $b'$ instead of $b$, the construction
yields the homology class $[a-\boundary e]$ instead of $[a]$, but
these two are homologous $[a-\boundary e]=[a]$. So it does not depend
on the choice of $b$!

The argument for independence of choice of $a$ is similarly, but
rather long and exhausting.
\end{proof}

\begin{definition}[Quotient chain complexes]
Let $X$ be a topological space, let $A\subset X$ be a subspace.
Then $C_{*}(A)\subset C_{*}(X)$ is a chain subcomplex since $C_{*}(A)$
is generated by singular simplices of $A$ (but singular simplices of
$A$ are also singular simplices of $X$). We then define the
\define{Relative Chain Complex} or \define{Quotient Chain Complex}
$C_{*}(X,A)=C_{*}(X)/C_{*}(A)$.
\end{definition}

\begin{definition}[Relative homology groups]
Let $X$ be a topological space, let $A\subset X$ be a subspace.
We define the \define{Relative Homology Groups} $H_{*}(X,A)$ to be the
homology groups of $C_{*}(X,A)$.
\end{definition}

\begin{remark}
We have a short exact sequence of chain complexes
\begin{equation}
0\to C_{*}(A)\xrightarrow{i_{*}}C_{*}(X)\xrightarrow{j_{*}}C(X,A)\to0.
\end{equation}
Then by Lemma~\ref{lem:fall-lec08}, we have a long exact sequence of
homology groups
\begin{equation}
\xymatrix{& &\dots\ar[dll]_{\boundary}\\
H_{n}(A)\ar[r]^{i_{*}}&H_{n}(X)\ar[r]^{j_{*}}&H_{n}(X,A)\ar[dll]_{\boundary}\\
H_{n-1}(A)\ar[r]^{i_{*}}&H_{n-1}(X)\ar[r]^{j_{*}}&H_{n-1}(X,A)\ar[dll]_{\boundary}\\
\dots}
\end{equation}
There is a geometric meaning to this connecting morphism
$\boundary$. Given a cycle $\alpha\in C_{n}(X,A)$, we see
\begin{equation}
\boundary\alpha=[0]\in C_{*}(X,A)=C_{*}(X)/C_{*}(A),
\end{equation}
so $\boundary\alpha\in C_{*}(A)$ is a singular chain in $A$, and since
$\boundary(\boundary\alpha)=0$, we see $\boundary\alpha$ is a
cycle. We see we have:
\begin{equation}
\vcenter{\hbox{\includegraphics{img/img.37}}}
\end{equation}
\end{remark}