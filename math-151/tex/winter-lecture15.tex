%%
%% winter-lecture15.tex
%% 
%% Made by Alex Nelson <pqnelson@gmail.com>
%% Login   <alex@lisp>
%% 
%% Started on  2026-02-07T11:30:08-0800
%% Last update 2026-02-07T11:30:08-0800
%% 

\lecture{}

\begin{recall}
Last time, we proved fiber bundles are Serre fibrations. We also
introduced the long exact sequence of homotopy groups for a
fibration $F\into E\onto B$. A simpler way to prove it is by looking
at the pair $(E,F)$ and the long exact sequence for homotopy groups
gives us $\pi_{n}(E,F)\iso\pi_{n}(B,b_{0})$ using the induced morphism
from the projection map $E\onto B$, and then the long exact sequence
is obtained
\begin{equation}
\dots\to\pi_{n}(F,x_{0})\to\pi_{n}(E,x_{0})\to\bigl(\pi_{n}(E,F)\iso\pi_{n}(B,b_{0})\bigr)\to\dots
\end{equation}
as desired.
\end{recall}

\begin{theorem}[Adams]
The Hopf bundles are the only fiber bundles between spheres; they are:
\begin{enumerate}
\item $\sphere{0}\into\sphere{1}\onto\sphere{1}\iso\RP^{1}$
\item $\sphere{1}\into\sphere{3}\onto\sphere{2}\iso\CP^{1}$
\item $\sphere{3}\into\sphere{7}\onto\sphere{4}\iso\HP^{1}$
\item $\sphere{7}\into\sphere{15}\onto\sphere{8}\iso\OP^{1}$.
\end{enumerate}
\end{theorem}

\begin{node}
We also discussed a little about the stable homotopy groups
$\pi_{n}(O(\infty))$, $\pi_{n}(U(\infty))$. But there is also
$\pi_{n}(\Sp(\infty))$. Recall the Symplectic group $\Sp(2n)$ is
defined by letting
\begin{equation}
\omega=\begin{pmatrix}0 & I_{n}\\-I_{n} & 0
\end{pmatrix}\in\Mat_{2n}(\RR)
\end{equation}
which is the standard symplectic form. Then
\begin{equation}
\Sp(2n)=\{A\in\Mat_{2n}(\RR)\mid A\omega\transpose{A}\}
\end{equation}
which is the change of basis for a bilinear form. This is a very
useful Lie group in physics. The homotopy groups stabilize in the
sense that $\pi_{i}(\Sp(n))=\pi_{i}(\Sp(n+k))$ for fixed $i$ and
sufficiently large $n$ and any $k$. This justifies looking at
$\pi_{n}(\Sp(\infty))$ as these stable homotopy groups, just as we did
for $\pi_{n}(O(\infty))$ and $\pi_{n}(U(\infty))$. The claim is $\pi_{n}(\Sp(\infty))$ is
periodic with period 8. This is studied under the umbrella term
\define{Bott Periodicity}.
\end{node}

\begin{node}
Today we will finish studying the ``concrete'' part of the course,
then pivot to study connections between homotopy theory and cohomology theory.
\end{node}

\begin{node}
The last things we will do for ``concrete'' topics:
\begin{enumerate}
\item we will study $\pi_{3}(\bigvee_{\alpha}\sphere{2}_{\alpha})$;
\item we will define the Whitehead product
\end{enumerate}
\end{node}

\begin{fact}
We know $\pi_{2}(\bigvee_{\alpha\in A}\sphere{2}_{\alpha})\iso\bigoplus_{\alpha\in A}\ZZ$.
\end{fact}

\begin{node}% Hatcher, example 4.52, PDF pages 389-390
We claim $\pi_{3}(\bigvee_{\alpha\in A}\sphere{2}_{\alpha})$ is a free
Abelian group generated by
\begin{enumerate}
\item the Hopf maps $\sphere{3}\to\sphere{2}_{\alpha}$ for each $\alpha\in A$, and
\item by maps of the form $\sphere{3}\to\sphere{2}_{\alpha}\vee\sphere{2}_{\beta}\subset\bigvee_{\alpha\in A}\sphere{2}_{\alpha}$
for distinct $\alpha\neq\beta$ given by the gluing maps of the cells
$e^{2}_{\alpha}\times e^{2}_{\beta}$ in the products $\sphere{2}_{\alpha}\times\sphere{2}_{\beta}$.
\end{enumerate}
\end{node}

\begin{proof}
Without loss of generality, we may assume the indexing set $A$ is finite.
Consider the long exact sequence of the pair $(\prod_{\alpha\in A}\sphere{2}_{\alpha},\bigvee_{\alpha\in A}\sphere{2}_{\alpha})$,
\begin{equation}
\vcenter{\xymatrix{& & \dots\ar[dll]\\
\pi_{n}(\bigvee_{\alpha\in A}\sphere{2}_{\alpha})\ar[r] &
\pi_{n}(\prod_{\alpha\in A}\sphere{2}_{\alpha})\ar[r]^-{0} & \pi_{n}(\prod_{\alpha\in A}\sphere{2}_{\alpha},\bigvee_{\alpha\in A}\sphere{2}_{\alpha})\ar[dll]\\
\dots & & }}
\end{equation}
where we know $\pi_{n}(\prod_{\alpha\in A}\sphere{2}_{\alpha})\iso\bigoplus_{\alpha\in A}\ZZ$.
The induced map
\begin{equation}
i_{*}\colon\pi_{n}(\bigvee_{\alpha\in A}\sphere{2}_{\alpha})\to\pi_{n}(\prod_{\alpha\in A}\sphere{2}_{\alpha})
\end{equation}
is surjective.
(Why is $i_{*}$ surjective? Well, for each $\beta\in A$, we have
$\sphere{2}_{\beta}\into\bigvee_{\alpha\in A}\sphere{2}_{\alpha}\xrightarrow{i}\prod_{\alpha}\sphere{2}_{\alpha}$
composes, everything has a preimage under this composition which
implies $i$ is surjective.)
This means the short exact sequence
\begin{equation}
0\to\pi_{n+1}(\prod_{\alpha\in A}\sphere{2}_{\alpha},\bigvee_{\alpha\in A}\sphere{2}_{\alpha})\to\pi_{n}(\bigvee_{\alpha\in A}\sphere{2}_{\alpha})\to\underbrace{\pi_{n}(\prod_{\alpha\in A}\sphere{2}_{\alpha})}_{\iso\bigoplus_{\alpha\in A}\ZZ}\to0
\end{equation}
splits. We can define a map
\begin{equation}
q\colon\bigoplus_{\alpha}\pi_{n}(\sphere{2}_{\alpha})\to\pi_{n}(\bigvee_{\alpha}\sphere{2}_{\alpha})
\end{equation}
such that
\begin{equation}
i_{*}\circ q=\id
\end{equation}
in the following way: For each $\beta\in A$, there is an inclusion
\begin{equation}
i_{\beta}\colon\sphere{2}_{\beta}\to\bigvee_{\alpha\in A}\sphere{2}_{\alpha}
\end{equation}
which gives
\begin{equation}
\bigoplus_{\beta}(i_{\beta})_{*}\colon\bigoplus_{\alpha}\pi_{n}(\sphere{2}_{\alpha})\to\pi_{n}(\bigvee_{\alpha}\sphere{2}_{\alpha})
\end{equation}
such that the composition
\begin{equation}
\vcenter{\xymatrix{S^{2}_{\beta}\ar[r]^{i_{\beta}} & \bigvee_{\alpha}\sphere{2}_{\alpha}\ar[r]^{i}&\prod_{\alpha}\sphere{2}_{\alpha}\\
\pi_{n}(S^{2}_{\beta})\ar[r]^{(i_{\beta})_{*}}\ar@/_2pc/[rr]_{\text{including the $\beta$ component}} & \pi_{n}(\bigvee_{\alpha}\sphere{2}_{\alpha})\ar[r]^{i_{*}}&\pi_{n}(\prod_{\alpha}\sphere{2}_{\alpha})}}
\end{equation}
Therefore
\begin{equation}
\pi_{n}(\bigvee_{\alpha}\sphere{2}_{\alpha})\iso\left(\bigoplus_{\alpha\in A}\pi_{n}(\sphere{2}_{\alpha})\right)\oplus\pi_{n+1}(\prod_{\alpha}\sphere{2}_{\alpha},\bigvee_{\alpha}\sphere{2}_{\alpha}).
\end{equation}
In particular,
\begin{equation}
\pi_{3}(\bigvee_{\alpha}\sphere{2}_{\alpha})\iso\left(\bigoplus_{\alpha\in A}\pi_{3}(\sphere{2}_{\alpha})\right)\oplus\pi_{4}(\prod_{\alpha}\sphere{2}_{\alpha},\bigvee_{\alpha}\sphere{2}_{\alpha}).
\end{equation}
Recall the quotient, since $\bigvee_{\alpha}\sphere{2}_{\alpha}$ is
1-connected and the pair $(\prod_{\alpha}\sphere{2}_{\alpha},\bigvee_{\alpha}\sphere{2}_{\alpha})$
is 3-connected, then by the quotient rule
\begin{equation}
\pi_{4}(\prod_{\alpha}\sphere{2}_{\alpha},\bigvee_{\alpha}\sphere{2}_{\alpha})\iso\pi_{4}\left(\prod_{\alpha}\sphere{2}_{\alpha}\middle/\bigvee_{\alpha}\sphere{2}_{\alpha}\right)
\end{equation}
Then the 5-skeleton of the quotient is given by
\begin{equation}
\bigvee_{\substack{(\alpha,\beta)\in A\times A\\ \alpha\neq\beta}}\sphere{4}_{\alpha,\beta}
\end{equation}
Then
\begin{subequations}
  \begin{align}
\pi_{4}(\prod_{\alpha}\sphere{2}_{\alpha},\bigvee_{\alpha}\sphere{2}_{\alpha})&\iso\pi_{4}\left(\prod_{\alpha}\sphere{2}_{\alpha}\middle/\bigvee_{\alpha}\sphere{2}_{\alpha}\right)\\
&\iso\bigoplus_{\alpha\neq\beta}\pi_{4}(\sphere{4}_{\alpha,\beta}).
  \end{align}
\end{subequations}
Returning to our short exact sequence, we see
\begin{equation}
0\to\underbrace{\pi_{n+1}(\prod_{\alpha\in A}\sphere{2}_{\alpha},\bigvee_{\alpha\in A}\sphere{2}_{\alpha})}_{\iso\bigoplus_{\alpha\neq\beta}\pi_{4}(\sphere{4}_{\alpha,\beta})}\xrightarrow{\boundary}\pi_{n}(\bigvee_{\alpha\in A}\sphere{2}_{\alpha})\to\underbrace{\pi_{n}(\prod_{\alpha\in A}\sphere{2}_{\alpha})}_{\iso\bigoplus_{\alpha\in A}\ZZ}\to0
\end{equation}
Under the map
\begin{equation}
\boundary\colon\pi_{4}(\prod_{\alpha}\sphere{2}_{\alpha},\bigvee_{\alpha}\sphere{2}_{\alpha})\to\pi_{3}(\bigvee_{\alpha}\sphere{2}_{\alpha})
\end{equation}
these generators of $\pi_{4}(\sphere{4}_{\alpha,\beta})$ are mapped
tothe gluing map of the cells $e^{2}_{\alpha}\times e^{2}_{\beta}$ to
$\bigvee_{\alpha}\sphere{2}_{\alpha}$ in $\prod_{\alpha}\sphere{2}_{\alpha}$.
\end{proof}

\begin{definition}
Given maps $f\colon\sphere{k}\to X$ and $g\colon\sphere{\ell}\to X$,
we may define their \define{Whitehead Product} denoted by $[f,g]$ such
that it is the composition of
\begin{equation}
\sphere{k+\ell-1}\xrightarrow{\phi}\sphere{k}\vee\sphere{\ell}\xrightarrow{f\vee g}X
\end{equation}
where $\phi$ is the attaching map of $\disk{k}\times\disk{\ell}$ to
$\sphere{k}\vee\sphere{\ell}$ in the CW structure of $\sphere{k}\times\sphere{\ell}$.
\end{definition}

\begin{remark}
The Whitehead product is well-defined on homotopy groups, giving us a map
\begin{equation}
[-,-]\colon\pi_{k}(X)\times\pi_{\ell}(X)\to\pi_{k+\ell-1}(X).
\end{equation}
\end{remark}

\begin{remark}
In particular, for $k=\ell=1$, we have the Whitehead product coincide
with the commutator of group elements for the fundamental group (hence
the notation). Specifically, we have
\begin{equation}
\begin{split}
[-,-]\colon&\pi_{1}(X)\times\pi_{1}(X)\to\pi_{1}(X)\\
&(f,g)\mapsto fgf^{-1}g^{-1}.
\end{split}
\end{equation}
Then we may view the Whitehead product as a generalization of the
commutator to other homotopy groups.
\end{remark}

\begin{remark}
In general, by a result of Hilton, we can compute
$\pi_{n}(\bigvee_{\alpha}\sphere{k}_{\alpha})$ in terms of $\pi_{n}(\sphere{k}_{\alpha})$.
\end{remark}

\subsection{Connections with Cohomology}

(This is \S4.3 in Hatcher~\cite{hatcher2002algebraic})

\begin{node}
So what is a (reduced) cohomology theory?

A (reduced) cohomology theory is a sequence of contravariant functors
\begin{equation}
(h^{k})_{k\in\NN_{0}}\colon\CW\to\GRP
\end{equation}
where $\CW$ is the category of pointed CW complexes (and the CW maps
between them), and $\GRP$ is the category of groups; such that
\begin{enumerate}
\item any $f\colon X\to Y$ becomes $f^{*}\colon h^{k}(Y)\to h^{k}(X)$;
\item if $f\homotopic g$ are homotopic, then $f^{*}=g^{*}$ (``descends
  to homotopy category''); and
\item ``Respects certain rules for topological operations'', things
  like $h^{k}(\bigvee_{\alpha}X_{\alpha})\iso\bigoplus_{\alpha}h^{k}(X_{\alpha})$,
  the Excision theorem, the Mayer--Vietoris sequence, etc.
\end{enumerate}
Then $h^{k}(X)$ is determined by $h^{k}(\sphere{0})$ the cohomology of
the point.
\end{node}

\begin{idea}
In the spirit of Yoneda lemma, there is a ``natural'' (in the informal
sense, an ``obvious'') way to get contravariant functors. For a given
CW complex $Y$ we define a functor
\begin{equation}
\begin{split}
[-,Y]\colon&\CW\to\SET\\
&X\mapsto[X,Y]
\end{split}
\end{equation}
where $[X,Y]$ consists of the homotopy equivalence classes of maps
$X\to Y$. Moreover, on a map $f\colon X_{0}\to X_{1}$, we get a map
\begin{equation}
\begin{split}
f^{*}\colon&[X_{1},Y]\to[X_{0},Y]\\
&g\mapsto g\circ f
\end{split}
\end{equation}
We just need to prove $[-,Y]$ satisfies the axioms for a cohomology theory---namely,
\begin{subequations}
  \begin{align}
    H^{k}(X;\ZZ)&=[X,K(\ZZ,k)]\\
\intertext{and for reduced cohomologies,}
\widetilde{H}^{k}(X;\ZZ)&=\langle X,K(\ZZ,k)\rangle,
  \end{align}
\end{subequations}
where $\langle-,-\rangle$ are the homotopy equivalence classes of
based maps up to based homotopy. Also note, we can replace the
coefficient group $\ZZ$ with any Abelian group $G$.
\end{idea}