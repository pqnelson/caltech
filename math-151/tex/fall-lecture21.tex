%%
%% fall-lecture21.tex
%% 
%% Made by Alex Nelson <pqnelson@gmail.com>
%% Login   <alex@lisp>
%% 
%% Started on  2025-11-15T10:28:01-0800
%% Last update 2025-11-15T10:28:01-0800
%% 

\lecture{}

\begin{definition}
Let $(C,\boundary)$ be a a chain complex (and $G$ is some fixed
Abelian group), define
\begin{equation}
C^{*}_{n}:=\hom(C_{n},G)
\end{equation}
to be the \define{Dual Group} of $C_{n}$ (where we fixed the Abelian
group of coefficients $G$).

We have the \define{Coboundary Map} $\coboundary_{n}\colon C^{*}_{n-1}\to C^{*}_{n}$
correspondding to the boundary map $\boundary\colon C_{n}\to C_{n-1}$.
Since $\boundary^{2}=0$, we find $\coboundary^{2}=0$.

Then we have
\begin{equation}
C^{*}_{n+1}\xleftarrow{\coboundary_{n+1}}C^{*}_{n}\xleftarrow{\coboundary_{n}}C^{*}_{n-1},
\end{equation}
where $\ker(\coboundary_{n+1})\supset\Im(\coboundary_{n})$. We define
the \define{Cohomology Group} of $C$ with coefficients $G$ as
\begin{equation}
H^{n}(C;G) := \ker(\coboundary_{n+1})/\Im(\coboundary_{n}).
\end{equation}
Lastly, we call $(C^{*},\coboundary)$ the \define{Cochain Complex}
with coefficients $G$.
\end{definition}

\begin{node}[Geometric meaning to this stuff]
Elements of $C_{n}^{*}$ are \define{$n$-Dimensional Cochains} which
gives us a way to evaluate a chain. That is to say, given some
$\alpha\in C^{*}_{n}$, it's just a morphism $\alpha\colon C_{n}\to G$
which maps $x\in C_{n}$ to $\alpha(x)$. We usually write
\begin{equation}
\langle\alpha, x\rangle :=\alpha(x)
\end{equation}
for the result of this mapping.

The coboundary map $\coboundary\colon C^{*}_{n}\to C^{*}_{n+1}$.
If $\alpha\in C^{*}_{n}$, then $\coboundary\alpha\in C^{*}_{n+1}$;
i.e., $\coboundary\alpha\colon C_{n+1}\to G$. For any $y\in C_{n+1}$,
we define
\begin{equation}
\langle\coboundary\alpha,y\rangle:=\langle\alpha,\boundary y\rangle.
\end{equation}
In this sense, $\coboundary$ is ``dual'' to the boundary $\boundary$.

A cohomology class $[\alpha]\in H^{n}(C;G)$ defines a morphism
$H_{n}(C)\to G$. How do we see this? Let $[x]\in H_{n}(C)$ be a
homology class. Then
\begin{equation}
\begin{split}
[\alpha]&\colon H_{n}(C)\to G\\
&[x]\mapsto\langle\alpha,x\rangle.
\end{split}
\end{equation}
Although this appears to depend on the choice of representatives
$x\in[x]$ and $\alpha\in[\alpha]$, using
\begin{equation}
\langle\coboundary\alpha,y\rangle=\langle\alpha,\boundary y\rangle
\end{equation}
and after a bit of work, we see this does not depend o nthe choice of
representatives. So we have a morphism $H^{n}(C;G)\to\hom(H_{n}(C),G)$.
(This is a surjective morphism, and its kernel is the Ext functor.)
\end{node}

\subsection{Universal Coefficient Theorem for Cohomology}

So how do we relate cohomology and homology? Well, homology really
determines the cohomology. We can make this precise in the following theorem:

\begin{universal-coef-thm}[Cohomology]
Suppose we have a chain complex $(C_{*},\boundary)$ where $C_{*}$ is
free Abelian (which is always true for chain complexes constructed
from topological spaces). Then we have a split short exact sequence
\begin{equation}
0\to\Ext(H_{n-1}(C),G)\to H^{n}(C;G)\to\hom(H_{n}(C),G)\to0
\end{equation}
and this is natural with respect to chain maps (in the sense that
chain maps will induce natural transformations making the diagram commute:
\begin{equation}
\vcenter{\xymatrix{0\ar[r]&\bullet\ar[r]&\bullet\ar[r]&\bullet\ar[r]&0\\
0\ar[r]&\bullet\ar[u]\ar[r]&\bullet\ar[u]\ar[r]&\bullet\ar[u]\ar[r]&0}}
\end{equation}
but the splitting is not natural.)
\end{universal-coef-thm}

\begin{corollary}
Let $G=\ZZ$. Suppose $H_{n}(C)$ and $H_{n-1}(C)$ are
finitely-generated. If $T_{i}$ is the torsion subgroup of $H_{i}(C)$,
then $H^{n}(C;\ZZ)\iso\bigl(H_{n}(C)/T_{n}\bigr)\oplus T_{n-1}$.
\end{corollary}

\begin{proof}[Proof (corollary)]
Let $H_{i}(C)=\ZZ^{b_{i}}\oplus T_{i}$ where $T_{i}=\bigoplus_{j}(\ZZ/m_{ij}\ZZ)$.
We see
\begin{subequations}
  \begin{align}
\hom(H_{n}(C),\ZZ) &= \hom(\ZZ^{b_{n}},\ZZ)\oplus\hom(T_{n},\ZZ)\\
&\iso\ZZ^{b_{n}}
  \end{align}
\end{subequations}
and
\begin{subequations}
  \begin{align}
\Ext(H_{n-1}(C),\ZZ) &=\Ext(\ZZ^{b_{n-1}}\oplus T_{n-1},\ZZ)\\
&=\Ext(T_{n-1},\ZZ)\\
&\iso T_{n-1}.
  \end{align}
\end{subequations}
Combining the results proves the corollary.
\end{proof}

\begin{proof}[Proof (Universal Coefficient Theorem)]
Recall, $Z_{n}=\ker(\boundary_{n})$ is the group of cycles, and
$B_{n}=\Im(\boundary_{n+1})$ is the group of boundaries, which are
both subgroups of $C_{n}$. In particular, they are free Abelian
groups. Now considerthe short exact sequence
\begin{equation}
0\to Z_{n+1}\to C_{n+1}\xrightarrow{\boundary}B_{n}\to 0.
\end{equation}
Since the $B_{n}$ are free, the short exact sequence splits. Moreover,
we can extend this since $\boundary\colon C_{n+1}\to C_{n}$, but
$\boundary|_{Z_{n+1}}$ and $\boundary|_{B_{n}}$ are just the zero
maps, so we have
\begin{equation}
\vcenter{\xymatrix{%
        & \vdots\ar[d] & \vdots\ar[d] & \vdots\ar[d] & \\
0\ar[r] & Z_{n+1}\ar[r]\ar[d]^{0} & C_{n+1}\ar[r]^{\boundary}\ar[d]^{\boundary} & B_{n}\ar[r]\ar[d]^{0} & 0\\
0\ar[r] & Z_{n}\ar[r]\ar[d]^{0} & C_{n}\ar[r]^{\boundary}\ar[d]^{\boundary} & B_{n-1}\ar[r]\ar[d] & 0\\
& \vdots & \vdots & \vdots & }}
\end{equation}
We can dualize these short exact sequences, and we get the short exact sequences;
\begin{equation}
\vcenter{\xymatrix{%
        & \vdots & \vdots & \vdots & \\
0 & Z^{*}_{n+1}\ar[l]\ar[u]^{0} & C^{*}_{n+1}\ar[l]^{\coboundary}\ar[u]^{\coboundary} & B^{*}_{n}\ar[l]\ar[u] & 0\ar[l]\\
0 & Z^{*}_{n}\ar[l]\ar[u]^{0} & C^{*}_{n}\ar[l]^{\coboundary}\ar[u]^{\coboundary} & B^{*}_{n-1}\ar[l]\ar[u]^{0} & 0\ar[l]\\
        & \vdots\ar[u] & \vdots\ar[u] & \vdots\ar[u] & }}
\end{equation}
The short exact sequences in the dualized version are all split (which
is not true in general!). We then have the long exact sequence by
extending each column and then we get (by applying cohomology)
\begin{equation}
\dots\gets B^{*}_{n}\xleftarrow{i^{*}_{n}}Z^{*}_{n}\gets
H^{n}(C;G)\gets B^{*}_{n-1}\xleftarrow{i^{*}_{n-1}}Z^{*}_{n-1}\gets\dots
\end{equation}
the connecting morphism is determined by the inclusion
\begin{equation}
i_{n}\colon B_{n}\into Z_{n}.
\end{equation}
So now we have a short exact sequence
\begin{equation}
0\gets\ker(i^{*}_{n})\gets H^{n}(C;G)\gets\coker(i^{*}_{n-1})\gets0.
\end{equation}
Now, we just need to understand the kernel and cokernel appearing in
this short exact sequence.

Consider the short-exact sequence from the definition of $H_{n}(C)$,
\begin{equation}
0\to B_{n}\xrightarrow{i_{n}}Z_{n}\to H_{n}(C)\to 0,
\end{equation}
then we dualize it
\begin{equation}
\vcenter{\xymatrix{
0=\Ext(Z_{n},G) & \ar[l]\Ext(H_{n}(C),G) & & \\
B^{*}_{n}\ar[ur] & \ar[l]^{i^{*}_{n}}Z^{*}_{n} & \hom(H_{n}(C),G)\ar[l] & 0\ar[l]}}
\end{equation}
So $\ker(i^{*}_{n})\iso\hom(H_{n}(C),G)$ and $\coker(i^{*}_{n})\iso\Ext(H_{n}(C),G)$.
This gives us the short exact sequence we sought. In other words, this
establishes the \emph{existence} of the short exact sequence.

Now, we just need to prove the short exact sequence is split. Plugging
in our results
\begin{equation}
0\to\Ext(H_{n-1}(C),G)\to H^{n}(C,G)\to\hom(H_{n}(C),G)\to 0.
\end{equation}
Consider the short exact sequence
\begin{equation}
0\to Z_{n}\xrightarrow{i}C_{n}\xrightarrow{\boundary}B_{n-1}\to0,
\end{equation}
which is split. So let
\begin{equation}
p\colon C_{n}\to Z_{n}
\end{equation}
be a left inverse of $i$ (i.e., $p|_{Z_{n}}$ is the identity morphism).
Suppose $\xi\in\ker(i^{*}_{n})$ (i.e., $\xi\colon Z_{n}\to G$ and $\xi|_{B_{n}}=0$).
Then consider
\begin{equation}
p^{*}\colon Z^{*}_{n}\to C^{*}_{n}
\end{equation}
where $p^{*}(\xi)=\xi\circ p\colon C_{n}\to G$. For $\boundary x\in B_{n}$
arbitrary, we find
\begin{subequations}
  \begin{align}
\langle p^{*}(\xi),\boundary x\rangle &= \xi\circ p\circ(\boundary x)\\
&=\xi(\boundary x)\\
&=0.
  \end{align}
\end{subequations}
So $p^{*}(\xi)\in\ker(\coboundary)$ since
\begin{equation}
\langle\coboundary p^{*}(\xi),x\rangle=\langle p^{*}(\xi),\boundary x\rangle=0,
\end{equation}
and this is true for every $x$. Then we have a map obtained by
composing
\begin{equation}
\ker(i^{*}_{n})\xrightarrow{p^{*}}\ker(\coboundary)\to H^{n}(C;G),
\end{equation}
which we claim is a right inverse to
\begin{equation}
H^{n}(C;G)\to\hom(H_{n}(C),G)\to 0
\end{equation}
appearing in the short exact sequence. Then the short exact sequence
is split.

Hence the result.
\end{proof}

\subsection{Universal Coefficient Theorem for Homology}

\begin{node}[Tor functor]
Recall (\textit{c.}\S\ref{defn:ext-functor}), Ext fnctor was derived from the $\hom(-,G)$ functor.
We can derive a Tor functor from the tensor product. What is this?
Suppose
\begin{equation}
A\to B\to C\to 0
\end{equation}
is exact. Then ``right exactness'' of the tensor product tells us
\begin{equation}
A\otimes G\to B\otimes G\to C\otimes G\to 0
\end{equation}
is exact. How do we define Tor?

Well, consider a free resolution
\begin{equation}
\dots\to F_{1}\xrightarrow{\boundary}F_{0}\to A\to0
\end{equation}
(there may be more entries in the free resolution, but we only need
this). Then tensor it with $G$ to give us
\begin{equation}
\dots\to F_{1}\otimes G\xrightarrow{\boundary_{*}}F_{0}\otimes G\to A\otimes G\to0,
\end{equation}
the homology can be determined using $\coker(\boundary_{*})=A\otimes G$
and $\ker(\boundary_{*})=:\Tor(A,G)$.
\end{node}

\begin{node}
From the short exact sequence
\begin{equation}
0\to A\to B\to C\to 0,
\end{equation}
we can form the long exact sequence
\begin{equation}
\vcenter{\xymatrix{
0\ar[r] &\Tor(A,G)\ar[r] &\Tor(B,G)\ar[r] &\Tor(C,G)\ar[dll] &\\
&A\otimes G\ar[r]&B\otimes G\ar[r]&C\otimes G\ar[r] & 0}}
\end{equation}
\end{node}


\begin{universal-coef-thm}[Homology]
Let $(C_{*},\boundary)$ be a chain complex, $C_{*}$ be a free Abelian group.
Then we have a split short exact sequence
\begin{equation}\label{eq:fall-2025:lec21:uct-homology}
0\to H_{n}(C)\otimes G\to H_{n}(C;G)\to\Tor(H_{n-1}(C),G)\to 0.
\end{equation}
It is natural, but the splitting is not.
\end{universal-coef-thm}

\begin{note}
All the groups in the short exact sequence in Equation~\eqref{eq:fall-2025:lec21:uct-homology}
are supposed to be the \underline{\emph{homology}} groups
(\underline{none} of them are cohomology groups!).
\end{note}