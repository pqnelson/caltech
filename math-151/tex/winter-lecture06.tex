%%
%% winter-lecture06.tex
%% 
%% Made by Alex Nelson <pqnelson@gmail.com>
%% Login   <alex@lisp>
%% 
%% Started on  2026-01-19T09:32:16-0800
%% Last update 2026-01-19T09:32:16-0800
%% 

\lecture[CW Approximation]{}

\begin{remark}
The
$\pi_{i}(\sphere{2})\iso\pi_{i}(\sphere{2}\times\sphere{\infty})\iso\pi_{i}(\sphere{2}\times\RP^{\infty})$
example has some subtle issues with it. In particular, these
isomorphisms hold for $i\geq2$, but for $i=1$ we have
$\pi_{1}(\RP^{2})\iso\ZZ/2\ZZ$, but $\pi_{1}(\sphere{2})=0$. So care
must be taken!
\end{remark}

\begin{lemma}[Review from last time]
A CW pair $(X,A)$ is $n$-connected if all the cells in $X\setminus A$
have dimension strictly greater than $n$. The first time you'll see a
difference between $X$ and $A$ is for $(n+1)$-dimensional cells, and
$A^{(i)}=X^{(i)}$ for all $i\leq n$.
\end{lemma}

\begin{theorem}[CW approximation]% Hatcher, Proposition 4.13, PDF pg 362
Every topological space $X$ has a CW approximation $f\colon Z\to X$.
(If $X$ is path-connected, then $Z^{(0)}$ can be taken to be a single point.)
\end{theorem}

\begin{proof}
Without loss of generality, assume $X$ is path-connected (otherwise
deal with each path-connected component separately). By induction, we
will keep adding cells of higher and higher dimensions.

Start with a single 0-cell $A^{(0)}=a_{0}$ with a map $f\colon A^{(0)}\to X$.

For 1-cells, consider a presentation of the fundamental group
$\pi_{1}(X,x_{0})$ where $f(a_{0})=x_{0}\in X$. Then for each
generator $\gamma$ of $\pi_{1}(X,x_{0})$ attach a 1-cell
$e_{1}^{\gamma}$ to $a_{0}$ and call the result $A^{(1)}$, and extend
the map $f\colon(A^{(1)},a_{0})\to(X,x_{0})$ to send $e^{\gamma}_{1}\mapsto\gamma$.
Then
\begin{equation}
f_{*}\colon\pi_{1}(A^{(1)},a_{0})\to\pi_{1}(X,x_{0})
\end{equation}
is a surjection \emph{but not necessarily an injection}.

The way to make it an injection: we attach a 2-cell to kill guys in
the kernel of $f_{*}$. This also makes $f_{*}$ a surjection on $\pi_{2}$.
We do it in two steps: (i) kill everything in the kernel, (ii) add
enough to make it surjective on $\pi_{2}$.

For $\pi_{1}$, we fix a presentation of $\ker(f_{*}\colon\pi_{1}(A^{(1)},a_{0})\to\pi_{1}(X,x_{0}))$
and for each generator $\beta$ of this kernel we attach a 2-cell
$\disk{2}_{\beta}$ such that the gluing map
\begin{equation}
\varphi_{\beta}\colon\disk{2}_{\beta}\to A^{(1)}
\end{equation}
is $\beta$. We call the resulting space $B^{(2)}$ obtained from
$A^{(1)}$ in this way.

Then we extend the map from $f$ to a map
\begin{equation}
f\colon B^{(2)}\to X.
\end{equation}
For the 2-cell $\disk{2}_{\beta}$ note that $\beta\in\ker(f_{*}\colon\pi_{1}(A^{(1)})\to\pi_{1}(X))$,
so
\begin{equation}
f_{*}\beta\homotopic c_{x_{0}}
\end{equation}
is described by the homotopy $H$. Then we extend the map $f$ to
include $\disk{2}_{\beta}$ using this very homotopy $H$. Then the
induced map
\begin{equation}
f_{*}\colon\pi_{1}(B^{(2)},a_{0})\to\pi_{1}(X,x_{0})
\end{equation}
is now an isomorphism.

For surjectivity on $\pi_{2}$: We fix a presentation of $\pi_{2}(X,x_{0})$.
For each generator $\delta$ of $\pi_{2}(X,x_{0})$ we attach a 2-cell
$\disk{2}_{\delta}$ such that $\boundary\disk{2}_{\delta}$ is mapped
to $a_{0}$. We call the resulting space $A^{(2)}$ and extend the map
$f$ to
\begin{equation}
f\colon A^{(2)}\to X
\end{equation}
using $\delta$ on $\disk{2}_{\delta}$. (This is the same as attaching
$(\sphere{2},\point{*})$ to the complex.)

Now, consider the long exact sequence of the CW pair $(A^{(2)},B^{(2)})$:
\begin{equation}
\pi_{2}(A^{2},B^{2})\xrightarrow{\boundary}\pi_{1}(B^{2})\to\pi_{1}(A^{2})\to\pi_{1}(A^{2},B^{2})=0.
\end{equation}
We claim $\pi_{1}(A^{2},B^{2})=0$ because we're attacing 2-spheres, so
by the Lemma it is zero. We also claim the
$\boundary\colon\pi_{2}(A,B)\to\pi_{1}(B)$ morphism is zero. Therefore
\begin{equation}
\pi_{1}(B^{(2)})\iso\pi_{1}(A^{(2)}).
\end{equation}
We also have $f$ induce an isomorphism
$\pi_{1}(B^{(2)})\to\pi_{1}(X)$. So our new $f$ induces
$f\colon\pi_{1}(A^{(2)})\to\pi_{1}(X)$ which must be an isomorphism by exactness.

Then we get
\begin{equation}
f\colon A^{(2)}\to X
\end{equation}
such that
\begin{equation}
f_{*}\colon\pi_{i}(A^{(2)},a_{0})\to\pi_{i}(X,x_{0})
\end{equation}
is an isomorphism for $i<2$ and surjective for $i=2$. This was the
situation we were in when we attached 2-cells. So we have a way to
attach 3-cells without affecting $\pi_{1}$ and $\pi_{2}$, and we cann
inductively add $n$-cells in a similar manner. Then we take the direct
limit
\begin{equation}
A = \lim_{n\to\infty}(A^{(0)}\into A^{(1)}\into A^{(2)}\into\dots A^{(n)}).
\end{equation}
Hence the result.
\end{proof}

\begin{remark}
We can also do a relative CW approximation for pairs $(X,X_{0})$ where
$X_{0}\subset X$ is a subspace, there exists a CW pair $(A,A_{0})$ and
a map
\begin{equation}
f\colon(A,A_{0})\to(X,X_{0})
\end{equation}
such that $f|_{A_{0}}\colon A_{0}\to X_{0}$ and also $f_{*}\colon\pi_{n}(A,A_{0})\to\pi_{n}(X,X_{0})$
are isomorphisms (and $f$ and $f_{0}$ (?) are weakly homotopy equivalent).
\end{remark}

\subsection{Some Applications}

\begin{proposition}% Hatcher, Proposition 4.15, PDF pg 362
If $(X,A)$ is an $n$-connected CW pair, then there exists a CW pair
$(Z,A)$ such that
\begin{enumerate}
\item $Z\setminus A$ has cells of dimension at least $n+1$
\item $(Z,A)\homotopic(X,A)\rel{A}$.
\end{enumerate}
\end{proposition}

\begin{proof}
(Modulo the $\rel{A}$ part) It is enough to construct a CW complex $Z$
such that
\begin{enumerate}
\item $f\colon Z\to X$ is a weak homotopy equivalence, and
\item $A\subset Z$ and
\item $f|_{A}=\id_{A}$ and
\item $Z\setminus A$ has cells of dimension strictly greater than $n$.
\end{enumerate}
Sounds good? Great!

To get $Z$, we repeat the construction in the CW approximation
starting with $(n+1)$-cells, and define $f\colon Z\to X$ as in the
proof. By the same argument, we see
\begin{equation}
f_{*}\colon\pi_{i}(Z)\to\pi_{i}(X)
\end{equation}
is an isomorphism for $i\geq n+1$, and also for $i<n$. The critical
question is what happens for $i=n$?

This is an injective morhpism, but we need to prove it is surjective;
i.e., we want to prove
\begin{equation}
f_{*}\colon\pi_{n}(Z)\to\pi_{n}(X)
\end{equation}
is surjective. We have the following commutative diagram
\begin{equation}
\vcenter{\xymatrix{Z\ar[rr]^{f} & & X\\
&\ar[ul]^{i}A\ar[ur]_{j} &}}
\end{equation}
where $i\colon A\into Z$ and $j\colon A\into X$ are the obvious
inclusions, and the diagram commutes $j=f\circ i$. Then we look at the
induced maps on the homotopy groups:
\begin{equation}
\vcenter{\xymatrix{\pi_{n}(Z)\ar[rr]^{f_{*}} & & \pi_{n}(X)\\
&\ar[ul]^{i_{*}}\pi_{n}(A)\ar[ur]_{j_{*}} &}}
\end{equation}
The long exact sequence of homotopy groups gives us
\begin{equation}
\pi_{n}(A)\xrightarrow{j_{*}}\pi_{n}(X)\to\pi_{n}(X,A)=0,
\end{equation}
where $\pi_{n}(X,A)=0$ is by the $n$-connectedness of $(X,A)$. The map
$j_{*}$ is therefore surjective by exactness. Hence $f_{*}$ must be
surjective by the commutative diagram.
\end{proof}