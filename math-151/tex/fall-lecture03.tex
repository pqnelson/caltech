%%
%% fall-lecture03.tex
%% 
%% Made by Alex Nelson <pqnelson@gmail.com>
%% Login   <alex@lisp>
%% 
%% Started on  2025-10-05T10:21:13-0700
%% Last update 2025-10-05T10:21:13-0700
%% 

\lecture{}

\begin{example}
If $X$ is contractible, then $\pi_{1}(X)\iso\TrivialGroup$ is the trivial group.
\end{example}

\begin{example}
For the circle, $\pi_{1}(\sphere{1})\iso\ZZ$.
\end{example}

\begin{example}
Let $X_{1}$ and $X_{2}$ be topological spaces.
Then $\pi_{1}(X_{1}\times X_{2})\iso\pi_{1}(X_{1})\times\pi_{1}(X_{2})$.
As a consequence, $\pi_{1}(\torus{n})\iso\ZZ^{n}$ where
$\torus{n}\iso\sphere{1}\times\cdots\times\sphere{1}$ is the $n$-torus
(where there are $n$ copies of $\sphere{1}$).
\end{example}

\begin{definition}
We say the topological space $X$ is \define{Simply-Connected} (or \emph{1-Connected}) if it is
path-connected and $\pi_{1}(X)=\TrivialGroup$.
\end{definition}

\begin{example}
We see $\RR^{n}$ is simply-connected since $\RR^{n}$ is contractible
and path-connected.
\end{example}

\begin{proposition}
Given a base-point preserving map $\varphi\colon(X,x_{0})\to(Y,y_{0})$,
we have an induced group morphism $\varphi_{*}\colon\pi_{1}(X,x_{0})\to\pi_{1}(Y,y_{0})$.
If further $\psi\colon(Y,y_{0})\to(X,x_{0})$ is a base-point
preserving map, if $\varphi\homotopic\psi$ is homotopic $\rel{x_{0}}$,
then $\varphi_{*}=\psi_{*}$ (they induce the same group morphism).
\end{proposition}

\begin{theorem}[Van Kampen]
Let $X=A\cup B$ and $A$, $B$ both be path-connected open sets, and
$A\cap B$ is path-connected, and $x_{0}\in A\cap B$.
Then the induced maps
\begin{subequations}
\begin{equation}
i_{A}\colon\pi_{1}(A\cap B,x_{0})\to\pi_{1}(A,x_{0}),
\end{equation}
and
\begin{equation}
i_{B}\colon\pi_{1}(A\cap B,x_{0})\to\pi_{1}(B,x_{0})
\end{equation}
\end{subequations}
we may construct the map (not group morphism)
\begin{equation}
  \begin{split}
    i\colon\pi_{1}(A\cap B,x_{0})&\to\pi_{1}(A,x_{0})*\pi_{1}(B,x_{0})\\
    \alpha&\mapsto i_{A}(\alpha)i_{B}(\alpha)^{-1}.
  \end{split}
\end{equation}
Let $N\subgroup\pi_{1}(A)*\pi_{1}(B)$ be normally generated by the
image $i\bigl(\pi_{1}(A\cap B)\bigr)$ --- this is the smallest normal
subgroup containing it.

\textbf{Then} $\pi_{1}(X)\iso\bigl(\pi_{1}(A)*\pi_{1}(B)\bigr)/N$,
where $G*H$ is the free product of the group $G$ with the group $H$.
\end{theorem}

\begin{remark}[van Kampen using group presentations]
Alternatively, suppose we have a presentation of
\begin{equation}
\pi_{1}(A)=\langle g_{\alpha}\mid r_{\alpha'}\rangle
\end{equation}
in terms of generators $g_{\alpha}$ satisfying relations $r_{\alpha'}$,
and we have a similar presentation for $\pi_{1}(B)$ as 
\begin{equation}
\pi_{1}(B)=\langle g_{\beta}\mid r_{\beta'}\rangle.
\end{equation}
We let $g_{\gamma}$ be the generators of $\pi_{1}(A\cap B)$.
Then
\begin{equation}
\pi_{1}(X) = \langle g_{\alpha}, g_{\beta}\mid r_{\alpha'},
r_{\beta'}, i_{A}(g_{\gamma})i_{B}(g_{\gamma})^{-1}\rangle.
\end{equation}
So we introduce a new set of relations (in addition to those for
$\pi_{1}(A)$ and $\pi_{1}(B)$) given by $i_{A}(g_{\gamma})i_{B}(g_{\gamma})^{-1}=1$.
\end{remark}

\begin{remark}[Picking $A$ and $B$ for van Kampen]
In practice, we pick $A$ and $B$ such that they are each deformation
retraction of a neighborhood. For example, when we have the following space
\begin{equation*}
\includegraphics{img/img.0}
\end{equation*}
we pick $A$ such that there exists a neighborhood $U_{A}$ (shaded
red region) of $X$
containing $A\propersubset U_{A}$ and $U_{A}$ deformation retracts to $A$:
\begin{equation*}
\includegraphics{img/img.1}
\end{equation*}
Then we can similarly pick $U_{B}$ (shaded blue region) to be an open set containing
$B\propersubset U_{B}$ and $U_{B}$ deformation retracts to $B$:
\begin{equation*}
\includegraphics{img/img.2}
\end{equation*}
Now when we have a path $a$ in $A$, and a path $b$ in $B$, we see the
loop given by $a\bar{b}$ is null-homotopic:
\begin{equation*}
\includegraphics{img/img.3}
\end{equation*}
\end{remark}

\begin{proposition}
If $A\cap B$ is simply-connected, then $\pi_{1}(X)\iso\pi_{1}(A)*\pi_{1}(B)$
is the free product of $\pi_{1}(A)$ and $\pi_{1}(B)$.
\end{proposition}

\begin{example}
Consider the ``figure 8'' graph $G$ with 1 node and 2 edges. Then
$\pi_{1}(G)\iso\ZZ*\ZZ$ is the free group of rank 2.
\end{example}

\begin{example}
Consider the 1-point union [``wedge sum''] of an arbitrary family of circles 
$X=\bigvee_{\alpha}\sphere{1}_{\alpha}$, then
\begin{equation}
\pi_{1}(X)\iso\freeprod_{\alpha}\ZZ
\end{equation}
is the free product of just as many copies of $\ZZ$.
\end{example}

\begin{example}
Any connected graph $G\homotopic \bigvee_{\alpha}\sphere{1}_{\alpha}$,
so $\pi_{G}$ is a free group.
\end{example}

\begin{example}
Let $\Sigma_{g}$ be a closed orientable surface of genus $g$. This can
be formed by taking $2g$ circles sharing the same base point, then
taking a $4g$-gon and attaching it to the $\bigvee^{2g}\sphere{1}$
by identifying its sides as doodled thus:
\begin{equation*}
\includegraphics{img/img.4}
\end{equation*}
We take $A$ to be the quotient of a tabular neighborhood of the
boundary of the $4g$-gon (the red region), so $A\homotopic\bigvee^{2g}\sphere{1}$.
Then we identify $B$ (the blue region) with the interior of the $4g$-gon, so that
$A\cap B\homotopic\boundary B\homotopic\sphere{1}$:
\begin{equation*}
\includegraphics{img/img.5}\qquad\includegraphics{img/img.6}
\end{equation*}
This means
\begin{equation}
\pi_{1}(\Sigma_{g})\langle \underbrace{a_{1},b_{1},\dots,a_{g},b_{g}}_{g_{\alpha}},\underbrace{1}_{g_{\beta}}\mid \underbrace{1}_{r_{\alpha'}}, \underbrace{1}_{r_{\beta'}}, \underbrace{[a_{1},b_{1}](\cdots)[a_{g},b_{g}]}_{i_{A}(\boundary{B})}\cdot1\rangle
\end{equation}
where $[a,b]=aba^{-1}b^{-1}$ is the commutator of group elements. This means
\begin{equation}
\pi_{1}(\Sigma_{g})\langle a_{1},b_{1},\dots,a_{g},b_{g}\mid [a_{1},b_{1}](\cdots)[a_{g},b_{g}]\rangle.
\end{equation}
\end{example}

\begin{node}
In general if $X$ is obtained by attaching $2$-cells $e^{2}_{\beta}$
to the one-point union of circles $\bigvee_{\alpha}\sphere{1}$, then
\begin{equation}
\pi_{1}(X)\iso\langle g_{\alpha}\mid r_{\beta}\rangle,
\end{equation}
where each $g_{\alpha}$ corresponds to a $\sphere{1}_{\alpha}$ and
each $r_{\beta}$ corresponds to $\boundary e^{2}_{\beta}$ and how it
is attached to $\bigvee_{\alpha}\sphere{1}_{\alpha}$.
\end{node}

\begin{proposition}
Attaching $q$-cells (for $q\geq3$) does not change the fundamental group.
\end{proposition}

(The boundary of the $q$-cell is simply-connected, which will not
change anything.)

This gives us a way to compute $\pi_{1}$ for any CW complex $X$,
\begin{equation}
\pi_{1}(X)\iso\pi_{1}(X^{2}).
\end{equation}
We can simply discard any higher cells. But we see for any CW Complex
$X$ that $X^{1}$ is a graph (if $X$ is connected) so
$X^{1}\homotopic\bigvee_{\alpha}\sphere{1}_{\alpha}$. So $X^{2}$ is
homotopy equivalent to 2-cells attached to
$\bigvee_{\alpha}\sphere{1}_{\alpha}$.
This gives us an easy way to compute the fundamental group of any CW
Complex.

\begin{definition}
A map $p\colon\widetilde{X}\to X$ is a \define{Covering Map} if for
each $x\in X$ there exists a $U\subset X$ such that $x\in U$ and $U$
is open and $p$ sends each component of $p^{-1}(U)$ (the pre-image of $U$)
onto $U$. Here $\widetilde{X}$ is called a \define{Covering Space}
(or just a \define{Cover}) of $X$.
\end{definition}

\begin{example}
Consider $p\colon\RR\to\sphere{1}$ sending $t\mapsto\exp(\I t)$ where
$\I=\sqrt{-1}$.
\begin{equation*}
\includegraphics{img/img.7}
\end{equation*}
The red intervals in $\RR$ are mapped to the black interval on the circle.
\end{example}

\begin{example}
Consider the wedge sum $\sphere{1}\wedge\sphere{1}$ drawn as:
\begin{equation*}
\includegraphics{img/img.8}
\end{equation*}
Some examples of 2-fold coverings, where each red edge is mapped to
the red circle (and the blue edges are mapped to the blue circle):
\begin{equation*}
\includegraphics{img/img.9}\qquad\includegraphics{img/img.10}
\end{equation*}
Some examples of 3-fold coverings:
\begin{equation*}
\includegraphics{img/img.11}\qquad\includegraphics{img/img.12}
\end{equation*}
\end{example}

\begin{slogan}
For ``good'' spaces $X$, there is a bijection between connected
covering spaces of $X$ and ubgroups of $\pi_{1}(X,x_{0})$. And the
subgroup $H\subgroup\pi_{1}(X,x_{0})$ is isomorphic to
$H\iso\pi_{1}(\widetilde{X})$. 
\end{slogan}