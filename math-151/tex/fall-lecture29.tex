%%
%% fall-lecture29.tex
%% 
%% Made by Alex Nelson <pqnelson@gmail.com>
%% Login   <alex@lisp>
%% 
%% Started on  2025-12-06T10:04:04-0800
%% Last update 2025-12-06T10:04:04-0800
%% 

\lecture{}

\begin{example}
We can use the same argument to compute the cohomology for the
quaternionic projective spaces $\HP^{n}$. Here $\HH$ is the set of
quaternions, guys which look like $a+\I b+\J c+\K d$ where
$\I^{2}=\J^{2}=\K^{2}=\I\J\K=-1$ and the generators $\I$, $\J$, $\K$
anticommute with each other (but otherwise it's an associative algebra
over the reals).

Then we can consider $\HH^{n+1}$ as a right module over $\HH$. Then we
define
\begin{equation}
\HP^{n}:=(\HH^{n+1}\setminus0)/(\HH\setminus0)
\end{equation}
where we mod out by right scalar multiplication by arbitrary nonzero
quaternions. 

It's not hard to see that the cohomology \underline{groups} for the
projective quaternionic space look like
\begin{equation}
  H^{k}(\HP^{n})\iso\begin{cases}\ZZ & \mbox{if }k=0,4,8,\dots,4n\\
  0 & \mbox{otherwise}
  \end{cases}
\end{equation}
But what is the cohomology \underline{ring} for $\HP^{n}$? We can use
the previous proposition from last lecture to find
\begin{equation}
H^{*}(\HP^{n})\iso\ZZ[x]/(x^{n+1}),
\end{equation}
where $x$ is a generator for $H^{4}(\HP^{n})$.
\end{example}

\begin{corollary}
When $n$ is even, $\CP^{n}$ and $\HP^{n}$ do not admit an
orientation-reversing homeomorphism.
\end{corollary}

\begin{proof}[Proof idea]
If there were a homeomorphism $f\colon\CP^{2}\to\CP^{2}$, then it
should induce an isomorphism on the cohomology ring,
\begin{equation}
f^{*}\colon H^{*}(\CP^{2})\xrightarrow{\iso}H^{*}(\CP^{2}),
\end{equation}
which is a ring isomorphism
\begin{equation}
f^{*}\colon\ZZ[x]/(x^{3})\to\ZZ[x]/(x^{3}).
\end{equation}
Then $f^{*}(x)=\pm x$, but $f^{*}(x^{2})=+x^{2}$. This means it must
preserve the orientation.
\end{proof}

\begin{definition}
In general, let $M$ and $N$ be two closed connected oriented
$n$-manifolds. Then $H_{n}(M)\iso H_{n}(N)\iso\ZZ$. If $f\colon M\to N$
is continuous, then $f_{*}\colon H_{n}(M)\to H_{n}(N)$ is just
determined by an integer $d\in\ZZ$ by the fact that on the class of
$M$ it yield the class of $N$, i.e., $f_{*}([M])=[N]=d$. This is
called the \define{Degree} of $f$ and denoted $\deg(f):=d$.
\end{definition}

\begin{question}
We just proved, if $n$ is even, there is no $f\colon\CP^{n}\to\CP^{n}$
of degree $-1$.
\begin{enumerate}
\item What happens if $n$ is odd?
\item What are the possible degrees for any $n$?
\end{enumerate}
\end{question}

\begin{example}
We can also compute the cohomology ring for the real projective
spaces, but we need to use $\ZZ/2\ZZ$ coefficients because they're not
orientable. We find
\begin{equation}
H^{*}(\RP^{n};\ZZ/2\ZZ)\iso(\ZZ/2\ZZ)[x]/(x^{n+1}),
\end{equation}
where $x$ is a generator of the cohomology group $H^{1}(\RP^{n};\ZZ/2\ZZ)$.
\end{example}

\begin{remark}
I asked if this general pattern extends to the Octonionic projective
plane. The professor said it was a reasonable conjecture and
expectation, but he was unfamiliar with its topology and hesitated to
give a misleading answer.

Curiously, there is a homeomorphism $\OP^{1}\iso\sphere{8}$, and
according to Lackmann's ``The Octonionic Projective Plane''
(\arXiv{1909.07047}), Corollary 4.1 says
\begin{equation}
H^{k}(\OP^{2})\iso\begin{cases}\ZZ & \mbox{if }k=0,8,16\\
0 & \mbox{otherwise}
\end{cases}
\end{equation}
So that's nice.
\end{remark}

\begin{proposition}
If $f\colon\RP^{n}\to\RP^{m}$ is a continuous map such that the
induced map $f_{*}\colon H_{1}(\RP^{n})\to H_{1}(\RP^{m})$ is nonzero,
then $n\leq m$.
\end{proposition}

\begin{proof}
By the Universal Coefficient Theorem,
\begin{equation}
f^{*}\colon H^{1}(\RP^{m};\ZZ/2\ZZ)\to H^{1}(\RP^{n};\ZZ/2\ZZ)
\end{equation}
is nonzero (by the naturality part of the theorem), and $f^{*}$ is
dual to $f_{*}$ and $f_{*}$ is nonzero by hypothesis, therefore
$f^{*}$ is nonzero. This gives us a nonzero morphism
\begin{equation}
f^{*}\colon(\ZZ/2\ZZ)[x]/(x^{m+1})\to(\ZZ/2\ZZ)[y]/(y^{n+1}).
\end{equation}
Then $f^{*}(x)=y$ and then $f^{*}(x^{n})=y^{n}\neq0$. This forces
$n\leq m$.
\end{proof}

\begin{theorem}
Suppose there is an odd continuous map $f\colon\sphere{n}\to\sphere{m}$.
Then $n\leq m$.
\end{theorem}

\begin{proof}
The map $f$ induces the map
\begin{equation}
\overline{f}\colon\RP^{n}\to\RP^{m}
\end{equation}
such that the induced map on homology groups
\begin{equation}
\overline{f}_{*}\colon H_{1}(\RP^{n})\to H_{1}(\RP^{m})
\end{equation}
is nonzero. Then the result follows from the previous theorem.
\end{proof}

\begin{corollary}[Borsuk--Ulam]
If $g\colon\sphere{n}\to\RR^{n}$, then there exists an
$x\in\sphere{n}$ such that $g(-x)=-g(x)$.
\end{corollary}

\subsection{Generalizing Poincar\'{e} Duality}

\begin{remark}
There are two generalizations of Poincar\'{e} duality. We need to
introduce a few notions before we get to them, however.
\end{remark}

\begin{definition}
A $n$-dimensional \define{Manifold with Boundary}, as far as
topologists care, is like a manifold except for each point $x\in M$
there is an open neighborhood $x\in U\subset M$ such that $U$ is
homeomorphic either to $\RR^{n}$ or $\RR^{n}_{+}=\{(x_{1},\dots,x_{n})\in\RR^{n}\mid x_{n}\geq0\}$.
\end{definition}

\begin{definition}
When $M$ is a manifold with boundary, a point $x\in M$ is called a
\define{Boundary Point} if it has a neighborhood $x\in U\subset M$
such that $\varphi\colon U\to\RR^{n}_{+}$ is a homeomorphism
satisfying $\varphi(x)\in\boundary\RR^{n}_{+}\iso\RR^{n-1}$.

The \define{Boundary} of $M$ is the set of all boundary points of $M$
and we denote the boundary of $M$ by $\boundary M$.
\end{definition}

\begin{theorem}[Lefschetz duality (Hatcher 3.43)]
Let $M$ be a compact oriented $n$-manifold and $\boundary M=A\cup B$
where $A$ and $B$ Are compact $(n-1)$-manifolds with a common boundary
$\boundary A=\boundary B=A\cap B$. Then
\begin{equation}
\begin{split}
D_{M}\colon & H^{i}(M,A)\to H_{n-i}(M,B)\\
&\alpha\mapsto\alpha\frown[M],
\end{split}
\end{equation}
is an isomorphism, where $[M]\in H_{n}(M,\boundary M)$ is the fundamental class of $M$.
\end{theorem}

We usually use Lefschetz duality when either $A=\emptyset$ or
$B=\emptyset$, since it gives us $D_{M}\colon H^{i}(M,\boundary M)\to H_{n-i}(M)$
isomorphic.

\begin{proof}
\textsc{Case 1:} If $B=\emptyset$, then apply Poincar\'{e} duality for
$M\setminus\boundary M$. Then we get
\begin{equation}
H^{i}_{c}(M\setminus\boundary M)\iso H_{n-i}(M\setminus\boundary M)\iso H_{n-i}(M,\boundary M),
\end{equation}
and since $M$ is compact we see $H^{i}_{c}(M\setminus\boundary M)\iso H^{i}(M\setminus\boundary M)$,
which is the idea for the proof.

\textsc{Case 2:} The general case where both $A\neq\emptyset$ and
$B\neq\emptyset$. We have 2 long exact sequences, with vertical arrows
given by the cap product
\begin{equation}
\vcenter{\xymatrix{
H^{k}(M,\boundary M)\ar[r]\ar[d]^{[M]\frown} & H^{k}(M,A)\ar[r]\ar[d]^{[M]\frown} & H^{k}(\boundary M,A)\ar[r]\ar[d]^{[B]\frown} & H^{k+1}(M,\boundary M)\ar[r]\ar[d]^{[M]\frown} & H^{k+1}(M,A)\ar[d]^{[M]\frown}\\
H_{n-k}(M)\ar[r] & H_{n-k}(M,B)\ar[r] & H_{n-k-1}(B)\ar[r] & H_{n-k-1}(M)\ar[r] & H_{n-k-1}(M,B)}}
\end{equation}
The bottom row describes the pair $(M,B)$. By the Excision theorem, we
can remove the interior of $A$ to give us $H^{*}(\boundary M,A)\iso H^{*}(B, \boundary B)$.
Then we see that all vertical morphisms except the second are obviously
isomorphisms (possibly relying on Case 1). Then the Five Lemma implies the result.
\end{proof}

\begin{corollary}
If $M$ is a compact $n$-dimensional manifold with boundary and $n$ is
odd, then the Euler characteristic of the boundary $\chi(\boundary M)=2\chi(M)$.
\end{corollary}

\begin{proof}
Consider the long exact triangle
\begin{equation}
\vcenter{\xymatrix{H_{*}(\boundary M)\ar[rr] & & H_{*}(M)\ar[ld]\\
& \ar[lu] H_{*}(M,\boundary M) &}}
\end{equation}
and then the Euler characteristic,
\begin{equation}\label{eq:fall2025:math151a:lecture29:pf-step-star}
\chi(\boundary M)-\chi(M)+\chi(M,\boundary M)=0,
\end{equation}
but then by Lefschetz duality, we know
\begin{equation}
H^{i}(M,\boundary M;\ZZ/2\ZZ)\iso H_{n-i}(M;\ZZ/2\ZZ),
\end{equation}
and by the Universal Coefficient Theorem, the left-hand side is
isomorphic to
\begin{equation}
H_{i}(M,\boundary M;\ZZ/2\ZZ)\iso H^{i}(M,\boundary M;\ZZ/2\ZZ).
\end{equation}
Since $n$ is odd, we have
\begin{equation}
\chi(M)=-\chi(M,\boundary M),
\end{equation}
since the right-hand side is an alternating sum of $\dim H_{i}(M,\boundary M)$.
Plugging this back into Equation~\eqref{eq:fall2025:math151a:lecture29:pf-step-star}
gives the result.
\end{proof}

\begin{theorem}[Alexander duality (Hatcher 3.45)]
If $K$ is a compact subset of $\sphere{n}$ which is a retract of some
neighborhood, then
\begin{equation}
\widetilde{H}_{i}(\sphere{n}\setminus K)\iso\widetilde{H}^{n-i-1}(K)
\end{equation}
for all $i$.
\end{theorem}

\begin{proof}[Proof idea]
If $K$ is an $n$-manifold with boundary, then consider
\begin{subequations}
  \begin{align}
H_{i}(\sphere{n}\setminus K) &\iso H^{n-i}(\sphere{n}\setminus K)\mbox{ by Lefschetz}\\
&\iso H^{n-i}(\sphere{n},K)\mbox{ by Excision}\\
&\iso H^{n-i-1}(K) \mbox{ if }i\neq0,n-1.
  \end{align}
\end{subequations}
Then we use the long exact sequence
\begin{equation}
\vcenter{\xymatrix{H^{*}(K)\ar[rr] & & H^{*}(\sphere{n})\ar[ld]\\
& \ar[lu]^{[+1]}H^{*}(\sphere{n},K) & }}
\end{equation}
But $H^{k}(\sphere{n})$ is mostly zero, so
\begin{equation}
H^{n-i-1}(K)\iso H^{n-i}(\sphere{n},K),
\end{equation}
as desired.
\end{proof}

\begin{remark}
In practice, we only really need to consider the case where $K$ is a
manifold. Otherwise, we could consider a tubular neighborhood of $K$,
which is a manifold.
\end{remark}

\begin{corollary}
If $N$ is a nonorientable closed surface, then $N$ does not embed into
$\sphere{3}$. 
\end{corollary}