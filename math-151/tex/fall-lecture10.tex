%%
%% fall-lecture10.tex
%% 
%% Made by Alex Nelson <pqnelson@gmail.com>
%% Login   <alex@lisp>
%% 
%% Started on  2025-10-21T11:39:40-0700
%% Last update 2025-10-21T11:39:40-0700
%% 

\lecture{}

Last time we discussed $H_{*}(\sphere{n})$ and $H_{*}(T^{2})$.

\begin{example}
For $\sphere{1}$, consider
\begin{equation}
\begin{split}
\gamma\colon[0,1]\to\sphere{1}\\
t\mapsto\exp(\I2\pi t)
\end{split}
\end{equation}
then $H_{1}(\sphere{1})$ is generated by the equivalence class
$[\gamma]$. We see that $\boundary\gamma=\gamma(1)-\gamma(0)=0$.

We have a similar description for $H_{n}(\sphere{n})$. We have
$\sigma\colon\Delta^{n}\to\sphere{n}$ where $\sigma$ sends
$\boundary\Delta^{n}$ to a point $s_{0}\in\sphere{n}$ and $\sigma$
also sends $\Delta^{n}\setminus\boundary\Delta^{n}\to\sphere{n}\setminus\{s_{0}\}$
homeomorphically. Then $H_{n}(\sphere{n})$ is generated by
$[\sigma]$. This
$\Delta^{n}\setminus\boundary\Delta^{n}\to\sphere{n}\setminus\{s_{0}\}$
is possible because both are homeomorphic to $\interior{\disk{n}}$. This
$[\sigma]$ is also denoted by $[\sphere{n}]$ and called the
\define{Fundamental Class}.
\end{example}

\begin{example}
For the torus $T^{2}$, we can divide it into two annuli $A$ and $B$.
We can draw this as
\begin{equation*}
\mbox{[figure 1]}
\end{equation*}
Then we have the exact sequence
\begin{equation}
\underbrace{H_{2}(A)\oplus H_{2}(B)}_{=0}\to H_{2}(T^{2})\to
H_{1}(A\cap B)\to H_{1}(A)\oplus H_{1}(B)\top H_{1}(T^{2})\to\underbrace{\ZZ\to0}_{\text{see note after next eqn}}
\end{equation}
and
\begin{equation}
H_{0}(A\cap B)\xrightarrow{f}H_{0}(A)\oplus H_{0}(B)\to H_{0}(T^{2}).
\end{equation}
We see that $\ker(f)\iso\ZZ$, which means the connecting morphism
sends $H_{1}(T^{2})\to\ZZ$ surjectively, hence why we could decompose
the long exact sequence into ``two parts''.

Now, $H_{1}(A\cap B)\iso H_{1}(\sphere{1}\sqcup\sphere{1})\iso\ZZ\oplus\ZZ$,
and $H_{1}(A)\oplus H_{1}(B)\iso H_{1}(\sphere{1})\oplus H_{1}(\sphere{1})\iso\ZZ\oplus\ZZ$.
This map
\begin{equation}
(i_{A*},-i_{B*})\colon H_{1}(A\cap B)\to H_{1}(A)\oplus H_{1}(B).
\end{equation}
We name and orient the components of $A\cap B$ as $P$ and $Q$:
\begin{equation*}
\mbox{[figure 2]}
\end{equation*}
We see
that $Q\homotopic A\homotopic P$ and the homotopy equivalence sends a
generator of $H_{1}(P)$ to $H_{1}(A)$ and a generator of $H_{1}(Q)$ to
$H_{1}(A)$. There is a similar result for $H_{1}(B)$. So if we look at
\begin{equation}
\vcenter{\xymatrix{H_{1}(A\cap B)\ar[r]\ar@{=}[d] & H_{1}(A)\oplus H_{1}(B)\ar@{=}[d]\\
H_{1}(P)\oplus H_{1}(Q)\ar[ur]&\langle[A],[B]\rangle}}
\end{equation}
where $[A]$ generates $H_{1}(A)$ and $[B]$ generated $H_{1}(B)$.
Now, the morphism
\begin{equation}
(i_{A*},-i_{B*})\colon\langle[P],[Q]\rangle\to H_{1}(A)\oplus H_{1}(B)
\end{equation}
can be described by examining what happens to $[P]$ and $[Q]$:
\begin{subequations}
\begin{align}
[P] &\mapsto ([A], [B])\\
[Q] &\mapsto (-[A], -[B]).
\end{align}
\end{subequations}
So the image of $\Im(i_{A*},-i_{B*})\iso\ZZ$ and $\ker(i_{A*},-i_{B*})\iso\ZZ$.
Then the long exact sequence breaks up into two parts:
\begin{subequations}
\begin{equation}
0\to H_{2}(T^{2})\to\ZZ\to0
\end{equation}
which implies $H_{2}(T^{2})\iso\ZZ$, and
\begin{equation}
0\to\ZZ\xrightarrow{\varphi}\ZZ^{2}\to H_{1}(T^{2})\to\ZZ\to0,
\end{equation}
\end{subequations}
but the $\Im(\varphi)=\ZZ\langle(1,-1)\rangle$ and $\ZZ^{2}/\ZZ\langle(1,-1)\rangle\iso\ZZ$.
\end{example}

\subsection{Reduced Homology}

\begin{node}
If we have a chain complex $C_{*}(X)$,
\begin{equation}
\dots\to C_{2}(X)\to C_{1}(X)\to C_{0}(X)\to 0,
\end{equation}
then we can introduce $\widetilde{C}_{*}(X)$ which is formed by:
\begin{equation}
\dots\to C_{2}(X)\to C_{1}(X)\xrightarrow{\boundary_{1}} C_{0}(X)\xrightarrow{\varepsilon}\ZZ\to 0,
\end{equation}
where $\varepsilon$ is defined by 
\begin{equation}
\varepsilon\left(\sum_{i}n_{i}x_{i}\right)=\sum_{i}n_{i}.
\end{equation}
We claim that $\widetilde{C}_{*}(X)$ is a chain complex. We just need
to show $\varepsilon\circ\boundary_{1}=0$. If $\gamma\colon[0,1]\to X$
is a path, then
\begin{equation}
(\varepsilon\circ\boundary_{1})(\gamma)=\varepsilon(\gamma(1)-\gamma(0))=1-1=0,
\end{equation}
as desired.
\end{node}

\begin{definition}
We call $\varepsilon$ an \define{Augmentation}, and we call
$\widetilde{C}(X)$ and \define{Augmented} chain complex (or
\define{Reduced} chain complex). We denote its homology
$\widetilde{H}_{*}(X)$ and call it the \define{Reduced Homology} of $X$.
\end{definition}

\begin{theorem}
We see that $H_{n}(X)=\widetilde{H}_{n}(X)$ for $n>0$.
\end{theorem}

This should be obvious.

For $n=0$, we see that $H_{0}(X)\neq\widetilde{H}_{0}(X)$. What's
going on?

\begin{theorem}
$H_{0}(X)\iso\widetilde{H}_{0}(X)\oplus\ZZ$.
\end{theorem}
\begin{proof}
We see that
\begin{equation}
H_{0}(X)=C_{0}(X)/\Im(\boundary_{1}),
\end{equation}
but
\begin{equation}
\widetilde{H}_{0}(X)=\ker(\varepsilon)/\Im(\boundary_{1}).
\end{equation}
Observe that
\begin{equation}
C_{0}(X)/\ker(\varepsilon)\iso\ZZ,
\end{equation}
we find
\begin{equation}
H_{0}(X)/\widetilde{H}_{0}(X)\iso\ZZ,
\end{equation}
or as a short-exact sequence:
\begin{equation}
0\to\widetilde{H}_{0}(X)\to H_{0}(X)\xrightarrow{\varepsilon}\ZZ\to0.
\end{equation}
In general, when we have a short exact sequence of Abelian groups
\begin{equation}
0\to A\to B\to C\to0,
\end{equation}
if $C$ is free, then $B\iso A\oplus C$. Then $H_{0}(X)\iso\widetilde{H}_{0}(X)\oplus\ZZ$,
as desired.
\end{proof}

\begin{node}
Suppose $A\subset X$ is a nonempty subset $A\neq\emptyset$. Consider
\begin{equation}
0\to C_{*}(A)\to C_{*}(X)\to C_{*}(X,A)\to0.
\end{equation}
We can augment this short exact sequence with ``an extra row at the bottom''
(since each $C_{*}(-)$ is a long exact sequence), appending:
\begin{equation}
0\to\ZZ\to\ZZ\to0\to0.
\end{equation}
This gives us another short exact sequence
\begin{equation}
0\to\widetilde{C}_{*}(A)\to\widetilde{C}_{*}(X)\to\widetilde{C}_{*}(X,A)\to0.
\end{equation}
Now, we obtain a long-exact sequence, which we can write as an exact
triangle
\begin{equation}
\vcenter{\xymatrix{\widetilde{H}_{*}(A)\ar[rr]& &\widetilde{H}_{*}(X)\ar[dl]\\
&\ar[ul]H_{*}(X,A)&}}
\end{equation}
In particular, if $A$ is a point $A=\{x_{0}\}$, then
$\widetilde{H}_{*}(A)=0$ and we get $\widetilde{H}_{*}(X)\iso H_{*}(X,\{x_{0}\})$
relating the reduced homology to the homology of a ``pointed space''
(i.e., a space with a specified base-point).
\end{node}

\begin{example}
\begin{enumerate}
\item If $X$ is contractible, then $\widetilde{H}_{*}(X)=0$
\item $\displaystyle\widetilde{H}_{k}(\sphere{n})=\begin{cases}\ZZ
  &\mbox{if }k=n\\
  0&\mbox{otherwise}
\end{cases}$
\end{enumerate}
\end{example}

\begin{example}
Looking at $\disk{n}$ and $\boundary\disk{n}=\sphere{n-1}$.
We see that
\begin{equation}
H_{*}(\disk{n},\boundary\disk{n})=H_{*}(\disk{n},\sphere{n-1}).
\end{equation}
We have the exact triangle
\begin{equation}
\vcenter{\xymatrix{\widetilde{H}_{*}(\sphere{n-1})\ar[rr]& &\widetilde{H}_{*}(\disk{n})=0\ar[dl]\\
&\ar[ul]_{[-1]}H_{*}(\disk{n},\sphere{n-1})&}}
\end{equation}
We see that $\widetilde{H}_{*}(\disk{n})=0$ since the disk is
contractible. Then
\begin{equation}
H_{k}(\disk{n},\sphere{n-1})\iso\begin{cases}\ZZ & \mbox{if }k=n\\
0 &\mbox{otherwise}.
\end{cases}
\end{equation}
We could try computing $H_{*}(\disk{n},\sphere{n-1})$ using the
long-exact sequence without reduced homologies, but it is longer and
more tedious (and more prone to error).
\end{example}

\begin{node}[Mayer--Vietoris Sequence]
If $A\subset X$ and $B\subset X$ form a Mayer--Vietoris pair, and the
short exact sequence
\begin{equation}
0\to C_{*}(A\cap B)\to C_{*}(A)\oplus C_{*}(B)\to C_{*}(A)+C_{*}(B)\to0.
\end{equation}
We can augment this short exact sequence, adding a row to the bottom:
\begin{equation}
0\to\ZZ\to\ZZ\oplus\ZZ\to\ZZ\to0,
\end{equation}
we can then obtain an exact sequence of the augmented chain complexes
\begin{equation}
0\to\widetilde{C}_{*}(A\cap B)\to\widetilde{C}_{*}(A)\oplus\widetilde{C}_{*}(B)\to\widetilde{C}_{*}(A)+\widetilde{C}_{*}(B)\to0.
\end{equation}
\end{node}

\begin{theorem}
If $(A,B)$ is a Mayer--Vietoris Pair, then we have the long exact sequence
\begin{equation}
\vcenter{\xymatrix{\widetilde{H}_{*}(A\cap B)\ar[r]& \ar[dl]\widetilde{H}_{*}(A)\oplus\widetilde{H}_{*}(B)\\
\ar[u]\widetilde{H}_{*}(X)&}}
\end{equation}
\end{theorem}

\begin{theorem}[Brouwer Fixed-Point]
Any continuous map $f\colon\disk{n}\to\disk{n}$ has a fixed point.
\end{theorem}

\begin{lemma}
There is no retraction $r\colon\disk{n}\to\sphere{n-1}$ (where $r|_{\sphere{n-1}}=\id_{\sphere{n-1}}$).
\end{lemma}

\begin{proof}[Proof (of lemma)]
If $i\colon\sphere{n-1}\into\disk{n}$ is the inclusion mapping, then
$r\circ i=\id_{\sphere{n-1}}$.
Then $r_{*}\circ i_{*}=\id\colon H_{*}(\sphere{n-1})\to H_{*}(\sphere{n-1})$,
and this means $i_{*}\colon\colon H_{*}(\sphere{n-1})\to H_{*}(\disk{n})$
must be injective. But that is not possible since $H_{n-1}(\sphere{n-1})\iso\ZZ$
but $H_{n-1}(\disk{n})=0$. Hence the result.
\end{proof}

\begin{proof}[Proof (Brouwer)]
Assume $f\colon\disk{n}\to\disk{n}$ has no fixed point.
Let $x$ be a point. It's not a fixed-point of $f$. So consider $f(x)\in\disk{n}$.
We form the ray $\overrightarrow{xf(x)}$. This ray will intersect the
boundary of the disk at $g(x)$. We form a function
$g\colon\disk{n}\to\sphere{n-1}$ in this manner, with $g(x)=x$ if
$x\in\sphere{n-1}$. But this means $g$ is a retraction. But that's not
possible by our lemma. Hence our contradiction.
\end{proof}