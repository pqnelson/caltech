%%
%% fall-lecture20.tex
%% 
%% Made by Alex Nelson <pqnelson@gmail.com>
%% Login   <alex@lisp>
%% 
%% Started on  2025-11-13T08:46:34-0800
%% Last update 2025-11-13T08:46:34-0800
%% 

\lecture[Cohomology]{}

We start ``Part II'' of the class. Today and next time will be
homological algebra.

\begin{node}[Functoriality of hom]
Let $A$ and $B$ be Abelian groups. Then
\begin{equation*}
\hom(A,B)=\{\phi\colon A\to B\mid\phi\mbox{ is a group morphism from $A$ to $B$}\}
\end{equation*}
Moreover, $\hom(A,B)$ is an Abelian group: we can add morphisms, the
identity element is the zero morphism (which just sends everything to
the zero element).

If we fix $B$, then $\hom(-,B)$ is a contravariant functor. Observe
for any morphism $f\colon A\to A'$ and morphism $g\colon A'\to B$, we
can lift $g$ to a unique morphism $A\to B$ such that the following diagram
commutes
\begin{equation}
\vcenter{\xymatrix{A\ar[dd]_{f}\ar@{..>}[dr] & \\
 & B\\
A'\ar[ur]^{g}}}
\end{equation}
This is how we define $f^{*}=\hom(f,B)\colon\hom(A',B)\to\hom(A,B)$
and treat $\hom(-,B)$ as a functor.

If we fix $A$, then $\hom(A,-)$ is a covariant functor and defines
$f_{*}=\hom(A,f)\colon\hom(A,B)\to\hom(A,B')$ such that the following
diagram commutes
\begin{equation}
\vcenter{\xymatrix{ & B\ar[dd]^{f}\\
    A\ar[ur]\ar@{..>}[dr] & \\
    & B'}}
\end{equation}
\end{node}

\begin{example}
$\hom(\ZZ,B)\iso B$ because a morphism $f\colon\ZZ\to B$ is determined
  by $f(1)$.
\end{example}

\begin{example}
$\hom(\ZZ/m\ZZ,B)\iso mB\iso\{b\in B\mid mb=0\}$ is just the
  $m$-Torsion of $B$.
\end{example}

\begin{definition}
If we fix an Abelian group $G$, then $\hom(A,G)$ is usually called the
\define{Dual Group} of $A$.
\end{definition}

\begin{node}[Properties of dual group]
\begin{enumerate}
\item\textsc{Left exactness}: If $A\to B\to C\to 0$ is exact, then
  $\hom(A,G)\gets\hom(B,G)\gets\hom(C,G)\gets0$ is exact.
\item\textsc{Direct sum}: If $0\to A\to B\to C\to 0$ is split exact,
then $0\gets\hom(A,G)\gets\hom(B,G)\gets\hom(C,G)\gets0$ is split exact.
\end{enumerate}
\end{node}

\begin{example}
Consider the short exact split sequence
\begin{equation}
0\to\ZZ\xrightarrow{m}\ZZ\to\ZZ/m\ZZ\to 0
\end{equation}
where $\ZZ\xrightarrow{m}\ZZ$ is defined by $x\mapsto mx$. Let $G=\ZZ/m\ZZ$.
Then we apply $\hom(-,G)$ to the short exact split sequence to obtain
\begin{equation}
0\gets\ZZ/m\ZZ\xleftarrow{m^{*}}\ZZ/m\ZZ\gets\ZZ/m\ZZ\gets0.
\end{equation}
Is this exact? Well, it fails at the last step $m^{*}\colon\ZZ/m\ZZ\to\ZZ/m\ZZ$
is not surjective, so there is no hope for
$0\gets\ZZ/m\ZZ\xleftarrow{m^{*}}\ZZ/m\ZZ$ being exact.
\end{example}

\begin{remark}
However, there is a functor $\Ext(-,G)$ which (in some sense) measures
the failure of exactness.
\end{remark}

\begin{definition}
Let $A$ be an Abelian group.
We define a \define{Free Resolution} of $A$ to be an exact sequence
\begin{equation}
\dots\to F_{2}\to F_{1}\to F_{0}\to A\to0,
\end{equation}
such that $F_{i}$ is a free Abelian group for every $i\in\NN_{0}$.
\end{definition}

\begin{fact}
Every Abelian group $A$ has a length 2 free resolution
\begin{equation}
0\to F_{1}\to F_{0}\to A\to 0.
\end{equation}
Take the generators $S$ of $A$. Then let $F_{0}$ be the free Abelian
group generated by $S$. But then for $F_{0}\to A\to 0$ to be exact, we
need $F_{1}$ to be freely generated by the relations of $A$ and the
image of $F_{1}\to F_{0}$ be these relations. But then there's nothing
more to do, since this map $F_{1}\to F_{0}$ is injective.
\end{fact}

\begin{node}
Now, consider the dual sequence (obtained from the free resolution of
$A$) with $A$ removed. The general case looks like
\begin{equation}
\dots\xleftarrow{\delta_{2}}\hom(F_{2},G)\xleftarrow{\delta_{1}}\hom(F_{1},G)\xleftarrow{\delta_{0}}\hom(F_{0},G),
\end{equation}
and the $\delta_{i}$ maps satisfy
\begin{equation}
\delta_{i+1}\circ\delta_{i}=0.
\end{equation}
(The professor asserts this follows from left exactness, just think
about the induced maps.) Consider the groups
\begin{equation}
H^{i}(F;G) :=\ker(\delta_{i})/\Im(\delta_{i-1}).
\end{equation}
\end{node}

\begin{lemma}
Given two free resolutions $F$ and $F'$ of groups $A$ and $A'$, then
every homomorphism $\alpha\colon A\to A'``$ can be extended to a chain
map $F\to F'$,
\begin{equation}
\vcenter{\xymatrix{\dots\ar[r]&F_{2}\ar@{..>}[d]\ar[r]&F_{1}\ar@{..>}[d]\ar[r]&F_{0}\ar@{..>}[d]\ar[r]&A\ar[r]\ar[d]^{\alpha}&0\\
\dots\ar[r]&F'_{2}\ar[r] & F'_{1}\ar[r] & F'_{0}\ar[r] & A'\ar[r] & 0}}
\end{equation}
And moreover any two such chain maps extending $\alpha$ are chain homotopic.
\end{lemma}

\begin{proof}[Proof sketch]
For any free Abelian group $F$ and Abelian groups $A$ and $A'$, if we
have $f\colon A\to A'$ and $\phi\colon F\to A'$, then we can always
lift $\phi$ to $\phi'\colon F\to A$ such that the following diagram
commutes (just pick generators):
\begin{equation}
\vcenter{\xymatrix{ & A\ar[d]^{\phi}\\
F\ar@{..>}[ur]\ar[r]^{f} & A'}}
\end{equation}
This proves the first part of the Lemma.
\end{proof}

\begin{lemma}
For any two free resolutions $F$, $F'$ of $A$, there are canonical
isomorphisms $H^{n}(F;G)\iso H^{n}(F';G)$.
\end{lemma}

\begin{proof}
Since free resolutions of Abelian groups are length 2, there are just
2 guys to look at:
\begin{equation}
\hom(F_{1},G)\xleftarrow{\delta}\hom(F_{0},G).
\end{equation}
Then
\begin{equation}
H^{1}=\coker(\delta),\quad\mbox{and}\quad H^{0}=\ker(\delta).
\end{equation}
Let us name the morphism in the free resolution
\begin{equation}
0\to F_{1}\xrightarrow{\partial}F_{0}\to A\to 0,
\end{equation}
then we see $H^{1}=\hom(F_{1},G)/\Im(\delta)$.

If some elements of $F_{0}$ are in the image of $\partial$, then we
realize
\begin{equation}
H^{0}=\ker(\delta)=\hom(A,G),
\end{equation}
since it's just imposing the relations on the generators of $A$. So
the only interesting thing is $H^{1}$, which leads us to the next definition.
\end{proof}

\begin{definition}
We define $\Ext(A,G) := H^{1}$.
\end{definition}

\begin{proposition}
Given a short exact sequence $0\to A\to B\to C\to 0$,
there is a dual long exact sequence
\begin{equation}
\vcenter{\xymatrix{
0 & \ar[l]\Ext(A,G) & \ar[l]\Ext(B,G) & \ar[l]\Ext(C,G) & \\
&\ar[urr]\hom(A,G) & \ar[l]\hom(B,G) & \ar[l]\hom(C,G) & \ar[l]0}}
\end{equation}
which is why we say $\Ext$ measures the failure of exactness.
\end{proposition}

\begin{proof}[Proof sketch]
If we look at the short exact sequence, then we can consider the free
resolution for each of the Abelian groups, giving us
\begin{equation}
\vcenter{\xymatrix{
        &      0\ar[d] &    0\ar[d] &       0\ar[d] &\\
        & F_{1}'\ar[d] & F_{1}\ar[d] & F''_{1}\ar[d] &\\
        & F_{0}'\ar[d] & F_{0}\ar[d] & F''_{0}\ar[d] &\\
0\ar[r] & A\ar[r]\ar[d] & B\ar[r]\ar[d] & C\ar[r]\ar[d] & 0\\
        &      0       &     0      &      0        & }}
\end{equation}
But then we can lift and get
\begin{equation}
\vcenter{\xymatrix{
        &      0\ar[d] &    0\ar[d] &       0\ar[d] &\\
0\ar[r] & F_{1}'\ar[d]\ar[r] & F_{1}\ar[d]\ar[r] & F''_{1}\ar[d]\ar[r] & 0\\
0\ar[r] & F_{0}'\ar[d]\ar[r] & F_{0}\ar[d]\ar[r] & F''_{0}\ar[d]\ar[r] & 0\\
0\ar[r] & A\ar[r]\ar[d] & B\ar[r]\ar[d] & C\ar[r]\ar[d] & 0\\
        &      0       &     0      &      0        & }}
\end{equation}
where the $0\to F'_{i}\to F_{i}\to F''_{i}\to 0$ (top two rows) are
split exact short sequences. Then we dualize using $\hom(-,G)$, and
this reverses the direction of the arrows, and the top 2 rows are
still split exact sequences. The result follows.
\end{proof}

\begin{proposition}
\begin{enumerate}
\item $\Ext(A\oplus A',G)\iso\Ext(A,G)\oplus\Ext(A',G)$
\item $\Ext(A,G)=0$ if $A$ is free
\item $\Ext(\ZZ/m\ZZ, G)\iso G/mG$
\end{enumerate}
\end{proposition}