%%
%% fall-lecture18.tex
%% 
%% Made by Alex Nelson <pqnelson@gmail.com>
%% Login   <alex@lisp>
%% 
%% Started on  2025-11-08T08:22:43-0800
%% Last update 2025-11-08T08:22:43-0800
%% 

\lecture{}

\begin{corollary}\label{cor:math-151a:fall-lec18}
Let $M$, $N$ be $n$-manifolds. If $M$ is compact and $N$ is connected,
then an injective continuous map $f\colon M\to N$ must be a homeomorphism.
\end{corollary}

\begin{proof}
\textsc{Step 1:} $f\colon M\to f(M)$ is a homeomorphism. We know it's
a bijection and continuous, so we need to prove $g\colon f(M)\to f$
(where $g=f^{-1}$) is continuous. We Let $B\subset M$ be a closed
set. We want to show $g^{-1}(B)$ is closed. But any closed subset of a
compact space must be compact. So $g^{-1}(B)=f(B)$ is also compact,
hence closed. This establishes $g$ is continuous, which concludes the
first step of the proof.

\textsc{Step 2.} By invariance of domain, $f(M)$ is open in
$N$. (Since $N$ is locally like $\RR^{n}$, we can use invariance of
domain.) Since $M$ is compact, $f(M)$ is also compact, hence $f(M)$ is
closed in $N$. So either $f(M)=\emptyset$ or $f(M)=N$ (since $f(M)$ is
open and closed in a connected space $N$). But $M\neq\emptyset$, which
implies $f(M)=N$.
\end{proof}

\begin{theorem}[Hopf]
$\RR$ and $\CC$ are the only finite-dimensional commutative division
algebras over $\RR$.
\end{theorem}

Recall, a division algebra over a field $\FF$ is a vector space $V$
equipped with a bilinear map $\mu\colon V\times V\to V$ called
``multiplication'' such that for all nonzero $x\in V\setminus\{0\}$ we
have $\mu(x,-)$ and $\mu(-,x)$ are bijections. We will write $\mu(x,y)=xy$.

\begin{proof}
Let $V\iso\RR^{n}$ be a finite-dimensional commutative division
algebra over $\RR$. Given $x\in\sphere{n-1}\subset\RR^{n}$, define
\begin{equation}
f(x) := \frac{\mu(x,x)}{\|\mu(x,x)\|}=\frac{x^{2}}{\|x^{2}\|}.
\end{equation}
This is well-defined, since $x\neq0$ so $x^{2}\neq0$. Then
$f\colon\sphere{n-1}\to\sphere{n-1}$ and $f(x)=f(-x)$. So $f$ induces
a map
\begin{equation}
\bar{f}\colon\RP^{n-1}\to\sphere{n-1}.
\end{equation}
This is also well-defined.

We claim $\bar{f}$ is injective. If $x,y\in\sphere{n-1}$ are such that
$f(x)=f(y)$, then $x^{2}/\|x^{2}\|=y^{2}/\|y^{2}\|$. We can rewrite
this as
\begin{equation}
x^{2}=\alpha^{2}y^{2},
\end{equation}
where $\alpha\in\RR$ is nonzero, and then we can rearrange this to be
(since $V$ is commutative)
\begin{equation}
(x+\sqrt{\alpha}y)(x-\sqrt{\alpha}y)=0.
\end{equation}
Then one of these factors must be zero (since division algebras are
integral domains). So we must have
\begin{equation}
x=\pm\sqrt{\alpha}y.
\end{equation}
But since $x,y\in\sphere{n-1}$, they are all unit vectors, and so this implies
\begin{equation}
x=\pm y.
\end{equation}
This implies $\bar{f}$ is injective.

Then $\bar{f}$ is a homeomorphism by Corollary~\ref{cor:math-151a:fall-lec18}.
But this is not true if $n>2$ --- just look at
$H_{1}(\RP^{n-1};\ZZ/2\ZZ)$ and $H_{1}(\sphere{n-1};\ZZ/2\ZZ)$, they
are different for $n>2$.

So $n\leq2$. For $n=1$, we see $V=\RR$. For $n=2$, we use need to show
there is a unit $1\in V$ such that $\mu(1,x)=\mu(x,1)=x$ for all $x\in V$
and there is a $j\in V$ such that $\mu(j,j)=-1$. This is elementary
algebra, and establishes $V\iso\CC$.
\end{proof}

\begin{definition}
A map $f\colon\sphere{n}\to\sphere{m}$ is \define{Even} if $f(-x)=f(x)$,
and \define{Odd} if $f(-x)=-f(x)$.
\end{definition}

\begin{theorem}
An odd map $f\colon\sphere{n}\to\sphere{n}$ has odd degree.
\end{theorem}

\begin{proof}
\textsc{Case 1: $n=1$.} When $n=1$, we can check it directly. This
case is trivial.

\textsc{Case 2: $n>1$.} We have the 2-fold covering
\begin{equation}
q\colon\sphere{n}\xrightarrow{2:1}\RP^{n}.
\end{equation}
Then we can cover $f$ by $\bar{f}$ to get
\begin{equation}
\vcenter{\xymatrix{\sphere{n}\ar[r]^{f}\ar[d]^{2:1} & \sphere{n}\ar[d]^{2:1}\\
\RP^{n}\ar[r]^{\bar{f}} & \RP^{n}}}
\end{equation}
We see $f(\{x,-x\})=\{f(x),-f(x)\}$ maps antipodal points to antipodal
points. Now consider
\begin{equation}
f_{*}\colon H_{*}(\RP^{n};\ZZ_{2})\to H_{*}(\RP^{n};\ZZ_{2}).
\end{equation}
Let $\gamma$ be a path on $\sphere{n}$ from $x_{0}=\gamma(0)$ to
$-x_{0}=\gamma(1)$. Let $f(\gamma)$ be a path from $f(x_{0})$ to $-f(x_{0})$.
Then $q(\gamma)$ is a loop which generates $\pi_{1}(\RP^{n})$.
[Lifting criteria: if $q(\gamma)$ is null-homotopic, then it would
  lift to a loop in $\sphere{n}$.] But $\pi_{1}(\RP^{n})\iso H_{1}(\RP^{n})$
always, and $H_{1}(\RP^{n})\iso H_{1}(\RP^{n};\ZZ_{2})$ if $n>1$.

Similarly, $[q(f(\gamma))]$ generates $H_{1}(\RP^{n};\ZZ_{2})$.

Hence $f_{*}\colon H_{1}(\RP^{n};\ZZ_{2})\to H_{1}(\RP^{n};\ZZ_{2})$
is an isomorphism.

Now we want to show the induced map is an isomorphism on $H_{j}$ for
$1\leq j\leq n$.

\textsc{Transfer Homomorphism:} Let $q\colon\widetilde{X}\to X$ be a
2-fold covering. The transfer morphism,
\begin{equation}
\tau\colon C_{*}(X;\ZZ_{2})\to C_{*}(\widetilde{X};\ZZ_{2}),
\end{equation}
where if we have a singular complex $\sigma$ for $X$,
$\tau(\sigma)=\widetilde{\sigma}_{1}+\widetilde{\sigma}_{2}$ sends it
to singular complexes for $\widetilde{X}$ are the two lifts of
$\sigma$.
\begin{equation}
\vcenter{\xymatrix{
&\widetilde{X}\ar[d]^{q}\\
\Delta\ar@{..>}[ur]\ar[r]^{\sigma} & X}}
\end{equation}
This works for $\ZZ$ coefficients, but we only need it for $\ZZ_{2}$ coefficients.
We have a short exact sequence
\begin{equation}
0\to C_{*}(X;\ZZ_{2})\xrightarrow{\tau}C_{*}(\widetilde{X};\ZZ_{2})\xrightarrow{q_{\sharp}}C_{*}(X;\ZZ_{2})\to0,
\end{equation}
which gives us a long exact sequence
\begin{equation}
\vcenter{\xymatrix{
H_{*}(X;\ZZ_{2})\ar[rr]^{\tau} && \ar[dl]^{q_{*}}H_{*}(\widetilde{X};\ZZ_{2})\\
&\ar[ul]_{[-1]}^{\boundary_{*}}H_{*}(X;\ZZ_{2})&}}
\end{equation}
We have a commutative diagram letting $X=\RP^{n}$:
\begin{equation}
\vcenter{\xymatrix{
\dots\ar[r]& H_{i}(\RP^{n};\ZZ_{2})\ar[d]^{\bar{f}_{*}}\ar[r] & H_{i}(\sphere{n};\ZZ_{2})\ar[r]\ar[d]^{f_{*}}
& H_{i}(\RP^{n};\ZZ_{2})\ar[d]^{\bar{f}_{*}}\ar[r] & H_{i-1}(\RP^{n};\ZZ_{2})\ar[d]^{\bar{f}_{*}}\ar[r] & H_{i-1}(\sphere{n};\ZZ_{2})\ar[r]\ar[d]^{f_{*}}& \dots\\
\dots\ar[r] & H_{i}(\RP^{n};\ZZ_{2})\ar[r] & H_{i}(\RP^{n};\ZZ_{2})\ar[r] & H_{i}(\RP^{n};\ZZ_{2})\ar[r] & H_{i-1}(\RP^{n};\ZZ_{2})\ar[r] & H_{i-1}(\RP^{n};\ZZ_{2})\ar[r]&\dots}}
\end{equation}
For $1<i\leq n-1$, the $H_{i}(\sphere{n};\ZZ_{2})=0=H_{i-1}(\sphere{n};\ZZ_{2})$,
which gives isomorphisms. Then $\bar{f}_{n-1}$ is an isomorphism. But
then this implies $\bar{f}_{n}$ is an isomorphism. So then $f_{n}$ is
an isomorphism, because
\begin{equation}
\vcenter{\xymatrix{0\ar[r] &
    H_{n}(\RP^{n};\ZZ_{2})\ar[d]^{\bar{f}_{n}}\ar[r] & H_{n}(\sphere{n};\ZZ_{2})\ar[d]^{f_{n}}\\
0\ar[r] & H_{n}(\RP^{n};\ZZ_{2})\ar[r] & H_{n}(\RP^{n};\ZZ_{2})}}
\end{equation}
Then $f_{n}\colon H_{n}(\RP^{n};\ZZ_{2})\to H_{n}(\RP^{n};\ZZ_{2})$ is
an isomorphism. Hence the result.
\end{proof}