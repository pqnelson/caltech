%%
%% fall-lecture05.tex
%% 
%% Made by Alex Nelson <pqnelson@gmail.com>
%% Login   <alex@lisp>
%% 
%% Started on  2025-10-09T10:42:48-0700
%% Last update 2025-10-09T10:42:48-0700
%% 

\lecture{}

\begin{definition}
Let $X$ be a $\Delta$-complex. Let $C_{n}^{\Delta}(X)$ be the free
Abelian group freely generated by all $n$-simplices in $X$. We call
the elements of $C^{\Delta}_{n}(X)$ \define{$n$-Chains}.
\end{definition}

\begin{definition}
We define the \define{Boundary Morphism} $\boundary_{n}\colon C^{\Delta}_{n}(X)\to C^{\Delta}_{n-1}(X)$
by
\begin{equation}
\boundary_{n}[v_{0},\dots,v_{n}] = \sum^{n}_{i=0}(-1)^{i}[v_{0},v_{1},\dots\widehat{v_{i}},\dots,v_{n}]
\end{equation}
where the hat indicates we do not include the vertex in the
$(n-1)$-simplex. This is the alternating sum of faces of the $n$-simplex.
\end{definition}

\begin{example}
For any 0-simplex $\bullet v_{0}$, we see $\boundary_{0}(\bullet v_{0})=0$.
\end{example}

\begin{example}
For a 1-simplex, $\boundary_{1}(v_{0}\bullet\!\!-\!\!\bullet v_{1})=v_{1}-v_{0}$.
\end{example}

\begin{example}
For a 2-simplex $\Delta^{2}=[v_{0},v_{1},v_{2}]$, let us draw it out as:
\begin{equation*}
\includegraphics{img/img.24}
\end{equation*}
Then we have
\begin{equation}
\boundary_{2}([v_{0},v_{1},v_{2}]) = [v_{1},v_{2}]-[v_{0},v_{2}]+[v_{0},v_{1}].
\end{equation}
We can similarly think of higher simplices having a similar boundary maps.
\end{example}

\begin{definition}
We define the \define{Simplicial Chain Complex} as the pair
$(C^{\Delta}(X),\boundary)$ where
\begin{subequations}
\begin{equation}
C^{\Delta}(X)=\bigoplus_{n=0}^{\infty}C^{\Delta}_{n}(X),
\end{equation}
and
\begin{equation}
\boundary = \bigoplus^{\infty}_{n=0}\boundary_{n}.
\end{equation}
\end{subequations}
\end{definition}

\begin{proposition}[Boundary of a boundary is zero]
For any $n$, $\boundary_{n-1}\circ\boundary_{n}=0$. Equivalently, $\boundary\circ\boundary=0$.
\end{proposition}

\begin{proof}
We see
\begin{equation}
\boundary_{n}[v_{0},\dots,v_{n}] = \sum^{n}_{j=0}(-1)^{j}[v_{0},\dots,\widehat{v_{j}},\dots,v_{n}].
\end{equation}
Then applying the next boundary map
\begin{align*}
\boundary_{n-1}\boundary_{n}[v_{0},\dots,v_{n}]
&= \sum^{n}_{j=0}(-1)^{j}\left(\sum^{n-1}_{k<j}(-1)^{k}[v_{0},\dots,\widehat{v_{k}},\dots,\widehat{v_{j}},\dots,v_{n}]+\sum^{n-1}_{j<k}[v_{0},\dots,\widehat{v_{j}},\dots,\widehat{v_{k}},\dots,v_{n}]\right)\\
&=\sum_{0\leq k<j\leq n}(-1)^{j+k}[v_{0},\dots,\widehat{v_{k}},\dots,\widehat{v_{j}},\dots,v_{n}]
+\sum_{0\leq j<k\leq n}(-1)^{j+k-1}[v_{0},\dots,\widehat{v_{j}},\dots,\widehat{v_{k}},\dots,v_{n}]\\
\intertext{then changing the dummy indices in the second sum}
&=\sum_{0\leq k<j\leq n}(-1)^{j+k}[v_{0},\dots,\widehat{v_{k}},\dots,\widehat{v_{j}},\dots,v_{n}]
+\sum_{0\leq k<j\leq n}(-1)^{j+k-1}[v_{0},\dots,\widehat{v_{k}},\dots,\widehat{v_{j}},\dots,v_{n}]\\
&=0
\end{align*}
for every generator of $C^{\Delta}_{n}(X)$, so it vanishes everywhere
on $C^{\Delta}_{n}(X)$, hence the result.
\end{proof}

\begin{corollary}
  We can find $\ker(\boundary_{n-1})\supset\Im(\boundary_{n})$.
  We have a short exact sequence
  \begin{equation*}
C^{\Delta}_{n}(X)\xrightarrow{\boundary_{n}}C^{\Delta}_{n-1}(X)\xrightarrow{\boundary_{n-1}}C^{\Delta}_{n-2}(X)
  \end{equation*}
\end{corollary}

\begin{definition}
We define $Z^{\Delta}_{n}(X):=\ker(\boundary_{n})$ to be the set of
\define{$n$-Cycles} of $X$ (a.k.a., \define{Closed $n$-Chains}).
\end{definition}

\begin{definition}
We define $B^{\Delta}_{n}(X):=\Im(\boundary_{n+1})$ the \define{$n$-Boundaries}.
\end{definition}

\begin{definition}
We define the \define{$n^{\text{th}}$ Simplicial Homology Group} of
$X$ to be
\begin{equation*}
  \begin{split}
H^{\Delta}_{n}(X) &:= Z^{\Delta}_{n}(X)/B^{\Delta}_{n}(X)\\
&= \ker(\boundary_{n})/\Im(\boundary_{n}).
  \end{split}
\end{equation*}
The elements of $H^{\Delta}_{n}(X)$ are called \define{Homology Classes}.

Two cycles representing the same homology class are called \define{Homologous}.
\end{definition}

\begin{example}
The point $\bullet p$ has $C^{\Delta}_{0}(X)=\langle p\rangle$ and
$C^{\Delta}_{n}(X)=0$ for $n>0$. Then $\boundary=0$. So
$\ker(\boundary)=C^{\Delta}(X)$ and $\Im(\boundary)=0$, so
$H^{\Delta}_{0}(X)\iso C^{\Delta}_{0}(X)$, so
\begin{equation*}
  H^{\Delta}_{n}(X)\iso\begin{cases}\ZZ & \mbox{if }n=0\\
  0 & \mbox{otherwise}.
  \end{cases}
\end{equation*}
\end{example}

\begin{example}
The line segment $X = (a \bullet\!\!-\!\!\bullet b)$, then
$C^{\Delta}_{0}(X)=\ZZ a\oplus\ZZ b$ and $C^{\Delta}_{1}(X)=\ZZ\langle[a,b]\rangle$.
There is only one nontrivial boundary map
\begin{equation}
\begin{split}
\boundary_{1}\colon& C^{\Delta}_{1}(X)\to C^{\Delta}_{0}(X)\\
&[a,b]\mapsto b-a.
\end{split}
\end{equation}
Then $\ker(\boundary_{1})=0$ and $\Im(\boundary_{1})=\ZZ(b-a)$.
So
\begin{equation}
H^{\Delta}_{1}(X) = \ker(\boundary_{1})/\Im(\boundary_{2})=0
\end{equation}
and
\begin{equation}
H^{\Delta}_{0}(X) = \ker(\boundary_{0})/\Im(\boundary_{1})=\ZZ\langle a,b\rangle/\ZZ\langle b-a\rangle\iso\ZZ\langle[a]\rangle
\end{equation}
where $[a]$ is the homology equivalence class of $a$.
\end{example}

\begin{example}
Consider $\sphere{1}$. There are a lot of different $\Delta$-complexes
we could use. For example,
\begin{equation*}
\includegraphics{img/img.25}
\end{equation*}
Then $\boundary_{1}(e)=a-a=0$. Then we find
\begin{equation}
H^{\Delta}_{n}(X)\iso\begin{cases}\ZZ & \mbox{if }n=0,1\\
0 & \mbox{otherwise}
\end{cases}
\end{equation}
since $H^{\Delta}_{n}(X)\iso C^{\Delta}_{n}(X)$ because
$\boundary_{n}=0$ for all $n$.

We could try a different $\Delta$-complex for $\sphere{1}$, like:
\begin{equation*}
\includegraphics{img/img.26}
\end{equation*}
Then $\boundary_{1}\colon C^{\Delta}_{1}(X)\to C^{\Delta}_{0}(X)$ is
given by
\begin{equation}
\boundary_{1}(e_{1})=b-a,\quad\mbox{and}\quad \boundary_{1}(e_{2})=b-a.
\end{equation}
Then $\ker(\boundary_{1})=\ZZ\langle e_{1}-e_{2}\rangle$ and 
$\Im(\boundary_{1})\ZZ\langle b-a\rangle$. We see
\begin{equation}
H^{\Delta}_{1}(X)=\ker(\boundary_{1})/\Im(\boundary_{2})\iso\ker(\boundary_{1})\iso\ZZ,
\end{equation}
and
\begin{equation}
H^{\Delta}_{0}(X)=\ker(\boundary_{0})/\Im(\boundary_{1})=\ZZ\langle a,b\rangle/\ZZ\langle b-a\rangle\iso\ZZ.
\end{equation}
So using this other $\Delta$-complex structure for $\sphere{1}$ we
obtain the same Homology groups as before.
\end{example}

\begin{example}
For $\sphere{2}$ with the $\Delta$-complex obtained by gluing two
triangles $\sigma_{+}$ and $\sigma_{-}$ together along their boundary, as doodled:
\begin{equation*}
\includegraphics{img/img.27}
\end{equation*}
We then find
\begin{subequations}
\begin{align}
C^{\Delta}_{0}(X) &= \ZZ\langle a_{0},a_{1},a_{2}\rangle\\
C^{\Delta}_{1}(X) &= \ZZ\langle e_{0},e_{1},e_{2}\rangle\\
C^{\Delta}_{2}(X) &= \ZZ\langle \sigma_{+},\sigma_{-}\rangle
\end{align}
\end{subequations}
Then we write down the boundary maps,
\begin{equation}
\begin{split}
\boundary_{2}\colon&C^{\Delta}_{2}(X)\to C^{\Delta}_{1}(X)\\
&\sigma_{\pm}\mapsto e_{0}+e_{1}+e_{2}
\end{split}
\end{equation}
and
\begin{subequations}
\begin{align}
  \boundary_{1}\colon&C^{\Delta}_{1}(X)\to C^{\Delta}_{0}(X)\\
  &e_{0}\mapsto a_{1}-a_{0}\\
  &e_{1}\mapsto a_{2}-a_{1}\\
  &e_{2}\mapsto a_{0}-a_{2}.
\end{align}
\end{subequations}
Then $\ker(\boundary_{2})=\ZZ\langle\sigma_{+}-\sigma_{-}\rangle$ and
$\Im(\boundary_{2})=\ZZ\langle e_{0}+e_{1}+e_{2}\rangle$,
$\ker(\boundary_{1})=\ZZ\langle e_{0}+e_{1}+e_{2}\rangle$ and
$\Im(\boundary_{1})=\ZZ\langle a_{1}-a_{0},a_{2}-a_{1}\rangle$.
We find
\begin{equation}
\left.\begin{array}{c}
H^{\Delta}_{2}(X)=\ker(\boundary_{2})/\Im(\boundary_{3})\iso\ZZ\\
H^{\Delta}_{1}(X)=\ker(\boundary_{1})/\Im(\boundary_{2})=0\\
H^{\Delta}_{0}(X)=\ker(\boundary_{0})/\Im(\boundary_{1})\iso\ZZ
\end{array}\right\}\implies
H^{\Delta}_{n}(\sphere{2})\iso\begin{cases}\ZZ &\mbox{if }n=0,2\\
0 & \mbox{otherwise}
\end{cases}
\end{equation}
More generally, for any $k\in\NN$,
\begin{equation}
H^{\Delta}_{n}(\sphere{k})\iso\begin{cases}\ZZ & \mbox{if }n=0,k\\
0 & \mbox{otherwise}
\end{cases}
\end{equation}
\end{example}