%%
%% winter-lecture02.tex
%% 
%% Made by Alex Nelson <pqnelson@gmail.com>
%% Login   <alex@lisp>
%% 
%% Started on  2026-01-08T11:12:01-0800
%% Last update 2026-01-08T11:12:01-0800
%% 

\lecture{}

\subsection{Overview}

\begin{node}
Remember $\pi_{n}(X,x_{0})$ is the set of homotopy equivalence classes
of continuous maps which preserve the specified point,
\begin{equation}
\pi_{n}(X,x_{0})=[(\sphere{n},p\in\sphere{n}),(X,x_{0})].
\end{equation}
We proved $\pi_{1}$ is the fundamental group, and $\pi_{n}$ is an
Abelian group for $n\geq2$. We hope to prove Hurewicz theorem,
Whitehead's theorem, discuss Eilienberg--Mac~Lane spaces $K(G,n)$,
Excision, and long exact sequences for fibrations/fiber bundles.
\end{node}

\begin{node}
If $F\into E\onto B$ is a fiber bundle, if $B$ is path-connected, then
\begin{equation}
\vcenter{\xymatrix{ & \dots\ar[r] & \pi_{n+1}(B,b_{0})\ar[dll]\\
\pi_{n}(F,x_{0})\ar[r] & \pi_{n}(E,x)\ar[r] & \pi_{n}(B,b_{0})\ar[dll]\\
\pi_{n-1}(F,x_{0})\ar[r] & \dots &  }}
\end{equation}
is a long exact sequence.
\end{node}

\begin{example}[Hopf fibration]
The Hopf fibration is the most famous example of a fiber bundle
$\sphere{1}\into\sphere{3}\onto\sphere{2}$ where $E=\sphere{3}$ is the
unit sphere in $\RR^{4}=\CC^{2}$ given by
\begin{equation}
E = \{(z,w)\in\CC^{2}\mid |z|^{2}+|w|^{2}=1\},
\end{equation}
and $B=\sphere{2}=\CP^{1}$ is given by
\begin{equation}
B=\{[z:w]\in\CC^{2}\}/[\lambda z:\lambda w]\sim[z:w].
\end{equation}
Then $J\colon\sphere{3}\to\sphere{2}$ sends $(z,w)\mapsto[z:w]$. The
fiber is given by $(\E^{\I\theta}z,\E^{\I\theta}w)$ for $\theta\in\sphere{1}$.
(The $J$ map is the generator for $\pi_{3}(\sphere{2})\iso\ZZ$.)
\end{example}

\begin{remark}[History]
Eduard \v{C}ech's ``H\"{o}herdimensionale Homotopiegruppen'' (\textit{Verh.\ Intern.\ Mathematikerkongress Z\"{u}rich}, 1932, O.\ F\"{u}ssli (1932) pp. 203)
first introduced homotopy groups, which was later separately
introduced in Witold Hurewicz's ``Beitr\"{a}ge zur Topologie der Deformationen I-II''
(\textit{Proc.~Ned.~Akad.~Weten.~Ser.~A} \textbf{38} (1935) pp.\ 112--119; 521--528)
and ``Beitr\"{a}ge zur Topologie der Deformationen III-IV''
(\textit{Proc.~Ned.~Akad.~Weten.~Ser.~A} \textbf{39} (1936) pp.\ 117--126; 215--224).
\end{remark}

\subsection{Review homotopy groups}

\begin{remark}
This is my own review of homotopy groups. There are two definitions
offered in the literature: one using pointed spheres, and another
using ``spheroids'' (cubes whose boundary are mapped to a specified point).
Albert Schwarz's book~\cite{schwarz1993quantum} is a great reference
for the spheroid definition, and also Fuchs and Fomenko~\cite[\S8]{fuchs2016homotopical}.
\end{remark}

\begin{definition}
Let $(X,x_{0})$ be a pointed topological space, let $n\in\NN$. An
$n$-\define{Spheroid} is a continuous map $f\colon(I^{n},\boundary I^{n})\to(X,x_{0})$
such that the boundary is mapped to the specified point $f(\boundary I^{n})=\{x_{0}\}$.

Equivalently, an $n$-spheroid is a continuous map
$g\colon(\sphere{n},s)\to(X,x_{0})$ such that it maps the marked point
to the marked point $g(s)=x_{0}$.

These two are related by the continuous map $(I^{n},\boundary I^{n})\to(\sphere{n},s)$
by taking the canonical continuous map to pointed quotient space $(I^{n}/\boundary I^{n},*)$
and observing this is canonically isomorphic to $(\sphere{n},s)$.
\end{definition}

\begin{remark}
We have $n=0$ spheroids by taking $I^{0}$ to be a point and $\boundary I^{0}=\emptyset$.
\end{remark}

\begin{definition}
Let $(X,x_{0})$ be a pointed topological space. Let
$f,g\colon(I^{n},\boundary I^{n})\to(X,x_{0})$ be two
$n$-spheroids. Then a \define{Homotopy} from $f$ to $g$ is a mapping
$f_{t}\colon[0,1]\times(I^{n},\boundary I^{n})\to(X,x_{0})$ such that
\begin{enumerate}
\item $f_{0}=f$ and
\item $f_{1}=g$ and
\item consists of spheroids: for all $t\in[0,1]$ we have
$f_{t}(\boundary I^{n})=\{x_{0}\}$
\end{enumerate}
It's trivial to check that this forms an equivalence relation among
$n$-spheroids. 
\end{definition}

\begin{notation}
We will write $[f]$ for homotopy equivalence classes with $f\in[f]$ as
a representative.
\end{notation}

\begin{definition}
Let $(X,x_{0})$ be a pointed topological space. We define the set
$\pi_{n}(X,x_{0})$ to be the set of all homotopy equivalence classes
of $n$-spheroids to $(X,x_{0})$.
\end{definition}

\begin{definition}
Let $(X,x_{0})$ be a pointed topological space.
We have a \define{Composition} of $n$-spheroids
$f,g\colon(I^{n},\boundary I^{n})\to(X,x_{0})$ giving us the spheroid
$g*f$ defined by
\begin{equation}
(g*f)(t_{1},\dots,t_{n})=\begin{cases}
  f(2t_{1},t_{2},\dots,t_{n}) & \mbox{if }0\leq t_{1}\leq1/2\\
  g(2t_{1}-1,t_{2},\dots,t_{n}) & \mbox{if }1/2\leq t_{1}\leq1.
  \end{cases}
\end{equation}
Observe that $f(1,t_{2},\dots,t_{n})=x_{0}=g(0,t_{2},\dots,t_{n})$.

This also descends to homotopy equivalence classes $[g]*[f]=[g*f]$.
\end{definition}

\begin{lemma}
The composition of spheroid homotopy equivalence classes is
associative $([h]*[g])*[f]=[h]*([g]*[f])$. However, it is not
associative on spheroids $(h*g)*f\neq h*(g*f)$.
\end{lemma}

\begin{lemma}
The constant spheroid $(I^{n},\boundary I^{n})\to(X,x_{0})$ (which just
maps everything to $x_{0}$) is the identity element for the
composition of homotopy equivalence classes of $n$-spheroids
\end{lemma}

\begin{lemma}
Let $f\colon(I^{n},\boundary I^{n})\to(X,x_{0})$ be an $n$-spheroid.
Then $[f]^{-1}$ given by the spheroid
$g(t_{1},t_{2},\dots,t_{n})=f(1-t_{1},t_{2},\dots,t_{n})$ is the
inverse of $[f]$ with respect to composing spheroids.
\end{lemma}

\begin{theorem}
The set $\pi_{n}(X,x_{0})$ equipped with the composition of (homotopy
equivalence classes of) spheroids forms a group for all $n\geq1$.
\end{theorem}

This follows from the preceding three lemmas.

\begin{proposition}
For $n\geq2$, $\pi_{n}(X,x_{0})$ is an Abelian group.
\end{proposition}

\begin{remark}
We adopt the standard convention of writing $[f]+[g]$ instead of
$[f]*[g]$ for applying the binary operator of $\pi_{n}(X,x_{0})$ when
$n\geq2$.
\end{remark}

\subsection{Independence of the base point for homotopy groups}

\begin{node}
Let $(X,x_{0})$ be a pointed topological space.
Let $x_{1}\in X$ and $\gamma\colon[0,1]\to X$ be a path from
$x_{0}=\gamma(0)$ to $x_{1}=\gamma(1)$. Let $f\colon(\sphere{n},s)\to(X,x_{0})$
be an $n$-spheroid.

Then we have another $n$-spheroid $\gamma f\colon(\sphere{n},s)\to(X,x_{1})$ defined by
\begin{equation}
(\gamma f)(r,\Omega) = \begin{cases} f(2r,\Omega) & 0\leq r\leq1/2\\
  \gamma(2r-1) & 1/2\leq r\leq1
  \end{cases}
\end{equation}
where $\Omega$ is all the angular components o
\end{node}