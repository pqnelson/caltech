%%
%% fall-lecture09.tex
%% 
%% Made by Alex Nelson <pqnelson@gmail.com>
%% Login   <alex@lisp>
%% 
%% Started on  2025-10-19T10:17:37-0700
%% Last update 2025-10-19T10:17:37-0700
%% 

\lecture{}

\begin{definition}
Some people prefer writing long exact sequences obtained from short
exact sequences as \define{Exact Triangles}:
\begin{equation}
\xymatrix{H_{*}(A)\ar[rr]^{[0]} & & H_{*}(X)\ar[dl]_{[0]}\\
& H_{*}(X,A)\ar[ul]^{[-1]} &}
\end{equation}
Eisenbud does this in his book on Commutative Algebra. Well, we suffer
what we must, I suppose.
\end{definition}

\begin{node}[Notation]
We sometimes write
\begin{equation}
f\colon(X,A)\to(Y,B)
\end{equation}
which is a continuous map $f\colon X\to Y$, where $X$ and $Y$ are
topological spaces, such that $f(A)\subset B$.

Then $f$ induces a chain map
\begin{equation}
f_{\sharp}\colon C_{*}(X,A)\to C_{*}(Y,B),
\end{equation}
which then induces the map
\begin{equation}
f_{*}\colon H_{*}(X,A)\to H_{*}(Y,B).
\end{equation}
Moreover, it commutes with the long exact sequence
\begin{equation}
  \vcenter{
\xymatrix{\dots\ar[r] & H_{n}(A)\ar[d]_{f_{*}}\ar[r] & H_{n}(X)\ar[d]_{f_{*}}\ar[r] & H_{n}(X,A)\ar[d]_{f_{*}}\ar[r]&\dots\\
\dots\ar[r] & H_{n}(B)\ar[r] & H_{n}(X)\ar[r] & H_{n}(X,B)\ar[r] & \dots}}
\end{equation}
\end{node}

\begin{node}
Let $(X,A,B)$ be a triple --- so $B\subset A\subset X$ and $X$ is a
topological space --- then we may consider the chain complex
$C_{*}(A,B)=C_{*}(A)/C_{*}(B)$ and $C_{*}(X,B)=C_{*}(C)/C_{*}(B)$. We
have an exact sequence
\begin{equation}
0\to C_{*}(A)/C_{*}(B)\to C_{*}(X)/C_{*}(B)\to C_{*}(X)/C_{*}(A)\to 0
\end{equation}
or equivalently
\begin{equation}
0\to C_{*}(A,B)\xrightarrow{i_{*}} C_{*}(X,B)\xrightarrow{j_{*}} C_{*}(X,A)\to 0
\end{equation}
induced by the inclusion maps $i\colon(A,B)\into(X,A)$ and $j\colon(X,A)\into(X,B)$.
Now we get a long exact sequence, which we write as an exact triangle
\begin{equation}
\vcenter{\xymatrix{H_{*}(A,B)\ar[rr]&&H_{*}(X,B)\ar[dl]\\
&H_{*}(X,A)\ar[ul]&}}
\end{equation}
\end{node}

\subsection{Mayer--Vietoris Sequences}

\begin{definition}
Let $A$ and $B$ be subspaces of $X$ such that $A\cup B=X$.
If the inclusion map $C_{*}(A)+C_{*}(B)\to C_{*}(X)$ induces an
isomorphism $H_{*}(C_{*}(A)+C_{*}(B))\to H_{*}(X)$, then $(A,B)$ is
called a \define{Mayer--Vietoris Pair} (or ``M-V Pair'' for short).

\textsc{Note:} when we write $C_{*}(A)+C_{*}(B)$ this \emph{is not}
the direct sum. We just add the elements together as subgroups of
$C_{*}(X)$. 
\end{definition}

\begin{theorem}
If $(A,B)$ is an M-V pair, then there is a long exact sequence
\begin{equation}
\dots\to H_{n}(A\cap B)\xrightarrow{(i_{A*},-i_{B*})}H_{n}(A)\oplus H_{n}(B)\xrightarrow{j_{A*}+j_{B*}}H_{n}(X)\xrightarrow{\boundary}H_{n-1}(A\cap B)\to\dots
\end{equation}
where
\begin{equation}
  \vcenter{\xymatrix{&\ar[dl]_{i_{A}}A\cap B\ar[dr]^{i_{B}}&\\
      A\ar[dr]_{j_{A}} & & B\ar[dl]^{j_{B}}\\
  &X&}}
\end{equation}
are the inclusion maps which give us the induced chain maps and
induced maps of homology groups.
\end{theorem}

\begin{question}
Why is that one morphism $(i_{A*},-i_{B*})$?
\end{question}

\begin{proof}[Answer]
We must have a short exact sequence, and this is the only possible way
to obtain one. The proof will show why.
\end{proof}

\begin{proof}
We have a short exact sequence
\begin{equation}
0\to C_{*}(A\cap B)\to%\xrightarrow{(i_{A\sharp},-i_{B\sharp})}
C_{*}(A)\oplus C_{*}(B)\xrightarrow{j_{A\sharp}+j_{B\sharp}}C_{*}(A)+C_{*}(B)\to0.
\end{equation}
We see that $j_{A\sharp}+j_{B\sharp}\colon C_{*}(A)\oplus C_{*}(B)\to C_{*}(A)+C_{*}(B)$
is clearly surjective. What is its kernel?

If we have an element in $\ker(j_{A\sharp}+j_{B\sharp})$, and if we have
$\alpha\in C_{*}(A)\cap C_{*}(B)$, then $(\alpha,-\alpha)\in\ker(j_{A\sharp}+j_{B\sharp})$.
But $C_{*}(A)\cap C_{*}(B)=C_{*}(A\cap B)$ since $C_{*}(A)$ is
generated by singular simplices in $A$, and $C_{*}(B)$ is generated by
singular simplices in $B$, so $C_{*}(A)\cap C_{*}(B)$ is generated by
singular simplices in both $A$ and $B$ (i.e., it is generated by
singular simplices in $A\cap B$). But we see
\begin{equation}
(\alpha,-\alpha)=(i_{A\sharp}(\alpha),-i_{B\sharp}(\alpha)),
\end{equation}
which answers the question just raised.

This short exact sequences induces the long exact sequence
\begin{equation}
\vcenter{\xymatrix{H_{*}(A\cap B)\ar[rr]&&\ar[dl]H_{*}(A)\oplus H_{*}(B)\\
&\ar[ul]H_{*}(C_{*}(A)+C_{*}(B))\ar[r]^-{\iso}&H_{*}(X).}}
\end{equation}
Hence the result.
\end{proof}

\begin{question}
So when do we have an M-V pair? The definition isn't very helpful
identifying which spaces admit them, and which subspaces we should use.
\end{question}

\begin{proposition}
If we have $\Interior(A)\cup\Interior(B)=X$ (where $\Interior(A)$ is
the interior of $A$, and $\Interior(B)$ is the interior of $B$), then
$(A,B)$ is an M-V pair.
\end{proposition}

\begin{proof}[Proof idea]
We want $H_{*}(C_{*}(A)+C_{*}(B))\to H_{*}(X)$ to be an isomorphism.
The idea is that given a singular simplex in $X$, we can subdivide
such that every simplex is contained in either $A$ or $B$.
\end{proof}

\begin{proposition}
Let $A$ and $B$ be closed subsets of $X$.
If $A\cap B$ has an open neighborhood $V$ which deformation retracts
onto $A\cap B$, then $(A,B)$ is an M-V pair.
\end{proposition}

\begin{proof}[Proof idea]
Let $A'=A\cup V$ and $B'=B\cup V$, then $\int(A')\cup\int(B')=X$,
and $A'$ deformation retracts onto $A$, and $B'$ deformation retracts
onto $B$. Then we have a long exact sequence
\begin{equation}
\vcenter{\xymatrix{H_{*}(A'\cap B')\ar[rr]&&H_{*}(A')\oplus H_{*}(B')\ar[dl]\\
&\ar[ul]H_{*}(X)&}},
\end{equation}
but we see that $H_{*}(A'\cap B')\iso H_{*}(A\cap B)$ and
$H_{*}(A')\oplus H_{*}(B')\iso H_{*}(A)\oplus H_{*}(B)$. The result
follows from these facts.
\end{proof}

\begin{remark}
The important thing is using M-V pairs to simplify calculating long
exact sequences.
\end{remark}

\begin{example}
Let us compute $H_{*}(\sphere{1})$ the homology of the circle. The
circle can be written as the union of two parts, so $A\cap B$ consists
of just two points. We have
\begin{equation}
H_{1}(A\cap B)\to\underbrace{H_{1}(A)\oplus H_{1}(B)}_{=0}\to
H_{1}(X)\to H_{0}(A\cap B)\to H_{0}(A)\oplus H_{0}(B)\to H_{0}(X)\to0.
\end{equation}
We can use a purely algebraic argument to deduce the rest. But we will
use the topological information. We see that
\begin{equation}
H_{0}(A\cap B)=\ZZ\langle[p],[q]\rangle,
\end{equation}
and
\begin{equation}
H_{0}(A)\iso\ZZ\langle[p]\rangle\iso H_{0}(B),
\end{equation}
because ``We can just pick a point''. Then
\begin{equation}
(i_{*A},-i_{*B})\colon H_{0}(A\cap B)\to H_{0}\oplus H_{0}(B)
\end{equation}
sends
\begin{equation}
[p]\mapsto ([p],-[p]),\quad\mbox{and}\quad[q]\mapsto([p],-[p]),
\end{equation}
and its kernel is precisely $\ker(i_{*A},-i_{*B})=\ZZ\langle[p]-[q]\rangle$.
(We have $[p]=[q]$ because they are on the same component, so they are homologous.)
\end{example}

\begin{example}
Consider $X=\sphere{2}$. Let $A$ be the Northern hemisphere, let $B$
be the Southern hemisphere.
The relevant bit of the long exact sequence looks like
\begin{equation}
\underbrace{H_{2}(A)\oplus H_{2}(B)}_{=0}\to H_{2}(X)\to\underbrace{H_{1}(A\cap B)}_{\iso H_{1}(\sphere{1})\iso\ZZ}\to
\underbrace{H_{1}(A)\oplus H_{1}(B)}_{=0}\to H_{1}(X)\to\dots
\end{equation}
We see then that $H_{2}(X)\iso\ZZ$. The connecting morphism
$\boundary\colon H_{1}(X)\to H_{0}(A\cap B)$ sends $\boundary(1)=(1,-1)$,
and it must be injective since $H_{1}(A)\oplus H_{1}(B)=0\to H_{1}(X)\xrightarrow{\boundary}H_{0}(A\cap B)$
is exact.
We see more generally, this scheme gives us
\begin{equation}
H_{m}(\sphere{n})\iso\begin{cases}\ZZ & \mbox{if }m=0,n\\
0 & \mbox{otherwise}
\end{cases}
\end{equation}
\end{example}