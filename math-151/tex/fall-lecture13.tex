%%
%% fall-lecture13.tex
%% 
%% Made by Alex Nelson <pqnelson@gmail.com>
%% Login   <alex@lisp>
%% 
%% Started on  2025-10-28T07:19:46-0700
%% Last update 2025-10-28T07:19:46-0700
%% 

\lecture[Cellular Homology]{}

\begin{node}
Let $X$ be a CW complex. Recall, this means we have
\begin{equation}
X = \bigcup^{\infty}_{n=0}X^{n},
\end{equation}
and $X^{0}$ has the discrete topology. We obtain $X^{n}$ by attaching
$n$-cells to $X^{n-1}$. The $X^{n}$ are called \define{$n$-Skeletons}
of $X$.
\end{node}

\begin{definition}
We can define the \define{Cellular Chain Complex} $C^{CW}_{*}(X)$ to
be the Abelian group freely generated by all $n$-cells.
\end{definition}

\begin{definition}
We define the \define{Boundary Map} as the morphism
\begin{subequations}
\begin{equation}
d_{n}\colon C^{CW}_{n}(X)\to C^{CW}_{n-1}(X)
\end{equation}
of the cellular chain complex, defined by
\begin{equation}
e_{\alpha}^{n}\mapsto\sum_{\beta}d_{\alpha\beta}e^{n-1}_{\beta},
\end{equation}
\end{subequations}
where the coefficients $d_{\alpha\beta}\in\ZZ$ are integers. Let us
outline how to construct this morphism.

First step, we orient each cell.
(Note: choosing the orienation of the $n$-cells = choosing which
generators of the local homology are positive.)
Let $f_{\alpha}\colon\boundary e^{n}_{\alpha}\to X^{n-1}$
be the attaching map of $e^{n}_{\alpha}$, so $\boundary e^{n}_{\alpha}=\sphere{n-1}$.
We also define the quotient map
\begin{equation}
q_{\beta}\colon X^{n-1}\to X^{n-1}/(X^{n-1}\setminus e^{n-1}_{\beta}),
\end{equation}
but we see that
\begin{equation}
X^{n-1}/(X^{n-1}\setminus e^{n-1}_{\beta})=\disk{n-1}/\boundary\disk{n-1}=\sphere{n-1}.
\end{equation}
Now we define
\begin{equation}
d_{\alpha\beta}=\deg(q_{\beta}\circ f_{\alpha}).
\end{equation}
\end{definition}

\begin{example}
When $n=1$, a $1$-cell $e^{1}_{\alpha}$, we put some orientation on it
$e^{1}_{\alpha}=[a,b]$, then we see that $d_{1}e^{1}_{\alpha}=b-a$.
\end{example}

\begin{example}
The circle $\sphere{1}$. We choose the CW complex with one 0-cell and
one 1-cell. Then $C^{CW}_{1}(\sphere{1})=\ZZ\langle e^{1}\rangle$ and
$C^{CW}_{0}(\sphere{1})=\ZZ\langle e^{0}\rangle$. We see that the
boundary map $\boundary e^{1}=e^{0}-e^{0}=0$. Hence the CW homology
\begin{equation}
H^{CW}_{k}(\sphere{1})=\begin{cases}\ZZ & \mbox{if }k=0,1\\
0 & \mbox{otherwise.}
\end{cases}
\end{equation}
\end{example}

\begin{example}
The $n$-sphere $\sphere{n}$. We choose the CW complex with one 0-cells
and one $n$-cell. Let us assume $n>1$. Then
\begin{equation}
C^{CW}_{n}(\sphere{n})=\ZZ\langle e^{n}\rangle,\quad\mbox{and}\quad
C^{CW}_{0}(\sphere{n})=\ZZ\langle e^{0}\rangle.
\end{equation}
All other groups in the cellular chain complex are zero. The boundary
maps $d_{k}\colon C^{CW}_{k}(\sphere{n})\to C^{CW}_{k-1}(\sphere{n})$
for grading reasons are always zero. From this we can compute the
homology
\begin{equation}
H^{CW}_{k}(\sphere{n}) = \begin{cases}\ZZ & \mbox{if }k=0,n\\
0 & \mbox{otherwise}
\end{cases}
\end{equation}
\end{example}

\begin{example}
The Torus $T^{2}$. There is a single 0-cell $p$, two 1-cells ($a$ and $b$),
and one 2-cell ($e$). We sketch it out:
\begin{equation*}
\includegraphics{img/img.38}
\end{equation*}
Then we see its cellular chain complexes are:
\begin{subequations}
\begin{align}
C^{CW}_{2}(T^{2}) &= \ZZ\langle e\rangle\\
C^{CW}_{1}(T^{2}) &= \ZZ\langle a,b\rangle\\
C^{CW}_{0}(T^{2}) &= \ZZ\langle p\rangle
\end{align}
\end{subequations}
We see that $d_{1}\colon C_{1}\to C_{0}$ is always zero.
We orient the
2-cell as doodled using the red arrow below:
\begin{equation*}
\includegraphics{img/img.39}
\end{equation*}
The attachment map $f_{e}\colon\boundary e\to X^{1}$ is given by
running along $a^{-1}bab^{-1}$, and we see that
\begin{equation}
X^{1} = \vcenter{\hbox{\includegraphics{img/img.40}}}
\end{equation}
So this means that
\begin{equation}
X^{1}/(X^{1}\setminus\{a\}) = \vcenter{\hbox{\includegraphics{img/img.41}}}
\end{equation}
Then $q_{a}\colon X^{1}\to X^{1}/(X^{1}\setminus\{a\})$ yields
$a^{-1}a$ as the path. This tells us that $d_{ea}=0$ since
$\deg(q_{a}\circ f_{e}=1-1=0$. Similar reasoning tells us $d_{eb}=0$.

We can instead: choose a generic point $y$ in $a$ such that its
presimage has finitely many points, as doodled thus:
\begin{equation*}
\includegraphics{img/img.42}
\end{equation*}
The red loop would pass through
$y$ twice. The first time we pass through $y$, it is orientation-reversing.
The second time we pass through $y$, it is
orientation-preserving. This gives us the local degrees of $f_{8}$ at
$y$. This simplifies computing $\deg(q_{a}\circ f_{e})$.
\end{example}

\begin{example}
Consider $\Sigma_{g}$ the genus $g$ surface. Recall, we can describe
its cellular complex using a $4g$-gon (drawn here for $g=2$):
\begin{equation*}
\includegraphics{img/img.43}
\end{equation*}
This means we have one 0-cell, $2g$ 1-cells ($a_{1}$, $b_{1}$, \dots,
$a_{g}$, $b_{g}$), and one 2-cell (which we'll call $e$). Then we see
that the boundary map for the 2-cell is
\begin{equation}
\boundary e=[a_{1},b_{1}](\cdots)[a_{g},b_{g}]
\end{equation}
is the product of commutators $[a,b]=aba^{-1}b^{-1}$. Then we see that
\begin{equation}
\begin{split}
d_{2}e &= a_{1}+b_{1}-a_{1}-b_{1}+\cdots+a_{g}+b_{g}-a_{g}-b_{g}\\
&= 0.
\end{split}
\end{equation}
We see that the homology
\begin{equation}
H^{CW}_{k}(\Sigma_{g}) = \begin{cases}\ZZ &\mbox{if }k=0,2\\
\ZZ^{2g} & \mbox{if }k=1,
\end{cases}
\end{equation}
and all others vanish.
\end{example}

\begin{example}
Let us consider the non-orientable closed surface of genus $g$, which
we'll denote by $\Pi_{g}$ (in analogy to the choice of $\Sigma_{g}$ to
denote the oriented closed surface of genus $g$). We construct it from
a $2g$-gon. For example, taking $g=3$ (and orienting the 2-cell with
its boundary attached along the red circle), we sketch it as:
\begin{equation*}
\includegraphics{img/img.44}
\end{equation*}
We see there is one 0-cell, $g$ 1-cells (labeled $a_{1}$, \dots, $a_{g}$),
and one 2-cell (labeled $e$). Then we see that the boundary of the 2-cell
\begin{equation}
\boundary e = a_{1}^{2}a_{2}^{2}(\cdots)a_{g}^{2}.
\end{equation}
We see that $d_{1}=0$. For $d_{2}(e)$, the trick to computing it is to
again compute the local degrees along the edges. For example, along
the edge $a_{1}$, we pick a generic point $y$, and we see how many
times $\boundary e$ passes through it (multiplying by $\pm1$ depending
on orientation-preserving or orientation-reversing):
\begin{equation*}
\includegraphics{img/img.45}
\end{equation*}
Since the orientations of $a_{1}$ and the boundary $\boundary e$
agree, this means we get a contribution of $+2$ from those two
edges. Taken altogether,
\begin{equation}
d_{2}(e) = 2(a_{1}+a_{2}+\cdots+a_{g}).
\end{equation}
Then $\ker(d_{2})=0$ and $\Im(d_{2})=\ZZ\langle 2(a_{1}+\cdots+a_{g})\rangle$,
so
\begin{subequations}
\begin{equation}
H^{CW}_{2}(\Pi_{g})=0,
\end{equation}
and
\begin{equation}
H^{CW}_{1}(\Pi_{g})=\ZZ\langle a_{1},\dots,a_{g}\rangle/\ZZ\langle2(a_{1}+\cdots+a_{g})\rangle.
\end{equation}
We can do a change of basis from $a_{i}$ generators to $b_{i}$
generators where $b_{1}=a_{1}+\cdots+a_{g}$ and $b_{i}=a_{i}$
otherwise, giving us
\begin{equation}
H^{CW}_{1}(\Pi_{g})=\ZZ\langle b_{1},\dots,b_{g}\rangle/\ZZ\langle2b_{1}\rangle\iso(\ZZ/2\ZZ)\oplus\ZZ^{g-1}.
\end{equation}
\end{subequations}
\end{example}

\begin{example}
The complex projective space $\CP^{n}$. We recall Example~\ref{ex:fall-lec02:complex-projective-spaces} the cellular
structure for this has one $2k$-cell for all $0\leq k\leq n$. Then
\begin{equation}
C^{CW}_{m}(\CP^{n}) = \begin{cases}\ZZ &\mbox{if }m=2k,\mbox{ and }0\leq k\leq n,\\
0 &\mbox{otherwise}
\end{cases}
\end{equation}
The boundary map must be zero for grading reasons, since it changes
the parity of the grading. Then we see that
$H_{*}^{CW}(\CP^{n})=C^{CW}_{*}(\CP^{n})$.
\end{example}

\begin{example}
The real projective spaces $\RP^{n}$. We recall the cellular structure
from Example~\ref{ex:fall-lec02:real-projective-spaces}. Well,
we obtained a cellular structure for $\sphere{n}$ by attaching two
disks to the equator $\sphere{n-1}$. We can use cellular structure for
$\sphere{n}$ to obtain a cellular structure for $\RP^{n}$ by modding
out by the antipodal map. In particular, the CW complex for $\RP^{n}$
obtained in this manner has one $k$-cell for $0\leq k\leq n$. This
tells us that
\begin{equation}
C^{CW}_{m}(\RP^{n}) = \begin{cases}\ZZ &\mbox{if }m=0,\dots,n\\
0 & \mbox{otherwise}
\end{cases}
\end{equation}
What happesn for $d_{k}\colon C^{CW}_{k}(\RP^{n})\to C^{CW}_{k-1}(\RP^{n})$,
we have only one coefficient to compute. We will want to consider the
composition
\begin{equation}
\boundary e^{k-1}\xrightarrow{f_{e^{k}}}\RP^{k-1}\xrightarrow{q}\RP^{k-1}/\RP^{k-2},
\end{equation}
since $\RP^{k-1}\setminus e^{k-1}\iso\RP^{k-2}$ because there is only
one $(k-1)$-cell. The quotient map is not all that important. We see
that
\begin{equation}
\left.\vcenter{\hbox{\includegraphics{img/img.46}}}\right/\mbox{antipodal map}=X^{k}
\end{equation}
But the antipodal map identifies $e^{k}_{+}$ with $e^{k}_{-}$, and
transforms the equator $\sphere{k-1}$ into $\RP^{k-1}$.

Now, we compute the local degrees. Taking a point
$y\in\RP^{k-1}\setminus\RP^{k-2}=e^{k-1}$, this has 2 preimages under
the antipodal map $x_{+}$ and $x_{-}$ in $\boundary e^{k-1}=\sphere{k-1}$,
which differ by the antipodal map.

Let us call the attachment map $f_{e^{k}}$ in the sequence,
\begin{equation}
\boundary e^{k-1}\xrightarrow{f_{e^{k}}}\RP^{k-1}\xrightarrow{q}\RP^{k-1}/\RP^{k-2},
\end{equation}
but $f_{e^{k}}$ is a covering map, so we can pick two disjoint
neighborhoods $x_{+}\in U_{+}$ and $x_{-}\in U_{-}$. So we see that
\begin{equation}
f_{e^{k}}|_{U_{+}}=f_{e^{k}}|_{U_{-}}\circ\rho,
\end{equation}
where $\rho$ is the antipodal map. Then we find the degrees
\begin{equation}
\deg f_{e^{k}}|_{U_{+}}=\deg(f_{e^{k}}|_{U_{-}}\circ\rho)=\deg f_{e^{k}}|_{U_{-}}\deg\rho
=(-1)^{k}\deg f_{e^{k}}|_{U_{-}}.
\end{equation}
Then
\begin{subequations}
\begin{align}
\deg(\mbox{composition}) &= \deg f|_{U_{+}} + \deg f|_{U_{-}}\\
  \intertext{since $f|_{U_{+}}$ is a local homeomorphism}
&= (\pm1)+(\pm1)(-1)^{k}\\
  &= (\pm1)(1 + (-1)^{k})\\
  &=\begin{cases}0 &\mbox{if }k\mbox{ is odd}\\
  2 & \mbox{if }k\mbox{ is even}
  \end{cases}
\end{align}
\end{subequations}
We end up with the coefficient for the boundary map looking like
\begin{equation}
\dots\xrightarrow{2}C^{CW}_{3}\xrightarrow{0}C^{CW}_{2}\xrightarrow{2}C^{CW}_{1}\xrightarrow{0}C^{CW}_{0},
\end{equation}
which tells us
\begin{equation}
H^{CW}_{k}(\RP^{n}) = \begin{cases}
\ZZ & \mbox{if }k=0\\
0 & \mbox{if }0<k<n\mbox{ is even}\\
\ZZ/2\ZZ & \mbox{if } 0<k<n\mbox{ is odd}\\
\ZZ & \mbox{if }k=n\mbox{ is odd}\\
0 & \mbox{all other cases}
\end{cases}
\end{equation}
\end{example}