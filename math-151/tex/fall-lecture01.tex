%%
%% fall-lecture01.tex
%% 
%% Made by Alex Nelson <pqnelson@gmail.com>
%% Login   <alex@lisp>
%% 
%% Started on  2025-10-02T11:14:33-0700
%% Last update 2025-10-02T11:14:33-0700
%%

\lecture[Review of homotopy]

The emphasis of this course will be on homology theory.

\begin{node}[Review of homotopy]
Many invariants are homotopy invariants, so let us review the notion
of homotopy. If we have two continuous maps between topological spaces,
\begin{equation}
f_{0},f_{1}\colon X\to Y,
\end{equation}
we say $f_{0}$ and $f_{1}$ are \define{Homotopic} if there exists a
map
\begin{equation}
F\colon X\times[0,1]\to Y
\end{equation}
such that $F(\cdot,1)=f_{1}(\cdot)$ and $F(\cdot,0)=f_{0}(\cdot)$. The
map $F$ is called a \define{Homotopy}. We may also view $F(\cdot,t)=f_{t}$
as a family of maps where $t\in[0,1]$. More abstractly, if we consider
\begin{equation}
\begin{split}
  [0,1]&\to\Maps(X,Y)\\
  t&\mapsto f_{t}
\end{split}
\end{equation}
then this gives us a path $\{f_{t}\}_{t\in[0,1]}$ in the space
$\Maps(X,Y)$ from $f_{0}$ to $f_{1}$. Here $\Maps(X,Y)$ is the set of
continuous functions from $X$ to $Y$, equipped with the compact-open topology.

We will write $f_{0}\homotopic f_{1}$ to indicate $f_{0}$ and $f_{1}$
are homotopic.
\end{node}

\begin{node}[Relative homotopy]
If $A\subset X$ is a subspace, and you have $\{f_{t}\}$ is a homotopy
between $f_{0}$ and $f_{1}$, then we say $f_{t}$ is a
\define{Relative Homotopy $\rel{A}$} if $f_{t}|_{A}$ does not depend
on $t$. In this case, we say $f_{0}$ and $f_{1}$ \define{are Homotopic $\rel{A}$},
and we write $f_{0}\homotopic f_{1}\rel{A}$.
\end{node}

\begin{definition}
A map $f\colon X\to Y$ is called a \define{Homotopy Equivalence}
if there exists a map $g\colon Y\to X$ such that $f\circ g\homotopic\id_{Y}$
and $g\circ f\homotopic\id_{X}$.
(This is similar to a homeomorphism, but we just replace equality with homotopy.)
In this case, we call $g$ a \define{Homotopy Inverse} of $f$, and we
say the spaces $X$ and $Y$ are \define{Homotopy Equivalent} (or they
have the same \emph{Homotopy Type}) and write this as $X\homotopic Y$.
\end{definition}

\begin{example}
The disk $\disk{n}$ is homotopic to a point $\point{x}$,
\begin{equation}
\disk{n}\homotopic\point{x}.
\end{equation}
You can pinch the disc to a point, and that's the homotopy.
\end{example}

\begin{example}
The graph $\infty$ is homotopy equivalent to $\theta$, which is
homotopy equivalent to a pair of glasses.
\end{example}

\begin{definition}[Retraction]
Let $A\subset X$ be a subspace. We call a map $r\colon X\to A$ a
\define{Retraction} if $r|_{A}=\id_{A}$. If we view $r\colon X\to X$,
then $r\circ r=r$.
\end{definition}

\begin{definition}[Deformation retraction]
Let $A\subset X$ be a subspace. We define a \define{Deformation Retraction}
of $X$ onto $A$ to be a family of maps $f_{t}\colon X\to X$ for $t\in[0,1]$
such that
\begin{enumerate}
\item $f_{0}=\id_{X}$
\item $f_{1}(X)=A$, and
\item $f_{t}|_{A}=\id_{A}$.
\end{enumerate}
In other words, it is a relative homotopy $\rel{A}$ connecting
$\id_{X}$ and a retraction.
\end{definition}

\begin{example}
Let $X=A\times[0,1]\supset A\times\{0\}$ and suppose we identify $A$
with $A\times\{0\}$. We define a family of maps (which we claim is a
deformation retraction)
\begin{equation}
f_{t}\colon X\to A,
\end{equation}
where we just collapse everything to $A$. We have
\begin{equation}
f_{t}(a,s) = (a, (1-t)s)
\end{equation}
where $(a,s)\in A\times[0,1]$.
\end{example}

\begin{example}
Consider $X$ being a ``pair of pants''. We can draw a figure 8 on our
pair of pants. We map $[0,1]$ to $X$, sending $0$ to the figure 8. We
have $f_{t}(s)=(1-t)s$ treating each of the red lines as an interval.

(A similar argument could be made for the theta graph as a subspace of
our pair of pants $X$.)
\end{example}

\begin{theorem}
If $f_{t}\colon X\to A$ is a deformation retraction, then $f_{1}$ is a
homotopy equivalence.
\end{theorem}

\begin{definition}[Mapping cylinder]
Let $f\colon X\to Y$ be a continuous map. We define the
\define{Mapping Cylinder} $\MappingCylinder{f}$ to be the space
$(Y\sqcup(X\times[0,1]))/\sim$ where the equivalence relation
identifies $(x,1)\sim f(x)$.

The complement of $Y$ may be decomposed as the (disjoint) union of intervals, if
we identify $f(X)\subset Y$ as a subset of $\MappingCylinder{f}$. So 
$\MappingCylinder{f}\setminus Y$ is a disjoint union of intervals such
that the endpoint of each interval is in $Y$.
\end{definition}

\begin{proposition}
The mapping cylinder of $f\colon X\to Y$ deformation retracts onto
$Y$. Hence $\MappingCylinder{f}\homotopic Y$.
\end{proposition}

\begin{remark}
The pair of pants $X$ with the pince-nez glasses graph $G$ turns out to be a mapping
cylinder of $f\colon S^{1}\sqcup S^{1}\sqcup S^{1}\to G$. The three
circles are just the boundary of the pair of pants.
\end{remark}

\begin{definition}[Null-homotopic]
A map $f\colon X\to Y$ is \define{Null-Homotopic} if it is homotopic
to a constant map.
\end{definition}

\begin{example}
Any map $f\colon X\to\RR^{n}$ is null homotopic. Just take $f_{t}(x)=(1-t)f(x)$.
When $t=1$, we map everything to the origin.
\end{example}

\begin{definition}[Contractible space]
A topological space is \define{Contractible} if its identity map is
null-homotopic.
\end{definition}

\begin{example}
We see $\disk{n}$, $\RR^{n}$, and any tree are all contractible.
\end{example}