%%
%% fall-lecture28.tex
%% 
%% Made by Alex Nelson <pqnelson@gmail.com>
%% Login   <alex@lisp>
%% 
%% Started on  2025-12-04T10:12:38-0800
%% Last update 2025-12-04T10:12:38-0800
%% 

\lecture{}

\begin{theorem}[Poincar\'{e} duality]
If $M$ is an oriented $n$-manifold, then there is an isomorphism
$D_{M}\colon H^{k}_{c}(M)\to H_{n-k}(M)$ for all $k$.
\end{theorem}

We won't prove this theorem, but we will sketch out its proof.

\begin{note}
In this form, we do not assume $M$ is closed. But when $M$ is closed,
the cohomology with compact support coincides with the usual
cohomology, i.e., $H^{k}_{c}(M)=H^{k}(M)$. 
\end{note}

\begin{node}[Initial step]
We need to define $D_{M}$. When $M$ is closed,
\begin{equation}
D_{M}(\alpha)=\alpha\frown[M],
\end{equation}
where $[M]=\mu_{M}$ is the fundamental class. But when $M$ is not
closed, we can write it as a directed limit.
\end{node}

\begin{lemma}\pushQED{\qed}
$\displaystyle{H^{*}_{c}(M)=\varinjlim_{K\text{ compact in }M}H^{*}(M,M\setminus K)}$\qedhere
\end{lemma}

\begin{node}
By a previous lemma or theorem, there exists a unique $\mu_{K}\in H_{n}(M,M\setminus K)$
such that $\rho_{x}(\mu_{K})=\mu_{x}\in H_{n}(M,M\setminus\{x\})$. So
we can us this to define
\begin{equation}
D_{K}\colon H^{k}(M,M\setminus K)\to H_{n-k}(M).
\end{equation}
Then we just take the direct limit, and we get $D_{M}\colon H^{k}_{c}(M)\to H_{n-k}(M)$.
This concludes the ``initial step'' (defining the Poincar\'{e} duality mapping).
\end{node}

\begin{proof}[Proof outline]\marginpar{{\footnotesize Proof sketch for Poincar\'{e} duality}}\ignorespaces%
\textsc{Case 1:} $M=\RR^{n}$ case. This is direct.

\textsc{Case 2:} If the theorem is true for $M=U$ and $M=V$ and
$M=U\cap V$, then it is also true for $M=U\cup V$. This uses
Mayer--Vietoris sequence argument and the Five lemma. We have the
exact sequence
\begin{equation}
\vcenter{\xymatrix{
H^{i}_{c}(U\cap V)\ar[r]\ar[d]^{D_{U\cap V}} & H^{i}_{c}(U)\oplus H^{i}_{c}(V)\ar[r]\ar@<-5ex>[d]^{D_{U}}\ar@<3ex>[d]^{D_{V}} & H^{i}_{c}(U\cup V)\ar[r]\ar[d]^{D_{U\cup V}} & H^{i+1}_{c}(U\cap V)\ar[r]\ar[d]^{D_{U\cap V}} & H^{i+1}_{c}(U)\oplus H^{i+1}_{c}(V)\ar@<-5ex>[d]^{D_{U}}\ar@<3ex>[d]^{D_{V}}\\
H_{n-i}(U\cap V)\ar[r] & H_{n-i}(U)\oplus H_{n-i}(V)\ar[r] & H_{n-i}(U\cup V)\ar[r] & H_{n-i-1}(U\cap V)\ar[r] & H_{n-i-1}(U)\oplus H_{n-i-1}(V)}}
\end{equation}
Since $D_{U\cap V}$, $D_{U}$, and $D_{V}$ are isomorphisms, the Five
lemma implies $D_{U\cup V}$ is an isomorphism.

\textsc{Case 3:} $M=\bigcup^{\infty}_{m=1}U_{m}$ where
$U_{m}\propersubset U_{m+1}$ and the theorem is true for each
$U_{m}$. Then we just apply the direct limit, and then the previous
case implies the result holds for $M$.

\textsc{Case 4:} If $U\subset\RR^{n}$ is open, then the theorem is
true for $U$ by cases 2 and 3.

\textsc{Case 5:} The general case. Since $M$ is covered by countably
many charts, using cases 2--4 implies the result.
\end{proof}

\begin{remark}
Just to reiterate: when $M$ is not orientable, then the isomorphism
still works for $\ZZ/2\ZZ$ coefficients.
\end{remark}

\subsection{Applications of Poincar\'{e} Cohomology}

\begin{corollary}
If $M$ is an odd-dimensional closed manifold, then the Euler
characteristic vanishes $\chi(M)=0$.
\end{corollary}

\begin{proof}
We know
\begin{equation}
\chi(M)=\sum^{n}_{i=0}(-1)^{i}\dim(H_{i}(M;\ZZ/2\ZZ)),
\end{equation}
but by Poincar\'{e} duality
\begin{equation}
H^{n-i}(M;\ZZ/2\ZZ)\iso H_{i}(M;\ZZ/2\ZZ).
\end{equation}
Then by the universal coefficient theorem
\begin{equation}
H^{i}(M;\ZZ/2\ZZ)\iso\hom(H_{i}(M),\ZZ/2\ZZ)\oplus\Ext(H_{i-1}(M),\ZZ/2\ZZ),
\end{equation}
and
\begin{equation}
H_{i}(M;\ZZ/2\ZZ)\iso (H_{i}(M)\otimes\ZZ/2\ZZ)\oplus\Tor(H_{i-1}(M),\ZZ/2\ZZ),
\end{equation}
and it turns out these two are isomorphic:
\begin{equation}
H_{i}(M;\ZZ/2\ZZ)\iso H^{i}(M;\ZZ/2\ZZ).
\end{equation}
Then we can write (since $n$ is odd),
\begin{equation}
\chi(M)=\sum^{(n-1)/2}_{i=0}\bigl((-1)^{i}\dim(H_{i}(M;\ZZ/2\ZZ))+(-1)^{i+1}\dim(H_{n-i}(M;\ZZ/2\ZZ))\bigr),
\end{equation}
but
\begin{subequations}
  \begin{align}
H_{n-1}(M;\ZZ/2\ZZ) &\iso H^{n-i}(M;\ZZ/2\ZZ) \quad\mbox{by Universal Coefficient}\\
&\iso H_{i}(M;\ZZ/2\ZZ)\quad\mbox{by Poincar\'{e} duality},
  \end{align}
\end{subequations}
so each summand in the Euler characteristic vanishes, giving
$\chi(M)=0$.
\end{proof}

\begin{remark}
People know more about even-dimensional manifolds than odd-dimensional
manifolds because $\chi(M)=0$ for $\dim(M)$ odd.
\end{remark}

\marginpar{{\footnotesize Connection to Cup Product}}

\begin{node}
Let $R$ be a commutative ring. Consider the mapping
\begin{equation}
\begin{split}
H^{k}(M;R)\times H^{n-k}(M;R)&\to R\\
(\varphi,\psi)&\mapsto\langle\varphi\smile\psi,[M]\rangle,
\end{split}
\end{equation}
which defines a bilinear pairing. Let us call the biliniear pairing
$Q\colon H^{k}(M;R)\times H^{n-k}(M;R)\to R$.

In general, if we have a function $Q\colon A\times B\to R$, then this
induces two morphisms
\begin{equation}
\begin{split}
A\to\hom(B,R)\\
a\mapsto Q(a,-),
\end{split}
\end{equation}
and similarly a morphism $B\to\hom(A,R)$.

We call a pairing \define{Nonsingular} if these two induced morphisms
are isomorphism.
\end{node}

\begin{proposition}
The cup product pairing $Q$ is nonsingular for closed oriented
manifolds when \emph{either} $R$ is a field \emph{or} $R=\ZZ$ and the
torsion in $H^{*}(M)$ is factored out.
\end{proposition}

(If $M$ is nonorientable, then this result holds when $R=\ZZ/2\ZZ$.)

\begin{proof}
For simplicity, we will prove the case when $R=\ZZ$ (but the argument
is similar for other cases). Consider the map
\begin{equation}
h\colon H^{n-i}(M)\to\hom(H_{n-i}(M),\ZZ)
\end{equation}
obtained by the Universal Coefficient Theorem. Recall we had
\begin{equation}
0\to\underbrace{\Ext(H_{n-i-1}(M),\ZZ)}_{\text{torsion part}}\to H^{n-i}(M)\xrightarrow{h}\underbrace{\hom(H_{n-i}(M),\ZZ)}_{\text{free part}}\to0
\end{equation}
If we mod out the torsion,
\begin{equation}
h\colon H^{n-i}(M)/\mbox{(torsion)}\to\hom(H_{n-i}(M),\ZZ)
\end{equation}
is an isomorphism by the Universal coefficient theorem. Then
$D^{*}_{M}$ is an isomorphism obtained by Poincar\'{e} duality as
follows: If $\varphi\in H^{n-i}(M)/\mbox{(torsion)}$, then
$(D^{*}_{M}\circ h)(\varphi)\in\hom(H^{i}(M),\ZZ)$. Let $\psi\in H^{i}(M)$,
then
\begin{subequations}
  \begin{align}
(D^{*}_{M}\circ h)(\varphi)(\psi)
&=h(\varphi)\bigl(D_{M}(\psi)\bigr)\mbox{ by def of } D^{*}_{M}\\
&=\langle\varphi,D_{M}(\psi)\rangle\mbox{ by def of }h\\
&=\langle\varphi,\psi\smile[M]\rangle\mbox{ by def of }D_{M}\\
&=\langle\varphi\smile\psi,[M]\rangle\\
&=Q(\varphi,\psi)\mbox{ by def of }Q.
  \end{align}
\end{subequations}
So the morphism defined by $Q$ is $D^{*}_{M}\circ h$ which is an isomorphism.
Hence the pairing $Q$ is nonsingular.
\end{proof}

\begin{example}
Suppose we want to compute $H^{*}(\CP^{2})$. We know that
\begin{equation}
H^{k}(\CP^{2})\iso\begin{cases}\ZZ & \mbox{if }k=0,2,4\\
0 & \mbox{otherwise}
\end{cases}
\end{equation}
But what is the cup product? For dimension reasons, we only need to
understand the cup product
\begin{equation}
H^{2}(\CP^{2})\times H^{2}(\CP^{2})\to H^{4}(\CP^{2})\iso\ZZ.
\end{equation}
The proposition implies this mapping is nonsingular. There are only 2
nonsingular pairings, they are determined by the image of $(1,1)$. So
what is $(1,1)$ mapped to? It must be a gerneator of $H^{4}(\CP^{2})\iso\ZZ$,
so $(1,1)\mapsto\pm1$. But we can always choose the generator such
that $(1,1)\mapsto+1$. So the cohomology ring for $\CP^{2}$ is
\begin{equation}
H^{*}(\CP^{2})\iso\ZZ[x]/(x^{3}),
\end{equation}
where $x$ is a generator of $H^{2}(\CP^{2})$.
\end{example}

\begin{fact}
Complex manifolds (like $\CP^{n}$) are orientable. But $\RP^{2}$ is
not orientable.
\end{fact}

\begin{example}
We can try this for $\CP^{3}$. Recall the cohomology groups for
$\CP^{3}$ are:
\begin{equation}
H^{k}(\CP^{3})\iso\begin{cases}\ZZ & \mbox{if }k=0,2,4,6\\
0 & \mbox{otherwise}
\end{cases}
\end{equation}
What is the ring structure of this guy? We observe there is an
inclusion map
\begin{equation}
i\colon\CP^{2}\into\CP^{3},
\end{equation}
which induces a ring morphism
\begin{equation}
i^{*}\colon H^{*}(\CP^{3})\to H^{*}(\CP^{2}),
\end{equation}
and a long exact sequence
\begin{equation}
\vcenter{\xymatrix{H^{*}(\CP^{2})\ar[dr]_{[+1]} & & \ar[ll]H^{*}(\CP^{3})\\
&H^{*}(\CP^{3},\CP^{2})\ar[ru]&}}
\end{equation}
where
\begin{equation}
H^{*}(\CP^{3},\CP^{2})\iso\widetilde{H}^{*}(\CP^{3}/\CP^{2})\iso\widetilde{H}^{*}(\sphere{6}).
\end{equation}
This gives us a short exact sequence
\begin{equation}
\begin{array}{ccrcccccc}
0 & \to & \widetilde{H}^{*}(\sphere{6}) & \to & H^{*}(\CP^{3}) & \to H^{*}(\CP^{2}) & \to & 0\\
  &     & \widetilde{H}^{6}(\sphere{6})\iso\ZZ & \xrightarrow{\iso} & H^{6}(\CP^{3}) & & & &\\
  &     &                               &     & H^{j}(\CP^{3}) & \xrightarrow{\iso} & H^{j}(\CP^{2}) & & 
\end{array}
\end{equation}
for $j<6$. Since $H^{*}(\CP^{2})\iso\ZZ[x]/(x^{2})$, we know
$H^{2}(\CP^{3})$ is generated by $x$, and $H^{4}(\CP^{3})$ is
generated by $x^{2}$. For dimension reasons, we only need to
understand the cup product for
\begin{equation}
H^{2}(\CP^{3})\times H^{4}(\CP^{3})\to H^{6}(\CP^{3}).
\end{equation}
We know it is nonsingular, so $x^{3}$ is a generator for $H^{6}(\CP^{3})$.
This implies the cohomology ring for $\CP^{3}$ satisfies
\begin{equation}
H^{*}(\CP^{3})\iso\ZZ[x]/(x^{4}).
\end{equation}
More generally, for any $\CP^{n}$, the cohomology ring satisfies
\begin{equation}
H^{*}(\CP^{n})\iso\ZZ[x]/(x^{n+1}),
\end{equation}
by a similar argument.
\end{example}