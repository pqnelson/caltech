%%
%% fall-lecture25.tex
%% 
%% Made by Alex Nelson <pqnelson@gmail.com>
%% Login   <alex@lisp>
%% 
%% Started on  2025-11-25T09:42:28-0800
%% Last update 2025-11-25T09:42:28-0800
%% 

\lecture{}

\begin{notation}
Let $X$ and $Y$ be CW complexes. We will write (in this lecture, at least)
$p_{X}\colon X\times Y\to X$ and $p_{Y}\colon X\times Y\to Y$
for the projection morphisms.
\end{notation}

\begin{definition}
Let $X$ and $Y$ be CW complexes. Let $R$ be a ring.
We define the \define{Cross Product} (or ``\emph{External Cup Product}'')
of cohomology groups as
\begin{equation}
\begin{split}
&H^{k}(X;R)\times H^{\ell}(Y;R)\xrightarrow{\times}H^{k+\ell}(X\times Y;R)\\
(a,b)\mapsto p_{X}^{*}(a)\smile p_{Y}^{*}(b),
\end{split}
\end{equation}
where $p_{X}^{*}$ is the pullback of the projection $p_{X}$, and
$p_{Y}^{*}$ is the pullback of $p_{Y}$.
\end{definition}

\begin{puzzle}
When is this cross product an isomorphism?
\end{puzzle}

\begin{remark}[Sources in the literature]
The only sources I could find on the ``cross product'' in homology or
cohomology appears to be Hatcher~\cite{hatcher2002algebraic},
Chapter~3 Section~B. After looking around, it appears that only
Hatcher discusses ``cross products'' of CW complexes.
\end{remark}

\begin{theorem}[{Hatcher~\cite[Th.~3.15]{hatcher2002algebraic}}]
The cross product $H^{*}(X;R)\times H^{*}(Y;R)\to H^{*}(X\times Y;R)$
is a ring isomorphism if $X$ and $Y$ are CW complexes and $H^{*}(Y;R)$
is a finitely-generated $R$-module.
\end{theorem}

\begin{proof}
Define two cohomology theories,
\begin{subequations}
\begin{equation}
h^{*}(X,A):=H^{*}(X,A;R)\otimes H^{*}(Y;R),
\end{equation}
and
\begin{equation}
k^{*}(X,A):=H^{*}(X\times Y,A\times Y;R).
\end{equation}
\end{subequations}
(By ``cohomology theories'', we mean functor from the category of
pairs of spaces to the category of rings satisfying the
Eilenberg--Steenrod axioms for cohomology.) We can check these
functors satisfy the axioms for cohomology; $k^{*}$ is easy. The hard
work is with $h^{*}$ where we need to use the fact $H^{*}(Y;R)$ is a
finitely-generated free module over $R$, then $h^{*}(X,A)$ is just
several copies of $H^{*}(X,A;R)$.

We see $h^{*}(\point{})=H^{*}(Y;R)$ and $k^{*}(\point{})=H^{*}(Y;R)$
and they are equal $h^{*}(\point{})=k^{*}(\point{})$. From this we can
see these two theories are isomorphic.

But we want \emph{more}. We define
\begin{equation}
\mu\colon h^{*}(X,A)\to k^{*}(X,A)
\end{equation}
to be the natural transformation defined by the cross product. So
$\mu$ induces an isomorphism for $(X,A)=(\point{},\emptyset)$. Then
$\mu$ is an isomorphism for all CW pairs $(X,A)$, a fact which follows
from the Eilenberg--Steenrod axioms for cohomology.
\end{proof}

\begin{example}
Consider the torus $\torus{n}=\sphere{1}\times\dots\times\sphere{1}$
which is the product of $n$ copies of the circle $\sphere{1}$. We see
that $H^{*}(\sphere{1})\iso\ZZ[x]/(x^{2})$ where $x$ is the generator
of $\ZZ$ and the grading of $x$ is $\deg(x)=1$.
Then
\begin{subequations}
  \begin{align}
H^{*}(\torus{n}) &=\bigotimes H^{*}(\sphere{1})\\
&=\ZZ\langle x_{1},\dots,x_{n}\rangle/(x_{1}^{2},\dots,x_{n}^{2},x_{i}x_{j}=-x_{j}x_{i})\\
&=\ExteriorAlgebra\ZZ^{n}.
  \end{align}
\end{subequations}
This is just the exterior algebra of $\ZZ^{n}$.
\end{example}

\begin{example}
Consider $X=\sphere{2}\times\dots\times\sphere{2}$ which is the
product of $n$ copies of the sphere $\sphere{2}$. We recall the
cohomology ring $H^{*}(\sphere{2})\iso\ZZ[x]/(x^{2})$ where $x$ is a
generator of $\ZZ$ and its grading is now $\deg(x)=2$. Then we see
\begin{subequations}
  \begin{align}
H^{*}(X) &=\bigotimes H^{*}(\sphere{2})\\
&=\ZZ[x_{1},\dots,x_{n}]/(x_{1}^{2},\dots,x_{n}^{2}).
  \end{align}
\end{subequations}
\end{example}

\subsection{Orientation of Manifolds}

\begin{definition}
A \define{$n$-Dimensional Manifold} $X$ is a Hausdorff space such that
for all $x\in X$ there exists an open subset $U\subset X$ such that
$x\in U$ and $U\iso\RR^{n}$ homeomorphic. We also require $X$ to be
separable (i.e., it has a countable dense subset).
\end{definition}

\begin{example}
Consider the space $\RR\sqcup\RR/\sim$ where the equivalence relation
is defined by $(x,0)\sim(x,1)$ if $x>0$. TThis is the line with two
origins, and it is not Hausdorff at the origin.

If we tried $(x,0)\sim(x,1)$ if $x\geq0$, then it's no longer locally Euclidean.
\end{example}

\begin{remark}
``The Long Line'' is Hausdorff, connected, but not path-connected, and
  it is not separable. Constructing it requires the axiom of choice.
\end{remark}

\begin{definition}
An \define{Orientation of $\RR^{n}$} at $x\in\RR^{n}$ is a choice of
generator of $H_{n}(\RR^{n},\RR^{n}\setminus\{x\})\iso\ZZ$.
\end{definition}

\begin{node}
Let $B\subset\RR^{n}$ be a ball containing distinct points $x\in B$ and $y\in B$.
Then
\begin{equation}
H_{n}(\RR^{n},\RR^{n}\setminus\{x\})\iso H_{n}(\RR^{n},\RR^{n}\setminus B)\iso
H_{n}(\RR^{n},\RR^{n}\setminus\{y\}).
\end{equation}
Then we see that the isomorphism
\begin{equation}
H_{n}(\RR^{n},\RR^{n}\setminus\{x\})\iso 
H_{n}(\RR^{n},\RR^{n}\setminus\{y\})
\end{equation}
is canonical in the sense that it is independent of the choice of ball $B$. 
\end{node}

\begin{node}
Now, for any space $X$,
we recall $H_{*}(X,X\setminus\{x\})\iso H_{*}(U,U\setminus\{x\})$ for
any open subset $U\subset X$ containing $x\in U$. (This is just the
Excision theorem.)

Now let $X$ be an $n$-dimensional manifold, and $U$ is a neighborhood
of $x\in X$ such that $U\iso\RR^{n}$ homeomorphic. Then
$H_{*}(X,X\setminus\{x\})\iso H_{*}(\RR^{n},\RR^{n}\setminus\{\point{}\})\iso\ZZ$.
\end{node}

\begin{definition}
Let $X$ be an $n$-dimensional manifold. Let $x\in X$.
We define a \define{Local Orientation} of $X$ at $x$ is a choice of
generator for $H_{n}(X,X\setminus\{x\})\iso\ZZ$.
\end{definition}

\begin{definition}
Let $X$ be an $n$-dimensional manifold.
An \define{Orientation} of $X$ is an assignment of a local orientation
$\mu_{x}$ at each $x\in X$ 
such that $\mu_{x}$ ``varies continuous'' with respect to $x$.
\end{definition}

\begin{node}[``Varies continuously''?]
More precisely, for each $x$, there exists a compact neighborhood $N$
and a class $\mu_{N}\in H_{n}(X,X\setminus N)$ such that the
pushforward of the inclusion $(\rho_{y})_{*}(\mu_{N})=\mu_{y}$ for
each $y\in N$ and the inclusion $\rho_{y}\colon(X,X\setminus N)\to(X,X\setminus\{y\})$.
So all the local orientations inside $N$ are determined by the local
orientation of $N$.
\end{node}

\begin{definition}
We say a manifold $X$ is \define{Orientable} if an orientation exists.
\end{definition}