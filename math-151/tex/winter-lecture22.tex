%%
%% winter-lecture22.tex
%% 
%% Made by Alex Nelson <pqnelson@gmail.com>
%% Login   <alex@lisp>
%% 
%% Started on  2026-02-24T08:12:57-0800
%% Last update 2026-02-24T08:12:57-0800
%% 

\lecture[Postnikov towers]{}

\begin{recall}
We had a construction (\S\ref{prop:existence-of-postnikov-tower}):
Given a path-connected space $X$, the Postnikov tower of $X$ is the
commutative diagram
\begin{equation}
\vcenter{\xymatrix{     &  \cdots\ar[d]\\ 
                        &  X_{2}\ar[d]\\
X\ar[uur]\ar[ur]\ar[r]  &  X_{1}}}
\end{equation}
such that
\begin{enumerate}
\item the map $X\to X_{k}$ induces an isomorphism
  $\pi_{i}(X)\iso\pi_{i}(X_{k})$ for $i\leq k$; and
\item $\pi_{i}(X_{k})=0$ for $i>k$.
\end{enumerate}
Such a tower is unique up to homotopy equivalence.
\end{recall}

\begin{node}
It turns out, we can do slightly better: to get a Postnikov tower such
that $X_{k}\to X_{k-1}$ is a \emph{fibration} for all $k$. More
explicitly (this is an inductive process):
\begin{enumerate}
\item For $X_{2}\to X_{1}$, we use the mapping path construction (\S\ref{defn:mapping-path})---recall
for $A\to B$ we constructed $A'$ and a unique [up to homotopy
  equivalence] fibration $A'\to B$ as doodled in the commutative
diagram
\begin{subequations}
\begin{equation}
\vcenter{\xymatrix{A\ar[dr]\ar[r]^{\txt{$\sim$}} & A'\ar@{..>}[d]^{\text{fibration}}\\
& B}}
\end{equation}
then the Postnikov tower's initial couple levels
\begin{equation}
\vcenter{\xymatrix{ & X_{2}\ar[d]\\
X\ar[ur]\ar[r] & X_{1}}}
\end{equation}
we extend $X_{2}\to X_{1}$ using the mapping path construction, we
introduce $X'_{1}:=X_{1}$ (the primed $X'_{i}$ will be the new
Postnikov tower we are constructing, and we draw in red the
corresponding entries used in the mapping path construction):
\begin{equation}
\vcenter{\xymatrix{ & {\color{red}X_{2}}\ar[d]\ar@[red][r]^{\homotopic} & {\color{red}X'_{2}\ar@[red][d]^{\text{fibration}}}\\
X\ar[ur]\ar[r] & X_{1}\ar@{=}[r] & {\color{red}X'_{1}}}}
\end{equation}
\end{subequations}
This gives us the ``base'' of the tower for the inductive construction.
\item Then inductively, to construct $X'_{k+1}$ (assuming that
  $X'_{k}$ is a fibration over $X'_{k-1}$), we have:
\begin{subequations}
\begin{equation}
\vcenter{\xymatrix{%
               & X_{k+1}\ar[d]                  & \\
X\ar[r]\ar[ur] & X_{k}\ar@{-}[r]^{\txt{$\sim$}} & X'_{k}}}
\end{equation}
We consider the mapping path space for the composition of
\begin{equation}
X_{k+1}\to X_{k}\homotopic X'_{k},
\end{equation}
i.e., the map $X_{k+1}\to X'_{k}$. This gives us a space $X'_{k+1}$
and a fibration $X'_{k+1}\to X'_{k}$, filling in our diagram with:
\begin{equation}
\vcenter{\xymatrix{%
               & X_{k+1}\ar[d]\ar[r]^{\txt{$\sim$}} & X'_{k+1}\ar[d]^{\text{fibration}} \\
X\ar[r]\ar[ur] & X_{k}\ar@{-}[r]^{\txt{$\sim$}}    & X'_{k}}}
\end{equation}
\end{subequations}
\end{enumerate}
Therefore, we can assume all of the $X_{k}\to X_{k-1}$ are fibrations.
\end{node}

\begin{question}
Then a natural question presents itself: What is the fiber of
$X_{k}\to X_{k-1}$?
\end{question}

\begin{node}
We know
\begin{subequations}
\begin{equation}
\pi_{i}(X_{k})=\begin{cases}\pi_{i}(X) & \mbox{for }i\leq k\\
0 & \mbox{otherwise},
\end{cases}
\end{equation}
and
\begin{equation}
\pi_{i}(X_{k-1})=\begin{cases}\pi_{i}(X) & \mbox{for }i\leq k-1\\
0 & \mbox{otherwise},
\end{cases}
\end{equation}
\end{subequations}
so these guys differ only at $\pi_{k}$.

We also know the commutative diagram,
\begin{equation}
\vcenter{\xymatrix{ & X_{k}\ar[d]^{p}\\
X\ar[r]\ar[ur] & X_{k-1}}}
\end{equation}
and the morphism $X\to X_{k}$ induces an isomorphism $\pi_{i}(X)\to\pi_{i}(X_{k})$
for $i\leq k$, and the induced morphism
\begin{equation}
p_{*}\colon\pi_{i}(X_{k})\to\pi_{i}(X_{k-1})
\end{equation}
is an isomorphism for $i\leq k-1$. Then by the long exact sequence for
fibration----let us call the fiber of $g_{k}\colon X_{k}\to X_{k-1}$
by $F_{k}$----we have
\begin{equation}
\dots\to\pi_{i+1}(X_{k})\to\pi_{i+1}(X_{k-1})\to\pi_{i}(F_{k})\to\pi_{i}(X_{k})\to\pi_{i}(X_{k-1})\to\dots
\end{equation}
But when $i=k$, we see that 
\begin{equation}
\dots\to\pi_{k+1}(X_{k})\to\underbrace{\pi_{k+1}(X_{k-1})}_{=0}\to\pi_{k}(F_{k})\to\pi_{k}(X_{k})\to\underbrace{\pi_{k}(X_{k-1})}_{=0}\to\dots
\end{equation}
which means $\pi_{k}(F_{k})\iso\pi_{k}(X_{k})$. But when $i\neq k$,
since $\pi_{i}(X_{k})\iso\pi_{i}(X_{k-1})$, this forces
$\pi_{i}(F_{k})=0$. So we have
\begin{equation}
\pi_{i}(F_{k})=\begin{cases}\pi_{k}(X_{k}) & \mbox{if }i=k\\
0 & \mbox{otherwise.}
\end{cases}
\end{equation}
Then
\begin{equation}
F_{k}=K(\pi_{k}(X),k).
\end{equation}
\end{node}

\begin{node}
Therefore, the Postnikov tower gives us a way to decompose $X$ as a
(homotopic version of) ``twisted products'' of Eilenberg--Mac~Lane spaces.
\end{node}

\begin{question}
In the opposite direction, we can ask the question: Given the homotopy
groups $\pi_{n}(X)$ for all $n\geq0$, to what extent can we
``reconstruct'' $X$? What ``extra information'' would we need? (The
extra information concerns the ``twisted product''---and how would we
characterize it in algebraic topological terms?)

We can restrict to a less general case to get an easier answer: when
all the fibrations in the Postnikov tower are ``principal''.
\end{question}

\begin{definition}
A fibration $F\to E\xrightarrow{p}B$ is called \define{Principal} if
there exists a commutative diagram
\begin{equation}
\vcenter{\xymatrix{F\ar[r]\ar[d] & E\ar[r]^{p}\ar[d] & B\ar[d] & \\
\Omega B'\ar[r] & F'\ar[r] & E'\ar[r] & B'}}
\end{equation}
where the bottom row is a fibration sequence of another fibration
$F'\to E'\to B'$, and the vertical maps are weak homotopy equivalences.

In particular, $F$ should look like a loop space of something else
$F\homotopic\Omega B$.
\end{definition}

\begin{remark}
In the definition, it is enough to have a fibration $B'\to E'\to F'$
and the commtuative diagram
\begin{equation}
\vcenter{\xymatrix{E\ar[r]^{p}\ar[d] & B\ar[d] &\\
F'\ar[r] &E'\ar[r] & B'}}
\end{equation}
with vertical morphisms are weak homotopy equivalence. Given this, the
map $F\to\Omega B'$ is automatically given and will be a weak homotopy equivalence.
\end{remark}

\begin{node}[Answering the question]
Returning to our question, assuming all the maps $X_{k}\to X_{k-1}$ in
the Postnikov tower are \emph{principal} fibrations. Then we have
\begin{equation}
\vcenter{\xymatrix{K(\pi_{n}(X),n)\ar[r]\ar[d] & X_{n}\ar[r]\ar[d] & X_{n-1}\ar[d] & \\
K(\pi_{n}(X),n)\ar[r] & X_{n}\ar[r] & X_{n-1}\ar[r] & k(\pi_{n}(X),n+1)}}
\end{equation}
where we recall $K(\pi_{n}(X),n)\homotopic\Omega K(\pi_{n}(X),n+1)$.
In the preceding commutative diagram, the homotopy group of this is
uniquely determined by the long exact sequence of fibration which will
be the same as $K(\pi_{n}(X),n+1)$.

Then (if we are given $X_{n-1}$, we want to get $X_{n}$), the $X_{n}$
is the homotopy fiber of a map $X_{n-1}\to K(\pi_{n}(X),n+1)$ up to
homotopy. This is the same as an element in $H^{n+1}(X_{n-1},\pi_{n}(X))$.

Combining these ideas together, we get the commutative diagram:
\begin{equation}
\vcenter{\xymatrix{
    &                       & \cdots\ar[d] & \\
    & K(\pi_{3}(X),3)\ar[r] & X_{3}\ar[d]\ar[r]^-{k_{3}} & K(\pi_{4}(X),5)\\
    & K(\pi_{2}(X),2)\ar[r] & X_{2}\ar[d]\ar[r]^-{k_{2}} & K(\pi_{3}(X),4)\\
X\ar[r]\ar[ur]\ar[uur] & K(\pi_{1}(X),1)\ar@{=}[r] & X_{1}\ar[r]^-{k_{1}} & K(\pi_{2}(X),3)}}
\end{equation}
\end{node}

\begin{definition}
The \define{$n^{\text{th}}$ $k$-Invariant of $X$} is the element of
$H^{n+2}(X_{n},\pi_{n+1}(X))$ which corresponds to the map
\begin{equation}
k_{n}\colon X_{n}(X)\to K(\pi_{n+1}(X),n+2)
\end{equation}
from the preceding commutative diagram.
\end{definition}

\begin{claim}
If the Postnikov tower consists of principal fibrations, then it is
determined by homotopy groups $\{\pi_{n}(X)\}$ and the $n^{\text{th}}$ $k$-invariants of $X$.
\end{claim}

\begin{question}
When can we guarantee the Postnikov tower consists of principal fibrations?
\end{question}

\begin{theorem}
A connected CW complex $X$ has a Postnikov tower consisting of
principal fibrations if and only if $\pi_{1}(X)$ acts trivially on
$\pi_{n}(X)$ for all $n>1$.
\end{theorem}

(In particular, if $X$ is simply-connected [so
  $\pi_{1}(X)=\TrivialGroup$], then this is guaranteed.)

\begin{proof}
\forwardproof\ Suppose $X$ has a Postnikov tower consisting of
principal fibrations. We want to show $\pi_{1}(X)$ acts trivially on
$\pi_{n}(X)$ for all $n>1$.

The question (of whether a fibration $p\colon A\to X$ is principal or
not) is the same as asking if there exists another fibration $F\to E\xrightarrow{p'}B$
such that
\begin{equation}
\vcenter{\xymatrix{A\ar[d]^{\homotopic}\ar[r] & X\ar[d]^{\homotopic}\\
F\ar[r] & E}}
\end{equation}
with vertical maps are homotopy equivalences.

\textsc{Claim:} If such a thing exists, then the action of
$\pi_{1}(A)$ on $\pi_{n}(X,A)$ must be trivial, for the following
reason: it's the same as the action of $\pi_{1}(F)$ on $\pi_{1}(B)$
which is trivial (by our commutative diagram and homotopy
equivalence).

Recall, we have the isomorphism $p_{*}\colon\pi_{n}(E,F)\to\pi_{n}(B,b_{0})$
for any $n$. Let $\gamma\in\pi_{1}(F)$, let
$\alpha\in\pi_{n}(E,F)$. Then under the isomorphism $p_{*}$, the
element $\gamma\bullet\alpha-\alpha$ becomes $p_{*}(\gamma)p_{*}(\alpha)-p_{*}(\alpha)$.
The diagram to have in mind is:
\begin{center}
\includegraphics{img/img.67}
\end{center}
Then $p_{*}(\gamma$ must be the constant loop at $b_{0}$. Then
\begin{equation}
p_{*}(\gamma)p_{*}(\alpha)-p_{*}(\alpha)=0,
\end{equation}
so $\pi_{1}(A)$ acts trivially on $\pi_{n}(X,A)$ when $A\to X$ is a
principal fibration.

In the Postnikov tower, $X_{n}\to X_{n-1}$ plays the role of $A\to X$,
and $\pi_{1}(X_{n})\iso\pi_{1}(X)$ for all $n\geq1$ and
$\pi_{n}(X_{n},X_{n-1})\iso\pi_{n}(X)$. 

\backwardproof\ Long, unenlightening, so we skip it.
\end{proof}

\begin{remark}
Hatcher uses the idiosyncratic term \define{Abelian CW Complex} for the situation
where $\pi_{1}(X)$ acts trivially on $\pi_{n}(X)$ for all $n>1$.
Others have called such things ``simple'', but there does not seem to
be any consensus.
\end{remark}