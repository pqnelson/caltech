%%
%% fall-lecture06.tex
%% 
%% Made by Alex Nelson <pqnelson@gmail.com>
%% Login   <alex@lisp>
%% 
%% Started on  2025-10-11T11:54:46-0700
%% Last update 2025-10-11T11:54:46-0700
%% 

\lecture{}

\begin{example}[Space with torsion]
Consider $\RP^{2}$. We have a $\Delta$-complex with 2 vertices
$\{a,b\}$, 3 edges $\{e_{1},e_{2},e_{3}\}$, and 2 triangles
$\{\sigma_{1},\sigma_{2}\}$. We draw it as:
\begin{equation*}
\includegraphics{img/img.28}
\end{equation*}
We have the boundary maps
\begin{equation}
\begin{split}
\boundary_{2}\colon&C^{\Delta}_{2}(X)\to C^{\Delta}_{1}(X)\\
& \sigma_{1}\mapsto e_{1}+e_{2}+e_{3}\\
& \sigma_{2}\mapsto e_{1}+e_{2}-e_{3}
\end{split}
\end{equation}
and
\begin{equation}
\begin{split}
\boundary_{1}\colon&C^{\Delta}_{1}(X)\to C^{\Delta}_{0}(X)\\
&e_{1}\mapsto b-a\\
&e_{2}\mapsto a-b\\
&e_{3}\mapsto 0
\end{split}
\end{equation}
We find $\ker(\boundary_{2})=0$ and 
\begin{equation}
\Im(\boundary_{2})=\ZZ\langle e_{1}+e_{2}+e_{3},e_{1}+e_{2}-e_{3}\rangle\iso\ZZ\langle e_{1}+e_{2}+e_{3},2e_{3}\rangle
\end{equation}
and $\ker(\boundary_{1})=\ZZ\langle e_{1}+e_{2},e_{3}\rangle$ and
$\Im(\boundary_{1})=\ZZ\langle a-b\rangle$.
Then we can compute the homology groups
\begin{subequations}
\begin{equation}
H^{\Delta}_{2}(X) = \ker(\boundary_{2})/\Im(\boundary_{3})=0
\end{equation}
\begin{equation}
\begin{split}
H^{\Delta}_{1}(X) &= \ker(\boundary_{1})/\Im(\boundary_{2})=\ZZ\langle e_{1}+e_{2}+e_{3},e_{3}\rangle/\ZZ\langle e_{1}+e_{2}+e_{3},2e_{3}\rangle\\
&\iso\ZZ/2\ZZ
\end{split}
\end{equation}
\begin{equation}
H^{\Delta}_{0}(X)\iso\ZZ.
\end{equation}
\end{subequations}
The $H^{\Delta}_{1}(X)\neq0$ is interpreted as ``torsion''.
\end{example}

\begin{theorem}
The group $H^{\Delta}(X)$ does not depend on the choice of the
$\Delta$-complex structure on $X$.
\end{theorem}

We will not prove this result, it is not easy to prove. We would need
a lot of technical infrastructure. Instead, we will work with singular
homology.

\subsection{Singular Homology}

\begin{definition}[From Homological Algebra]\label{defn:fall-lec06:chain-complex}
A \define{Chain Complex} (in homological algebra) is an Abelian group
$C_{*}=\bigoplus_{n\in\ZZ}C_{n}$ (Algebraists allow $n\in\ZZ$ or any
graded ring, topologists only need $n\in\NN_{0}$) equipped with a morphism
$\boundary\colon C_{*}\to C_{*}$ decomposed as $\boundary=\bigoplus_{n\in\ZZ}\boundary_{n}$
where $\boundary_{n}\colon C_{n}\to C_{n-1}$ satisfies $\boundary\circ\boundary=0$.

Since $\boundary\circ\boundary=0$, we have
$\ker(\boundary_{n})\supset\Im(\boundary_{n+1})$. So we may define
$H_{n}(C_{*})=\ker(\boundary_{n})/\Im(\boundary_{n+1})$ and similarly
$H_{*}(C_{*})=\bigoplus_{n\in\ZZ}H_{n}(C_{*})$. This is called the
\define{Homology} of $(C_{*},\boundary)$.

The $\boundary$ map is called the \define{Boundary} or \define{Differential}.
\end{definition}

\begin{definition}
Let $X$ be a topological space.
Let $\Delta^{n}=[v_{0},\dots,v_{n}]$ where $v_{i}\in\RR^{n+1}$ form a
canonical orthonormal basis.
We call $\Delta^{n}$ the \define{Standard $n$-Simplex}.

We define a \define{Singular $n$-Simplex} in $X$ is just a continuous
map $\sigma\colon\Delta^{n}\to X$, no restrictions.

We do not need $\interior\Delta{}^{n}$ to be embedded in $X$, we may
have non-injective continuous $\sigma$.

Then $C_{n}(X)$ is the Free Abelian Group freely generated by all
singular $n$-simplices. (Its basis is uncountable, usually.) Elements
of $C_{n}(X)$ are called \define{(Singular) $n$-Chains}.
\end{definition}

\begin{node}
We need to define $\boundary_{n}\colon C_{n}(X)\to C_{n-1}(X)$ which
we can define by
\begin{equation}
\boundary_{n}(\sigma) = \sum^{n}_{j=0}(-1)^{j}\sigma|_{[v_{0},\dots,\widehat{v_{j}},\dots,v_{n}]}
\end{equation}
where we take the alternating sum of faces of $\sigma$, each face
identified with $\Delta^{n-1}$ in the obvious way. So intuitively,
this corresponds closely to the definitions of simplicial notions (of
boundary maps, homologies, etc.).

(Since $\sigma|_{[v_{0},\dots,\widehat{v_{j}},\dots,v_{n}]}$ is a
restriction to a face of $\Delta^{n}$, this is a map $\Delta^{n-1}\to X$
which is an $(n-1)$-simplex.)
\end{node}

\begin{lemma}
For any $n$, we have
$\boundary_{n}\circ\boundary_{n+1}=0$. Equivalently, $\boundary^{2}=0$.
\end{lemma}

The argument is the same as before.

\begin{definition}
We define the \define{$n^{\text{th}}$ Singular Homology} of $X$ to be
the group $H_{n}(X)=\ker(\boundary_{n})/\Im(\boundary_{n+1})$.
\end{definition}

\begin{node}[Digression on homological algebra]\label{defn:fall-lec06:chain-map}
Suppose $(C,\boundary)$ and $(C',\boundary')$ are two chain
complexes. We define a \define{Chain Map} to be a morphism $f\colon C\to C'$
preserving the grading $f=\bigoplus_{n\in\ZZ}f_{n}$ and $f_{n}\colon C_{n}\to C'_{n}$
preserves the boundary map $\boundary'\circ f=f\circ\boundary$, i.e.,
the following diagram commutes
\begin{equation}
\xymatrix{C_{n}\ar[d]_{\boundary_{n}}\ar[r]^{f_{n}}&C'_{n}\ar[d]^{\boundary'_{n}}\\
C_{n-1}\ar[r]_{f_{n-1}} & C'_{n-1}}
\end{equation}
for all $n$.

A chain map induces a morphism on the homology groups
\begin{equation*}
f_{*}\colon H_{*}(C)\to H_{*}(C')
\end{equation*}
is the ``induced'' mapping. This is because the chain map $f$
preserves the boundary map.
\end{node}

\begin{lemma}
We have $f(\ker(\boundary))\subset\ker(\boundary')$ and $f(\Im(\boundary))\subset\Im(\boundary')$.
\end{lemma}

\begin{corollary}
The chain map $f$ induces a map
$f_{*}\colon\ker(\boundary)/\Im(\boundary)\to\ker(\boundary')/\Im(\boundary')$.
\end{corollary}

\begin{node}[Sharp maps]
Let $X$ and $Y$ be topological spaces. Let $f\colon X\to Y$ be a
continuous mapping. Then $f$ induces a chain map $f_{\sharp}\colon C_{*}(X)\to C_{*}(Y)$
sending a singular $n$-chain on $X$, $\sigma\colon\Delta^{n}\to X$, to
a singular $n$-chain on $Y$ given by $f_{\sharp}(\sigma)=\sigma\circ f\colon\Delta^{n}\to Y$.
This means it preserves grading.

It is routine to check
$f_{\sharp}\circ\boundary^{(X)}=\boundary^{(Y)}\circ f_{\sharp}$.

We also get the induced map $f_{*}\colon H_{*}(X)\to H_{*}(Y)$ from
this mapping.
\end{node}

\begin{proposition}
Let $X\xrightarrow{g}Y\xrightarrow{f}Z$ be continuous maps of
topological spaces. Then $(f\circ g)_{\sharp}=f_{\sharp}\circ g_{\sharp}$ and 
$(f\circ g)_{*}=f_{*}\circ g_{*}$.
\end{proposition}

\begin{proposition}
The identity map $\id\colon X\to X$ induces $\id_{*}\colon H_{*}(X)\to H_{*}(X)$
the identity morphism of homology groups.
\end{proposition}

\begin{corollary}
The homology group $H_{*}(X)$ is homeomorphism invariant:
If $f\colon X\to Y$ is a homeomorphism of topological spaces (so there
exists $g\colon Y\to X$ such that $g\circ f=\id_{X}$ and $f\circ g=\id_{Y}$),
then $g_{*}\circ f_{*}=\id_{H_{*}(X)}$ and $f_{*}\circ g_{*}=\id_{H_{*}(Y)}$.
\end{corollary}