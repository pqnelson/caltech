%%
%% winter-lecture11.tex
%% 
%% Made by Alex Nelson <pqnelson@gmail.com>
%% Login   <alex@lisp>
%% 
%% Started on  2026-01-29T09:26:12-0800
%% Last update 2026-01-29T09:26:12-0800
%% 

\lecture{}

Our next application: quotient spaces.

\begin{theorem} % Hatcher, Proposition 4.28, PDF page 375
If a CW pair $(X,A)$ is $r$-connected and $A$ is $s$-connected (and
$r,s\geq0$), then $f_{*}\colon\pi_{i}(X,A)\to\pi_{i}(X/A,A/A)$
[induced by the quotient map $f\colon X\to X/A$] is an isomorphism for
$i\leq r+s$ and surjective for $i=r+s$.
\end{theorem}

Note that $\pi_{i}(X/A,A/A)\iso\pi_{i}(X/A)$.

\begin{proof}
Consider the mapping cone for the inclusion $i\colon A\into X$ which is
\begin{equation}
C_{f}=X\cup_{A}CA.
\end{equation}
More generally, for $f\colon Y\to X$, its mapping cone $C_{f}$ is its mapping
cylinder $\MappingCylinder{f}$ collapsing $Y$ to a point $C_{f}=\MappingCylinder{f}/Y\times\{0\}$
(recall $\MappingCylinder{f}=((Y\times I)\sqcup X)/(z,1)\sim f(z)$ is the mapping
cylinder (\S\ref{defn:mapping-cylinder})).
This is equivalent to $(CY\cup X)/(y,1)\sim f(y)$. Since $A$ is
$r$-connected and $CA$ is contractible, the long exact sequence for
the pair $(CA,A)$ implies $(CA,A)$ is $(s+1)$-connected. So by the
Excision theorem, the map
\begin{equation}
i_{*}\colon\pi_{i}(X,A)\to\pi_{i}(X\cup CA,A)
\end{equation}
is an isomorphism for $i\leq r+s$ and surjective if $i=r+s+1$.

Since $CA$ is contractible, the map
\begin{equation}
X\cup CA\xrightarrow{r}(X\cup CA)/CA=X/A,
\end{equation}
since $CA$ collapses to a point but that's the same as $A$ collapsing
to a point (i.e., $X/A$). Then $r$ is a deformation retraction (by
Proposition 0.17 of Hatcher). The quotient map
\begin{equation}
q\colon X\to X/A
\end{equation}
can be written as the following commutative diagram
\begin{equation}
\vcenter{\xymatrix{ & X\cup CA\ar[dr]^{r} & \\
X\ar@{^{(}->}[ur]^{i}\ar[rr]^{q} & & X/A}}
\end{equation}
In other words, $q=r\circ i$. Hence we have the following
\begin{equation}
\vcenter{\xymatrix{\pi_{i}(X,A) \ar@/_/[drr]_{q_{*}}\ar[r]^-{i_{*}} & \pi_{i}(X\cup CA,CA)\ar[r]^-{r_{*}} & \ar[d]^{\iso}\pi_{i}\bigl((X\cup CA)/CA, CA/CA\bigr)\\
& & \pi_{i}(X,A)}}
\end{equation}
where $r_{*}$ is an isomorphism because it's' a deformation
retraction. Hence $q_{*}$ has the desired property.
\end{proof}

\begin{example}
Compute $\pi_{n}(\bigvee_{\alpha\in A}\sphere{n}_{\alpha})$. When
$n=1$, the Seifert--van Kampen theorem says this is just the free
product of $|A|$ copies of $\ZZ$.

When $n\geq2$, the situation is similar, we claim
\begin{equation}
\pi_{n}(\bigvee_{\alpha\in A}\sphere{n}_{\alpha})\iso\bigoplus_{\alpha\in A}\ZZ
\end{equation}
Note: $A$ may be an infinite set, or it may be a finite set.
\end{example}

\begin{proof}
\textsc{Case 1:} $A$ is a finite indexing set. If each
$\sphere{n}_{\alpha}$ is given the standard cell structure (i.e., one
$0$-cell and one $n$-cell). Then consider $\prod_{\alpha\in A}\sphere{n}_{\alpha}$
with the product cell structure---so for each $\alpha\in A$, we pick
either the $0$-cell or $n$-cell from $\sphere{n}_{\alpha}$. In
particular, each cell in $\prod_{\alpha\in A}\sphere{n}_{\alpha}$ has
dimension $kn$ (for some $k=0,1,\dots,n$). Then the wedge sum
$\bigvee_{\alpha\in A}\sphere{n}_{\alpha}$ is the $n$-skeleton of
$\prod_{\alpha\in A}\sphere{n}_{\alpha}$.
So each cell in the wedge product is one $n$-cell producted with the
remaining $0$-cells. And $(\prod_{\alpha\in A}\sphere{n}_{\alpha},\bigvee_{\alpha\in A}\sphere{n}_{\alpha})$
is $(2n-1)$-connected since cells in 
$(\prod_{\alpha\in A}\sphere{n}_{\alpha})\setminus(\bigvee_{\alpha\in A}\sphere{n}_{\alpha})$
has dimension at least $2n$.

By the long exact sequence of the pair $(\prod_{\alpha\in A}\sphere{n}_{\alpha},\bigvee_{\alpha\in A}\sphere{n}_{\alpha})$
we have
\begin{equation}
\pi_{n}(\bigvee_{\alpha\in A}\sphere{n}_{\alpha})\iso\pi_{n}(\prod_{\alpha\in A}\sphere{n}_{\alpha})\iso\prod_{\alpha\in A}\pi_{n}(\sphere{n}_{\alpha})\iso\bigoplus_{\alpha\in A}\ZZ,
\end{equation}
as desired.

\textsc{Case 2:} $A$ is infinite. Define a map
\begin{equation}
\Phi\colon\bigoplus_{\alpha\in A}\pi_{n}(\sphere{n}_{\alpha})\to\pi_{n}(\bigvee_{\alpha\in A}\sphere{n}_{\alpha}),
\end{equation}
by the universal property of the coproduct this is defined by how it
behaves on components. Then define it to be
\begin{equation}
\Phi:=\bigoplus_{\alpha\in A}(i_{\alpha})_{*}.
\end{equation}
We want to show $\Phi$ is surjective and injective.

\textsc{Surjectivity:} Suppose $[f]\in\pi_{n}(\bigvee_{\alpha\in A}\sphere{n}_{\alpha})$,
i.e., we have $f\colon\sphere{n}\to\bigvee_{\alpha\in A}\sphere{n}_{\alpha}$.
Since $\sphere{n}$ is compact and $f$ is continuous, $f(\sphere{n})$
is compact and therefore touches only finitely many $\sphere{n}_{\alpha}\setminus\{\point{*}\}$.
So $f\in\Im(\Phi)$.

\textsc{Injectivity:} Suppose $\Phi(g)=0$.. Then there is a homotopy
$H$ from $(\bigoplus_{\alpha\in A}i_{\alpha})\circ g$ to the constant
map in $\bigvee_{\alpha\in A}\sphere{n}_{\alpha}$. But the image of
$H$ is compact, so it only intersects finitely of the $\sphere{n}_{\alpha}$
away from the point. Then it follows $[g]=0$.
\end{proof}

\begin{example}
$\pi_{n}(\sphere{1}\vee\sphere{n})$ for $n\geq2$.
\end{example}

\begin{proof}
\textsc{Claim 1:} $\pi_{n}(\sphere{1}\vee\sphere{n})\iso\ZZ[t,t^{-1}]$.
(We see $\ZZ[t,t^{-1}]\iso\bigoplus_{n\in\ZZ}\ZZ$ indexed by $\deg(t)$.)
We can consider the universal cover of $\sphere{1}\vee\sphere{n}$
which is roughly attaching $\sphere{n}$ to each point of $\RR$ modulo
translation. Let $\widetilde{X}$ be this covering space. Then
$\widetilde{X}\homotopic\bigvee_{\alpha\in\ZZ}\sphere{n}_{\alpha}$, so
\begin{equation}
\pi_{n}(\sphere{1}\vee\sphere{n})
\iso\pi_{n}(\widetilde{X})
\iso\pi_{n}(\bigvee_{\alpha\in\ZZ}\sphere{n}_{\alpha})
\iso\bigoplus_{\alpha\in\ZZ}\ZZ
\iso\ZZ[t,t^{-1}].
\end{equation}
Hence the result.
\end{proof}

\begin{remark}
This example shows given some finite CW complex, its homotopy group
could be infinitely-generated. (The source of this infinity is the
action of $\pi_{1}$.)
\end{remark}

\begin{example}
We can realize any Abelian group as $\pi_{n}$ for $n\geq2$.
Consider
\begin{equation}
X:=(\bigvee_{\alpha\in A}\sphere{n}_{\alpha})\underbrace{\cup_{\varphi_{\beta}}(\bigcup_{\beta\in B}e^{n+1}_{\beta})}_{\text{kills the relations for Abelian group}},
\end{equation}
where the gluing map is
\begin{equation}
\varphi_{\beta}\colon(\sphere{n}\iso\boundary e^{n+1}_{\beta})\to\bigvee_{\alpha\in A}\sphere{n}_{\alpha}.
\end{equation}
\textsc{Claim:}
\begin{equation}
\pi_{n}(X)\iso\frac{\bigoplus_{\alpha\in A}\ZZ}{\langle[\varphi_{\beta}]\mid\beta\in B\rangle}.
\end{equation}
\begin{proof}
Consider the long exact sequence of the pair $(X,\bigvee_{\alpha\in A}\sphere{n}_{\alpha})$,
\begin{equation}
\vcenter{\xymatrix{\pi_{n+1}(X,\bigvee_{\alpha\in A}\sphere{n}_{\alpha})\ar[r]^{\boundary}
&\ar[d]^{\iso}\pi_{n}(\bigvee_{\alpha\in A}\sphere{n}_{\alpha})\ar[r]
&\pi_{n}(X)\ar[r] &\pi_{n}(X,\bigvee_{\alpha\in A}\sphere{n}_{\alpha})\ar@{=}[d]\\
& \bigoplus_{\alpha\in A}\ZZ & & 0}}
\end{equation}
Exactness then says
\begin{subequations}
  \begin{align}
\pi_{n}(X)&\iso\pi_{n}(\bigvee_{\alpha\in A}\sphere{n}_{\alpha})/\Im(\boundary)\\
&\iso(\bigoplus_{\alpha\in A}\ZZ)/\Im(\boundary),
  \end{align}
\end{subequations}
where
\begin{equation}
\begin{split}
\boundary\colon&\pi_{n+1}(X,\bigvee_{\alpha\in A}\sphere{n}_{\alpha})\to\pi_{n}(\bigvee_{\alpha\in A}\sphere{n}_{\alpha})\\
&[e^{n+1}_{\beta}]\mapsto[\varphi_{\beta}(\boundary e^{n+1}_{\beta})],
\end{split}
\end{equation}
so $\Im(\boundary)=\langle[\varphi_{\beta}]\mid\beta\in B\rangle$.
\end{proof}
\end{example}

\subsection{Eilenberg--MacLane Spaces}

\begin{definition}
A space $X$ with only one nontrivial homotopy group $\pi_{n}(X)=G$ is
called a (the?) \define{Eilenberg--Mac~Lane Space} usually denoted
$K(G,n)$ for $n\geq1$.
\end{definition}

\begin{example}
$G=\ZZ$, $n=1$, $k(\ZZ,1)=\sphere{1}$.
\end{example}

\begin{example}
$K(\ZZ/2\ZZ, 1)=\RP^{\infty}=\sphere{\infty}/x\sim-x$ where we mod out
  by the antipodal map. 
\end{example}

\begin{example}
$K(\ZZ,2)=\CP^{\infty}=\sphere{\infty}/\sphere{1}$ where recall we had
\begin{equation}
\CP^{1}=\underbrace{\langle[z:w]\in\CC^{2}\mid |z|^{2}+|w|^{2}=1\rangle}_{=\sphere{3}}/\underbrace{[\E^{\I\theta}z:\E^{\I\theta}w]\sim[z:w]}_{=\sphere{1}},
\end{equation}
a sphere in $\CC^{n+1}\iso\RR^{2n+2}$ is $\sphere{2n+1}$,
so using the same reasoning we have more generally $\CP^{n}=\sphere{2n+1}/\sphere{1}$.
\end{example}

\begin{node}
In general, for $n=1$ and any group $G$, Hatcher \S1{.}B constructs a contractible space
$EG$ together with a free $G$-action, so $K(G,1):=(EG)/G=BG$.
\end{node}

\begin{node}
For $n\geq2$ and any Abelian group $G$, in the previous example we
constructed a CW complex $X$ of dimension $n+1$ and it is
$(n-1)$-connected and $\pi_{n}(X)=G$. We just need to kill all higher
homotopy groups. So by applying the same argument used in the CW
approximation by attaching cells of dimension at least $n+2$, we get a
CW complex which is $K(G,n)$.
\end{node}