%%
%% fall-lecture16.tex
%% 
%% Made by Alex Nelson <pqnelson@gmail.com>
%% Login   <alex@lisp>
%% 
%% Started on  2025-11-04T09:30:07-0800
%% Last update 2025-11-04T09:30:07-0800
%% 

\lecture[Categories and Functors]

\subsection{Categories}

\begin{definition}
We define a \define{Category} $\cat{C}$ to consist of
\begin{enumerate}
\item a \emph{collection} $\Ob(\cat{C})$ of ``\emph{objects\/}'';
\item disjoint \emph{collections}\footnote{The professor defined
categories to be locally small, and required the collection of
morphisms to be sets. I could not live with myself if I left that in
my notes.} $\Mor(X,Y)$ of ``\emph{morphisms\/}'' for each pair of
  objects $X,Y\in\Ob(\cat{C})$;
\item a ``\emph{composition of morphisms\/}'' function $\circ\colon\Mor(X,Y)\times\Mor(Y,Z)\to\Mor(X,Z)$
  for any three objects $X,Y,Z\in\Ob(\cat{C})$;
\item for each object $X\in\Ob(\cat{C})$, there is an ``\emph{identity morphism}''
  $\id_{X}\in\Mor(X,X)$;
\end{enumerate}
such that
\begin{enumerate}
\item\textsc{Unit laws}: for each $f\in\Mor(X,Y)$ we have $f\circ\id_{X}=f$
  and $\id_{Y}\circ f=f$;
\item \textsc{Associativity:} for any $X,Y,Z,W\in\Ob(\cat{C})$, and
  morphisms
  $X\xrightarrow{f}Y\xrightarrow{g}Z\xrightarrow{h}W$, we have
  $(f\circ g)\circ h=f\circ(g\circ h)$.
\end{enumerate}
\end{definition}

\begin{remark}[Notation]
It has become idiomatic to write:
\begin{enumerate}
\item $X\in\cat{C}$ instead of $X\in\Ob(\cat{C})$;
\item $\hom(X,Y)$ instead of $\Mor(X,Y)$;
\item if there's any ambiguity about which category $\Mor(X,Y)$
  belongs to, we will write it as a subscript $\Mor_{\cat{C}}(X,Y)$;
\item we write $f\colon X\to Y$ instead of $f\in\hom(X,Y)$.
\end{enumerate}
\end{remark}

\begin{definition}
We call a category $\cat{C}$ \define{Locally Small} if for all pairs
of objects $X,Y\in\Ob(\cat{C})$, we have $\Mor(X,Y)$ be a set.
\end{definition}

\begin{definition}
We call a locally small category $\cat{C}$ \define{Small} if $\Ob(\cat{C})$
is a set. Otherwise, we call $\cat{C}$ \define{Large}.
\end{definition}

\begin{example}
\begin{enumerate}
\item $\TOP$ is the large locally-small category whose objects are
  topological spaces and morphisms are continuous maps;
\item $\GRP$ is the large locally-small category whose objects are
  groups and morphisms are group homomorphisms;
\item $\AB$ is another large locally-small category whose objects are
  Abelian groups and morphisms are group morphisms;
\item $\LINEAR$ is the large locally-small category whose objects are
  linear spaces and morphisms are linear maps.
\end{enumerate}
\end{example}

\begin{example}
We define the cat $\BIMOD$ whose objects are rings, the morphisms are
$(R_{1},R_{2})$-bimodules, and composition of an
$(R_{1},R_{2})$-bimodule $M_{1}$ with an $(R_{2},R_{3})$-bimodule
$M_{2}$ is the tensor product $M_{1}\otimes_{R_{2}}M_{2}$.
\end{example}

\begin{example}
Let $X$ be a topological space. We can construct a category $\cat{C}$
whose objects $\Ob(\cat{C})$ is the set of points of $X$. The
morphisms are just $\Mor(X,Y)$ all paths from $X$ to $Y$, the
$\id_{X}$ is just the constant path. Composition is just concatenating paths.
\end{example}

\begin{remark}
The previous two examples do not have functions be morphisms. That's
important to realize: morphisms are not always functions.
\end{remark}

\begin{node}[Monoids as categories]
When a category $\cat{C}$ has a single object $\Ob(\cat{C})=\{*\}$,
then this really describes a monoid using $\Mor(*,*)$ as the elements
of the monoid and $\circ$ is the monoid's law of composition.
\end{node}

\begin{definition}
Let $\cat{C}$ be a category. Let $X,Y\in\Ob(\cat{C})$ be objects.
Let $f\in\Mor(X,Y)$ be a morphism. We call $f$ an \define{Isomorphism}
if there exists a $g\in\Mor(Y,X)$ such that $f\circ g=\id_{Y}$ and
$g\circ f=\id_{X}$.
\end{definition}

\begin{node}[Groups as categories]
If we have a category $\cat{C}$ where $\Ob(\cat{C})=\{*\}$ and all
morphisms are isomorphisms, then we really have something equivalent
to a group: $\Mor(*,*)$ are the elements of the group, $\circ$ is the
law of composition for the group.
\end{node}

\begin{definition}
A \define{Groupoid} is a small category $\cat{C}$ such that all
morphisms of $\cat{C}$ are isomorphisms.
\end{definition}

\begin{example}[Fundamental Groupoid]
Let $X$ be a topological space. The category $\Pi_{1}(X)$ has its
objects be points of $X$ and morphisms $\Mor(x,y)$ are justs paths
from $x$ to $y$ modulo homotopy $\rel\boundary$ of the path. The
composition are just concatenating these paths.
Observe $\Mor(x,x)=\pi_{1}(X,x)$ is the fundamental group with
base-point $x$.
\end{example}

\subsection{Functors}

\begin{definition}
Let $\cat{C}$ and $\cat{D}$ be categories.
A \define{Covariant Functor} $F\colon\cat{C}\to\cat{D}$ assigns an
object $F(X)\in\Ob(\cat{D})$ to each object $X\in\Ob(\cat{C})$, and
$F(f)\in\Mor_{\cat{D}}(F(X),F(Y))$ in $\cat{D}$ to each
$f\in\Mor_{\cat{C}}(X,Y)$ in $\cat{C}$ such that
\begin{enumerate}
\item\textsc{Preserves identity morphism}: $F(\id_{X})=\id_{F(X)}$ for each $X\in\Ob(\cat{C})$;
\item\textsc{Covariance}: $F(f\circ g)=F(f)\circ F(g)$ for all $X,Y,Z\in\Ob(\cat{C})$ and
  $f\in\Mor_{\cat{C}}(Y,Z)$ and $g\in\Mor_{\cat{C}}(X,Y)$.
\end{enumerate}
\end{definition}

\begin{example}
Let $\Chain$ be the category whose objects are singular chain complexes and
morphisms are chain maps. Then we have a covariant functor
$C_{*}\colon\TOP\to\Chain$ assigning a singular chain complex to each
topological space $X$, and to each continuous map $f\colon X\to Y$
it assigns the chain map $f_{\sharp}\colon C_{*}(X)\to C_{*}(Y)$.
\end{example}

\begin{example}
Let $\GrAB$ be the category of graded Abelian groups and
grade-preserving group morphisms.
We have a covariant functor $H_{*}\colon\Chain\to\GrAB$
sending a chain complex $(C,\boundary)$ to the homological group $H_{*}(C)$,
and chain maps $f\colon(C,\boundary)\to(C',\boundary')$ to the induced
maps $f_{*}\colon H_{*}(C)\to H_{*}(C')$.
\end{example}

\begin{example}
We can compose these two functors to obtain a covariant functor
$H_{*}\circ C_{*}\colon\TOP\to\GrAB$.
\end{example}

\begin{example}
We see that the fundamental group is also a functor from the category
$\TOP_{*}$ of pointed topological spaces to the category of groups --- $\pi_{1}\colon\TOP_{*}\to\GRP$.
\end{example}

\begin{example}
The fundamental groupoid is a functor $\Pi_{1}\colon\TOP\to\GRPD$
where $\GRPD$ is the category of groupoids and its morphisms are functors.
\end{example}

\begin{definition}
A \define{Contravariant Functor} $F\colon\cat{C}\to\cat{D}$ is like a
covariant functor, except when assigning morphisms:
\begin{enumerate}[start=2]
\item\textsc{Contravariance}: $F(f\circ g)=F(g)\circ F(f)$.
\end{enumerate}
\end{definition}

\begin{example}
The ``dual'' functor $*\colon\LINEAR\to\LINEAR$ which assigns a vector
space $V$ to its dual space $V^{*}$, and a linear map $f\colon V\to W$
is mapped to $f^{*}\colon W^{*}\to V^{*}$.
\end{example}

\subsection{Natural Transformations}

\begin{definition}
Let $\cat{C}$ and $\cat{D}$ be categories.
Let $F$, $G\colon\cat{C}\to\cat{D}$ be covariant functors.
We define a \define{Natural Transformation} $T$ from $F$ to $G$
assigns a morphism $T_{X}\colon F(X)\to G(X)$ for each object $X\in\Ob(\cat{C})$
(called the \emph{component} of $T$ at $X$)
such that for each morphism $f\in\hom_{\cat{C}}(X,Y)$ in $\cat{C}$ the
following diagram commutes
\begin{equation}
\vcenter{\xymatrix{%
F(X)\ar[d]_{T_{X}}\ar[r]^{F(f)} & F(Y)\ar[d]^{T_{Y}}\\
G(X)\ar[r]^{G(f)} & G(Y)}}
\end{equation}
We write $T\colon F\To G$ for the natural transformation.
\end{definition}

\begin{example}
Let $\cat{C}$ have its objects be pairs of spaces, and morphisms are
continuous maps of pairs of spaces. We have $H_{*}\colon\cat{C}\to\GrAB$
sending $(X,A)\mapsto H_{*}(X,A)$. Then $\boundary_{*}\colon H_{*}(X,A)\to H_{*-1}(A)$
is a natural transformation.
\end{example}

\begin{example}
Let $G$ and $H$ be Abelian groups. Let $\alpha\colon G\to H$ be a
group morphism. Now we have two functors $H_{*}(-;G),H_{*}(-;H)\colon\TOP\to\GrAB$.
Then we have a natural transformation $\alpha_{*}\colon H_{*}(-;G)\To H_{*}(-;H)$
such that the diagram commutes
\begin{equation}
\vcenter{\xymatrix{%
                 & H_{*}(X;G)\ar[dd]^{\alpha_{*}}\\
X\ar[ur]\ar[dr]  & \\
                 & H_{*}(X;H)}}
\end{equation}
\end{example}