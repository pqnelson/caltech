%%
%% winter-lecture08.tex
%% 
%% Made by Alex Nelson <pqnelson@gmail.com>
%% Login   <alex@lisp>
%% 
%% Started on  2026-01-24T08:44:30-0800
%% Last update 2026-01-24T08:44:30-0800
%% 

\lecture{}

We ended the last lecture in the middle of proving the following lemma:

\begin{lemma}
If $(X,A)$ is an $n$-connected CW pair, and if $A\neq\emptyset$, then
there exists a CW pair $(Z,A)\homotopic_{\rel{A}}(W,A)$ such that
$Z\setminus A$ has cells of dimension strictly greater than $n$.
\end{lemma}

\begin{proof}
We still need to show the ``rel $A$'' part of the homotopy equivalence.
Consider $f\colon Z\to X$. .We know $f|_{A}=\id_{A}$ by
construction. Then using cellular approximation, we can assume $f$ is
a cellular map such that $f|_{A}=\id_{A}$. Then consider the mapping
cylinder
\begin{equation}
M_{f}=(Z\times I\sqcup X)/(z,1)\sim f(z).
\end{equation}
Define $W:=M_{f}/(a,t)\sim(a,0)$ for all $a\in A$, where we are
smashing the cylinder $A\times I$ ``down''.

We claim $(Z,A)\homotopic_{\rel{A}}(W,A)\homotopic_{\rel{A}}(X,A)$.

To see $(W,A)\homotopic_{\rel{A}}(X,A)$, recall $W$ deformation
retracts to $X$ by sending $(X\setminus A)\times I$ to $(X\setminus A)\times\{1\}$
which is the identity on $A$.

Now, to see $(Z,A)\homotopic_{\rel{A}}(W,A)$. Let $r\colon W\to X$
be the deformation retraction we just mentioned. Let $j\colon Z\into W$
be the inclusion, then the following diagram commutes
\begin{equation}
\vcenter{\xymatrix{& W\ar[rd]^{r} &\\
Z \ar@{^{(}->}[ur]^{j}\ar[rr]^{f} & & X}}
\end{equation}
That is to say, $f=r\circ j$. Since $f\colon Z\to X$ is a weak
homotopy equivalence, and $r\colon W\to X$ is a deformation
retraction, then $j$ is also a weak homotopy equivalence. (Since
$r_{*}\colon\pi_{n}(W)\iso\pi_{n}(X)$ and
$f_{*}\colon\pi_{n}(Z)\iso\pi_{n}(X)$, so\dots)
Then by the long exact sequence of $(W,Z)$ we have
\begin{equation}
\vcenter{\xymatrix{ & \dots\ar[r]& \pi_{n+1}(Z,W)=0\ar@/_/[dll]_-{\boundary}\\
\pi_{n}(Z)\ar[r]^{j_{*}} & \pi_{n}(W)\ar[r] & \pi_{n}(Z,W)=0\ar@/^/[dll]_-{\boundary}\\
\dots & &}}
\end{equation}
(So $\pi_{n}(Z)\iso\pi_{n}(W)$ by exactness.)
Then applying the compression lemma to the identity map
\begin{equation}
\id\colon(W,Z,A)\to(W,Z,A)
\end{equation}
gives a deformation retract of $(W,A)$ to $(Z,A)$.
\end{proof}

\begin{observation}
The punchline is: it is easier to use the mapping cylinder than a map.
\end{observation}

\begin{definition}
Let $(X,A)$ be a CW pair such that $A\neq\emptyset$.
Then an \\define{$n$-Connected CW Model} is an $n$-connected CW pair
$(Z,A)$ and a map $f\colon Z\to X$ such that
\begin{enumerate}
\item $f|_{A}=\id_{A}$
\item $f_{*}\colon\pi_{i}(Z)\to\pi_{i}(X)$ is an isomorphism if $i>n$
  and injective if $i=n$
\end{enumerate}
\end{definition}

\begin{proposition}% Hatcher, prop 4.17
For any CW pair $(X,A)$ with $A\neq\emptyset$, for any $n\in\NN$,
there exists an $n$-connected CW model of $(Z,A)$ (such that $Z$ is
obtained by attaching cells of dimension greater than $n$).
\end{proposition}

\begin{remark}
Think of this $Z$ as a transition from $X$ to $A$ in some homotopic
way. As $n\to\infty$, it becomes more and more like $A$.
\end{remark}

\begin{remark}
An $n$-connected CW model of $(X,A)$ is unique up to homotopy
equivalence rel~$A$. (Hatcher Proposition~4.18 and Corollary~4.19):
for any two such $n$-connected CW models $(Z,A)$ and $(Z',A)$ we have $(Z,A)\homotopic_{\rel{A}}(Z',A)$.
\end{remark}

Another application of the mapping cylinder.

\begin{proposition}
A weak homotopy equivalence $f\colon X\to Y$ (for any spaces, no
assumptions imposed on $X$ and $Y$) induces isomorphisms
\begin{subequations}
  \begin{align}
f_{*}\colon H_{n}(X,G)\to H_{n}(Y,G)\\
\intertext{and}
f^{*}\colon H^{n}(Y,G)\to H^{n}(X,G)
  \end{align}
\end{subequations}
for all $n\in\NN_{0}$ and coefficient groups $G$.
\end{proposition}

\begin{proof}
Suffices to prove one of these, since the Universal Coefficient
theorem gives us $f^{*}$ from $f_{*}$.

Consider the mapping cylinder $M_{f}$ of $f$, which gives us the
following commutative diagram
\begin{equation}
\vcenter{\xymatrix{& M_{f}\ar[rd]^{r} &\\
X \ar@{^{(}->}[ur]^{j}\ar[rr]^{f} & & Y}}
\end{equation}
It suffices to prove the inclusion $j\colon X\into M_{f}$ satisfies
this statement, since $r$ is a deformation retraction. It suffices to
prove the following using the long exact sequence of $H_{*}$ for $(Z,X)$:
\begin{quote}\it
If $(Z,X)$ is an $n$-connected pair of path-connected spaces, then
$H_{i}(Z,X;G)=0$ for all $i\leq n$ and for all $G$.
\end{quote}
This is because
\begin{equation}
H_{i}(X)\xrightarrow{j_{*}}H_{i}(Z)\to H_{i}(Z,X)=0
\end{equation}
gives the isomorphism. Let us use singular homology for $H_{i}$. Let
\begin{equation}
\alpha=\sum_{j}n_{j}\sigma_{j}
\end{equation}
where $n_{j}\in G$, $\sigma_{j}\colon\Delta^{i}\to Z$, and
\begin{equation}
[\alpha]\in H_{i}(Z,X;G).
\end{equation}
Then we can build an $i$-dimensional simplicial complex $K$ with
boundary $L$ and a map
\begin{equation}
\sigma\colon(K,L)\to(Z,X)
\end{equation}
such that $\sigma_{*}([K])=[\alpha]$.
(It may be necessary to replace $(Z,X)$ by a CW pair using the CW
approximation at this point.) Then by the Compression Lemma, $\sigma$
is homotopic $\rel{L}$ to a map
\begin{equation}
\tau\colon(K,L)\to(X,X)
\end{equation}
but
\begin{equation}
0=\tau_{*}([K])=\sigma_{*}([K])=\alpha.
\end{equation}
Hence the claim.
\end{proof}

\subsection{Postinikov Tower}

\begin{node}
For a space $X$, we want to build a sequence of spaces approximating it.
\end{node}

\begin{proposition}
Given a path-connected CW complex $X$. For any $n\geq1$, we can
construct a CW complex $X_{n}$ containing $X$ as a subspace such that
\begin{enumerate}
\item $\pi_{i}(X_{n})=0$ if $i>n$, and
\item The inclusion $i_{n}\colon X\into X_{n}$ induces an isomorphism
  on $\pi_{j}(X)\iso\pi_{j}(X_{n})$ for $j\leq n$.
\end{enumerate}
\end{proposition}

\begin{proof}
It follows from the CW approximation: to get $X_{n}$, consider the
construction applied to $f\colon X\to\point{*}$. Then start attaching
cells of dimension at least $(n+2)$. Then this will kill
$\pi_{i}(X_{n})$ for $i\geq n+1$, and $i_{n}\colon X\into X_{n}$ is an
isomorphism on $\pi_{j}$ for $j\leq n$ as we only attach cells of
dimension strictly greater than $(n+1)$.

By the extension lemma applying to the inclusions
\begin{equation}
\vcenter{\xymatrix{X_{2} \ar@{.>}[dr] & \\
X\ar@{^{(}->}[u]^{j_{2}}\ar@{^{(}->}[r]^{j_{1}} & X_{1}}}
\end{equation}
we have an extension from $X_{2}$ to $X_{1}$ (since the dimension of
cells in $X_{2}\setminus X_{1}$ is $2+2=4$, and $\pi_{i}(X_{1})=0$ for $i\geq2$.)
Then we have a tower by iteratively doing this:
\begin{equation}
\vcenter{\xymatrix{ & \vdots\ar[d] \\
& X_{2}\ar[d] \\
X\ar@{^{(}->}[uur]\ar@{^{(}->}[ur]^{j_{2}}\ar@{^{(}->}[r]^{j_{1}} & X_{1}}}
\end{equation}
This vertical tower of $X_{n}$ is precisely the \define{Postinikov Tower}.
\end{proof}