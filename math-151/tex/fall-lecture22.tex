%%
%% fall-lecture22.tex
%% 
%% Made by Alex Nelson <pqnelson@gmail.com>
%% Login   <alex@lisp>
%% 
%% Started on  2025-11-18T10:57:51-0800
%% Last update 2025-11-18T10:57:51-0800
%% 

\lecture[Cohomology of Spaces]{}

\begin{definition}
Let $X$ be a topological space and let $G$ be an Abelian group.
We have the \define{Cochain Complex} of $X$ be
\begin{equation}
C^{*}(X,G) := \hom(C_{*}(X), G).
\end{equation}
We have the \define{Coboundary Map} $\coboundary\colon C^{*}(X;G)\to C^{*+1}(X;G)$
which is the dual of the boundary map $\boundary$.

More precisely, let $\varphi\in C^{n}(X;G)$. Then $\varphi\colon C_{n}(X)\to G$.
Then $\coboundary\varphi\colon C_{n+1}(X)\to G$. To do this, let
\begin{equation}
\sigma\colon\Delta^{n+1}=[v_{0},\dots,v_{n+1}]\to G
\end{equation}
be a singular simplex, then
\begin{equation}
\coboundary\varphi(\sigma)=\sum^{n+1}_{i=0}(-1)^{i}\varphi(\sigma|_{[v_{0},\dots,\widehat{v_{i}},\dots,v_{n+1}]}),
\end{equation}
where $\sigma|_{[v_{0},\dots,\widehat{v_{i}},\dots,v_{n+1}]}$ is the
face of $\sigma$ without the vertex $v_{i}$.
\end{definition}

\begin{definition}
When we have $C^{n+1}\xleftarrow{\coboundary_{n+1}}C^{n}\xleftarrow{\coboundary_{n}}C^{n-1}$,
then we have the $n^{\text{th}}$ \define{Cohomology Group} of $X$
(with coefficients in $G$) to be $H^{n}(X;G):=\ker(\coboundary_{n+1})/\Im(\coboundary_{n})$.
\end{definition}

\begin{remark}
Thus far, we have define the \emph{singular} cohomology group, the
\emph{singular} cochain complex, the \emph{singular} coboundary
map. All singular stuff.

We can similarly define these gadgets for $\Delta$-complexes. This
gives us a \emph{simplicial} cochain complexes with a \emph{simplicial}
coboundary maps yielding \emph{simplicial} cohomology groups.

We can define analogous gadgets for $CW$-complexes. This gives us
\emph{cellular} cochain complexes with \emph{cellular} coboundary maps
yielding \emph{cellular} cohomology groups.
\end{remark}

\begin{remark}
On thing you can observe is that $C_{*}(X)$ is a free Abelian group,
so we can apply the universal coefficient Theorem to get $H^{*}(C;G)$
from $H_{*}(X)$. So although it \emph{looks} like you don't get
anything new, we will see cohomology has extra structure which makes
it more interesting than homology.
\end{remark}

\subsection{Properties}

\begin{node}
If $A\subset X$ is a subspace, then there exists a short exact sequence
\begin{equation}
0\to C_{*}(A)\to C_{*}(X)\to C_{*}(X,A)\to 0
\end{equation}
which is split, so its dual
\begin{equation}
0\gets C^{*}(A)\gets C^{*}(X)\gets C^{*}(X,A)\gets 0
\end{equation}
is also a split short exact sequence. Hence we obtain the long exact
sequence
\begin{equation}
\vcenter{\xymatrix{H^{*}(A;G)\ar[dr]^{[+1]} & & \ar[ll]H^{*}(X;G)\\
& H^{*}(X,A;G)\ar[ur] &}}
\end{equation}
where not only are the arrows in the opposite direction as in the
homological case, but the grading increases by 1 (instead of decreases
by 1).
\end{node}

\begin{node}[Reduced cochain complex and reduced cohomology]
Recall the reduced chain complex looked like
\begin{equation}
\dots\to C_{1}\to C_{0}\xrightarrow{\varepsilon}\ZZ\to0,
\end{equation}
where $\varepsilon$ is just the sum of points. This is $\widetilde{C}_{*}(X)$.
We can dualize it to get $\widetilde{C}^{*}(X;G)$ which produces the
reduced cohomology $\widetilde{H}^{*}(X;G)$.
\end{node}

\begin{node}
If $f\colon(X,A)\to(Y,B)$ is a continuous map of pairs of spaces, then
there is a chain map $f_{\sharp}\colon C_{*}(X,A)\to C_{*}(Y,B)$,
which then induces the \define{Cochain Map}
\begin{equation}
f^{\sharp}\colon C^{*}(Y,B;G)\to C^{*}(X,A;G),
\end{equation}
and from it we get the induced map
\begin{equation}
f^{*}\colon H^{*}(Y,B;G)\to H^{*}(X,A;G).
\end{equation}
If further $f\homotopic g\colon(X,A)\to(Y,B)$, then $f^{*}=g^{*}$.
\end{node}

\begin{theorem}[Exicision for cohomology]
If $Z\propersubset A\propersubset X$ is such that $\cl(Z)\propersubset\Interior(A)$,
then the inclusion
\begin{equation*}
i\colon(X\setminus Z,A\setminus Z)\into(X,A)
\end{equation*}
induces an isomorphism
\begin{equation*}
i^{*}\colon H^{*}(X,A;G)\xrightarrow{\iso}H^{*}(X\setminus Z,A\setminus Z;G).
\end{equation*}
\end{theorem}

\begin{node}[Mayer--Vietoris pairs]
If $X=\Interior(A)\cup\Interior(B)$, then we have the long exact sequence
\begin{equation}
\vcenter{\xymatrix{
H^{*}(A\cap B;G)\ar[dr]^{[+1]} & & \ar[ll]H^{*}(A;G)\oplus H^{*}(B;G)\\
& H^{*}(X;G)\ar[ur] & }}
\end{equation}
\end{node}

\begin{node}[Eilenberg--Steenrod Axioms for cohomology]
We can write down Eilenberg--Steenrod axioms for cohomology similar to
what we did for homology (\S\ref{subsec:fall-2025:eilenberg-steenrod-axioms-for-homology}).
This uses $h^{*}$ a contravariant functor from the category of pairs
of spaces to the category of graded Abelian groups, and analogous data
(coboundary maps, etc.) satisfying 7 axioms.

There is an analogous uniqueness theorem for cellular cohomology
groups as for cellular homology stated in Theorem~\ref{thm:eilenberg-steenrod:homology:uniqueness}.
\end{node}

\subsection{Examples}

\begin{example}[Spheres]
We recall
\begin{equation}
H_{k}(\sphere{n})\iso\begin{cases}\ZZ & \mbox{if $k=0$ or $k=n$}\\
0 & \mbox{otherwise}
\end{cases}
\end{equation}
So the homology groups for the sphere are free Abelian groups. The
universal coefficient theorem tells us
\begin{equation}
H^{n}(C;G)\iso\hom(H_{n}(C),G)\oplus\Ext(H_{n-1}(C),G).
\end{equation}
But for $H_{n-1}(C)$ being free Abelian groups $\Ext(H_{n-1}(C),G)=0$,
which means
\begin{equation}
H^{n}(C;G)\iso\hom(H_{n}(C),G).
\end{equation}
Then for $G=\ZZ$ we see:
\begin{equation}
H^{k}(\sphere{n};\ZZ)\iso\begin{cases}\ZZ & \mbox{if $k=0$ or $k=n$}\\
0 & \mbox{otherwise}
\end{cases}
\end{equation}
\end{example}

\begin{example}[Real projective plane]
We recall
\begin{equation}
H_{k}(\RP^{2})\iso\begin{cases}
\ZZ & \mbox{if }k=0,\\
\ZZ/2\ZZ & \mbox{if }k=1,\\
0 & \mbox{otherwise}.
\end{cases}
\end{equation}
Then we can use the universal coefficient theorem for the
cohomology. The torsion part shifts up by 1,
\begin{equation}
H^{k}(\RP^{2})\iso\begin{cases}
\ZZ & \mbox{if }k=0,\\
\ZZ/2\ZZ & \mbox{if }k=2,\\
0 & \mbox{otherwise}.
\end{cases}
\end{equation}
But we can also compute it explicitly with $G=\ZZ/2\ZZ$.
We find
\begin{subequations}
  \begin{align}
H^{1}(\RP^{2};\ZZ/2\ZZ) &\iso\hom(H_{1}(\RP^{2}),\ZZ/2\ZZ)\oplus\Ext(H_{0}(\RP^{2}),\ZZ/2\ZZ)\\
&\iso\hom(\ZZ/2\ZZ,\ZZ/2\ZZ)\oplus0\mbox{ since $H_{0}(\RP^{2})$ is free}\\
&\iso\ZZ/2\ZZ,
  \end{align}
\end{subequations}
and
\begin{subequations}
  \begin{align}
H^{2}(\RP^{2};\ZZ/2\ZZ) &\iso\hom(H_{2}(\RP^{2}),\ZZ/2\ZZ)\oplus\Ext(H_{1}(\RP^{2}),\ZZ/2\ZZ)\\
&\iso 0\oplus\Ext(H_{1}(\RP^{2}),\ZZ/2\ZZ)\\
&\iso\ZZ/2\ZZ,
  \end{align}
\end{subequations}
giving us
\begin{equation}
H^{k}(\RP^{2},\ZZ/2\ZZ)\iso\begin{cases}\ZZ/2\ZZ & \mbox{if }k=0,1,2\\
0 & \mbox{otherwise}
\end{cases}
\end{equation}
This is the singular cohomology for $\RP^{2}$.

We can also compute this using the cellular chain complex for
$\RP^{2}$ which is
\begin{equation}
C_{2}=\ZZ\xrightarrow{2} C_{1}=\ZZ\xrightarrow{0}C_{0}=\ZZ\to0,
\end{equation}
where the morphisms multiply by 2 or 0. We dualize it with
coefficients $G=\ZZ$, then we get
\begin{equation}
C^{2}=\ZZ\xleftarrow{2}C^{1}=\ZZ\xleftarrow{0}C^{0}=\ZZ\gets0.
\end{equation}
From this, it's not too difficult to compute $H^{k}(\RP^{2};\ZZ)$.
If we dualize the chain complex with coefficients in $G=\ZZ/2\ZZ$,
then we get
\begin{equation}
C^{2}=\ZZ/2\ZZ\xleftarrow{0}C^{1}=\ZZ/2\ZZ\xleftarrow{0}C^{0}=\ZZ/2\ZZ\gets0.
\end{equation}
\end{example}

\subsection{Cup product and cap product}

\begin{definition}
Suppose the coefficients for the cochain complex are in a commutative
ring $R$. For cochains $\varphi\in C^{k}(X;R)$ and $\psi\in C^{\ell}(X;R)$,
we define the \define{Cup Product} $\varphi\smile\psi\in C^{k+\ell}(X;R)$
by defining its evaluation on every $(k+\ell)$-singular simplex
$\sigma\colon\Delta^{k+\ell}=[v_{0},\dots,v_{k+\ell}]\to X$ as
\begin{equation}
(\varphi\smile\psi)(\sigma)=\varphi(\sigma|_{[v_{0},\dots,v_{k}]}\psi(\sigma|_{[v_{k},\dots,v_{k+\ell}]})
\end{equation}
where on the right-hand side of this equation, we use the
multiplication of ring elements.
\end{definition}

\begin{lemma}[Leibniz rule]
If $\varphi\in C^{k}(X;R)$ and $\psi\in C^{\ell}(X;R)$, then
\begin{equation}
\coboundary(\varphi\smile\psi)=(\coboundary\varphi)\smile\psi+(-1)^{k}\varphi\smile(\coboundary\psi).
\end{equation}
\end{lemma}

\begin{proof}
Evaluate on $\sigma\colon\Delta^{k+\ell}=[v_{0},\dots,v_{k+\ell}]\to X$.
Then
\begin{subequations}
  \begin{align}
((\coboundary\varphi)\smile\psi)(\sigma)
&= (\coboundary\varphi)(\sigma|_{[v_{0},\dots,v_{k+1}]})(\psi(\sigma|_{[v_{k+1},\dots,v_{k+\ell}]}))\\
&= \left(\sum^{k+1}_{i=0}(-1)^{i}\varphi(\sigma|_{[v_{0},\dots,\widehat{v_{i}},\dots,v_{k+1}]})\right)(\psi(\sigma|_{[v_{k+1},\dots,v_{k+\ell}]})),
  \end{align}
\end{subequations}
and similarly
\begin{subequations}
  \begin{align}
(\varphi\smile(\coboundary\psi))(\sigma)
&= (\varphi(\sigma|_{[v_{0},\dots,v_{k}]}))((\coboundary\psi)(\sigma|_{[v_{k},\dots,v_{k+\ell}]}))\\
&= (\varphi(\sigma|_{[v_{0},\dots,v_{k}]}))\left(\sum^{k+\ell+1}_{i=k}(-1)^{i-k}\psi(\sigma|_{[v_{k},\dots,\widehat{v_{i}},\dots,v_{k+\ell}]})\right).
  \end{align}
\end{subequations}
Then we see that the last term of
$(\coboundary\varphi\smile\psi)(\sigma)$ cancels with the first term
of $(-1)^{k}(\varphi\smile\coboundary\psi)(\sigma)$, so we're left
with $k+\ell+1$ terms, and it's a trivial exercise to see
$\coboundary(\varphi\smile\psi)=(\coboundary\varphi)\smile\psi+(-1)^{k}\varphi\smile(\coboundary\psi)$.
\end{proof}

\begin{corollary}
The cup product of 2 cocylces is a cocycle.

The cup product of a cocycle with a coboundary is a coboundary.

The cup product of a coboundary with a cocycle is a coboundary.
\end{corollary}

(A cocycle $\varphi$ has $\coboundary\varphi=0$, and a coboundary
$\varphi$ has a $\psi$ such that $\varphi=\coboundary\psi$.)

\begin{node}
From this corollary, we see that the cup product $\smile$ on the
cochain complex $C^{*}$ induces a cup product $H^{*}$. Or more
precisely,
\begin{equation}
H^{k}(X;R)\times H^{\ell}(X;R)\xrightarrow{\smile}H^{k+\ell}(X;R)
\end{equation}
such that $[\varphi]\smile[\psi]=(-1)^{k\ell}[\psi]\smile[\varphi]$
and under this product $H^{*}(X;R)$ becomes a graded ring.
\end{node}