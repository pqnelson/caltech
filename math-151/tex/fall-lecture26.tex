%%
%% fall-lecture26.tex
%% 
%% Made by Alex Nelson <pqnelson@gmail.com>
%% Login   <alex@lisp>
%% 
%% Started on  2025-11-27T11:04:48-0800
%% Last update 2025-11-27T11:04:48-0800
%% 

\lecture{}

\begin{node}
If we have any manifold $M$, then we may form its cover
\begin{equation}
\widetilde{M}=\{(x,\mu_{x})\mid x\in M,\mu_{x}\mbox{ is a local orientation at x}\}
\end{equation}
with $p\colon\widetilde{M}\to M$ which is a 2-to-1 map. We may give
$\widetilde{M}$ a topology such that $p$ is continuous. In fact, $p$
is a covering map.
\end{node}

\begin{node}
We can take a local orientation $\widetilde{\mu}_{(x,\mu(x))}$ such
that 
\begin{equation}
p_{*}(\widetilde{\mu}_{(x,\mu(x))})=\mu(x)
\end{equation}
which is an orientation of $\widetilde{M}$. This is a canonical local
orientation for $\widetilde{M}$, so $\widetilde{M}$ is orientable.
\end{node}

\begin{definition}
This $\widetilde{M}$ thus described is called the \define{Orientable Double Cover} of $M$.
\end{definition}

\begin{proposition}
Let $M$ be connected. Then $\widetilde{M}$ has 2 components iff $M$ is orientable.
\end{proposition}

\begin{proof}
\backwardproof\ If $M$ is orientable, then $\widetilde{M}$ consists of
2 copies of $M$ due to connectedness.

\forwardproof\ By contrapositive. If $M$ is not orientable, then
$\widetilde{M}$ is connected.
\end{proof}

\begin{node}
When $M$ is connected, we can fix a base-point $x_{0}\in M$. Then we
can consider a loop $C$ with base-pont $x_{0}$. We can consider how
the local orientation varies continuously around the loop $C$
(represented by the frames doodled in red):
\begin{equation*}
\includegraphics{img/img.51}
\end{equation*}
At the end, we get a new local orientation at $x_{0}$. If the new
local orientation agrees with the initial local orientation, then we
call $C$ \define{Orientation-Preserving}. Otherwise we call $C$ \define{Orientation-Reversing}.
\end{node}

\begin{node}
We can define a map $\phi\colon\pi_{1}(M,x_{0})\to\ZZ/2\ZZ$ such that
orientation-preserving loops $C$ are mapped to $\phi([C])=0$, but
orientation-reversing loops $C'$ are mapped to $\phi([C'])=1$.

Further, this is a group morphism iff $M$ is nonorientable.
\end{node}

\begin{node}
We see that $\widetilde{M}$ is the 2-fold cover of $M$ corresponding
to $\ker(\phi)$.
\end{node}

\begin{node}
We want to generalize the notion of ``orientation''.
\end{node}

\begin{definition}
Let $R$ be a unital commutative ring. Then an \define{$R$-Orientation}
of $M$ is an assignment of an invertible element $\mu_{x}\in H_{n}(M,M\setminus\{x\};R)\iso R$
to each $x\in M$ such that it ``varies continuously''.
\end{definition}

\begin{proposition}
Every manifold is $(\ZZ/2\ZZ)$-orientable.
\end{proposition}

(This is because there's exactly one invertible element of $\ZZ/2\ZZ$.)

Now, the next ``big thing'' we want to prove requires a lemma.

\begin{lemma}\label{lemma:fall2025:lecture26:math151:inductive-step}
Let $M$ be an $n$-manifold, let $K\subset M$ be compact. Let
$\rho\colon K\into M$ be the inclusion map.
\begin{enumerate}
\item $H_{i}(M,M\setminus K)=0$ if $i>n$
\item A homology class $\alpha\in H_{n}(M,M\setminus K)$ is zero if
  and only if
  for all $x\in K$, $(\rho_{x})_{*}(\alpha)=0\in H_{n}(M,M\setminus\{x\})$ where 
  we recall $H_{n}(M,M\setminus\{x\})\iso\ZZ$.
\end{enumerate}
\end{lemma}

\begin{proof}
We claim if the Lemma is true for $K_{1}$, $K_{2}$, and $K_{1}\cap K_{2}$,
then it is true for $K_{1}\cup K_{2}$ (and by induction for $K$). To
see this, we use the Mayer--Vietoris sequence
\begin{equation}
\vcenter{\xymatrix{\dots\ar[r] & H_{i}(M,K_{1}\setminus K_{2})\ar[r] & H_{i}(M,M\setminus K_{1})\oplus
H_{i}(M,M\setminus K_{2})\ar[r] & H_{i}(M,M\setminus K_{1}\cap K_{2})\ar[dll]\\
& H_{i-1}(M,K_{1}\setminus K_{2})\ar[r] & \dots &}}
\end{equation}
The first statement is obviously true, the second statement we look at
$i=n$. The forward proof is easy. The backwards proof: we have
$0\mapsto\alpha\mapsto0\oplus0$ which proves the claim.
\end{proof}

\begin{theorem}[Hatcher 3.26]
\begin{enumerate}
\item If $M$ is a closed, connected, orientable $n$-manifold, then $H_{n}(M)\iso\ZZ$;
\item If $M$ is a connected manifold but either nonorientable or
  noncompact, then $H_{n}(M)=0$.
\end{enumerate}
\end{theorem}

\begin{proof}
Let $K\subset M$ be compact, let $x\in K$, let
$\rho_{x}\colon(M,M\setminus K)\to(M,M\setminus\{x\})$ be the inclusion map.
We will prove this by cases, in a sort of induction procedure.

\textsc{Case 1: $M=\RR^{n}$, $K$ is convex.} 
Then $(M,M\setminus K)\homotopic(M,M\setminus\{x\})$ homotopic. So $\rho_{x}\colon H_{*}(M,M\setminus K)\to H_{*}(M,M\setminus\{x\})$
is an isomorphism. Hence the result for this case.

\textsc{Case 2: $M=\RR^{n}$, $K$ is union of finitely many convex compact subsets.}
When $K=K_{1}\cup \dots\cup K_{m}$, we can inductively apply Lemma~\ref{lemma:fall2025:lecture26:math151:inductive-step}
and the previous case to the $K_{i}$ and we're done.

\textsc{Case 3: $M=\RR^{n}$, $K$ compact.}
Let
\begin{equation}
N = \{x\in\RR^{n}\mid d(x,K)\leq\varepsilon\},
\end{equation}
so we effectively have added an $\varepsilon$ amount of ``wiggle
room'' to $K$. Obviously $N$ is compact. We can cover $K$ by finitely
many convex sets in $N$---call them $K_{1}$, \dots, $K_{m}$. Then we
have a sequence
\begin{equation}
H_{n}(M,M\setminus N)\to H_{n}(M,M\setminus\bigcup_{i}K_{i})\to
H_{n}(M,M\setminus K)\to H_{n}(M,M\setminus\{x\}).
\end{equation}
Let $\alpha$ (a homology class) be represented by a singular chain $a$.
So $a\in C_{n}(M,M\setminus K)$ is a cycle, i.e., $\boundary a$ has an
image in $M\setminus K$ (but the image is compact), so we can choose
$\varepsilon$ small enough such that the image is in $M\setminus N$.

We want to say $\rho_{x}(a)$ for every $x\in K_{1}\cup\dots\cup K_{m}$.
Then we just consider the two possible subcases:

\textsc{Subcase 1:} If $x\in K$, then it is true by assumption

\textsc{Subcase 2:} If $x\notin K$, then we can find a chain $b$ representing $\alpha$
  with image disjoint from $x$, so $\rho_{x}(\alpha)=0$. Then we apply
  the previous subcase.

\textsc{Case 4: general case.} We have $K=K_{1}\cup\dots\cup K_{m}$
such that for each $i$ there is a $U_{i}\subset M$ open and locally
Euclidean $U_{i}\iso\RR^{n}$ such that $K_{i}\subset U_{i}\iso\RR^{n}$.
Then we can prove our theorem by induction on $m$. Each case is like
one of the 3 preceding cases. Hence the result.
\end{proof}

\begin{theorem}\label{thm:math151a:fall2025:lec26:thm-star}
Let $M$ be an oriented $n$-manifold. Let $K\subset M$ compact.
Then there exists a unique $\mu_{K}\in H_{n}(M,M\setminus K)$ such that
for each $x\in K$ we have $\rho_{x}(\mu_{K})=\mu_{x}$.
\end{theorem}

\begin{proof}
Uniqueness follows from Lemma~\ref{lemma:fall2025:lecture26:math151:inductive-step}:
if we had 2 different $\mu_{K}$ and $\mu'_{K}$, then their difference
$\mu_{K}-\mu'_{K}=0$, and the lemma gives us everything.

Existence: induction. If we have existence for $K_{1}$, $K_{2}$, and
$K_{1}\cap K_{2}$, then we look at the long exact sequences, we see
how it behaves on this portion:
\begin{equation}
\begin{array}{ccccl}
H_{n}(M,M\setminus K_{1})&\oplus&H_{n}(M,M\setminus K_{2})&\to&H_{n}(M,M\setminus K_{1}\cap K_{2})\\
\mu_{K_{1}} & & \mu_{K_{2}} & \mapsto & \rho_{x}(j_{K_{1}}(\mu_{K_{1}})-j_{K_{2}}(\mu_{K_{2}}))=0.
\end{array}
\end{equation}
Then there is a preimage $\mu_{K_{1}\cup K_{2}}\in H_{n}(M,M\setminus K_{1}\cup K_{2})$.
Then by induction o nthe number of charts covering $K$ (which must be
finite since it's compact) we get the result.
\end{proof}