%%
%% winter-lecture20.tex
%% 
%% Made by Alex Nelson <pqnelson@gmail.com>
%% Login   <alex@lisp>
%% 
%% Started on  2026-02-19T09:54:20-0800
%% Last update 2026-02-19T09:54:20-0800
%% 

\lecture{}

\begin{example}
If $f\colon E\to B$ is a fibration, then there is a fiber homopy
equivalence
\begin{equation}
\vcenter{\xymatrix{E\,\ar[dr]_{f} \ar@{^{(}->}[rr]^{i} & & E_{f}\ar[dl]_{p}\\
& B &}}
\end{equation}
where $i$ is a homotopy equivalence and the ``actual fiber''
$f^{-1}(b_{0})$ is homotopy equivalent to the ``homotopy fiber'' $F_{f}=p^{-1}(b_{0})$.
This requires justification. It follows from the Five-Lemma applied to
the 2 long exact sequences of a fibration. There isa map
$i\colon(E,F)\to(E_{f},F_{f})$ which gives us
\begin{equation}
\vcenter{\xymatrix{
\dots\ar[r] & \pi_{n+1}(E)\ar[r]\ar[d]^{i_{*}\iso} & \pi_{n+1}(B)\ar[r]\ar[d]^{i_{*}=\id} & \pi_{n}(F)\ar[r]\ar[d]^{i_{*}}& \pi_{n}(E)\ar[r]\ar[d]^{i_{*}\iso} & \pi_{n}(B)\ar[r]\ar[d]^{i_{*}=\id} & \dots\\
\dots\ar[r] & \pi_{n+1}(E_{f})\ar[r] & \pi_{n+1}(B)\ar[r] & \pi_{n}(F_{f})\ar[r]& \pi_{n}(E_{f})\ar[r] & \pi_{n}(B)\ar[r] & \dots
}}
\end{equation}
Then by Whitehead's theorem, $F\homotopic F_{f}$.
\end{example}

\begin{node}
Cofibration is ``inclusion'' with ``good properties''. Another way to
think of fibration is the following slogan: ``It is a homotopic
version of fiber bundles''. More precisely:
\end{node}

\begin{proposition}
For a fibration $p\colon E\to B$, the fibers $F_{b}=p^{-1}(b)$ over
each path-connected component of $B$ are homotopy equivalent.
\end{proposition}

\begin{proof}
Choose $b_{0}$, $b_{1}$ in the same path-connected component of
$B$. We want to show $F_{b_{0}}\homotopic F_{b_{1}}$. Choose a path
$\gamma\colon I\to B$ from $b_{0}$ to $b_{1}$. Then we get a homotopy
\begin{equation}
\begin{split}
G\colon&F_{b_{0}}\times I\to B\\
&(a,t)\mapsto\gamma(t)
\end{split}
\end{equation}
(or $g_{t}\colon F_{b_{0}}\to B$ sending $a\mapsto\gamma(t)$). Then
there is a lift
\begin{equation}
\widetilde{g}_{0}\colon F_{b_{0}}\to E,
\end{equation}
given by the inclusion. Then by the Homotopy Lifting Property with
respect to $F_{b_{0}}$ (\S\ref{defn:homotopy-lifting-property}), we
get a lift $\widetilde{g}_{t}$ of $g_{t}$ extending $\widetilde{g}_{0}$.
Then
\begin{equation}
p(\widetilde{g}_{1})(a)=g_{1}(a),
\end{equation}
which is $b_{1}$ for all $a\in F_{b_{1}}$. That is to say,
$\widetilde{g}_{1}$ is a map from $F_{b_{0}}$ to $F_{b_{1}}$. Denote
this map by
\begin{equation}
L_{\gamma}\colon F_{b_{0}}\to F_{b_{1}}.
\end{equation}
The map $L_{\gamma}$ satisfies the following properties:
\begin{enumerate}
\item If $\gamma\homotopic_{\rel{I}}\gamma'$, then
  $L_{\gamma}\homotopic L_{\gamma'}$;
\item $L_{\gamma\bullet\gamma'}\homotopic L_{\gamma'}\circ L_{\gamma}$
  where $\gamma\bullet\gamma'$ is concatenation of paths (do $\gamma$
  first, then do $\gamma'$).
\end{enumerate}
Assuming these properties, choose $\gamma$ to bea path from $b_{0}$ to
$b_{1}$ and choose $\overline{\gamma}$ to be the reverse map. Then
\begin{equation}
\gamma\bullet\overline{\gamma}\homotopic\mbox{const}_{b_{0}}
\end{equation}
and
\begin{equation}
\overline{\gamma}\bullet\gamma\homotopic\mbox{const}_{b_{1}}.
\end{equation}
So
\begin{equation}
L_{\gamma\bullet\overline{\gamma}}\homotopic
L_{\overline{\gamma}}\circ L_{\gamma}\homotopic L_{\mbox{const}_{b_{0}}}\homotopic\id_{F_{b_{0}}}.
\end{equation}
Similarly,
\begin{equation}
L_{\gamma}\circ L_{\overline{\gamma}}\homotopic\id_{F_{b_{1}}}.
\end{equation}
Hence the claim, provided we accept these properties stipulated of $L_{\gamma}$.
\end{proof}

\begin{claim}
If $\gamma\homotopic_{\rel{I}}\gamma'$, then
$L_{\gamma}\homotopic L_{\gamma'}$.
\end{claim}

\begin{proof}
Let $h(s,t)$ be the homotopy between $h(0,t)=g_{\gamma}(t)$ and
$h(1,t)=g_{\gamma'}(t)$ where
\begin{equation}
h(-,-)\colon F_{b_{0}}\times I\times I\to B.
\end{equation}
We have partial lifting of $h(s,t)$ over $L_{\gamma}=F_{b}\times I\times\{0\}$
and $F_{b_{0}}\times I\times\{1\}=L_{\gamma'}$ and
$F_{b}\times\{0\}\times I$ is the inclusion $F_{b_{0}}\into B$.
We have
\begin{equation}
\vcenter{\hbox{\includegraphics{img/img.65}}}\quad\homotopic\quad\vcenter{\hbox{\includegraphics{img/img.66}}}
\end{equation}
Then we apply the homotopy lifting property to $F_{b_{0}}\times I$ to
get a lift
\begin{equation}
\widetilde{h}(s,t)\colon F_{b_{0}}\times I\times I\to E,
\end{equation}
extending the partial lift. In particular, restricting to $t=1$, we
get a homotopy from $L_{\gamma}=\widetilde{h}(0,1)$
to $L_{\gamma'}=\widetilde{h}(1,1)$.
\end{proof}

\begin{claim}
$L_{\gamma\bullet\gamma'}\homotopic L_{\gamma'}\circ L_{\gamma}$
  where $\gamma\bullet\gamma'$ is concatenation of paths (do $\gamma$
  first, then do $\gamma'$).
\end{claim}

\begin{proof}
Reparametrizing and lifting consecutively, you see
$\widetilde{g}_{\gamma'}\widetilde{g}_{\gamma}$ is a lifting of $g_{\gamma\bullet\gamma'}$.
\end{proof}

\subsection{Pullback fibration}

\begin{definition}
Given a fibration $p\colon E\to B$ and a map $f\colon A\to B$, the
\define{Pullback Fibration} of $f^{*}p$ is given by
\begin{equation}
p^{*}\colon f^{*}(E)\to A,
\end{equation}
where
\begin{equation}
f^{*}(E):=\{(a,e)\in A\times E\mid f(a)=p(e)\},
\end{equation}
and $p^{*}(a,e)=a$; i.e.,
\begin{equation}
\vcenter{\xymatrix{f^{*}E\ar[r]\ar[d]^{p^{*}}\ar@{}[dr]|>{\txt{$\ulcorner$}} & E\ar[d]^{p}\\
A\ar[r]^{f} &  B}}
\end{equation}
\end{definition}

\begin{xca}
Check $p^{*}\colon f^{*}E\to A$ is a fibration (i.e., satisfies the
homotopy lifting property).
\end{xca}

\begin{node}[Fibration sequence]
We start with a fibration $p\colon E\to B$. We get a sequence by
repeatedly taking the homotopy fiber; it takes the form
\begin{equation}
\dots\to\Omega^{2}B\to\Omega F\to\Omega E\xrightarrow{\Omega p}\Omega B\to
F\into E\xrightarrow{p}B.
\end{equation}
This sequence has the following property:
For any $X$, we have the following long exact sequence
\begin{equation}\label{eq:math151b:lecture20:long-exact-sequence:fibration-sequence}
\dots\to\langle X,\Omega B\rangle\to\langle X,F\rangle\xrightarrow{i_{*}}\langle
X,E\rangle\xrightarrow{p_{*}}\langle X,B\rangle.
\end{equation}
It may be useful to compare this to the cofibration sequence,
\begin{equation}
A\into B\to A/B\to\Sigma A\to\Sigma B\to\Sigma(A/B)\to\Sigma^{2}A\to\dots.
\end{equation}
In particular, if you take $X=\sphere{0}$, this gives you the long
exact sequence [of the homotopy groups] of the fibrations.
\end{node}

\begin{construction}
The idea is to look at every 3 consecutive terms in the sequence in
Equation~\eqref{eq:math151b:lecture20:long-exact-sequence:fibration-sequence},
we want to find
\begin{equation}
\Omega B\xrightarrow{?}F\to E.
\end{equation}
The trick is to use
\begin{equation}
\dots\to F_{j}\to F_{i}\xrightarrow{j}F_{f}\xrightarrow{i}E_{f}\to B
\end{equation}
where $i\colon F_{f}\to E_{f}$ is also a fibration. We just keep
iterating ``take homotopy fiber of previous map'', then we show
$\Omega B\homotopic F_{i}$, and then $\Omega E\homotopic F_{j}$,
and so on. Take $F_{p}$ the homotopy fiber of $p\colon E\to B$.
We need to show $F_{p}\homotopic F$ and $i\colon F_{p}\into E$ is a
fibration.

Then take the homotopy fiber $F_{i}$ of $i\colon F_{p}\to E$ and then
we need to show $j\colon F_{i}\into F_{p}$ is a fibration and
$F_{i}\homotopic\Omega B$. Then take the homotopy fiber of $F_{j}$ and
we need to show $F_{j}\homotopic\Omega E$. Then repeat.

Recall
\begin{equation}
F_{p}=\{(e,\gamma)\in E\times B^{I}\mid\gamma(0)=p(e),\gamma(1)=b_{0}\}
\end{equation}
where $b_{0}\in B$ is the basepoint of $B$. We define
\begin{equation}
\begin{split}
i\colon&F_{p}\into E\\
&(e,\gamma)\mapsto e
\end{split}
\end{equation}
just projects out the first component. Then $F_{p}$ is the pullback
fibration of
\begin{equation}
\vcenter{\xymatrix{F_{p}\ar[d]\ar[r] & PB\ar[d] & \gamma\ar@{|->}[d]\\
E\ar[r] & B & \gamma(0)}}
\end{equation}
where
\begin{equation}
PB=\{\gamma\in B^{I}\mid \gamma(1)=b_{0}\}.
\end{equation}
Because $p\colon E\to B$ is a fibration, we have $F\homotopic F_{p}$.
Then we take the homotopy fiber $F_{i}$ of $F_{p}\xrightarrow{i}E$. 
We see that
\begin{equation}
F_{i}=\{\bigl((e,\gamma),\eta\bigr)\in F_{p}\times E^{I}\mid i(e,\gamma)=e=\eta(0),\eta(1)=e_{0},\gamma(0)=p(e),\gamma(1)=b_{0}\},
\end{equation}
where $e_{0}$ is the basepoint of $E$. We can recover $e$ by looking
at $\eta(0)$, so this is kind of redundant, and the condition comes
from $F_{p}$. That is to say,
\begin{equation}
F_{i}\iso\{(\gamma,\eta)\in B^{I}\times E^{I}\mid\eta(1)=e_{0},\gamma(1)=b_{0},\gamma(0)=p(\eta(0))\},
\end{equation}
and the map $j\colon F_{i}\to F_{p}$ is given by sending $(\gamma,\eta)\mapsto(e=\eta(0),\gamma)$.

The proof $j$ is a fibration is the same as before.

We need to show $\Omega B\homotopic F_{i}$. The actual fiber of the map
$i\colon F_{p}\to E$ is given by loops in $B$ based at $b_{0}$ where
\begin{equation}
\begin{split}
F_{p}\to E\\
(e,\gamma)\mapsto e.
\end{split}
\end{equation}
Then since $i\colon F_{p}\to E$ is a fibration, then its actual fiber
$\Omega B$ is homotopic to the homotopy fiber $F_{i}$.

Then we take the homotopy fiber $F_{j}$ of $j\colon F_{i}\to F_{p}$
and we need to show the actual fiber of $j$ is $\Omega E$. The actual
fiber of $j$ is given by
\begin{subequations}
  \begin{align}
j^{-1}\bigl((e_{0},\mbox{const}_{b_{0}})\bigr)
&=\{(\gamma,\eta)\in B^{I}\times E^{I}\mid\eta(1)=e_{0},\gamma(1)=b_{0},\eta(0)=e_{0},\gamma=\mbox{const}_{b_{0}}\}\\
&=\{\eta\in B^{I}\mid\eta(0)=\eta(1)=e_{0}\}\\
&=\Omega E.
  \end{align}
\end{subequations}
\end{construction}