%%
%% fall-lecture12.tex
%% 
%% Made by Alex Nelson <pqnelson@gmail.com>
%% Login   <alex@lisp>
%% 
%% Started on  2025-10-25T11:06:30-0700
%% Last update 2025-10-25T11:06:30-0700
%% 

\lecture[Degree of a map $\sphere{n}\to\sphere{n}$]{}

\begin{definition}
Recall $H_{n}(\sphere{n})\iso\ZZ$.
Then the induced map of $f\colon\sphere{n}\to\sphere{n}$ on the homology
$f_{*}\colon H_{n}(\sphere{n})\to H_{n}(\sphere{n})$ is the
multiplication by an integer $k$. We define the \define{Degree} of $f$
to be the integer denoted $\deg(f)=k$.
\end{definition}

\begin{node}[Strategy for computing degree of map]
Recall, $\sphere{n}=\disk{n}_{-}\cup\disk{n}_{+}$ is the union of two
disks (one for the Northern hemisphere, the other for the Southern
hemisphere). Let $\sigma_{1}\colon\Delta^{n}\to\disk{n}_{+}$ and then let
$\sigma_{2}\colon\Delta^{n}\to\disk{n}_{-}$ be the mirror image of
$\sigma_{1}$ across the equator $\sphere{n-1}$. Then
$\boundary\sigma_{1}=\boundary\sigma_{2}$, so
$\boundary\sigma_{1}-\boundary\sigma_{2}=0$, and therefore
$\sigma_{1}-\sigma_{2}$ is a cycle. This means $[\sigma_{1}-\sigma_{2}]$
is a generator of $H_{n}(\sphere{n})$. Thus we have obtained a
generator of $H_{n}(\sphere{n})$, and when computing the degree of $f$
we just need to consider what happens to this cycle.
\end{node}

\begin{example}[Degree of constant map]
Let $f\colon\sphere{n}\to\sphere{n}$ be a constant map. Then
\begin{subequations}
\begin{align}
f_{*}[\sigma_{1}-\sigma_{2}] &= [f\circ\sigma_{1}-f\circ\sigma_{2}]\\
&= 0
\end{align}
\end{subequations}
since $f$ is constant. Hence $\deg(f)=0$.
\end{example}

\begin{example}[Identity map]
Consider $f=\id\colon\sphere{n}\to\sphere{n}$. Then
\begin{equation}
f_{*}[\sigma_{1}-\sigma_{2}]=[\sigma_{1}-\sigma_{2}],
\end{equation}
which means $\deg(\id)=1$.
\end{example}

\begin{example}
Consider the unit circle in the complex plane $\sphere{1}\subset\CC$.
Consider $f\colon\sphere{1}\to\sphere{1}$ defined by $f(z)=z^{k}$ for
some fixed $k\in\ZZ$. What is the degree of $f$?

Let us consider $H_{1}(\sphere{1})$ generated by $\sigma\colon[0,1]\to\sphere{1}$,
where
\begin{equation}
\sigma(t)=\exp(\I 2\pi t).
\end{equation}
Then $[\sigma]$ generates $H_{1}(\sphere{1})$.
We see
\begin{equation}
(f\circ\sigma)(t)=\exp(k\cdot\I2\pi t)
\end{equation}
and then $f_{*}(\sigma)$ sends $t$ to $\exp(k\I2\pi t)$. Hence $\deg(f)=k$.
\end{example}

\begin{example}[Reflection]
Consider $\rho\colon\sphere{n}\to\sphere{n}$ given by reflection
across the equator. What is $\deg(\rho)$?

We see its induced morphism on the homology group
\begin{subequations}
\begin{align}
\rho_{*}[\sigma_{1}-\sigma_{2}] &= [\rho\circ\sigma_{1}-\rho\circ\sigma_{2}]\\
&= [\sigma_{2}-\sigma_{1}]\\
&= -[\sigma_{1}-\sigma_{2}],
\end{align}
\end{subequations}
hence $\deg(\rho)=-1$.
\end{example}

\begin{example}[Antipodal map]
Let $f\colon\sphere{n}\to\sphere{n}$ be the antipodal map. What is this?

We think of $\sphere{n}\subset\RR^{n+1}$ as the unit sphere, and
$f\colon\RR^{n+1}\to\RR^{n+1}$ sends
\begin{equation}
f(x_{1},\dots,x_{n+1})=(-x_{1},\dots,-x_{n+1}).
\end{equation}
Then $f$ acting on $\sphere{n}$ is the antipodal map.

We claim if $\sphere{n}\xrightarrow{g}\sphere{n}\xrightarrow{f}\sphere{n}$,
then $\deg(f\circ g)=\deg(f)\deg(g)$. Then we see that
\begin{equation}
(f\circ g)_{*}=f_{*}\circ g_{*}\colon H_{n}(\sphere{n})\to H_{n}(\sphere{n}),
\end{equation}
so the degree of the left-hand side is the product of degrees.

Now we can deduce the degree of the antipodal map, which is just the
degree of $(n+1)$ reflections. But we just saw the degree of a single
reflection is $-1$, so we conclude $\deg(f)=(-1)^{n+1}$.
\end{example}

\begin{fact}
If $f\homotopic g$ homotopic, then $\deg(f)=\deg(g)$.
\end{fact}

\begin{corollary}
If $f(x)\neq-g(x)$ for all $x\in\sphere{n}$, then $\deg(f)=\deg(g)$.
\end{corollary}

\begin{proof}
We define a homotopy
\begin{equation}
F_{t}(x) = \frac{(1-t)f(x)+tg(x)}{\|(1-t)f(x)+tg(x)\|},
\end{equation}
and we claim this is a well-defined homotopy since $F_{1}=g$ and
$F_{0}=f$ and the denominator is never zero $(1-t)f(x)+tg(x)\neq0$ for
all $x$ --- otherwise $\|f(x)\|=\|g(x)\|=1$ which implies $t=1/2$
which then means $f(x)=-g(x)$. But this contradicts our hypothesis
that $f(x)\neq-g(x)$ for all $x\in\sphere{n}$. Hence the result (using
our handy fact).
\end{proof}

\begin{corollary}
If $f$ has no fixed point, then $\deg(f)=(-1)^{n+1}$.
\end{corollary}

\begin{theorem}
The sphere $\sphere{n}$ has a continuous nonvanishing vector field if
and only if $n$ is odd.
\end{theorem}

\begin{proof}
We can think of a continuous nonvanishing vector field as a mapping
\begin{equation}
\vec{v}\colon\sphere{n}\to\RR^{n+1}\setminus\{0\}
\end{equation}
such that $\vec{v}(x)\perp x$ for all $x\in\sphere{n}$, when we view
$\sphere{n}$ as the unit sphere in $\RR^{n+1}$. The vectors must be
tangent to the sphere. If $n$ is odd,
\begin{equation}
\vec{v}(x_{1},x_{2},\dots,x_{n+1})=(-x_{2},x_{1},-x_{4},x_{3},\dots,-x_{n+1},x_{n})
\end{equation}
defines the vector field and it is obvious non-vanishing.

If $n$ is even. Assume we have such a $\vec{v}$. We can define
\begin{equation}
f(x) = \frac{x+\vec{v}(x)}{\|x+\vec{v}(x)\|}.
\end{equation}
Then
\begin{equation}
F_{t}(x) = \frac{x+t\vec{v}(x)}{\|x+t\vec{v}(x)\|}
\end{equation}
is a homotopy between $F_{0}=\id$ and $F_{1}=f$. So $\deg(f)=1$. Since
$\vec{v}(x)\neq0$ for all $x\in\sphere{n}$, then $f(x)\neq x$ for any $x\in\sphere{n}$.
Then $f$ has no fixed-point.
So $\deg(f)=(-1)^{n+1}=-1$ if $n$ is even.
Then we get our contradiction.
\end{proof}
% TODO: draw picture for this proof

\begin{theorem}[Hopf]
Two maps $f,g\colon\sphere{n}\to\sphere{n}$ are homotopic if and only if $\deg(f)=\deg(g)$.
\end{theorem}

\begin{definition}[Local degree of map]
Let $f\colon\sphere{n}\to\sphere{n}$. Let $x\in\sphere{n}$.
Suppose $y=f(x)$. Let $x\in U$ and $y\in V$ be open neighborhoods in $\sphere{n}$.
Recall
\begin{equation}
\begin{array}{rcl}
H_{n}(U,U\setminus\{x\}) & \iso & H_{n}(V,V\setminus\{y\})\\
\rotatebox{-90}{$\!\iso\;$} & & \rotatebox{90}{$\!\!\iso$}\\
H_{n}(\sphere{n},\sphere{n}\setminus\{x\}) & & H_{n}(\sphere{n},\sphere{n}\setminus\{y\})\\
\rotatebox{-45}{$\!\iso$} & H_{n}(\sphere{n}) & \rotatebox{45}{$\!\iso$}
\end{array}
\end{equation}
where the second and third rows follows from the long exact sequence
\begin{equation}
0=H_{n}(\sphere{n}\setminus\{x\})\to H_{n}(\sphere{n})\to H_{n}(\sphere{n},\sphere{n}\setminus\{x\})\to
H_{n-1}(\sphere{n}\setminus{x})=0,
\end{equation}
which gives us the isomorphisms.
Assume $U\subset f^{-1}(V)$ (so $f(U)\subset V$).
Assume $x$ is an isolated point in $f^{-1}(\{y\})$.
Then we can choose $U$ such that $U\cap f^{-1}(\{y\})=\{x\}$.
Then  $f\colon(U,U\setminus\{x\})\to(V,V\setminus\{y\})$.
Then the induced map on homology groups
$f_{*}\colon H_{*}(U,U\setminus\{x\})\to H_{*}(V,V\setminus\{y\})$.
We see this map $f_{*}$ is just multiplication with an integer $k\in\ZZ$.
So we define the \define{Local Degree} of $f$ at $x$ to be this
integer $\deg f|_{x}=k$.

It does not depend on the choice of $U$ and $V$ (so it is
well-defined)\dots well, it is well-defined for isolated points. If
$x\in f^{-1}(\{y\})$ is an isolated point, then we may define this
local degree of a map.

If $f^{-1}(\{y\})$ consists of isolated points (which is closed in
$\sphere{n}$, which itself is compact), then $f^{-1}(\{y\})$ is also
compact --- well, $f^{-1}(\{y\})$ is \emph{always} compact --- but
moreover $f^{-1}(\{y\})$ must be a finite set.
\end{definition}

\begin{proposition}
If $f^{-1}(\{y\})=\{x_{1},\dots,x_{m}\}$, then
$\deg(f)=\sum_{i=1}^{m}\deg f|_{x_{i}}$.

Let $U_{i}$ be an open neighborhood of each $x_{i}$ and $V$ an open
neighborhood of $y$.
If $f|_{U_{i}}$ is homeomorphic onto its image, then $\deg f|_{x_{i}}=\pm1$
(depending on whether $f|_{U_{i}}$ preserves or reverses its orientation).
\end{proposition}

\begin{node}
So we pick our point $y$, then perturb $f$ such that $f^{-1}(\{y\})$
is a finite set. Then we find sufficiently small neighborhoods $U_{i}$
of each $x_{i}\in f^{-1}(\{y\})$, and we see if $f|_{U_{i}}$ is
orientation reversing or preserving, and we use it to calculate the
degree of $f$.
\end{node}

