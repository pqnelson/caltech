%%
%% winter-lecture21.tex
%% 
%% Made by Alex Nelson <pqnelson@gmail.com>
%% Login   <alex@lisp>
%% 
%% Started on  2026-02-21T14:41:37-0800
%% Last update 2026-02-21T14:41:37-0800
%% 

\lecture{}

\begin{node}
The duality between fibration and cofibration goes under the name
\define{Eckmann--Hilton duality}. This is a natural bijection
\begin{equation}
\langle\Sigma X,Y\rangle\iso\langle X,\Omega Y\rangle,
\end{equation}
which is the starting point of everything. This gives us the
correspondence:
\begin{itemize}
\item $\mbox{cofibration}\longleftrightarrow\mbox{fibration}$
\item $\mbox{homotopy extension}\longleftrightarrow\mbox{homotopy lifting}$
\item The (reduced) cohomology of a space $X$ is measured by maps from $X$
to a space with a single nontrivial homotopy group
\begin{equation}
\widetilde{H}^{n}(X,G)\stackrel{1:1}{\longleftrightarrow}\langle X,K(G,n)\rangle
\end{equation}
\item Homotopy groups of $X$ are measured by maps from a space with a
  very simple (reduced) cohomology to $X$
\begin{equation}
\pi_{n}(X)=\langle\sphere{n},X\rangle
\end{equation}
\end{itemize}
For more about this, see Hatcher \S4.H.
\end{node}

\begin{node}[Homotopy groups with other coefficients?]
But the homotopy group does not take \emph{any} (Abelian) group of
coefficients, which spoils the duality. We can discuss ``rational
homotopy groups'',\index{Homotopy group!Rational}\index{Rational homotopy group}
defined as $\pi_{n}(X)\otimes_{\ZZ}\QQ$ which kills the torsion part
of the homotopy groups. See Quillen's ``Rational Homotopy Groups''.
\end{node}

\begin{node}[Co-homotopy groups]\index{Co-homotopy groups}
We could try constructing the ``co-homotopy group''' studying maps
$\langle X,\sphere{n}\rangle$ which should be dual to homotopy groups,
but it's not studied that much.

Confusingly, these were originally
named ``cohomotopy sets'' by Borsuk who introduced them in ``Sur les groupes des classes de transformations continues''
(\textit{Comptes Rendue de Academie de Science}, Paris \textbf{202} (1936) no.1400--1403, no.2),
but it wasn't until Spanier's ``Borsuk's cohomotopy groups''
(\textit{Annals of Mathematics} \textbf{50} (1949) pp.203--245, \doi{10.2307/1969362})
that they were systematically studied.
\end{node}

\begin{node}
We should think of fibration sequence as a ``space-level long exact sequence''.
What happens if $E$ is contractible ($E\homotopic\point{*}$)?
The long exact sequence for homotopy groups of the fibration
simplifies considerably, can we hope for something similar?
\end{node}

\begin{proposition}
If $F\into E\xrightarrow{p}B$ is a fibration and if $E$ is
contractible, then there is a weak homotopy equivalence between $F$
and $\Omega B$.
\end{proposition}

\begin{remark}
Milnor proved: If $B$ is a CW complex, then $PB$ and $\Omega B$ are of
the homotopy type of a CW complex, and therefore $f$ is a homotopy equivalence.
\end{remark}

\begin{example}
Recall the fibration $\sphere{1}\into\sphere{\infty}\to\CP^{\infty}$,
the space $\sphere{\infty}$ is contractible. Then
\begin{equation}
\vcenter{\xymatrix@R=0.5pc@C=0.25pc{\sphere{1} \ar@{=}[r]^-{\txt{$\sim$}} & \Omega\CP^{\infty}\\
K(\ZZ,1)\ar@{=}[u] & K(\ZZ,2)\ar@{=}[u]}}
\end{equation}
In general, we have the following result (again by Milnor):
\end{example}

\begin{theorem}[Milnor]
For any topological group $G$, there is a fiber bundle with fiber
$G\into EG\to BG$ such that the space $EG$ is contractible (hence
$G\homotopic\Omega BG$).
\end{theorem}

\begin{remark}
The $p\colon EG\to BG$ is called the \define{Universal $G$-Bundle} if $EG$ is
contractible. We refer to $BG$ as the \define{Classifying Space of $G$}.
Every $G$-bundle is a pullback of this one, so the universal one
contains all the information.
\end{remark}

\begin{theorem}
The loop space of a CW complex is homotopy equivalent to a topological group.
\end{theorem}

\begin{proof}[Proof idea]
The basic idea is to allow reparametrization for based loops, i.e.,
elements of $\Omega X$. Compare this to the homotopy groups, which are
based loops in $X$ modulo homotopy equivalence. The reparametrization
allows us to obtain associativity of concatenation of based loops.
\end{proof}
