%%
%% fall-lecture27.tex
%% 
%% Made by Alex Nelson <pqnelson@gmail.com>
%% Login   <alex@lisp>
%% 
%% Started on  2025-12-02T09:43:23-0800
%% Last update 2025-12-02T09:43:23-0800
%% 

\lecture{}

\begin{remark}
Truthfully, I am confused with the material as I review it. It appears
to be a different approach to proving what we ended last lecture, but
that was never made explicitly clear (at least, not that I recollect).
\end{remark}

\begin{lemma}
The groups $H_{i}(M,M\setminus K)=0$ for $i>n$. A homology class
$\alpha\in H_{n}(M,M\setminus K)$ is zero iff
$\rho_{x}(\alpha)\in H_{n}(M,M\setminus\{x\})$
is zero for all $x\in M$.
\end{lemma}

\begin{proof}
First, we consider the space
\begin{equation}
\widetilde{M}=\{(x,a)\mid x\in M,a\in H_{n}(M,M\setminus\{x\})\}.
\end{equation}
We have $p\colon\widetilde{M}\to M$, $p^{-1}(x)=H_{n}(M,M\setminus\{x\})\iso\ZZ$.
This is known as a \define{Bundle of a Group}, but the important thing
is we can put a topology on $\widetilde{M}$ such that
$p\colon\widetilde{M}\to M$ is a covering map. Given an $\alpha\in H_{n}(M)$,
we can define
\begin{equation}
\begin{split}
s_{\alpha}\colon& M\to\widetilde{M}\\
&x\mapsto(x,\rho_{x}(\alpha))
\end{split}
\end{equation}
where $\rho_{x}\colon H_{n}(M)\to H_{n}(M,M\setminus\{x\})$. It's not
hard to see that $s_{\alpha}$ is continuous. We also see $\{(x,0)\mid x\in M\}$
is both open and closed in $\widetilde{M}$. Then
\begin{equation}
s^{-1}_{\alpha}(0)=\{x\in M\mid \rho_{x}(\alpha)=0\}
\end{equation}
(is the preimage of $0$ of a section of the bundle of a group) is both
open and closed in $M$. If $M$ is connected, then either
$s^{-1}_{\alpha}(0)=\emptyset$ or $s^{-1}_{\alpha}(0)=M$.
\end{proof}

\begin{corollary}
If $M$ is closed and connected, then $\rho_{x}$ is injective for each
$x\in M$.
\end{corollary}

\begin{proof}
If $\rho_{x}(\alpha)=0$ for some $x\in M$, then $s^{-1}_{\alpha}(0)\neq\emptyset$.
This means $s^{-1}_{\alpha}(0)=M$. Then $\rho_{y}(\alpha)=0$ for all
$y\in M$. Then by the previous Lemma, $\alpha=0$. So we've established
the kernel of $\rho_{y}$ is trivial for all $y\in M$. Hence the result.
\end{proof}

\begin{theorem}
\begin{enumerate}
\item If $M$ is a closed, connected orientable $n$-manifold, then $H_{n}(M)\iso\ZZ$.
\item If $M$ is an $n$-manifold which is either connected noncompact
  or nonorientable, then $H_{n}(M)=0$.
\end{enumerate}
\end{theorem}

\begin{proof}
(1) We have a map $\rho_{x}\colon H_{n}(M)\to H_{n}(M,M\setminus\{x\})\iso\ZZ$
is injective. Then we know from
Theorem~\ref{thm:math151a:fall2025:lec26:thm-star} $\rho_{x}(\mu_{M})=\mu_{x}$
which implies $\rho_{x}$ is surjective. Hence $\rho_{x}$ is an isomorphism.


(2) If $M$ is nonorientable and closed, then since $\rho_{x}$ is
injective we know $H_{n}(M)\iso0$ or $H_{n}(M)\iso\ZZ$. But if
$H_{n}(M)\iso\ZZ$, then we can consider
\begin{equation}
\mu_{x}:=\frac{\rho_{x}(1)}{|\rho_{x}(1)|}\in H_{n}(M,M\setminus\{x\}),
\end{equation}
which is a well-defined local orientation, and it varies
continuously with respect to $x$ (since $\rho_{x}$ varies
continuously with $x$). But this implies $M$ is orientable, which is
a contradiction.

If $M$ is noncompact, then let $z\in C_{n}(M)$ be a cycle. Let $x\in M$
be away from the support of $z$, then $\rho_{x}([z])=0$. (We can
always find such an $x\in M$ since $M$ is noncompact, and $z$ is a
\emph{finite} linear combination of simplices.) This implies
$\rho_{y}([z])=0$ for all $y\in M$.

(Lemma: for all $K\subset M$ compact, $[z]=0$ in $H_{n}(M,M\setminus K)$.)

Let $K$ be a compact subset of $M$ which contains the support of $z$
in its interior. Let $V=M\setminus K$. We have a commutative diagram
\begin{equation}
\vcenter{\xymatrix{0=H_{n+1}(M,\Interior(K)\cup V)\ar[r] & H_{n}(\Interior(K)\cup V,V)\ar[r] & H_{n}(M,V)\\
& H_{n}(\Interior(K))\ar[u]^{\text{inclusion}}\ar[r]_{\text{inclusion}} & H_{n}(M)\ar[u]}}
\end{equation}
The inclusion $H_{n}(\Interior(K))\to H_{n}(\Interior(K)\cup V,V)$ is
an isomorphism by the Excision theorem. We claim $[z]\in H_{n}(\Interior(K))$
and $[z]\in H_{n}(M)$ so by the commutative diagram
$H_{n}(\Interior(K))\to H_{n}(M)\to H_{n}(M,V)$ being exact, this maps
to $0$. So we have $0\mapsto[z]\mapsto0$, which forces $[z]=0$.
\end{proof}

\begin{proposition}
If $M$ is connected and $\boundary M\neq0$, then $H_{n}(M)=0$.

If $M$ is compact, connected, and oriented $n$-manifold, then
$H_{n}(M,\boundary M)\iso\ZZ$. 
\end{proposition}

(This is the analogous result for manifolds with boundary.)

\begin{remark}
If $M$ is a closed, connected $n$-manifold, then $H_{n}(M,\ZZ/2\ZZ)\iso\ZZ/2\ZZ$
(since every manifold is $\ZZ/2\ZZ$-orientable).
\end{remark}

\subsection{Poincar\'{e} Duality}

\begin{definition}
Let $M$ be a connected, compact, orientable $n$-manifold, let $R$ be a ring.
A \define{Fundamental Class} for $M$ with coefficients in $R$ is an
element $[M]:=\mu_{M}$ of $H_{n}(M;R)$ whose image in
$H_{n}(M,M\setminus\{x\};R)$ is a generator for all $x\in M$.
\end{definition}

\begin{remark}
\begin{enumerate}
\item The ``Manifold Atlas'' wiki cites: M.J.~Greenberg and J.R.~Harper, \textit{Algebraic topology}, Benjamin/Cummings Publishing Co. Inc. Advanced Book Program, 1981. 
\item  This definition can be modified to work with an $(n-1)$-connected topological space.
\end{enumerate}
\end{remark}

\begin{theorem}
Let $M$ be a closed orientable $n$-manifold with ``\emph{Fundamental Class}''
$[M]:=\mu_{M}\in H_{n}(M;\ZZ)$, then the map $D\colon H^{k}(M)\to H_{n-k}(M)$
defined by $D(\alpha)=[M]\frown\alpha$ is an isomorphism for all $k$.
\end{theorem}

\begin{remark}
This is \emph{the} fundamental result in homology theory.
\end{remark}

\begin{remark}
The proof of this is highly nontrivial.

How do we see it intuitively? There's no good way with singular
homology. But in differential topology, there are ways to gain an intuition.

We will prove an even stronger result, which requires a few new notions.
\end{remark}

\begin{definition}
We define the \define{Cochains with compact support} $C^{i}_{c}(X)$
which consists of cochains $\varphi\colon C_{i}(X)\to\ZZ$ such that
there exists compact $K\subset X$ such that $\varphi(\sigma)=0$ if the
image of $\sigma\subset X\setminus K$.

Then we consider $C^{*}_{c}(X)$ which is a chain subcomplex of $C^{*}(X)$.

Then we consider the \define{Cohomology with Compact Support} $H^{*}_{c}(X)$
which is the cohomology of $C^{*}_{c}(X)$.
\end{definition}

\begin{example}
What is $H^{*}(\RR^{n})$? It turns out to be the limit
\begin{equation}
H^{*}(\RR^{n})=\lim_{r\to\infty}H^{*}(\RR^{n},\RR^{n}\setminus B^{n}_{r}),
\end{equation}
where $B^{n}_{r}$ is the ball of radius $r$ centered at $0$. A cochain
in compact $K$ may be placed in a large ball. We know that for fixed
$r$ that
\begin{equation}
H^{k}(\RR^{n},\RR^{n}\setminus B^{n}_{r})\iso\begin{cases}\ZZ &
\mbox{if }k=n,\\
0&\mbox{otherwise},
\end{cases}
\end{equation}
then
\begin{equation}
H^{k}_{c}(\RR^{n},\RR^{n}\setminus B^{n}_{r})\iso\begin{cases}\ZZ &
\mbox{if }k=n,\\
0&\mbox{otherwise}.
\end{cases}
\end{equation}
\end{example}

\begin{remark}
We want to prove Poincar\'{e} duality using $H_{c}$, then we just
remove the hyporhesis $M$ is closed. We also modify the definition of
$D$ since the fundamental class is no longer well-defined. We'll do
this next time.
\end{remark}
