%%
%% winter-lecture23.tex
%% 
%% Made by Alex Nelson <pqnelson@gmail.com>
%% Login   <alex@lisp>
%% 
%% Started on  2026-02-26T09:12:44-0800
%% Last update 2026-02-26T09:12:44-0800
%% 

\lecture{}

\begin{node}
A lot of problems in Algebraic Topology can be phrased as:
\begin{subequations}
\begin{enumerate}
\item\textsc{Extension problem}: Given a CW pair $(W,A)$ (in
  particular, $A\into W$ is a cofibration) and a map $A\to X$, can we
  extend it to a map $W\to X$? The diagram looks like:
\begin{equation}
\vcenter{\xymatrix{A\ar@{^{(}->}[d]\ar[r] & X\\
W\ar@{..>}[ur] & }}
\end{equation}
\item\textsc{Lifting problem}: Given a fibration $X\to Y$ and a map
  $W\to Y$ can we lift it to $W\to X$?
\begin{equation}
\vcenter{\xymatrix{ & X\ar[d]\\
W\ar@{..>}[ur]\ar[r] & Y}}
\end{equation}
\item\textsc{Relative lifting problem}: Given a CW pair $(W,A)$ and a
  fibration $X\to Y$,
given a map $W\to Y$ and a ``partial lifting'' $A\to X$, can we extend
this ``partial lifting'' to a lift $W\to X$?
\begin{equation}
\vcenter{\xymatrix{A\ar@{^{(}->}[d]\ar[r] & X\ar[d]\\
W\ar@{..>}[ur]\ar[r] & Y}}
\end{equation}
\end{enumerate}
Observe the other problems are special cases of the relative lifting
problem: $A=\emptyset$ recovers the lifting problem, and when $Y$ is a
singleton $Y=\{y\}$ we recover the extension problem.
\end{subequations}
\end{node}

\begin{node}
Obstruction theory is the procedure to identify a family of chomology
classes which are obstructions to the lifting/extension problems (in
the sense that a lift/extension exists if and only if the cohomology
classes vanish).
\end{node}

\begin{node}
We have already seen some preliminary statement in this direction by
cell-by-cell argument (e.g., in the compression lemma and the
extension lemma).
\end{node}

\begin{node}[Goal]
Get some finer answer using Postnikov tower. Let us consider the
extension problem. (In general, there is a construction called the
Moore--Postnikov tower which can be used to run the same argument).
\end{node}

\begin{node}
Suppose we consider the extension problem:
\begin{equation}
\vcenter{\xymatrix{A\ar@{^{(}->}[d]\ar[r] & X\\
W\ar@{..>}[ur] & }}
\end{equation}
Suppose that $X$ has a Postnikov tower consisting of principal
fibrations. Then
we can write it as
\begin{equation}
\vcenter{\xymatrix{
                      &    & \dots\ar[d] & \\
                      &    & X_{3}\ar[r]\ar[d] & K(\pi_{4}(X),5)\\
                      &    & X_{2}\ar[r]\ar[d] & K(\pi_{3}(X),4)\\
A\ar@{^{(}->}[d]\ar[r] & X\ar[ur]\ar[uur]\ar[uuur]\ar[r]  & X_{1}\ar[r] & K(\pi_{2}(X),3)\\
W\ar@{..>}[ur] & }}
\end{equation}
Suppose further that $X$ is a connected CW complex. Then we can extend
the Postnikov tower one more layer to $X_{0}=\{*\}$ a point
\begin{equation}
\vcenter{\xymatrix{
                      &    & \dots\ar[d] & \\
                      &    & X_{3}\ar[r]\ar[d] & K(\pi_{4}(X),5)\\
                      &    & X_{2}\ar[r]\ar[d] & K(\pi_{3}(X),4)\\
A\ar@{^{(}->}[d]\ar[r] & X\ar[ur]\ar[uur]\ar[uuur]\ar[r]  & X_{1}\ar[r]\ar[d] & K(\pi_{2}(X),3)\\
W\ar@{..>}[ur] & & X_{0}\ar[r] & K(\pi_{1}(X),2)}}
\end{equation}
where we view $X_{1}\to X_{0}\to K(\pi_{1}(X),2)$ as a fibration.
In the above argument roughly $X$ can be viewed as an \emph{inverse limit}
of $\varprojlim(\dots\to X_{2}\to X_{1}\to X_{0})$, so the extension
problem may be viewed as an inductive argument or a recursive construction (where the $X_{0}$ case
is rather trivial):

Given
\begin{equation}
\vcenter{\xymatrix{A\ar@{^{(}->}[d]\ar[r] & X_{n}\ar[d]\\
W\ar[r]\ar@{..>}[ur] & X_{n-1}}}
\end{equation}
can we extend the partial lift $W\to X_{n-1}$ to a lift $W\to X_{n}$?
\end{node}

\begin{node}
For the vertical map $X_{n}\to X_{n-1}$, by assumption it is a
\emph{principal} fibration, i.e., $X_{n}$ is the homotopy fiber of the
map
\begin{equation}
k_{n}\colon X_{n-1}\to K:=K(\pi_{n}(X),n+1).
\end{equation}
Recall the construction of homotopy fiber of a fibration
\begin{equation}
\vcenter{\xymatrix{\phantom{A} & X_{n}\ar[r]\ar[d] & PK\ar[d]\\
\phantom{W} & X_{n-1}\ar[r]^{k_{n}} & K}}
\end{equation}
The question is whether it can be extended to the red portion of the diagram:
\begin{equation}
\vcenter{\xymatrix{{\color{red}A}\ar@[red][r]\ar@[red][d] & X_{n}\ar[r]\ar[d] & PK\ar[d]\\
{\color{red}W}\ar@[red][r]^{\color{red}f}\ar@[red]@{..>}[ur]^{\color{red}\widetilde{f}} & X_{n-1}\ar[r]^{k_{n}} & K}}
\end{equation}
A lift of a map $f\colon W\to X_{n-1}$ would be something like
\begin{equation}
\begin{split}
\widetilde{f}\colon&W\to X_{n}\\
&w\mapsto(f(w),\gamma_{w}),
\end{split}
\end{equation}
i.e., $\widetilde{f}$ can be viewed as a map
\begin{equation}
\begin{split}
\gamma\colon & W\to PK\\
& w\mapsto\gamma_{w}
\end{split}
\end{equation}
such that $\gamma_{w}(0)=k_{n-1}(f(w))$ and $\gamma_{w}(b)$ is the
base point.

But there is another way to view a lifting $\gamma\colon W\to PK$ is
the same as a homotopy from $k_{n-1}\circ f$ to a constant map
$\operatorname{const}_{b}$.

We want a null homotopy of the composition
\begin{equation}
k_{n-1}\circ f\colon W\to K,
\end{equation}
extending the given one in $A$.

One more thing: the map $k_{n-1}\circ f\colon W\to K$ together with a
null-homotopy to $A$ (this is just the given data) can be viewed as a
map $W\cup CA\to K$ where, when on $W$ it is just $k_{n-1}\circ f$,
and when on $CA$ it is just the null-homotopy.

Recall the definition of a fibration $X\to Y$ meaning for any $W\times I\to Y$
and any partial lift $W\times\{0\}\to X$ there exists a lift $W\times I\to X$
\emph{of homotopies}.

The map $W\cup CA\to K=K(\pi_{n}(X),n+1)$ gives us an element $\omega_{n}$ in
\begin{equation}
H^{n+1}(\underbrace{X\cup CA}_{\homotopic X/A};\pi_{n}(X))\iso H^{n+1}(W,A;\pi_{n}(X)).
\end{equation}
\end{node}

\begin{definition}
The \define{$n^{\text{th}}$ Obstruction Class} $\omega_{n}\in H^{n+1}(W,A;\pi_{n}(X))$
is defined to be the above element.
\end{definition}

\begin{lemma}
A lift $\widetilde{f}_{n}\colon W\to X_{n}$ of $f\colon W\to X_{n-1}$
extending the given one $A\to X_{n}$ exists if and only if the
$n^{\text{th}}$ obstruction class vanishes $\omega_{n}=0$.
\end{lemma}

\begin{proof}
Such a lifty $\widetilde{f}\colon W\to X_{n}$ exists if and only if
the map $W\cup CA\to K$ can be extended to a map $CW\to K$. (We want
to extend the null-homotopy from $CA$ to $CW$.

Then $\omega_{n}=0$ if and only if the map $W\cup CA\to K$ is
null-homotopic. Therefore it is enough to show: a given map
\begin{equation}
g\colon W\cup CA\to K
\end{equation}
is null-homotopic if and only if $g$ can be extended to a map
\begin{equation}
\widetilde{g}\colon CW\to K.
\end{equation}
This is the claim we will prove.

\backwardproof\ Suppose we have an extension $\widetilde{g}\colon CW\to K$.
Since $CW$ is contractible, there is a null-homotopy from
$\widetilde{g}$ to $\operatorname{const}_{b}$. Then restricting this
null-homotopy to $W\cup CA$ gives the desired null-homotopy from $g$
to $\operatorname{const}_{b}$.

\forwardproof\ Suppose $g_{t}\colon W\cup CA\to K$ is a null-homotopy
of $g$. That is to say,
\begin{equation}
\begin{split}
g_{1}\colon & W\cup CA\to K\\
& w\mapsto b,
\end{split}
\end{equation}
and
\begin{equation}
g_{0}=g.
\end{equation}
Then by the \emph{homotopy extension property} of CW pair $(CW,W\cup CA)$,
there exists an extension of the homotopy
\begin{equation}
\widetilde{g}_{t}\colon W\to K
\end{equation}
such  that
\begin{equation}
\begin{split}
\widetilde{g}_{1}\colon & CW\to K\\
& w\mapsto b
\end{split}
\end{equation}
is the constant map. Then $\widetilde{g}_{0}$ is a map $CW\to K$
extending $g_{0}=g$.
\end{proof}

\begin{remark}
This proof is kind of unsatisfactory, but it's the machinery which
allows us to translate the problem simply.
\end{remark}

\begin{corollary}
If $X$ is a connected CW complex such that $\pi_{1}(X)$ acts trivially
on $\pi_{n}(X)$ for all $n>1$, and if $(W,A)$ is a CW pair such that
for all $n$ we have $H^{n+1}(W,A;\pi_{n}(X))=0$, then every map $A\to X$
can be extended to a map $W\to X$.
\end{corollary}