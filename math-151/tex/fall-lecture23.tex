%%
%% fall-lecture23.tex
%% 
%% Made by Alex Nelson <pqnelson@gmail.com>
%% Login   <alex@lisp>
%% 
%% Started on  2025-11-20T10:02:02-0800
%% Last update 2025-11-20T10:02:02-0800
%% 

\lecture{}

\begin{node}
The cup product is bilinear over $R$.
\end{node}

\begin{definition}
Let $R$ be a commutative ring. We have a ring structure on $H^{*}(X;R)$,
called the \define{Cohomology Ring} of $X$.
\end{definition}

\begin{theorem}
If $[\varphi]\in H^{k}(X;R)$ and $[\psi]\in H^{\ell}(X;R)$, then $[\varphi]\cup[\psi]=(-1)^{k\ell}[\psi]\cup[\varphi]$.
\end{theorem}

In other words, the cup product is \emph{graded commutative}.

\begin{convention}
If we omit the ring of coefficients when writing the cohomology group
(or cohomology ring), then we always mean the integers --- e.g.,
$H^{*}(X)$ always means $H^{*}(X;\ZZ)$.
\end{convention}

\begin{node}[Unital cohomology rings]
If $R$ is a unital commutative ring with $1_{R}\in R$, then we can
consider the augmentation $\varepsilon\in H^{0}(X;R)$ --- the mapping
$C_{0}(X)\xrightarrow{\varepsilon}R$ which sends $\sum_{i}n_{i}x_{i}\mapsto\sum_{i}n_{i}$
(i.e., every point to $1_{R}$). This $\varepsilon$ is the identity
element of the cohomology ring $H^{*}(X;R)$.
\end{node}

\begin{example}
The Torus $T^{2}$. What is $H^{*}(T^{2};\ZZ)$? We have already computed,
\begin{equation}
H^{k}(T^{2};\ZZ)\iso\begin{cases}\ZZ & \mbox{if }k=0,2\\
\ZZ^{2} & \mbox{if }k=1\\
0 & \mbox{otherwise}
\end{cases}
\end{equation}
We see then that $H^{0}(T^{2})=\langle\varepsilon\rangle$. We find
that $H^{2}(T^{2})\smile H^{k}(T^{2})=0$ for any $k>0$, so we only
need to understand the cup product
\begin{equation}
H^{1}(T^{2})\times H^{1}(T^{2})\xrightarrow{\smile}H^{2}(T^{2}).
\end{equation}
For $[\varphi],[\psi]\in H^{1}(T^{2})$, we know
$[\varphi]\smile[\psi]\in H^{2}(T^{2})\iso\ZZ$, but
\begin{equation}
[\varphi]\smile[\psi]=-[\psi]\smile[\varphi]
\end{equation}
which implies $[\varphi]\smile[\varphi]=0$.

By the universal coefficient theorem, $H^{1}(T^{2})\iso\hom(H_{1}(T^{2}),\ZZ)$.
We know $H_{1}(T^{2})$. We also know the singular complex for
$T^{2}$. We doodle its singular complex as
\begin{equation}
\vcenter{\hbox{\includegraphics{img/img.49}}}
\end{equation}
We have two singular simplices
\begin{equation}
\begin{split}
  \sigma_{1}\colon&\Delta^{2}\to\Delta ABC\\
&[v_{0},v_{1},v_{2}]\mapsto[A,B,C]
\end{split}
\end{equation}
\begin{equation}
\begin{split}
  \sigma_{2}\colon&\Delta^{2}\to\Delta ADC\\
&[v_{0},v_{1},v_{2}]\mapsto[A,D,C]
\end{split}
\end{equation}
Then $H_{1}(T^{2})=\ZZ\langle[a],[b]\rangle$. Then we find the first
cohomology group $H^{1}(T^{2})=\ZZ\langle[\alpha],[\beta]\rangle$
where
\begin{equation}
\langle\alpha,a\rangle=1,\quad\mbox{and}\quad\langle\alpha,b\rangle=0
\end{equation}
\begin{equation}
\langle\beta,a\rangle=0,\quad\mbox{and}\quad\langle\beta,b\rangle=1
\end{equation}
We find
\begin{subequations}
  \begin{align}
\langle\alpha\smile\beta,\sigma_{1}\rangle
&=\alpha([AB])\beta([BC])\\
&=\alpha(a)\beta(b)\\
&=1
  \end{align}
\end{subequations}
and
\begin{subequations}
  \begin{align}
\langle\alpha\smile\beta,\sigma_{2}\rangle
&=\alpha([AD])\beta([DC])\\
&=0.
  \end{align}
\end{subequations}
Then
\begin{equation}
\langle\alpha\smile\beta,\sigma_{1}-\sigma_{2}\rangle=1,
\end{equation}
so $\sigma_{1}-\sigma_{2}$ generates $H^{2}(T^{2})$.
This tells us also that $[\alpha]\smile[\beta]$ generates $H^{2}(T^{2})$.
\end{example}

\begin{example}
In general, if $\Sigma_{g}$ is the orientable closed surface of genus $g$,
then we can find a basis
\begin{equation}
H_{1}(\Sigma_{g})=\ZZ\langle a_{1},b_{1},\dots,a_{g},b_{g}\rangle
\end{equation}
We find the duals to these generators
\begin{equation}
\langle\alpha_{i},a_{j}\rangle=\delta_{i,j}\quad\mbox{and}\quad\langle\alpha_{i},b_{j}\rangle=0,
\end{equation}
\begin{equation}
\langle\beta_{i},a_{j}\rangle=0\quad\mbox{and}\quad\langle\beta_{i},b_{j}\rangle=\delta_{i,j}.
\end{equation}
Then $\alpha_{i}\smile\alpha_{j}=0$ and $\beta_{i}\smile\beta_{j}=0$, but
$\alpha_{i}\smile\beta_{j}=\delta_{i,j}$. This gives us the structure
of $H^{*}(\Sigma_{g};R)$.
\end{example}

\begin{example}
The real projective plane $\RP^{2}$. 
We recall the cohomology groups for $\RP^{2}$ are:
\begin{equation}
H^{k}(\RP^{2};\ZZ)\iso\begin{cases}
\ZZ & \mbox{if }k=0\\
\ZZ/2\ZZ & \mbox{if }k=2\\
0 & \mbox{otherwise}
\end{cases}
\end{equation}
Then we have no interesting cup product, due to grading reasons.

However, using $\ZZ/2\ZZ$ for the coefficients, we find
\begin{equation}
H^{k}(\RP^{2},\ZZ/2\ZZ)\iso\begin{cases}\ZZ/2\ZZ & \mbox{if }k=0,1,2\\
0 & \mbox{otherwise}
\end{cases}
\end{equation}
What is the cohomology ring in this case?
Again, we can draw the singular complex
for it:
\begin{equation}
\vcenter{\hbox{\includegraphics{img/img.50}}}
\end{equation}
We have two simplices
\begin{equation}
\begin{split}
\sigma_{1}\colon&\Delta^{2}\to\Delta ABC\\
&[v_{0},v_{1},v_{2}]\mapsto[A,B,C]
\end{split}
\end{equation}
\begin{equation}
\begin{split}
\sigma_{2}\colon&\Delta^{2}\to\Delta CDA\\
&[v_{0},v_{1},v_{2}]\mapsto[C,D,A]
\end{split}
\end{equation}
Eh\dots we skip the calculations. It turns out
\begin{equation}
H_{1}(\RP^{2};\ZZ/2\ZZ)\iso\ZZ_{2}\langle a+b\rangle,
\end{equation}
so
\begin{equation}
\alpha(a+b)=1,
\end{equation}
and then we find $\alpha\smile\alpha$ generates $H^{2}(\RP^{2};\ZZ/2\ZZ)$.
It turns out the ring structure for the cohomology ring looks like
\begin{subequations}
  \begin{align}
H^{*}(\RP^{2};\ZZ/2\ZZ) &= \langle1,\alpha,\alpha^{2}\rangle\\
&=\ZZ_{2}[\alpha]/(\alpha^{3}).
  \end{align}
\end{subequations}
\end{example}

\begin{example}
More generally, for $\Pi_{g}$ (the nonorientable closed surface of
genus $g$), we find
\begin{equation}
H^{k}(\Pi_{g};\ZZ/2\ZZ)\iso\begin{cases}\ZZ/2\ZZ & \mbox{if }k=0,2\\
(\ZZ/2\ZZ)^{g} & \mbox{if }k=1\\
0 & \mbox{otherwise}
\end{cases}
\end{equation}
Then we find
\begin{equation}
H^{1}(\Pi_{g};\ZZ/2\ZZ)=(\ZZ/2\ZZ)\langle\alpha_{1},\dots,\alpha_{g}\rangle
\end{equation}
and $\alpha_{1}^{2}=\alpha_{2}^{2}=\dots=\alpha_{g}^{2}$ is the
generator for $H^{2}(\Pi_{g};\ZZ/2\ZZ)$. We also find
$\alpha_{i}\smile\alpha_{j}=0$ if $i\neq j$.
\end{example}

\begin{proposition}
Let $R$ be a commutative ring (possibly unital).
If we have $f\colon X\to Y$ continuous, then we have the induced map
$f^{*}\colon H^{*}(Y;R)\to H^{*}(X;R)$ is a ring morphism.
\end{proposition}

\begin{node}[Cup product for relative cohomology]
Let $R$ be a commutative ring (possibly unital).
If $A\subset X$ and $B\subset X$ are open subsets,
then we can define the cup product for relative cohomology groups
\begin{equation}
\smile\colon H^{*}(X,A;R)\times H^{*}(X,B;R)\to H^{*}(X,A\cup B;R).
\end{equation}
If we have $\alpha\in C^{*}(X,A;R)$ and $\beta\in C^{*}(X,B;R)$,
then $\alpha$ evaluates to zero on chains in $A$, and $\beta$
evaluates to zero on chains in $B$. Then $\alpha\smile\beta$ evaluates
to zero on chains in $C_{*}(A)+C_{*}(B)\homotopic C_{*}(A\cup B)$ if
$A$, $B$ overlap.
\end{node}