%%
%% fall-lecture07.tex
%% 
%% Made by Alex Nelson <pqnelson@gmail.com>
%% Login   <alex@lisp>
%% 
%% Started on  2025-10-14T12:12:24-0700
%% Last update 2025-10-14T12:12:24-0700
%% 

\lecture{}

\begin{example}
For $X=\point{x}$, then for each $n$ there is exactly one singular
$n$-simplex $\sigma\colon\Delta^{n}\to X$ which is the cosntant
map. Then
\begin{equation}
C_{n}(X)\iso\ZZ\quad\mbox{for all }n\geq0.
\end{equation}
The boundary map
\begin{equation}
\begin{split}
\boundary_{n}\colon&C_{n}(X)\to C_{n-1}(X)\\
&\sigma\mapsto\sum^{n}_{i=0}(-1)^{i}\sigma|_{[v_{0},\dots,\widehat{v_{i}},\dots,v_{n}]}
\end{split}
\end{equation}
but all these $\sigma|_{\text{faces}}$ are the same map, so this means
$\boundary_{n}\colon\ZZ\to\ZZ$ corresponds to multiplication by
$0=\sum^{n}_{i=0}(-1)^{i}$ when $n$ is odd, or by $1=\sum(-1)^{i}$
when $n$ is even. So we have
\begin{equation}
\cdots\to C_{3}\xrightarrow{0}C_{2}\iso C_{1}\xrightarrow{0}C_{0}\to 0.
\end{equation}
Then $H_{0}(X)\iso C_{0}\iso\ZZ$, $H_{1}(X)=\ker(\boundary_{1})/\Im(\boundary_{2})\iso0$.
We find
\begin{equation}
H_{n}(\point{x})=\begin{cases}\ZZ & \mbox{if }n=0\\
0 & \mbox{otherwise}
\end{cases}
\end{equation}
\end{example}

\begin{proposition}
If $X_{\alpha}$ are path-connected components of $X$,
then we see that $H_{n}(X) = \bigoplus_{\alpha}H_{n}(X_{\alpha})$.
\end{proposition}

Every singular $n$-simplex must be contained in one of the
path-connected components, so we just need to compute the homology of
those path-connected components.

\begin{proposition}
$H_{0}(X)\iso\bigoplus_{\alpha}\ZZ$ where $X_{\alpha}$ are the
  path-connected components of $X$.
\end{proposition}

\begin{proof}
If $X$ is path-connected, then
\begin{equation}
C_{1}(X)\xrightarrow{\boundary_{1}} C_{0}(X)\xrightarrow{0}0
\end{equation}
so $H_{0}(X)=C_{0}(X)/\Im(\boundary_{1})$. Then $C_{0}(X)$ is freely
generated by $\sigma\colon\Delta^{0}\to X$ but $\Delta^{0}$ is just a
point, so $C_{0}(X)$ is just freely generated by points in $X$.

Then signular $1$-simplex is $\sigma\colon\Delta^{1}\to X$, but this
is homeomorphic to the closed unit interval $\Delta^{1}\iso[0,1]$, so
$\sigma$ is just a path in $X$. Then $\boundary(\sigma)=\sigma(1)-\sigma(0)$.
Then $\Im(\boundary_{1})$ is freely generated by $\sigma(1)-\sigma(0)$
for all paths $\sigma$. But $X$ is path-connected, so \emph{every} two
points is connected by a path: $\forall x,y\in X\ldotp x-y\in\Im(\boundary_{1})$.

Taken together, these facts imply $H_{0}(X)\iso\ZZ$ is generated
freely by a single generator. Hence the result.
\end{proof}

We have exhausted the ability of computing singular homology by
definition, or at least the examples become more and more tedious (and
continuing to work on them becomes increasingly de-motivating). So we
need to build some results. This will involve digressions to
homological algebra!

\subsection{Homotopy Invariance}

\begin{definition}
Let $(C,\boundary)$ and $(C',\boundary')$ be chain complexes, and let
$f,g\colon C\to C'$ be chain maps. We define a \define{Chain Homotopy}
between $f$ and $g$ to be a morphism $H\colon C\to C'$ which can be
decomposed as $H=\bigoplus_{n\in\ZZ}H_{n}$ where $H_{n}\colon C_{n}\to C'_{n+1}$
increases the grading by 1, such that $\boundary'\circ H+H\circ\boundary=f-g$,
or $\boundary'_{n+1}\circ H_{n}+H_{n-1}\circ\boundary_{n}=f_{n}-g_{n}$.

We stress, \emph{a chain homotopy \textbf{is not} a chain map!}
\end{definition}

\begin{remark}
We can look at ``higher homotopies'', which increase the grading by $k>1$.
This is useful in a lot of modern mathematics.
\end{remark}

\begin{theorem}
If $f$ and $g$ are chain homotopic, then
\begin{equation}
f_{*}=g_{*}\colon H_{*}(C)\to H_{*}(C')
\end{equation}
(they induce the same morphism on the chain Homology groups).
\end{theorem}

\begin{proof}
We just need to prove for every cycle $\alpha$ in $C$
that $f(\alpha)$ and $g(\alpha)$ are homologous. We just need to apply
the definition of chain homotopy. We find
\begin{subequations}
\begin{align}
f(\alpha) - g(\alpha) &= \boundary'H(\alpha)+H\boundary(\alpha)\\
&=\boundary'H(\alpha) + 0
\end{align}
\end{subequations}
since $\alpha$ is a cycle. Recall two elements are homologous if their
difference is an element of the image of the boundary map, which we
just proved.
\end{proof}

\begin{proposition}
Let $X$ and $Y$ be topological spaces, let $f,g\colon X\to Y$ be homotopic
maps.
Then $f_{\sharp},g_{\sharp}\colon C_{*}(X)\to C_{*}(Y)$ are chain
homotopic, and therefore $f_{*}=g_{*}\colon H_{*}\to H_{*}(Y)$.
\end{proposition}

\begin{proof}
We want to construct a chain homotopy. Let us call this chain homotopy
\begin{equation}
P\colon C_{*}(X)\to C_{*+1}(Y),
\end{equation}
where we emphasize the grading increases by 1. For any singular
$n$-simplex $\sigma\colon\Delta^{n}\to X$, we want to send it to some
$P(\sigma)$ which is an $(n+1)$-\underline{chain} in $Y$.

Now, we use $f$ and $g$ are homotopic. There exists $F\colon X\times[0,1]\to Y$
such that
\begin{equation}
F|_{X\times\{0\}}=f,\quad\mbox{and}\quad F|_{X\times\{1\}}=g.
\end{equation}
We have $\Delta^{n}\xrightarrow{\sigma}X$ induce
\begin{equation}
\Delta^{n}\times[0,1]\xrightarrow{\widetilde{\sigma}}X\times[0,1]\xrightarrow{F}Y,
\end{equation}
the idea is that $P(\sigma)=F\circ\widetilde{\sigma}$. We view
$\Delta^{n}\times[0,1]$ as a prism
\begin{equation}
\vcenter{\hbox{\includegraphics{img/img.29}}}\xrightarrow{\widetilde{\sigma}}X\times[0,1]\xrightarrow{F}Y
\end{equation}
Now, we triangulate the prism so it is the union of
$(n+1)$-simplices. Then restricting $F\circ\widetilde{\sigma}$ to each
$(n+1)$-simplex in the triangulation gives us a singular
$(n+1)$-simplex. We take the prism as the linear combination of the
singular $(n+1)$-simplices.

If we look at the boundary of the prism $\boundary(\Delta^{n}\times[0,1])$,
it consists of three parts:
\begin{equation}
\boundary(\Delta^{n}\times[0,1])=\bigl((\boundary\Delta^{n})\times[0,1]\bigr)\cup(\Delta^{n}\times\{0\})\cup(\Delta^{n}\cup\{1\}).
\end{equation}
We have $\boundary P+P\boundary=f_{\sharp}+g_{\sharp}$.
We look at the restriction of $F\circ\widetilde{\sigma}$ on these
components, we see
$F\circ\widetilde{\sigma}|_{\text{prism}}=P$ so
\begin{subequations}
\begin{equation}
\boundary(F\circ\widetilde{\sigma}|_{\text{prism}})=\boundary P
\end{equation}
and
\begin{align}
F\circ\widetilde{\sigma}|_{\boundary X\times[0,1]} &= P\boundary\\
F\circ\widetilde{\sigma}|_{X\times\{0\}} &= f_{\sharp}\\
F\circ\widetilde{\sigma}|_{X\times\{1\}} &= g_{\sharp}.
\end{align}
\end{subequations}
Now we can write down the chain homotopy. Formally, we have
\begin{equation}
P(\sigma) = \sum^{n}_{i=0}(-1)^{i}F\circ\widetilde{\sigma}|_{[v_{0},\dots,v_{i},w_{i},\dots,w_{n}]}
\end{equation}
where the vertices on the ``left simplex'' are $v_{0}$, \dots,
$v_{n}$, and the vertices on the ``right simplex'' are $w_{0}$, \dots, $w_{n}$:
\begin{equation}
\vcenter{\hbox{\includegraphics{img/img.30}}}
\end{equation}
For the case when $n=2$, we can draw the situations. When $i=0$, we
get the tetrahedron spanned by $[v_{0},w_{0},w_{1},w_{2}]$:
\begin{equation*}
[v_{0},w_{0},w_{1},w_{2}] = \vcenter{\hbox{\includegraphics{img/img.31}}}
\end{equation*}
For $i=1$, the tetrahedron is spanned by $[v_{0},v_{1},w_{1},w_{2}]$
\begin{equation*}
\vcenter{\hbox{\includegraphics{img/img.32}}}
\end{equation*}
For $i=2$, the tetrahedron is spanned by $[v_{0},v_{1},v_{2},w_{2}]$
\begin{equation*}
\vcenter{\hbox{\includegraphics{img/img.33}}}
\end{equation*}
We see the tetrahedron looks like
\begin{equation*}
i=0:\quad\vcenter{\hbox{\includegraphics{img/img.34}}}
\end{equation*}
\begin{equation*}
i=1:\quad\vcenter{\hbox{\includegraphics{img/img.35}}}
\end{equation*}
\begin{equation*}
i=2:\quad\vcenter{\hbox{\includegraphics{img/img.36}}}
\end{equation*}
\end{proof}

\begin{corollary}
A homotopy equivalence $f\colon X\to Y$ induces an isomorphism
$f_{*}\colon H_{*}(X)\to H_{*}(Y)$.
\end{corollary}

\begin{proof}
There exists a $g\colon Y\to X$ such that $g\circ f\iso\id_{X}$ and
$f\circ g\iso\id_{Y}$. Then by the previous theorem,
$f_{*}\circ g_{*}=\id_{H_{*}(Y)}$ and $g_{*}\circ f_{*}=\id_{H_{*}(X)}$,
which implies $f_{*}$ is an isomorphism of homology groups.
\end{proof}