%%
%% fall-lecture04.tex
%% 
%% Made by Alex Nelson <pqnelson@gmail.com>
%% Login   <alex@lisp>
%% 
%% Started on  2025-10-07T09:20:02-0700
%% Last update 2025-10-07T09:20:02-0700
%%

\lecture{}

\begin{theorem}[Lifting criterion]
Let $p\colon\widetilde{X}\to X$ be a covering map, let
$\widetilde{x}_{0}\in\widetilde{X}$ and $x_{0}=p(\widetilde{x}_{0})$.
Let $Y$ be a path-connected, locally path-connected space. Let $f\colon(Y,y_{0})\to(X,x_{0})$.
A \define{Lift} $\widetilde{f}\colon(Y,y_{0})\to(\widetilde{X},\widetilde{x}_{0})$
exists iff $f_{*}\bigl(\pi_{1}(Y,y_{0}))\subgroup p_{*}\bigl(\pi_{1}(\widetilde{X},\widetilde{x}_{0})\bigr)$,
where $p_{*}\colon\pi_{1}(\widetilde{X},\widetilde{x}_{0})\to\pi_{1}(X,x_{0})$
is the induced group morphism from the projection.
\end{theorem}

This means we have the following commutative diagram for the universal
property of lifts:
\begin{equation}
\xymatrix{ & (\widetilde{X},\widetilde{x}_{0})\ar[d]^{p}\\
(Y,y_{0})\ar@{-->}[ur]^{\widetilde{f}} \ar[r]_{f} &(X,x_{0})}
\end{equation}
Observe, in particular, if $Y$ is simply-connected, then $f$ can
always be lifted.

\subsection{Homology Theory}

There are many different homology theories. Historically the first was
simplicial homology. It is still used today, since it is very easy to
compute (in the sense that: a computer can efficiently compute it).

\begin{definition}
We define an \define{$n$-Simplex} to be the convex hull of $n+1$
points $v_{0}$, $v_{1}$, \dots, $v_{n}\in\RR^{n+1}$ which are not
contained in any $(n-1)$-dimensional hyperplane. The simplex is
denoted by $[v_{0},v_{1},\dots,v_{n}]$ and defined as:
\begin{equation}
[v_{0},v_{1},\dots,v_{n}] :=
\{t_{0}v_{0}+\cdots+t_{n}v_{n}\mid(\forall
j\ldotp t_{j}\geq0),\;\sum^{n}_{j=0}t_{j}=1\}.
\end{equation}
These coefficients $t_{j}$ are called the \define{Barycentric Coordinates}.
\end{definition}

\begin{remark}
The criterion is that the $n+1$ vectors $v_{j}$ are ``affinely independent'',
meaning that the $n$ vectors $v_{1}-v_{0}$, \dots, $v_{n}-v_{0}$ are
linearly independent.
\end{remark}

\begin{definition}[Subsimplices and faces of a simplex]
In general, any subset of the vertices $\{v_{0},\dots,v_{n}\}$ span a
\define{Subsimplex} of $[v_{0},\dots,v_{n}]$, and the vertices are
ordered according to their order in the larger simplex.

The $(n-1)$ subsimplices of the $n$-simplex are called the
\define{Faces} of the simplex.
\end{definition}

\begin{definition}
If $\Delta^{n}$ is a simplex, then its interior
$\interior\Delta^{n}=\Delta^{n}\setminus\boundary\Delta^{n}$ is called
an \define{Open $n$-Simplex}.
\end{definition}

\begin{example}
When $n=0$, then a 0-simplex is simply a point $\bullet$.

When $n=1$, then a 1-simplex is simply a line segment.

When $n=2$, then a 2-simplex is just a triangle:
\begin{equation*}
\includegraphics{img/img.13}
\end{equation*}
When $n=3$, then a 3-simplex is a tetrahedron:
\begin{equation*}
\includegraphics{img/img.14}
\end{equation*}
For $n>3$, we cannot easily draw $n$-simplices.
\end{example}

\begin{definition}
A \define{$\Delta$-Complex Structure} is a CW complex such that
\begin{enumerate}
\item every cell is an open simplex, and
\item the interior of each face of this simplex is also a cell.
\end{enumerate}
\end{definition}

\begin{remark}
Hatcher (Chapter 2, Section 1) says that a $\Delta$-complex structure
on a space $X$ is a
collection of maps $\sigma_{\alpha}\colon\Delta^{n}\to X$ with $n$
depending on the index $\alpha$ such that
\begin{enumerate}
\item The restriction $\sigma_{\alpha}|_{\interior\Delta^{n}}$ is
  injective and each point of $X$ is in the image of exactly one such
  restriction;
\item Each restriction of $\sigma_{\alpha}$ to a face of $\Delta^{n}$
  is one of the maps $\sigma_{\beta}\colon\Delta^{n-1}\to X$. Here we
  are identifying the faces of $\Delta^{n}$ with $\Delta^{n-1}$ by the
  canonical linear homeomorphism between them that preserves the
  ordering of the vertices; and
\item A set $A\subset X$ is open iff $\sigma^{-1}_{\alpha}(A)$ is open
  in $\Delta^{n}$ for each $\sigma_{\alpha}$.
\end{enumerate}
The conditions (2) and (3) are necessary when $X$ is the quotient
space $Y/\sim$ and we have a $\Delta$-complex for $Y$ (and we wish to
use it to construct a $\Delta$-complex for $X$).
\end{remark}

\begin{example}
For $\sphere{1}$, what kind of $\Delta$-complex structure can we impose?
We see the following CW complex for $\sphere{1}$,
\begin{equation*}
\includegraphics{img/img.15}
\end{equation*}
consists of a 0-simplex and a 1-simplex. So this is a valid
$\Delta$-complex for the circle.

The CW complex,
\begin{equation*}
\includegraphics{img/img.16}
\end{equation*}
consists of two 0-simplices and two 1-simplices. This is another valid
$\Delta$-complex for the circle.

We also see the CW complex,
\begin{equation*}
\includegraphics{img/img.17}
\end{equation*}
consists of three 0-simplices and three 1-simplics. It's another valid
$\Delta$-complex for the circle.
\end{example}

\begin{example}
For the sphere $\sphere{2}$, we can draw a CW complex
\begin{equation*}
\includegraphics{img/img.18}
\end{equation*}
We place three vertices on the equator. The upper hemisphere is one
triangle, and the lower hemisphere is another triangle. So we glue two
tetrahedra together.
\end{example}

\begin{example}
Consider the torus $T^{2}$. We can form it by gluing the sides of the
unit square together as in the following diagram:
\begin{equation*}
\includegraphics{img/img.19}
\end{equation*}
This is a CW complex, but not a $\Delta$-complex. (There is no
consistent ordering of 1-cells which preserves the ordering of the
faces of the 2-simplex.) We can turn it into
a $\Delta$ complex by adding a diagonal edge
\begin{equation*}
\includegraphics{img/img.20}
\end{equation*}
This gives us a $\Delta$-complex consisting of one 0-simplex, three
1-simplices, and two 2-simplices.
\end{example}

\begin{example}
More generally, the regular $4g$-gon may be given a $\Delta$-complex
using the following triangulation:
\begin{equation*}
\includegraphics{img/img.21}
\end{equation*}
(Observe this will require $4g-2$ 2-simplices.)
This will be useful for constructing the $\Delta$-complex of an
orientable surface of genus $g$.
\end{example}

\begin{definition}[Simplicial complex]
A \define{Simplicial Complex} is a $\Delta$-complex such that every
cell is determined by its set of vertices. (This means any two distinct cells
cannot have the same set of vertices.)
\end{definition}

\begin{example}
The following is \textbf{NOT} a simplicial complex of $\sphere{1}$:
\begin{equation*}
\includegraphics{img/img.15}
\end{equation*}
The vertices for the zero-cell = the vertices for the 1-cell. But we
can form a simplicial complex with the following CW complex:
\begin{equation*}
\includegraphics{img/img.22}
\end{equation*}
We see each 1-simplex is written $e^{1}_{i,j}$ is determined as $[v_{i},v_{j}]$.
This is uniquely determined from the set of vertices $\{v_{i},v_{j}\}$.
\end{example}

\begin{example}
For the sphere $\sphere{n}$, we may form a simplicial complex using
$\boundary\Delta^{n+1}$. 
\end{example}

\begin{example}
For the torus $T^{2}$, the $\Delta$-complex fails to be a simplicial
structure. We need to subdivide it further (into a 3-by-3 subsquares,
each subdivided into two 2-simplices, for a total of eighteen 2-simplices).
\end{example}

\subsection{Barycentric Subdivisions}

\begin{node}
Suppose $\Gamma$ is a $\Delta$-complex structure. Then we may obtain
$\Gamma^{(1)}$ whose vertices are the baricenters [centers of mass] of each simplex in $\Gamma$
adjoined to the vertices of $\Gamma$.
\end{node}

\begin{example}
The barycenters of the 2-simplex may be drawn as follows:
\begin{equation*}
\includegraphics{img/img.23}
\end{equation*}
Here we have drawn the ``old'' 2-simplex using dashed lines and open
circles for the vertices. The baricenters for each 1-simplex (and the
entire 2-simplex) are drawn as black dots, and the new edges we would
add are drawn as solid lines. We would need six 2-simplices to
describe this new complex (whereas initially we had one 2-simplex).
\end{example}

\begin{remark}
More generally, each $n$-simplex is subdivided into $(n+1)!$ $n$-simplices,
so this grows hyperexponentially.
\end{remark}

\begin{remark}[Obtaining a $\Delta$-complex]
The point is that applying the barycentric subdivision \underline{\textbf{\emph{twice}}}
will give us a simplicial complex from a $\Delta$-complex. But this is
a bit of a distraction, let us not fixate on the details.
\end{remark}

\begin{remark}
The simplicial complex is a bit too restrictive, the $\Delta$-complex
is more relaxed. However, we specify a simplicial complex by the set
of vertices $V^{(0)}$, say $N=|V^{(0)}|$. [Then we need $N$-bits to
  specify an $n$-simplex --- it's a $k$-simplex if there are exactly
  $k$ ones in the $N$-bit object\dots I think?]
\end{remark}

\begin{definition}
A simplicial structure on a space $X$ is called a
\define{Triangulation} of $X$.
\end{definition}

\begin{theorem}[Manolescu, 2013]
For each $n\geq5$, there exists $n$-manifolds without any triangulations.
\end{theorem}

The proof may be found in \arXiv{1303.2354}.

\begin{theorem}
Every topological space $X$ is weakly homotopically equivalent to a CW complex.
\end{theorem}

\begin{remark}
We often put an orientation on $n$-simplices. Here an ``orientation''
means an ordering on the vertices modulo even permutations.
\end{remark}

\begin{example}
For $n=1$, $[a,b]=-[b,a]$.

For $n=2$, we have
\begin{equation}
\begin{split}
[a,b,c] &= -[b,a,c] = +[b,c,a]\\
&= -[a,c,b] = +[c,a,b] = -[c,b,a]
\end{split}
\end{equation}
For $n=0$, we just artificially impose some order in a purely formal
way, writing $+a$ or $-a$.
\end{example}