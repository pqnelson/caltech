%%
%% winter-lecture12.tex
%% 
%% Made by Alex Nelson <pqnelson@gmail.com>
%% Login   <alex@lisp>
%% 
%% Started on  2026-01-31T09:59:57-0800
%% Last update 2026-01-31T09:59:57-0800
%% 

\lecture{}

We will prove a ``uniqueness theorem'' for $K(G,n)$, but first we need
to prove a useful lemma.

\begin{lemma}
Let $X$ be a CW complex of the form $X=\left(\bigvee_{\alpha\in A}\sphere{n}_{\alpha}\right)\cup_{\varphi_{\beta}}\bigcup_{\beta\in B}e^{n+1}_{\beta}$
for some $n\geq1$, then for any path-connected $Y$ and for any map
$f\colon\pi_{n}(X)\to\pi_{n}(Y)$ there exists a continuous
$\varphi\colon X\to Y$ such that the induced map $\varphi_{*}=f$.
\end{lemma}

\begin{proof}
We define $\varphi$ cell-by-cell (we have one 0-cell for the wedge
sum, a bunch of $n$-cells, and a bunch of $(n+1)$-cells).

The basepoint $x_{0}\in\bigvee_{\alpha\in A}\sphere{n}_{\alpha}$ is
sent to the basepoint $y_{0}\in Y$. We don't have a choice here.

For each $\sphere{n}_{\alpha}$, we define $\varphi$ as follows:
$\pi_{n}(X)$ is generated by (the induced morphisms from the inclusions):
\begin{equation}
\begin{split}
[i_{\alpha}]\colon&\sphere{n}\to X\\
&\sphere{n}\mapsto\sphere{n}_{\alpha}
\end{split}
\end{equation}
Then $f([i_{\alpha}])\in\pi_{n}(Y)$ is represented by an actual map
$\varphi_{\alpha}\colon\sphere{n}\to Y$. Then we define $\varphi$ on
$\sphere{n}_{\alpha}$ to be this $\varphi_{\alpha}$. (Now, our task is
to extend $\varphi$ to each cell $e^{n+1}_{\beta}$.)

Then to extend $\varphi$ to $e^{n+1}_{\beta}$, it is the same as
giving a homotopy from
\begin{equation}
\varphi|_{\psi_{\beta}(\boundary e^{n+1}_{\beta})}\colon\sphere{n}\to Y,
\end{equation}
to the constant map (i.e., a null homotopy of
$\varphi\circ\psi_{\beta}\colon\sphere{n}\to Y$). But
\begin{subequations}
  \begin{align}
[\varphi\circ\varphi_{\beta}] &= \varphi_{*}[\psi_{\beta}]\\
&=0\in\pi_{n}(Y).
  \end{align}
\end{subequations}
So we can extend $\varphi$ to $e^{n+1}_{\beta}$.
\end{proof}

\begin{proposition}[Uniqueness of $K(G,n)$]
The homotopy type of a CW complex $K(G,n)$ is uniquely determined by
$G$, $n$.
\end{proposition}

(We could imagine a $K(G,n)$ which is not a CW complex---this
proposition says nothing about that situation. So the uniqueness claim
is, from a skeptical perspective, rather limited.)

\begin{proof}
Let $K$ be the Eilenberg--MacLane space $K(G,n)$ we constructed last
time (by patching cells of dimension at least $n+2$ to $X$). Let $K'$
be another CW $K(G,n)$. It suffices to show $K\homotopic K'$ (since
homotopy equivalence is an equivalence relation, specifically due to
the transitivity property).

By Whitehead's Theorem~\ref{thm:whitehead}, it is enough to construct
a map $f\colon K\to K'$ such that $f_{*}=\id$ on $\pi_{n}$ for all
$n$.

By our previous Lemma, we get a map $f\colon X\to K'$ such that
$f_{*}=\id\colon\pi_{n}(X)\to\pi_{n}(K'$ where $X\subset K$. Then
since $K\setminus X$ consists of cells of dimension at least $n+2$, it
is enough to extend $f$ to a map $f\colon K\to K'$ (then automiatcally
$f_{*}$ is an isomorphism for all $i$). By the Extension
Lemma~\ref{lemma:extension}, such extension exists if $\pi_{i}(K')=0$
for $i\geq n+1$---which is true as $K'$ is a $K(G,n)$.
\end{proof}

\subsection{Hurewicz Theorem}

\begin{theorem}[Hurewicz]\label{thm:homotopy:Hurewicz}
If a space $X$ is $(n-1)$-connected for $n\geq2$, then
$\widetilde{H}_{i}(X;\ZZ)=0$ for $i\leq n-1$ and $H_{n}(X;\ZZ)\iso\pi_{n}(X)$.
\end{theorem}

\begin{node}[Relative version]
If a pair $(X,A)$ is $(n-1)$-connected for $n\geq2$ and if
$A\neq\emptyset$ is simply-connected, then $H_{i}(X,A)=0$ for $i\leq n-1$
and $H_{n}(X,A;\ZZ)\iso\pi_{n}(X,A)$.
\end{node}

\begin{remark}
\begin{enumerate}
\item The slogan is ``The first nontrivial homology group and homotopy
  group agrees.''
\item The $n\geq2$ is necessary, since in general the first homology
  group is the Abelianization of the fundamental group $H_{1}(X;Z)=\pi_{n}(X)/[\pi_{n}(X),\pi_{n}(X)]$.
\item If $X$ is not simply-connected, there is a more general
  statement (a surjection relating homology and homotopy groups).
\item This is the best we can do for the relations between $H_{i}$ and
  $\pi_{i}$. (Just look at $\sphere{n}$ where we have
  $\widetilde{H}_{i}$ is nontrivial for exactly one $i$, but $\pi_{i}$
  is nontrivial for $i\geq n$.)

  (Moore spaces are another class of examples where things get tricky.)
\end{enumerate}
\end{remark}

\begin{proof}
Using the CW approximation (\S\ref{thm:CW-approximation}), we can
assume (for the absolute case) $X$ is a CW complex (for the relative
case, we assume $(X,A)$ is a CW pair).

Note: the relative case can be reduced to the absolute case by the
quotient property $\pi_{i}(X,A)\iso\pi_{i}(X/A)$ for $i\leq n$ (since
$(X,A)$ is $(n-1)$-connected and $A$ is 1-connected) and
$H_{i}(X,A)\iso\widetilde{H}_{i}(X/A)$ for all $i$. So we're left to
prove the absolute case.

Also using the proof of the CW approximation, we may assume $X$ is a
CW complex such that the $(n-1)$-skeleton of $X$ is a single
point. That is to say, all nontrivial cells have dimension at least
$n$ (since $X$ is $n$-connected). Since cells of dimension at least
$n+2$ do not affect $\pi_{i}(X)$ or $H_{i}(X)$ for $i\leq n$,
we may assume that $X$ is of the form
\begin{equation}
X=\left(\bigvee_{\alpha\in A}\sphere{n}_{\alpha}\right)\cup_{\varphi_{\beta}}\bigcup_{\beta\in B}e^{n+1}_{\beta}.
\end{equation}
Then using the cellular homology for $\widetilde{H}_{i}(X)$, we have
$\widetilde{H}_{i}(X)=0$ for $i\leq n-1$. So it is enough to show that
$\pi_{n}(X)\iso\widetilde{H}_{n}(X)$.

We computed previously the homotopy group
\begin{equation}
\pi_{n}(X)=\coker\left(\bigoplus_{\beta\in B}\ZZ\xrightarrow{\bigoplus[\varphi_{\beta}]}\bigoplus_{\alpha\in A}\ZZ\right).
\end{equation}
Now, we just need the homology group.

For $\widetilde{H}_{n}(X)$, the claim is the part that's relevant is
the following
\begin{equation}
\vcenter{\xymatrix{
H_{n+1}(X^{n+1},X^{n})\ar@{=}[d] & H_{n}(X^{n},X^{n-1})\ar@{=}[d] & \\
C_{n+1}\ar[r]^{d}\ar[d]^{\iso} & C_{n}\ar[r]\ar[d]^{\iso} & C_{n-1}=0\\
\bigoplus_{\beta\in B}\ZZ\langle
e^{n+1}_{\beta}\rangle\ar[r]^{d}&\bigoplus_{\alpha\in A}\ZZ\langle e^{n}_{\alpha}\rangle& }}
\end{equation}
Defining $d$ by
\begin{equation}
d(e^{n+1}_{\beta})=\sum_{\alpha\in A}\deg(\varphi_{\beta}|_{\alpha})\langle e^{n}_{\alpha}\rangle
\end{equation}
which is the same as $\bigoplus\varphi_{\beta}$, where we write
\begin{equation}
\begin{array}{rccc}
\varphi_{\beta}|_{\alpha}\colon&\pi_{n}(\sphere{n})&\to&\pi_{n}(\sphere{n})\\
& \iso  & & \iso\\
\deg(\varphi_{\beta}|_{\alpha})\colon& H_{n}(\sphere{n}) & \to & \ZZ
\end{array}
\end{equation}
using the gluing map
\begin{equation}
\varphi_{\beta}\colon(\boundary e^{n+1}_{\beta}\iso\sphere{n})\to\bigvee_{\alpha\in A}\sphere{n}_{\alpha}
\end{equation}
restricted to the $\alpha$-th sphere,
\begin{equation}
\varphi_{\beta}|_{\alpha}\colon\boundary e^{n+1}_{\beta}\xrightarrow{\varphi_{\beta}}\bigvee_{\alpha\in A}\sphere{n}_{\alpha}\to\sphere{n}_{\alpha},
\end{equation}
the result follows.
\end{proof}

\begin{corollary}
A map $f\colon X\to Y$ between simply-connected CW complexes $X$ and
$Y$ is a homotopy equivalence if the induced map $f_{*}\colon H_{i}(X;\ZZ)\to H_{i}(Y;\ZZ)$ 
is an isomorphism for all $i$.
\end{corollary}

\begin{proof}
The trick: we can replace $Y$ by the mapping cylinder (\S\ref{defn:mapping-cylinder})
\begin{equation}
M_{f} = \bigl((X\times[0,1])\sqcup Y\bigr)/\bigl((x,1)\sim f(x)\bigr),
\end{equation}
we may assume the map $f$ is just an inclusion $X\into Y$.

Since $X$ and $Y$ are both simply-connected, we have $\pi_{1}(Y,X)=0$
by the long exact sequence of homotopy groups, so we can use Hurewicz
Theorem~\ref{thm:homotopy:Hurewicz}. By the long exact sequence for
the homology of the pair $(Y,X)$, we have
\begin{equation}
f_{*}\colon H_{i}(X)\to H_{i}(Y)
\end{equation}
is an isomorphism since $H_{i}(Y,X)=0$ for all $i$. So by Hurewicz,
$\pi_{i}(Y,X)=0$ for all $i$. Then by the long exact sequence for
homotopy groups of the pair, we have
\begin{equation}
f_{*}\colon\pi_{n}(X)\to\pi_{n}(Y)
\end{equation}
is an isomorphism for all $n$. So $f$ is a homotopy equivalence by Whitehead's Theorem~\ref{thm:whitehead}.
\end{proof}

\begin{definition}
The \define{Hurewicz Map} $h\colon\pi_{n}(X,A,x_{0})\to H_{n}(X,A;\ZZ)$
is defined as follows: Fix a generator $\alpha\in H_{n}(\sphere{n};\ZZ)\iso\ZZ$,
then for all $[f]\in\pi_{n}(X,A,x_{0})$, $f\colon(\disk{n},\boundary\disk{n},\point{p})\to(X,A,x_{0})$,
define
\begin{equation}
h([f]):=f_{*}(\alpha)\in H_{n}(X,A;\ZZ).
\end{equation}
\end{definition}

\begin{remark}
This Hurewicz map $h$ is a concrete map which realizes the isomorphism
in Hurewicz's theorem.
\end{remark}

\begin{xca}% Hatcher, prop 4.36, PDF page 378
Prove the Hurewicz map $h$ is a group morphism for $n>1$ (otherwise,
we don't have groups).
\end{xca}

\begin{theorem}[General Hurewicz]% Hatcher, Theorem 4.37, PDF page 380
If $(X,A)$ is an $(n-1)$-connected CW pair, and $n\geq2$ and $A\neq\emptyset$,
then
\begin{enumerate}
\item $H_{i}(X,A)=0$ for all $i<n$; and
\item  $h\colon\pi_{n}(X,A)\to H_{n}(X,A)$ is surjective with the
  kernel generated by things of form $[\gamma]\cdot[f]-[f]$ where
  $[f]\in\pi_{n}(X,A)$ (or $n=2$, $[f]$ is in the normal closure of $\pi_{n}(X,A)$) and $[\gamma]\in\pi_{1}(A)$
\end{enumerate}
\end{theorem}

\begin{proof}
Omitted because it is unenlightening. The curious reader may consult Hatcher~\cite[Th~4.37]{hatcher2002algebraic} for a proof.
\end{proof}