%%
%% fall-lecture19.tex
%% 
%% Made by Alex Nelson <pqnelson@gmail.com>
%% Login   <alex@lisp>
%% 
%% Started on  2025-11-11T07:55:46-0800
%% Last update 2025-11-11T07:55:46-0800
%% 

\lecture{}

\begin{corollary}[Borsuk--Ulam]
Let $f\colon\sphere{n}\to\RR^{n}$. Then there exists $x\in\sphere{n}$
such that $f(x)=f(-x)$.
\end{corollary}

\begin{proof}
Assume for contradiction $f(x)\neq f(-x)$ for all $x\in\sphere{n}$.
Then we define
\begin{equation}
g(x) := \frac{f(x)-f(-x)}{\|f(x)-f(-x)\|},
\end{equation}
which is a well-defined function $g\colon\sphere{n}\to\sphere{n-1}$
and $g$ is odd. Consider $g|_{\sphere{n-1}}$ where
$\sphere{n-1}\subset\sphere{n}$ is the equator. Then this restriction
is also odd. Then $\deg(g|_{\sphere{n-1}})\neq0$ is odd. But
$g|_{\disk{n}_{+}}$ restricted to the Northern hemisphere will give us
a nullhomotopy of $g|_{\sphere{n-1}}$. So $\deg(g|_{\sphere{n-1}})=0$
which is a contradiction. Hence the result.
\end{proof}

\begin{corollary}[Ham Sandwich]
Suppose $A_{1}$, \dots, $A_{n}\subset\RR^{n}$ are measurable sets with
finite measures. Then there exists an $(n-1)$-dimensional hyperplane
$P$ which divides each $A_{i}$ in half (as determined by the measure).
\end{corollary}

The intuition is that the $A_{i}$ are the bread and slices of ham, and
we can cut the sandwhich into exactly half.

\begin{proof}
We first place $\RR^{n}\subset\RR^{n+1}$ with the identification
$(x_{1},\dots,x_{n})\mapsto(x_{1},\dots,x_{n},0)$. Then we chooise a
point $x_{0}\in\RR^{n+1}$ equal to $x_{0}=(0,\dots,0,1)$. Given an
$(n-1)$-dimensional hyperplane $P\subset\RR^{n}$, we can extend it to
an $n$-dimensional hyperplane $Q\subset\RR^{n+1}$ such that $P\subset Q$ 
and $x_{0}\in Q$. Now, we can choose a unit normal vector $\vec{v}$ of
$Q$ with base-point $x_{0}$. (Importantly, each possible $\vec{v}$ corresponds
to a different $P$.) Suppose $P$ divides $A_{i}$ into two parts, let
$m_{i}$ be the measure of the part of $A_{i}$ on the same side of $P$
as $\vec{v}$ points towards. Then we have a map
\begin{equation}
\begin{split}
f\colon &\sphere{n}\to\RR^{n}\\
&\vec{v}\mapsto(m_{1},\dots,m_{n}),
\end{split}
\end{equation}
which is a continuous map. Then the Borsuk--Ulam theorem implies
$f(\vec{v})=f(-\vec{v})$ which implies the hyperplane divides each
$A_{i}$ of equal size subsets. Hence the results.
\end{proof}

\begin{theorem}[Polynomial Sandwich Theorem]
Given $N=\binom{k+n}{n}-1$ measurable sets with finite measure $A_{1}$,
\dots, $A_{N}$.
Then there exists a polynomial $p\in\RR[x_{1},\dots,x_{n}]$ of degree $\deg(p)\leq k$
such that for all $i$, $\mu(\{x\in A_{i}\mid p(x)\geq0\})=\mu(\{x\in A_{i}\mid p(x)\leq0\})$.
(When $k=1$, we recover the usual Ham Sandwich Theorem.)
\end{theorem}

We will not prove this.

\subsection{Simplicial Approximation}

\begin{node}
Let $K$ and $L$ be two simplicial complexes. Let $f\colon K\to L$ be continuous.

\textsc{Goal:} Approximate $f$ with a \define{simplicial map} $g$, which
means if $\sigma$ is a simplex in $K$ then $g(\sigma)$ is a simplex in
$L$ and $g|_{\sigma}$ is affine. (Simplicial maps are determined by
where it sends vertices of a simplex $\sigma$ of $K$.) Such a $g$ is
determined by its values on vertices.
\end{node}

\begin{node}
If $g\colon K\to L$ is a simplicial map, then $g$ induces a chain map
$g_{\sharp}\colon C^{\Delta}(K)\to C^{\Delta}(L)$. If $\sigma=[v_{0},\dots,v_{n}]$
is a simplex in $K$, then
\begin{equation}
g_{\sharp}(\sigma) = \begin{cases}[g(v_{0}),\dots,g(v_{n})] & \mbox{in }L\\
0 & \mbox{if $g(v_{i})=g(v_{j})$ for some $i$, $j$}
\end{cases}
\end{equation}
\end{node}

\begin{definition}
Let $K$ be a simplicial complex, let $\sigma$ be a simplex.
We define the \define{[Closed] Star} of $\sigma$ to be $\ClosedStar(\sigma)$ the union
of all simplices containing $\sigma$.

We define the \define{Open Star} of $\sigma$ to be $\OpenStar(\sigma)$
the union of interiors of all simplices containing $\sigma$.
\end{definition}

\begin{theorem}[Simplicial Approximation]
Let $K$ be a finite simplicial complex, let $L$ be a simplicial complex.
Let $f\colon K\to L$ be continuous.
Then $f$ is homotopic to a map $g$ which is simplicial with respect to
some Barycentric subdivision of $K$.

Moreover, $g$ satisfies $f(\sigma)\subset\OpenStar(g(\sigma))$ for
each simplex $\sigma$ of the subdivision. (So ``$f$ and $g$ are not
too far apart from each other''.)
\end{theorem}

We will not prove this theorem, but instead consider its applications.

\begin{definition}
Let $X$ be a space such that $H_{*}(X)$ has finite rank.
Let $f\colon X\to X$ be continuous. Let $f_{q}\colon H_{q}(X)\to H_{q}(X)$
be the induced map, in the sense that: if $g\colon A\to A$ is a map of
an Abelian group $A$ with torsion subgroup $T$, then $\bar{g}\colon A/T\to A/T$
may be viewed as a matrix $\bar{g}\colon\ZZ^{n}\to\ZZ^{n}$ for some $n\in\NN$
which we refer to as the induced map.

We define the \define{Lefschetz Number} of $f$ is
$\tau(f) := \sum_{q}(-1)^{q}\tr(f_{q})$.
\end{definition}

\begin{remark}
\begin{enumerate}
\item This isn't a typo, I didn't mean ``Lipshitz''. This is named
  after Solomon Lefschetz.
\item The Lefschetz number may be viewed as a generalization of the
  Euler characteristic.
\end{enumerate}
\end{remark}

\begin{example}
$\tau(\id_{X})=\chi(X)$ is just the Euler characteristic. (See? The
  Lefschetz number really does generalize the Euler characteristic.)
\end{example}

\begin{lemma}
Let $C_{*}$ be a chain complex and $\rank(C_{*})<\infty$ is finite rank.
Let $\varphi\colon C_{*}\to C_{*}$ be a chain map.
Then $\tau(\varphi)=\tau(\varphi_{*})$, where
$\tau(\varphi)=\sum_{q}(-1)^{q}\rank(\varphi|_{C_{q}})$. 
\end{lemma}

In other words, the Lefschetz number on the chain complex is the same
as the Lefschetz number on the homology group. This generalizes the
same property of the Euler characteristic.

\begin{theorem}[Lefschetz fixed-point theorem]
Let $X$ be a finite simplicial complex.
Let $f\colon X\to X$ be continuous.
If $\tau(f)\neq0$, then $f$ has at least one fixed-point.
\end{theorem}

% This proof was actually at the start of the next lecture, but it
% felt "cleaner" to place it here.

\begin{proof}
Since $X$ is a finite simplicial complex, we can put a metric $d$ on $X$.
If $f$ has no fixed-point, then $d(x,f(x))>0$ for all $x\in X$.
But $X$ is compact, and this means there exists an $\varepsilon>0$
such that $d(x,f(x))>\varepsilon$ for all $x\in X$.

Now, we just choose a barycentric subdivision $L$ of $X$ such that
every simplex has a diameter less than $\varepsilon$.

Moreover, by the simplicial approximation theorem, there exists a
simplicial map $g$ homotopic to $f$ such that [the conditions of the
  theorem are satisfied].

Assume for contradiction there exists a simplex $\sigma$ such that $g(\sigma)\cap\sigma\neq\emptyset$,
then let $x\in g(\sigma)\cap\sigma$.
Then $f(x)\in f(\sigma)\propersubset\star(g(\sigma))$.
So there exists a simplex $\sigma'$ such that $f(x)\in\sigma'$ and $g(\sigma)\propersubset\sigma'$.
But $x\in g(\sigma)\implies x\in\sigma'$, so $f(x)\in\sigma'$ and $x\in\sigma'$,
so
\begin{equation}
d(x,f(x))\leq\diam(\sigma')<\varepsilon,
\end{equation}
which gives us a contradiction. This means
$g(\sigma)\cap\sigma=\emptyset$ for every simplex $\sigma$.

Consider $g_{\sharp}\colon C^{\Delta}_{*}(L)\to C^{\Delta}_{*}(L)$,
which (if we were to write it out as a matrix) has no nonzero diagonal entry
--- otherwise $g(\sigma)\cap\sigma\neq\emptyset$ if $g_{\sharp}$ has a
nonzero diagonal entry. So $\tau(f)=\tau(g)=\tau(g_{\sharp})=0$.
Hence we proved the contrapositive of the theorem.
\end{proof}

\begin{remark}
The Brouwer fixed-point theorem is an immediate corollary, since
$\tau(f)=1$ for $f\colon\disk{1}\to\disk{1}$.
\end{remark}