%%
%% fall-lecture14.tex
%% 
%% Made by Alex Nelson <pqnelson@gmail.com>
%% Login   <alex@lisp>
%% 
%% Started on  2025-10-30T11:15:50-0700
%% Last update 2025-10-30T11:15:50-0700
%% 

\lecture{}

\begin{node}
\begin{enumerate}
\item Our goal will be to relate the homology of CW complexes to the
  homology of singular complexes.
\item It is a useful fact that, for CW complexes $X$, if
  $e^{n-1}_{\beta}$ is an $(n-1)$-cell, then the quotient of the
  $(n-1)$-skeleton satisfies $X^{n-1}/(X^{n-1}\setminus e^{n-1}_{\beta})=\closure{e^{n-1}_{\beta}}/\boundary\closure{e^{n-1}_{\beta}}$
  (which is just homeomorphic to the $\sphere{n}$).
\end{enumerate}
\end{node}

\begin{theorem}
The cellular homology is isomorphic to the singular homology, $H^{CW}_{*}(X)=H_{*}(X)$.
\end{theorem}

First, we need to relate the chain complexes.

\begin{observation}
We see that the CW chain complex satisfies
\begin{subequations}
\begin{equation}
C^{CW}_{n}(X)\iso H_{n}(X^{n},X^{n-1}),
\end{equation}
but
\begin{equation}
H_{n}(X^{n},X^{n-1})\iso \widetilde{H}_{n}(X^{n}/X^{n-1}),
\end{equation}
and from the opening remark, we see
\begin{equation}
\widetilde{H}_{n}(X^{n}/X^{n-1}) = \widetilde{H}_{n}(\bigvee_{\alpha}\sphere{n}_{\alpha})=\bigoplus_{\alpha}\widetilde{H}_{n}(\sphere{n}_{\alpha}),
\end{equation}
since
\begin{equation}
X^{n}/X^{n-1}\iso\bigvee_{\alpha}\sphere{n}_{\alpha}.
\end{equation}
\end{subequations}
\end{observation}

\begin{node}[Boundary map]
What is the boundary map? Well,
\begin{equation}
\vcenter{\xymatrix{H_{n}(X^{n},X^{n-1})\ar[rr]^{d_{n}}\ar[dr]_{\boundary_{n}} & & H_{n-1}(X^{n-1},X^{n-2})\\
& H_{n-1}(X^{n-1})\ar[ur]_{j_{*}}}}
\end{equation}
where $\boundary_{*}$ is the connecting morphism in the long exact
sequence for the pair $(X^{n},X^{n-1})$, and $j_{*}$ is the induced
morphism from the quotient map
\begin{equation}
j_{*}\colon C_{*}(X^{n-1})\to C_{*}(X^{n-1},X^{n-2}).
\end{equation}
Why? Well, just look at
\begin{equation}
d_{\alpha\beta}=\deg(\boundary e^{n}_{\alpha}\xrightarrow{\text{bdry}}X^{n-1}\xrightarrow{\text{quotient}}X^{n-1}/(X^{n-1}\setminus e^{n-1}_{\beta}))
\end{equation}
and that's what's going on here. We won't prove it here.

Now, we can look at the fragment of the long-exact sequences
\begin{equation}
\vcenter{\xymatrix{
H_{n+1}(X^{n+1},X^{n})\ar[d]_{\boundary_{n+1}}\ar[dr]^{d_{n+1}} & & \\
H_{n}(X^{n})\ar[r]^{j_{n}} & H_{n}(X^{n},X^{n-1})\ar[r]^{\boundary_{n}}\ar[dr]_{d_{n}} & H_{n-1}(X^{n-1})\ar[d]^{j_{n-1}}\\
& & H_{n-1}(X^{n-1},X^{n-2})}}
\end{equation}
When we look at the second row, it's just part of the long exact
sequence for the pair $(X^{n},X^{n-1})$. We can complete the diagram,
filling in a few objects and arrows
\begin{equation}\label{eq:fall-lec14:messy-diagram}
\vcenter{\xymatrix{
&H_{n+1}(X^{n+1},X^{n})\ar[d]_{\boundary_{n+1}}\ar[dr]^{d_{n+1}} & & H_{n-1}(X^{n})\ar[d]\\
H(X^{n-1})\ar[r]&H_{n}(X^{n})\ar[d]\ar[r]^-{j_{n}} & H_{n}(X^{n},X^{n-1})\ar[r]^{\boundary_{n}}\ar[dr]_{d_{n}} & H_{n-1}(X^{n-1})\ar[d]^{j_{n-1}}\\
&H_{n}(X^{n+1})\ar[d]& & H_{n-1}(X^{n-1},X^{n-2})\\
&H_{n}(X^{n+1},X^{n})& &}}
\end{equation}
The left and right columns (which we have added more arrows vertically
to) are obtained using long exact sequences.
\end{node}

\begin{lemma}
If $X'$ is attached from $X$ by attaching an $n$-cell, then
$H_{i}(X')\iso H_{i}(X)$ if $i\neq n-1,n$.
\end{lemma}

\begin{proof}
  Look at the long exact sequence
  \begin{equation}
H_{i+1}(X',X)\to H_{i}(X)\to H_{i}(X')\to H_{i}(X',X).
  \end{equation}
  But $H_{i+1}(X',X)=0$ and $H_{i}(X',X)=0$ when $i\neq n-1,n$. This
  gives us the isomorphism. Furthermore
  \begin{equation}
H_{k}(X',X)\iso\widetilde{H}_{k}(X'/X)\iso\widetilde{H}_{k}(\sphere{n})
  \end{equation}
  only in dimension $k=n$.
\end{proof}

\begin{corollary}
We have
$H_{i}(X^{n})=0$ if $i>n$, and $H_{i}(X^{n})\iso H_{i}(X)$ if $i<n$.
\end{corollary}

\begin{node}
Returning to the diagram in Equation~\eqref{eq:fall-lec14:messy-diagram},
we see that the previous results allow us to replace some of the
entries with zeros
\begin{equation}
\vcenter{\xymatrix{
&H_{n+1}(X^{n+1},X^{n})\ar[d]_{\boundary_{n+1}}\ar[dr]^{d_{n+1}} & & 0\ar[d]\\
0\ar[r]&H_{n}(X^{n})\ar[d]\ar[r]^-{j_{n}} & H_{n}(X^{n},X^{n-1})\ar[r]^{\boundary_{n}}\ar[dr]_{d_{n}} & H_{n-1}(X^{n-1})\ar[d]^{j_{n-1}}\\
&H_{n}(X^{n+1})\ar[d]& & H_{n-1}(X^{n-1},X^{n-2})\\
&0& &}}
\end{equation}
\textsc{Goal:} We want to prove $H_{n}\iso\ker(d_{n})/\Im(d_{n+1})$.

We see that $j_{n+1}$ is injective, which implies
\begin{equation}
\ker(d_{n})=\ker(\boundary_{n})=\Im(j_{n})=j_{n}(H_{n}(X^{n})).
\end{equation}
Now, what is $\Im(d_{n+1})$? We see by commutativity of the diagram,
\begin{subequations}
\begin{align}
\Im(d_{n+1}) &= j_{n}(\Im(\boundary_{n}))\\
&= j_{n}(\ker(i_{n})).
\end{align}
\end{subequations}
Since $j_{n}$ is injective, we know
\begin{subequations}
\begin{align}
\ker(d_{n})/\Im(d_{n+1}) &= j\bigl(H_{n}(X^{n})/\ker(i_{n})\bigr)\\
&\iso H_{n}(X^{n})/\ker(i_{n}),
\end{align}
but $i_{n}$ is surjective, so basic algebra gives us
\begin{align}
\ker(d_{n})/\Im(d_{n+1}) &\iso \Im(i_{n})\\
&\iso H_{n}(X^{n+1}).
\end{align}
\end{subequations}
Hence the result.
\end{node}

\subsection{Euler Characteristic}

\begin{definition}
Let $C_{*}$ be a finitely-generated chain complex, so there are only
finitely many $C_{n}$ and each are finitely-generated as Abelian groups.
Let $\alpha_{n}=\rank(C_{n})$. (Recall, a finitely-generated Abelian
group $A\iso\ZZ^{r}\oplus T$ where $T$ is a finite group called the
Torsion group; here $r=\rank(A)$.) Then the \define{Euler Characteristic} of $C$
is defined to be
\begin{equation*}
\chi(C):=\sum_{n}(-1)^{n}\alpha_{n}.
\end{equation*}
\end{definition}

\begin{definition}
The $n^{th}$ \define{Betti Number} of $C$ is defined to be $\beta_{n}:=\rank(H_{n}(C_{*}))$.
\end{definition}

\begin{definition}
The \define{Poincar\'{e} Polynomial} of $C$ is defined to be 
\begin{subequations}
\begin{equation}
P(t) := \sum_{n}\beta_{n}t^{n}
\end{equation}
the generating function for the Betti numbers of $C$. Some authors
give the following as the Poincar\'{e} polynomial
\begin{equation}
Q(t) := \sum_{n}\alpha_{n}t^{n}
\end{equation}
(the generating function for the $\alpha_{n}=\rank(C_{n})$).
\end{subequations}
\end{definition}

\begin{theorem}
There exists a polynomial with non-negative coefficients $R(t)$ such
that $(1-t)R(t) = Q(t)-P(t)$.
\end{theorem}

\begin{proof}
Let
\begin{equation}
C_{n+1}\xrightarrow{\boundary_{n+1}} C_{n}\xrightarrow{\boundary_{n}}C_{n-1},
\end{equation}
let $\xi_{n}=\rank(\ker(\boundary_{n}))$ and $b_{n}=\rank(\Im(\boundary_{n+1}))$.
Then $C_{n}/\ker(\boundary_{n})\iso\Im(\boundary_{n})$.

This implies
\begin{equation}
\alpha_{n}-\xi_{n}=b_{n}-1.
\end{equation}
We also know that, by definition,
\begin{equation}
\ker(\boundary_{n})/\Im(\boundary_{n+1})=H_{n}(C).
\end{equation}
Then $\xi_{n}-b_{n}=\beta_{n}$, by looking atthe ranks of the groups 
$\ker(\boundary_{n})$, $\Im(\boundary_{n+1})$, and $H_{n}(C)$. Then
\begin{subequations}
\begin{align}
Q(t)-P(t) &= \sum_{n}(\xi_{n}+b_{n-1})t^{n}-(\xi_{n}-b_{n})t^{n}\\
&= \sum_{n}(b_{n-1}+b_{n})t^{n}\\
&= (1+t)\sum_{n}b_{n}t^{n},
\end{align}
\end{subequations}
and $b_{n}\geq0$ since it's a rank. Hence the result.
\end{proof}

\begin{corollary}[Euler-Poincar\'{e} formula]
We see $Q(-1)=P(-1)$, i.e.,
\begin{equation*}
\chi(C) = \sum_{n}(-1)^{n}\beta_{n}.
\end{equation*}
\end{corollary}

\begin{example}
Given a convex polyhedron, we know
\begin{equation}
\#(\mbox{vertices})-\#(\mbox{edges})+\#(\mbox{faces})=2.
\end{equation}
This is Euler's famous formula. The polyhedron gives a CW complex of
$\sphere{2}$, and Euler's formula is an alternating sum of Betti numbers.
\end{example}

\begin{corollary}[Morse inequality]
For any $q\geq0$,
\begin{equation*}
\alpha_{q}-\alpha_{q-1}+\cdots+(-1)^{q}\alpha_{0}\geq\beta_{q}-\beta_{q-1}+\cdots+(-1)^{q}\beta_{0}.
\end{equation*}
\end{corollary}

\begin{proof}
This follows immediately from $Q(t)-P(t)=(1-t)R(t)$, just truncate it
up to order $q$ and then set $t=-1$.
\end{proof}

\begin{remark}
Why prove formulas like this? Well, the right-hand side of Morse's
inequality depends only on the topology of the space. The left-hand
side tells us something about the number of cells in the complex. So
from thjis, we can estimate the number of cells (or a lower bound for
it). Why do this? Well, Morse theory needs this. So it really tells us
something about the critical points of a function.
\end{remark}

\begin{definition}
If $X$ is a finite CW complex, we can just define the
\define{Euler Characteristic} of $X$ to be,
\begin{equation}
\chi(X) := \sum_{n}(-1)^{n}\alpha_{n}=\sum_{n}(-1)^{n}\rank(H_{n}(X))
\end{equation}
by Euler-Poincar\'{e}.
\end{definition}

\begin{example}
If $\Sigma_{g}$ is an orientable closed surface of genus $g$, we know
\begin{equation}
H_{k}(\Sigma_{g}) = \begin{cases}
\ZZ & \mbox{if }k=0,2\\
\ZZ^{2g} & \mbox{if }k=1,
\end{cases}
\end{equation}
and therefore $\chi(\Sigma_{g})=2-2g$.
\end{example}

\begin{example}
Closed nonorientable surface of genus $g$ which we denoted as $\Pi_{g}$
has its Euler characteristic be $\chi(\Pi_{g})=2-g$.
(Why $\Pi_{g}$ as notation for this space? Because $\Pi$ is Greek for
``P'', and since the projective plane is not orientable, we use $\Pi$
for it.)
\end{example}

\begin{example}
If $T$ is a tree of $n$ vertices, then $T$ has $n-1$ edges.
We see $T$ is a contractible graph, then $\chi(T)=1$. Hence we see
there must be $n-1$ edges for this to hold.
\end{example}