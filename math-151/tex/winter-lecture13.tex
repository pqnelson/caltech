%%
%% winter-lecture13.tex
%% 
%% Made by Alex Nelson <pqnelson@gmail.com>
%% Login   <alex@lisp>
%% 
%% Started on  2026-02-03T10:03:34-0800
%% Last update 2026-02-03T10:03:34-0800
%% 

\lecture[Fiber Bundles]{}

\begin{definition}
A \define{Fiber Bundle} is a map $p\colon E\to B$ (where $E$ is
called the ``total space'' and $B$ is called the ``base'' [or ``base space''])
such that
\begin{enumerate}
\item\textsc{Fibers are the same:} there exists a space $F$ such that
  for any $b\in B$, we have $p^{-1}(\{b\})\iso F$ --- we usually just
  refer to $F$ as the \define{Fiber} of the fiber bundle; and
\item\textsc{Locally trivializable:} for each $b\in B$ there exists a
  \define{Chart} $U\subset B$ such that $b\in U$ and there exists a
  \define{Local Trivialization} homeomorphism $h\colon p^{-1}(U)\to U\times F$ such that the
  following diagram commutes
\begin{equation}
\vcenter{\xymatrix{p^{-1}(U)\ar[rr]^{h}\ar[dr]_{p}& &\ar[dl]^{\pi_{1}}U\times F\\
& U &}}
\end{equation}
where $\pi_{1}\colon U\times F\to U$ is the usual projection $\pi_{1}(u,f)=u$.
\end{enumerate}
\end{definition}

\begin{example}
The covering space $p\colon\widetilde{X}\to X$ where for each $x\in X$,
there exists a neighborhood $U\subset X$ such that $x\in U$ and
$p^{-1}(U)=\bigsqcup_{i}U_{i}$. This is when the fiber is a discrete set.
\end{example}

\begin{example}
Mobius strip has the fiber be an interval, and the base $B=\sphere{1}$.
\end{example}

\begin{example}
Vector bundles where the fiber $F$ is a vector space. (The linear
structure on $p^{-1}(U)$ is the one imposed by $h\colon p^{-1}(U)\to U\times F$.)
\end{example}

\begin{example}[Hopf fibration]
We can consider a fibration $p\colon\sphere{3}\to\sphere{2}$ with
fiber $\sphere{1}$ obtained as follows: we look at $\CP^{1}$,
\begin{subequations}
  \begin{align}
\sphere{2}\iso\CP^{1}&=\bigl(\{(z,w)\in\CC^{2}\setminus\{(0,0)\}\bigr)/(\lambda
z,\lambda w)\sim(z,w)\\
\intertext{where $\lambda=r\E^{\I\theta}$ lets us factorize the
  symmetry as}
&=\{(z,w)\in\CC^{2}\mid |z|^{2}+|w|^{2}=1\}/(\E^{\I\theta}z,\E^{\I\theta}w)\sim(z,w)\\
&=\sphere{3}/\sphere{1}
  \end{align}
\end{subequations}
There is a smooth action of $\sphere{1}$ on $\sphere{3}$, so we get a map
\begin{equation}
\begin{split}
&\sphere{1}\times\sphere{3}\to\sphere{3}\\
&(\E^{\I\theta},(z,w))\mapsto(\E^{\I\theta}z,\E^{\I\theta}w)
\end{split}
\end{equation}
and this is the action we modded out by to give us $\sphere{2}=\sphere{3}/\sphere{1}$.
\end{example}

\begin{node}[Long exact sequence]
There is a long exact sequence of homotopy groups associated to fiber
bundles, which looks like
\begin{equation}
\vcenter{\xymatrix{
& \dots\ar[r] & \pi_{n+1}(B)\ar[lld]\\
\pi_{n}(F) \ar[r]& \pi_{n}(E)\ar[r] & \pi_{n}(B)\ar[lld]\\
\pi_{n-1}(F) \ar[r] & \dots & }}
\end{equation}
We divide the proof into two steps:
\begin{enumerate}
\item For any map $p\colon E\to B$ satisfying the \emph{homotopy
lifting property} with respect to $\disk{k}$, we have such a long
  exact sequence;
\item AFor a fiber bundle $p\colon E\to B$, this property is always satisfied.
\end{enumerate}
\end{node}

\begin{definition}\label{defn:homotopy-lifting-property}
A map $p\colon E\to B$ is said to have the \define{Homotopy Lifting Property
with respect a space $X$} if a given homotopy $g_{t}\colon X\to B$
(i.e., a $G\colon X\times[0,1]\to B$) and given a lifting
$\widetilde{g}_{0}\colon X\to E$ of $g_{0}$ such that
$p\circ\widetilde{g}_{0}=g_{0}$, then there exists a lifting
$\widetilde{g}_{t}\colon X\to E$ (of the whole family) such that
$p\circ\widetilde{g}_{t}=g_{t}$, i.e., the following diagram commutes
\begin{equation}
\vcenter{\xymatrix{ & E\ar[d]^{p}\\
X\ar@{..>}[ur]^{\widetilde{g}_{t}}\ar[r]_{g_{t}} & B}}
\end{equation}
Note: there is also a ``pair'' version of this notion.
\end{definition}

\begin{definition}
A map $g\colon E\to B$ is said to have the \define{Homotopy Lifting
  Property with respect to a pair $(X,A)$} if for each homotopy
$f_{t}\colon X\to B$ with $g_{t}=f_{t}|_{A}\colon A\to B$ and a given
lift $\widetilde{f}_{0}\colon X\to E$ and a lift
$\widetilde{g}_{0}\colon A\to E$, there exists a lift
$\widetilde{f}_{t}\colon X\to E$ extending both.
\end{definition}

\begin{definition}
A map $p\colon E\to B$ which has the homotopy lifting property with
respect to any space $X$ is called a \define{Fibration}.
\end{definition}

\begin{remark}
The homotopy lifting property with respect to $\disk{k}$ is equivalent
to the homotopy lifting property with respect to $(\disk{k},\boundary\disk{k})$.
The pictures we should have in mind are, when drawing this as a
homotopy, the disk $\disk{k}$ [in red] and its boundary
$\boundary\disk{k}$ [in blue]:
\begin{center}
\includegraphics{img/img.60}
\end{center}
and the relevant part for the homotopy lifting property is
$(\disk{k}\times\{0\})\cup(\boundary\disk{k}\times I)$ (where
$\disk{k}\times\{0\}$ is identified as $\disk{k}$ [drawn in red], and
$\boundary\disk{k}\times I$ is drawn in blue):
\begin{center}
\includegraphics{img/img.61}
\end{center}
So what we're saying is that if we know the boundary conditions (drawn
in red and blue), we
can extend it to the whole cube $\disk{k}\times I$.

\textsc{Note:}\marginpar{{\footnotesize $J^{k}$ for boundary conditions}} it is common to work with $I^{k}\times I$ instead of
$\disk{k}\times I$, and it is common to refer to the 3 edges of this
$I^{k+1}\iso I^{k}\times I$ are $J^{k}$.
\end{remark}

\begin{remark}
By induction on each cell, the homotopy lifting property with respect
to $\disk{k}$ for all $k\geq0$ is equivalent to the homotopy lifting
property with respect to any CW pair $(X,A)$.
\end{remark}

\begin{definition}
A map $p\colon E\to B$ is called a \define{Serre Fibration} if it has
the homotopy lifting property with respect to all the discs $\disk{k}$
for all $k\geq0$.
\end{definition}