%%
%% fall-lecture02.tex
%% 
%% Made by Alex Nelson <pqnelson@gmail.com>
%% Login   <alex@lisp>
%% 
%% Started on  2025-10-02T11:47:57-0700
%% Last update 2025-10-02T11:47:57-0700
%% 

\lecture[CW Complexes]

\begin{definition}[$n$-Cell]
Let $n\in\NN_{0}$ be a natural number. We define an \define{$n$-Cell}
to be an \emph{open} $n$-dimensional disc when $n\neq0$, and when
$n=0$ it is just a point.
\end{definition}

\begin{definition}[CW Complex]
A \define{CW Complex} is a topological space
\begin{equation}
X = \bigcup^{\infty}_{n=0}X^{n}
\end{equation}
where $X^{0}$ consists of 0-cells with discrete topology.

When $n>0$, $X^{n}$ is obtained from $X^{n-1}$ by attaching $n$-cells,
so
\begin{equation}
X^{n}=X^{n-1}\cup_{f}\left(\bigsqcup\disk{n}\right)
\end{equation}
where $f\colon\boundary(\bigsqcup\disk{n})\to X^{n-1}$ is used to
attach the $n$-cells to $X^{n-1}$. Some authors write
$f^{n}_{\alpha}\colon\disk{n}_{\alpha}\to X$ and call it ``the
characteristic map of the cell $\disk{n}_{\alpha}$''.
\end{definition}

\begin{example}
Consider $X=S^{1}$ the circle, and $X^{0}=\point{x}$. Then
$X^{1}=S^{1}$ is obtained by mapping the boundary of $\disk{1}$ to $X^{0}$.
\end{example}

\begin{example}
Let $X=S^{n}$ be the $n$-sphere. We have $X^{0}=\point{x}$, and
$X^{k}=X^{0}$ for all $k<n$. Then $X^{n}=S^{n}$ using $f\colon\boundary\disk{n}\to\point{x}$.
This is what Albert Schwarz called ``spheroids''!
\end{example}

\begin{example}
Let $X=T^{2}$ be the torus. Then $X^{0}=\point{x}$, $X^{1}$ is drawn
thus:

Then $f\colon\boundary\disk{2}\to X^{1}$ sends $\boundary\disk{2}$ to
$ab\bar{a}\bar{b}$ where $\bar{a}$ is the $a$ in the opposite
orientation.

We can obtain a CW complex for the torus $T^{2}$ by gluing the
opposite edges of the unit square together, the interior of the square
is the 2-cell.
\end{example}

\begin{remark}
The $n$-cell is not necessarily an open subset in $X$, the use of the word
``open'' is confusing. We're referring to the interior of the closed
disk, i.e., $\interior{\disk{n}}$.
\end{remark}

\begin{theorem}
If $X_{1}$ and $X_{2}$ are CW complexes, then $X_{1}\times X_{2}$ is a
CW complex.
\end{theorem}

\begin{proof}[Proof sketch]
If $e_{1}$ is a $k$-cell in $X_{1}$, and if $e_{2}$ is an $m$-cell in
$X_{2}$, then $e_{1}\times e_{2}$ is an $(k+n)$-cell in $X_{1}\times X_{2}$.
This gives us the method for constructing all the cells in
$X_{1}\times X_{2}$.
\end{proof}

\begin{node}[Two conditions]
\begin{enumerate}
\item \textsc{Closure finiteness:} the closure of each cell meets only
  finitely many other cells.
\item \textsc{Weak topology:} A set $W\subset X$ is closed iff its
  preimage $(f^{n}_{\alpha})^{*}W$ is closed in $\disk{n}_{\alpha}$
  for all $n$, $\alpha$. (This tells us how to put a topology on the space.)
\end{enumerate}
(These two conditions are why it's called a CW complex --- it
satisfies the ``\underline{\textbf{C}}losure finiteness'' and
``\underline{\textbf{W}}eak topology'' conditions.)

Observe these conditions are satisfied automatically for finite CW
complexes.
\end{node}

\begin{example}
We can pick for $S^{1}$ the CW complex where $X^{0}$ consists of two
distinct points, and $X^{1}$ is obtained by attaching one interval for
``the Northern hemisphere'' and another interval for ``the Southern hemisphere''.
This construction can be iterated by treating $S^{1}$ as the equator,
then attach a disk $\disk{2}$ as its ``Northern hemisphere'', and
another $\disk{2}$ as its ``Southern hemisphere''.

We can continue iterating this by attaching two $\disk{n}$ to
$S^{n-1}$ to obtain the CW complex for $S^{n}$. We see that
$S^{0}\subset S^{1}\subset S^{2}\subset\dots$. Then we could define
\begin{equation}
S^{\infty} := \bigcup^{\infty}_{n=0}S^{n}.
\end{equation}
This differs from, e.g., defining the unit sphere in a Banach space as
consisting of all points which are a distance 1 from the origin.

What good is this construction? Well, it is invariant under
the antipodal map, which sends one $k$-cell to the other $k$-cell, for
each $k=0,\dots,n$.
\end{example}

\begin{example}[Real projective spaces]
Now, we see that 
\begin{equation}
S^{n}/(\mbox{antipodal map})=\RP^{n}.
\end{equation}
We thus get a CW complex describing $\RP^{n}$, with one $k$-cell for
each $k=0,1,\dots,n$. This also gives us a way to define
\begin{equation}
\RP^{\infty}:=S^{\infty}/(\mbox{antipodal map}).
\end{equation}
\end{example}

\begin{example}[Complex projective spaces]
Consider
\begin{equation}
\CP^{n}=\bigl(\CC^{n+1}\setminus\{0\}\bigr)/\sim
\end{equation}
where $z\sim\lambda z$ for any $z\in\CC^{n+1}\setminus\{0\}$ and
$\lambda\in\CC\setminus\{0\}$. We can use homogeneous coordinates to
write the underlying set as
\begin{equation}
\CP^{n}=\{[x_{0}:x_{1}:\cdots:x_{n}]\in\CC^{n+1}\mid\exists i\ldotp x_{i}\neq0\}.
\end{equation}
Now, suppose we set $x_{n}=0$, which lets us see that $\CP^{n-1}\propersubset\CP^{n}$
by the identification. We see
\begin{subequations}
\begin{align}
\CP^{n}\setminus\CP^{n-1}&=\{[x_{0}:x_{1}:\cdots:x_{n}]\in\CC^{n+1}\mid x_{n}\neq0\}\\
&=\{[\widetilde{x}_{0}:\widetilde{x}_{1}:\cdots:1]\in\CC^{n+1}\}
\end{align}
\end{subequations}
where $\widetilde{x}_{i}=x_{i}/x_{n}$ and $\widetilde{x}_{n}=1$. We
see then that there are $n$ complex variables $\widetilde{x}_{i}$ for
$i=0,1,\dots,n-1$, which means we have a copy of $\CC^{n}$. In symbols,
\begin{equation}
\CP^{n}\setminus\CP^{n-1}=\CC^{n}.
\end{equation}
We can describe $\CC^{n}$ using a $2n$-cell. This gives us a CW
complex for $\CP^{n}$ by adding a $2n$-cell to a CW complex for
$\CP^{n-1}$. Then by induction, $\CP^{n}$ has a CW complex consisting
of a single $2k$-cell for each $k=0,1,\dots,n$.
\end{example}

\begin{definition}[CW Subcomplex]
Let $X$ be a CW complex, we call $A\subset X$ a \define{Subcomplex} of $X$
if $A$ is closed and $A$ is the union of cells in $X$. In this way, we
find $(X,A)$ sometimes called a \define{CW Pair} in the literature.
\end{definition}

\begin{example}
$S^{1}\subset S^{2}\subset\dots\subset S^{\infty}$ gives us many subcomplexes.
\end{example}

\begin{example}
$\CP^{1}\subset\CP^{2}\subset\dots\subset\CP^{\infty}$ where
  $\CP^{\infty}=\bigcup^{\infty}_{n=0}\CP^{n}$, is another way to
  obtain many subcomplexes.
\end{example}

\begin{theorem}[Morse]
Every smooth manifold is homotopy equivalent to a CW complex.
\end{theorem}

(So if you want to construct something which is not a CW complex, then
you have to try very hard.)

\subsection{Fundamental Groups}

\begin{definition}[Fundamental group at a basepoint]
If you have a space $X$ and a base point $x_{0}\in X$,
then the \define{Fundamental Group at $x_{0}$} is defined to be:
\begin{equation}
\pi_{1}(X,x_{0}) = \{f\colon[0,1]\to X\mid f(0)=f(1)=x_{0}\}/\sim,
\end{equation}
where $f\sim g$ if $f\homotopic g\rel\boundary[0,1]$.

The group structure on $\pi_{1}(X,x_{0})$ is given by the binary
operator $[f]\cdot[g]$ represented by the path
\begin{equation}
(f\cdot g)(t) = \begin{cases}f(2t) & \mbox{if }t\leq 1/2\\
g(2t-1) & \mbox{if }t>1/2.
  \end{cases}
\end{equation}
Then $[f]^{-1}$ is represented by $\bar{f}(t)=f(1-t)$, and the
identity element $[e]$ corresponds to the path $e(t)=x_{0}$ for all $t\in[0,1]$.
\end{definition}

\begin{node}[Relating fundamental group at different base points]
Suppose we have a path $h\colon[0,1]\to X$ where $h(0)=x_{0}$ and $h(1)=x_{1}$.
Then we obtain a map
\begin{equation}
\beta_{h}\colon\pi_{1}(X,x_{0})\to\pi_{1}(X,x_{1})
\end{equation}
sending $[f]\mapsto[h]\cdot[f]\cdot[h]^{-1}$ (topologically: travel
along $h$, then $f$, then backwards along $h$). This $\beta_{h}$ is a
group isomorphism. In particular, if $X$ is path connected, then
$\pi_{1}(X)$ is the isomorphism type of $\pi_{1}(X,x_{0})$ for any
base point $x_{0}\in X$. Often we just omit the $x_{0}$.
\end{node}
