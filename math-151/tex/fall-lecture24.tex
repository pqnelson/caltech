%%
%% fall-lecture24.tex
%% 
%% Made by Alex Nelson <pqnelson@gmail.com>
%% Login   <alex@lisp>
%% 
%% Started on  2025-11-22T11:21:57-0800
%% Last update 2025-11-22T11:21:57-0800
%% 

\lecture{}

\subsection{Cap Product}

\begin{definition}
Let $R$ be a commutative ring, $X$ be a space.
We define the \define{Cap Product} to be
\begin{equation}
-\frown-\colon C_{k}(X;R)\times C^{\ell}(X;R)\to C_{k-\ell}(X;R)
\end{equation}
for all $k\geq\ell$, defined by acting on the simplex
$\sigma\colon\Delta^{k}=[v_{0},\dots,v_{k}]\to X$ and any $\varphi\in C^{\ell}(X;R)$ by
\begin{equation}
\sigma\frown\varphi:=\varphi(\sigma|_{[v_{0},\dots,v_{\ell}]})\cdot\sigma|_{[v_{\ell},\dots,v_{k}]}.
\end{equation}
\end{definition}

\begin{proposition}[Leibniz rule]
We have
$\boundary(\sigma\frown\varphi)=(-1)^{\ell}((\boundary\sigma)\frown\varphi-\sigma\frown\coboundary\varphi)$.
\end{proposition}

\begin{corollary}
The cap product induces a map
$H_{k}(X;R)\times H^{\ell}(X;R)\to H_{k-\ell}(X;R)$,
which makes $H_{*}(X;R)$ a right module over $H^{*}(X;R)$.
\end{corollary}

\begin{proof}[Proof idea]
One thing we'd need to check is on the chain level
$(\sigma\frown\varphi)\frown\psi=\sigma\frown(\varphi\smile\psi)$.
From this, we'd get $H_{*}(X;R)$ is a module over the cohomology ring.
\end{proof}

\begin{node}
If $\alpha\in C_{k+\ell}(X;R)$, $\varphi\in C^{k}(X;R)$, and $\psi\in C^{\ell}(X;R)$,
then $\psi(\alpha\frown\varphi)=(\varphi\smile\psi)(\alpha)$. This may
be viewed as a special case of the corollary.
\end{node}

\begin{corollary}
The map $[\varphi]\smile-\colon H^{\ell}(X;R)\to H^{k+\ell}(X;R)$ is
dual to $-\frown[\varphi]\colon H_{k+\ell}(X;R)\to H_{k}(X;R)$.
\end{corollary}

\begin{proof}[Proof idea]
From the universal coefficient theorem, we can write $H^{\ell}(X;R)\iso\hom(H_{\ell}(X),R)\oplus\Ext(H_{\ell-1}(X),R)$,
and this gives us the way to replace the dual map $\frown[\varphi]$ on
a short-exact sequence yielding $[\varphi]\smile$ acting on a short
exact sequence.
\end{proof}

\begin{remark}
In infinite-dimensional homology, it does not always make sense to
speak of the cohomology ring, but the cap product remains a
well-defined notion.
\end{remark}

\subsection{K\"{u}nneth Formula}

\begin{node}
Let $X$ and $Y$ be spaces. We want to compute $H_{*}(X\times Y)$ from
$H_{*}(X)$ and $H_{*}(Y)$. The K\"{u}nneth formula accomplishes this task.
\end{node}

\begin{definition}[Tensor product of chain complexes]
Let $(C_{*},\boundary)$ and $(C'_{*},\boundary')$ be two chain complexes.
Then their \define{Tensor Product} is the chain complex
$(\overline{C}_{*},\overline{\boundary})$ where
\begin{equation}
\overline{C}_{n}=\bigoplus_{i+j=n}C_{i}\otimes C'_{j}\mbox{ as Abelian groups},
\end{equation}
and for any $x\in C_{i}$ and $y\in C'_{j}$, we have
\begin{equation}
\overline{\boundary}(x\otimes y)=(\boundary x)\otimes y+(-1)^{i}x\otimes\boundary'y.
\end{equation}
To prove this is well-defined, we should check that
$\overline{\boundary}^{2}=0$, so we see
\begin{subequations}
  \begin{align}
\overline{\boundary}^{2}(x\otimes y) &= (\boundary^{2}x)\otimes
y+(-1)^{i+1}(\boundary x)\otimes(\boundary'y)+(-1)^{i}(\boundary x)\otimes(\boundary'y)
+(-1)^{2i}x\otimes(\boundary'^{2}y)\\
&=(-1)^{i}(1-1)(\boundary x)\otimes(\boundary'y)\\
&=0
  \end{align}
\end{subequations}
\end{definition}

\begin{lemma}
Suppose $0\to A_{*}\to B_{*}\to C_{*}\to )$ is a split short exact
sequence of chain complexes, and suppose $G_{*}$ is a chain complex.
Then
\begin{equation}
0\to A_{*}\otimes G_{*}\to B_{*}\otimes G_{*}\to C_{*}\otimes G_{*}\to0
\end{equation}
is a split short exact sequence.
\end{lemma}

\begin{theorem}[Algebraic K\"{u}nneth Formula]
Suppose $C$ and $C'$ are chain complexes. If $C$ is free,
then for all $n\in\NN_{0}$ there exists a natural short exact sequence
\begin{equation}
0\to\bigoplus_{i}H_{i}(C)\otimes H_{n-i}(C')\to H_{n}(C\otimes C')\to\bigoplus_{i}\Tor(H_{i}(C),H_{n-i-1}(C'))\to0
\end{equation}
which is split.
\end{theorem}

This should remind us of the universal coefficient theorem for
homology (\S\ref{universal-coefficient-theorem-homology}). Even the requirement that one of the guys is free should
ring a bell.

\begin{proof}
First, if $\boundary=0$ in $C$, then the conclusion is true. We can
see this because
\begin{subequations}
\begin{equation}
\bigoplus_{i}H_{i}(C)\otimes H_{n-i}(C')=\bigoplus C_{i}\otimes H_{n-i}(C'),
\end{equation}
and
\begin{equation}
H_{n}(C\otimes C')=\bigoplus_{i}C_{i}\otimes H_{n-i}(C'),
\end{equation}
and
\begin{equation}
\bigoplus_{i}\Tor(H_{i}(C),H_{n-i-1}(C'))=\bigoplus\Tor(C_{i},H_{n-i-1}(C')),
\end{equation}
\end{subequations}
and then the conclusion follows immediately.

Now, let $Z_{*}=\ker(\boundary)$ and $B_{*}=\Im(\boundary)$. Then we
have
\begin{equation}
0\to Z_{k}\to C_{k}\xrightarrow{\boundary} B_{k-1}\to0
\end{equation}
is a short exact sequence by definition (and it's split since $C$ is free).
Then we can use our Lemma nad tensor with $C'$, we get the short exact sequence
\begin{equation}
0\to Z_{k}\otimes C'_{*}\to C_{k}\otimes C'_{*}\xrightarrow{\boundary\otimes\id} B_{k-1}\otimes C'_{*}\to0.
\end{equation}
This gives us a long exact sequence
\begin{equation}
\dots\to H_{n}(Z\otimes C')\to H_{n}(C\otimes C')\to H_{n-1}(B\otimes
C')\to H_{n-1}(Z\otimes C')\to\dots
\end{equation}
which is (using the induced map $i_{n}\colon B_{n}\into Z_{n}$ from
the inclusion $i\colon B\into Z$)
\begin{equation}
\dots\xrightarrow{i_{n}}(Z_{*}\otimes H_{*}(C'))_{n}\to H_{n}(C\otimes
C')\to(B_{*}\otimes H_{*}(C'))_{n-1}\xrightarrow{i_{n-1}}(Z_{*}\otimes H_{*}(C))_{n-1}\to\dots
\end{equation}
Hence
\begin{equation}
0\to\coker(i_{n})\to H_{n}(C\otimes C')\to\ker(i_{n-1})\to0
\end{equation}
is a short exact sequence.

We can consider the short exact sequence (from the definition of
homology alone)
\begin{equation}
0\to B_{k}\xrightarrow{i_{k}}Z_{k}\to H_{k}(C)\to0.
\end{equation}
Then tensoring with $H_{n-k}(C')=G$ gives us
\begin{equation}
\vcenter{\xymatrix{
0\ar[r] & \Tor(B_{k},G)\ar[r]        & \Tor(Z_{k},G)=0\ar[r] &\ar[lld]\Tor(H_{k}(C),G)&\\
        & B_{k}\otimes G\ar[r]_{i_{n}} & Z_{k}\otimes G\ar[r] & H_{k}(C)\otimes G\ar[r]&0}}
\end{equation}
So
\begin{equation}
\bigoplus_{k}\Tor(H_{k}(C),H_{n-k}(C'))=\ker(i_{n}\otimes G),
\end{equation}
where
$i_{n}\otimes G\colon B\otimes G\to Z\otimes G$, and
\begin{equation}
\bigoplus_{k}\Tor(H_{k}(C),H_{n-k}(C'))=\coker(i_{n}\otimes G).
\end{equation}
This gives us an exact short sequence. We still need to prove it is
split, but it is the same as in the universal coefficient theorem, so
we're calling it a day here.
\end{proof}

\begin{remark}
When $C_{*}'=C'_{0}=G$ has only one term, then the tensor product
$C_{*}\otimes C'_{*}=C_{*}\otimes G$. In this case, the K\"{u}nneth
formula is just the Universal Coefficient Theorem for homology (\S\ref{universal-coefficient-theorem-homology}).
Unsurprisingly, the proof for the Algebraic K\"{u}nneth formula
resembles the proof for the Universal coefficient theorem.
\end{remark}

\begin{remark}
See also Hatcher~\cite[Theorem 3B.5]{hatcher2002algebraic} for his presentation of the Algebraic
K\"{u}nneth formula.
\end{remark}

\begin{node}
If $X$ and $Y$ are CW complexes, then $X\times Y$ has an induced CW
complex structure. If $e_{\alpha}$ is a cell in $X$ and $e_{\beta}$ is
a cell in $Y$, then $e_{\alpha}\times e_{\beta}$ is a cell in $X\times Y$.
\end{node}

\begin{proposition}
Let $C_{*}$ be the cellular chain complex, then $C_{*}(X\times Y)\iso C_{*}(X)\otimes C_{*}(Y)$.
\end{proposition}

\begin{proof}[Proof idea]
Check the boundary map (since it's obviously true for the underlying
Abelian groups). Check the boundary maps on two cubes
$\boundary(I^{k}\times I^{\ell})$, then every cell can be
approximated by a cube.
\end{proof}

\begin{corollary}[K\"{u}nneth formula for topological spaces]
If $X$ and $Y$ are CW complexes, then
\begin{equation}
0\to\bigoplus_{i}H_{i}(X)\otimes H_{n-i}(Y)\to H_{n}(X\times Y)\to\bigoplus_{i}\Tor(H_{i}(X),H_{n-i-1}(Y))\to0
\end{equation}
is a natural short exact sequence and it is split.
\end{corollary}