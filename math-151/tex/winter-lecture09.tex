%%
%% winter-lecture09.tex
%% 
%% Made by Alex Nelson <pqnelson@gmail.com>
%% Login   <alex@lisp>
%% 
%% Started on  2026-01-24T09:22:12-0800
%% Last update 2026-01-24T09:22:12-0800
%% 

\lecture{}

\begin{node}
We stress, we have seen how to use a mapping cone to decompose a map
into an inclusion and a deformation retraction.
\end{node}

\begin{remark}
  We are now starting Section~4.2 of Hatcher.

  The goal is to do more computations of $\pi_{n}$.

  We have two main tools: (1) Excision theorem (discussed today), and
  (2) Fiber bundles and fibrations. For $\pi_{1}$, these are
  analogous to (1) van Kampen and (2) covering spaces (respectively).
\end{remark}

\begin{theorem}[Excision]
Let $X$ be a CW space. Suppose $X=A\cup B$ where $A$ and $B$ are CW
subcomplexes of $X$ and $A\cap B=C$ is a subcomplex of both $A$ and $B$.
Assume $C\neq\emptyset$ and $C$ is connected. If $(A,C)$ is
$m$-connected and $(B,C)$ is $n$-connected, then the induced map
\begin{equation*}
i_{*}\colon\pi_{i}(A,C)\to\pi_{i}(X,B)
\end{equation*}
(induced by the inclusion $i\colon (A,C)\into(X,B)$) is an isomorphism
for $i<m+n$ and surjective for $i=m+n$.
\end{theorem}

(We defer the proof until next time!)

\subsection{Application: Suspensions}

\begin{node}
Recall, the \define{Suspension} of a space $X$ is defined as
\begin{equation}
SX := \frac{X\times[0,1]}{(x,0)\sim(y,0), (x,1)\sim(y,1)}
\end{equation}
where we can write this as joining two cones over $X$ together along
$X$, i.e., as $SX=C_{+}X\cup_{X}C_{-}X$.
\end{node}

\begin{example}
We see $S(\sphere{0})=\sphere{1}$, $S(\sphere{1})=\sphere{2}$,
$S(\sphere{2})=\sphere{3}$, \dots, $S(\sphere{n})=\sphere{n+1}$.
\end{example}

\begin{proposition}
If $f\colon X\to Y$, then
\begin{equation}
\begin{split}
Sf\colon&SX\to SY\\
&(x,t)\mapsto(f(x),t)
\end{split}
\end{equation}
is a continuous map. (That is to say, $S-$ is a functor.)
\end{proposition}


\begin{definition}
Let $(X,x_{0})$ be a pointed space. We may define its
\define{Reduced Suspension}
\begin{equation}
\Sigma X:=SX/(x_{0},t)\sim(x_{0},1).
\end{equation}
\end{definition}

\begin{node}
In general, if $X$ is homotopy equivalent to a CW complex, we have
$\Sigma X\homotopic SX$. In particular,
\begin{equation}
\vcenter{\xymatrix@C=0.5pc{\pi_{i}(X)\ar[d]^{S}\ar@{=}[r] & [S^{i},X]& \\
\pi_{i+1}(SX)\ar@{=}[r] & [\sphere{i+1},SX]\ar@{=}[r] &[S(\sphere{i}),S(X)]}}
\end{equation}
where $S\colon\pi_{i}(X)\to\pi_{i+1}(SX)$ is the suspension map.
\end{node}

\begin{theorem}[Freudenthal suspension]
If $X$ is an $(n-1)$-connected CW complex, then the suspension map
$S\colon\pi_{i}(X)\to\pi_{i}(S(X))$ is an isomorphism for $i<2n-1$ and
surjective for $i=2n-1$. (In particular, for $X=\sphere{n}$, we have
$\pi_{i}(\sphere{n})\to\pi_{i}(\sphere{n+1})$ is an isomorphism for $i<2n-1$.)
\end{theorem}

\begin{proof}
Let $A=C_{+}X$, $B=C_{-}X$, $C=A\cap B\iso X$, writing the long exact
sequence of $(CX,X)$
\begin{equation}
\dots\to\pi_{j}(X)\to\underbrace{\pi_{j}(CX)}_{=0}\to\pi_{j}(CX,X)\xrightarrow{\boundary}\pi_{j-1}(X)\to\underbrace{\pi_{j-1}(CX)}_{=0}\to\dots
\end{equation}
We know $CX$ is contractible, so $\pi_{j}(CX,X)\iso\pi_{j-1}(X)$ so
$(CX,X)$ is $n$-connected.

Then by the Excision theorem, the induced map
\begin{equation}
i_{*}\colon\pi_{k}(C_{+}X,X)\to\pi_{k}(SX,C_{-}X)
\end{equation}
is an isomorphism for $k<2n$ and surjective for $k=2n$. By the long
exact sequence of the pair $(SX,C_{-}X)$:
\begin{equation}
\dots\to\pi_{j}(C_{-}X)\to\pi_{j}(SX)\xrightarrow{j_{*}}\pi_{j}(SX,C_{-}X)\to\pi_{j-1}(C_{-}X)\to\dots
\end{equation}
gives $j_{*}\colon\pi_{j}(SX)\to\pi_{j}(SX,C_{-}X)$ is an
isomorphism. The goal is to prove the following diagram:
\begin{equation}
\vcenter{\xymatrix{\pi_{j}(X)\ar[d]^{\boundary^{-1}}\ar[r]^{S} & \pi_{j+1}(SX)\ar[d]^{j_{*}}\\
\pi_{j+1}(C_{+}X,X)\ar[r]^{i_{*}} & \pi_{j+1}(SX,C_{-}X)}}
\end{equation}
We will compute $j_{*}\circ S$ and $i_{*}\circ(\boundary^{-1})$.

(1) Take $f\colon\sphere{i}\to X$ as an element of $\pi_{i}(X)$. Then
take the suspension $\pi_{i}(X)\xrightarrow{S}\pi_{i+1}(SX)$, we have
\begin{equation}
Sf\colon(\sphere{i+1},\point{*})\to(SX,p).
\end{equation}
Then we apply $j$ to $Sf$.

(2) We have $\boundary\colon\pi_{i+1}(C_{+}X,X)\to\pi_{i}(X)$. For
$f\in\pi_{i}(X)$, $[\boundary^{-1}f]$ is represented by
\begin{equation}
\bigl((\disk{i+1},\boundary\disk{i+1})\xrightarrow{g}(C_{+}X,X)\bigr)\mapsto g|_{\boundary\disk{i+1}}
\end{equation}
for any
\begin{equation}
g\colon(\disk{i+1},\boundary\disk{i+1})\to(C_{+}X,X)
\end{equation}
such that $g|_{\boundary\disk{i+1}}=f$. So we take $\boundary^{-1}f$
to be $Cf\colon C(\sphere{i})\to C_{+}X$ when $\disk{i+1}\iso C(\sphere{i})$.
Then this map $i_{*}(\boundary^{-1}f)$ is just viewing $Cf$ as a map
\begin{equation}
(\disk{i+1},\boundary\disk{i+1})\to(C_{+}X,X)\into(SX,C_{-}X).
\end{equation}
Then the claim is
\begin{equation}
[i_{*}(\boundary^{-1}f)]=[j_{*}(Sf)]\in\pi_{i+1}(SX,C_{-}X),
\end{equation}
making the diagram commute.
\end{proof}

\begin{remark}
Since $\pi_{i}(\sphere{n})\iso\pi_{i+1}(\sphere{n+1})$ for $i<2n-1$,
then we can conclude, e.g., for $i=1$ that $n>(i+1)/2=1$ we have
\begin{equation}
0=\pi_{1}(\sphere{2})=\pi_{2}(\sphere{3})=\pi_{3}(\sphere{4})=\dots.
\end{equation}
For $i=2$, we have $n>3/2$, so
\begin{equation}
\pi_{2}(\sphere{2})\iso\pi_{2}(\sphere{3})\iso\pi_{4}(\sphere{4})\iso\pi_{4}(\sphere{5})\iso\dots.
\end{equation}
Generally, for $i=n+k$, we define $\pi_{n+k}(\sphere{k})$ is
independent of $n$ if $n\geq k+2$. We define the \define{Stable Homotopy Group of Spheres}
to be $\pi_{k}^{S}$ or $\pi_{k}(\mathbb{S})$ equal to
\begin{equation}
\pi_{k}(\mathbb{S})=\lim_{n\to\infty}\pi_{n+k}(\sphere{n}).
\end{equation}
\end{remark}

\begin{corollary}
$\pi_{n}(\sphere{n})\iso\ZZ$.
\end{corollary}

\begin{proof}
We just need to compute $\pi_{2}(\sphere{2}$. The rest are the same by
suspensions. We know $\pi_{1}(\sphere{1})\to\pi_{2}(\sphere{2})$ is
surjective, so $\pi_{2}(\sphere{2})$ is either $\ZZ$ or $\ZZ/n\ZZ$.
We want to rule out the latter case. Recall the degree map: if
$f\colon\sphere{2}\to\sphere{2}$ we define $\deg(f)$ to be such that
\begin{equation}
f_{*}[\sphere{2}]=\deg(f)[\sphere{2}]\in H_{2}(\sphere{2};\ZZ).
\end{equation}
Then the $\deg$ is a homotopy invariant. So $\deg\colon\pi_{2}(\sphere{2})\to\ZZ$
is surjective. Hence $\pi_{2}(\sphere{2})\iso\ZZ$.
\end{proof}