%%
%% fall-lecture04.tex
%% 
%% Made by Alex Nelson <pqnelson@gmail.com>
%% Login   <alex@lisp>
%% 
%% Started on  2025-10-07T07:51:53-0700
%% Last update 2025-10-07T07:51:53-0700
%% 

\lecture{}

\begin{example}
Suppose we are given this identity:
\begin{equation}
\frac{1}{(1-t)(1-tq)(1-tq^{2})(\cdots)}=\sum^{\infty}_{n=0}\frac{t^{n}}{(1-q)(\cdots)(1-q^{n})}.
\end{equation}
We may look at this from the perspective of a 19th century
Mathematician (or early 20th century, like Ramanujan) and just
``accept it'' as true because it works. Or we may ask, ``What ring are
we working in?'' We may say something like $\QQ(q)[[t]]$. Is that the
``smallest" ring where such an identity holds?

Well, we can look at the denominator of the right-hand side
\begin{equation}
\varphi=(1-t)(1-tq)(1-tq^{2})(\cdots)
\end{equation}
and we see the coefficients are integers, so we see $\varphi\in\ZZ[q][[t]]$.
Perhaps we could use a smaller ring (ahem, like the
\emph{localization} of some multiplicative subset of $\ZZ[q][[t]]$?).
The professor got this example from Peter Scholze's lecture notes (on $q$-series?).
\end{example}

\begin{proposition}[{Matsumara~\cite[1.H]{matsumura1970commutative}}]
Let $P\in\Spec(A)$, take $S:=A\setminus P$ as a multiplicative set. We
have the following notation for localization of $A$ with respect to $S$:
\begin{equation}
A_{P} := S^{-1}A,
\end{equation}
and for $A$-modules $M$ we have the following notation for the
localization of $M$ with respect to $S$:
\begin{equation}
M_{P}:=S^{-1}M.
\end{equation}
\end{proposition}

\begin{lemma}[{Matsumara~\cite[1.1]{matsumura1970commutative}}]
If $x\in M$ is mapped to $0\in M_{P}$ for all $P\in\MSpec(A)$, then
$x=0$. In other words, the natural map
$M\into\prod_{P\in\MSpec(A)}M_{P}$
is injective.
\end{lemma}

\begin{proof}
We have $x=0$ in $M_{P}$ iff $s\in A\setminus P$ such that $sx=0$ in $M$
iff the annihiliator of $x=m/s$ (where $s\in A\setminus P$) is not
contained in $P$  
\begin{equation}
\Annihilator(x):=\{a\in A\mid ax=0\}\nsubset P.
\end{equation}
But if $x=0$ in all $M_{P}$ for each maximal ideal $P\ideal A$, then
$\Annihilator(x)=A$ which implies $1\cdot x=0$ which implies $x=0$.
\end{proof}

\begin{lemma}[{Matsumara~\cite[1.2]{matsumura1970commutative}}]
Let $A$ be an integral domain with field of fractions $\kk$. Then all
localizations of $A$ may be viewed as subrings of $\kk$, i.e., we have
\begin{equation}
A = \bigcap_{P\in\MSpec(A)}A_{P}.
\end{equation}
\end{lemma}

\begin{proof}
Let $x\in\kk$. Then
\begin{equation}
D=\{a\in A\mid ax\in A\}\ideal A
\end{equation}
is the ideal of denominators of $x$. (Ex: $\frac{1}{6}\in\QQ$, $D(1/6)=(6)\ideal\ZZ$.)
Then $x\in A$ iff $D=A$, and $x\in A_{P}$ iff $D\nsubset P$ (i.e., $D$
is not contained in any maximal ideal) which is a contradiction. Hence
the result.
\end{proof}

\begin{remark}[Big picture: why are we doing this?]
So why are we doing this? The idea is that localization removes a lot
of stuff. The Ring is usually too big to solve a problem we're facing.
So we solve our Problem in some localization (this is a common trick
in number theory). If we can solve it in \emph{every} localization,
then we can solve it in the Ring.
\end{remark}

\begin{proposition}[{Matsumara~\cite[1.I]{matsumura1970commutative}}]
Let $f\colon A\to B$ be a ring morphism. Let $S\subset A$ be a
multiplicative set. Let $S'=f(S)$ be the image of $S$ under $f$. Let
$S^{-1}B$ be the localization of $B$ \emph{as an $A$-module} (since
$f$ gives us an $A$-action on $B$, we can view $B$ as an
$A$-module). Consider $S'^{-1}B$ the localization of $B$ \emph{as a $B$-module}.
Then $S'^{-1}B=S^{-1}B=S^{-1}A\otimes_{A}B$ as sets and as $A$-modules.

In particular, if $I\ideal A$ is an ideal, and $S'$ is the image of
$S$ in $A/I$, we obtain
\begin{equation}
S'^{-1}(A/I) = S^{-1}A/I(S^{-1}A).
\end{equation}
We should think of this as ``dividing by $I$ commutes with localization''.
\end{proposition}

\begin{definition}[Local rings and residue fields (1.K)]
Let $A$ be a ring. We call $A$ \define{Local} if it has only one
maximal ideal. Observe every field is local.

If $A$ is local and $M\ideal A$ is maximal, then we define the
\define{Residue Field} of $A$ to be the field $\kk:=A/M$.

We will write $(A,M)$ or $(A,M,\kk)$ for local rings, their maximal
ideals, and optionally their residue fields.
\end{definition}

\begin{example}
Let $\ZZ_{(7)}=\{\frac{a}{b}\mid a\in\ZZ, 1\leq b\leq 6\}$.
There is an ideal $(7)\ZZ_{(7)}\ideal\ZZ_{(7)}$ and it is the only
maximal ideal. Then
\begin{equation}
\ZZ_{(7)}/((7)\ZZ_{(7)})\iso\ZZ/7\ZZ.
\end{equation}
\end{example}

\begin{proposition}
Let $A$ be any ring, let $P\in\Spec(A)$ be a prime ideal, then $A_{P}$
is local, and we write the residue field for it as $\kk(P)=A_{P}/(PA_{P})$
which is the ring of fractions of $A/P$.
\end{proposition}

\begin{proposition}[{Matsumara~\cite[1.J]{matsumura1970commutative}}]
Let $S\subset A$ be a multiplicative set. Suppose
\begin{equation}
A\xrightarrow{f}B\xrightarrow{g}S^{-1}A
\end{equation}
be morphisms such that $g\circ f\colon A\to S^{-1}A$ is the natural map.
Suppose for any $b\in B$ there exists an $s\in S$ with $f(s)b\in f(A)$.
Then $S^{-1}B=f(S)^{-1}B=S^{-1}A$.

If $A$ is a domain, $P\in\Spec(A)$, $A\subset B\subset A_{P}$,
then $A_{P}=B_{\mathfrak{p}}\iso B_{P}$ where $\mathfrak{p}=PA_{P}\cap B$
and $B_{P}=B\otimes A_{P}$.
\end{proposition}

\begin{proof}
Assume $\forall b\in B\ldotp\exists s\in S\ldotp f(s)b\in f(A)$.
Then $S^{-1}B=f(S)^{-1}B\iso S^{-1}A$.

We see that using the universal property of localization (\S\ref{node:fall:lecture3:universal-property-of-localization}),
the following diagram commutes:
\begin{equation}
\xymatrix{
A\ar[r]^{f} & B\ar[r]^{g}\ar[d] & \ar@{-->}[dl]^{!} S^{-1}A\\
            & S^{-1}B           &}
\end{equation}
Viewed as $A$-modules, we could describe localization as a
functor. Then functoriality gives us
\begin{equation}
\xymatrix{B\ar[rr]^{g}& & S^{-1}A\ar@{=}[d]\\
S^{-1}B\ar[r]^{S^{-1}g} & S^{-1}S^{-1}A \ar@{=}[r] & S^{-1}A}
\end{equation}
Functoriality means
\begin{equation}
  \begin{split}
    g\colon &M\to N\\
    & m\mapsto n=g(m)
  \end{split}
\end{equation}
maps to
\begin{equation}
\begin{split}
  S^{-1}g\colon&S^{-1}M\to S^{-1}N\\
&  \frac{m}{s}\mapsto\frac{g(m)}{s}
\end{split}
\end{equation}
The universal property can be used again
\begin{equation}
\xymatrix{M \ar[d]\ar[r]^{g}\ar[dr] & N\ar[d]\\
S^{-1}M\ar@{-->}[r]^{!} & S^{-1}N}
\end{equation}
Hence the result.
\end{proof}

\begin{example}
An example where the hypotheses do not hold:
\begin{equation*}
\begin{split}
\ZZ\to &\;\ZZ[x]\to\ZZ_{(7)}\\
n\mapsto&\; n\\
&n_{0}+n_{1}x+\cdots \mapsto\frac{n_{0}}{1}
\end{split}
\end{equation*}
But $(\ZZ\setminus(7))^{-1}\ZZ[x]\not\iso\ZZ_{(7)}$.

On the other hand, $\ZZ\subset 3^{-1}\ZZ\subset\ZZ_{(7)}$.
\end{example}

\begin{definition}[Local ring morphisms (1.K)]
Let $(A,\mathfrak{m},\kk)$ and $(B,\mathfrak{m}',\kk')$ be local rings.
If $\psi\colon(A,\mathfrak{m},\kk)\to(B,\mathfrak{m}',\kk')$ is a ring
morphism, then we call $\psi$ \define{Local} if
$\psi(\mathfrak{m})\subset\mathfrak{m}'$. This means it induces a
morphism of fields $\kk\to\kk'$.
\end{definition}

\begin{proposition}[(1.K)]
Let $\psi\colon A\to B$ be a ring morphism.
Then this induces a morphism $\preimage\psi\colon\Spec(B)\to\Spec(A)$
mapping $P\in\Spec(B)$ to $\preimage\psi(P)=\mathfrak{p}=\{a\in A\mid \psi(a)\in P\}$.
But now we see $A\setminus\mathfrak{p}$ is a multiplicative set of
$A$, so $\psi(A\setminus\mathfrak{p})\subset B\setminus P$, and it
induces a mapping
\begin{equation}
\psi_{P}\colon A_{\mathfrak{p}}\to B_{P},
\end{equation}
which \emph{is} a local morphism,
where $\psi_{P}(\mathfrak{p}A_{\mathfrak{p}})\subset\psi_{P}(\mathfrak{p})B_{P}\subset PB_{P}$.
We can factor it as follows:
\begin{equation}
A_{\mathfrak{p}}\to B_{\mathfrak{p}} = A_{\mathfrak{p}}\otimes_{A}B\to B_{P}
\end{equation}
where $B_{\mathfrak{p}}$ is not necessarily a local ring.
\end{proposition}