%%
%% fall-lecture05.tex
%% 
%% Made by Alex Nelson <pqnelson@gmail.com>
%% Login   <alex@lisp>
%% 
%% Started on  2025-10-09T10:13:21-0700
%% Last update 2025-10-09T10:13:21-0700
%% 

\lecture{}

\begin{definition}
Let $A\neq0$ be a ring. Then the \define{Jacobson Radical} of $A$
is the ideal $\JacobsonRadical(A)=\bigcap\MSpec(A)$ obtained by taking
the intersection of all maximal ideals of $A$.
\end{definition}

\begin{remark}
The previous definition works for commutative rings $A$. If we wanted
to consider a general unital ing, we would define
$\JacobsonRadical(A)$ as the ideal consisting of all elements $a\in A$
such that $aM=0$ whenever $M$ is a simple $A$-module. When $A$ is
commutative, we recover the usual Jacobson radical.
\end{remark}

\begin{definition}
A \define{Semi-Local Ring} is a ring $A$ with finitely many maximal
ideals $M_{1}$, \dots, $M_{r}$.
\end{definition}

\begin{proposition}
For a semilocal ring $A$, we see
$\JacobsonRadical(A)=M_{1}\cap\cdots\cap M_{r}=M_{1}(\cdots)M_{r}$
(by the Chinese remainder theorem, $I\cap J=A$ and $IJ=I\cap J$).
\end{proposition}

\begin{proposition}
Let $x\in\JacobsonRadical(A)$. Then $1+x$ is a unit in  $A$.
\end{proposition}

\begin{proof}
Consider the maximal ideal $M$ such that $x\in M$. Then $1+x\in M$,
which implies $1\in M$, which is a contradiction. Hence the result.
\end{proof}

\begin{proposition}
If $I\ideal A$ is an ideal, and if $1+x$ is a unit for all $x\in I$,
then $I\subset\JacobsonRadical(A)$.
\end{proposition}

\begin{nakayama}[Matsumara 1.M]\label{lemma:NAK}
Let $A$ be a ring, let $M$ be a finitely-generated $A$ module. Suppose
$IM=M$ where $I\ideal A$ is an ideal.
Then there exists some $a\in A$ of the form $a=1+x$ where $x\in I$ and $aM=0$.

Moreover, if $I\subset\JacobsonRadical(A)$, then $M=0$.
\end{nakayama}

\begin{proof}
Let $M=Aw_{1}+\cdots+Aw_{s}$ where the $w_{j}$ are generators of $M$
(since $M$ is finitely-generated, by hypothesis). Write $M'=M/A$.
Now by induction on $s$ (the number of generators).

By induction, $\exists x\in I\ldotp(1+x)M'=0\iff(1+x)M\subset Aw_{s}$.
Since $IM=M$, we have
\begin{subequations}
\begin{align}
(1+x)M &= (1+x)(IM)\\
  &=(1+x)IM\\
  &=I(1+x)M\\
  &\subset I(Aw_{s})=Iw_{s}
\end{align}
\end{subequations}
Now $(1+x)w_{s}=yw_{s}$ for some $y\in I$.
Then $(1+x-y)(1+x)M=0$ so $(1+x-y)(1+x)\equiv1\pmod{I}$.
\end{proof}

\begin{remark}
\begin{enumerate}
\item You'd use this when $A$ is local and $I$ is the maximal ideal of
  $A$.
\item There are other formulations of the NAK lemma.
\item The NAK lemma is due to Nakayama, Azumaya, and Krull. It is
  usually just referred to as ``the Nakayama lemma'', although
  Nakayama did not like the name.
\end{enumerate}
\end{remark}

\begin{corollary}
Let $A$ be a ring, let $M$ be an $A$-module, let $I\ideal A$ be an ideal of $A$,
let $N,N'\subset M$ be submodules.
Suppose that $M=N+IN'$ and either
\begin{enumerate}
\item $I$ is nilpotent, or
\item $I\subset\JacobsonRadical(A)$ and $N'$ is a finitely-generated module.
\end{enumerate}
Then $M=N$.
\end{corollary}

\begin{proof}
Per cases.
\begin{enumerate}
\item Assume $I$ is nilpotent. Then $M/N=I(M/N)=I^{2}(M/N)=\cdots=0$
  and eventually $I^{n}=0$. Hence the result.
\item Assume $I\subset\JacobsonRadical(A)$ and $N'$ is a finitely-generated module.
  Then use NAK for $M/N$.
\end{enumerate}
Hence the result.
\end{proof}

\begin{proposition}[1.N]
Let $(A,\mathfrak{m},\kk)$ be a local ring. Let $M$ be an $A$-module.
Suppose either $\mathfrak{m}$ is nilpotent or $M$ is finitely-generated.
Let $G\subset M$ be a subset. Then $G$ generates $M$ iff its image
$\overline{G}$ in $M/\mathfrak{m}M$ generates $M\otimes\kk$.
\end{proposition}

The trick is to realize $M/\mathfrak{m}M=M\otimes_{A}\kk$ which is a
vector space over $\kk$.

(When is a local ring's maximal ideal nilpotent? One example,
$\CC[x]/(x^{2})$ has $(x)$ be maximal and nilpotent.)

\begin{remark}
NAK is analogous to the implicit function theorem from Analysis. We
can take generators of the vector space $M/\mathfrak{m}M$ and lift
them. This then corresponds to extending $M/(x)M$ to an infinitesimal
neighborhood. We are taking $A=\CC[x]$, $M=A^{n}$ free module,
$S=A\setminus(x)$, and $S^{-1}M$ the localization. Elements of
$S^{-1}M$ are like $M$ when we can divide by $x$. Then
$M/\mathfrak{m}M=\CC^{n}$ and we can pick the usual canonical basis
$e_{1}$, \dots, $e_{n}$ and lift this basis.
\end{remark}


\begin{definition}[1.O]
Let $A$ be a ring, let $M$ be an $A$-module. An element $a\in A$ is
called \define{$M$-Regular} if it is not a zero-divisor on $M$; i.e.,
if $M\to aM$ defined by $m\mapsto am$ is injective.

The set of $M$-regular elements is a multiplicative subset of $A$.

Let $S_{0}$ be the set of $A$-regular elements (= the set of nonzero
divisors), then $S_{0}^{-1}A$ is called the \define{Total Ring of Fractions} of $A$,
which Matsumara denoted by $\Phi(A)$.
\end{definition}

\begin{definition}[1.P]
Let $A$ be a ring, let $\alpha\colon\ZZ\to A$ be the canonical ring morphism
sending $1\mapsto 1$. Then $\ker(\alpha)=n\ZZ$ for some $n\geq0$.
We call this $n\geq0$ the \define{Characteristic} of $A$ and denote it
by $\Char(A)$.

Observe: if $A$ is local, then $\Char(A)$ is either 0 or a power of a
prime number.
\end{definition}

\subsection{Noetherian and Artinian Rings}

\begin{definition}
We call a ring $A$ \define{Noetherian} (resp., \define{Artinian}) if
the ascending chain condition (resp., descending chain condition)
holds in it:
if we have a chain of ideals of $A$, $I_{1}\subset I_{2}\subset\cdots$,
then there exists an $n\in\NN$ such that $I_{n}=I_{n+k}$ for all $k\in\NN$.

(The descending chain condition is the same, but with $I_{1}\supset I_{2}\supset\cdots$.)
\end{definition}

\begin{remark}
This is a finiteness condition on rings. The Noetherian property turns
out to be ``the right one''.
\end{remark}

\begin{example}
Is $\ZZ$ Noetherian? Take $I_{n}=(k_{n})$ for $k_{n}>0$ and
$k_{n+1}\leq k_{n}$. It is Noetherian. But it is not Artinian.
\end{example}