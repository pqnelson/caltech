%%
%% fall-lecture07.tex
%% 
%% Made by Alex Nelson <pqnelson@gmail.com>
%% Login   <alex@lisp>
%% 
%% Started on  2025-10-14T11:53:54-0700
%% Last update 2025-10-14T11:53:54-0700
%% 

\lecture{}

\begin{theorem}[Lasker--Noether]
Let $R$ be a Noetherian ring. Let $I\ideal R$ be an ideal.
Then there exists a primary decomposition of $I=Q_{1}\cap\cdots\cap Q_{n}$
where $Q_{i}\ideal R$ are primary ideals for all $i=1,\dots,n$.
\end{theorem}

\begin{example}
We see $(x^{2},xy)=(x)\cap(x^{2},xy,y^{n})$ for any $n\in\NN$ nonzero.
It is useful to think of ideals as systems of equations.
For us, the system of equations are $x^{2}=0$ and $xy=0$.
This has the family of solutions $\{(0,y)\in\CC^{2}\}$. We can look at
$\CC[x,y]/(x^{2},xy)$ as a $\CC[x,y]$-module, then look at
annihilators of elements of the module. Looking at $\bar{x}$, what
$q\in\CC[x,y]$ satisfies $q\bar{x}=0$? Well, $q=x$ and $q=y$.
So $(x,y)\bar{x}=0$ and similarly we find $(x)\bar{y}=0$.

We have 2 prime ideals which arise as annihilators of some elements in
$M=\CC[x,y]/(x^{2},xy)$. These are called \define{Associated Primes}
of $M$ and by abuse of language we call them \define{Associated Primes}
of the ideal $(x^{2},xy)$.

How are these primes associated to this decomposition? We see $(x,y)$
is the radical of $(x^{2},xy,y^{n})$. We call $(x,y)$ an
\define{Embedded Ring}.
\end{example}

\begin{remark}
The previous example shows us, and this cannot be stressed enough,
\emph{The primary decomposition is highly non-unique}.
\end{remark}

\begin{example}
Looking at $\CC[x,y]/(x^{2},xy,y^{n})$ which is a ring, but it's also
a vector space over $\CC$. It has a basis spanned by the vectors $1$,
$x$, $y$, \dots, $y^{n-1}$, so $\dim(M)=n+1$. We want to be more
general than linear algebra, because we are doing \emph{commutative algebra}.
This motivates us to introduce the next definition.
\end{example}

\begin{definition}
We define the \define{Length} of a module $M$ over ring $A$ to be the
maximal $n$ among the chains of proper submodules such that
\begin{equation}
0=M_{0}\propersubset M_{1}\propersubset\cdots\propersubset M_{n}=M.
\end{equation}
If all such chains have their length be bounded by an $n\in\NN$, then
the least such $n$ is the length of $M$. Otherwise, if there is no
such bound, we say the module is of \define{Infinite Length}.
\end{definition}

\begin{example}
Returning to the example with $M=\CC[x,y]/(x^{2},xy,y^{n})$, we see
submodules are ideals containing $(x^{2},xy,y^{n})$. So we have the
endpoints of the chain be $(1)\propersupset\cdots\propersupset(x^{2},xy,y^{n})$.
Then we fill in the ``$\cdots$'' bit,
\begin{equation}
(x^{2},xy,y^{n}\propersubset(x^{2},xy,y^{n-1})\propersubset\cdots\propersubset(x^{2},xy,y)=(x^{2},y)\propersubset(1).
\end{equation}
This is of length $n+1$, so it coincides with dimension of $M$ as a vector
space over $\CC$, cool beans.
\end{example}

\begin{example}
We see $\RR[x]/(x^{2}+1)\iso\CC$. Observe that $\RR[x]/(x^{2}+1)$ has
length 1 over itself, but it has dimension 2 over $\RR$, so be careful
when thinking about \emph{length} and \emph{dimension}.
\end{example}

\begin{proposition}[2.C]
A ring $A$ is Artinian if and only if the length of $A$ as an
$A$-module is finite: $\length_{A}(A)<\infty$.
\end{proposition}

\begin{proof}
$(\Longleftarrow)$ We see $\length_{A}(A)<\infty$ implies $A$ is
  Noetherian and $A$ is Artinian.

$(\Longrightarrow)$ This is more involved. Assume $A$ is Artinian.
There are three central claims:

\textsc{Claim 1:} $A$ has finitely many maximal ideals.
If there were infinitely many prime and maximal ideals $(P_{1},P_{2},\dots)$,
then we could form the chain $P_{1}\supset P_{1}P_{2}\supset P_{1}P_{2}P_{3}\supset\dots$,
which contradicts $P_{i}$ being prime and maximal.

\medbreak
\textsc{Claim 2:} Let $P_{1}$, \dots, $P_{r}$ be maximal ideals from
claim 1. Then form their Jacobson radical $I=\JacobsonRadical(A)=P_{1}(\cdots)P_{r}$.
We claim there exists an $s\in\NN$ such that $I^{s}=0$.

We form a chain $I\supset I^{2}\supset I^{3}\supset\dots$. Eventually
this stabilizes so $I^{s}=I^{s+1}$. We write
\begin{equation}
(I:J)=\{x\in A\mid xJ\subset I\}
\end{equation}
(e.g., $((4):(2))=(2)$ but $((5):(2))=(5)$ since $2x=5n$ has solutions
$x\in(5)$).
Now consider $J:=((0):I^{s})=\{x\in A\mid xI^{s}=0\}$. We want to show
$J=A$, this is our goal.

Assume for contradiction $J\neq A$. Then take $J'$ the minimal member
of this set of ideals strictly containing $J$. (Can we do this? We
know there is some ideal containing $J$, since $A$ is Artinian there
is some $J_{1}\supset J$. If $J_{1}$ is not minimal, we can form a
chain $J_{1}\propersupset J_{2}\propersupset\dots\propersupset J$
which must be of finite length by $A$ being Artinian. Hence there is a
minimal ideal containing $J$.)

Let $J'=Ax+J$ for some $x\in J'\setminus J$. We want to say
$Ix+J\subset J'$. We use the Nakayama Lemma (\S\ref{lemma:NAK}): for any $N,N'\subset M$
if $M=N+IN'$, then $M=N$ where $I$ sits inside the radical and $N'$ is
finite.
Then $J=J'$ which contradicts $J'$ being a minimal proper ideal
containing $J\neq J'$. Then $Ix+J=J$ which implies $Jx\subset J$ which
implies $x\in(J:I)$. We see
\begin{subequations}
\begin{align}
  (J:I) &= \bigl(((0) : I^{s}) : I\bigr) = ((0) : I^{s+1})\\
  &= ((0) : I^{s})\\
  &= J.
\end{align}
\end{subequations}
Hence $x\in J$.

\medbreak
\textsc{Claim 3:} Using $I^{s}=0$, we form the longest possible chain
of subideals involving the $P_{i}$, and we see that this is an upper
bound on $\length_{A}(A)$. But it's finite, and that concludes the proof.

Take the following chain:
\begin{equation}
A\supset P_{1}\supset P_{1}P_{2}\supset\dots\supset
P_{1}P_{2}(\cdots)P_{r-1}\supset I\supset IP_{1}\supset\cdots\supset
I^{2}\supset\cdots\supset I^{s}=0.
\end{equation}
We claim there is no possible longer chain. We could look at the
quotients (e.g., $IP_{1}P_{2}/IP_{1}$) which are vector spaces over
$A/P_{i}$ (which is a field since $P_{i}$ is a maximal ideal in $A$)
since $M/PM$ is an $A/PA$-vector space, and $IP_{1}P_{2}/IP_{1}$ is a
finite-dimensional vector space, which means there are only
finitely-many ideals between $IP_{1}P_{2}$ and $IP_{1}$. The length of
this chain $IP_{1}\supset\cdots\supset IP_{1}P_{2}$ is at most
$\dim_{A/P_{i}}(IP_{1}P_{2}/IP_{1})$. Then we have
\begin{equation}
\length_{A}(A)=\sum\mbox{length of factor modules},
\end{equation}
we know each summand is a finite number, and there are only finitely
many summands. Hence $\length_{A}(A)$ is finite.
\end{proof}