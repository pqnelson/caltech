%%
%% fall-lecture02.tex
%% 
%% Made by Alex Nelson <pqnelson@gmail.com>
%% Login   <alex@lisp>
%% 
%% Started on  2025-10-02T09:50:51-0700
%% Last update 2025-10-02T09:50:51-0700
%% 

\lecture{}

\begin{lemma}[Prime avoidance]\label{math120a:prime-avoidance}
Let $A$ be a ring, let $I$, $p_{1}$, \dots, $p_{r}$ be ideals of
$A$. Suppose at most 2 of the ideals $p_{1}$, \dots, $p_{r}$ are not prime.
Then if $I\nsubset p_{i}$ for all $i$, then $I\nsubset\bigcup p_{i}$.
\end{lemma}

\begin{proof}
Assume $p_{i}\nsubset p_{j}$ for all $i\neq j$. Now we perform a
proof by induction on $r$.

\textsc{Base case} ($r=2$): Suppose for contradiction $I\subset p_{1}\cup p_{2}$.
Take $x\in I\setminus p_{2}$ and $s\in I\setminus p_{1}$.
Then take $x+s$. Either $x+s\in p_{1}$ or $x+s\in p_{2}$.
If $x+s\in p_{1}$, then $x+s-x=s\in p_{1}$, which is a contradiction
(hence the result).
If $x+s\in p_{2}$, then $x+s-s=x\in p_{2}$, which is a contradiction
(hence the result). This establishes the $r=2$ case.

\textsc{Case $r>2$:} Assume $p_{r}$ is prime.
Then $Ip_{1}\cdots p_{r}\nsubset p_{r}$ (otherwise one of $I$,
$p_{1}$, \dots, $p_{r-1}$ would be contained in $p_{r}$). Set
$S=I\setminus(p_{1}\cup\cdots\cup p_{r-1})$.
By the inductive hypothesis, $S\neq\emptyset$.

Now, suppose for contradiction that $I\subset p_{1}\cup\cdots\cup p_{r}$.
Then $S\subset p_{r}$. If $s\in S$ and $x\in Ip_{1}\cdots p_{r-1}$,
then $x+s\in S$. Then $s\in p_{r}$ and $x+s\in p_{r}$ (trick to see
this is that $x+s\in I$), hence $x\in p_{r}$. This contradicts our
assumption, hence the result.
\end{proof}

\begin{remark}
If $A$ contains an infinite field $\kk$, then we do not need $p_{3}$,
\dots, $p_{r}$ be prime ideals. We view $A$ as a vector space over
$\kk$. Looking at $I\subset(p_{1}\cup\cdots\cup p_{r})$, observe
\begin{align*}
(I\cap p_{1})\cup\cdots\cup(I\cap p_{r}) &= I\cap(p_{1}\cup\cdots\cup p_{r})\\
&= I.
\end{align*}
This critically depends on $\kk$ being infinite.
\end{remark}

\begin{lemma}
Let $A$ be a ring, let $I_{1}$, \dots, $I_{r}$ be ideals of $A$ such
that $I_{i}+I_{j}=A$ for $i\neq j$. Then $I_{1}\cap\cdots\cap I_{r}=I_{1}(\cdots)I_{r}$
(their intersection is their product). In particular,
\begin{equation}
A/\left(\bigcap_{i}I_{i}\right)\iso (A/I_{1})\times\cdots\times(A/I_{r}).
\end{equation}
\end{lemma}

\begin{example}
Consider $(2)\ideal\ZZ$ and $(4)\ideal\ZZ$. Then $(2)(4)=(8)$ and
$(2)+(4)=(2)$. This is because the hypothesis does not hold.
But for $(3)\ideal\ZZ$ the hypothesis \emph{does} hold, and we see
$(2)+(3)=(6)$ and $(2)(3)=(6)$.
\end{example}

\begin{proposition}
Any ring $A\neq0$ has at least one maximal ideal.
\end{proposition}

\begin{proof}
Let $\mathcal{M}=\{I\ideal A\mid I\neq A\}$. We see it is a nonempty
set since $(0)\in\mathcal{M}$. Therefore by Zorn's lemma, any chain
has a maximal element, which is a maximal ideal.
\end{proof}

\begin{proposition}
Let $A\neq0$ be a ring. Then $\Spec(A)$ has a maximal element. For any
prime $p$ there is a minimal prime contained in $p$.
\end{proposition}

\begin{proof}[Proof sketch]
By Zorn's lemma, using reverse inclusion.
\end{proof}

\begin{remark}
Recall geometrically, maximal ideals correspond to points. We have
$(x,y)\in\CC[x,y]$ or $(x-x_{0},y-y_{0})$ corresponds to the point
$(x_{0},y_{0})\in\CC$. The minimal ideal of $\CC[x,y]$ corresponds to
``chunks'' of $\CC$ (in this case, the entire plane).

Then $\CC[x,y]/(xy)$ corresponds to the union of axes. The minimal
prime ideals in this quotient ring would be $(x)$ and $(y)$, but not $(0)$.

The minimal prime ideals correspond to ``components'' of varieties.

Let $J\ideal A$ be a proper ideal of $A\neq J$. Then there exists a
map $\Spec(A/J)\to\Spec(A)$. (Any ring morphism $\varphi\colon A\to B$
such that any prime ideal $Q\ideal B$ has its preimage
$\varphi^{-1}(Q)\ideal A$ be a prime ideal of $A$. This induces a
contravariant functor on categories.) In particular, this induces
$\preimage\varphi\colon\Spec(B)\to\Spec(A)$ where $\varphi\colon A\to B$ is
a ring morphism.
\end{remark}

\begin{example}
Consider $\varphi\colon\ZZ\to\ZZ[\sqrt{2}]$ given by the usual
inclusion, then we have $\preimage\varphi\colon\Spec(\ZZ[\sqrt{2}])\to\Spec(\ZZ)$.
If we want to find $(p)$ such that $\preimage\varphi(a+b\sqrt{2})=(p)$,
then we need to solve
\begin{equation}\label{eqn:fall-lecture02:pell}
p = (a+b\sqrt{2})(a-b\sqrt{2})=a^{2}-2b^{2}.
\end{equation}
If there is a solution, then $\preimage\varphi(a+b\sqrt{2})=(p)$ and 
$\preimage\varphi(a-b\sqrt{2})=(p)$ are the only possibilities.
If there is no solution, then $\preimage\varphi(p)=(p)$ is the only possibility.
Since there is no $a$, $b$ such that $3=a^{2}-2b^{2}$, this means only
$\preimage\varphi(3)=(3)$.

(We also see that $\preimage\varphi(\sqrt{2})=2$, since this
corresponds to the $a=0$ and $b=1$ solution to Equation~\eqref{eqn:fall-lecture02:pell}.)

Equation~\eqref{eqn:fall-lecture02:pell} is also known as the
\emph{Generalized Pell's Equation}. A quick way to check if there is
no solution is to rewrite it as
\begin{equation}
p + 2b^{2} = a^{2}.
\end{equation}
If $p+2b^{2}\equiv0\bmod{4}$ or $p+2b^{2}\equiv1\bmod{8}$, then there
is possibly a solution. Otherwise, it is impossible. This is because a
perfect square of an even number is $(2n)^{2}=4n^{2}\equiv0\bmod{4}$,
and the perfect square of an odd number is $(2n+1)^{2}=4n(n+1)+1$ and
$n(n+1)$ is always an even integer, so $(2n+1)^{2}\equiv1\bmod{8}$. We
also see that if $a$ is even, then $a^{2}-2b^{2}$ is even (which can
only possibly describe the $p=2$ case). Therefore we need $a$ to be odd.

This means there are at most 1 solution to the generalized Pell's equation,
and possibly zero solutions. When there is a solution, then we have
two prime ideals of $\ZZ[\sqrt{2}]$ which map to $(p)$. When there is
no solution, then $(p)$ is a prime ideal of $\ZZ[\sqrt{2}]$ and it's
the only one which maps to its doppleganger in $\ZZ$.

There was some mention that this has some connection to Galois theory.
\end{example}