%%
%% fall-lecture27.tex
%% 
%% Made by Alex Nelson <pqnelson@gmail.com>
%% Login   <alex@lisp>
%% 
%% Started on  2025-12-02T08:31:02-0800
%% Last update 2025-12-02T08:31:02-0800
%% 

\lecture

\begin{node}
We studied Noetherian rings and found to do anything useful, we need
it. Then we studied Artinian rings, which turned out to be a special
case of Noetherian rings. Then we discussed zero-dimensional
rings. But we want to study all rings, not just Noetherian rings,
which is why we introduced dimension. We introduced graded rings,
which allowed us to prove things by induction. We saw Hilbert
polynomials which was a great tool for proving things about graded rings.
\end{node}

\subsection{Dimension}

(This is Chapter 5 of Matsumara)

\begin{recall}
For a prime ideal $P\in\Spec(A)$, we have
$\height(P)=\dim(A_{P})\mathrel{\mbox{``=''}}$ codimension of the point.
\end{recall}

\begin{definition}
Let $M\neq0$ be a module over $A$. We define the \define{Dimension} of
$M$ is $\dim(M):=\dim(A/\Annihilator(M))$.
\end{definition}

\begin{proposition}[Matsumara 12B]
Let $A$ be a Noetherian ring, let $M\neq0$ be a finitely-generated
module over $A$. The following are equivalent:
\begin{enumerate}
\item $M$ is a module of finite length $\length(M)<\infty$;
\item $A/\Annihilator(M)$ is Artinian;
\item $\dim(M)=0$.
\end{enumerate}
\end{proposition}

\begin{proof}
We have proved $(2)\iff(3)$ before.

$(2)\implies(1)$ Matsumara appears to argue that $M$ is also a module
over $A/\Annihilator(M)$, then using
Proposition~\ref{prop:math120a:fall2025:lec7:matsumara-7a} gives the
results.

But this was unsatisfying to the audience. Assume $A/\Annihilator(M)$
is Artinian. If there were an infinite chain of submodules, their
annhiliators give us a chain of ascending ideals which must terminate,
and this is the same as $\length(M)<\infty$. But if
$N_{1}\propersubset N_{2}$, is it true that
$\Annihilator(N_{2})\propersubset\Annihilator(N_{1})$? We know
$\Annihilator(N_{2})\subset\Annihilator(N_{1})$ and
$\Annihilator(N_{1})\subset\Annihilator(M)$. We want to show
$\Annihilator(N_{2})\propersubset\Annihilator(N_{1})$. Suffices to
show $N_{2}/N_{1}$ has a nontrivial annihilator as a module over $A/\Annihilator(N_{1})$.
We know $A/\Annihilator(N_{1})$ is Artinian, so it has only finitely
many prime ideals
(\S\ref{artinian-rings-have-finitely-many-maximal-and-prime-ideals}). Their
product must annihilate $N_{2}/N_{1}$.

$(1)\implies(3)$ Assume $\length(M)<\infty$. Replace $A$ by
$A/\Annihilator(M)$. Then we can assume $\Annihilator(M)=0$ as a
module over $A/\Annihilator(M)$. If $\dim(A)>0$, then we can take
$P\in\Spec(A)$ minimal which is not maximal. Then $M$ is finite, so
$\Annihilator(M)=0$. Then $M_{P}\neq0$. Then $P\in\Assassinator(M)$ is
an associated prime. But this means $A/P$ is a submodule of $M$. We
see $\dim(A/P)>0$, which implies $\length(A/P)=\infty$, which implies
$\length(M)=\infty$ which is a contradiction. Hence the result.
\end{proof}

\begin{definition}[Matsumara 12C]
Let $A$ be a Noetherian semilocal ring (i.e., $\MSpec(A)$ is finite) and
$\mathfrak{m}=\JacobsonRadical(A)$. An ideal $I\ideal A$ is called an
\define{Ideal of Definition} if there is some $\nu\in\NN$ such that $\mathfrak{m}^{\nu}\subset I\subset\mathfrak{m}$.
This is equivalent to $I\subset\mathfrak{m}$ and $A/I$ is Artinian.
\end{definition}

\begin{remark}
This defines what Mizar would call a ``soft type''. We need to prove
the existence of an ideal of definition for a Noetherian semilocal
ring $A$. The Stacks Project \href{https://stacks.math.columbia.edu/tag/00KQ}{\texttt{[00KQ]}}
establishes that there exists an ideal of definition generated by
$\dim(A)$ number of elements.
\end{remark}

\begin{example}
We see $(x^{2},y^{3})\ideal\CC[x,y]$ is an ideal of definition since
$\mathfrak{m}=(x,y)$ gives us $(x,y)^{4}\subset(x^{2},y^{3})\subset(x,y)$.
\end{example}

\begin{remark}
We see $\JacobsonRadical(A/I)$ is nilpotent when $A/I$ is Artinian, so $\mathfrak{m}^{\nu}=0$.
\end{remark}

\begin{node}[Hilbert polynomial of module with respect to ideal of definition]
We take 
\begin{equation}
A^{*}=\gr^{I}A=\bigoplus_{n}I^{n}/I^{n+1},
\end{equation}
which is a graded ring, and when $M$ is a finite module over $A$,
\begin{equation}
M^{*}=\gr^{I}M=\bigoplus_{n}I^{n}M/I^{n+1}M,
\end{equation}
which is a finitely-generated graded module. When
\begin{equation}
I=Ax_{1}+\dots+Ax_{r},
\end{equation}
then $A$ is a quotient
\begin{equation}
B=(A/I)[x_{1},\dots,x_{r}],
\end{equation}
and $M^{*}$ is a finitely-generated, graded module over $A^{*}$. So
we're in a situation where we can consider Hilbert polynomials. We
take
\begin{equation}
F_{M^{*}}(n)=\length(I^{n}M/I^{n+1}M)<\infty
\end{equation}
is a polynomial in $n$ of degree $\deg(F_{M^{*}})\leq r-1$ for $n\gg0$.
We define the functions
\begin{equation}
\chi(M,I;n):=\length(M/I^{n}M)=\sum^{n-1}_{j=0}F_{M^{*}}(j),
\end{equation}
which is also a polynomial in $n$, and $\deg(\chi(M,I;-))\leq r$ for $n\gg0$.
The polynomial which represents $\chi(M,I;n)$ for $n\gg0$ is called
the \define{Hilbert polynomial} of $M$ with respect to $I$.

If $J$ is another ideal of definition of $A$, then $J^{s}\subset I$
for some $s>0$. Then we have $\chi(M,I;n)\leq\chi(M,J;sn)$. If
\begin{equation}
\chi(M,I;n)\sim a_{d}n^{d}+(\mbox{lower order terms})
\end{equation}
and
\begin{equation}
\chi(M,J;sn)\sim b_{d'}n^{d'}+(\mbox{lower order terms}),
\end{equation}
then $d\leq d'$. This implies $d$ is independent of $I$, so we may
simply write $d$ as a function of the module $d:=d(M)$.
\end{node}

\begin{remark}
We want to understand dimension, but it's rather cumbersome. We want
to work with $d(M)$ obtainable from the Hilbert polynomial.
\end{remark}

\begin{proposition}[Matsumara 12D]
Let $A$ be a Noetherian semilocal ring, let $I$ be an ideal of definition of $A$.
If
\begin{equation}
0\to M'\to M\to M''\to 0
\end{equation}
is a short exact sequence of finite $A$-modules, then $d(M)\leq\max(d(M'),d(M''))$.
Moreover, $\chi(M,I;n)-\chi(M',I,n)-\chi(M'',I;n)$ is a polynomial of
degree strictly smaller than $d(M')$ for $n\gg0$.
\end{proposition}

\begin{proof}
We see
\begin{equation}
\length(M''/I^{n}M'')=\length(M/(M'+I^{n}M))\leq\length(M/I^{n}M),
\end{equation}
which implies $d(M'')\leq d(M)$. We take
\begin{subequations}
  \begin{align}
\chi(M,I;n)-\chi(M'',I;n) &= \length(M/I^{n}M)-\length(M/(M'+I^{n}M))\\
&= \length((M'+I^{n}M)/I^{n}M)\\
&= \length(M'/(M'\cap I^{n}M)).
  \end{align}
\end{subequations}
There exists some $r>0$ such that $M'\cap I^{n}M\subset I^{n-r}M'$ by
Artin--Rees Theorem~\ref{math120a:artin--rees}, which gives us a bound
\begin{equation}
\length(M'/I^{n}M')\geq\length(M'/(M'\cap I^{n}M))\geq\length(M'/I^{n-r}M'),
\end{equation}
which means they have same degree and same leading coefficients. (This
is a generic fact about polynomials $f(n)\geq g(n)\geq f(n-r)$ implies
$f$ and $g$ have the same degree and same leading coefficients.)
\end{proof}

\begin{lemma}[Matsumara 12E]
Let $A$ be a Noetherian semilocal ring. Then $d(A)\geq\dim(A)$.
\end{lemma}

\begin{proof}
By induction on $d(A)$.

Base case: $d(A)=0$. Then $I=M$ is a maximal ideal, which implies
$M^{\nu}=M^{\nu-1}=\dots$, which implies
\begin{equation}
\bigcap^{\infty}_{k=0}M^{\nu+k}=M^{\nu}.
\end{equation}
But $M\subset\JacobsonRadical(A)$, so by the Krull Intersection
Theorem~\ref{thm:math-120a:krull-intersection} we find $M^{\nu}=0$,
which implies $\length(A)$ is finite. Hence $A$ is Artinian, and this
means $\dim(A)=0$.

Inductive case: skipped for time, but it uses Matsumara's Proposition 12D.
\end{proof}

\begin{lemma}[Matsumara 12G]
Let $A$ be a Noetherian semilocal ring, let $M\neq0$ be a
finitely-generated module over $A$, and let $r=\dim(M)$.
Then there exists $r$ elements $x_{1}$, \dots, $x_{r}\in\JacobsonRadical(A)$
such that
\begin{equation}
\length(M/(x_{1}M+\dots+x_{r}M))<\infty.
\end{equation}
\end{lemma}

The proof uses prime avoidance (\S\ref{math120a:prime-avoidance}).

\begin{theorem}[Matsumara 12H]
Let $A$ be a Noetherian semilocal ring, let
$\mathfrak{m}=\JacobsonRadical(A)$, and let $M\neq0$ a
finitely-generated $A$-module.
Then $d(M)=\dim(M)$ which is the smallest $r>0$ such that there exists
$x_{1}$, \dots, $x_{r}\in\mathfrak{m}$ such that $\length(\mathfrak{m}/(x_{1}\mathfrak{m}+\dots+x_{r}\mathfrak{m}))<\infty$.
\end{theorem}