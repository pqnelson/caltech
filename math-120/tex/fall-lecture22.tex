%%
%% fall-lecture22.tex
%% 
%% Made by Alex Nelson <pqnelson@gmail.com>
%% Login   <alex@lisp>
%% 
%% Started on  2025-11-18T09:55:31-0800
%% Last update 2025-11-18T09:55:31-0800
%% 

\lecture[Primary decomposition of modules]{}

\begin{note}
We will reserve $A$ to be a Noetherian ring, and $M$ for a module over $A$.
\end{note}

\begin{definition}[Matsumara 8A]
Let $A$ be a Noetherian ring. We call an $A$-module
\define{Co-primary} if it has only one associated prime.
\end{definition}

\begin{definition}
Let $A$ be a Noetherian ring, $M$ a module over $A$, and $N\subset M$
a submodule. We say $N$ is \define{Primary} (``a primary submodule of $M$'')
if the quotient $M/N$ is co-primary.

If $\Assassinator(M/N)=\{P\}$, then we say $N$ is \define{$P$-Primary}
(or ``$N$ belongs to $P$'').
\end{definition}

\begin{definition}
Let $A$ be a ring, let $M\neq0$ be a module over $A$.
Let $a\in A$. We say $a$ is \define{Locally Niplotent} on $M$ if for
each $m\in M$ there is a corresponding $n_{m}\in\NN$ such that $a^{n_{m}}m=0$.

(I don't know how standard this terminology is, Matsumara uses it for
Proposition~8.1 only. It's not even carved out as a definition.)
\end{definition}

\begin{proposition}[Matsumara 8B]
The following are equivalent:
\begin{enumerate}
\item The module $M$ is coprimary
\item $M\neq0$, if $a\in A$ is a zero-divisor for $M$, then $a$ is
  locally nilpotent on $M$.
\end{enumerate}
\end{proposition}

\begin{proof}
$(1)\implies(2)$ Assume $M$ is co-primary, i.e., that $\Assassinator(M)=\{P\}$.
Then let $x\in M$ be a nonzero element $x\neq0$, and we can consider
$\Assassinator(Ax)=\{P\}$. Then by a homework problem, we know
\begin{equation}
\Support(Ax)=V(\Annihilator(x)).
\end{equation}
Then $P$ is the unique maximal prime in $V(\Annihilator(x))$. Then we
can infer $P=\Radical{\Annihilator(x)}$, so if $a\in P$, then there is
an $n\in\NN$ such that $a^{n}\in\Annihilator(x)$.

$(2)\implies(1)$ Write
\begin{equation}
P=\{a\in A\mid a\mbox{ is locally nilpotent on } M\}.
\end{equation}
We assert that $P$ is an ideal (which can easily be verified). Let $Q$
be an associated prime $Q\in\Assassinator(M)$. Then there exists some
$x\in M$ such that $Q=\Annihilator(x)$. Then $P\subset Q$ by
definition of $P$, so if $a\in P$ then $a^{n}\in Q$. Buut $Q$ is
prime, so this means $a\in Q$. We can write
\begin{equation}
  P=\bigcup_{r\in\Assassinator(M)}r,
\end{equation}
which implies $Q\subset P$. Then $P=Q$, hence $\Assassinator(M)=\{P\}$.
\end{proof}

\begin{remark}
\begin{enumerate}
\item When $M=A/Q$, then we have
\begin{align*}
(2) &\iff\mbox{all zero-divisors of $M$ are nilpotent}\\
&\iff Q\mbox{ is primary},
\end{align*}
so $A/Q$ is coprimary $\iff$ $Q$ is primary.
\item Atiyah and Macdonald has a slicker proof of the primary
  decomposition theorem (c.f., Propositions 4.3--4.5 and exercises 21,
  22 in their \textit{Introduction to Commutative Algebra}).
\end{enumerate}
\end{remark}

\begin{lemma}[Matsumara 8C]
Let $P\in\Spec(A)$. Let $Q_{1}$, $Q_{2}$ be $P$-primary submodules of $M$.
Then $Q_{1}\cap Q_{2}$ is $P$-primary.
\end{lemma}

\begin{proof}
Look at the quotient
\begin{equation}
M/(Q_{1}\cap Q_{2})\into(M/Q_{1})\oplus(M//Q_{2}),
\end{equation}
then since $M\neq0$ we have
\begin{equation}
\emptyset\neq\Assassinator(M/(Q_{1}\cap Q_{2}))\subset\Assassinator(M/Q_{1})\cup[\Assassinator(M//Q_{2})=\{P\}.
\end{equation}
Hence the result.
\end{proof}

\begin{definition}[Matsumara 8D]
Let $N$ be a submodule of $M$.
\begin{enumerate}
\item A \define{Primary Decomposition} of $N$
is an equation of the form
\begin{equation}
N = Q_{1}\cap\cdots\cap Q_{r},
\end{equation}
where each of the $Q_{i}$ are primary in $M$.
\item We say this primary decomposition is \define{Nonredunant} if (i)
  no $Q_{i}$ may be omitted, and (ii) $\Assassinator(M/Q_{i})$ are all distinct.
\end{enumerate}
\end{definition}

\begin{xca}
Find an example of a primary decomposition where
$\Assassinator(M/Q_{i})$ are all distinct, but is redundant.
\end{xca}

\begin{lemma}[Matsumara 8E]
If $N=Q_{1}\cap\cdots\cap Q_{r}$ is an irredundant primary
decomposition and if $Q_{i}$ belongs to $P_{i}$, then
$\Assassinator(M/N)=\{P_{1},\dots,P_{r}\}$. 
\end{lemma}

\begin{proof}
Look at the natural map $M/N\to M/Q_{1}\oplus\dots\oplus M/Q_{r}$,
then
\begin{equation}
\Assassinator(M/N)\subset\Assassinator(M/Q_{1}\oplus\dots\oplus M/Q_{r})=\{P_{1},\dots,P_{r}\}.
\end{equation}
But $(Q_{2}\cap\dots\cap Q_{r})/N\subset M//Q_{1}$, so
$\Assassinator((Q_{2}\cap\dots\cap Q_{r})/N)\subset\{P_{1}\}$, and
then
$Q_{2}\cap\cdots\cap Q_{r}/N\subset M/N$, which implies
$P_{1}\in\Assassinator(M/N)$.
Hence the result.
\end{proof}

\begin{theorem}[Matsumara 8G]
Let $A$ be a Noetherian ring, let $M$ be a module over $A$.
Then there exists $Q(P)$ a $P$-primary submodule of $M$ for each
$P\in\Assassinator(M)$ which we can pick in such a way that
\begin{equation}\label{eq:fall-lec22:p-primary-decomposition-8g}
(0)=\bigcap_{P\in\Assassinator(M)}Q(P).
\end{equation}
\end{theorem}

\begin{proof}
  (We will use Zorn's lemma.)
Let $P$ be an associated prime of $M$. Write 
\begin{equation}
\mathcal{N}=\{N\subset M\mid P\notin\Assassinator(N)\}.
\end{equation}
We see $(0)\in\mathcal{N}$, so it's nonempty. If
$\mathcal{N}'=\{N_{\lambda}\}_{\lambda}$ is a linearly ordered subset
of $\mathcal{N}$, then $\bigcup_{\lambda}N_{\lambda}$ is an element of
$\mathcal{N}$ since $\Assassinator(\bigcup_{\lambda}N_{\lambda})=\bigcup_{\lambda}\Assassinator(N_{\lambda})$.
Then by Zorn's lemma, $\mathcal{N}$ has a maximal element which we
call $Q=Q(P)$. (We want to show each $Q(P)$ is $P$-primary and satisfies Equation~\eqref{eq:fall-lec22:p-primary-decomposition-8g}.)

We see $P\in\Assassinator(M)$ and $P\notin\Assassinator(Q)$, which
implies $M\neq Q$.

If $M/Q$ had a different associated prime $P'\neq P$, then there is a
$Q'$ such that $Q'/Q\iso A/P'$, then $Q'/Q\iso M/Q$. This would mean
$Q'\in\mathcal{N}$, and we would contradict $Q$ being maximal. Hence
$Q=Q(P)$ is a $P$-primary submodule of $M$.

From
\begin{equation}
\Assassinator(\bigcap_{P\in\Spec(A)}Q(P))=\bigcap_{P}\Assassinator(Q(P))=\emptyset,
\end{equation}
we have $\bigcap Q(P)=(0)$ as desired.
\end{proof}