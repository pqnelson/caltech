%%
%% fall-lecture20.tex
%% 
%% Made by Alex Nelson <pqnelson@gmail.com>
%% Login   <alex@lisp>
%% 
%% Started on  2025-11-13T08:00:24-0800
%% Last update 2025-11-13T08:00:24-0800
%% 

\lecture{}

\begin{node}
Returning to associated primes of a module, recall:
\begin{subequations}
\begin{align}
\Assassinator(M) &= \{P\in\Spec(A)\mid P=\Annihilator(x)\mbox{ for some }x\in M\}\\
&= \{P\in\Spec(A)\mid A/P\into M\}.
\end{align}
\end{subequations}
\end{node}

\begin{proposition}[Matsumara 7B]
If $P$ is maximal among $\Annihilator(x)$ for some nonzero $x\in M$,
then $P$ is prime and moreover $P\in\Assassinator(M)$.
\end{proposition}

\begin{proof}
Suffices to prove that $P$ is prime. Let $P=\Annihilator(x)$.
Suppose $ab\in P$ and $b\notin P$. Then $bx\neq0$ and $abx=0$. Since
$\Annihilator(bx)\supset\Annihilator(x)=P$, we must have
$\Annihilator(bx)=P$ by maximality of $P$. Hence $a\in P$.
\end{proof}

\begin{corollary}
$\Assassinator(M)=\emptyset$ iff $M=0$.
\end{corollary}

\begin{corollary}
The set of all zero divisors of $M$ is the union of associated primes
of $M$ ($=\bigcup\Assassinator(M)$).
\end{corollary}

\begin{definition}[Matsumara, Exercise 1(ii)]
Let $A$ be a ring, let $M$ be an $A$-module.
We define the \define{Support} of $M$ to be
\begin{equation}
\Support(M):=\{P\in\Spec(A)\mid M_{P}\neq0\}.
\end{equation}
\end{definition}

\begin{remark}
\begin{enumerate}
  \item Since the notion of a ``prime ideal'' only really makes sense
    for commutative rings, we should note that the support of a module
    \emph{should} be considered when working over \emph{commutative}
    rings. (Observe: the noncommutative ring of $n\times n$ matrices over $\CC$ has
    no prime ideals.)
  \item Although this was not discussed in class, Eisenbud's \emph{Commutative Algebra}
points out that $P\in\Support(M)$ if and only if $P\supset\Annihilator(M)$
($P$ contains the annihilator of $M$).
This is Corollary~2.7 on page 67.
\end{enumerate}
\end{remark}

\begin{example}
If we take a free module $M=A^{n}$, then $\Support(M)=\Spec(A)$.
\end{example}

\begin{example}
Let $A$ be a ring, let $I\ideal A$ be an ideal.
Then $\Support(A/I)=\{P\in\Spec(A)\mid I\subset P\}$.
\end{example}

\begin{remark}
Geometrically, an $A$-module $M$ ``is'' a bundle over the ``space'' $\Spec(A)$.
Or more precisely, the ``space of sections of a vector bundle''
corresponds to a ``finitely-generated projective module'' --- this is
the Serre--Swan theorem.

The $M_{P}\neq0$ condition demands we have a nontrivial fibre over the
point $P\in\Spec(A)$.
\end{remark}

\begin{xca}[Matsumara, Chapter 1(ii)]
If $M$ is a finite module (i.e., a finitely-generated module), then 
\begin{subequations}
  \begin{align}
\Support(M) &= V(\Annihilator(M))\\
&= \{P\in\Spec(A)\mid\Annihilator(M)\subset P\}.
  \end{align}
\end{subequations}
\end{xca}

\begin{lemma}[Matsumara 7C]\label{lemma:fall-lec20:7c}
Let $S\subset A$ be a multiplicative set.
Let $A'=S^{-1}A$ and $M'=S^{-1}M$.
Then
\begin{equation}
\Assassinator_{A}(M')=f(\Assassinator_{A'}(M'))=\Assassinator_{A}(M)\cap\{P\in\Spec(A)\mid P\cap S=\emptyset\}
\end{equation}
where $f\colon\Spec(A')\to\Spec(A)$ is the natural map.
\end{lemma}

\begin{proof}
Homework. Hint: $S^{-1}P\subset S^{-1}A=A'$.
\end{proof}

\begin{theorem}[Matsumara 7D]
Let $A$ be a Noetherian ring, let $M$ be an $A$-module.
Then associated primes $\Assassinator(M)\subset\Support(M)$ and any
\emph{minimal} element of $\Support(M)$ is in $\Assassinator(M)$.
\end{theorem}

\begin{proof}
\textsc{Claim 1.}
If $P\in\Assassinator(M)$, then we can write an exact sequence
\begin{equation}
0\to A/P\to M.
\end{equation}
We know $A_{P}$ is flat over $A$. Then tensoring the exact sequence
with $A_{P}$ gives us the exact sequence
\begin{equation}
0\to(A/P)\otimes_{A} A_{P}\to M\otimes_{A}A_{P},
\end{equation}
or equivalently,
\begin{equation}
0\to A_{P}/P\to M_{P}.
\end{equation}
Then $A_{P}/PA_{P}\neq0$, so $M_{P}\neq0$. Hence $\Assassinator(M)\subset\Support(M)$.

\textsc{Claim 2.} Let $P$ be a minimal prime ideal in the support $P\in\Support(M)$.
Then by Lemma~\ref{lemma:fall-lec20:7c},
\begin{equation}
P\in\Assassinator(M)\iff PA_{P}\in\Assassinator_{A_{P}}(M_{P}),
\end{equation}
so we can replace $M$ by $M_{P}$, and $A$ by $A_{P}$. We may assume
$(A,P)$ is a local ring and $M\neq0$ and $M_{q}=0$ for every
$q\propersubset P$ since $P$ is a minimal prime in $\Support(M)$ so
any smaller prime sub-ideal would kill $M$.
Then $\Support(M)=\{P\}$.
Hence $P\in\Assassinator(M)$.
\end{proof}

\begin{corollary}[Matsumara 7.3]
If $I\ideal A$ is an ideal, then the minimal associated primes of the
module $A/I$ are precisely the minimal prime over-ideals of $I$.
\end{corollary}

\begin{definition}
We say associated primes which are not minimal in $\Support(M)$ are
called \define{Embedded}.
\end{definition}

\begin{fact}
\begin{enumerate}
\item (7E) Let $M$ is a finite module over $A$, let $A$ be Noetherian ring.
  Then there exists the chain of modules
  \begin{equation}
(0)=M_{0}\propersubset\cdots\propersubset M_{n-1}\propersubset M_{n}=M
  \end{equation}
  such that $M_{i}/M_{i-1}=A/P_{i}$ with $P_{i}\in\Spec(A)$ are also
  associated primes.
\item (7F) If $0\to M'\to M\to M''$ is an exact sequence, then
\begin{equation}
\Assassinator(M)\subset\Assassinator(M')\cup\Assassinator(M'')
\end{equation}
\item (7G) If $M$ is a finite module over $A$, and if $A$ is a
  Noetherian ring, then $\Assassinator(M)$ is finite. This follows by
  combining the previous two facts to see the chain of successive
  quotients of nested submodules have assassinators
  \begin{equation}
\Assassinator(M)\subset\Assassinator(M_{1})\cup\Assassinator(M_{2}/M_{1})\cup\cdots\cup\Assassinator(M_{n}/M_{n-1}),
  \end{equation}
  where $\Assassinator(M_{i}/M_{i-1})\iso\Assassinator(A/P_{i})=\{P_{i}\}$.
\end{enumerate}
\end{fact}