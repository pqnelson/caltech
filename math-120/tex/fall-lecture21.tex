%%
%% fall-lecture21.tex
%% 
%% Made by Alex Nelson <pqnelson@gmail.com>
%% Login   <alex@lisp>
%% 
%% Started on  2025-11-15T09:31:29-0800
%% Last update 2025-11-15T09:31:29-0800
%% 

\lecture[Graded Rings and Modules]

\begin{definition}[Graded ring]
A \define{Graded Ring} consists of a ring $R$ equipped with a
decomposition of its additive group $R=\bigoplus_{n\geq0}R_{n}$ such
that $R_{n}R_{m}\subset R_{n+m}$. 

Elements of $R_{n}$ are called \define{Homogeneous Elements of Degree $n$}.
\end{definition}

\begin{example}
The ring $R=\CC[x]$ is a graded ring with $R_{n}$ consisting of all
degree $n$ monomials.
\end{example}

\begin{definition}[Graded module]
Let $R$ be a graded ring. A \define{Graded Module} over $R$ consists
of a module $M$ over $R$ equipped with a decomposition
$M=\bigoplus_{n\in\ZZ}M_{n}$ such that $R_{n}M_{m}\subset M_{n+m}$.

Elements of $M_{n}$ are called \define{Homogeneous Elements of Degree $n$}.
\end{definition}

\begin{example}
The module $M=\CC[x,y]$ is a graded module over $\CC[x]$. 
\end{example}

\begin{node}[Morphisms of graded modules]
Let $R$ be a graded ring. Let $M$ and $N$ be a graded modules over $R$.
A morphism $\varphi\colon M\to N$ must preserve the grading in the
sense that $\varphi(M_{n})\subset\varphi(M)_{n}$ for all $n\in\ZZ$.
\end{node}

\begin{definition}[Graded submodule]
Let $M$ be a graded module over a graded ring $R$.
A submodule $N\subset M$ is called a \define{Graded Submodule} of $M$
if $N=\bigoplus_{n\in\ZZ}(M_{n}\cap N)$.
\end{definition}

\begin{proposition}
Let $M$ be a graded module over a graded ring $R$.
Let $N\subset M$ be a submodule. The following are equivalent:
\begin{enumerate}
\item $N$ is a graded submodule of $M$
\item $N$ is generated by homogeneous elements
\item for all $x\in N$: if $x=x_{1}+\dots+x_{\xi}$ where $x_{i}$ is
  homogeneous for all $i$, then $x_{i}\in N$.
\end{enumerate}
\end{proposition}

\begin{proposition}
If $M$ is a graded module and $N\subset M$ is a homogeneous graded
submodule, then $M/N$ is a graded module.
\end{proposition}

\begin{definition}[Graded ideal]
Let $R$ be a graded ring. Let $I\ideal R$ be an ideal. We say the
ideal $I$ is \define{Graded} if it is a graded $R$-module.

We call the ideal $I$ \define{Homogeneous} if it consists of
homogeneous elements (or, equivalently, if it is generated by
homogeneous elements).
\end{definition}

\begin{example}
The ideal $I_{1}=(x_{1},x_{2})\ideal\CC[x_{1},x_{2},x_{3}]$ is a
homogeneous ideal. But $I_{2}=(x_{1}^{2}+x_{2})$ is not a homogeneous ideal.
\end{example}

\begin{proposition}
If $I\ideal R$ is a homogeneous ideal of a graded ring, then $R/I$ is a
graded module over $R$.
\end{proposition}

\begin{node}[Rosetta stone]
We can also ask what is the Spec of a graded ring (and how does it
differ from the Spec of a ``vanilla'' ring)?
\end{node}

\begin{definition}
Let $A$ be a ring (possibly graded, possibly ungraded). We define a
\define{Filtration} of $A$ to be a descending sequence of ideals
\begin{equation}
A=J_{0}\supset J_{1}\supset J_{2}\supset\dots
\end{equation}
such that for all $n$ and $m$ we have $J_{n}J_{m}\subset J_{n+m}$.
\textsc{Caution:} some authors use the condition $J_{n}J_{m}\supset J_{n+m}$,
and still other authors do not demand any condition at all.
\end{definition}

\begin{example}
For the ring $\ZZ$, for any prime $p$ we have the filtration
given by $J_{n}=(p^{n})$ for $n\geq1$.
\end{example}

\begin{example}
For the ring $\CC[x]$, we have the filtration given by $I_{n}=(x^{n})$.
\end{example}

\begin{definition}
We call a ring $A$ \define{Filtered} if it is equipped with a
filtration. This is extra ``data'', not an extra ``property''.
\end{definition}

\begin{construction}
We can construct a graded ring $A'$ from a filtered ring $(A, J_{0}\supset\dots)$
by writing $A'=\bigoplus_{n\geq0}J_{n}/J_{n+1}$.

Let $\alpha\in J_{n}/J_{n+1}$ and $\beta\in J_{m}/J_{m+1}$. These are
really equivalence classes of the form $\alpha=x + J_{n+1}$ and
$\beta=y + J_{m+1}$ where $x\in J_{n}$ and $y\in J_{m}$. Then
\begin{equation}
\alpha\beta = xy + J_{n+m+1},
\end{equation}
where $xy\in J_{n+m}$ and $\alpha\beta\in J_{n+m}/J_{n+m+1}$.
\end{construction}

\begin{definition}
If $I\ideal A$ is a ``vanilla'' ideal in a ``vanilla'' ring, then we
can consider the filtration $A=I^{0}\supset I^{1}\supset I^{2}\supset\dots$
which is called an \define{$I$-Adic Filtration}.

Moreover, the \define{Associated Graded Ring} with respect to the
filtration $I$ is written
\begin{equation}
\gr^{I}(A) := \bigoplus_{n\geq0}I^{n}/I^{n+1}.
\end{equation}
Eisenbud~\cite[Ch.~5]{eisenbud1994commutative} discusses this in
detail---there is geometric meaning to it.
\end{definition}

\begin{proposition}[Matsumara 10.2]
If $A$ is a Noetherian ring and $I\ideal A$ is an ideal,
then $\gr^{I}(A)$ is Noetherian.
\end{proposition}

\begin{proof}
We see $\gr^{I}(A)=\bigoplus_{n\geq0}A'_{n}$ where $A'_{n}=I^{n}/I^{n+1}$.
Observe $A'_{0}=A/I$ is a subring of $\gr^{I}(A)$ (which is
generically true for a graded ring --- $R_{0}$ is a subring of the
graded ring $R$), but $A'_{0}$ is Noetherian. We can write
$I=(a_{1},\dots,a_{n})$ so then $\bar{a}_{1}$, \dots, $\bar{a}_{n}\in I/I^{2}$
generate $\gr^{I}(A)$ over $A'_{0}$. This is a finitely generrated
algebra over a Noetherian ring, hence Noetherian.
\end{proof}

\begin{node}[Matsumara 10E]
Let $A$ be an \emph{Artinian} ring. Consider $B=A[x_{1},\dots,x_{m}]$, and
$M$ a finitely-generated module over $B$. Then we define the
\define{Hilbert--Samuel Function}
$F_{M}(n):=\length(M_{n})$ for $n\geq0$. Matsumara asserts this
measures the ``largeness'' of $M$.

\textsc{Caution:} We can generalize this from $A$ being Artinian to
more general settings, see Eisenbud~\cite[p.274]{eisenbud1994commutative}.
\end{node}

\begin{notation}
In the preceding setup, we write $B(d)$ for shifting the grading of $B$,
but keeping the underlying module the same. So $B(d)_{n}=B_{n-d}$.
\end{notation}

\begin{proof}[Proof (well-definedness of Hilbert--Samuel function)]
Why is the length finite? Since $M$ is finitely-generated, we can
wwrite it as $M=(\xi_{1},\dots,\xi_{p})$ where
$\deg(\xi_{i})=d_{i}$. Then we can consider a map
\begin{equation}
\begin{split}
f\colon&\bigoplus B(d)\to M\\
&1_{d_{i}}\mapsto\xi_{i}
\end{split}
\end{equation}
The claim is that this is surjective. Then
\begin{equation}
\length(M_{n})\leq\sum\length(B_{n}(d_{i}))=\sum\length(B_{n-d_{i}})<\infty
\end{equation}
is finite. The number of monomials of degree $n$ in $x_{1}$, \dots, $x_{m}$
equals $\binom{n+m-1}{m-1}$. Then
\begin{equation}
F_{B}(n)=\length(B_{n})=\binom{n+m-1}{m-1},
\end{equation}
as desired.
\end{proof}

\begin{theorem}[Matsumara 10F]\marginpar{\footnotesize Most important theorem in this section of Matsumara}\ignorespaces%
The function $F_{M}(n)$ is eventually polynomial in $n$ with
$\QQ$-coefficients for large enough $n$.
\end{theorem}