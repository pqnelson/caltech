%%
%% fall-lecture24.tex
%% 
%% Made by Alex Nelson <pqnelson@gmail.com>
%% Login   <alex@lisp>
%% 
%% Started on  2025-11-22T10:36:08-0800
%% Last update 2025-11-22T10:36:08-0800
%% 

\lecture{}

\begin{remark}
  We can always write a polynomial of degree $d$ as
  \begin{equation*}
f(x) = c_{d}\binom{x+d}{d}+c_{d-1}\binom{x+d-1}{d-1}+\dots+c_{0}\binom{x}{0}.
  \end{equation*}
\end{remark}

\begin{node}
Let $f\in\QQ[x]$.
If there exists an $n_{0}\in\NN$ such that $f(n)\in\ZZ$ for all
integer $n>n_{0}$,
then $c_{i}\in\ZZ$ for all $i$. Moreover, this means for all $n$ we
have $f(n)\in\ZZ$ for all $n\in\ZZ$.
\end{node}

\begin{proof}
By induction, writing
\begin{equation}
f(x)-f(x-1)=c_{d}\binom{x+d-1}{d-1}+\dots+c_{1}\binom{x}{0},
\end{equation}
Then $c_{i}\in\ZZ$ for $i>0$. But $f(n)\in\ZZ$ requires $c_{0}\in\ZZ$.
Hence the result.
\end{proof}

\begin{node}
If $F(n)-F(n-1)=f(n)$ for all $n>n_{0}$,
then 
\begin{equation*}
F(n) = c_{d}\binom{n-d+1}{d+1}+\dots+c_{0}\binom{n+1}{1}+C
\end{equation*}
where $C$ is some constant. This may be viewed as a discrete form of integration.
\end{node}

\subsection{Artin--Rees Theorem}

\begin{remark}
We can form a topology for a filtered module. Specifically, the
filtration gives rise to a topology.
If we go to a submodule of a filtered module, do we get the induced
topology from the filtration restricted to the submodule?
\end{remark}

\begin{definition}[Matsumara 11A]
Let $A$ be a ring, $I\ideal A$ is an ideal of $A$, and $M$ is a module
over $A$.
\begin{enumerate}
\item A \define{Filtration} of $M$ is a descending chain of submodules
  $M=M_{0}\supset M_{1}\supset M_{2}\supset\dots$
\item An \define{$I$-Adic Filtration} of $M$ is a filtration such that $M_{n}=I^{n}M$
\item A filtration of $M$ is called \define{$I$-Admissible} if
  $IM_{j}\subset M_{j+1}$ for all $j$;
\item A filtration of $M$ is called \define{Essentially $I$-Adic} if
  (i) it is $I$-Admissible, and (ii) there exists an $i_{0}$ such that
  $IM_{i}=M_{i+1}$ for all $i>i_{0}$.
\end{enumerate}
\end{definition}

\begin{construction}[Matsumara 11A]
Let $A$ be a ring, $I\ideal A$ is an ideal of $A$, and $M$ is a module over $A$.
Suppose we have a filtration $M=M_{0}\supset M_{1}\supset M_{2}\supset\dots$.
Then we may form a topology on $M$ by taking for any $x\in M$ the collection $\mathcal{N}(x)=\{x+M_{n}\mid n=1,2,\dots\}$
as a ``fundamental system of neighborhoods'' for the topology. I think
this means that $\bigcup_{x\in M}\mathcal{N}(x)$ forms a basis for the topology.

This topology is separated iff $\bigcap^{\infty}_{n=1}M_{n}=(0)$.
\end{construction}

\begin{definition}[Matsumara 11A]
Let $A$ be a ring, $I\ideal A$ is an ideal of $A$, and $M$ is a module over $A$.
The topology formed by an $I$-adic filtration on $M$ is called the \define{$I$-Adic Topology} of $M$.
\end{definition}

\begin{proposition}[Matsumara 11A]
Let $A$ be a ring, $I\ideal A$ is an ideal of $A$, and $M$ is a module over $A$.
The topology formed by an essentially $I$-adic filtration on $M$
is the same as the $I$-adic topology on $M$. 
\end{proposition}

\begin{proof}
Observe $I^{i}M\subset M_{i}$ by $I$-admissibility,
and $M_{i}\subset I^{i-i_{0}}M_{i_{0}}$
and $I^{i-i_{0}}M_{i_{0}}\subset I^{i-i_{0}}M$ by $I$-admissibility.
This means that every neighborhood of the $I$-adic topology is
contained within, and contains, a neighborhood of the topology formed
by an essentially $I$-adic filtration, so they must be the same topologies.
\end{proof}

\begin{lemma}[Matsumara 11B]\label{lemma:fall25-lec24:preceding-lemma}
Let $A$ be a ring, $I\ideal A$ is an ideal of $A$, and $M$ is a module over $A$.
Suppose $M=M_{0}\supset M_{1}\supset M_{2}\supset\dots$ is an
$I$-admissible filtration such that all $M_{i}$ are finite $A$-modules.
Let $x$ be an indeterminate (i.e., a formal variable), so we can write
$A'=\sum_{n=0}^{\infty}I^{n}x^{n}$ (which is then just a graded ring)
and $M'=\sum^{\infty}_{n=0}M_{n}x^{n}$.

Then the filtration $M_{i}$ is essentially $I$-adic if and only if
$M'$ is finitely-generated over $A'$.
\end{lemma}

\begin{proof}
We see that $A'\subset A[x]$ is a graded subring,
and similarly $M'\subset M\otimes_{A}A[x]$ is an Abelian subgroup such
that $A'M'\subset M'$ which implies $M'$ is a graded $A'$-module.

\backwardproof\ Assume $M'$ is finitely-generated over $A'$. Then we
can write $M'=A'\xi_{1}+\dots+A'\xi_{r}$ with $\xi_{i}\in M'_{d_{i}}$
and $\deg(\xi_{i})=d_{i}$. Then $M'_{n}=(Ix)M'_{n-1}$ implies
$M_{n}=IM_{n-1}$ for all $n>\max(d_{1},\dots,d_{r})$. This implies the
filtration $M_{i}$ is essentially $I$-adic.

\forwardproof\ Assume the filtration $M_{i}$ is essentially $I$-adic.
So $M_{n}=IM_{n-1}$ for all $n>d$. Then $M'$ generated over $A'$ by
$M_{d-1}x^{d-1}+\dots+M_{1}x+M_{0}$ which is generated by
finitely-many elements of $A$.
\end{proof}

\begin{theorem}[Artin--Rees]\label{math120a:artin--rees}
If $A$ is a Noetherian ring, $I\ideal A$ is an ideal, and $M$ is a
finite $A$-module, and $N\subset M$ is some submodule of $M$, then
there exists some $r\in\ZZ$ such that $I^{n}M\cap N=I^{n-r}(I^{r}M\cap N)$
for all $n>r$.
\end{theorem}

(Lemma 5.1 in Eisenbud, Theorem 11C in Matsumara.)

So $I^{n}M\cap N$ is a filtration on $N$, and Artin--Rees implies this
filtration is essentially $I$-adic.

\begin{proof}
(1) The filtration is $I$-admissible (obvious).

(2) 
We introduce
\begin{subequations}
  \begin{align}
N' &= \sum^{\infty}_{n=0}(I^{n}M\cap N)x^{n}\\
M' &= \sum^{\infty}_{n=0}I^{n}Mx^{n}\\
A' &= \sum^{\infty}_{n=0}I^{n}x^{n}.
  \end{align}
\end{subequations}
So $N'$ is a graded submodule of $M'$, and $M'$ is a graded module
over $A'$, and $A'$ is a graded ring.

Since $A$ is Noetherian, the ideal $I$ is finitely-generated. Let
us name these generators as:
\begin{equation}
I = a_{1}A+\dots+a_{r}A.
\end{equation}
Then $A'=A[a_{1}x,\dots,a_{r}x]$ is Noetherian as an Algebra over $A$.
Then we use Lemma~\ref{lemma:fall25-lec24:preceding-lemma} to prove
the claim.
\end{proof}

\begin{remark}
There are two critical conditions in the Artin--Rees theorem:
\begin{enumerate}
\item $A$ is Noetherian, otherwise we can generate counter-examples to
  the claim;
\item $M$ is finitely-generated module over $A$.
\end{enumerate}
\end{remark}