%%
%% fall-lecture11.tex
%% 
%% Made by Alex Nelson <pqnelson@gmail.com>
%% Login   <alex@lisp>
%% 
%% Started on  2025-10-23T11:08:02-0700
%% Last update 2025-10-23T11:08:02-0700
%% 

\lecture{}

\begin{remark}
The $\Tor$ functor is a ``higher tensor product'' measuring failure of flatness.
\end{remark}

\begin{remark}
If we have an extension of an $A$ module $M$ by $N$, then
\begin{equation}
0\to N\to\widehat{M}\to M\to 0
\end{equation}
is a short exact sequence, and then $\Ext^{1}(M,N)$ classifies the
possible extensions of $M$ by $N$.
\end{remark}

\begin{proposition}[3E]
Let $\varphi\colon A\to B$ be flat, let $M$ and $N$ be modules over
$A$. We write $M_{(B)}=M\otimes_{A}B$. Then
\begin{enumerate}
\item $\Tor^{A}_{i}(M,N)\otimes_{A}B=\Tor^{B}_{i}(M_{(B)},N_{(B)})$
\item If $A$ is Noetherian and $M$ is finitely-generated over $A$,
  then we also have $\Ext^{i}_{A}(M,N)\otimes_{A}B=\Ext^{i}_{B}(M_{(B)},N_{(B)})$.
\end{enumerate}
\end{proposition}

\begin{proof}[Proof (of (1))]
If we want to compute $\Tor$ functors, we need to take some
resolution. We take a projective resolution of $M$,
\begin{equation}
X_{*}\colon\quad\dots\to X_{1}\to X_{0}\to M\to 0,
\end{equation}
and then we tensor it with $B$:
\begin{equation}
X_{*}\otimes_{A} B\colon\quad\dots\to X_{1(B)}\to X_{0(B)}\to M_{(B)}\to 0.
\end{equation}
Is this a projective resolution? Recall, $P$ is projective means there is
another module $Q$ such that their direct sum is free $P\oplus Q = F$.
But then base change gives us $P_{(B)}\oplus Q_{(B)}=F_{(B)}$. So all
the modules in the tensor product $X_{i(B)}$ are projective. Then by
flatness, the sequence,
\begin{equation}
X_{*}\colon\quad\dots\to X_{1(B)}\to X_{0(B)}\to M_{(B)}\to 0,
\end{equation}
is exact. Now we have $\Tor^{A}_{i}(M,N)=H_{i}(X_{*}\otimes N)$ by definition.
Then
\begin{equation}
\Tor^{B}_{i}(M_{(B)},N_{(B)})=H_{i}(X_{*}\otimes_{A}\underbrace{N\otimes B}_{=N_{(B)}}).
\end{equation}
So $M_{(B)}$ has a projective resolution (where we go from the first
line to the second line by an exercise):
\begin{subequations}
\begin{align}
X_{*(B)} &= (X_{*}\otimes_{A}B)\otimes_{B}(N\otimes_{A}B)\\
&= X_{*}\otimes_{A}N\otimes_{A}B.
\end{align}
\end{subequations}
Now we use $B$ is a flat module over $A$. We then have
\begin{subequations}
\begin{align}
\Tor^{B}_{i}(M_{(B)},N_{(B)}) &= H_{i}(X_{*}\otimes_{A}N\otimes_{A}B)\\
&= H_{i}(X_{*}\otimes_{A}N)\otimes_{A}B\\
\intertext{and by flatness}
&= \Tor^{A}_{i}(M,N)\otimes_{A}B.
\end{align}
\end{subequations}
For a generic complex of modules over $A$,
\begin{equation}
M'\xrightarrow{f}M\xrightarrow{g}M'',
\end{equation}
we see the homology of $M$ is defined to be
$H_{1}=\ker(g)/\Im(f)$. Now tensoring the generic complex by $B$,
\begin{equation}
M'_{(B)}\xrightarrow{f\otimes_{A}B}M_{(B)}\xrightarrow{g\otimes_{A}B}M''_{(B)},
\end{equation}
and the homology group is then
\begin{equation}
H^{(B)}_{1}=\ker(g\otimes_{A}B)/\Im(f\otimes_{A}B)=\bigl(\ker(g)/\Im(f)\bigr)\otimes_{A}B
\end{equation}
where flatness justifies the second equality. Hence the result.
\end{proof}

\begin{proof}[Proof sketch (of (2))]
Take $\hom_{B}(X_{i}\otimes_{A}B,N\otimes_{A}B)\stackrel{?}{=}\hom_{A}(X_{i},N)\otimes_{A}B$?
There is some slightly nonobvious map we need to find. We need all the
$X_{i}$ free (we have a free resolution this time). Recall if $A$ is
Noetherian and $M$ is finite, then we can cook up a free and
finitely-generated resolution
\begin{equation}
\dots\to A^{n_{2}}\to A^{n_{1}}\to A^{n_{0}}\to M\to 0.
\end{equation}
For this situation, we have the identities
\begin{equation}
\Tor^{A_{\mathfrak{p}}}_{i}(M_{\mathfrak{p}},N_{\mathfrak{p}})=\bigl(\Tor^{A}_{i}(M,N)\bigr)_{\mathfrak{p}}
\end{equation}
and similarly for $\Ext$ under the assumptions of the claim.
\end{proof}

\begin{slogan}
Exact functors (like tensoring with flat modules) preserves kernels,
images, and homologies.
\end{slogan}

\subsection{Faithful Flatness}

\begin{theorem}[Matsumara (4A)]\label{thm:fall-lec11:equivalent-criteria-for-ff}
Let $A$ be a ring, let $M$ be a module over $A$. Then the following
are equivalent:
\begin{enumerate}
\item $M$ is faithfully flat over $A$
\item $M$ is flat and for any nonzero $N\neq0$ module over $A$,
  $M\otimes N\neq0$,
\item $M$ is flat over $A$ and for any maximal ideal
  $\mathfrak{m}\ideal A$, we have $\mathfrak{m}M\neq M$.
\end{enumerate}
So, the intuition is: $M$ is flat iff it is not a zero-divisor using
the tensor product.
\end{theorem}

\begin{proof}
$(1)\implies(2)$ Suppose $N\otimes M=0$. Then take the sequence $0\to N\to0$
and tensor it with $M$:
\begin{equation}
0\to \underbrace{N\otimes M}_{=0}\to 0.
\end{equation}
But this means, since $M$ is faitfully flat, that $N=0$.

$(2)\implies(3)$ Obvious. Eh, we'll skip the rest of the proof.
\end{proof}

\begin{example}
Consider $\CC[x]$ and $\CC[[x]]$. Is $\CC[[x]]$ a faithfully flat
$\CC[x]$ module?

Consider the module $\CC[x]/(x-1)\iso\CC$. What happens if we tensor
it with
\begin{equation}
\CC[[x]]\otimes_{\CC[x]}\bigl(\CC[x]/(x-1)\bigr)=?
\end{equation}
Well, consider the element $x\otimes1=1\otimes x=1\otimes 1$ in this
module. This feels bad, since we can then convert every term in the
formal power series into a constant term, but we cannot evaluate an
arbitrary power series. So this suggests we should think of zeroes,
divide by zeroes, and zero modules.

Let us try to prove
\begin{equation}
\CC[[x]]\otimes_{\CC[x]}\bigl(\CC[x]/(x-1)\bigr)=0.
\end{equation}
Consider the fact that $(1-x)$ has an inverse in $\CC[[x]]$, namely
$\sum_{n=0}^{\infty}x^{n}$. Then we see
\begin{subequations}
\begin{align}
1\otimes1 &= \left[(1-x)\left(\sum_{n=0}^{\infty}x^{n}\right)\right]\otimes1\\
&=\left(\sum_{n=0}^{\infty}x^{n}\right)\otimes(1-x)\\
&=\left(\sum_{n=0}^{\infty}x^{n}\right)\otimes0\\
&=1\otimes0=0.
\end{align}
\end{subequations}
The intuition: although $\CC[x]$ corresponds to a line, $\CC[[x]]$ is
an infinitesimal neighborhood of $0\in\CC$ distinct from $\CC[x]/(x-1)$
(which is ``some other'' neighborhood of $0\in\CC$).
\end{example}

\begin{remark}
There is a map
\begin{equation}
\CC[x]_{(x)}\to\CC[[x]],
\end{equation}
which is just a power series expansion.

J.P.~Serre investigates these examples. He showed that
\begin{equation*}
\xymatrix{
& \CC[[x]] &\\
\CC[x]_{(x)}\ar[ur]^{\text{f.f.}}\ar@{-->}[rr]^{\text{f.f.}}& &\ar[ul]_{\text{f.f.}}\CC(x)}
\end{equation*}
then we can relate algebraic geometry to complex geometry, which
revolutionized things.
\end{remark}