%%
%% fall-lecture03.tex
%% 
%% Made by Alex Nelson <pqnelson@gmail.com>
%% Login   <alex@lisp>
%% 
%% Started on  2025-10-05T09:30:30-0700
%% Last update 2025-10-05T09:30:30-0700
%% 

\lecture[Localization, Ring of Fractions]

\begin{node}
For any ideal $J\neq A$, we have this induce a map $\Spec(A/J)\into\Spec(A)$
which is an injective map. The image is just the primes $P\ideal A$
containing $J\subset P$. We should intuitively think $\Spec(A/J)\sim V(J)$
where $V(J)=\{P\in\Spec(A)\mid J\subset P\}$ is a closed set in the
Zariski topology.
\end{node}

\begin{example}
For $A=\CC[x,y]$, $J=\langle x^{2}-y^{3}\rangle$, $\Spec(A/J)$ is the
curve doodled in my notes.
\end{example}

\begin{definition}[Multiplicative set]
Let $A$ be a [commutative] ring.
We call the subset $S\subset A$ a \define{Multiplicative Set}
if
\begin{enumerate}
\item $1\in S$ (which means that multiplicative sets are always nonempty), and
\item $\forall x,y\in S\ldotp xy\in S$.
\end{enumerate}
(This could be defined for any unital nonempty magma $A$.)
\end{definition}

\begin{example}
Let $R$ be any ring. Let $x\in R$ be any element of the ring.
Then the set $\{x^{n}\in R\mid n\in\NN_{0}\}$ is a multiplicative set.
\end{example}

\begin{proposition}
Let $S$ be a multiplicative set of $A$, let $0\notin S$.
Let $\mathcal{M}$ be the set of all prime ideals of $A$ disjoint from $S$.
Then $(0)\in\mathcal{M}$. By Zorn's lemma, we get a maximal element
$P$ [a chain] of $\mathcal{M}$. Then $P$ is a prime ideal.
\end{proposition}

\begin{definition}[Nilradical]
Let $A$ be a commutative ring.
We define the \define{Nilradical} of $A$ to be the ideal
$\nilradical{A}$ consisting of all nilpotent elements of $A$.
\end{definition}

\begin{proposition}\label{fall:lec3:prop-ii}
The nilradical of $A$ is the intersection of all prime ideals,
\begin{equation*}
\nilradical{A}=\bigcap\Spec(A).
\end{equation*}
\end{proposition}

\begin{proof}
Every prime ideal $P$ contains $0$, so $a\cdot a^{n-1}=0$ is in the
prime ideal $P$. This proves $\nilradical{A}\subset\bigcap\Spec(A)$.

Proving $\nilradical{A}\supset\bigcap\Spec(A)$: let $b\in\bigcap\Spec(A)$,
then form $\{b^{n}\mid n\in\NN_{0}\}=S$ which is a multiplicative set
and does not contain zero. Then there is a prime ideal not containing
$S$ which is a contradiction.
\end{proof}

\begin{proposition}
If $J\ideal A$, then
\begin{equation*}
\Radical{J}=\bigcap\{P\in\Spec(A)\mid J\subset P\}.
\end{equation*}
\end{proposition}

\begin{proof}[Proof sketch]
This followed from Proposition~\ref{fall:lec3:prop-ii} applied to the
quotient ring $A/J$.
\end{proof}

\begin{definition}
We say $A$ is \define{Reduced} if $\nilradical{A}=0$. We can always form
\begin{equation}
A_{\text{red}} := A/\bigcap\Spec(A),
\end{equation}
which is reduced.
\end{definition}

\begin{example}
Consider $\kk[x]/(x^{2})$. We see its nilradical is
$\nilradical{\kk[x]/(x)^{2}}=(x)$, so
\begin{equation*}
\bigl(\kk[x]/(x^{2})\bigr)/(x)\iso\kk.
\end{equation*}
\end{example}

\begin{definition}
Let $S\subset A$ be a multiplicative set.
We define the \define{Localization} (or \emph{Ring of Fractions}) of
$A$ with respect to $S$ as
\begin{equation}
S^{-1}A = \{{\textstyle\frac{a}{s}}\mid a\in A, s\in S\}/\sim
\end{equation}
where $\sim$ is the equivalence relation
\begin{equation}
\frac{a}{s}\sim\frac{b}{s'}\iff\exists s''\in S\ldotp s''(s'a-sb)=0,
\end{equation}
where the $s''$ factor is needed (if we omit it, we would not have an
equivalence relation, because we can have zero divisors). Addition and
multiplication are both defined in the usual way.
\end{definition}

\begin{proposition}
$S^{-1}A=0\iff 0\in S$ because we can divide by zero.
\end{proposition}
\begin{proof}
\begin{equation*}
\frac{1}{0}=\frac{a}{s}\iff s'(s-0\cdot a)=0,
\end{equation*}
since $s'=0$.
\end{proof}

\begin{definition}[Natural map]
Let $S\subset A$ be a multiplicative set.
We define the \define{Natural Map} to be the morphism $A\to S^{-1}A$
sending $a\mapsto \frac{a}{1}$.
\end{definition}

\begin{proposition}[Existence of injective natural map]
We have a map $A\to S^{-1}A$, $a\mapsto\frac{a}{1}$.
(This is analogous to the inclusion $\ZZ\to\QQ$.)
Its kernel is zero and all zero-divisors.
\end{proposition}

\begin{example}
For $\ZZ$, there are a lot of multiplicative subsets. For example,
$(p)$ is a multiplicative set for each prime $p\in\ZZ$.

For $S=\ZZ\setminus\{0\}$, which is a multiplicative subset of $\ZZ$,
we see $S^{-1}\ZZ=\QQ$.
\end{example}

\begin{definition}
Let $A$ be an integral domain. We define the \define{Field of Fractions}
to be the ring $\Frac(A)$ equal to the localization of $A$ with
respect to the nonzero elements $(A\setminus\{0\})^{-1}A$.
\end{definition}

\begin{example}
Let $S\subset\ZZ$ consist of all odd integers. Then
\begin{equation*}
S^{-1}\ZZ=\left\{\frac{a}{b}\mid\mbox{$b$ is odd}\right\}.
\end{equation*}
\end{example}

\begin{proposition}
Let $P\in\Spec(A)$, $S:=A\setminus P$. Then $S^{-1}A=AP$.
\end{proposition}

\begin{node}[Universal property of localization]\label{node:fall:lecture3:universal-property-of-localization}
For any construction of gadgets, we can do it explicitly (as we have
done) or we can use a universal property. For localizations, we have
for any ring $B$ and ring morphism $f\colon A\to B$, the following
diagram commutes:
\begin{equation}
\xymatrix{
A\ar[dr]_{f}\ar[rr] & & \ar@{.>}[dl] S^{-1}A\\
& B & }
\end{equation}
The map $A\to S^{-1}A$ is the ``canonical injection'' sending
$a\mapsto\frac{a}{1}$. The dashed arrow is unique. After a bit of
tinkering, we see it sends
\begin{equation}
\frac{a}{s}\mapsto\frac{f(a)}{f(s)}.
\end{equation}
When we have ``another'' ring $B$ proposed as the localization, the
universal property says there exists a unique morphism $S^{-1}A\to B$
and another unique morphism $B\to S^{-1}A$ such that the compositions
are identity morphisms.

This means that the localization of $A$ with respect to $S$ is unique
up to unique isomorphism (i.e., if two distinct localizations exist,
then there is a unique isomorphism from one to the other). However,
universal properties never assert the existence of the construction, only
the (comparative) uniqueness of the construction.
\end{node}

\begin{node}
Let $S$ be a multiplicative set of $A$.
Let $P$ be a prime (resp., primary) ideal of $A$ such that it is
disjoint from $S$, i.e., $P\cap S=\emptyset$.
Then we can form $P(S^{-1}A)=i(P)(S^{-1}A)$ --- where $i\colon A\into S^{-1}A$
is the canonical injection --- is a prime (resp., primary) ideal of $S^{-1}A$.

This means there exists a canonical mapping $\Spec(S^{-1}A)\to\Spec(A)$
which is a bijection to
\begin{equation}
\Spec(S^{-1}A)\iso\{P\in\Spec(A)\mid P\cap S=\emptyset\}.
\end{equation}
\end{node}

\begin{example}
For $i\colon\ZZ\into\QQ$, we see $i(2)$ is not an ideal in $\QQ$, but
$i(2)\QQ$ \emph{is} the smallest ideal of $\QQ$ containing
\begin{equation*}
\{aq\mid a\in i(2), q\in\QQ\}.
\end{equation*}
\end{example}

\begin{example}
  Let $A_{P}=(A\setminus P)^{-1}A$, then
  \begin{equation}
\Spec(A_{P}) = \{Q\in\Spec(A)\mid Q\subset P\}.
  \end{equation}
\end{example}

\begin{example}
Roughly, the localization $\CC[x,y]_{(x,y)}$ describes geometrically
all curves passing through $(x,y)$.
\end{example}

\subsection{Tensor Product}

\begin{node}
The basic example of tensor products of vector spaces $A$ and $B$ over
the same field $\kk$ is a vector space $A\otimes B$. We define it
roughly as follows: let $\{e_{1},\dots, e_{m}\}$ be a basis of $A$,
let $\{f_{1},\dots,f_{n}\}$ be a basis of $B$. Then
$\{e_{i}\otimes b_{j}\mid i=1,\dots,m; j=1,\dots,n\}$ is a basis of $A\otimes B$.

More generally, for modules $A$ and $B$ over a commutative ring $R$,
we form their tensor product $A\otimes_{R}B$ which is a module over
$R$ as consisting of vectors $m\otimes n\in A\otimes_{R}B$ (where
$m\in A$, $n\in B$) such that
\begin{subequations}
\begin{align}
m\otimes(n+n')&=(m\otimes n)+(m\otimes n')\\
(rm)\otimes n&=m\otimes(rn)=r(m\otimes n),
\end{align}
\end{subequations}
for any $r\in R$, $m\in A$, $n,n'\in B$ --- there are more equations,
but you get the idea.

We can also form the localization for a module $S^{-1}M$ as we did for
the ring, which can be viewed as $M\otimes_{A}S^{-1}A$ since $S^{-1}A$
is a module over $A$.
\end{node}