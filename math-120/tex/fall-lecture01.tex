%%
%% fall-lecture01.tex
%% 
%% Made by Alex Nelson <pqnelson@gmail.com>
%% Login   <alex@lisp>
%% 
%% Started on  2025-09-30T09:51:31-0700
%% Last update 2025-09-30T09:51:31-0700
%% 

\lecture{}

\begin{node}
This course is going to be about commutative algebra. The rough plan
is to follow Matsumara's \textit{Commutative Ring Theory} (last year
it was Atiyah and MacDonald). Matsumara's book was an improvement on
his earlier \textit{Commutative Algebra}. [Addendum: the professor
seems mixed up, at least in the following sense: We are using
Matsumara's \textit{Commutative Algebra}~\cite{matsumura1970commutative}. Citations to Matsumara refer
to \textit{Commutative Algebra}.]

But if you want even more detail, Eisenbud's
\textit{Commutative Algebra with a View Toward Algebraic Geometry}
(Springer, 1994) has a lot of examples. The professor will be skipping
a lot of proofs, but expects us to read the proofs from Matsumara (or
another text) on our own. (There will not be much category theory in
this course, the focus will be on explicit calculations.)
\end{node}

\begin{node}[Open problem]
Consider the infinite series
\begin{equation}
(1 - q) + (1 - q)(1 - q^{2}) + (1 - q)(1 - q^{2})(1 - q^{3}) + \cdots.
\end{equation}
What happens if we expand it? Well, each term contributes a constant
$1$ term, of which there are infinitely many. So the constant term
diverges. Also, every term linear in $q$ contributes $-q$, which means
the linear term would diverge.

But \emph{as a function},
\begin{equation}
f(q) = (1 - q) + (1 - q)(1 - q^{2}) + (1 - q)(1 - q^{2})(1 - q^{3}) + \cdots,
\end{equation}
we see $f(1)=0$, $f(-1)=2$, and for any root of unity $\exp(\I 2\pi/n)$
we see this function will be well-defined (since it only involves the
first $n-1$ terms). This function is actually an element of the
projective limit
\begin{equation}
\widehat{\ZZ[q]} = \varprojlim_{n}{\ZZ[q]/(q)_{n}}
\end{equation}
where $(q)_{n}=(1-q)(1-q^{2})(\cdots)(1-q^{n})$. Although the units of
$\ZZ$ are $\pm1$, and the units of $\ZZ[q]$ are still $\pm1$, in
$\widehat{\ZZ[q]}$ we find $q$ is a unit because
\begin{equation}
\sum_{n\geq0}q^{n+1}(q)_{n}=\sum\bigl(1-(1-q^{n+1})\bigr)(q)_{n}=\sum_{n\geq0}(q)_{n}-(q)_{n+1}=(q)_{0}=1.
\end{equation}
\textbf{The conjecture:} $\widehat{\ZZ[q]}$ has units and $\{\pm q^{m}\mid m\in\ZZ\}$
are units, but are these all the units? Kazuo Habiro conjectured about
this in his ``Cyclotomic completions of polynomial rings'' (c.f.,
Conjecture~7.2 in \arXiv{math/0209324}).
\end{node}

\begin{convention}
Rings are always commutative and unital.
\end{convention}

\begin{definition}[The radical of an ideal]
Let $A$ be a ring, let $\mathfrak{a}$ be an ideal of $A$. We make the
set of $x$ such that $x^{n}\in\mathfrak{a}$ for some $n\in\NN$. This
is called the \define{Radical} of $\mathfrak{a}$, denoted $\Radical{\mathfrak{a}}$.
\end{definition}

\begin{definition}[Prime ideals]
An ideal $P$ of $A$ is \define{Prime} in $A$ if $A/P$ is an integral
domain (in particular, $A/P\neq0$, and also $A\neq P$). This means for
any $\bar{x}$, $\bar{y}\in A/P$, if $\bar{x}\bar{y}=0$,
then necessarily either $\bar{x}=0$ or $\bar{y}=0$.

Equivalently, for any $x$, $y\in A$, if $xy\in P$, then either $x\in P$ or
$y\in P$.
\end{definition}

\begin{definition}[Primary ideals]
An ideal $Q$ of $A$ is called \define{Primary} if $Q\neq A$ and if the
only zero-divisors of $A/Q$ are \emph{nilpotent}; equivalently, if $xy\in Q$
and $x\notin Q$, then $y^{n}\in Q$ for some $n\in\NN$.
\end{definition}

\begin{definition}
If $Q$ is a primary ideal of $A$, then $\Radical{Q}=P$ is a prime
ideal of $A$. We say in this case $P$ and $Q$ \define{belong to each other}
(the exact terminology varies author to author).
\end{definition}

\begin{proposition}
Let $Q$ is a primary ideal of $A$, and let $\mathfrak{m}$ is a maximal
ideal of $A$.
If $Q$ contains $\mathfrak{m}^{n}$ for some $n\in\NN$,
then $Q$ is a primary ideal belonging to $\mathfrak{m}$.
\end{proposition}

\begin{example}
We see that $(x,y)\subset\CC[x,y]$ is a maximal ideal, and
$(x^{2}-y^{2})\subset\CC[x,y]$ is a primary ideal. The maximal ideals
correspond to \emph{points}. The prime ideals correspond to \emph{curves}.
The ideal $(x^{2},y)\subset\CC[x,y]$ is a primary ideal and
corresponds to a ``bigger point'', an ``infinitesimal blob'' in the
$x$ direction (but a point in the $y$ direction). This is drawn as an
elongated point at $(x,y)$.
\end{example}

\begin{example}
The ring $\ZZ$ has prime ideals $(2)$, $(3)$, $(5)$, $(7)$, etc., and the
zero ideal. The primary ideals are generated by powers of primes
$(p^{n})$ where $p$ is a prime integer and $n\in\NN$.
\end{example}

\begin{definition}[Prime spectrum of a ring]
Let $A$ be a ring. We define the \define{Spectrum} of $A$ to be the
set $\Spec(A)$ consisting of the set of all prime ideals of $A$. We
pronounce it as ``speck $A$''.

Some authors refer to this as the \emph{prime spectrum} of $A$, others
just refer to it as the spectrum of $A$. Everyone appears to use the
same notation $\Spec(A)$.
\end{definition}

\begin{definition}[Maximal spectrum of ring]
We define the set $\MSpec(A)$ to be the set of all maximal ideals of
$A$. (Matsumara uses the notation $\Omega(A)$.)
\end{definition}

\begin{example}
We see $\MSpec(\CC[x,y])$ is intuitively the set of all points. The
$\Spec(\CC[x,y])$ intuitively is the set of all curves.
\end{example}

\begin{node}
The $\Spec(A)$ is a topological space, but it is a \emph{very}
pathological space. If we take any subset $M\subset A$, we may form
\begin{equation}
V(M) = \{p\in\Spec(A)\mid M\subset p\}
\end{equation}
and take $V(M)$ as the \emph{closed subsets} of $\Spec(A)$. We would
need to check the axioms of a topological space are satisfied
(arbitrary intersections of closed sets is closed, finite unions of
closed sets is closed). This topology is known as the \define{Zariski Topology}.

What are the open sets for the line? The line is described as
$\CC[x]$. Then the ideals are of the form $(x-a)$ for any $a\in\CC$,
and the zero ideal. The only ideals look like $\bigl(f(x)\bigr)$ for
some polynomial $f\in\CC[x]$ (but this factorizes into a product of
guys of the form $(x-a)$ thanks to the fundamental theorem of
algebra). So $(x^{2}-1)$ corresponds to two points since
\begin{equation}
x^{2}-1=(x-1)(x+1),
\end{equation}
so the points are $\pm1$. Then the closed sets are just finite sets of
points, the open sets are everything else. The open set associated
with $f$ is
\begin{equation}
D(f) = \Spec(A)\setminus V(f).
\end{equation}
For $M=\{(x^{2}-y^{2})\}$, we see it is contained in the ideals
$M\subset(x,y)$ and $M\subset(x^{2},y)$, but $(x^{2},y)$ is not
prime. We could consider $(x_{0},y_{0})$ satisfying
\begin{equation}
x_{0}^{2}-y_{0}^{3}=0,
\end{equation}
then $(x-x_{0},y-y_{0})$ contains $M$. To see this, we could also
Taylor expand $x^{2}-y^{3}$ about, say, $(1,1)=(x,y)$. We could also
take $\CC[x,y]/(x-1,y-1)$ and observe that $(x^{2}-y^{3})$ vanishes
under the natural map of the quotient. This is an example of $D(f)$
where $f=x^{2}-y^{3}$, and we call these \define{Elementary Open Sets}.
\end{node}