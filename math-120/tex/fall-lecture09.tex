%%
%% fall-lecture09.tex
%% 
%% Made by Alex Nelson <pqnelson@gmail.com>
%% Login   <alex@lisp>
%% 
%% Started on  2025-10-19T10:03:29-0700
%% Last update 2025-10-19T10:03:29-0700
%% 

\lecture{}

We want to find equivalent criteria for flatness.

\begin{theorem}
The following are equivalent:
\begin{enumerate}
\item $M$ is flat;
\item For any exact sequence of modules $0\to N'\to N$, we see $0\to N'\otimes M\to N\otimes M$ is exact;
\item For any finitely-generated ideal $I\ideal A$, $0\to I\otimes M\to M$
  is exact (the sequence $0\to I\to A$ is always exact), so we can say
  $I\otimes M\iso IM$;
\item $\Tor^{A}_{1}(M,A/I)=0$ for all finitely-generated ideals
  $I\ideal A$;
\item $\Tor^{A}_{1}(M,N)=0$ for any $A$-module $N$;
\item If $a_{i}\in A$ and $x_{i}\in M$ for some $i=1,\dots,r$, and
  \begin{equation}
\sum^{r}_{i=1}a_{i}x_{i}=0,
  \end{equation}
  then there exists some $s\in\NN$ and $b_{ij}\in A$ and $y_{j}\in M$
  for $j=1,\dots,s$ such that
  \begin{enumerate}[label=(\roman*)]
  \item $\sum^{r}_{i=1}a_{i}b_{ij}=0$ for all $j$ --- i.e.,
    $\vec{a}^{T}B=0$; and
  \item $x_{i}=\sum^{s}_{j=1}b_{ij}y_{j}$ for all $i$ --- i.e., $\vec{x}=B\vec{y}$.
  \end{enumerate}
\end{enumerate}
\end{theorem}

\begin{example}
The $\ZZ$-module $\QQ$ is flat but not projective. If we take
$2\cdot\frac{1}{2}+(-3)\cdot\frac{1}{3}=0$, then we can rewrite it as
$2\cdot3\cdot\frac{1}{6}+2\cdot(-3)\cdot\frac{1}{6}=0$. This
satisfies the last criterion. The indexes
here are $i=1,2$ and $j=1$. The $y_{1}=1/6$.
\end{example}

\begin{example}
Consider $\ZZ/5\ZZ$. Let us view $\bar{1}\in\ZZ/5\ZZ$ as an element of a $\ZZ$-module.
Then $5\cdot\bar{1}=0$, but it fails to satisfy the last criterion,
therefore $\ZZ/5\ZZ$ is not a flat $\ZZ$-module.

These two examples should demonstrate the usefulness of the seemingly
random last criterion in our theorem.
\end{example}

\begin{proposition}[Transitivity (3.B)]
If $\phi\colon A\to B$ is a ring morphism and $\phi$ makes $B$ a flat
$A$-module. Then a flat module $N$ over $B$ is also a flat module over $A$.
\end{proposition}

\begin{proposition}[Change of basis (3.C)]
If we have a ring morphism $\phi\colon A\to B$ and a flat module $M$
over $A$, then $M\otimes_{A}B$ is flat (as a module over $B$).
\end{proposition}

\begin{remark}
This is a ``change of basis'' in the sense that we are changing the
base ring of scalars.
\end{remark}

\begin{remark}
We should think of rings as functions on topological spaces (like the
ring of continuous functions, the ring of smooth functions, the ring
of analytic functions, the ring of holomorphic functions, etc.). Then
flat and free modules correspond to vector bundles. So this tells us
pulling back vector bundles gives us vector bundles.
\end{remark}


\begin{proposition}[Localization (3.D)]
Let $S\subset A$ be a multiplicative subset. Then $S^{-1}A$ is a flat
module over $A$.
\end{proposition}


\begin{proposition}[3H]
Let $A\to B$ be a flat map of rings. Let $I_{1},I_{2}\ideal A$ be ideals.
Then
\begin{enumerate}
\item $(I_{1}\cap I_{2})B=(I_{1}B)\cap(I_{2}B)$
\item if $I_{2}$ is finitely-generated, then $(I_{1}:I_{2})B = (I_{1}B:I_{2}B)$.
\end{enumerate}
\end{proposition}