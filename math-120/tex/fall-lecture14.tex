%%
%% fall-lecture14.tex
%% 
%% Made by Alex Nelson <pqnelson@gmail.com>
%% Login   <alex@lisp>
%% 
%% Started on  2025-10-30T08:16:22-0700
%% Last update 2025-10-30T08:16:22-0700
%% 

\lecture{}

\begin{definition}[Going-up property]
Let $\phi\colon A\to B$ be a morphism of rings.
We say the rings satisfy the \define{Going-Up Property} (or ``going-up
holds for $\phi$'') if:
for any prime ideals $P$, $P'\in\Spec(A)$ such that $P\subset P'$,
for any prime ideal $Q\in\Spec(B)$ lying over $P$,
there exists a prime ideal $Q'\in\Spec(B)$ lying over $P'$
such that $Q\subset Q'$.

Graphically, given the black data, the
going-up property asserts the existence of the red data:
\begin{equation*}
\begin{array}{lcccc}
B\colon&\quad& Q & {\color{red}\subset} & {\color{red}Q'}\\
       &     & \rotatebox{90}{$\subset$} & & {\color{red}\rotatebox{90}{$\subset$}}\\
A\colon&\quad& P & \subset & P'
\end{array}
\end{equation*}
\end{definition}

\begin{definition}[Going-down property]
Let $\phi\colon A\to B$ be a morphism of rings.
We say the rings satisfy the \define{Going-Down Property} (or, ``going-down
holds for $\phi$'') if:
for any prime ideals $P$, $P'\in\Spec(A)$ such that $P\subset P'$,
for any prime ideal $Q'\in\Spec(B)$ lying over $P'$,
there exists a prime ideal $Q\in\Spec(B)$ lying over $P$
such that $Q\subset Q'$.

Graphically, given the black data, the
going-down property asserts the existence of the red data:
\begin{equation*}
\begin{array}{lcccc}
B\colon&\quad& {\color{red}Q} & {\color{red}\subset} & Q'\\
       &     & {\color{red}\rotatebox{90}{$\subset$}} & & \rotatebox{90}{$\subset$}\\
A\colon&\quad& P & \subset & P'
\end{array}
\end{equation*}
\end{definition}

\begin{remark}
This terminology (``going up'' and ``going down'') refers to the
existence of an ascending or descending chain of ideals. Algebraists
named these properties, which is why it refers to ``going up'' the
chain (or ``going down'' the chain) of ideals.
\end{remark}

\begin{proposition}[Matsumara GD']
Let $\phi\colon A\to B$ be a ring morphism.
Then the going-down property is satisfied if and only if:
for any prime ideal $P\in\Spec(A)$ and for any minimal prime
over-ideal $Q$ of $PB$, we have $Q\cap A=P$.
\end{proposition}

\begin{example}[Matsumara 5.1]
Let $\kk[x]$ be a ring, and suppose we have polynomials
\begin{equation}
x_{1}=x(x-1)
\end{equation}
and
\begin{equation}
x_{2}=x^{2}(x-1).
\end{equation}
The function field of $\kk[x]$ is $\kk(x)=\kk(x_{1},x_{2})$ --- since
$x_{2}/x_{1}=x$. We may consider a map
\begin{equation}
\begin{split}
\kk[x_{1},x_{2}]&\to\kk[x]\\
x_{1}&\mapsto x(x-1)\\
x_{2}&\mapsto x^{2}(x-1)
\end{split}
\end{equation}
Note/caution: the domain of this map is the ring of polynomials in two
unknowns, we are abusing notation (``punning''?) writing the unknowns
as $x_{1}$ and $x_{2}$, and mapping them to the polynomials of the
same name in $\kk[x]$.

Now, we want to find a relation between the polynomials $x_{1}$ and
$x_{2}$ such that $f\in\kk[x_{1},x_{2}]$ is mapped to zero under this
transformation. For example, we see
\begin{equation}
x_{1}^{3}-x_{2}^{2}+x_{1}x_{2}\mapsto0.
\end{equation}
This means we have a nodal curve geometrically.
The difficulty occurs at $x=0$ and $x=1$ are both mapped to
$x_{1}=x_{2}=0$, the picture we should have is something like:
\begin{equation*}
\includegraphics{img/img.1}
\end{equation*}
We have two ideals $(x=0)$ and $(x=1)$ sent to the same $(0,0)$ ideal
under the induced $\phi^{*}$ morphism.

Now, let $B=\kk[x,y]$ and
$A=\kk[x_{1},x_{2},y]/(x_{1}^{3}-x_{2}^{2}+x_{1}x_{2})$.
We extend $\phi$ to send $y\mapsto y$.
The intuition is that we're just ``adding another dimension''.
We take an ideal corresponding to the red curves in the following
sketch of the situation (the planar line is $y=ax$ for some $a\neq0$):
\begin{equation*}
\includegraphics{img/img.2}
\end{equation*}
We see that $B$ corresponds to the plane, $A$ corresponds to the\dots
exotic situation on the left. There are 4 points in the plane which
are mapped to the 2 points on the nodal surface.
\end{example}

\begin{theorem}[Matsumara 5D]
If $\phi\colon A\to D$ is flat, then going down holds for $\phi$.
\end{theorem}

\begin{proof}
Let $P'\subset P$ be prime ideals in $A$, and $Q\ideal B$ is lying over $P$.
Then $B_{Q}$ is flat over $A_{P}$ (by Theorem~\ref{thm:matsumara-3j}).
But $A_{P}\to B_{Q}$ is a local morphism, and so faithfully flat.
Therefore the map $\Spec(B_{Q})\onto\Spec(A_{P})$ is surjective
(because it's faithfully flat).
Let $Q'^{*}\in\Spec(B_{Q})$ lying over $P'A_{P}$ by surjectivity.
Then $Q'=Q'^{*}\cap B$ is a prime ideal of $B$ lying over $P'$.
\end{proof}

\begin{theorem}[Matsumara 5E]
Let $B$ be a ring, let $A\subset B$ be a subring over which $B$ is integral
(i.e., for each $b\in B$, there exists a monic $f\in A[x]$ such that $f(b)=0$).
Then
\begin{enumerate}
\item $\Spec(B)\to\Spec(A)$ is surjective
\item There is no inclusion relation between the prime ideals of $B$
  lying over a fixed ideal of $A$ (if $Q\in\Spec(A)$ and
  $P_{1},P_{2}\in\Spec(B)$ lie over $Q$, then we are not allowed to
  have either $P_{1}\subset P_{2}$ or $P_{2}\subset P_{1}$).
\item The going-up property holds.
\item If $A$ is local, and $P\ideal A$ is its maximal ideal, then the
  prime ideals of $B$ lying over $P$ are all the maximal ideals of $B$.
\end{enumerate}
\medbreak\noindent%
If further $A$ and $B$ are integral domains and $A$ is integrally
closed, then
\begin{enumerate}[resume]
\item Going-down holds.
\end{enumerate}
\end{theorem}