%%
%% fall-lecture10.tex
%% 
%% Made by Alex Nelson <pqnelson@gmail.com>
%% Login   <alex@lisp>
%% 
%% Started on  2025-10-21T10:38:13-0700
%% Last update 2025-10-21T10:38:13-0700
%% 

\lecture{}

\begin{remark}
The word ``flat'' (as in \emph{flat module}) is translated from the
French word \textit{plat}. Jean-Pierre Serre introduced it in his
paper ``G\'{e}ometrie Alg\'{e}brique et G\'{e}om\'{e}trie Analytique''
(1956).
\end{remark}

\begin{example}[Matsumara 3.3]
Let $A=\kk[x,y]$ be our ring, so $\Spec(A)\sim$ the plane
$\kk^{2}$. It's useful to think $\kk=\CC$.

We want a module $B=\kk[x,y,z]$ where $z=y/x$ formally. So
$B\sim\CC[x,z]$ because $y=zx$.

We have a map $f\colon A\to B$ sending $f(x)=x$ and $f(y)=zx$.

Let us now look at ideals $I_{1}$ and $I_{2}$ of $A$. Then flatness
for $B$ implies $(I_{1}\cap I_{2})B=(I_{1}B)\cap(I_{2}B)$ if $B$ were
flat. Take $I_{i}=xA$ and $I_{2}=yA$. Then $I_{1}\cap I_{2}=xyA$.

Observe
\begin{subequations}
\begin{align}
  (I_{1}\cap I_{2})B &= xy B\\
  &= x^{2}zB,
\end{align}
\end{subequations}
but
\begin{subequations}
\begin{align}
(I_{1}B)\cap (I_{2}B) &= (xB)\cap(yB)\\
  &=(xB)\cap(xzB)\\
  &=xzB
\end{align}
\end{subequations}
since $x$ divides $xz$. So the punchline: $B$ is not flat over $A$.

So $B=\kk[x,y,z]/(xz-y)$, right? We have some maps
\begin{equation}
\begin{array}{ccccc}
\kk[x,y,z] & \gets & A=\kk[x,y] & \to & B\\
x & \mapsfrom & x & \mapsto & x\\
y & \mapsfrom & y & \mapsto & xz
\end{array}
\end{equation}
Great, so what?

Well, these induce maps $\Spec(\kk[x,y,z])\to\Spec(A)$ which is a
projection to the plane, and $\Spec(B)\to\Spec(A)$. Let us draw the
solution set to $xz-y=0$. At $x=0$, we have $y=0$ and $z$ is
arbitrary. At $x=1$, we have $z-y=0$ or $z=y$. At $x=2$, we have
$2z-y=0$ or $2z=y$. At $x=-1$, we have $y=-z$. At $x=-2$, we have $y=-2z$.
We can draw some of these curves along the $yz$-plane:
\begin{equation*}
\includegraphics{img/img.0}
\end{equation*}
This $\Spec(B)\to\Spec(A)$ gives us another projection from this
surface defined by $xz-y=0$ to the $xy$-plane. At $x=0$, we project
the entire $z$-axis to a signle point on the plane.

The ideal $I_{1}$ corresponds to the $x=0$ plane, and $I_{2}$
corresponds to the $y=0$ plane. When we look at $I_{1}B$ this
corresponds to the intersection of the $x=0$ plane with the $B$
surface; and if we look at the $I_{2}B$, this corresponds to the
intersection of the $y=0$ plane with the $B$ surface.

So we see $xyA$ looks like the union of the $x$-axis with the
$y$-axis. The pre-image of the $x$-axis under the projection is the
union of the $x$-axis with the $z$-axis. The pre-image of the $y$-axis
is again the $z$-axis. But $x=0$ in $A$ corresponds to the
$z$-axis. (Well $x=0$ is the $y$-axis and $y=0$ is the $x$-axis\dots)
The $x$-axis appears with multiplicity 2, and we reflect this with the
ideal $x^{2}yA$. (Question: Is this a blow-up? Answer: Yes, it is!)
\end{example}

\begin{remark}
The previous example was offered for the justification of the word
``flat'': $B$ is not flat over $A$. If we look at the surface, the
surface contains the \emph{entire} $z$-axis, and therefore does not
look ``flat'' over the $(x,y)$-plane. I'm not convinced this is why
Serre called it flat, but here we are.
\end{remark}

\begin{example}[Matsumara 3.4]
Consider $A=\CC[x,y]$ and $B=\CC[x,y,z]$ and now consider the surface
$z^{2}=f(x,y)$ where $f\in A$. Then we see that $B=A\oplus Az$ so $A$
is free (and projective) and therefore flat.
\end{example}

\begin{remark}[Moral]
The lesson to be learned here: Examples are alwyas so much harder than
the theory. That's the moral of the story.
\end{remark}

\begin{proposition}[Matsumara, 3.G]
Let $(A,\mathfrak{m},\kk)$ be a local ring. Let $M$ be an $A$-module.
Suppose either $\mathfrak{m}$ is nilpotent or $M$ is finite. Then $M$
is free iff $M$ is projective iff $M$ is flat.
\end{proposition}

So, locally, these three attributes (free, projective, flat) are the
same. Since projective modules are analogous to vector bundles, we can
view this as the algebraic counterpart to vector bundles
[projective modules] locally look like products [free modules].

\begin{proof}
It suffices to prove $M$ is flat implies $M$ is free.
We see that $M/\mathfrak{m}M$ is a $\kk=(A/\mathfrak{m})$-vector space.
We will lift a basis of $M/\mathfrak{m}M$ to a minimal generating set
of $M$ (by NA Lemma). We want to show this generating set is linearly independent.

Let $x_{1}$, \dots, $x_{n}\in M$ be such that $\bar{x}_{1}$, \dots,
$\bar{x}_{n}$ is a basis for $M/\mathfrak{m}M$. We want $x_{1}$,
\dots, $x_{n}$ to be linearly independent. We do this by induction on
$n$.

\textsc{Base case} $n=1$: Then $ax=0$ implies
$\exists y_{1},\dots,y_{r}\in M$ and $b_{1}$, \dots, $b_{r}\in A$ such that
\begin{equation}
x=\sum_{i}b_{i}y_{i},
\end{equation}
and
\begin{equation}
ab_{i}=0\quad\mbox{for all }i.
\end{equation}
This is from the last theorem of the previous lecture. Assume
$b_{1}\notin\mathfrak{m}$. Then $b_{1}$ is a unit. Then $ab_{1}=0$
implies $a=0$. Hence the result for $n=1$.

\textsc{Inductive Case} $n>1$: Then
\begin{equation}
\sum_{i}a_{i}x_{i}=0.
\end{equation}
Then there exists $y_{1}$, \dots, $y_{r}\in M$ and $b_{ij}\in A$ (for $j=1,\dots,r$)
such that
\begin{equation}
x_{i}=\sum_{j}b_{ij}y_{j}
\end{equation}
and
\begin{equation}
\sum_{j}a_{i}b_{ij}=0.
\end{equation}
If $\bar{x}_{r}\neq0$, then the $b_{nj}\notin\mathfrak{m}$ for some $j$.
The rest of the argument is as before.
\end{proof}

\begin{proposition}[Matsumara 3J]\label{thm:matsumara-3j}%
Let $A\to B$ be a ring morphism. Then the following are equivalent:
\begin{enumerate}
\item $B$ is flat over $A$
\item $B_{P}$ is flat over $A_{\mathfrak{p}}$ (where
  $\mathfrak{p}=P\cap A$) for all prime ideals $P\in\Spec(B)$
\item $B_{P}$ is flat over $A_{\mathfrak{p}}$ (where
  $\mathfrak{p}=P\cap A$) for all maximal ideals $P\in\Spec(B)$.
\end{enumerate}
\end{proposition}

\begin{proof}
$(1)\implies(2)$: The ring $B_{\mathfrak{p}}=B\otimes A_{\mathfrak{p}}$
is flat over $A_{\mathfrak{p}}$ (by base change), and $B_{P}$ is a
localization of $B_{\mathfrak{p}}$, so $B_{P}$ is flat over
$A_{\mathfrak{p}}$ by transitivity.

$(2)\implies(3)$: Obvious.

$(3)\implies(1)$: Suffices to show $\Tor^{A}_{1}(B,N)=0$ for any
$A$-module $N$. (We know from homework: for any $N$ being an $A$-module,
and for any $P$ being a maximal ideal of $A$, that $N_{P}=0$ implies $N=0$.)
$\Tor^{A}_{i}(B,N)_{P}=\Tor^{A_{\mathfrak{p}}}_{i}(B_{P},N_{\mathfrak{p}})$. Etc.
\end{proof}