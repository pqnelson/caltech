%%
%% fall-lecture25.tex
%% 
%% Made by Alex Nelson <pqnelson@gmail.com>
%% Login   <alex@lisp>
%% 
%% Started on  2025-11-25T08:18:09-0800
%% Last update 2025-11-25T08:18:09-0800
%% 

\lecture{}
% The professor could not make it, and the TA Edward Hou lectured today

\begin{theorem}[Krull intersection]\marginpar{{\footnotesize Matsumara 11D}}\ignorespaces%
Let $A$ be a Noetherian ring, let $I\ideal A$ be an ideal, and let $M$
be a finite module over $A$. If $N=\bigcap^{\infty}_{n=0}I^{n}M$, then $IN=N$.
\end{theorem}

\begin{proof}
By Artin--Rees, there exists an $r\in\NN$ such that for $n>r$,
\begin{subequations}
  \begin{align}
N&=I^{n}M\cap N\\
 &=I^{n-r}(I^{r}M\cap N)\\
 &\subset IN\subset N.
  \end{align}
\end{subequations}
Hence the result.
\end{proof}

\begin{corollary}[Matsumara 11.1--3]
Let $A$ be Noetherian.
\begin{enumerate}
\item If $M$ is a finite module over $A$, and $I\subset\JacobsonRadical(A)$,
then $\bigcap_{n\geq0}I^{n}M=0$.
\item (Krull) If $I\subset\JacobsonRadical(A)$, then
  $\bigcap_{n\geq0}I^{n}=0$.
\item (Krull) If $A$ is moreover an integral domain and $I\properideal A$ is a proper ideal,
  then $\bigcap_{n\geq0}I^{n}=0$.
\end{enumerate}
\end{corollary}

\begin{proof}
\begin{enumerate}
\item Krull's intersection theorem + NAK Lemma~\ref{lemma:NAK}.
\item Obvious.
\item Let $N=\bigcap_{n\geq0}I^{n}$, so $IN=N$. By NAK Lemma~\ref{lemma:NAK},
  $\exists x\in I\ldotp (1+x)N=0$. But $1+x\neq0$. Hence $N=0$.\qedhere
\end{enumerate}
\end{proof}

\begin{example}[$p$-adic valuation in $\ZZ$]
Let $p\in\ZZ$ be prime, let $a,b\in\ZZ$ be distinct $a\neq b$. Then
there exists some $n>0$ such that $a\not\equiv b\bmod{p^{n}}$.

\begin{proof}
Proof by contrapositive: If $a\equiv b\bmod{p^{n}}$, then
$a-b\in\bigcap_{n\geq0}(p)^{n}=(0)$, which implies $a=b$.
\end{proof}
\end{example}

\subsection{Dimension}

\begin{definition}[Matsumara 12A]
Let $A\neq0$ be a ring. Its \define{Krull Dimension} $\dim(A)$ is
defined to be either:
\begin{enumerate}
\item the
length of the longest chain of prime ideals $P_{0}\propersubset \dots\propersubset P_{n}$;
in this case $\dim(A)=n$; or
\item if no longest chain exists, $\dim(A)=\infty$.
\end{enumerate}
\end{definition}

\begin{example}
The $\dim(\ZZ)=1$ because the prime ideals of $\ZZ$ look like $(p)$ or $(0)$,
so the longest chain would be $(0)\propersubset(p)$.
\end{example}

\begin{example}
If $A$ is a principal ideal domain, $\dim(A)\leq1$.
\end{example}

\begin{example}
If $A$ is a field, $\dim(A)=0$.
\end{example}

\begin{example}
Let $\kk$ be a field. Then $\dim(\kk[x_{1},\dots,x_{n}])=n$.
\end{example}

\begin{remark}
In general, if $A$ is Noetherian, then $\dim(A[x])=1+\dim(A)$.
\end{remark}

\begin{definition}[Matsumara 12A]
Let $A$ be a ring (possibly Noetherian, possibly not). Let $P\in\Spec(A)$
be a prime ideal. We define the \define{Height} of $P$ to be the
number $\height(P)$ defined by cases:
\begin{enumerate}
\item If all chains of prime subideals
  $P_{0}\propersubset\dots\propersubset P_{n}=P$ are finite length,
  then $\height(P)$ is the length of the longest such chain;
\item Otherwise, $\height(P)=\infty$.
\end{enumerate}
\end{definition}

\begin{example}
In $\CC[x,y]$ we see $\height(y-x^{2})=1$ and $\height(x,y-1)=2$.
\end{example}

\begin{node}[Matsumara 12A]
For any commutative ring $A$, $\dim(A)=\sup\{\height(P)\mid P\in\Spec(A)\}$.
\end{node}

\begin{node}[Matsumara 12A]
For any commutative ring $A$, $\height(P)=\dim(A_{P})$ since
$\Spec(A_{P})$ consists of all prime ideals below $P$.
\end{node}

\begin{definition}
Let $A$ be a commutative ring, let $I\properideal A$ be a proper ideal.
We define the \define{Height} of $I$ to be
\begin{equation}
\height(I) := \min\{\height(P)\mid P\in\Spec(A), I\subset P\}.
\end{equation}
\end{definition}

\begin{fact}
We have $A$ is Artinian iff $A$ is Noetherian and $\dim(A)=0$.
\end{fact}

\begin{definition}[Matsumara 12C]
Let $(A,\mathfrak{m})$ be a local Noetherian ring.
We define an \define{Ideal of Definition} of $A$ to be an ideal
$I\properideal A$ such that $\Radical{I}=\mathfrak{m}$. This means
there is some $\nu>0$ such that $\mathfrak{m}^{\nu}\subset\mathfrak{q}\subset\mathfrak{m}$.

(See Stacks project \href{https://stacks.math.columbia.edu/tag/07DU}{\texttt{[07DU]}}, for example.)
\end{definition}

\begin{remark}
It seems to me that the phrase ``ideal of definition'' is rather
obscure. I've only seen Matsumara, the Stacks project, and Springer's
\textit{Encyclopedia of Mathematics} use it (and the
\textit{Encyclopedia} uses it only once). The earliest use of the
phrase appears to be D.G.\ Nothcott, \textit{Lessons on rings, modules, and multiplicities} (\S4.9, Cambridge Univ.\ Press, 1968).
\end{remark}

\begin{proposition}[Matsumara 12C]
Let $(A,\mathfrak{m})$ be a Noetherian local ring with $\mathfrak{q}$
a $\mathfrak{m}$-primary ideal such that for some $\nu>0$ we have $\mathfrak{m}^{\nu}\subset\mathfrak{q}\subset\mathfrak{m}$.
Then $A/\mathfrak{q}$ is Artinian.
\end{proposition}

\begin{proof}
If $P\in\Spec(A/\mathfrak{q})$, then
$\mathfrak{m}^{\nu}\subset\mathfrak{q}\subset P$.
Since $P$ is prime, we have $\mathfrak{m}\subset P$, which implies
$\Spec(A/\mathfrak{q})=\{\mathfrak{m}\}$.
This means $\dim(A/\mathfrak{q})=0$. Hence the result.
\end{proof}

\begin{proposition}
Let $M$ be a finite module over $A$,
$A^{*}=\gr^{\mathfrak{q}}A=A/\mathfrak{q}\oplus\mathfrak{q}/\mathfrak{q}^{2}\oplus\dots$,
$M^{*}=\gr^{\mathfrak{q}}M=\bigoplus_{n\geq0}\mathfrak{q}^{n}M/\mathfrak{q}^{n+1}M$.
If $\mathfrak{q}$ is finitely-generated by $x_{1}$, \dots, $_{r}$,
then $A^{*}$ is a homomorphic image of $B=(A/\mathfrak{q})[x_{1},\dots,x_{r}]$.
So $M$ is a finite $A^{*}$-module and a finite $B$-module.
\end{proposition}

\begin{definition}[Matsumara 12C]
Let $A$ be a ring. Let $\mathfrak{q}$ be an ideal of $A$ satisfying
the above. Let $M$ be a module over $A$.

We define the \define{Characteristic Function} of $M$ with respect to
$\mathfrak{q}$ to be the function
$\chi(M,\mathfrak{q};n):=\length_{A/\mathfrak{q}}(M^{*}/M^{*}_{\geq n})$
Observe that $M^{*}/M^{*}_{\geq n}\iso\bigoplus_{i<n}M^{*}_{i}$.
We see that
\begin{subequations}
  \begin{align}
\chi(M,\mathfrak{q};n)&=\length_{A/\mathfrak{q}}(M^{*}/M^{*}_{\geq n})\\
&=\length_{A/\mathfrak{q}}(M^{*}_{0})+\dots+\length_{A/\mathfrak{q}}(M^{*}_{n-1})\\
&=\length_{A}(M/\mathfrak{q}^{n}A),
  \end{align}
\end{subequations}
since length is additive.
\end{definition}

\begin{definition}[Matsumara 12C]
Let $(A,\mathfrak{m})$ be local Noetherian ring, let $\mathfrak{q}$ be
an ideal of definition of $A$. Let $M$ be a finite module over $A$.
We can define the \define{Hilbert Polynomial} of $M$ with respect to
$\mathfrak{q}$ to be the unique polynomial $g(n)$ such that there
exists some $n_{0}\in\NN$ such that for all $n>n_{0}$ we have $g(n)=\chi(M,\mathfrak{q};n)$. 
\end{definition}

\begin{proposition}
$\deg(\chi(M,\mathfrak{q};n))=\deg(\chi(M,\mathfrak{m};n))$
\end{proposition}

\begin{notation}
We define $d(M):=\deg(\chi(M,\mathfrak{m};n))$ for this degree.
\end{notation}

\begin{theorem}[Dimension theorem]
Let $(A,\mathfrak{m})$ be a Noetherian local ring. The following are
equal
\begin{enumerate}
\item $\dim(A)$
\item $d(A)=\deg\chi(A,\mathfrak{m};n)$
\item $\delta(A)=$ the least number of generators for an
  $\mathfrak{m}$-primary proper ideal of $A$
\end{enumerate}
\end{theorem}

\begin{corollary}
Let $A$ be Noetherian, $I=(x_{1},\dots,x_{r})\ideal A$ be an ideal.
Every minimal prime over-ideal $P\supset I$ has $\height(P)\leq r$ and
therefore $\height(I)\leq r$.
\end{corollary}

\begin{corollary}
$\dim(A)\leq\dim_{A/\mathfrak{m}}(\mathfrak{m}/\mathfrak{m}^{2})$
\end{corollary}