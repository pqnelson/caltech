%%
%% fall-lecture15.tex
%% 
%% Made by Alex Nelson <pqnelson@gmail.com>
%% Login   <alex@lisp>
%% 
%% Started on  2025-11-01T09:50:02-0700
%% Last update 2025-11-01T09:50:02-0700
%% 

\lecture{}

\begin{example}
Returning to an example from last time: let $\kk[x]$ be a ring of
polynomials, let $x_{1}=x(x-1)$ and $x_{2}=x^{2}(x-1)$ be polynomials
in this ring. We had a map
\begin{equation}
\begin{split}
\kk[x_{1},x_{2}]&\to\kk[x]\\
x_{1}&\mapsto x(x-1)\\
x_{2}&\mapsto x^{2}(x-1),
\end{split}
\end{equation}
which gave us a map
\begin{equation}
b\colon \Spec(\kk[x])\to\Spec(\kk[x_{1},x_{2}]/(x_{1}^{3}-x_{2}^{2}-x_{1}x_{2})),
\end{equation}
and we also had the natural morphism $\kk[x_{1},x_{2}]\onto \kk[x_{1},x_{2}]/(x_{1}^{3}-x_{2}^{2}-x_{1}x_{2})$,
which induces the mapping:
\begin{equation}
\pi_{*}\colon\Spec(\kk[x_{1},x_{2}]/(x_{1}^{3}-x_{2}^{2}-x_{1}x_{2}))\to\Spec(\kk[x_{1},x_{2}]).
\end{equation}
We see that $b$ is a birational map.

(As an aside, this allows us to do things in number theory like: try
to find integer solutions to $x^{2}+y^{2}=z^{2}$. The first step is to
divide through by $z$ to obtain $(x/z)^{2} + (y/z)^{2}=1$. But this is
just the equation for a unit circle, with rational points since $x/z\in\QQ$
and $y/z\in\QQ$ when $z\neq0$. We can then use stereographic
projection to find such solutions.)
\end{example}

\begin{theorem}
\begin{enumerate}
\item If $A\to B$ is an integral extension of rings, then Going-Up holds.
\item Let $A$ and $B$ be integral domains. If $A$ is integrally closed
  and $A\to B$ is an integral extension, then Going-Down holds.
\end{enumerate}
\end{theorem}

\begin{xca}[Hartshorne]
Show that $\CC[x,y]/(x^{2}-y^{3})$ is not integrally closed, and
$\CC[x,y,z](x^{2}+y^{3}+z^{5})$ is integrally closed.
\end{xca}

\begin{theorem}[Matsumara 5E]
Let $A\subset B$ be rings with $B$ integral over $A$.
Suppose $A$ is integrally closed.
\begin{enumerate}[start=6]
\item If $B$ is an integral closure of $A$ in a normal field extension
  $L$ of $K=\Frac(A)$ [the field of fractions of $A$), then any two
  prime ideals $Q$ and $Q'$ of $B$ lying over the same prime ideal
  $P\in\Spec(A)$ are conjugate to each other by an automorphism of $L$
  over $K$.
\end{enumerate}
\end{theorem}

Remember: A Galois extension = splitting extension + normal.

\begin{proof}
Let $G=\Aut(K)$ be the automorphism group of $K$.
Assume that $L$ is a finite extension of $K$.
Then $G$ is finite.
Write $G=\{\sigma_{1},\dots,\sigma_{n}\}$.
Let $Q$, $Q'$ be primes lying over $P$.
Write $Q_{i}:=\sigma_{i}(Q)$.
We want to prove that $Q'=Q_{i}$ for some $i$.
Since $\sigma_{i}(B)\subset B$, $Q_{i}\in\Spec(B)$.
If $Q'\neq Q_{i}$ for all $i=1,\dots,n$, then $Q'\nsubset Q_{i}$.
Then there exists $x\in Q'$ such that $x\in Q_{i}$ for all $i=1,\dots,n$.
Write
\begin{equation}
y = \left(\prod^{n}_{i=1}\sigma_{i}(x)\right)^{q}
\end{equation}
where $q=1$ if $\Char(K)=0$, and $q=p^{N}$ if $\Char(K)=p$ and $N$ is
large enough.
Then $y\in K$, and (since $A$ is integrally closed) we have $y\in B$.
Then $y\in A$. But $y\notin P$ (because $x\notin\sigma_{i}^{-1}(P)$,
which implies $\sigma_{i}(x)\notin P$), while $y\in P'\cap A=P\cap A$
which is a contradiction.
\end{proof}

\begin{xca}
Why do we need a normal field extension?? Why would a separable field
extension fail?
\end{xca}

\subsection{Constructible Sets}

A digression through topology\dots

\begin{definition}[Matsumara 6A]
A topological space $X$ is said to be \define{Noetherian} if the
descending chain condition holds for closed subsets.
\end{definition}

\begin{example}
The real line $\RR$ with the standard topology is \emph{not} Noetherian.
\end{example}

\begin{example}
Let $A$ be a Noetherian ring. Then $\Spec(A)$ is a Noetherian
topological space.
\end{example}

\begin{fact}
  If a space $X$ is covered by finitely many Noetherian subspaces,
  then $X$ is Noetherian.
\end{fact}

\begin{fact}
  Any subspace of a Noetherian space is Noetherian.
\end{fact}

\begin{definition}
We say a space $X$ is \define{Quasicompact} if every open cover has a
finite subcover.

If further $X$ is Hausdorff, we say $X$ is \define{Compact}. That is
to say,
\begin{equation}
\mbox{Compact} = \mbox{Quasicompact} + \mbox{Hausdorff}.
\end{equation}
\end{definition}

\begin{fact}
Noetherian implies quasicompact: If $X$ is a Noetherian space, then
$X$ is quasicompact.
\end{fact}

\begin{node}
We study Noetherian spaces to discuss irreducible components.
\end{node}

\begin{definition}
A closed set $Z$ of a topological space $X$ is \define{Irreducible} if
we cannot write $Z=Z_{1}\cup Z_{2}$ where $Z_{1}$ and $Z_{2}$ are
proper closed subsets of $X$ (which may or may not be disjoint).
\end{definition}

\begin{remark}
``Irreducible'' corresponds to prime ideals.
\end{remark}

\begin{fact}
In a Noetherian space, every closed set $Z$ has a unique decomposition
into irreducible sets $Z=Z_{1}\cup\cdots\cup Z_{n}$ such that
$Z_{i}\nsubset Z_{j}$ when $i\neq j$. We call these $Z_{i}$
\define{Irreducible Components}.
\end{fact}

Take a ring, take an ideal, then look at their minimal polynomial(?).
