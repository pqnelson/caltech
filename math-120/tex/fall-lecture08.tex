%%
%% fall-lecture08.tex
%% 
%% Made by Alex Nelson <pqnelson@gmail.com>
%% Login   <alex@lisp>
%% 
%% Started on  2025-10-16T07:06:20-0700
%% Last update 2025-10-16T07:06:20-0700
%% 

\lecture{}

\begin{corollary}
A ring $R\neq0$ is Artinian if and only if $R$ is Noetherian and $R$
is of dimension zero.
\end{corollary}

\begin{definition}
We say a ring $A$ \define{has Dimension Zero} (or ``is of dimension zero'')
if all its prime ideals are maximal ideals.
\end{definition}

\begin{remark}
There is a more general notion of a ``dimension of a ring'' (the Krull
dimension). The preceding definition is logically equivalent to when
the Krull dimension is zero.
\end{remark}

\begin{proof}[Proof (of Corollary)]
\forwardproof{} Assume $R$ is Artinian. Let $P$ be a prime ideal of $R$.
We want to prove $P$ is among the finitely-many maximal ideals
$P_{1}$, \dots, $P_{r}$ of $R$. We form from the proof of the previous
proposition from last lecture
$I^{s}=(P_{1}(\cdots)P_{r})^{s}=(0)\subset P$, so therefore there
exists an $i$ such that $P_{i}\subset P$ which implies $P_{i}=P$.

\backwardproof{} Assume $R$ is Noetherian and zero-dimensional. Then
we can take its primary decomposition
\begin{equation}
(0) = (q_{1})\cap\cdots\cap(q_{r})
\end{equation}
where the $(q_{i})$ are primary ideals of $R$. Take $P_{i}=\Radical{(q_{i})}$
for each $i=1,\dots,r$. Then each $P_{i}$ is finitely generated. Then
there exists $n\in\NN$ such that $P_{i}^{n}\subset Q_{i}=(q_{i})$
since $P_{i}$ are finitely-generated. Then take
\begin{equation}
\bigl(P_{1}(\cdots)P_{r}\bigr)^{n}\subset Q_{1}(\cdots)Q_{r}\subset Q_{1}\cap\cdots\cap Q_{r}=(0).
\end{equation}
Recall from last time, when we get to this point, we got
$\length_{R}(R)<\infty$ is finite, which implies $R$ is Artinian.
\end{proof}

\begin{remark}
\begin{enumerate}
\item A ring is Noetherian iff every prime ideal is finitely-generated.
\item ``Recently'' (1968) it has been shown if $R\subset R'$ and $R'$
  is a finitely-generated module over $R$, if $R'$ is Noetherian then
  $R$ is Noetherian.
\end{enumerate}
\end{remark}

\subsection{Flat Modules}

\begin{definition}[Tensoring a sequence by a module]
Let $A$ be a ring. Let $M$ be a module over $A$.
When we have any sequence of the form
\begin{equation}
S\colon\quad\cdots\to N\to N'\to N''\to\cdots
\end{equation}
we denote $S\otimes M$ the sequence
\begin{equation}
S\otimes M\colon\quad \cdots\to N\otimes M\to N'\otimes M\to
N''\otimes M\to\cdots.
\end{equation}
\end{definition}

\begin{definition}[Flat and faithfully flat modules]
Let $A$ be a ring. Let $M$ be a module over $A$.
We call $M$ \define{Flat} if $S\otimes M$ is an exact sequence
whenever $S$ is an exact sequence.

We call $M$ \define{Faithfully Flat} (or ``f.f.'') if $S\otimes M$ is exact
iff $S$ is exact.
\end{definition}

\begin{remark}
We have a hierarchy of attributes for modules. Let us review them here:
\begin{enumerate}
\item Free modules (nicest) are of the form $A^{n}$ or
  $\bigoplus_{i\in I}A$
\item Projective modules $P$ are defined by the property
  \begin{equation}
\xymatrix{P\ar@{-->}[r]\ar[dr] & M\ar@{->>}[d]\\
& N}
  \end{equation}
  which lifts along surjective maps. There are many ways to
  characterize projective modules.

  We see for any sequence
  \begin{equation}
S\colon\quad \cdots\to N\to N'\to N''\to\cdots
  \end{equation}
  thaat the sequence
  \begin{equation}
\cdots\to\hom(P,N)\to\hom(P,N')\to\hom(P,N'')\to\cdots
  \end{equation}
  Recall $\hom(M,-)\colon\Mod[A]\to\Ab$ is a functor and it is
  \textbf{\emph{always}} left exact (for every module $M$), meaning if
  $0\to N\to N'\to N''\to\cdots$ is exact, then
  \begin{equation}
0\to\hom(M,N)\to\hom(M,N')\to\hom(M,N'')\to\cdots
  \end{equation}
  is exact. So $P$ is projective if $\hom(P,-)$ is exact (on left and right).
\item Flat modules $M$ means $M\otimes-$ is an exact functor.
\end{enumerate}
\medbreak
We have in summary
\begin{equation}
\mbox{Free}\subset\mbox{Projective}\subset\mbox{Flat}.
\end{equation}
\end{remark}

\begin{proposition}
Let $A$ be a ring, let $P$ be a module over $A$.
The module $P$ is projective iff there exists a module $Q$ over $A$
such that $P\oplus Q$ is a free module.
\end{proposition}

\begin{question}
What is an example of a projective module which is not free?
\end{question}

\begin{proof}[Answer]
Take $R=\ZZ/6\ZZ$. We see that $M=\ZZ/2\ZZ$ is a module over $R$.
We see that $(\ZZ/2\ZZ)\oplus(\ZZ/3\ZZ)=\ZZ/6\ZZ$. But $M$ is not free.

More generally, if $A=B\times C$ as rings, then $B$ is a module over
$A$ and $B$ is flat over $A$ (but $B$ is not faithfully flat).
\end{proof}

\begin{node}
If we have a functor $\hom(M,-)$ for $M$ being a flat module, or
$M\otimes-\colon\Mod[A]\to\Ab$ but it is not left exact in general. We
can write a short exact sequence
\begin{equation}
0\to N\to N'\to N''\to 0
\end{equation}
the wish is to find a module $M$ such that
\begin{equation}
0\to M\otimes N\to M\otimes N'\to M\otimes N''\to0
\end{equation}
to preserve the injective map.

Recall the projective resolution of $N$,
\begin{equation}
\cdots\to P_{2}\to P_{1}\to P_{0}\to N,
\end{equation}
the idea is to apply the functor $M\otimes-$ to the projective
resolution
\begin{equation}
\cdots M\otimes P_{2}\xrightarrow{f_{2}} M\otimes P_{1}\xrightarrow{f_{1}} M\otimes P_{0}\xrightarrow{f_{0}} M\otimes N.
\end{equation}
Then we can look at the homology of $C=\mbox{(projective resolution)}\otimes M$
which measures th failure of exactness. We call the $i^{\text{th}}$
homology group $H^{i}(C)=\Tor^{i}(N,M)$.

The point: even if $M$ is not flat, there is a way to obtain flat
stuff by tensoring with the projective resolution.
\end{node}

\begin{xca}
Compute $\Tor(\ZZ/2\ZZ,\ZZ)$.
\end{xca}