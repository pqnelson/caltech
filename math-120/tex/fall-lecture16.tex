%%
%% fall-lecture16.tex
%% 
%% Made by Alex Nelson <pqnelson@gmail.com>
%% Login   <alex@lisp>
%% 
%% Started on  2025-11-04T08:57:54-0800
%% Last update 2025-11-04T08:57:54-0800
%% 

\lecture{}

\begin{definition}[Matsumara 6B]
\begin{enumerate}
\item Let $X$ be a topological space and $Z\subset X$. We say that $Z$ is
\define{Locally Closed} if $Z=U\cap F$ for some $U\subset X$ open and
$F\subset X$ closed.
\item Let $X$ be a Noetherian topological space. We say that a set
  $Z\subset X$ is \define{Constructible} if $Z$ is a finite union of
  locally closed sets in $X$, i.e., $Z=\bigcup^{n}_{i=1}(U_{i}\cap F_{i})$
  for some open $U_{i}\subset X$ and closed $F_{i}\subset X$ for $i=1,\dots,n$
\end{enumerate}
\end{definition}

\begin{remark}
\begin{enumerate}
\item We can define ``constructible'' for generic topological spaces,
  not just Noetherian spaces, but it is rather unwieldy in general.
\item If $Z$ and $Z'$ are constructible, then so are $Z\cup Z'$ and
  $Z\cap Z'$ and $Z\setminus Z'$ are all constructible.
\end{enumerate}
\end{remark}

\begin{proposition}[Matsumara 6C]\label{prop:fall-lec16:prop-6c}
If $X$ is Noetherian and $Z\subset X$, then $Z$ is constructible if
and only if for each irreducible closed subset $X_{0}\subset X$,
either $X_{0}\cap Z$ is not dense in $X_{0}$, or $X_{0}\cap Z$
contains a nonempty open subset of $X_{0}$.
\end{proposition}

\begin{example}
The canonical example: consider a map $\AA^{2}\to\AA^{2}$ sending
$(x,y)\mapsto(x,xy)$. What is the image of this map?

We see that $(0,y)\mapsto(0,0)$ for all $y$, but nothing is mapped to
$(0,y)$. For all other $(x,y)$ with $y\neq 0$, we can find a point
$(x,y/x)$ which is mapped to it. Thus the image is
$\AA^{2}\setminus\{(0,y)\mid y\neq0\}$.

Is this locally closed? [No]

Is this a constructible set? Yes. Intuitively, we could take
$U_{+}=\{(x,y)\mid x>0\}$ and $U_{-}=\{(x,y)\mid x<0\}$ with $F_{\pm}=\AA^{2}$, and $U_{0}=\AA^{2}$
with $F_{0}=\{(0,0)\}$. This gives us our family of locally closed
sets whose union is the image of the map.
(Or, we could see, the complement of the line $x=0$ is an open set,
and use that instead of $U_{0}$.)
\end{example}

\begin{proof}[Proof (6C)]
\forwardproof\
Assume $Z$ is constructible. Let $X_{0}$ be an irreducible closed
subset of $X$. THen $X_{0}\cap Z$ is constructible. So we can consider
\begin{equation}
X_{0}\cap Z=\bigcup^{m}_{i=1}(U_{i}\cap F_{i}).
\end{equation}
Without loss of generality: (1) since $F_{i}$ are closed, they are irreducible;
(2)~$\forall i,U_{i}\cap F_{i}\neq\emptyset$ otherwise we could just
discard any disjoint pair $U_{j}\cap F_{j}=\emptyset$ from consideration.

Now, consider $\closure{U_{i}\cap F_{i}}=F_{i}$ since $F_{i}$
irreducible (and $F_{i}\setminus U_{i}$ would give the irreducible
decomposition). Then $\closure{X_{0}\cap Z}=\bigcup^{m}_{i=1}F_{i}$.

If $X_{0}\cap Z$ is dense in $X_{0}$, then
\begin{equation}
X_{0}=\bigcup^{m}_{i=1}F_{i},
\end{equation}
which implies $X_{0}=F_{1}$ (where we just reorder the indices to make
$F_{1}$ the particular closed set). Then $U_{1}\cap X_{0}=U_{1}\cap F_{1}$
is a nonempty open subset of $X_{0}$ contained in $X_{0}\cap Z$. Hence
the result.

\backwardproof\ A bit harder, use induction and the fact the space is Noetherian.
\end{proof}


\begin{example}
The integers are not a constructible subset of $\RR$ in the Zariski topology.
But $\ZZ$ is dense in the Zariski topology (since the closed subsets
of $\RR$ are either finite or $\RR$, and the closure of $\ZZ$ cannot
be finite, so the closure of $\ZZ$ must be $\RR$ in the Zariski
topology --- hence $\ZZ$ is dense in $\RR$ in the Zariski topology).
\end{example}

\begin{definition}
Let $A$ be a ring, let $F$ be an irreducible closed subset of $X=\Spec(A)$.
Then we define the \define{Generic Point} of $F$ to be the unique
prime ideal $P\in\Spec(A)$ such that $F=V(P)$ (where
$V(I)=\{P\in\Spec(A)\mid I\subset P\}$).
\end{definition}

\begin{remark}
Remember, in Hausdorff spaces, points are closed subsets. But in
non-Hausdorff spaces, ``points'' can be\dots well, big.
\end{remark}

\begin{lemma}[Generic point is well-defined, Matsumara 6D]
Let $A$ be a ring, let $F$ be a closed subset of $X=\Spec(A)$.
Then $F$ is irreducible if and only if $F=V(P)$ for some prime ideal
$P$. This $P$ is unique and it's precisely the generic point of $F$.
\end{lemma}

\begin{proof}
\forwardproof\ Suppose $F$ is irreducible.
Since $F$ is closed in the Zariski topology, there is some $I\ideal A$
such that $F=V(I)$ where $I=\bigcup_{P\in F}P$. We want to show that
$I$ is prime. If $I$ is not prime, then there exists some $ab\in A\setminus I$
such that $ab\in I$. Look at $V(a)$ (the ideals containing $a$) and $V(b)$.
Then $F\nsubset V(a)$ and $F\nsubset V(b)$. But $F\subset V(a)\cap V(b)=V(ab)$.
Then $F=F(F\cap V(a))\cup(F\cap V(b))$ which contradicts irreducibility.

\backwardproof\ If $V(P)=F$, then $P\in V(P)$. Suppose $F=Z_{1}\cup Z_{2}$.
Then either $P\in Z_{1}$ or $P\in Z_{2}$. But its closure is the
entire thing $\closure{P}=V(P)=F$ which implies $Z_{1}=\emptyset$ or
$Z_{2}=\emptyset$ which implies $F$ is irreducible.
\end{proof}

% This was actually a "sub-lemma" used in the proof of Matsumara 6.2
\begin{lemma}\label{lm:fall-lec16:sub-lemma-for-6-2}
If $Z\subset\Spec(A)=X$ is any sset in the Zariski topology, then
\begin{equation}
\closure{Z}=V\left(\bigcap_{P\in Z}P\right)=\{Q\in\Spec(A)\mid Q\supset\bigcap_{P\in Z}P\}.
\end{equation}
\end{lemma}

\begin{proof}
Observe:
\begin{enumerate}
\item $V(\bigcap P)$ is a closed set
\item $Z\subset V(\bigcap P)$
\item For any closed $Y$ such that $Z\subset Y$, then $Y=V(I)$ implies
  $\bigcap P\subset I$ which implies $V(\bigcap P)\subset V(I)$. Then
  $P\in V(I)$ implies $I\subset P$, which implies $V(\bigcap P)\subset V(I)$. 
\end{enumerate}
Hence the result.
\end{proof}

\begin{lemma}[Matsumara 6.2]\label{lm:fall-lec16:matsumara-6-2}
Let $\phi\colon A\to B$ be a map of rings. Let $X=\Spec(A)$ and $Y=\Spec(B)$.
Let $f=\preimage\varphi\colon Y\to X$. Then $f(Y)$ is dense in $X$ if
and only if $\ker(\varphi)\subset\nilradical{A}$. In particular, if
$A$ is reduced, then $f(Y)$ is dense iff $\phi$ is injective.
\end{lemma}

% The proof of the lemma was the start of lecture 17

\begin{proof}
\forwardproof\
To prove something is dense, we look at its (topological) closure. So
we look at $\closure{f(Y)}$ in $X$. Now, we apply Lemma~\ref{lm:fall-lec16:sub-lemma-for-6-2}
to
\begin{subequations}
\begin{align}
\closure{f(Y)} &= V(\bigcap{P\in Y}\phi^{-1}(P))\\
&= V(\phi^{-1}(\bigcap_{P\in Y}P))\quad\mbox{preimage commutes with interscetion}\\
&= V(\phi^{-1}(\nilradical(B)))\quad\mbox{by def of nilradical},
\end{align}
\end{subequations}
so now let us look at the kernel of $\phi$. Denoting
$I=\phi^{-1}(\nilradical(B))$, then we have $\ker(\phi)\subset I$. Now
suppose $f(Y)$ is dense in $X$ (i.e., $\closure{f(Y)}=X$).
Then $I\subset\nilradical(A)$. We have $\forall P\in\Spec(A)\ldotp I\subset P$.
Then $I\subset\bigcap P=\nilradical(A)$. Hence $\ker(\phi)\subset I\subset \nilradical(A)$.
\end{proof}

\begin{remark}
Note: Matsumara's proof is flawed. We proved one direction, but the
other is easy, and we will skip it.
\end{remark}