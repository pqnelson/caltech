%%
%% winter-lecture03.tex
%% 
%% Made by Alex Nelson <pqnelson@gmail.com>
%% Login   <alex@lisp>
%% 
%% Started on  2026-01-10T09:13:23-0800
%% Last update 2026-01-10T09:13:23-0800
%% 

\lecture{}

\begin{corollary}\label{cor:2.7}
If $L$, $L's$ are algebraic closures of $F$, then there exists an
$F$-isomorphism $L\to L'$.
\end{corollary}

\begin{proof}
By Proposition~\ref{prop:2.6}, there existsan $F$-embedding
$\sigma\colon L\to L'$. Since $L$ is algebraic over $L's$,
$\sigma(L)\subset L'$ and $\sigma(L)$ is algebraically closed, we have
$\sigma(L)=L'$. 
\end{proof}

\begin{node}
If $E/F$ is algebraic, then their algebraic closures coincide $\closure{E}=\closure{F}$.
\end{node}

\begin{proof}
By Proposition~\ref{prop:2.6}, there exists a $\sigma\colon E\to\closure{F}$
which is an $F$-embedding. By Proposition 1.3 we have
$\closure{F}\supset E\supset F$ and $\closure{E}\supset E\supset F$
(which gives $\closure{E}\supset\sigma(E)\supset F$).
\end{proof}

\begin{proposition}\label{prop:2.8}
Let $L/E$ and $E/F$ be algebraic extensions. Then for any
$F$-embedding $\tau\colon E\to\closure{F}$ can be extended to
$\tau'\colon L\to\closure{F}$. That is, the restriction
$\hom_{F}(L,\closure{F})\to\hom_{F}(E,\closure{F})$ sending $f\mapsto f|_{E}$
is surjective.
\end{proposition}

\begin{proof}
First consider $L=\closure{F}$ (i.e., $\closure{F}\supset E\supset F$).
We write
\begin{equation}
S=\{(K,\sigma)\mid\closure{F}/K,K/E,\sigma\colon K\to\closure{F}\mbox{ is $F$-embedding}, \sigma|_{\closure{E}}=\tau\}
\end{equation}
and $(K,\sigma)\leq(K',\sigma')$ iff $K'/K$ and
$\sigma'|_{K}=\sigma$. This gives us a poset, then by Zorn's lemma
there is a maximal element --- there exists an $F$-embedding
$\tau's\colon\closure{F}\to\closure{F}$ such that $\tau'|_{E}=\tau$,
so the maximal element of $S$ is $(\closure{F},\tau')$. Moreover,
$\tau'$ is surjective (hence $\tau\in\Aut_{F}(\closure{F})$).
In fact, for any $\alpha\in\closure{F}$ and its minimal polynomial
\begin{equation}
p_{\alpha}(x)=\prod_{i}(x-\alpha_{i})\in\closure{F}[x],
\end{equation}
and
\begin{equation}
\tau'(p_{\alpha}(x))=\prod_{i}(x-\tau'(\alpha_{i}))\in\Im(\tau)[x]
\end{equation}
is a polynomial of $\alpha=\tau'(\alpha_{i})$ for some
$\alpha_{i}$. So $\alpha\in\Im(\tau')$, hence $\tau'$ is surjective.

Now, second, we can prove the general case $L/E$ and $E/F$ (N.B.:
$\closure{L}=\closure{E}=\closure{F}$), composition of restriction of
maps
\begin{equation}
\vcenter{\vbox{\xymatrix{\Aut_{F}(\closure{F})\ar[r]\ar@/_1.5pc/@{>>}[rr] & \hom_{F}(L,\closure{F})\ar@{>>}[r] & \hom_{F}(E,\closure{F})}
\vskip1pc}}
\end{equation}
where we proved $\hom_{F}(L,\closure{F})\to\hom_{F}(E,\closure{F})$ is
surjective in the first part of the proof. Hence the result.
\end{proof}

\subsection{Splitting fields}

\begin{definition}
Let $K/F$ be a field extension, let $f(x)\in F[x]$ be a monic polynomial.
We say $f(x)$ \define{Splits} in $K$ if we can write
\begin{equation}
f(x)=\prod_{i}(x-\alpha_{i}),
\end{equation}
where $\alpha_{i}\in K$ for each $i$.

We also say that $K$ is the \define{Splitting Field} of $f(x)$ if
$K=F(\alpha_{1},\dots,\alpha_{n})$ for all $n$ roots $\alpha_{1}$,
\dots, $\alpha_{n}$ of $f$.
\end{definition}

\begin{note}
For every polynomial $f(x)\in F[x]$, there exists a splitting field
for $f(x)$ since there are roots of $f(x)$ in $\closure{F}$.
\end{note}

\begin{node}
For a family $P$ of polynomials $P\subset F[x]$, the splitting field
of $P$ is defined similarly.
\end{node}

\begin{proposition}\label{prop:3.1}
For splitting fields $E$, $K$ of $f\in F[x]$ such that $K\subset\closure{F}$,
then every $F$-embedding $E\to\closure{F}$ induces an isomorphism
$E\iso K$.
\end{proposition}

\begin{proof}
By Proposition~\ref{prop:2.6} there exists an embedding $\sigma\colon E\to\closure{F}$.
We denote $E=F(\alpha_{1},\dots,\alpha_{n})$ for the roots
$\alpha_{i}$ of $f$ in $E$, and $K=F(\beta_{1},\dots,\beta_{n})$ for
the roots $\beta_{j}\in\closure{F}$ of $f$. Then
$\{\sigma(\alpha_{i})\}_{i=1,\dots,n}=\{\beta_{j}\}_{j=1,\dots,n}$
because the left-hand side has distinct roots and by definition of
splitting fields (minimality).
\end{proof}

\begin{corollary}\label{cor:3.2}
For splitting fields $E$ and $K$ of a family of polynomials $P\subset F[x]$
such that $K\subset\closure{F}$, then every $F$-embedding
$\sigma\colon E\to\closure{F}$ induces an isomorphism $E\iso K$.

Also $K=\Im(\sigma)=\sigma(E)$ is the compositum of splitting fields
$K_{i}$ of each $f_{i}\in P$.
\end{corollary}

\begin{proof}
The first half of this proof is the same as the preceding proposition.

Second, since both the compositum and $K=\Im(\sigma)$ are generated by
every root of every polynomial in $P$.
\end{proof}

\begin{theorem}\label{thm:well-definedness:normal-extension}\label{thm:3.3}
Let $\closure{F}/E$ and $E/F$ be algebraic extensions. Then the
following are equivalent:
\begin{enumerate}
\item\label{criteria:normal:2} $E$ is a splitting field for a family of polynomials $P\subset F[x]$;
\item\label{criteria:normal:1} every $F$-embedding $E\to\closure{F}$ is an automorphism of $E$;
\item\label{criteria:normal:3} For any irreducible polynomial $p\in F[x]$ which has a root in
  $E$ must have all of its roots in $E$.
\end{enumerate}
\end{theorem}

\begin{proof}
$\ref{criteria:normal:2}\implies\ref{criteria:normal:1}$ It follows from the above uniqueness of the splitting
  field in $\closure{F}$ (from the second part of the previous corollary).

$\ref{criteria:normal:1}\implies\ref{criteria:normal:3}$ Let $\alpha\in E$ and $p_{\alpha}\in F[x]$ be the
  minimal polynomial of $\alpha$. Then for another root
  $\beta\in\closure{F}$, by composing
\begin{equation}
\begin{array}{ccccl}
F(\alpha) & \iso & F[x]/(p_{\alpha}(x)) & \iso & F(\beta)\into\closure{F}\\
\alpha & \mapsto & x & \mapsto & \beta
\end{array}
\end{equation}
Then by Proposition~\ref{prop:2.8}, it is extended to an $F$-embedding
$E\to\closure{F}$ such that $\alpha\mapsto\beta$ and hence by
hypothesis it's an automorphism of $E$, which means $\beta\in E$.
(Lemma 1.1 proves $F(\alpha)\iso F[x]/(p_{\alpha}(x))$.)

$\ref{criteria:normal:3}\implies\ref{criteria:normal:2}$
Let $P$ be the set of all minimal polynomials for every $\alpha\in E$.
Then $E=F(S)$ by hypothesis --- well, $E\subset F(S)$ obviously, and
$F(S)\subset E$ by hypothesis. Hence the claim.
\end{proof}

\begin{definition}
An algebraic extension $E/F$ satisfying Theorem~\ref{thm:well-definedness:normal-extension}
is called a \define{Normal Extension}.
\end{definition}

\begin{proposition}[Properties of normal extensions]\label{prop:3.4}
\begin{enumerate}
\item Let $L/E$ and $E/F$ be field extensions. If $L/F$ is normal,
  then $L/E$ is normal.
\item Let $L/E_{1}$, $L/E_{2}$, $E_{1}/F$, $E_{2}/F$ be field
  extensions. If $E_{1}/F$ is normal, then $E_{1}E_{2}/E_{2}$ is normal.
\item Let $L/E_{1}$, $L/E_{2}$, $E_{1}/F$, $E_{2}/F$ be field
  extensions. If $E_{1}/F$ and $E_{2}/F$ are normal, then
  $E_{1}E_{2}/F$ is normal.
\end{enumerate}
\end{proposition}

\begin{proof}
$(1)$ Let $\phi\colon L\to\closure{E}$ be an $E$-embedding (we have
$\closure{E}=\closure{F}$), so $\phi\in\hom_{E}(L,\closure{F})$.
Since $\phi|_{F}=\id_{F}$ since $F\subset E$, $\phi$ is also an $F$-embedding.
By criterion~\ref{criteria:normal:1} of normal extensions, $\phi$
induces an automorphism of $L$. Hence $L/F$ is normal.

$(2)\implies(3)$ Let $\phi\colon E_{1}E_{2}\to\closure{E_{2}}$ is an
$E_{2}$-embedding, and so for the same reason $\phi_{E_{1}}\in\hom_{F}(E_{1},\closure{E_{2}})$.
Since $E_{1}/F$ is normal, $\phi$ is an automorphism of $E_{1}$. Since
$\phi$ is an $E_{2}$-morphism, we conclude
\begin{equation*}
\phi(E_{1}E_{2})=\phi(E_{1})\phi(E_{2})=E_{1}E_{2},
\end{equation*}
as desired.

$(3)\implies(1)$ If $\phi\in\hom_{F}(E_{1}E_{2},\closure{F})$,
then $\phi|_{E_{i}}\in\hom_{F}(E_{i},\closure{F})$.
Since $E_{i}/F$ is normal, $\phi(E_{i})=E_{i}$ and thus
$\phi(E_{1}E_{2})=E_{1}E_{2}$.
\end{proof}