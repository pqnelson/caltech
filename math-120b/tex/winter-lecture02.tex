%%
%% winter-lecture02.tex
%% 
%% Made by Alex Nelson <pqnelson@gmail.com>
%% Login   <alex@lisp>
%% 
%% Started on  2026-01-08T10:33:32-0800
%% Last update 2026-01-08T10:33:32-0800
%% 

\lecture{}

\begin{definition}
Let $L/F$ and $E/F$ be field extensions. We define an
\define{$F$-Embedding} to be a field morphism $\sigma\colon E\to L$
such that $\sigma|_{F}=\id_{F}$ (and $\sigma$ is injective, of course).
\end{definition}

\begin{lemma}\label{lemma:math120b:winter2026:lecture2:1}
For an algebraic extension $E/K$, any $K$-embedding $E\to E$ is an
automorphism. 
\end{lemma}

\begin{proof}
Suffices to check surjectivity. For any $\alpha\in E$, let
$p_{\alpha}(x)\in K[x]$ be the minimal polynomial of $\alpha$. We
write the roots of $p_{\alpha}$ in $E$ as the set,
\begin{equation}
S=\{c\in K\mid p_{\alpha}(c)=0\}\cap E,
\end{equation}
and we write
\begin{equation}
E' = K(S).
\end{equation}
Since $E'$ is finitely-generated and algebraic over $K$, by
Proposition 1.2 $E'/K$ is finite. Since $\sigma$ is a $K$-morphism,
$\sigma(S)\subset S$. Therefore $\sigma(E')\subset E'$.
Since $\sigma$ is injective and $\dim_{K}(E')$ is finite, $\sigma$ is
surjective (i.e., $\sigma(E')=E'$). Since this was for arbitrary
$\alpha$, $\sigma$ is surjective on all of $E$.
\end{proof}

\begin{definition}
A field $E$ is \define{Algebraically Closed} if every polynomial
$f(x)\in E[x]$ has a root in $E$.
\end{definition}

\begin{definition}
For any field extension $E/F$, we say $E$ is an \define{Algebraic Closure}
of $F$ if $E/F$ is an algebraic extension and $E$ is algebraically closed.
\end{definition}

\begin{lemma}\label{lemma:math120b:winter2026:lecture2:2}
For a field $F$ and finitely many polynomials $p_{1},\dots,p_{n}\in F[x]$,
there exists an extension $E/F$ such that $p_{i}$ has a root in $E$.
\end{lemma}

\begin{proof}
By induction, it is enough to check for $n=1$.
We may assume that $p_{1}$ is irreducible (otherwise it would be a
product of irreducible factors, and we could just work with one of
those irreducible factors). Then $(p_{1}(x))$ is a maximal ideal in
$F[x]$. Then $F_{1}=F[c]/(p_{1}(c))$ has $c$ as a root of $p_{1}$.
\end{proof}

\begin{theorem}
For any field $F$, there exists an extension $E/F$ such that $E$ is
algebraically closed.
\end{theorem}

\begin{proof}
It suffices to show that there is an extension $F_{1}/F$ such that
every polynomial $f(x)\in F[x]$ has a root in $F_{1}$. In fact,
\begin{equation}
F=F_{0}\subset F_{1}\subset\dots
\end{equation}
(which, in general, is infinite) such that $F_{i+1}/F_{i}$ satisfies
the condition. Then
\begin{equation}
F_{\infty}:=\bigcup^{\infty}_{i=0}F_{i}
\end{equation}
is algebraically closed and it's the field extension we're seeking,
i.e., $E=F_{\infty}$. It's algebraically closed since every polynomial
$f(x)\in F[x]$ there exists only finitely many coefficients of $f(x)$,
so there exists some $N\gg0$ such that $f(x)\in F_{N}[x]$ has a root
in $F_{N+1}\subset E$.

For the set $S$ of all monic irreducible polynomials in $F[x]$ and
ring
\begin{equation}
R:=F[x_{f}]_{f\in S}
\end{equation}
which is the polynomial ring with unknowns indexed by elements of $S$
(which, in general, is infinitely many unknown variables). Let
\begin{equation}
I = \langle\{f(x_{f})\in R\mid f\in S\}\rangle
\end{equation}
be the ideal of $R$ generated by all finite sums of the form
\begin{equation}
\sum_{\substack{f\in S\\\text{finite}}}g_{f}f(x_{f})\quad\mbox{where }g_{f}\in R.
\end{equation}

We claim $I\neq R$.
Otherwise there are finitely many $g_{1},\dots,g_{n}\in R$ such that
\begin{equation}\label{eq:math120b:winter-lecture02:star}
\sum^{n}_{i=1}g_{i}(x_{f_{1}},\dots,x_{f_{n}},x_{n+1},\dots,x_{N})f_{i}(x_{f_{i}})=1
\end{equation}
By Lemma~\ref{lemma:math120b:winter2026:lecture2:2}, there exists
$K/F$ such that each $f_{i}$ has a root $\alpha_{i}\in K$. Then the
$F$-morphism $R\to K$ such that
\begin{equation}
  x_{f}\mapsto\begin{cases}\alpha_{i} & \mbox{if }f=f_{i}\\
  0 & \mbox{otherwise},
  \end{cases}
\end{equation}
we apply this to Equation~\eqref{eq:math120b:winter-lecture02:star} to get
\begin{equation}
\sum g_{i}(\alpha_{1},\dots,\alpha_{n},0,\dots,0)\underbrace{f_{i}(\alpha_{i})}_{=0}=1,
\end{equation}
but the left-hand side's sum is a finite sum of zeroes. That means we
have deduced $0=1$, which is a contradiction.

So $I\neq R$. Then there exists a maximal ideal $\mmm\subset R$
containing $I\subset\mmm$, and $R/\mmm$ is a field containing all the
roots of all the polynomials in $F[x]$.
\end{proof}

\begin{corollary}
For any field $F$, there exists a field extension $E/F$ which is an
algebraic closure of $F$.
\end{corollary}

\begin{proof}
For algebraically closed $F_{\infty}/F$, let $F^{(a)}_{\infty}$ be the
union of all algebraic subextensions of $F_{\infty}/F$ --- that is to
say,
\begin{equation}
F^{(a)}_{\infty}=\bigcup\{F'\subset F_{\infty}\mid F'/F\mbox{ is algebraic}\}.
\end{equation}
If $\alpha\in F_{\infty}$ is algebraic over $F^{(a)}_{\infty}$, then
$\alpha$ is algebraic over $F$. Then $\alpha\in F^{(a)}_{\infty}$ ---
that is to say, $F^{(a)}_{\infty}$ is algebraically closed. (Observe:
$F_{\infty}/F$ is not necessarily an algebraic extension; for example
$F=\QQ$, $F_{\infty}=\CC$ is not an algebraic extension since
transcendental guys like $\pi\in F_{\infty}$.)
\end{proof}

\begin{proposition}
For any extension $E/F$, for any $\alpha\in E$ algebraic over $F$.
Let $p_{\alpha}$ be the minimal polynomial of $\alpha$. Then the
number of $F$-embeddings $F(\alpha)\to E$ is equal to the number of
distinct roots of $p_{\alpha}$ in $E$ and this is bounded above by $[F(\alpha):F]=\deg(p_{\alpha})$.
\end{proposition}

\begin{proof}
An $F$-embedding $\sigma\colon F(\alpha)\to E$ sends $\alpha\mapsto\sigma(\alpha)$
which is also a root of the minimal polynomial $p_{\alpha}(x)$. Hence
the result.
\end{proof}

\begin{proposition}\label{prop:2.6}
Let $L/F$ be an algebraic extension and suppose $L$ is algebraically closed.
Then for any $E/F$ there exists an $F$-embedding $E\to L$.
\end{proposition}

\begin{proof}
We have a poset
\begin{equation*}
\{(K,\sigma)\mid F\subset K\subset E, \sigma\colon K\to L\mbox{ is an $F$-embedding}\},
\end{equation*}
such that $(K,\sigma)\leq(K',\sigma')$ iff $K'/K$ and $\sigma'|_{K}=\sigma$.

Then for any totally ordered subset $\{(K_{i},\sigma_{i})\}_{i\in I}\subset S$
an upper bound $(\bigcup_{i\in I}K_{i},\sigma)$ exists where
$\sigma|_{K_{i}}=\sigma_{i}$ for each $i\in I$. Then Zorn's lemma
asserts the existence of a maximal element $(K,\lambda)\in S$ with
respect to this ordering. If $K\neq E$, then there is an $\alpha\in E\setminus K$
and an $F$-embedding $K(\alpha)\to E$ which leads to a
contradiction. Hence the result.
\end{proof}