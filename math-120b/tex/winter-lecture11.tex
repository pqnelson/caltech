%%
%% winter-lecture11.tex
%% 
%% Made by Alex Nelson <pqnelson@gmail.com>
%% Login   <alex@lisp>
%% 
%% Started on  2026-01-29T08:41:48-0800
%% Last update 2026-01-29T08:41:48-0800
%% 

\lecture[Examples and Applications]{}

\begin{example}\label{ex:8.1}
Consider the polynomial
\begin{equation}
f(x)=x^{4}-2\in\QQ[x].
\end{equation}
We denote by
\begin{equation}
\alpha=\sqrt[4]{2},\quad\mbox{and}\quad\I=\sqrt{2}.
\end{equation}
Then the roots of $f(x)$ are $\pm\alpha$, $\pm\I\alpha$. In
particular, $f(x)$ is irreducible and separable. Thus the splitting
field of $f$ is $\QQ(\alpha,\I)/\QQ$, which is a finite Galois
extension.

What is its Galois group? Denote $G=\Gal(\QQ(\alpha,\I)/\QQ)$. Since
\begin{equation}
[\QQ(\alpha):\QQ]=4,\quad\mbox{and}\quad[\QQ(\I):\QQ]=2,
\end{equation}
we have
\begin{subequations}
  \begin{align}
[\QQ(\alpha,\I):\QQ]
&=[\QQ(\alpha,\I):\QQ(\alpha)][\QQ(\alpha):\QQ]\\
&=[\QQ(\I):\QQ(\alpha)\cap\QQ(\I)][\QQ(\alpha):\QQ]\quad\mbox{by Theorem~\ref{thm:7.6}}\\
&=[\QQ(\I):\QQ][\QQ(\alpha):\QQ]\\
&=2\times4=8
  \end{align}
\end{subequations}
Define $\sigma\in\Gal(\QQ(\alpha,\I)/\QQ(\I))$ and
$\tau\in\Gal(\QQ(\alpha,\I)/\QQ(\alpha))$ such that
\begin{equation}
\sigma(\alpha)=\I\alpha,\quad\mbox{and}\quad\tau(\I)=-\I.
\end{equation}
Then clearly we have $|\langle\sigma\rangle|=4$ and $|\langle\tau\rangle|=2$,
and $\tau\notin\langle\sigma\rangle$ and
\begin{equation}
8=|G|\geq|\langle\sigma,\tau\rangle|=|\langle\sigma\rangle|\cdot[\langle\sigma,\tau\rangle:\langle\tau\rangle]=8,
\end{equation}
so by cardinality reasons $G=\langle\sigma,\tau\rangle$.

We can verify $\sigma\tau\sigma=\tau$ and this is the only relation
for $\langle\sigma,\tau\rangle$. So the underlying set of elements
is:;
\begin{equation}
G=\{1,\sigma,\sigma^{2},\sigma^{3},\tau,\tau\sigma,\tau\sigma^{2},\tau\sigma^{3}\},
\end{equation}
since $\sigma^{3}\tau=\tau\sigma$, $\sigma^{2}\tau=\tau\sigma^{2}$, $\sigma\tau=\tau\sigma^{3}$.
This allows us to classify the subgroups $H$ of $G$, let us tabulate
them by cardinality:
\begin{center}
\begin{tabular}{r|l}
$|H|$ & $H$\\\hline
1 & $\mathbf{1}$ (trivial group)\\
2 & $\langle\tau\rangle$, $\langle\sigma^{2}\rangle$, $\langle\tau\sigma\rangle$, $\langle\tau\sigma^{2}\rangle$, $\langle\tau\sigma^{3}\rangle$\\
4 & $\langle\sigma\rangle$, $\langle\tau,\sigma^{2}\rangle$, $\langle\tau,\tau\sigma^{2}\rangle$\\
8 & $H=G=\langle\sigma,\tau\rangle$
\end{tabular}
\end{center}
The corresponding subextensions $\QQ(\alpha,\I)^{H}$ are:
\begin{center}
\begin{tabular}{r|l}
$|H|$ & $\QQ(\alpha,\I)^{H}$\\\hline
1 & $\QQ(\alpha,\I)$ \\
2 & $\QQ(\alpha)$, $\QQ(\alpha^{2},\I)$, $\QQ((1-\I)\alpha)$, $\QQ(\I\alpha)$, $\QQ((1+\I)\alpha)$\\
4 & $\QQ(\I)$, $\QQ(\alpha^{2})$, $\QQ(\alpha^{2}\I)$\\
8 & $\QQ$
\end{tabular}
\end{center}
\end{example}

\begin{example}[Fundamental theorem of Algebra]\label{ex:8.2}
That is, $\CC$ is algebraically closed. Here $\CC=\RR(\I)$ wherre
$\I=\sqrt{-1}$ is a root of the polynomial $x^{2}+1$.We use the
following properties of $\RR$:
\begin{enumerate}
\item $(\RR,\leq)$ is an ordered field --- recall, ordered fields have
  the properties that, specifically for the reals,
  \begin{enumerate}
  \item $\forall a,b,c\in\RR\ldotp a\geq b\implies a+c\geq b+c$
  \item $\forall a,b\in\RR\ldotp a\geq0\land b\geq0\implies ab\geq0$
  \end{enumerate}
\item\textsc{Existence of squareroot:} $\forall a\in\RR\ldotp a\geq0\implies x^{2}-a$ has a root in
  $\RR$ (namely, $\sqrt{a}\in\RR$)
\item For any polynomial $f(x)\in\RR[x]$, if $\deg(f)$ is odd, then
  $f$ has a root in $\RR$.
\end{enumerate}

For any monic polynomial
\begin{equation}
f(x)=x^{n}+a_{n-1}x^{n-1}+\dots+a_{1}x+a_{0},
\end{equation}
where $n$ is odd, we have
\begin{equation}
\lim_{x\to+\infty}f(x)=+\infty,\quad\mbox{and}\quad
\lim_{x\to-\infty}f(x)=-\infty.
\end{equation}
Then by the intermediate value Theorem, there exists an $x_{0}\in\RR$
such that $f(x_{0})=0$.

Now, we also have two facts needed for the proof:
\begin{enumerate}
\item For any separable finite extension $K/\RR$, if $[K:\RR]$ is odd
  then $K=\RR$;
  \begin{proof}
By Theorem~\ref{thm:4.5}, there exists $\alpha\in K$ such that
$K=\RR(\alpha)$ and thus the minimal polynomial of $p_{\alpha}$ of
$\alpha$ in $\RR$, we have $[K:\RR]=\deg(p_{\alpha})$ is odd. But
since $p_{\alpha}$ has a root in $\RR$, $\deg(p_{\alpha})=1$.
  \end{proof}
\item Every quadratic polynomial $x^{2}+ax+b\in\CC[x]$ has roots in $\CC$
\begin{proof}
We can just use the quadratic formula
\begin{equation}
x_{\pm}=\frac{-a\pm\sqrt{a^{2}-4b}}{2}
\end{equation}
which is an element of $\CC$.
\end{proof}
\end{enumerate}

\textsc{Claim:} $\CC$ is algebraically closed. We will show for any
$P(x)\in\CC[x]$ the splitting field of $P$ is $K=\CC$.

\begin{proof}
Since $\Char(\RR)=0$ (otherwise it contradicts the first property of
ordered fields), then $\RR$ and $\CC$ and $K$ are perfect by
Proposition~\ref{prop:5.5}. Since every algebraic extension of a
perfect field is separable, we can assume that $P$ is irreducible and
separable, and thus $K/\CC/\RR$ and $K/\RR$ are finite Galois extensions.

For $G=\Gal(K/\RR)$, we can write
\begin{equation}
|G|=[K:\RR]=2^{m}q,
\end{equation}
where $m\geq0$ and $q\in\NN_{0}$ is odd. If $m=0$, then $[K:\RR]=q$ is
an odd extension, which means $q=1$ which is a contradiction (it
implies $\CC\subset K=\RR$). So $m>0$.

By Sylow's theorem for $p=2$, there exists a maximal Sylow 2-subgroup
$H\leq G$ such that $|H|=2^{m}$ and $[G:H]=q$. By Theorem~\ref{thm:7.4},
\begin{equation}
[K:K^{H}]=|H|=2^{m},
\end{equation}
and
\begin{equation}
[K^{H}:\RR]=\frac{[K:\RR]}{[K:K^{H}]}=q=1,
\end{equation}
which means
\begin{equation}
[K:\RR]=2^{m}=|G|.
\end{equation}

Next, we study $K/\CC$. For $G'=\Gal(K/\CC)$, we find
\begin{equation}
|G'|=[K:\CC]=2^{m-1}.
\end{equation}
If $m>1$, then there exists a subgroup $H'\subgroup G'$ such that
$|H'|=2^{m-2}$. (Note: if $m>1$, then $m-1\geq1$.) Thus by similar
reasoning as before,
\begin{equation}
[K^{H'}:\CC]=\frac{[K:\CC]}{[K:K^{H'}]}=\frac{2^{m-1}}{|H'|}=2.
\end{equation}
But as we'd check every quadratic polynomial in $\CC[x]$ has all its
roots in $\CC$, this gives us a contradiction. Hence $m=1$, so $[K:\CC]=2^{1-1}=1$.
Therefore $K=\CC$.
\end{proof}
\end{example}

\begin{example}[Fundamental theorem of symmetric polynomials/functions]
Let $\FF$ be a field, let $t_{1}$, \dots, $t_{n}$ be formal variables
(or, at least, algebraically independent). Let $L:=\FF(t_{1},\dots,t_{n})$.
Then the symmetric group $S_{n}$ acts on $L$ by $\sigma(t_{i})=t_{\sigma(i)}$
for $\sigma\in S_{n}$. Clearly this $\sigma$ is an automorphism
$\sigma\in\Aut_{\FF}(L)$. Then $L^{S_{n}}$ is a field of all symmetric
functions over $\FF$. Denote this by $E=L^{S_{n}}$. Denote
\begin{equation}
s_{k}:=\sum_{1\leq i_{1}<\dots<i_{k}\leq n}t_{i_{1}}(\cdots)t_{i_{k}}
\end{equation}
for the elementary $k^{\text{th}}$ symmetric polynomial. Then
$s_{k}\in E$ for $k=0,1,\dots,n$. Writing
$K=\FF(s_{1},\dots,s_{n})\subset E$. We obviously have $K\subset E$.
Then
\begin{subequations}
  \begin{align}
P(x) &:= \prod(x-t_{i})\in E[x]\\
&=\sum^{n}_{i=0}(-1)^{n-i}s_{n-i}x^{i}\in K[x].
  \end{align}
\end{subequations}
We show that $K=E$.

Since $L$ is the splitting field of $P(x)$, then by a homework problem
$[L:K]$ divides $(\deg(P))!=n!$. On the other hand, since $K\subset
E$, we have
\begin{subequations}
  \begin{align}
[L:E] &= [L:L^{S_{n}}]\\
&=|S_{n}|\quad\mbox{by Theorem~\ref{thm:7.4}}\\
&=n!.
  \end{align}
\end{subequations}
But
\begin{equation}
n!=[L:E]\leq[L:K]\leq n!,
\end{equation}
which implies $E=K$. Hence every symmetric rational function may be
written using elementary symmetric polynomials.
\end{example}

\begin{remark}
The polynomial version of this theorem (every symmetric polynomial may
be written as $f\in\FF[s_{1},\dots,s_{n}]$).
\end{remark}

\begin{proof}
Writing $f=\frac{g}{h}$ with $g,h\in\FF[s_{1},\dots,s_{n}]$ when $g$
is irreducible and $h$ is irreducible or a unit. If $h$ is
irreducible, then $g=fh$ which is a contradiction unless $f$ is a
unit. Hence $h$ must be a unit.
\end{proof}