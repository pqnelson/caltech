%%
%% winter-lecture01.tex
%% 
%% Made by Alex Nelson <pqnelson@gmail.com>
%% Login   <alex@lisp>
%% 
%% Started on  2026-01-08T10:42:46-0800
%% Last update 2026-01-08T10:42:46-0800
%% 

\lecture{}

I missed the first lecture, it was mostly review of field extensions.
A good reference I have found is Steven Roman's \textit{Field Theory}
(second ed., Springer, 2006)~\cite{roman2006field}.

\begin{proposition}\label{prop:1.2}
If $L_{1}/L_{2}$ is a simple extension $L_{1}=L_{2}(\alpha)$, then
\begin{equation}
[L_{1}:L_{2}]=\deg(p_{\alpha}),
\end{equation}
where $p_{\alpha}\in L_{1}[x]$ is the
minimal polynomial of $\alpha$.
\end{proposition}

\begin{definition}[{Roman~\cite[\S2.2]{roman2006field}}]
Let $\mathcal{C}$ be a class of field extensions. We say it is
\define{Distinguished} if both of the following conditions hold:
\begin{enumerate}
\item\textsc{Tower Property}: For any tower of extensions $L/E/F$, we
  have $L/F$ is in $\mathcal{C}$ if and only if both $L/E$ and $E/F$
  are in $\mathcal{C}$
\item\textsc{Lifting Property}: For any extension $E/F$ in
  $\mathcal{C}$ and any arbitrary other field extension $K/F$, we have
  the extension $EK/K$ is in $\mathcal{C}$.
\end{enumerate}
\end{definition}

\begin{remark}
Serge Lang apparently invented this notion of distinguished classes of
field extensions. See Lang's \textit{Algebra} (third ed., Springer;
Chapter~V \S1, pages 227 \textit{et seq}.).
\end{remark}