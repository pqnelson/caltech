%%
%% winter-lecture09.tex
%% 
%% Made by Alex Nelson <pqnelson@gmail.com>
%% Login   <alex@lisp>
%% 
%% Started on  2026-01-24T08:04:37-0800
%% Last update 2026-01-24T08:04:37-0800
%% 

\lecture[Finite Galois theory]{}

\begin{definition}
We call the field extension $E/F$ a \define{Galois extension}
if it is normal and separable.
\end{definition}

\begin{definition}
For any Galois extension $E/F$, we define its \define{Galois Group}
$\Gal(E/F):=\Aut_{F}(E)=\End_{F}(E)\iso\End_{F}(E,\closure{F})$.
\end{definition}

\begin{note}\label{note:galois-extension-properties}
By Proposition~\ref{prop:3.4} and Proposition~\ref{prop:4.3}, we have:
\begin{enumerate}
\item\label{item:note:galois-extension-properties:transitivity} For any field extensions $L/E$ and $E/F$, if $L/F$ is Galois,
  then $L/E$ and $E/F$ are both Galois
\item If $E_{1}/F$ is Galois and $E_{2}/F$ is Galois, then
  $E_{1}E_{2}/E_{1}$ and $E_{1}E_{2}/E_{2}$ are both Galois
\item If $E_{1}/F$ and $E_{2}/F$ are both Galois extensions, then
  both $(E_{1}\cap E_{2})/F$ and $E_{1}E_{2}/F$ are both Galois.
\end{enumerate}
\end{note}

\begin{remark}
In this section, we consider only \emph{finite} Galois extensions.
\end{remark}

\begin{fundamental-thm-galois-thy}[Finite extension version]\label{thm:7.1}
Let $L/F$ be a finite Galois extension with $G=\Gal(L/F)$. Let
\begin{subequations}
  \begin{align}
\mathcal{E} &= \{\mbox{all intermediate field extensions } L/E/F\},
\intertext{and}
\mathcal{H} &= \{\mbox{subgroups }H\subgroup G\}
  \end{align}
\end{subequations}
We define
\begin{equation}
\begin{split}
\Phi\colon & \mathcal{E}\to\mathcal{H}\\
& E\mapsto\Gal(L/E),
\end{split}
\end{equation}
and
\begin{equation}
\begin{split}
\Psi\colon & \mathcal{H}\to\mathcal{E}\\
& H\mapsto L^{H},
\end{split}
\end{equation}
where $L^{H}=\{\alpha\in L\mid\forall\sigma\in H\ldotp\sigma(\alpha)=\alpha\}$.
For $E=L^{H}$ and $H=\Gal(L/E)$ we denote this relation by $E\sim H$.
Then
\begin{enumerate}
\item $\Phi^{-1}=\Psi$ and $\Psi^{-1}=\Phi$
\item\label{item:thm:7.1:item-2} Let $E_{i}\sim H_{i}$ for $i=1,2$. Then $E_{1}\subset E_{2}$
  if and only if $H_{1}\supset H_{2}$
\item\label{item:thm:7.1:item-3} Let $E_{i}\sim H_{i}$ for $i=1,2$. Then $E_{1}E_{2}\sim H_{1}\cap H_{2}$
  and $E_{1}\cap E_{2}\sim\langle H_{1},H_{2}\rangle$.
\item\label{item:thm:7.1:item-4} For $E\sim H$ and $\varphi\in G$, then $\varphi(E)\sim\varphi H\varphi^{-1}$
\item\label{item:thm:7.1:item-5} For $E\sim H=\Gal(L/E)$, then $E/F$ is a (finite)
  Galois extension if and only if $H\normalsubgroup G$ is
  a normal subgroup of $G$. Furthermore, $\Gal(E/F)\iso G/H$.
\end{enumerate}
\end{fundamental-thm-galois-thy}

The proof of the fundamental theorem of Galois theory will be delayed
until next week.

\begin{lemma}[Dedekind's theorem]\label{lemma:7.2}
Let $M/F$ and $N/F$ be arbitrary field extensions. Let $f_{1}$, \dots,
$f_{n}\in\hom_{F}(M,N)$ be distinct. Then $f_{1}$, \dots, $f_{n}$ are distinct.
\end{lemma}

\begin{proof}
By induction on $n$. The base case $n=1$ is obvious.

Suppose
\begin{equation}
\sum^{n+1}_{j=1}a_{j}f_{j}=0,
\end{equation}
where $a_{j}\in N$. Since $f_{1}\neq f_{2}$, there exists an $x\in M$
such that $f_{1}(x)\neq f_{2}(x)$.
Then for all $y\in M$ we have
\begin{subequations}
  \begin{align}
0 &= \sum^{n+1}_{j=1}a_{j}f_{j}(xy) = \sum^{n+1}_{j=1}a_{j}f_{j}(x)f_{j}(y)\\
  &= f_{1}(x)\cdot0=f_{1}(x)\sum^{n+1}_{j=1}a_{j}f_{j}(y)=\sum^{n+1}_{j=1}a_{j}f_{1}(x)f_{j}(y).
  \end{align}
\end{subequations}
Then subtracting the second line from the first equation, we obtain
\begin{equation}
0=\sum^{n+1}_{j=2}a_{j}\bigl(f_{j}(x)-f_{1}(x)\bigr)f_{j}.
\end{equation}
Then by the inductive hypothesis
\begin{equation}
a_{j}(f_{j}(x)-f_{1}(x))f_{j}=0
\end{equation}
for all $j=2,\dots,n+1$. Then by assumption
\begin{equation}
a_{j}(f_{j}(x)-f_{1})=0,
\end{equation}
which means that $f_{j}(x)-f_{1}(x)\neq0$ so $a_{j}=0$ for all $j$.
\end{proof}

\begin{lemma}\label{lemma:7.3}
For any finite extension $M/F$ and arbitrary extension $N/F$, we have
\begin{equation*}
|\hom_{F}(M,N)|\leq[M:F].
\end{equation*}
\end{lemma}

\begin{proof}
Let $n=[M:F]$ and $a_{1},\dots,a_{n}\in M$ be a basis of $M$ as a
vector space over $F$. Assume for contradiction that there exists
distinct (and thereby linearly independent) $f_{1},\dots,f_{n+1}\in\hom_{F}(M,N)$.
Let
\begin{equation}
\begin{split}
\varphi\colon& N^{n+1}\to N^{n}\\
&(b_{j})_{j=1}^{n+1}\mapsto\bigl(\sum^{n+1}_{j=1}b_{j}f_{j}(a_{i})\bigr)_{i=1}^{n}=(f_{j}(a_{i}))_{ij}(b_{j})_{j}
\end{split}
\end{equation}
By comparing the dimension of both sides, the map is not
injective. Then there exists a $(b_{j})_{j}\neq(0)_{j}$ such that
\begin{equation}
\sum_{j}b_{j}f_{j}(a_{i})=0
\end{equation}
for all $i$. For any
\begin{equation}
x=\sum_{i}^{n}c_{i}a_{i}\in M,
\end{equation}
where each $c_{i}\in F$, we have
\begin{subequations}
  \begin{align}
\sum_{j}b_{j}f_{j}(x)
&=\sum_{j}f_{j}\left(\sum_{i}^{n}c_{i}a_{i}\right)\\
&=\sum_{i}c_{i}\left(\sum_{j}b_{j}f(a_{i})\right)\\
&=\sum_{i}c_{i}\cdot0\\
&=0.
  \end{align}
\end{subequations}
Then
\begin{equation}
\sum_{j}b_{j}f_{j}=0,
\end{equation}
and each $b_{j}=0$. Hence we obtain our contradiction. Then we cannot
take distinct $f_{1}$, \dots, $f_{n+1}$.
\end{proof}

\begin{theorem}[Artin]\label{thm:7.4}
Let $E$ be a field, let $G\subgroup\Aut(E)$ be a finite subgroup of
the automorphisms of $E$. Let $E^{G}$ be the fixed field of $E$. Then
$[E:E^{G}]=|G|$ and $\Aut(E/E^{G})=G$.
\end{theorem}

\begin{proof}
By Lemma~\ref{lemma:7.3}, let
\begin{equation}
n:=|G|\leq|\Aut(E/E^{G})|\leq[E:E^{G}].
\end{equation}
It suffices to show that $[E:E^{G}]\leq n$. 
Set $G=\{f_{1},\dots,f_{n}\}$. Then by Lemma~\ref{lemma:7.2} these are
linearly independent, so
\begin{equation}
\sum^{n}_{i=1}f_{i}(a)\neq0
\end{equation}
for some $a\in E$. If $[E:E^{G}]\geq n+1$, then there exists
\emph{linearly independent} $b_{1}$, \dots, $b_{n+1}\in E$. As in the
last lemma, we construct a map
\begin{equation}
\begin{split}
E^{n+1}\to E^{n}\\
(y_{j})_{j=1}^{n+1}\mapsto(\sum^{n+1}_{j=1}f_{i}^{-1}(b_{j})y_{j})^{n}_{i=1}
\end{split}
\end{equation}
is a linear map. By the same reason, there exists a nonzero element of
the kernel $(x_{1},\dots,x_{n+1})$ such that
\begin{equation}
\sum f^{-1}_{i}(b_{j})x_{j}=0
\end{equation}
for all $i$. We may assume that $x_{N}=a$ for some $1\leq N\leq n+1$.
Since
\begin{equation}
\sum_{j}b_{j}f_{i}(x_{j})=f_{i}(\sum_{j}f^{-1}_{i}(b_{j})x_{j})=0,
\end{equation}
we have
\begin{equation}
\sum_{i}\sum_{j}b_{j}f_{i}(x_{j})=\sum_{j}\left(\sum_{i}f_{i}(x_{j})\right)b_{j}=0.
\end{equation}
Let us denote
\begin{equation}
c_{j}=\sum_{i}f_{i}(x_{j}),
\end{equation}
i.e., every $c_{j}\in E^{G}$. Because
\begin{equation}
c_{N}=\sum_{i}f_{i}(x_{N})=\sum_{i}f_{i}(a)\neq0
\end{equation}
by assumption, we have $(b_{j})_{j}$ are linearly dependent. But
this contradicts the assumption that they are linearly
indendent. Hence the result.
\end{proof}

\begin{proposition}\label{prop:7.5}
For any finite extension $E/F$, the following are equivalent:
\begin{enumerate}
\item $E/F$ is a Galois extension
\item $|\Aut(E/F)|=[E:F]$
\item\label{item:prop:7.5:3} $E^{\Aut(E/F)}=F$
\item $E$ is a splitting field of a separable polynomial in $F[x]$
\end{enumerate}
\end{proposition}

\begin{proof}
$(1)\implies(2)$ $|\Aut(E/F)|=|\hom_{F}(E,\closure{F})|$ since Galois
  extensions are normal, and $|\hom_{F}(E,\closure{F})|=[E:F]$ since
  Galois extensions are separable. Hence the claim.

$(2)\implies(3)$ By Theorem~\ref{thm:7.4},
\begin{equation}
[E:E^{\Aut(E/F)}]=|\Aut(E/F)|=[E:F]
\end{equation}
and
\begin{equation}
[E:F]=[E:E^{\Aut(E/F)}][E^{\Aut(E/F)}:F].
\end{equation}
Hence the result.

$(3)\implies(4)$ For $a\in E$, write
\begin{equation}
S=\{\varphi(a)\mid \varphi\in\Aut(E/F)\}=\{a_{1},\dots,a_{n}\}.
\end{equation}
Assume $a_{1}=a$. For each $a_{i}\in S$, let
$\varphi_{i}\in\Aut(E/F)$ be such that $\varphi_{i}(a)=a_{i}$. Let
\begin{equation}
P(x)=\prod^{n}_{i=1}(x-a_{i})\in E[x].
\end{equation}
We will show
\begin{equation}
P\in E^{\Aut(E/F)}[x]=F[x].
\end{equation}
For any $\varphi\in\Aut(E/F)$, $\varphi$ permiutes $S$. Since the
$k^{\text{th}}$ coefficient of $P(x)$ is given by the $k^{\text{th}}$
elementary symmetric polynomial
\begin{equation}
s_{k}(a_{1},\dots,a_{n})=\sum_{1\leq i_{1}<\dots<i_{k}\leq n}a_{i_{1}}(\dots)a_{i_{k}},
\end{equation}
which is fixed by $\varphi$. Hence $P\in E^{\Aut(E/F)}[x]=F[x]$. Let
$p(x)$ be a minimal polynomial of $a$. Then $p\divides P$. Since $P$
is separable, then $p$ is separable. Hence $E/F$ is separable. By
Theorem~\ref{thm:4.5}, we may take $a$ to be a primitive element
$E=F(a)$. Then by definition, $S$ is the set of all roots of $P$ and
so $E$ is the splitting field of $P$.

% (4)\implies(1) was part of the next lecture, but fits better here
$(4)\implies(1)$ Assume $E$ is a splitting field for a separable
polynomial $f\in F[x]$. We want to prove $E/F$ is Galois. We have
\begin{equation}
E=F(a_{1},\dots,a_{n})
\end{equation}
for
\begin{equation}
f(x)=\prod_{i=1}^{n}(x-a_{i}).
\end{equation}
By Theorem~\ref{thm:3.3}, $E/F$ is normal. Since thte minimal
polynomial $f_{i}$ of $a_{i}$ divides $f$, we see $f_{i}$ (and
therefore $a_{i}$) are separable. Hence $E/F$ is also separable.
\end{proof}