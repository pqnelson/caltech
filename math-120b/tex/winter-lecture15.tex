%%
%% winter-lecture15.tex
%% 
%% Made by Alex Nelson <pqnelson@gmail.com>
%% Login   <alex@lisp>
%% 
%% Started on  2026-02-07T09:18:31-0800
%% Last update 2026-02-07T09:18:31-0800
%% 

\lecture{}

\begin{definition}
A group $G$ is \define{Solvable} if there exists a finite sequence of
subgroups $\TrivialGroup=G_{0}\normalsubgroup G_{1}\normalsubgroup\dots\normalsubgroup G_{n}=G$
(i.e., each $G_{i}$ is a subgroup of $G$, and $G_{i}$ is a normal
subgroup of $G_{i+1}$) such that $G_{i+1}/G_{i}$ is Abelian for all $i$.
\end{definition}

\begin{remark}
For finite groups, an equivalent criteria is for the quotient groups
$G_{i+1}/G_{i}$ to be cyclic groups of prime order.
\end{remark}

\begin{definition}
A finite field extension $E/k$ is \define{Solvable} if there exists a field
extension $L/E$ such that $L/k$ is Galois and $\Gal(L/k)$ is a
solvable group.

\textsc{Note}: if $E$ is Galois and solvable, then $\Gal(E/k)$ is solvable (by
the next lemma---and also see Theorem~\ref{thm:7.1}~\ref{item:thm:7.1:item-5}).
\end{definition}

\begin{lemma}\label{lemma:12.1}
Let $G$ be a solvable group, let $G'\subgroup G$ be any subgroup of $G$.
Let $N\normalsubgroup G$ be a normal subgroup of $G$. Then both $G'$
and $G/N$ are solvable groups.
\end{lemma}

\begin{proof}
Let
\begin{equation}
\TrivialGroup=G_{0}\normalsubgroup G_{1}\normalsubgroup\dots\normalsubgroup G_{n}=G,
\end{equation}
where each quotient $G_{i+1}/G_{i}$ is Abelian. Then
$G_{i}N\normalsubgroup G_{i+1}N$ and we have the canonical map
\begin{equation}
\phi\colon G_{i+1}/G_{i}\onto(G_{i+1}N)/(G_{i}N)
\end{equation}
where $(G_{i+1}N)/G_{i}\normalsubgroup\ker(\phi)\normalsubgroup G_{i+1}/G_{i}$,
so $(G_{i+1}N)/(G_{i}N)$ must be Abelian.

The argument for $G'$ is similar, we just take $G_{i}\cap G'$ in the
subgroup series.
\end{proof}

\begin{remark}
If $E/k$ is solvable, then there exists a solvable Galois extension
$L/k$ such that $E\subset L$. We often will just use this in the
abbreviated form ``Since $E/k$ is solvable, then consider the solvable
Galois extension $L/k$ such that $L/E/k$.''
\end{remark}

\begin{proposition}\label{prop:12.2}
The class of solvable extensions is distinguished.
\end{proposition}

\begin{proof}
We just need to check two things:
\begin{enumerate}
\item Tower property: For any field extensions $E/F/k$, we have $E/k$ is solvable if
  and only if both $E/F$ and $F/k$ are solvable
\item Lifting property: For any solvable extension $E/k$ and any arbitrary field
  extension $F/k$, we have $EF/F$ is solvable.
\end{enumerate}
We will prove (2), then use it to prove (1).

\textsc{Subproof (Lifting property)}: Let $E/k$ be a solvable
extension. Then consider $K/k$ a solvable Galois extension such that $K/E/k$.
Let $F/k$ be an arbitrary field extension.
So we have the following situation (where dotted lines are arbitrary
field extensions, solid lines are solvable field extensions)
\begin{equation}
\vcenter{\xymatrix{
K \ar@/_2pc/@{-}_{\text{Galois solvable}}[ddrr]\ar@{-}[dr] & & & \\
& E\ar@{-}[dr] & & F\ar@{..}[dl]\\
& & k & }}
\end{equation}
By Proposition~\ref{prop:3.4} and Proposition~\ref{prop:4.3}, it follows
that $KF/F$ is Galois.
Now, both $KF/K$ and $KF/F$ are Galois extensions, so we have:
\begin{equation}
\vcenter{\xymatrix{
& & KF\ar@{..}[dll]_{\text{Galois}}\ar@{..}[ddr]^{\text{Galois}} &\\
%    & & & \\      
K \ar@/_2pc/@{-}_{\text{Galois solvable}}[ddrr]\ar@{-}[dr] & & & \\
& E\ar@{-}[dr] & & F\ar@{..}[dl]\\
& & k & }}
\end{equation}
Then by Theorem~\ref{thm:7.6}, we have
\begin{subequations}
  \begin{align}
\Gal(KF/F) &\iso\Gal(K/K\cap F)\\
&\subgroup\Gal(K/k)
  \end{align}
\end{subequations}
and since $\Gal(K/k)$ is solvable, we have $\Gal(KF/F)$ is solvable by
Lemma~\ref{lemma:12.1}.
In particular, this means $EF/F$ is solvable, so we have the following diagram:
\begin{equation}
\vcenter{\xymatrix{
& & KF\ar@{-}[dll]_{\text{Galois}}\ar@{-}[ddr]^{\text{Galois}}\ar@{-}[d] &\\
K \ar@/_2pc/@{-}_{\text{Galois solvable}}[ddrr]\ar@{-}[dr] & & EF\ar@{..}[dl]\ar@{-}[dr] & \\
& E\ar@{-}[dr] & & F\ar@{..}[dl]\\
& & k & }}
\end{equation}
Hence the lifting property holds (i.e., $EF/F$ is solvable).

\textsc{Subproof (Tower property)}: \backwardproof\ Let $E/F/k$ be
such that $E/F$ and $F/k$ are solvable extensions; we want to prove
$E/k$ is solvable. Since $F/k$ is
solvable, there exists a solvable Galois $K/k$ such that $K/F/k$. So
we have the following situation:
\begin{equation}
\vcenter{\xymatrix{E\ar@{-}[d] & \\
F\ar@{-}[d] & \ar@{..}[l]\ar@{-}[dl]^{\text{Galois}}K \\
k & }}
\end{equation}
Then by the lifting property, both $E/F$ and $K/F$ implies $EK/K$ is
solvable. That is to say, there exists an $L/EK$ such that $L/K$ is
solvable Galois.
\begin{equation}
\vcenter{\xymatrix{ & L\ar@{..}[dl]\ar@{..}[d]\ar@/^2pc/@{-}[dd]^{\text{Galois}}\\
E\ar@{-}[d]\ar@{..}[r] & EK\ar@{-}[d] \\
F\ar@{-}[d] & \ar@{..}[l]\ar@{-}[dl]^{\text{Galois}}K \\
k & }}
\end{equation}
We can conclude $L/k$ is Galois by Note~\ref{note:galois-extension-properties}~\ref{item:note:galois-extension-properties:transitivity}.
We just need to prove $\Gal(L/k)$ is solvable. We see that
\begin{equation}
\Gal(K/k)\iso\frac{\Gal(L/k)}{\Gal(L/K)},
\end{equation}
by Theorem~\ref{thm:7.1}. Since both $\Gal(K/k)$ and $\Gal(L/K)$ are
solvable, then $\Gal(L/k)$ is solvable by concatenating the subnormal
series. Hence $E/k$ is solvable.

\forwardproof\ Assume $E/k$ is solvable. Then consider the Galois extension $L/E$ such that
$L/k$ is solvable Galois. Then by the lifting property, clearly $F/k$
is solvable. So we have the current situation:
\begin{equation}
\vcenter{\xymatrix{
E\ar@{..}[d]\ar@/_2pc/@{-}[dd]_{\text{Galois}} & L\ar@{..}[dl]\ar@{..}[ddl]\ar@{-}[l]_{\text{Galois}} \\
F\ar@{-}[d] & \\
k & }}
\end{equation}
Now, we want to prove $E/F$ is solvable. Since $L/k$ is Galois, then
$L/F$ is Galois. Suffices to show that $L/F$ is solvable (we only need
to show $\Gal(L/F)$ is a solvable group). Since $\Gal(L/F)\subgroup\Gal(L/k)$
and $\Gal(L/k)$ is solvable, then by Lemma~\ref{lemma:12.1} we have
the result that $E/F$ is solvable.
\end{proof}

\begin{definition}
A finite extension $E/k$ is \define{Solvable by Radicals} if there
exists an extension $L/E$ such that $L/k$ is \define{Radical}, i.e.,
there is a finite toiwer of extensions
\begin{equation}
L=L_{n}/L_{n-1}/\cdots/L_{1}/L_{0}=k
\end{equation}
such that $L_{i+1}=L_{i}(\alpha_{i+1})$ where $\alpha_{i+1}$ is a root
of either $x^{n}-a_{i}\in L_{i}[x]$ (provided $\Char(k)\ndivides n$)
or $x^{p}-x-a_{i}\in L_{i}[x]$ where $\Char(k)=p$.
\end{definition}

\begin{proposition}\label{prop:11.3}
Let $F$ be a field with $\Char(F)=p>0$. Let
\begin{equation}
f_{a}(x)=x^{p}-x-a\in F[x].
\end{equation}
Then the following all hold:
\begin{enumerate}
\item $f_{a}(x)$ either splits in $F$ or is an irreducible polynomial.
\item Moreover, if $f_{a}$ is irreducible, then
  $\Gal(F(\alpha)/F)\iso\ZZ/p\ZZ$ for any root $\alpha$ of $f_{a}$
\item If $E/F$ is cyclic of degree $p$ (i.e., $E/F$ is Galois and
  $\Gal(E/F)$ is cyclic and $\Gal(E/F)\iso\ZZ/p\ZZ$), then there
  exists some $\alpha\in E$ such that $E=F(\alpha)$ and
  $f_{a}(\alpha)=0$ for some $a\in F$.
\end{enumerate}
\end{proposition}

\begin{proof}
Homework.
\end{proof}

\begin{theorem}
Let $E/k$ be a separable extension.\marginnote{``$E/k$ separable'' is
  not needed for the result, it only simplifies the proof}
Then $E/k$ is solvable if and only if $E/k$ is solvable-by-radicals.
\end{theorem}

\begin{proof}
\forwardproof\ Assume $E/k$ is a solvable extension. Then consider
$K/E$ such that $K/k$ is solvable Galois.
\begin{equation}
\vcenter{\xymatrix{K\ar@{..}[d]\ar@/^2pc/@{-}[dd]^{\text{Galois}}\\
E\ar@{-}[d] \\
k}}
\end{equation}
Then we write
\begin{equation}
m := \prod_{i}p_{i}
\end{equation}
where $p_{i}$ are prime numbers such that $p_{i}\ndivides\Char(k)$ and
$p_{i}\divides[K:k]$. Then we introduce the field
\begin{equation}
F:=k(\zeta_{m})
\end{equation}
where $\zeta_{m}$ is the primitive $m^{\text{th}}$ root of unity. So
now our situation looks like:
\begin{equation}
\vcenter{\xymatrix{K\ar@{..}[dr]\ar@/_2pc/@{-}[ddrr]_{\text{Galois}} & & & \\
& E\ar@{-}[dr] & & \ar@{..}[dl] F \\
&   & k & }}
\end{equation}
By Proposition~\ref{prop:9.1}, $F/k$ is Abelian since
$\Gal(F/k)\subgroup\MultGroup{(\ZZ/m\ZZ)}$.

On the other hand, $KF/F$ is solvable Galois by Proposition~\ref{prop:12.2}
and Note~\ref{note:galois-extension-properties}. This gives us
\begin{equation}
\vcenter{\xymatrix{& & KF\ar@{..}[dll]\ar@{-}[ddr]^{\text{Galois}}\\
K\ar@{..}[dr]\ar@/_2pc/@{-}[ddrr]_{\text{Galois}} & & & \\
& E\ar@{-}[dr] & & \ar@{..}[dl] F \\
&   & k & }}
\end{equation}
Now, by definition of $m$, $KF/k$ is radical. In fact, $F/k$ is
radical because $\Char(k)\ndivides m$. On the other hand, by
\begin{equation}
\Gal(KF/F)\iso\Gal(K/K\cap F)\subgroup\Gal(K/k),
\end{equation}
and Lagrange's theorem
\begin{equation}
[KF:F]\divides[K:k],
\end{equation}
we see that all prime factors of $[KF:F]$ appear in $m=\prod_{i}p_{i}$.
Therefore each step of the tower of extensions $KF/\cdots/F$
corresponding to the composition series of $\Gal(KF/F)$ [each step
  $(KF)_{i}$ is a simple extension by adjoining a root of unity]
satisfies the hypotheses of Proposition~\ref{prop:11.2}. By using
Proposition~\ref{prop:11.2} or Proposition~\ref{prop:11.3} repeatedly,
$KF/F$ is radical. Hence $KF/k$ is radical, which forces $E/k$ to be
solvable by radicals.
\end{proof}