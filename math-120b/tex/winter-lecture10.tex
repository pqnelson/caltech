%%
%% winter-lecture10.tex
%% 
%% Made by Alex Nelson <pqnelson@gmail.com>
%% Login   <alex@lisp>
%% 
%% Started on  2026-01-27T07:46:57-0800
%% Last update 2026-01-27T07:46:57-0800
%% 

\lecture{}

\begin{remark}
We deferred proving the Fundamental Theorem of Galois theory until
now. Note: we did not use it in proving any result from last lecture
(so there is no danger of circular reasoning). Let us re-state the
theorem, then prove it.
\end{remark}

\begin{fundamental-thm-galois-thy}[Finite extension version]
Let $L/F$ be a finite Galois extension with $G=\Gal(L/F)$. Let
\begin{subequations}
  \begin{align}
\mathcal{E} &= \{\mbox{all intermediate field extensions } L/E/F\},
\intertext{and}
\mathcal{H} &= \{\mbox{subgroups }H\subgroup G\}
  \end{align}
\end{subequations}
We define
\begin{equation}
\begin{split}
\Phi\colon & \mathcal{E}\to\mathcal{H}\\
& E\mapsto\Gal(L/E),
\end{split}
\end{equation}
and
\begin{equation}
\begin{split}
\Psi\colon & \mathcal{H}\to\mathcal{E}\\
& H\mapsto L^{H},
\end{split}
\end{equation}
where $L^{H}=\{\alpha\in L\mid\forall\sigma\in H\ldotp\sigma(\alpha)=\alpha\}$.
For $E=L^{H}$ and $H=\Gal(L/E)$ we denote this relation by $E\sim H$.
Then
\begin{enumerate}
\item $\Phi^{-1}=\Psi$ and $\Psi^{-1}=\Phi$
\item Let $E_{i}\sim H_{i}$ for $i=1,2$. Then $E_{1}\subset E_{2}$
  if and only if $H_{1}\supset H_{2}$
\item Let $E_{i}\sim H_{i}$ for $i=1,2$. Then $E_{1}E_{2}\sim H_{1}\cap H_{2}$
  and $E_{1}\cap E_{2}\sim\langle H_{1},H_{2}\rangle$.
\item For $E\sim H$ and $\varphi\in G$, then $\varphi(E)\sim\varphi H\varphi^{-1}$
\item For $E\sim H$ and $E/F$ is Galois, then $H\normalsubgroup G$ is
  a normal subgroup of $G$.
\end{enumerate}
\end{fundamental-thm-galois-thy}

Note: this is a \emph{longer} proof than normal, requiring proofs for
5 results. We will state the claim, then have a ``subproof'' for each
claim. Each subproof will end with a checkmark (``\checkmark'').

\begin{proof}
We assume $L/F$ is a finite Galois extension. We write
\begin{subequations}
\begin{equation}
G=\Gal(L/F)=\Aut_{F}(L),
\end{equation}
\begin{equation}
\mathcal{E}=\{\mbox{fields }E\mid L/E/F\}
\end{equation}
\begin{equation}
\mathcal{H}=\{\mbox{subgroups }H\subgroup G\}.
\end{equation}
\end{subequations}
Now, we will start proving the claims.

\begin{enumerate}
\item \textsc{Claim:} $\Phi^{-1}=\Psi$ and $\Psi^{-1}=\Phi$
\begin{proof}[Subproof]\let\qedsymbol\checkmark
For $E\in\mathcal{E}$, by Proposition~\ref{prop:7.5}~\ref{item:prop:7.5:3},
  we have
  \begin{equation}
(\Psi\circ\Phi)(E)=\Psi(\Gal(L/E))=L^{\Gal(L/E)}=E.
  \end{equation}
For $H\in\mathcal{H}$, by Theorem~\ref{thm:7.4} we have
\begin{equation}
(\Phi\circ\Psi)(H)=\Phi(L^{H})=\Gal(L/L^{H})=H.
\end{equation}
Hence the result.
\end{proof}
%(2)
\item\textsc{Claim:} Let $E_{i}\sim H_{i}$ for $i=1,2$. Then $E_{1}\subset E_{2}$
  if and only if $H_{1}\supset H_{2}$
\begin{proof}[Subproof]\let\qedsymbol\checkmark
\forwardproof\ If $E_{1}\subset E_{2}$, then for any $f\in\Gal(L/E_{2})$ we have
$f\in\Gal(L/E_{1})$. Hence $H_{1}\supset H_{2}$.

\backwardproof\ If $H_{1}\supset H_{2}$, then for any $x\in L^{H_{1}}$
we have $x\in L^{H_{2}}$. Hence $E_{1}\subset E_{2}$.
\end{proof}
%(3)
\item\textsc{Claim:} Let $E_{i}\sim H_{i}$ for $i=1,2$. Then $E_{1}E_{2}\sim H_{1}\cap H_{2}$
  and $E_{1}\cap E_{2}\sim\langle H_{1},H_{2}\rangle$.
\begin{proof}[Subproof]\let\qedsymbol\checkmark
\textsc{Step 1:} $\Gal(L/E_{1}E_{2})\subset H_{1}\cap H_{2}$.
Since $E_{1}$, $E_{2}\subset E_{1}E_{2}$ by Item~\ref{item:thm:7.1:item-2},
we see that
\begin{equation}
\Gal(L/E_{1}E_{2})\subset\Gal(L/E_{1})\cap\Gal(L/E_{2}),
\end{equation}
which establishes the first step.

\textsc{Step 2:} $\Gal(L/E_{1}E_{2})\supset H_{1}\cap H_{2}$.
If $f\in H_{1}\cap H_{2}$, then $f$ fixes any element of $E_{1}$ and
$E_{2}$ (and then fixes every element of $E_{1}E_{2}$). Hence $f\in\Gal(L/E_{1}E_{2})$.

(Proof of the other claim omitted.)
\end{proof}
%(4)
\item\textsc{Claim:} For $E\sim H$ and $\varphi\in G$, then $\varphi(E)\sim\varphi H\varphi^{-1}$
\begin{proof}[Subproof]\let\qedsymbol\checkmark
For $f\in\Gal(L/\varphi(E))$ and for any $x\in E$, we have
\begin{equation}
(\varphi^{-1}\circ f\circ\varphi)(x)=\varphi^{-1}(f(\varphi(x)))=\varphi^{-1}(\varphi(x))=x;
\end{equation}
which is to say
\begin{equation}
\varphi^{-1}\circ f\circ\varphi\in\Gal(L/E)=H\implies f\in\varphi H\varphi^{-1}.
\end{equation}
Now, for any $f=\varphi g\varphi^{-1}\in\varphi H\varphi^{-1}$ (where
$g\in H$) and for any $y=\varphi(x)\in\varphi(E)$ (where $y\in E$), we
see
\begin{equation}
f(y)=\varphi g\varphi^{-1}\varphi(x)=\varphi(g(x))=\varphi(x)=y,
\end{equation}
which is to say $f\in\Gal(L/\varphi(E))$. Hence the claim.
\end{proof}
%(5)
\item\textsc{Claim:} For $E\sim H=\Gal(L/E)$, then $E/F$ is a (finite)
  Galois extension if and only if $H\normalsubgroup G$ is
  a normal subgroup of $G$. Furthermore, $\Gal(E/F)\iso G/H$.
\begin{proof}[Subproof]\let\qedsymbol\checkmark
\textsc{Step 1:}\forwardproof\ Let $\phi\in G$. Since $E/F$ is Galois (and, in
particular, normal), $\phi(E)=E$. That is to say
\begin{equation}
\Gal(L/\phi(E))=\Gal(L/E)=H
\end{equation}
and by Item~\ref{item:thm:7.1:item-4}, we have
\begin{equation}
\Gal(L/\phi(E))=\phi H\phi^{-1}.
\end{equation}
Hence $H$ is normal.

\textsc{Step 2:}\backwardproof\ Assume $H$ is a normal subgroup. Then
$E/F$ is separable by Proposition~\ref{prop:4.3}. We will show that
$E/F$ is normal. Let $L\subset\closure{F}$ be the normal
closure(?). By Proposition~\ref{prop:2.8}, for any
$\phi\in\hom_{F}(E,\closure{F})$ is extended to
$\overline{\phi}\in\hom_{F}(L,\closure{F})$. Since $L/F$ is normal,
\begin{equation}
\overline{\phi}(L)=L,
\end{equation}
which is to say, $\overline{\phi}\in\Gal(L/F)=G$. Since
$H\normalsubgroup G$, we have
\begin{equation}
H=\overline{\phi}H\overline{\phi}^{-1}.
\end{equation}
By Item~\ref{item:thm:7.1:item-4},
\begin{equation}
E=\overline{\phi}(E)=\phi(E),
\end{equation}
which means $E$ is normal.

\textsc{Step 3:} $\Gal(E/F)\iso G/H$. Let
\begin{equation}
\begin{split}
i\colon&G\to\Gal(E/F)\\
&\phi\mapsto\phi|_{E}.
\end{split}
\end{equation}
As we checked, $i$ is surjective and
\begin{equation}
\phi\in\ker(i)\iff\phi|_{E}=\id_{E}\iff\phi\in H=\Gal(L/E).
\end{equation}
So $\Gal(E/F)\iso G/\ker(i)=G/H$.
\end{proof}
\end{enumerate}
Hence the fundamental theorem of Galois theory.
\end{proof}

\begin{theorem}[Translation theorem]\label{thm:7.6}
Let $L/E_{i}/F$ be a tower of extensions (with $i=1,2$).
\begin{enumerate}
\item If $E_{1}/E_{1}\cap E_{2}$ is a finite Galois extension, then so
  is $E_{1}E_{2}/E_{2}$ and $\Gal(E_{1}/E_{1}\cap E_{2})\iso\Gal(E_{1}E_{2}/E_{2})$.
\item If both $E_{i}/E_{1}\cap E_{2}$ are finite Galois extensions,
  then so is $E_{1}E_{2}/E_{1}\cap E_{2}$ and
  \begin{equation}
\Gal\left(\frac{E_{1}E_{2}}{E_{1}\cap E_{2}}\right)\iso\Gal(E_{1}/E_{1}\cap E_{2})\times\Gal(E_{2}/E_{1}\cap E_{2})
  \end{equation}
\end{enumerate}
\end{theorem}

\begin{corollary}\label{cor:7.7}
Let $E_{i}/F$ ($i=1,\dots,n$) be a finite collection of finite Galois
extensions. Let $G_{i}=\Gal(E_{i}/F)$ be their Galois groups. If for
each $i$ we have
\begin{equation}
E_{i+1}\cap(E_{1}\cdots E_{i})=F,
\end{equation}
then $(E_{1}\cdots E_{n})/F$ is a
finite Galois extension and
\begin{equation}
\Gal((E_{1}\cdots E_{n})/F)\iso G_{1}\times\dots\times G_{n}.
\end{equation}
\end{corollary}