%%
%% winter-lecture05.tex
%% 
%% Made by Alex Nelson <pqnelson@gmail.com>
%% Login   <alex@lisp>
%% 
%% Started on  2026-01-15T09:32:04-0800
%% Last update 2026-01-15T09:32:04-0800
%% 

\lecture{}

% This was technically the end of lecture 4, and reiterated at the
% start of lecture 5, but I just decided to write it once.
\begin{proposition}\label{prop:4.3}
The class of separable extensions is distinguished, i.e.,
\begin{enumerate}
\item if $L/E$ and $E/F$ are algebraic extensions, then $L/F$ is
  separable if and only if $L/E$ is separable and $E/F$ is separable.
\item For separable extensions $E/K$ and any other extension $F/K$, we
  have $EF/F$ is separable.
\end{enumerate}
\end{proposition}

\begin{proof}
\begin{enumerate}
\item \forwardproof\ Assume $L/F$ is separable. Then every element of
  $L$ is separable over $E$, and thus $L/E$ is separable.

  On the other hand, $E\subset L$ is a subfield, and so by the same
  reasoning $E/F$ is separable.

  \backwardproof Assume $L/E$ and $E/F$ are separable.

  \textsc{Subcase 1:} Suppose $L/F$ is a finite extension.
  Then the result follows by Proposition~\ref{prop:4.1} since
  \begin{subequations}
    \begin{align}
[L:F]_{s} &= [L:F]\\
&= [L:E]_{s}[E:F]_{s}=[L:E][E:F],
    \end{align}
  \end{subequations}
  hence the result.

  \textsc{Subcase 2:} Suppose $L/F$ is an infinite extension.
  Let $\alpha\in L$. Since $L/E$ is separable by hypothesis, then
  $\alpha$ is separable over $E$. Let $p_{\alpha}\in E[x]$ be its
  minimal polynomial
\begin{equation}
p_{\alpha}(x)=\sum^{n}_{i=1}a_{i}x^{i}\in E[x]
\end{equation}
which is separable (i.e., has no multiple roots). Then adjoining the
coefficients $a_{0}$, \dots, $a_{n}$ to $F$,
\begin{equation}
F' := F(a_{0},a_{1},\dots,a_{n})\subset E
\end{equation}
is separable over $F$ by assumption and $\alpha$ is separable over
$F'$ (then the minimal polynomial for $\alpha$ over $F'$ is also
$p_{\alpha}(x)$). By $F'(\alpha)/F'$ and $F'/F$ both being separable
finite extensions, $\alpha$ is separable over $F$. Since $\alpha\in L$
was arbitrary, we have the desired result.
\item Let $E/K$ be separable, let $F/K$ be any arbitrary field
  extension. Let $\alpha\in E\subset EF=F(E)$ be separable over $K$
  (hence $\alpha$ is separable over $F$). Then its minimal polynomial
  $p_{\alpha}(x)\in F[x]$ is separable. Hence $\alpha$ is separable
  over $F$. By Corollary~\ref{cor:4.2}, $F(E)=EF$ is separable over $F$.
  \qedhere
\end{enumerate}
\end{proof}

\begin{proposition}\label{prop:4.4}
Let $E/F$ be a splitting field of a separable polynomial $f(x)\in F[x]$.
Then $E/F$ is separable.
\end{proposition}

\begin{proof}
Let $\alpha_{1}$, \dots, $\alpha_{n}\in E$ be distinct roots of
$f(x)$, then by definition of a splitting field we have
\begin{equation}
E=F(\alpha_{1},\dots,\alpha_{n}).
\end{equation}
Let $p_{i}(x)\in F[x]$ be the minimal polynomial of $\alpha_{i}$. Then
$p_{i}$ is separable (so $p_{i}$ divides $f$). So each $\alpha_{i}$ is
separable. Hence $E/F$ is separable by Corollary~\ref{cor:4.2}.
\end{proof}

\begin{definition}
We say a field extension $E/F$ is a \define{Simple} extension if there
exists an $\alpha\in E$ such that $E=F(\alpha)$. We call this $\alpha$
a \define{Primitive Element}.
\end{definition}

\begin{theorem}[Primitive element theorem]\label{thm:4.5}
For any finite extension $E/F$, we see:
\begin{enumerate}
\item $E/F$ is simple if and only if there are finitely many
  subextensions between $E$ and $F$; and
\item If $E/F$ is (finite and) separable, then $E/F$ is simple.
\end{enumerate}
\end{theorem}

The proof of (1) is pretty standard across most texts.

\begin{proof}[Proof (1)]
\textsc{Case 1:} Suppose $F$ is a finite field. We will show next week
there exists an $\alpha\in E$ such that every nonzero in $E$ is a
power of $\alpha$ (i.e., any $\beta\in E$ there is an $m\in\NN$ such
that $\beta=\alpha^{m}$). Hence $E=F(\alpha)$. Since $E/F$ is finite,
$E$ is also a finite field. Then there are only finitely many subsets
of $E$, hence only finitely many subfields.

\textsc{Case 2:} Suppose $F$ is an infinite field.

\forwardproof\ Suppose $F(\alpha)=E$ for some $\alpha\in E$. For any
intermediate subextensions $E/K/F$, we write the minimal polynomial of
$\alpha$ over $K$ as
\begin{equation}
g_{K}(x)=\sum_{i}a_{i,K}x^{i}\in K[x],
\end{equation}
and also
\begin{equation}
S=\{f\in E[x]\mid f\mbox{ is monic and }f\mbox{ divides }g_{F}\}.
\end{equation}
Then $S$ is a finite set because there are only finitely many roots of
$g_{F}$, and $g_{K}\in S$. Then the map
\begin{equation}\label{eq:mapping-is-injective}
\begin{split}
  \{\mbox{subextensions }K/F\mid E/K\}&\to S\\
  K&\mapsto g_{K}
\end{split}
\end{equation}
is injective. In fact, for $K/F(a_{0,K},\dots,a_{n,K})$ where
$a_{i,K}$ are the coefficients of $g_{K}$, we have
\begin{equation}
[E:K]=[E:F(a_{0,K},\dots,a_{n,K})]=\deg(g_{K})
\end{equation}
by assumption $E=F(\alpha)$ and Proposition~\ref{prop:1.2}. So $K=F(a_{0,K},\dots,a_{n,K})$,
i.e., $g_{K}$ uniquely determines $K$. (This proves the mapping in
Equation~\eqref{eq:mapping-is-injective} is injective.\footnote{If
further $g_{F}$ splits in $E$, this mapping is a bijection.}) Hence there
are only finitely many subextensions.

\backwardproof\ Assume there are only finitely many subextensions
between $E$ and $F$. Suffices to show every $\alpha,\beta\in E$
has $F(\alpha,\beta)/F$ is simple. Then by induction
$F(\alpha_{1},\dots,\alpha_{n})$ is simple for any $n\in\NN$.

For any $\lambda\in F$, we write
\begin{equation}
F_{\lambda}:=F(\alpha+\lambda\beta)\subset F(\alpha,\beta).
\end{equation}
Since $F$ is infinite and by hypothesis there exists distinct
$\lambda_{1},\lambda_{2}\in F$ such that $\lambda_{1}\neq\lambda_{2}$
and $F_{\lambda_{1}}=F_{\lambda_{2}}$. Then
\begin{equation}
\alpha\beta\in F_{\lambda_{i}},
\end{equation}
for each $i=1,2$. (If there were no such $\lambda_{1},\lambda_{2}\in F$,
then $|\{F_{\lambda}\mid \lambda\in F\}|=\infty$ which violates the
hypothesis that $E/F$ is a finite extension.) Hence the result.
\end{proof}

\begin{proof}[Proof (2)]
Assume $E/F$ is a separable finite extension. By the same reason as
above, consider the case $E=F(\alpha\beta)$. Let
$\hom_{F}(F(\alpha\beta),\closure{F})\setminus\{0\}$ be the set of
$F$-embeddings of $E$ into $\closure{F}$. We claim this is a finite
set
\begin{equation}
\hom_{F}(F(\alpha\beta),\closure{F})\setminus\{0\}=\{\sigma_{1},\dots,\sigma_{n}\}.
\end{equation}
Now let us write
\begin{equation}
P(x)=\sum_{1\leq i< j\leq n}(\sigma_{i}-\sigma_{j})(\alpha+x\beta)\in\closure{F}[x].
\end{equation}
Then $P(x)\neq0$ because if $P(x)=0$, then there is an $i\neq j$
such that
\begin{equation}
\sigma_{i}(\alpha)+x\sigma_{i}(\beta)=\sigma_{i}(\alpha+x\beta)=\sigma_{j}(\alpha+x\beta)=\sigma_{j}(\alpha)+x\sigma_{j}(\beta)
\end{equation}
then $\sigma_{i}(\alpha)=\sigma_{j}(\alpha)$ and
$\sigma_{i}(\beta)=\sigma_{j}(\beta)$, which is a contradiction.
Then there exists a $c\in F$ such that $P(c)\neq0$. (If every $c\in F$
has $P(c)=0$, then there exists $i\neq j$ and there exists
$c_{1},c_{2}\in F$ such that $(\sigma_{i}-\sigma_{j})(\alpha+c_{1}\beta)=(\sigma_{i}-\sigma_{j})(\alpha+c_{2}\beta)$
which implies $\sigma_{i}(\alpha)=\sigma_{j}(\alpha)$ and $\sigma_{i}(\beta)=\sigma_{j}(\beta)$
which is a contradiction.)
Then $\sigma_{i}(\alpha+c\beta)$ are distinct for all
$i=1,\dots,n$. Hence
\begin{equation}
[F(\alpha+c\beta):F]=[F(\alpha+c\beta):F]_{s}\geq n
\end{equation}
and
\begin{equation}
[F(\alpha+c\beta):F]_{s}=|\hom_{F}(F(\alpha+c\beta),F)|,
\end{equation}
which implies $\sigma_{i}|_{F(\alpha+c\beta)}$ distinct $n$ elements,
but
\begin{equation}
[F(\alpha+\beta):F]=[F(\alpha\beta):F]_{s}=n,
\end{equation}
which implies $F(\alpha\beta)=F(\alpha+c\beta)$, i.e.,
$F(\alpha\beta)$ is simple.
\end{proof}