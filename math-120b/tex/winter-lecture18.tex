%%
%% winter-lecture18.tex
%% 
%% Made by Alex Nelson <pqnelson@gmail.com>
%% Login   <alex@lisp>
%% 
%% Started on  2026-02-14T07:24:13-0800
%% Last update 2026-02-14T07:24:13-0800
%% 

\lecture{}


\begin{corollary}\label{cor:14.4}
By inducing the Krull topology to $\Gal(\Omega/F)\iso\varprojlim\Gal(E/F)$
this is isomorphic as topological groups. In particular,
\begin{equation}
U_{\sigma}:=\{\Gal(\Omega/E)\mid\sigma\in\Gal(\Omega/E), E/F\mbox{ is a finite Galois extension}\}
%U_{\sigma}:=\{\sigma\in\Gal(\Omega/E)\mid E/F\mbox{ is a finite Galois extension}\}
\end{equation}
is a neighborhood basis of $\sigma\in\Gal(\Omega/F)$ and
\begin{equation}
  \begin{split}
    \Gal(\Omega/F)\to\Gal(E/F)\\
    \sigma\mapsto\sigma|_{E}
  \end{split}
\end{equation}
is continuous and surjective.
\end{corollary}

\begin{lemma}\label{lemma:14.5}
Let $G=\Gal(\Omega/F)$ and $S\subset\Omega$ be a finite subset such
that $GS=S$. Then $F(S)/F$ is a finite Galois and $\Gal(F(S)/F)\iso G/\Gal(\Omega/F(S))$.

(In particular, $\Gal(\Omega/F(S))$ is an open subgroup with finite index.)
\end{lemma}

\begin{proof}
We see $F(S)/F$ is a finite Galois extension.
\begin{enumerate}
\item It is normal by Proposition~\ref{prop:4.3}.
\item It is normal by Theorem~\ref{thm:3.3},
  Proposition~\ref{prop:2.8}, and $GS=S$ (which implies $GF(S)=F(S)$).
  So if $\sigma\in\hom_{F}(F(S),\closure{F})$, then
  $\sigma'\in\Gal(\Omega/F)=F$ by Proposition~\ref{prop:2.8}
\item It is clearly a finite extension.
\end{enumerate}
The other claim is also clear.
\end{proof}

\begin{lemma}\label{lemma:14.6}
Let $S\subset\Omega$ be a finite subset. Then 
$G_{S}=\{\sigma\in G\mid\sigma|_{S}=\id_{S}\}$ is an open subgroup of
$G$ with finite index. Moreover, $G_{S}$ is normal if and only if $GS=S$.
\end{lemma}

\begin{proof}
Let $\overline{S}=GS$. Then by Lemma~\ref{lemma:14.5},
\begin{equation}
G_{\overline{S}}=\Gal(\Omega/F(\overline{S}))\subset G_{S}  
\end{equation}
is an open subgroup with finite index.

We claim $G_{S}=\bigcup_{\sigma\in G_{S}}\sigma G_{\overline{S}}$. 
One direction $G_{S}\subset\bigcup_{\sigma\in G_{S}}\sigma G_{\overline{S}}$ is obvious.
The other direction: Let $\tau\in\bigcup_{\sigma\in G_{S}}\sigma G_{\overline{S}}$.
Then there exists a $\sigma\in G_{S}$ and a $\tau'\in G_{\overline{S}}$
such that $\tau=\sigma\tau'$ where $\tau'|_{G_{\overline{S}}}=\id_{G_{\overline{S}}}$
and $\tau|_{S}=\sigma\tau'|_{S}=\id_{S}$. Then we see $\tau\in G$ which
proves the claim $G_{S}\supset\bigcup_{\sigma\in G_{S}}\sigma G_{\overline{S}}$.

This implies $G_{S}$ is a union of open sets, and in particular
$G_{S}$ is open.

Since $\tau G_{S}\tau^{-1}=G_{\tau S}$, we see $G_{S}$ is normal iff
for all $\tau\in G$ we have $\tau S=S$ (or equivalently $GS=S$).
\end{proof}

\begin{proposition}\label{prop:14.7}
\begin{enumerate}
\item For $\Omega/E/F$ (where $E/F$ is a possibly infinite extension),
  $\Gal(\Omega/E)$ is closed in $G=\Gal(\Omega/F)$
\item For any subgroup $H\subgroup G$, $\Gal(\Omega/\Omega^{H})$ is
  the closure of $H$ (the smallest closed [topological] subgroup of
  $G$ containing $H$).
\end{enumerate}
\end{proposition}

\begin{proof}
\begin{enumerate}
\item Since
\begin{equation}
\Gal(\Omega/E)=\bigcap_{\substack{S\subset E\\S\text{ finite}}}G_{S},
\end{equation}
$G_{S}$ is closed by Lemma~\ref{lemma:14.6} and
Proposition~\ref{prop:14.2}, hence $\Gal(\Omega/E)$ is closed (since
it's the arbitrary intersection of closed subsets).
\item We apply (1) to $E=\Omega^{H}$, then $\Gal(\Omega/\Omega^{H})$
  is closed and by definition $H\subgroup\Gal(\Omega/\Omega^{H})$, and
  thus $\closure{H}\subgroup\Gal(\Omega/\Omega^{H})$.

  Suppose $\sigma\in\Gal(\Omega,\Omega^{H})\setminus\closure{H}$. Then
  by Corollary~\ref{cor:14.4}, there exists a finite Galois extension
  $E/F$ such that
\begin{equation}
\sigma\Gal(\Omega/E)\cap H=\emptyset.
\end{equation}
By $\pi=\pi_{E}\colon G\to\Gal(E/F)$, we have
\begin{equation}
\pi(\sigma\Gal(\Omega/E))=\{\pi(\sigma)\},
\end{equation}
and thus
\begin{equation}\label{eq:math120b:winter2026:lecture18:contradiction}
\{\pi(\sigma)\}\cap\pi(H)=\emptyset.
\end{equation}
Since
$\pi(H)\subgroup\Gal(E/F)$ by Proposition~\ref{prop:7.5},
\begin{equation}
\pi(H)=\Gal(E/E^{\pi(H)})=\Gal(E/E\cap\Omega^{H}).
\end{equation}
On the other hand, since $\sigma\in\Gal(\Omega/\Omega^{H})$, for any
$\alpha\in E|cap\Omega^{H}$ we have $\sigma(\alpha)=\alpha$. That is
to say, $\pi(\sigma)\in\Gal(E/E\cap\Omega^{H})=\pi(H)$. But this
contradicts Equation~\eqref{eq:math120b:winter2026:lecture18:contradiction}.
\end{enumerate}
\end{proof}

\begin{fundamental-thm-galois-thy}[Infinite version]\label{thm:14.8}
Let $\Omega/F$ be Galois, $G=\Gal(\Omega/F)$, $\mathcal{E}=\{E\mid\Omega/E/F\}$
(where $E/F$ may possibly be infinite), $\mathcal{H}=\{\mbox{closed subgroups }H\subgroup G\}$.
Then
\begin{equation}
\begin{split}
\Phi\colon&\mathcal{E}\to\mathcal{H}\\
&E\mapsto\Gal(\Omega/E)
\end{split}
\end{equation}
and
\begin{equation}
\begin{split}
\Psi\colon&\mathcal{H}\to\mathcal{E}\\
&H\mapsto\Omega^{H}
\end{split}
\end{equation}
are bijective and inverse of each other $\Phi^{-1}=\Psi$, $\Psi^{-1}=\Phi$.
\end{fundamental-thm-galois-thy}

\begin{proof}
First we note $\Phi$ is injective. In fact, if $\Gal(\Omega/E)=\Gal(\Omega/E')$,
and $\alpha\in E\setminus E'$, then there exist
$\sigma_{1},\dots,\sigma_{n}\in\Gal(\Omega/E'')$ such that the roots
of the minimal polynomial $p_{\alpha}\in E'[x]$ of $\alpha$ are
$\{\sigma_{i}(\alpha)\mid i=1,\dots,n\}$. But since
$\Gal(\Omega/E')=\Gal(\Omega/E)$ we must have
$\{\sigma_{i}(\alpha)\mid i=1,\dots,n\}=\{\alpha\}$ and
$[E'(\alpha):E']=1$, so $\alpha\in E'$ which is a contradiction.

By Proposition~\ref{prop:14.7}, $\Phi(E)$ is closed [by part (1) of
  the proposition] and $\Gal(\Omega/\Omega^{\Phi(E)})=\Gal(\Omega/E)$
[by part (2)], i.e., $\Omega^{\Phi(E)}=E$. Then we have
\begin{equation}
\Psi\circ\Phi(E)=\Omega^{\Phi(E)}=E.
\end{equation}
On the other hand,
\begin{subequations}
  \begin{align}
\Phi\circ\Psi(H) &=\Gal(\Omega/\Omega^{H})\\
&=H
  \end{align}
\end{subequations}
by Proposition~\ref{prop:14.7}~(2).
\end{proof}

\begin{proposition}\label{prop:14.9}
Let $\Omega/F$ be a Galois extension, let $G=\Gal(\Omega/F)$,
and let $N\normalsubgroup G$ be a normal open subgroup.
Then $\Omega^{N}/F$ is a finite Galois extension.
\end{proposition}

\begin{proof}
We will first prove $\Omega^{N}/F$ is a Galois extension. We see it is
separable by Proposition~\ref{prop:4.3}. Now, to prove it is
normal. Let $\sigma\in\hom_{F}(\Omega^{N},\closure{F})$. Let its
extension be $\sigma'\in\Gal(\Omega/F)$ be such that
$\sigma'|_{\Omega^{N}}=\sigma$ by Proposition~\ref{prop:2.8}. Since
$N$ is normal, for any $\alpha\in\Omega^{N}$ we have
\begin{equation}
(\sigma')^{-1}N\sigma'(\alpha)=N(\alpha)=\alpha,
\end{equation}
i.e., $N\sigma(\alpha)=\sigma(\alpha)$. Hence $\Omega^{N}/F$ is
normal and, in particular, Galois.

We claim $\Omega^{N}/F$ is finite. Since $N$ is open by
Corollary~\ref{cor:14.4}, there exists finite Galois extension $E/F$
such that $\Gal(\Omega/E)\subset N$. (Note: $\Gal(\Omega/E)$ is an
open neighborhood of 1 in Galois group.) Then by
Proposition~\ref{prop:14.2}, $N$ and $\Gal(\Omega/E)$ are closed. Then
by Lemma~\ref{lemma:14.5}, we have
\begin{equation}
|G/N|\leq|G/\Gal(\Omega/E)|=|\Gal(E/F)|<\infty.
\end{equation}
If $\Omega^{N}/F$ is infinite, then there exists $\alpha\in\Omega^{N}$
such that
\begin{equation}\label{eq:contradiction:prop:14.9}
m:=[F(\alpha):F]>|G/N|.
\end{equation}
(In fact, every subextension of $\Omega^{N}/F$ is separable, so by
Theorem~\ref{thm:4.5} such an $\alpha$ exists.) Denote
\begin{equation}
\hom_{F}(F(\alpha),\closure{F})=\{\sigma_{1},\dots,\sigma_{m}\}
\end{equation}
and its extension be $\{\sigma'_{1},\dots,\sigma'_{m}\}\subset\Gal(\Omega/F)$.
If $\sigma'_{i}N=\sigma'_{j}N$, then $\sigma'_{i}(\alpha)=\sigma'_{j}(\alpha)$,
and therefore $\sigma_{i}=\sigma_{j}$. So $|G/N|\geq m$, which
contradicts the assumption in Equation~\eqref{eq:contradiction:prop:14.9}.
Hence $\Omega^{N}/F$ is finite.
\end{proof}