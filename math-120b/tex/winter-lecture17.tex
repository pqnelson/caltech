%%
%% winter-lecture17.tex
%% 
%% Made by Alex Nelson <pqnelson@gmail.com>
%% Login   <alex@lisp>
%% 
%% Started on  2026-02-12T08:34:43-0800
%% Last update 2026-02-12T08:34:43-0800
%% 

\lecture{}

We now are shifting gears towards infinite Galois extensions.

\begin{definition}
Let $(I,\leq)$ be a directed poset (i.e., for all $a,b\in I$, there
exists some $c\in I$ such that $c\geq a$ and $c\geq b$). Let
$\{G_{i}\}_{i\in I}$ be a family of groups,
let $\{f_{ij}\colon G_{i}\to G_{j}\}_{i,j\in I}$ be a family of group morphisms.
We call $(G_{i},f_{ij})_{i,j\in I}$ an \define{Inverse System} if
\begin{enumerate}
\item for all $i\in I$, $f_{i,i}=\id_{G_{i}}$
\item for all $i,j,k\in I$, $f_{i,k}=f_{j,k}\circ f_{i,j}$
\end{enumerate}
\end{definition}

\begin{definition}
The \define{Inverse Limit} of an inverse system
$(G_{i},f_{ij})_{i,j\in I}$ is a group
\begin{equation}
G:=\varprojlim_{i\in I}{G_{i}}:=\{(g_{i})_{i\in I}\in\prod_{i\in I}G_{i}\mid\forall i,j\in I\ldotp i\geq j\implies g_{j}=f_{ij}(g_{i})\}
\end{equation}

Note: for any $g_{i}$ and $g_{j}$ appearing in $\vec{g}\in G$, there exists a
$k\in I$ such that $k\geq i$ and $k\geq j$ such that $g_{k}\to g_{i}$,
$g_{k}\to g_{j}$---i.e., $f_{ki}g_{k}=g_{i}$ and $f_{kj}g_{k}=g_{j}$. 
\end{definition}

\begin{node}
The inverse limit of an inverse system of groups $G=\varprojlim_{i\in I}{G_{i}}$
is a group.
\end{node}

\begin{proof}
We see $f_{ij}(g_{i}g'_{i})=f_{ij}(g_{i})f_{ij}(g'_{i})=g_{j}g'_{j}$.
Similar reasoning shows there is an inverse operator and an identity element.
\end{proof}

\begin{definition}
Let $(G_{i},f_{ij})_{i,j\in I}$ be an inverse system, let $G=\varprojlim_{i\in I}{G_{i}}$.
For any $i\in I$, we define the \define{Canonical Projection}
to be the group morphism $\pi_{i}\colon G\to G_{i}$ sending
$\pi_{i}\bigl((g_{j})_{j\in I}\bigr)=g_{i}$.

Since $G\subset\prod_{i\in I}G_{i}$, these are just the usual projection
morphisms restricted to $G$.
\end{definition}

\begin{node}[Review of point-set topology]
A brief review of topology. Note that the ``minimal topology'' is used
instead of ``initial topology''.
\end{node}


\begin{definition}
Let $(G_{i},f_{ij})_{i\in I}$ be an inverse system of groups.
The \define{Krull Topology} on $G=\varprojlim_{i\in I}{G_{i}}$ is the
initial topology with respect to the projection morphisms, i.e., the
coarsest topology such that $\pi_{i}\colon G\to G_{i}$ is continuous
for each $i\in I$.
\end{definition}

\begin{proposition}
Let $(G_{i},f_{ij})_{i\in I}$ be an inverse system of groups.
A basis of the Krull topology on $G=\varprojlim_{i\in I}{G_{i}}$ is
\begin{equation*}
\mathcal{B}:=\{\pi^{-1}_{i}(g_{i})\mid i\in I, g_{i}\in G_{i}\}.
\end{equation*}
\end{proposition}

\begin{proof}
In fact, the open subsets of $G$ contains
[the collection of open subsets generated by $\mathcal{B}$] $\mathcal{O}'=\{\bigcup_{i}\pi^{-1}_{i}(S_{i})\mid i\in I, S_{i}\subset G_{i}\}$.
Therefore it suffices to show $\mathcal{O}'$ defines the topology on
$G$. The first two axioms are obviously satisfied ($G\in\mathcal{O}'$
and $\emptyset\in\mathcal{O}'$ and it's closed under arbitrary unions).
We just need to check it is closed under intersecting two elements,
\begin{equation}
\left(\bigcup_{i}\pi^{-1}_{i}(S_{i})\right)\cap\left(\bigcup_{i}\pi^{-1}_{i}(S'_{i})\right)
=\bigcup_{i}\left(\pi^{-1}_{i}(S_{i})\cap\pi^{-1}_{i}(S'_{i})\right)
\end{equation}
By definition of an inverse limit of an inverse system, for any
$i,j\in I$ there exists $k\in I$ such that $k\geq i$ and $k\geq j$ and
$g_{k}\to g_{i}$ and $g_{k}\to g_{j}$. This means
$\pi^{-1}_{i}(S_{i})\cap\pi^{-1}_{j}(S'_{j})$ is generated by
\begin{equation}
\{\pi^{-1}_{k}(g_{k})\mid g_{i}\in S_{i},g_{j}\in S'_{j}\}\subset\mathcal{B}.
\end{equation}
This means it is generated by $\mathcal{B}$, and therefore belongs to $\mathcal{O}'$.
\end{proof}

\begin{proposition}\label{prop:14.1}
Let $(G_{i},f_{ij})_{i\in I}$ be an inverse system of groups.
The Krull topology on $G=\varprojlim_{i\in I}{G_{i}}$ turns $G$ into a
topological group.
\end{proposition}

\begin{proof}
Let us check the binary operator $G\times G\to G$ is continuous (we
omit the proof $G\to G$, $g\mapsto g^{-1}$ is continuous, since the
argument is similar). It suffices to check for the basis
$\{\pi^{-1}(g_{n})\mid n\in I, g_{n}\in G_{n}\}$, i.e.,
\begin{equation}
\mathcal{O}(g_{n}):=\pi^{-1}(g_{n})=\{\bigl((a_{i}),(b_{i})\bigr)\mid a_{n}b_{n}=g_{n}\}.
\end{equation}
Then by definition of the product topology on $G\times G$ (the
projection onto factors are continuous and it is the coarsest such
topology), write
\begin{subequations}
  \begin{align}
\mathcal{O}_{a_{n}}=\pi^{-1}_{1}(\pi^{-1}_{n}(a_{n}))=\{\bigl((a'_{i}),(b_{i})\bigr)\mid a'_{n}=a_{n}\}\\
\intertext{and}
\mathcal{O}'_{a_{n}^{-1}g_{n}}=\pi^{-1}_{2}(\pi^{-1}_{n}(a_{n}))=\{\bigl((a'_{i}),(b_{i})\bigr)\mid b_{n}=a_{n}^{-1}g_{n}\}
  \end{align}
\end{subequations}
are both open in $G\times G$ for a \emph{fixed} $a_{n}\in G_{n}$.
So if we take the union of their intersections (which is the union of
open sets, hence open), we find
\begin{subequations}
  \begin{align}
\bigcup_{a_{n}\in G_{n}}(\mathcal{O}_{a_{n}}\cap\mathcal{O}'_{a_{n}^{-1}g_{n}})
&=\{\bigl((a_{i}),(b_{i})\bigr)\mid a_{n}b_{n}=g_{n}\}\\
&=\mathcal{O}(g_{n}),
  \end{align}
\end{subequations}
which proves the product is continuous.
\end{proof}

\begin{remark}
We will henceforth just consider $\varprojlim_{i\in I}G_{i}$ is a topological
group equipped with the Krull topology.
\end{remark}

\begin{proposition}\label{prop:14.2}
Let $(G_{i},f_{ij})_{i\in I}$ be an inverse system of groups,
let $G=\varprojlim_{i\in I}{G_{i}}$ (viewed as a topological group).
Then every open (respectively, ``finite index ($[G:H]<\infty$) closed
[as a subset of a topological space]'') subgroup $H\subgroup G$ is
closed (resp., open).
\end{proposition}

\begin{proof}
\begin{enumerate}
\item Let $H\subgroup G$ be an open subgroup. We can write
  $\{g_{\lambda}=(g_{\lambda,i})_{i\in I}\}_{\lambda}\in G$ a set of
  representatives of the cosets $G/H=\{gH\}$. Since $H$ is open,
\begin{equation}
H=\bigcup_{i}\pi^{-1}_{i}(S_{i})
\end{equation}
for some $S_{i}\subset G_{i}$, and thus
\begin{equation}
g_{\lambda}H=\bigcup_{i\in I}\pi^{-1}_{i}(g_{\lambda}S_{i})
\end{equation}
is also open. Then
\begin{equation}
H=G\setminus\left(\underbrace{\bigcup_{i\in I}\overbrace{g_{\lambda}H}^{\text{open}}}_{\text{open}}\right)
\end{equation}
is closed.
\item If $H\subgroup G$ is a finite-index closed subgroup, then in the
  same manner
\begin{equation}
H=G\setminus\left(\underbrace{\bigcup_{i\in I}\overbrace{g_{\lambda}H}^{\text{closed}}}_{\text{closed}}\right)
\end{equation}
which is open.\qedhere
\end{enumerate}
\end{proof}

\begin{proposition}
Let $\Omega/F$ be a (possibly infinite) Galois extension. Then the set
of all finite Galois extensions $E/F$ (which are subextensions, i.e.,
$\Omega/E/F$) forms a directed poset by inclusion (i.e., $E_{i}\geq E_{j}\iff E_{i}/E_{j}$)\footnote{Let
$\Omega/F$ be an algebraic extension, let $E/F$ be a finite Galois
subextension $E\subset\Omega$, let $\alpha\in\Omega\setminus E$. Then
$E(\alpha)/F$ is another finite Galois extension, and
$E(\alpha)\supset E$. Hence they form a poset.}
and this turns the Galois groups $\{\Gal(E/F) \mid \Omega/E/F, E/F \mbox{ finite Galois}\}$
into an inverse system of groups. Then
\begin{equation}
\Gal(\Omega/F)\iso\varprojlim\Gal(E/F)
\end{equation}
as groups.
\end{proposition}


\begin{proof}
The first half follows from Theorem~\ref{thm:7.1}~\ref{item:thm:7.1:item-5},
in particular,
\begin{equation}
\Gal(E/F)\iso\frac{\Gal(\Omega/F)}{\Gal(\Omega/E)},
\end{equation}
and for an intermediate Galois subextension $E/E'/F$ we also have
\begin{equation}
\Gal(E'/F)\iso\frac{\Gal(\Omega/F)}{\Gal(\Omega/E')},
\end{equation}
so we have a surjection $\Gal(E'/F)\onto\Gal(E/F)$. This can be
iterated to other finite subextensions, which gives us the system
\begin{equation}
\dots\gets\Gal(E/F)\gets\Gal(E'/F)\gets\dots.
\end{equation}
Now, we have a group morphism
\begin{equation}
\begin{split}
\varphi\colon&\Gal(\Omega/F)\to\varprojlim\Gal(E/F)\\
&\sigma\mapsto(\sigma|_{E})_{E}
\end{split}
\end{equation}
is bijective.

\textsc{Claim 1: Injectivity}. If $\sigma\in\ker(\varphi)$, then
$\sigma|_{E}=\id_{E}$ for all finite Galois extensions $E/F$. But
since $\Omega/F$ is algebraic, every $\alpha\in\Omega$ is in some
finite Galois extension $E/F$; i.e., $\Omega=\bigcup E$. Hence it is injective.

\textsc{Claim 2: Surjectivity}. For $(\sigma_{E})_{E}\in\varprojlim\Gal(E/F)$,
we can define $\sigma\in\Gal(\Omega/F)$ by $\sigma|_{E}:=\sigma_{E}$
(well-definedness follows from inverse limits of inverse
systems). Hence surjectivity, and we obtain the isomorphism $\varphi$
as desired.
\end{proof}