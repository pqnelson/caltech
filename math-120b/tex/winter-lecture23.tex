%%
%% winter-lecture23.tex
%% 
%% Made by Alex Nelson <pqnelson@gmail.com>
%% Login   <alex@lisp>
%% 
%% Started on  2026-02-26T08:19:44-0800
%% Last update 2026-02-26T08:19:44-0800
%% 

\lecture{}

\begin{definition}
Let $B$ be a commutative ring, let $A\subset B$ be a subring of $B$.
\begin{enumerate}
\item We say an element $\alpha\in B$ is \define{Integral} over $A$ if there
  exists a monic polynomial $P\in A[x]$ such that $P(\alpha)=0$.
\item We say $B$ is \define{Integral} if every $\alpha\in B$ is
  integral over $A$.
\item The \define{Integral Closure} of $A$ over $B$ is the set of all
  elements of $B$ which are integral over $A$.
\end{enumerate}
\end{definition}

\begin{note}
When $A\subset B$ is a subring of a commutative ring $B$, we have the
following are equivalent:
\begin{enumerate}
\item $\alpha\in B$ is integral
\item $A[\alpha]$ is a finite $A$-module
\item There exists a finite $A$-module $C$ such that $A[\alpha]\subset C$
\end{enumerate}
\end{note}

\begin{node}
In particular, if $\alpha,\beta\in B$ are integral over $A\subset B$,
then so are $\alpha+\beta$ and $\alpha\beta$ both integral over $A$.
\end{node}

\begin{proof}
We see $A(\alpha,\beta)$ is a finite $A$-module and both
$A(\alpha+\beta)\subset A[\alpha,\beta]$ and $A(\alpha\beta)\subset A[\alpha,\beta]$.
Hence the result.
\end{proof}

\begin{node}[Intuition]
We should think of ``integral'' for ring extensions as analogous to
``algebraic'' for field extensions.
\end{node}

\begin{definition}
Let $A$ be an integral domain and $F=\Frac(A)$ its field of fractions.
We say $A$ is \define{Integrally Closed} if the integral closure of
$A$ in $F$ is $A$ itself.
\end{definition}

\begin{proposition}\label{prop:17.1}
The class of integral extensions is distinguished.
\end{proposition}

\begin{proof}
\textsc{Tower property}: We want to prove, given commutative rings
$A\subset B\subset C$ that $C$ is integral over $A$ if and only if $C$
is integral over $B$ and $B$ is integral over $A$.

\forwardproof\ Assume $C$ is integral over $A$. Then $C$ is integral
over $B$, and also $B$ is integral over $A$ (both by definition of integral).

\backwardproof\ Assume $C$ is integral over $B$, and assume $B$ is
integral over $A$. Let $c\in C$ be integral over $B$. Then there
exists a monic polynomial
\begin{equation}
f(x)=x^{n}+b_{n-1}x^{n-1}+\dots+b_{1}x+b_{0}\in B[x]
\end{equation}
such that $f(c)=0$ and the coefficients belong to $b_{i}\in B$. So $c$
is integral over $A[b_{0},\dots,b_{n-1}]$. But since $B$ is integral
over $A$, this means $A[b_{0},\dots,b_{n-1}]$ is integral over
$A$. Therefore $A[c,b_{0},\dots,b_{n-1}]$ is integral over $A$. Hence
$c$ is integral over $A$.

\textsc{Lifting property}: Let $B$ be integral over $A$, let $R$ be
any commutative ring containing $A\subset R$ as a subring. We will
prove $RB$ is integral over $R$. For any arbitrary linear combination
\begin{equation}
\sum_{i}r_{i}b_{i}\in RB
\end{equation}
where each $b_{i}$ is integral over $A$, then we see that $b_{i}$ is
integral over $R$. Then $\sum_{i}r_{i}b_{i}\in R[b_{i}]$ is integral
over $R$. Since the linear combination was arbitrary, this means every
element of $RB$ is integral over $R$.
\end{proof}

\begin{corollary}\label{cor:17.2}
Let $A$ be integrally closed, let $k=\Frac(A)$ its field of fractions,
and let $L/k$ be an algebraic extension. Let $B$ be the algebraic
closure of $A$ in $L$. Then $B$ is integrally closed and $L=\Frac(B)$.
\end{corollary}

\begin{proof}
We draw the current ``state of affairs'' (solid lines indicate one
ring is integral over another, dotted lines indicate one ring is a
subring [possibly integral but not proven] of another):
\begin{equation}
\vcenter{\xymatrix{& L\ar@{..}[ddl]\ar@{..}[ddr]\ar@{..}[drr] & & & \\
  &   &   & \Frac(B)\ar@{-}[dl]\ar@{-}[dr] & \\
k\ar@{..}[dr] &   & B\ar@{..}[dl]\ar@{..}[rr] &     & C\ar@{..}[dlll] \\
  & A &   &     & }}
\end{equation}
Since $B$ is integral over $A$, then by Proposition~\ref{prop:17.1}
the integral closure $C$ of $B$ in $\Frac(B)$ is integral over $A$.
\begin{equation}
\vcenter{\xymatrix{& L\ar@{..}[ddl]\ar@{..}[ddr]\ar@{..}[drr] & & & \\
  &   &   & \Frac(B)\ar@{-}[dl]\ar@{-}[dr] & \\
k\ar@{..}[dr] &   & B\ar@{..}[dl]\ar@{-}[rr] &     & C\ar@{..}[dlll] \\
  & A &   &     & }}
\end{equation}
Since $\Frac(B)\subset L$ and the definition of $B$ shows $B=C$,
therefore $B$ is integrally closed. Since $L/k$ is algebraic, any
$\gamma\in L$ has a minimal polynomial
\begin{equation}
P(\gamma)=\gamma^{n}+c_{n-1}\gamma^{n-1}+\dots+c_{1}\gamma+c_{0}=0,
\end{equation}
where the coefficients $c_{i}\in k$. Then there exists an $a\in A$
such that
\begin{equation}
0=a^{n}P(\gamma)=(a\gamma)^{n}+a^{1}c_{n-1}(a\gamma)^{n-1}+\dots+a^{n-1}c_{1}(a\gamma)+a^{n}c_{0}=\widetilde{P}(a\gamma)
\end{equation}
so each coefficient $a^{n-i}c_{i}\in A$ which gives us a monic polynomial
$\widetilde{P}\in A[x]$. Then $a\gamma$ is integral over $A$ and
$\gamma\in\Frac(B)$. (Therefore $L=\Frac(B)$.)
\end{proof}

\begin{definition}
Let $E/F$ be a separable extension. A \define{Galois Closure} $L$ of
$E$ is the smallest Galois extension $L/E$.
\end{definition}

\begin{proposition}\label{prop:17.3}
Let $A$ be integrally closed and Noetherian. Let $L$ bea finite
separable extension of $k=\Frac(A)$. Let $B$ be the integral closure
of $A$ in $L$. Then $B$ is a finite $A$-module.
\end{proposition}

\begin{proof}
%Suffices to prove $L/k$ is finite Galois.
Let $C$ be the integral closure of $A$ (and thus $B$) in the Galois
closure of $L$. If $C$ is a finite $A$-module, then $B$ is a finite
$A$-module [Proof: $A$ is Noetherian, then $C$ is a finite Noetherian
$A$-module, and therefore every submodule of $C$ is finite.]
Therefore we may assume $L//k$ is finite Galois.

Then by Theorem~\ref{thm:4.5}, there exists an $\alpha\in L$ such that
$L=k(\alpha)$. By clearing the denominators of the minimal polynomial
of $\alpha$, we may assume $\alpha$ is integral over $A$ and thus
$\alpha\in B$.

For $\beta\in B\subset L=k(\alpha)$, we have
\begin{equation}
\beta=\sum^{n-1}_{j=0}c_{j}\alpha^{j}
\end{equation}
where $c_{j}\in k$ and $n=[L:k]=|\Gal(L/k)|$. We suppose
\begin{equation}
\Gal(L/k)=\{\sigma_{1},\dots,\sigma_{n}\}.
\end{equation}
By applying $\sigma_{i}\in\Gal(L/k)$ we have
\begin{equation}
\sigma_{i}(\beta)=\sum_{j}c_{j}\sigma_{i}(\alpha)^{j},
\end{equation}
which is to say
\begin{equation}
\underbrace{\begin{pmatrix}
\sigma_{1}(1) & \sigma_{1}(\alpha) & \dots & \sigma_{1}(\alpha)^{n-1}\\
\sigma_{2}(1) & \sigma_{2}(\alpha) & \dots & \sigma_{2}(\alpha)^{n-1}\\
\vdots & \vdots & \ddots & \vdots\\
\sigma_{n}(1) & \sigma_{n}(\alpha) & \dots & \sigma_{n}(\alpha)^{n-1}
\end{pmatrix}}_{=(\sigma_{i}(\alpha)^{j})=\text{Vandermonde matrix}}
\begin{pmatrix}
c_{1}\\ \vdots\\ c_{n}
\end{pmatrix}
=
\begin{pmatrix}
\sigma_{1}(\beta)\\ \vdots\\ \sigma_{n}(\beta)
\end{pmatrix}
\end{equation}
Then we multiply both sides by the matrix inverse of the Vandermonde
matrix to give us
\begin{subequations}
  \begin{align}
c_{k}
&=\sum_{\ell}(\sigma_{i}(\alpha)^{j})^{-1}_{k,\ell}b_{\ell}\\
\intertext{then using Cramer's rule, we write the matrix of cofactors
  as $(\Delta_{i,j})$ to give us:}
&=\frac{1}{\prod_{1\leq i<j\leq n}(\sigma_{j}(\alpha)-\sigma_{i}(\alpha))}(\Delta_{i,j})\begin{pmatrix}
\sigma_{1}(\beta)\\ \vdots\\ \sigma_{n}(\beta)
\end{pmatrix}
  \end{align}
\end{subequations}
Since $\alpha,\beta\in B$ and $B$ is an integral closure of $A$, then
$\sigma_{i}(\alpha)\in B$ and $\sigma_{i}(\beta)\in B$ for all $i=1,\dots,n$,
and also the Vandermonde determinant belongs to $B$. Then we have
\begin{equation}
\left(\prod(\sigma_{i}(\alpha)-\sigma_{j}(\alpha))\right)c_{k}\in B.
\end{equation}
Moreover, if we multiply $c_{k}$ by this product again, we have
$Dc_{k}\in B$ where
\begin{equation}
D=\left(\prod(\sigma_{i}(\alpha)-\sigma_{j}(\alpha))\right)^{2}
\end{equation}
is the \define{Discriminant} of the minimal polynomial of $\alpha$
over $k$, which is contained in $k$. Therefore $D\in k\cap B$.
Therefore $Dc_{k}\in k\cap B=A$. Then
\begin{equation}
D\beta=\sum_{j}c_{j}\alpha^{j}\in A(\alpha),
\end{equation}
hence $DB\subset A[\alpha]$. Since $A[\alpha]$ is a finite $k$-module
and $A$ is Noetherian, then
\begin{equation}
\begin{split}
B\to DB\\
x\mapsto Dx
\end{split}
\end{equation}
is an isomorphism whose inverse is given by $y\mapsto D^{-1}y$. Then
$DB$ is also a finite $A$-module. Hence the claim.
\end{proof}

\begin{proposition}\label{prop:17.4}
If $A$ is an integral domain and a unique factorization domain, then
$A$ is integrally closed.
\end{proposition}

\begin{proof}
Suppose that we take $r/s\in\Frac(A)$ to be an integral element over
$A$, and suppose further that $\gcd(r,s)=1$. Then
\begin{equation}
(r/s)^{n}+a_{n-1}(r/s)^{n-1}+\dots+a_{0}=0
\end{equation}
where each $a_{i}\in A$ for $i=0,\dots,n-1$. Then multiplying by
$s^{n}$:
\begin{equation}
r^{n}+(a_{n-1}s)r^{n-1}+\dots+(a_{1}s^{n-1})r+a_{0}s^{n}=0
\end{equation}
For a prime factor $p$ of $s$, we have $p$ divides $(a_{n-1}s)r^{n-1}+\dots+(a_{1}s^{n-1})r+a_{0}s^{n}$
so therefore $p$ divides $r^{n}$. Therefore $r$ must be divisible by
$p$. But since $\gcd(r,s)=1$, we must have $(r/s)\in A$.
\end{proof}

\begin{lemma}\label{lemma:17.5}
If $B$ is an integral extension over $A$, and if $Q\in\Spec(B)$ is a
prime ideal which lies above $P\in\Spec(A)$, then
\begin{enumerate}
\item $B/Q$ is integral over $A/P$, and
\item $B_{Q}$ is integral over $A_{P}$.
\end{enumerate}
\end{lemma}

\begin{proof}
For a representative in $B$ of an element of $B/Q$ taking the minimal
polynomial (mod $P$) shows the integrality over $A/P$.

For $(b/s)\in B_{Q}$, since $b\in B$ and $s\in Q\subset B$, there
exists an $n\in\NN$ sufficiently large $n\gg0$ such that $b^{n}\in A$
and $s^{n}\in A\cap(B\setminus Q)=A\setminus P$, and the claim follows
(we have a finite module).
\end{proof}