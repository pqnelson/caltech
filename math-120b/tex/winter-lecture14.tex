%%
%% winter-lecture14.tex
%% 
%% Made by Alex Nelson <pqnelson@gmail.com>
%% Login   <alex@lisp>
%% 
%% Started on  2026-02-05T09:38:06-0800
%% Last update 2026-02-05T09:38:06-0800
%% 

\lecture{}

\begin{theorem}\label{thm:10.4}
Let $E/F$ be a finite Galois extension, let $n=[E:F]=|G|$ be its degree
where $G=\Gal(E/F)=\{\sigma_{1},\dots,\sigma_{n}\}$ is its Galois group. 
If $|F|=\infty$, then there exists a $w\in E$ such that
$\{\sigma_{1}(w),\dots,\sigma_{n}(w)\}$ is a basis of $E$ as a vector
space over $F$.
\end{theorem}

\begin{proof}
By Theorem~\ref{thm:4.5}, there is some $\alpha\in E$ such that $E=F(\alpha)$.
We may reindex the $\sigma_{i}$ such that $\sigma_{1}=\id$. We write
the minimal polynomial of $\alpha$ as
\begin{equation}
f(x)=\prod^{n}_{i=1}(x-\alpha_{i})\in F[x]
\end{equation}
where $\alpha_{i}=\sigma_{i}(\alpha)$. (Note that if $i\neq j$ then $\alpha_{i}\neq\alpha_{j}$.)

Now, using the formal derivative $f'(x)$ of the minimal polynomial of
$\alpha$, we write
\begin{equation}
g_{i}(x)=\frac{f(x)}{(x-\alpha_{i})f'(x)}=\prod_{j\neq i}\frac{x-\alpha_{j}}{\alpha_{i}-\alpha_{j}}.
\end{equation}
Since $G$ is a group, we will denote the index $(i\cdot j)$ for the
index of the element such that
\begin{equation}
\sigma_{i}^{-1}\sigma_{j}=\sigma_{(i\cdot j)}.
\end{equation}
This is because there is some $k\in\{1,\dots,n\}$ such that
$\sigma_{i}^{-1}\sigma_{j}=\sigma_{k}$, so we just write $k=(i\cdot j)$.
Using this, let us define the $n\times n$ matrix
\begin{equation}
A(x)=(a_{i,j}(x))=(g_{(i\cdot j)}(x)),
\end{equation}
and its determinant
\begin{equation}
D(x):=\det(A(x))=\sum_{\tau\in S_{n}}\sgn(\tau)\prod_{i=1}^{n}g_{(i\cdot\tau(i))}(x).
\end{equation}
Then, as
\begin{equation}
g_{ij}(\alpha)=\delta_{i,j}=\begin{cases}1 & \mbox{if }i=j\\
0 & \mbox{otherwise},
\end{cases}
\end{equation}
we have $A(\alpha)=I$ is the identity matrix, so its determinant is
unity $D(\alpha)=1\neq0$. In particular, the determinant is always
nonzero $D\neq0$, and thus has only finitely many roots in $E$.

Now, using the hypothesis $|F|=\infty$, there exists some $b\in E$
such that $D(b)\neq0$. Let us show these $\{\beta_{j}:=g_{j}(b)\}_{j=1,\dots,n}$
forms a basis for $E$ as a vector space over $F$. Then $w := g_{1}(b)$.
If
\begin{equation}
\sum^{n}_{j=1}a_{j}\beta_{j}=0
\end{equation}
for some $a_{1}$, \dots, $a_{n}$, then applying $\sigma_{i}^{-1}$ to
it gives us
\begin{equation}
0=\sum^{n}_{j=1}\sigma_{i}^{-1}g_{j}(b)a_{j}=A(b)\vec{a},
\end{equation}
where we just rewrite it in matrix notation. But since $D(b)\neq0$,
this means $A(b)$ is an invertible matrix, and therefore $\vec{a}=\vec{0}$.
Hence the result.
\end{proof}

\subsection{Cyclic Extensions}

\begin{definition}
A Galois extension $E/F$ is called \define{Cyclic} if its Galois group
$\Gal(E/F)$ is a cyclic group.
\end{definition}

\begin{theorem}[Hilbert's theorem 90]\label{thm:11.1}
Let $E/F$ be a cyclic Galois extension of degree $n$ (i.e.,
$G=\Gal(E/F)\iso\ZZ/n\ZZ$). Let $\sigma\in G$ be a generator. Then the
following all hold:
\begin{enumerate}
\item\label{item:thm:11.1:item:1} There exists some $\alpha\in E$ such that: $N_{E/F}(\alpha)=1$ if
  and only if there exists some $\beta\in E$ such that
  $\alpha=\beta/\sigma(\beta)$;
\item\label{item:thm:11.1:item:2} For any $\alpha\in E$, we have $\Tr_{E/F}(\alpha)=0$ if and only
  if there exists a $\beta\in E$ such that $\alpha=\beta-\sigma(\beta)$.
\end{enumerate}
\end{theorem}

(This seems to be \emph{literally} the 90th Theorem in David Hilbert's
\textit{Die Theorie der algebraischen Zahlk\"{o}rper} (1897), which is
the origin of the name.)

\begin{proof}
\backwardproof\ $N_{E/F}$ and $\Tr_{E/F}$ are preserved by the group
action, which gives both results.

(1)~\forwardproof\ Assume that
\begin{equation}
N_{E/F}(\alpha)=\prod^{n}_{i=0}\sigma^{i}(\alpha)=1.
\end{equation}
By Lemma~\ref{lemma:7.2}, the $\sigma^{i}$ are linearly independent,
and so
\begin{equation}\label{eq:math120b:winter:lec14:hilbert90:def-of-f}
f:=\sum^{n-1}_{i=0}\left(\prod^{i-1}_{j=0}\sigma^{j}(\alpha)\right)\sigma^{i}=\id+\alpha\sigma+\alpha\sigma(\alpha)\sigma^{2}+\dots\neq0
\end{equation}

Then there exists some $\gamma\in E$ such that
\begin{equation}
\beta=f(\gamma)\neq0.
\end{equation}
Then
\begin{subequations}
  \begin{align}
\alpha\sigma(\beta)
&=\alpha\sigma(\gamma)+\alpha\sigma(\alpha)\sigma^{2}(\gamma)+\dots+\underbrace{\alpha\left(\prod^{n}_{j=1}\sigma^{j}(\alpha)\right)}_{=N_{E/F}(\alpha)=1}\underbrace{\sigma^{n}(\gamma)}_{=\gamma}\\
&=\beta
  \end{align}
\end{subequations}
In particular, since we just proved $\alpha\sigma(\beta)=\beta$, then
we find $\alpha=\beta/\sigma(\beta)$ as desired.

(2) \forwardproof\ For this, we take
\begin{equation}
\beta=\frac{1}{\Tr_{E/F}(\gamma)}f(\gamma)
\end{equation}
where we use $f$ from
Equation~\eqref{eq:math120b:winter:lec14:hilbert90:def-of-f}. By
similar reasoning, the result follows.
\end{proof}

\begin{proposition}\label{prop:11.2}
Let $\zeta_{n}\in F$ (so $F$ contains the $n^{\text{th}}$ root of
unity $\zeta_{n}$) for some $n$ such that $\Char(F)\ndivides n$. Then
the following all hold:
\begin{enumerate}
\item\label{item:prop:11.2:1} If there exists some $\alpha\in E$ such that $E=F(\alpha)$
  (where $\alpha^{n}\in F$ but $\alpha^{m}\notin F$ for all $0<m<n$),
  then $E/F$ is cyclic of degree $n$.
\item\label{item:prop:11.2:2} If $E/F$ is cyclic of degree $n$, then there exists an
  $\alpha\in E$ such that $E=F(\alpha)$ and $\alpha^{n}\in F$.
\end{enumerate}
\end{proposition}

\begin{proof}
(1) For $i=1$, \dots, $n$, we have $\zeta_{n}^{i}\in F(\alpha)$ are
distinct roots of $x^{n}-\alpha^{n}\in F[x]$. Therefore $F(\alpha)$
is a splitting field for the separable polynomial $x^{n}-\alpha^{n}$,
and in particular $F(\alpha)/F$ is a finite Galois extension.

For $\sigma\in G:=\Gal(E/F)$, we see $\sigma(\alpha)=\zeta_{n}^{i}\alpha$
for some $i$, and thus we have a map (by Proposition~\ref{prop:9.1}):
\begin{equation}
\begin{split}
&G\to\mu_{n}\\
&\sigma\mapsto\frac{\sigma(\alpha)}{\alpha}=\zeta_{n}^{i}.
\end{split}
\end{equation}
This is a group morphism. If $\sigma$ is in its kernel, then
$\sigma(\alpha)=\alpha$ and therefore $\sigma=\id$ (i.e., the kernel
is trivial). Hence this group morphism is injective. Therefore
\begin{equation}
G\iso\ZZ/d\ZZ
\end{equation}
for some $d\divides n$. But
\begin{equation}
(\sigma(\alpha)/\alpha)^{d}=1,
\end{equation}
which is to say: $\sigma(\alpha^{d})=\alpha^{d}$ so $\alpha^{d}\in E^{G}$
and by Theorem~\ref{prop:7.5}~\ref{item:prop:7.5:3} we have $E^{G}=F$. This means that
$d=n$ by the hypothesis that $\alpha^{n}\in F$ but $\alpha^{m}\notin F$
for any $1<m<n$.

(2) Let $\sigma$ be a generator of $G=\Gal(E/F)\iso\ZZ/n\ZZ$. Since
\begin{subequations}
  \begin{align}
N_{E/F}(\zeta_{n}^{-1})
&= \prod^{n}_{i=0}\sigma^{i}(\zeta_{n}^{-1})\\
&=(\zeta^{-1}_{n})^{n}=1,
  \end{align}
\end{subequations}
we have by Theorem~\ref{thm:11.1}~\ref{item:thm:11.1:item:1} there
exists $\alpha\in E$ such that $\sigma(\alpha)=\zeta_{n}\alpha$. Then
\begin{subequations}
  \begin{align}
\sigma(\alpha^{n})
&=\bigl(\sigma(\alpha)\bigr)^{n}\\
&=(\zeta_{n}\alpha)^{n}\\
&=\alpha^{n}
  \end{align}
\end{subequations}
and thus $\alpha^{n}\in E^{G}=F$. Hence $x^{n}-\alpha^{n}\in F[x]$,
and it has distinct roots $\zeta^{i}_{n}\alpha\in F(\alpha)$. That is
to say, $F(\alpha)/F$ is a Galois extension. So we have the following
summary of which extensions are Galois (solid lines) and which we have
no information about (dotted lines):
\begin{equation}
\vcenter{\xymatrix{&E\ar@/_2pc/@{-}[dd]_{\text{galois}}\ar@{-}[d]^{\text{galois}}\\
      &F(\alpha)\ar@{..}[d]\\
    &F}}
\end{equation}
(Note that
$\Gal(E/F(\alpha))$ is a normal subgroup of $\Gal(E/F)$ by
Theorem~\ref{thm:7.1}.) We have
\begin{equation}
\ZZ/n\ZZ\iso\Gal(F(\alpha)/F)\iso\frac{\Gal(E/F)}{\Gal(E/F(\alpha))},
\end{equation}
but
\begin{equation}
\Gal(E/F)\iso\ZZ/n\ZZ,
\end{equation}
which means that $\Gal(E/F(\alpha))$ must be trivial. Hence $E=F(\alpha)$.
\end{proof}