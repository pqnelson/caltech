%%
%% winter-lecture13.tex
%% 
%% Made by Alex Nelson <pqnelson@gmail.com>
%% Login   <alex@lisp>
%% 
%% Started on  2026-02-03T09:49:45-0800
%% Last update 2026-02-03T09:49:45-0800
%% 

\lecture{}

\begin{definition}
Let $E/F$ be a finite field extension. Let $r=[E:F]_{s}$ be the
separable degree (which equals $|\hom_{F}(E,\closure{F})|$). Let
$\sigma_{1},\dots,\sigma_{r}\in\hom_{F}(E,\closure{F})$ be
distinct.
\begin{enumerate}
\item We define the \define{Norm} $N_{E/F}\colon E\to F$ to be such
  that $N_{E/?F}(\alpha)=(\prod_{j}\sigma_{j}(\alpha))^{[E:F]_{i}}$
\item We define the \define{Trace} $T_{E/F}\colon E\to F$ equal to
  $\Tr_{E/F}(\alpha)=[E:F]_{i}\sum_{j}\sigma_{j}(\alpha)$.
\end{enumerate}
Note: some people define the norm on $\MultGroup{E}$, but it can be
defined on all of $E$. The question is just what do we do with $0\in E$?
(It will be obvious shortly that it should be mapped to zero.)
\end{definition}

\begin{note}
When the norm $N_{E/F}$ is restricted to $\MultGroup{E}$, it is a
group morphism.
\end{note}

\begin{proposition}\label{prop:10.1}
Let $E/F$ be a finite extension. Then $N_{E/F}(\alpha)\in\MultGroup{F}$
and $\Tr_{E/F}(\alpha)\in F$. Moreover, for $L/E/F$, we have
$N_{L/F}=N_{L/E}\circ N_{E/F}$ and $\Tr_{L/F}=\Tr_{L/E}\circ\Tr_{E/F}$.
\end{proposition}

\begin{proposition}\label{prop:10.2}
For any finite extension $E/F$ and for any $\alpha\in E$, we may write
\begin{equation}
\begin{split}
m_{\alpha}\colon & E\to E\\
& \beta\mapsto\alpha\beta.
\end{split}
\end{equation}
We may view $m_{\alpha}$ as an $F$-linear transformation and hence as
an $[E:F]\times[E:F]$ matrix. Then $N_{E/F}(\alpha)=\det(m_{\alpha})$
and $\Tr_{E/F}(\alpha)=\tr(m_{\alpha})$, and the minimal polynomial of
$\alpha$ is equal to the characteristic polynomial of $m_{\alpha}$.
\end{proposition}

\begin{proposition}\label{prop:10.3}
For any finite separable extension $E/F$, and for any $\alpha\in\MultGroup{E}$,
we define the inner product $\langle x,y\rangle := \Tr_{E/F}(\alpha xy)$.
This is a nondegenerate and perfect pairing, and the mapping
\begin{equation}
\begin{split}
&E\to\hom_{F}(E,F)=E^{*}\\
&y\mapsto\langle-,y\rangle,
\end{split}
\end{equation}
is an isomorphism (where $E^{*}$ is the dual of $E$).
\end{proposition}