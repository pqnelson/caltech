%%
%% winter-lecture22.tex
%% 
%% Made by Alex Nelson <pqnelson@gmail.com>
%% Login   <alex@lisp>
%% 
%% Started on  2026-02-24T07:37:03-0800
%% Last update 2026-02-24T07:37:03-0800
%% 

\lecture{}

\begin{definition}
Let $K/F$ be an arbitrary field extension, let $\{\alpha_{1},\dots,\alpha_{n}\}$
be a finite subset of $K$. We say the $\alpha_{i}$ are
\define{Algebraically Independent} if the mapping
\begin{equation}
\begin{split}
F[x_{1},\dots,x_{n}]&\to K\\
x_{i}&\mapsto\alpha_{i}
\end{split}
\end{equation}
is injective (equivalently: if $P(\alpha_{1},\dots,\alpha_{n})=0$,
then $P$ is identically zero).

Note: $F(\alpha)/F$ is \define{Transcendental} if $\alpha$ is
algebraically independent over $F$.
\end{definition}

\begin{lemma}\label{lemma:16.1}
Let $K/F$ be a field extension, let $\gamma\in K$, let $A\subset K$.
Then the following are equivalent:
\begin{enumerate}
\item $\gamma$ is algebraic over $F(A)$;
\item There exists a polynomial $P\in F[x_{1},\dots,x_{m},y]$ and
  there exists $\alpha_{1},\dots,\alpha_{m}\in A$ such that
  $P(\alpha_{1},\dots,\alpha_{m},y)\neq0$ but $P(\alpha_{1},\dots,\alpha_{m},\gamma)=0$.
\end{enumerate}
\end{lemma}

\begin{proof}
$(1)\implies(2)$ Let $f\in F(A)[y]$ be the [irreducible] minimal
  polynomial of $\gamma$ over $F(A)$. Since there exists only finitely
  many coefficients of $f$, we can assume $f\in
  F[\alpha_{1},\dots,\alpha_{m}][y]$. Then we can take $P$ defined by
  replacing each $\alpha_{i}$ to $x_{i}$. Hence the result.

$(2)\implies(1)$ Obvious.
\end{proof}

\begin{lemma}\label{lemma:16.2}
Let $K/F$ be some field extension. Let $\alpha_{1},\dots,\alpha_{m}\in K$.
If $\beta$ is algebraic over $F(\alpha_{1},\dots,\alpha_{m})$ but not
over $F(\alpha_{1},\dots,\alpha_{m-1})$, then $\alpha_{m}$ is
algebraic over $F(\alpha_{1},\dots,\alpha_{m-1},\beta)$.
\end{lemma}

\begin{proof}
Since $\beta$ is algebraic over $F(\alpha_{1},\dots,\alpha_{m})$, then
by Lemma~\ref{lemma:16.1} there exists a polynomial $P\in F[x_{1},\dots,x_{m},y]$
such that
\begin{equation}
P(\alpha_{1},\dots,\alpha_{m},y)\neq0\quad\mbox{but}\quad P(\alpha_{1},\dots,\alpha_{m},\beta)=0.
\end{equation}
We can write the polynomial as a polynomial in $x_{m}$:
\begin{equation}
P=\sum_{i}P_{i}(x_{1},\dots,x_{m-1},y)x_{m}^{i}.
\end{equation}
Since $P(\alpha_{1},\dots,\alpha_{m},y)\neq0$, there must exist some
$n\in\NN_{0}$ such that
\begin{equation}
P_{n}(\alpha_{1},\dots,\alpha_{m-1},y)\neq0\quad\mbox{and}\quad P_{n}(\alpha_{1},\dots,\alpha_{m-1},\beta)=0.
\end{equation}
[Proof: $\beta$ is transcendental over
  $F(\alpha_{1},\dots,\alpha_{m-1})$ by hypothesis.]
Therefore
\begin{equation}
x^{n}_{m}P_{n}(\alpha_{1},\dots,\alpha_{m-1},y)\neq0\quad\mbox{and}\quad \alpha_{m}^{n}P_{n}(\alpha_{1},\dots,\alpha_{m-1},\beta)=0.
\end{equation}
By Lemma~\ref{lemma:16.1} again, we have the result.
\end{proof}

\begin{proposition}\label{prop:16.3}
Let $K/F$ be a field extension, let
$A=\{\alpha_{1},\dots,\alpha_{m}\}\subset K$ and
$B=\{\beta_{1},\dots,\beta_{n}\}\subset K$ be finite subsets.
If $A$ is algebaically independent over $F$ but $A$ is algebraic over
$F(B)$, then $|A|\leq|B|$.
\end{proposition}

\begin{proof}
Let $\{\alpha_{1},\dots,\alpha_{k}\}=A\cap B$ (possibly reindexing $A$
and $B$, if necessary). Then we have two cases.

\textsc{Case 1}: If $k=m$, then $A\subset B$, and the result follows
immediately.

\textsc{Case 2}: If $k\neq m$, then we can take $\alpha_{k+1}$ which
is algebraic over $F(B)=F(\alpha_{1},\dots,\alpha_{k},\beta_{k+1},\dots,\beta_{n})$ but not algebraic over $F(\alpha_{1},\dots,\alpha_{k})$
(by algebraic independence of $A$ over $F$).
Let us take the smallest $j$ between $k+1\leq j\leq n$ such that
$\alpha_{k+1}$ is algebraic over
$F(\alpha_{1},\dots,\alpha_{k},\beta_{k+1},\dots,\beta_{j})$.
For
\begin{equation}
B_{1}:=(B\setminus\{\beta_{j}\})\cup\{\alpha_{k+1}\},
\end{equation}
by Lemma~\ref{lemma:16.2} we see $\beta_{j}$ is algebraic over $F(\alpha_{1},\dots,\alpha_{k},\beta_{k+1},\dots,\beta_{j-1})$.
Then $B\subset B_{1}\cup\{\beta_{j}\}$ is algebraic over $F(B_{1})$.
Since $A$ is algebraic over $F(B)$, then by Proposition~\ref{prop:1.3}
(algebraic extensions are a distinguished class), we find $A$ is
algebraic over $F(B_{1})$, and also $|A\cap B_{1}|=k+1$. By repeating
this process, we find $A\subset B_{m-k}$, and so
\begin{equation}
|A|\leq|B_{m-k}|.
\end{equation}
But $|B_{m-k}|=|B|$, which gives us the result.
\end{proof}

\begin{definition}
A \define{Transcendental Basis} for $K/F$ is an algebraically
independent set $A\subset K$ such that $K/F(A)$ is algebraic.
\end{definition}

\begin{proposition}\label{prop:16.4}
Every maximal algebraically independent subset of $K/F$ is a
transcendental basis for $K/F$.
\end{proposition}

\begin{proof}
Let $A\subset K$ be such a maximal algebraically independent subset,
and let $\beta\notin A$. Then $A\cup\{\beta\}$ is algebraically
independent by Lemma~\ref{lemma:16.1} there exists a polynomial
\begin{equation}
P=\sum_{n}P_{n}(x_{1},\dots,x_{m})y^{n}\in F[x_{1},\dots,x_{m},y]
\end{equation}
and there exists $\alpha_{1},\dots,\alpha_{m}\in A$ such that
\begin{equation}
P(\alpha_{1},\dots,\alpha_{m},\beta)=0,
\end{equation}
but for some $n$ (since the $\alpha_{i}$ are algebraically independent),
\begin{equation}
P_{n}(\alpha_{1},\dots,\alpha_{m})\neq0.
\end{equation}
Therefore $\beta$ is algebraic over
$F(\alpha_{1},\dots,\alpha_{m})\subset F(A)$ and thus $K$ is algebraic
over $F(A)$; i.e., $A$ is a transcendental basis.
\end{proof}

\begin{proposition}\label{prop:16.5}
For an algebraically independent $A\subset K$, there exists a
transcendental basis for $K/F$ such that $A\subset B$.
\end{proposition}

\begin{proof}
Let us consider the poset $\mathcal{A}$ of algebraically independent
subsets containing $A$, ordered by inclusion. For a totally ordered
subset $\mathcal{A}'\subset\mathcal{A}$, we see $\bigcup_{A\in\mathcal{A}'}A$
is an upper bound of $\mathcal{A}'$. Now we can use Zorn's lemma:
there exists a maximal element of $\mathcal{A}$; i.e., a maximal
algebraically independent subset $B$ containing $A$. Then by
Proposition~\ref{prop:16.4}, $B$ is a transcendental basis.
\end{proof}

\begin{definition}
Let $K/F$ be a field extension. The \define{Transcendental Degree} of
$K$ over $F$ is the number $\trdeg_{F}K$ equal to the cardinality of a
transcendental basis for $K/F$.
\end{definition}

\begin{xca}
The transcendental degree of $K$ over $F$ is independent of choice of
transcendental basis.
\end{xca}

\begin{xca}
For a subextension $K/E/F$, $\trdeg_{F}K=\trdeg_{E}K+\trdeg_{F}E$.
\end{xca}