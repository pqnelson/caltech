%%
%% fall-lecture23.tex
%% 
%% Made by Alex Nelson <pqnelson@gmail.com>
%% Login   <alex@lisp>
%% 
%% Started on  2025-11-20T09:38:30-0800
%% Last update 2025-11-20T09:38:30-0800
%% 

\lecture{}

\begin{example}
Consider the ring $A=\kk[x,y,z]$ where $\kk$ is a field.
We look at $A$ as a graded ring. Let's try counting the number of
monomials of different degrees:
\begin{center}
\begin{tabular}{clc}
degree & monomials & total number\\
0 & 1 & 1\\
1 & $x$, $y$, $z$ & 3\\
2 & $x^{2}$, $xy$, $xz$, $y^{2}$, $yz$, $z^{2}$ & 6\\
3 & \dots & 10
\end{tabular}
\end{center}
How many monomials are there of degree $n$? It's the same as asking
for how many ways can we write $n=i+j+k$ for $i,j,k\geq0$. ``Stars and bars''
from combinatorics tells if we write down $n+2$ stars and try to place
2 bars (so we're counting the gaps between the stars as $i$, $j$, $k$),
then there are $\binom{n+2}{2}$ ways to place the bars down. This
means we can complete our table:
\begin{center}
\begin{tabular}{clc}
degree & monomials & total number\\
0 & 1 & 1\\
1 & $x$, $y$, $z$ & 3\\
2 & $x^{2}$, $xy$, $xz$, $y^{2}$, $yz$, $z^{2}$ & 6\\
3 & \dots & 10\\
$n$ & \dots & $\binom{n+2}{2}$
\end{tabular}
\end{center}
\end{example}

\begin{example}
Now, consider the number of monomials in the ring $\kk[x,y,z]/(x^{2}-y^{2}-z^{2})$.
We can write down a similar table
\begin{center}
\begin{tabular}{clc}
degree & monomials & total number\\
0 & 1 & 1\\
1 & $x$, $y$, $z$ & 3\\
2 & $xy$, $xz$, $y^{2}$, $yz$, $z^{2}$ & 5\\
3 & $xy^{2}$, $xz^{2}$, $y^{3}$, $y^{2}z$, $yz^{2}$, $z^{3}$ & 7
\end{tabular}
\end{center}
What's the general pattern? Well, we can keep grinding through the
calculations to find
\begin{center}
\begin{tabular}{clc}
degree & monomials & total number\\
0 & 1 & 1\\
1 & $x$, $y$, $z$ & 3\\
2 & $xy$, $xz$, $y^{2}$, $yz$, $z^{2}$ & 5\\
3 & $xy^{2}$, $xz^{2}$, $y^{3}$, $y^{2}z$, $yz^{2}$, $z^{3}$ & 7\\
4 & \dots & 9\\
5 & \dots & 11,
\end{tabular}
\end{center}
then it becomes painfully obvious there are $2n+1$ monomials of degree $n$.
\end{example}

\begin{theorem}[Matsumara 10F]
Let $A$ be an Artinian ring, let $B=A[x_{1},\dots,x_{m}]$,
and $M=\bigoplus_{n\geq0}M_{n}$ be a graded module over $B$. Then
there exists some polynomial $f_{M}(x)\in\QQ[x]$ such that
the function defined as $F_{M}(n):=\length_{A}(M_{n})$ is such that $F_{M}(n)=f_{M}(n)$ for some sufficiently large $n\gg0$.
\end{theorem}

Note that $M$ is a finitely-generated module over $B$, that $B$ is
Noetherian, and therefore $M$ is Noetherian.

I will confess, I am not terribly fond of Matsumara's proof. It's
rather maundering. The bulk of the lecture was working through
Matsumara's proof. At the end of class, one of the graduate students
(Tasmin Chu?) found a slicker proof for the inductive case:

\begin{proof}
Let $P[M']$ be the assert holds for a graded module $M'$ over $B$.
We will prove by induction: for any submodule $N$, the assertion
$P[M/N]$ holds.

\textsc{Base case:} $P[M/M]$ obviously holds.

\textsc{Inductive hypothesis:} Suppose $P[M/N']$ is true for all
submodules $N'\propersupset N$ strictly.
  
\textsc{Inductive case:}
Let $\mathcal{N}=\{N\subset M\mid P[M/N]\mbox{ does not hold}\}$.
If $N$ is a maximal element of $\mathcal{N}$, then for all
$N'\propersupset N$ we have $P[M/N']$ holds. Then $P[M/N]$ holds,
which is a contradiction. So $\mathcal{N}=\emptyset$, and
$(0)\notin\mathcal{N}$ implies $P[M/(0)]$ holds as desired.
\end{proof}

This works because $B$ is Noetherian and $M$ is a finitely-generated
module over $B$.