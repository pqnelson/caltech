%%
%% fall-lecture06.tex
%% 
%% Made by Alex Nelson <pqnelson@gmail.com>
%% Login   <alex@lisp>
%% 
%% Started on  2025-10-11T13:49:34-0700
%% Last update 2025-10-11T13:49:34-0700
%% 

\lecture{}

The professor is trying a new approach: every Friday is a more relaxed
discussion than a lecture, with emphasis on puzzles and history.

\begin{definition}
Let $A$ be a ring. If $M$ is a finitely-generated $A$-module, then
there exists an $n\in\NN$ such that
\begin{equation}
A^{n}\xrightarrow{p}M\xrightarrow{q}0
\end{equation}
is a short exact sequence. We can do this for \emph{any} $A$-module,
but non-finitely-generated $A$-modules would have their short exact
sequence look like
\begin{equation}
\bigoplus^{\infty}_{i=0}A_{i}\xrightarrow{p}M\xrightarrow{q}0
\end{equation}
where $A_{i}=A$ is just a copy of $A$ indexed by $i\in\NN_{0}$.

The $\ker(p)$ describes the relations among the generators. We can
extend this short exact sequence to
\begin{equation}
\cdots\to A^{m}\xrightarrow{r}A^{n}\xrightarrow{p}M\xrightarrow{q}0
\end{equation}
such that $\ker(p)=\Im(r)$. This is really a \define{Complex}, it's
exact when $\ker(p)=\Im(r)$ and $\ker(q)=\Im(p)$ and\dots
\end{definition}

\begin{node}
There is another variant of the definition for a Noetherian ring:
A ring $R$ is Noetherian iff every ideal is finitely-generated. (If
$R$ is Noetherian, take the union of the chain of ideals, it is
necessarily finitely-generated.)
\end{node}

\begin{proposition}
A submodule of a finitely-generated module is finitely-generated.
\end{proposition}

\begin{example}
Let $\kk[x_{1},x_{2},\dots]$ be a ring of formal polynomials in
infinitely many unknowns $x_{i}$ indexed by $i\in\NN$. This is not
Noetherian, because we can form the chain of ideals
$I_{k}=(x_{1},\dots,x_{k})$ which will not stabilize.
\end{example}

\begin{node}
Emmy Noether, ``Ideal Theory in Rings'' (English translation
\arXiv{1401.2577}) introduced the notion in \S1 as a finiteness
condition on rings.

Algebraic integers refers to all algebraic elements of $\CC$ adjoined
to $\ZZ$. That is to say, if $f\in\ZZ[x]$ is monic, it's roots
$\alpha\in\CC$ satisfying $f(\alpha)=0$ are adjoined to $\ZZ$. If we
write $\mathcal{O}$ for the integral closure of $\ZZ$, we have
\begin{equation}
\xymatrix@R=1pc{\ZZ\;\ar@{^{(}->}[r]\ar@{}[d]|-*[@]{\rotatebox{-90}{$\subset$}} & \raisebox{0pt}[0.9\height][0.3\height]{$\mathcal{O}$}\ar@{^{(}->}[d] \\
\QQ\;\ar@{^{(}->}[r] & \overline{\QQ}\;\ar@{^{(}->}[r] & \CC}
\end{equation}
where $\overline{\QQ}$ is the algebraic closure of $\QQ$. The integral
closure $\mathcal{O}$ of $\ZZ$ is not Noetherian, just consider the
chain of ideals generated by square roots of primes which is not terminating $(\sqrt{2},\sqrt{3},\sqrt{5},\dots)$.
\end{node}

\begin{node}
Let $A$ be Noetherian, let $M$ be a finitely-generated $A$-module. We
have the short-exact sequence
\begin{equation}
A^{n_{1}}\onto M\to 0.
\end{equation}
But we can form a complex of free modules:
\begin{equation}
\underbrace{\cdots\to A^{n_{2}}\to A^{n_{1}}}_{\text{free modules}}\to M\to 0
\end{equation}
We can see
\begin{equation}
\varphi_{i}\colon A^{n_{i+1}}\to A^{n_{i}}
\end{equation}
and
\begin{equation}
\varphi_{0}\colon A^{n_{1}}\to M
\end{equation}
satisfy the expected relations $\Im(\varphi_{i+1})\subset\ker(\varphi_{i})$.
This is called the \define{Free Resolution} of $M$.
\end{node}

\begin{definition}[Projective modules]
A module $P$ is called \define{Projective} if there exists a module
$Q$ such that $P\oplus Q\iso F$ for some free module $F$.
\end{definition}

\begin{example}
More examples of Noetherian rings: $\ZZ$ is Noetherian; any principal
ideal domain is Noetherian; if $R$ is Noetherian, then $R[x]$ is
Noetherian (and so is $R[[x]]$).

If $R$ is Noetherian and $I\ideal R$, then $R/I$ is Noetherian.
\end{example}

\begin{example}
If $B$ is a Noetherian ring, and $A\subset B$ is a subring, then is
$A$ Noetherian?

The algebraic integers $\mathcal{O}\subset\overline{Q}$ are not
Noetherian despite $\overline{Q}$ being a field.
\end{example}

\begin{example}
Let $A\into B$ be an integral ring extension. If $B$ is Noetherian,
then is $A$ Noetherian?
\end{example}

\begin{node}
Any proper ideal $I$ in a Noetherian ring $A$ has a primary
decomposition: $I=Q_{1}\cap\cdots\cap Q_{n}$ where $Q_{i}$ is a
primary ideal of $A$.
\end{node}

\begin{node}[Puzzle]
We know ``If $Q\ideal A$ is primary, then $\Radical{Q}$ is prime''.

But is the converse true, ``If $\Radical{Q}$ is prime, then $Q$ is primary''?
\end{node}

\begin{example}
Consider $Q=(x^{2},xy)\ideal\CC[x,y]$. Is $Q$ primary?

We have $x\cdot x\in Q$ implies $x^{n}\in Q$ or $x\in Q$ (which is
true).

We have $y\cdot x\in Q$ implies $y\in Q$ or $x^{n}\in Q$ (which is
true).

But $x\cdot y\notin Q$ since this would imply $x\in Q$ or $y^{n}\in Q$
which is not true. This means $Q$ is not primary.

However, $\Radical{Q}=(xy,x)=(x)$ which is prime. This gives a
counter-example to the claim ``$\Radical{Q}$ is prime implies $Q$ is primary''.
\end{example}