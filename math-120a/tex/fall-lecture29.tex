%%
%% fall-lecture29.tex
%% 
%% Made by Alex Nelson <pqnelson@gmail.com>
%% Login   <alex@lisp>
%% 
%% Started on  2025-12-06T09:32:48-0800
%% Last update 2025-12-06T09:32:48-0800
%% 

\lecture[Hilbert's Nullstellensatz]

\begin{remark}
Hilbert's Nullstellensatz enabled us to assert that maximal ideals of
$\CC[x,y]$ look like $(x-a,y-b)$ for some $a,b\in\CC$. We should
probably \emph{prove} that before the end of the quarter. Since
today's the last lecture, that means we should prove it today.
\end{remark}

\subsection{Morphisms and Dimensions}

\begin{definition}
Let $\phi\colon A\to B$ be a ring morphism. Let $P\in\Spec(A)$ and
$\kappa(P)=A_{P}/PA_{P}$ be the residue field. We can consider
$\Spec(B\otimes_{A}\kappa(P))$ which is called the \define{Fober} over
$B$ (of the map $\phi^{*}\colon\Spec(B)\to\Spec(A)$).

Observe that the preimage
$(\phi^{*})^{-1}(P)\iso\Spec(B\otimes_{A}\kappa(P))$ homeomorphic.
\end{definition}

\begin{remark}
Consider some ideal $Q$ lying over $P$, where $Q\in\Spec(B)$. This has
the corresponding prime
\begin{equation}
QB_{P}/PB_{P}\subset B_{P}/PB_{P}\iso B\otimes_{A}\kappa(P),
\end{equation}
if $Q$ is some appropriate preimage. Writing $Q^{*}=QB_{P}/PB_{P}$,
we find
\begin{subequations}
  \begin{align}
(B\otimes_{A}\kappa(P))_{Q^{*}} &\iso B_{Q}\otimes_{A}\kappa(P)\\
&\iso B_{Q}/PB_{Q}.
  \end{align}
\end{subequations}
\end{remark}

\begin{theorem}[Matsumara 13B]
Let $A$, $B$ be Noetherian rings. Let $\phi\colon A\to B$ be a ring morphism.
Let $Q\in\Spec(B)$ and $P=Q\cap A$. Then the following all hold:
\begin{enumerate}
\item $\height(Q)\leq\height(P)+\height(QB_{P}/PB_{P})$ and in
  particular this means $\dim(B_{Q})\leq\dim(A_{P})+\dim(B_{Q}\otimes_{A}\kappa(P))$;
\item If the Going-Down theorem holds for $\phi$ (e.g., if $\phi$ is
  flat), then the inequalities in the previous item are equalities.
\item If also $\phi^{*}\colon\Spec(B)\to\Spec(A)$ is surjective and if
  the Going-Down theorem holds then $\dim(B)\geq\dim(A)$ and also for
  each ideal $I\ideal A$ we have $\height(I)=\height(IB)$.
\end{enumerate}
\end{theorem}

\begin{theorem}[Matsumara 13C]
Let $B$ be a Noetherian ring. Let $A$ be a Noetherian subring of $B$
such that $B$ is integral over $A$. Then $\dim(A)=\dim(B)$ and for
$P\in\Spec(B)$, $\height(P)\leq\height(P\cap A)$.
\end{theorem}

\subsection{Finitely-Generated Extensions}

\begin{theorem}[Matsumara 14A]
Let $A$ be a Noetherian ring, then $\dim(A[x_{1},\dots,x_{n}])=n+\dim(A)$.
\end{theorem}

\begin{remark}
This is not obvious if $A$ is not Noetherian. Atiyah and Macdonald has
a fun exercise about this, it's the very last one it their book. In
fact, they ask (in the last two exercises on page 126) to also prove
that for any commutative ring $A$ (possibly not Noetherian),
\begin{equation}
1+\dim(A)\leq\dim(A[x])\leq 1+2\dim(A).
\end{equation}
\end{remark}

\begin{corollary}[Matsumara 14.1]
Let $\kk$ be a field. Then $\dim(\kk[x_{1},\dots,x_{n}])=n$.
\end{corollary}

\begin{theorem}[Normalization theorem of E.Noether, Matsumara 14G]
Let $A=\kk[x_{1},\dots,x_{n}]$ be a finitely-generated algebra over a
field $\kk$.
Then there exists elements $y_{1}$, \dots, $y_{r}\in A$ which are
algebraically independent over $\kk$ such that $A$ is integral over $\kk[y_{1},\dots,y_{r}]$.
We have $r=\dim(A)$. If further $A$ is an integral domain, we have $r=\trdeg_{\kk}(A)$.
\end{theorem}

\begin{corollary}[Matsumara 14.4]
Let $\kk$ be an algebraically closed field.
Then any maximal ideal $\mmm\ideal\kk[x_{1},\dots,x_{n}]$ is of the
form $(x_{1}-a_{1},\dots,x_{n}-a_{n})$.
\end{corollary}

\begin{theorem}[Hilbert's Nullstellensatz, Matsumara 14J]
Let $\kk$ be a field, let $A$ be a finitely-generated $\kk$-algebra,
and $I\properideal A$ be a proper ideal of $A$.
Then 
\begin{equation}
\Radical{I}=\bigcap_{\substack{\mmm\in\MSpec(A)\\I\subset\mmm}}\mmm.
\end{equation}
\end{theorem}

\begin{remark}
This may be viewed as a generalization of the fundamental theorem of algebra.
\end{remark}

\begin{remark}
There are two forms of Hilbert's Nullstellensatz:
\begin{enumerate}
\item Weak form: $I\properideal\kk[x_{1},\dots,x_{n}]$,
  $V(I)\neq\emptyset$;
\item Strong form: $I(V(I))=\Radical{I}$ for any $I\properideal\kk[x_{1},\dots,x_{n}]$.
\end{enumerate}
\end{remark}

\begin{remark}
This is the end of the course. And it's only about a third of the
material in Matsumara. If you want to continue learning Commutative
Algebra, and this is your first ``real'' exposure to it as a subject,
then continuing on with Matsumara is a great way to learn the material.
\end{remark}