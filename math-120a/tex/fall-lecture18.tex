%%
%% fall-lecture18.tex
%% 
%% Made by Alex Nelson <pqnelson@gmail.com>
%% Login   <alex@lisp>
%% 
%% Started on  2025-11-08T07:52:08-0800
%% Last update 2025-11-08T07:52:08-0800
%% 

\lecture{}

\begin{definition}
Let $A$ be a ring. Very frequently, we look at two [prime] ideals
$P\subset P'$ of $A$. We will say that $P'$ is a
\define{Specialization} of $P$, and that $P$ is a
\define{Generalization} of $P'$.
\end{definition}

\begin{node}
Geometrically $P$ corresponds to a curve and $P'$ lies on the curve.

What is the topological closure of $\{P'\}$ (in the Zariski topology)?
It's just $\{P'\}$ itself.

What is the closure of the \emph{point} $\{P\}$ --- what is $\closure{\{P\}}$?
These are the ideals $I$ containing $P\subset I$. This is all the
points lying over the curve $P$. This should be very non-intuitive
topologically. The classic geometric intuition is that maximal ideals
are points, then we ``add'' more points corresponding to prime
ideals. Then it makes sense to ask if a point lies in/on the closure
of another point.
\end{node}

\begin{node}
Can we formulate the going-up (\S\ref{defn:going-up-property}) and going-down (\S\ref{defn:going-down-property})
using this language? We have $A\to B$, and this induces the map
$f\colon\Spec(B)\to\Spec(A)$.
\begin{itemize}
\item Going-up = image of $f$ is stable under specialization;
\item Going-down = image of $f$ is stable under generalization.
\end{itemize}
\end{node}

\begin{lemma}[Matsumara 6.4]\label{lemma:fall-lec18:math-120a:lemma-6-4}
Let $A$ be a Noetherian ring. Let $X=\Spec(A)$. Let $Z$ be
constructible. If $Z$ is stable under generalization, then $Z$ is
closed in $X$.
\end{lemma}

\begin{proof}
Take the topological closure $\closure{Z}$. We want to prove $\closure{Z}=Z$.
Take some irreducible component $W$ of $\closure{Z}$, and $X$ is its
generic point. (Then $\closure{\{X\}}=W$.) Then $W\cap Z$ is dense in
$W$. Then $W\cap Z$ contains an open subset of $W$ (by Theorem~\ref{prop:fall-lec16:prop-6c}).
Then $X\in Z$. Hence the result.
\end{proof}

\begin{xca}
Why is $W\cap Z$ dense in $W$?
\end{xca}

\begin{theorem}[Matsumara 8 (6I)]
Let $A$ be a Noetherian ring, let $B$ be an $A$-algebra of finite
type. Suppose that Going-Down (\S\ref{defn:going-down-property}) holds
for the natural map $i\colon A\to B$. Then the canonical map
$f\colon\Spec(B)\to\Spec(A)$ is an open map.
\end{theorem}

(Recall, an open map sends open sets to open sets, it occurs in
complex analysis, and in the open mapping theorem in Functional
Analysis.)

\begin{proof}
Let $U$ be an open set in $\Spec(B)$. Then $f(U)$ is constructible by
Chevalley's Theorem~\ref{thm:120a:chevalley}. But Going-Down means
that $f(U)$ is stable under generalization (if $P\subset P'$ and
$P'\subset f(U)$, then $P\subset f(U)$).

Now, we should use Lemma~\ref{lemma:fall-lec18:math-120a:lemma-6-4} on
$\Spec(A)\setminus f(U)$. If $P\notin f(U)$, then $P'\notin f(U)$ by
contrapositive of the above claim.

But if $P\in f(U)$, then we get a contradiction. Hence the result.
\end{proof}

\begin{proposition}[Matsumara 6J]
Let $\phi\colon A\to B$ be a ring morphism, let $B$ be Noetherian, and
$\phi$ satisfies Going-Up.
Then $\preimage\phi=f\colon\Spec(B)\to\Spec(A)$ is closed.
\end{proposition}

Matsumara says this is an ``easy exercise''.

\subsection{Associated Primes}

\begin{node}
This is Chapter~III of Matsumara. Henceforth, we consider all rings to
be Noetherian (and we will not announce it).
\end{node}

\begin{definition}
Let $A$ be a ring, let $M$ be an $A$-module. We say that the prime
ideal $P\in\Spec(A)$ is an \define{Associated Prime} of $M$ if one of
the following equivalent conditions holds:
\begin{enumerate}
\item There exists some $x\in M$ such that $\Annihilator(x)=P$.
\item $M$ contains a submodule isomorphic to $A/P$.
\end{enumerate}
We write $\Assassinator_{A}(M)$ for the associated primes of $M$ in
the ring $A$ (and pronounce it as ``the assassinator of $M$'' --- get
your mind out of the gutter).
\end{definition}

\begin{example}
Consider $M=\CC[x,y]/(x^{2},xy)$ as a $\CC[x,y]$ module. We see that
$(x)\cap(x^{2},y)$ and $(x)\cap(x,y)^{2}$ are primary decompositions
of $I=(x^{2},xy)$.

Let $x\in\CC[x,y]_{(x^{2},xy)}$. Then $(x,y)=\Annihilator(x)$, so
$(x,y)\in\Assassinator(M)$.

Let $y\in\CC[x,y]_{(x^{2},xy)}$. Then $(x)=\Annihilator(y)$, so
$(x)\in\Assassinator(M)$.

We see $(x)=\Radical{(x)}$ and $(x,y)=\Radical{(x^{2},y)}$.
\end{example}