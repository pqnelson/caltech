%%
%% fall-lecture28.tex
%% 
%% Made by Alex Nelson <pqnelson@gmail.com>
%% Login   <alex@lisp>
%% 
%% Started on  2025-12-04T09:15:33-0800
%% Last update 2025-12-04T09:15:33-0800
%% 

\lecture{}

%% \begin{theorem}[Review from last time]
%% Let $A$ be a Noetherian semilocal ring, let $\mathfrak{m}=\JacobsonRadical(A)$,\dots
%% Then $d(A)=\dim(A)$
%% \end{theorem}

\begin{theorem}[Matsumara 12I]
Let $A$ be a Noetherian ring, let $I=(a_{1},\dots,a_{r})$ be an ideal
generated by $r$ elements. Then any minimal overprime [i.e., prime over-ideal]
$P$ of $I$ has height $\height(P)\leq r$. In particular,
$\height(I)\leq r$.
\end{theorem}

\begin{xca}
Think of a situation where $\height(P)<r$.
\end{xca}

\begin{proof}
Take $PA_{P}$ which is the only prime ideal of $A_{P}$ containing $IA_{P}$.
This means $A_{P}/IA_{P}=A_{P}/(a_{1}A_{P}+\dots+a_{r}A_{P})$ is Artinian.
Then $\height(P)=\dim(A_{P})\leq r$ by Theorem~\ref{matsumara-ca:theorem-12-h}.
\end{proof}

\begin{definition}[Matsumara 12J]
Let $(A,\mmm,\kk)$ be a local ring. Then an ideal of
definition of $A$ is the same thing as a primary ideal belonging to $\mmm$.

If $\dim(A)=d$, then all ideals of definition are generated by at
least $d$ elements and there exists some generated by exactly $d$ elements.

If $(x_{1},\dots,x_{d})$ is an ideal of definition generated by
exactly $d$ elements, then we call the set $\{x_{1},\dots,x_{d}\}$ a
\define{System of Parameters} of $A$. (We can view a system of
parameters as a sort of ``local coordinates''.)

If there exists a system of parameters that generated $\mmm$, then
$(A,\mmm,\kk)$ is a \define{Regular Local Ring}.\marginpar{{\footnotesize Important definition}}

(Observe: the number of elements in a minimal basis of $\mmm$ is equal
to $\rank_{\kk}(\mmm/\mmm^{2})\leq\dim(A)$.)
\end{definition}

\begin{example}
The ring $\CC[x,y]_{(x,y)}$ has $\mmm/\mmm^{2}=(x,y)$, so everything
has dimension 2.
\end{example}

\begin{non-example}
The ring $(\CC[x,y]/(x^{2}-y^{3}))_{(x,y)}$ has dimension 1, but
$\mmm=(x,y)$ is the maximal ideal, which has 2 generators. Can we find
a system of parameters?

Take $I=(x^{2})$. We see $I\subset\mmm$, and then we want to find an
$n\in\NN$ such that $\mmm^{n}\subset I$. Well,
\begin{equation*}
\mmm^{2}=(x^{2},xy,y^{2})
\end{equation*}
is inadequate,
\begin{equation*}
\mmm^{3}=(x^{3},x^{2}y,xy^{2},y^{3})
\end{equation*}
also turns out to be inadequate, but
\begin{equation}
\mmm^{4}=(x^{4},x^{3}y,x^{2}y^{2},xy^{3},y^{4})\subset I
\end{equation}
works. But we find $\dim(\mmm/\mmm^{2})=2$. Geometrically, what's
going on is that $\mmm/\mmm^{2}$ is the (co)tangent space at the
singular point of the nodal curve, and it has tangent (co)vectors in
every direction there.

If we tried a \emph{different} point like $(1,1)$ for our ring
$A=(\CC[x,y]/(x^{2}-y^{3}))_{(x-1,y-1)}$, then we can try changing
variables $(\CC[a,b]/((a+1)^{2}-(b+1)^{3}))_{(a,b)}$ where $a=x-1$ and $b=y-1$.
Does $I=(a)$ work as a system of parameters? Well, we'd need to solve
\begin{equation}
a^{2}+2a+1-b^{3}-3b^{2}-3b-1=0
\end{equation}
for $a$ in terms of $b$. We claim
\begin{equation}
a = \frac{(b^{2}+3b+3)}{(a+2)}b
\end{equation}
where $b\in\mmm$ and $(a+2)\in A_{\mmm}$ works.
\end{non-example}

\begin{node}
Ifwe studied algebra from a different approach (e.g., algebraic number
theory), then we start with Galois theory and study a number field $K$
(i.e., a finite extension of $\QQ$), then study the ring of integers
in number fields like $\mathcal{O}_{K}$. We find $\mathcal{O}_{K}$ is
a Noetherian domain ($\dim\mathcal{O}_{K}=1\iff\forall p\neq0$ prime
is maximal). What happens in a primary decomposition
$I=Q_{1}\cap\dots\cap Q_{r}$
where $P_{i}$ is a prime ideal associated with $Q_{i}$ and $P_{i}$ is
maximal. Then their pairwise decomposition is the whole ring $P_{i}+P_{j}=A$.
We want to claim $Q_{1}\cap\dots\cap Q_{r}=Q_{1}(\dots)Q_{r}$ is a product.

We can look at regular rings in dimension 1.
\end{node}

\begin{definition}
A \define{Dedekind domain} is a domain $A$ of dimension $1$ such that
$A_{P}$ is a regular local ring for each $P\neq0$ (it's also true for
$P=0$, but we want to do $P$ prime).
\end{definition}

\begin{remark}
A Dedekind domain is a ``global ring'' such that localizing at a point
is regular (i.e., a regular local ring).
\end{remark}

\begin{definition}
A Noetherian regular local ring of dimension $1$ is called a
\define{Discrete Valuation Ring}. (Equivalently, the maximal ideal
$\mmm$ is principal.)
\end{definition}

\begin{proposition}[Atiyah--Macdonal 9.2]
Let $(A,\mmm,\kk)$ be a Noetherian local domain of dimension $\dim(A)=1$.
The following are equivalent:
\begin{enumerate}
\item $A$ is a discrete valuation ring;
\item $A$ is integrally closed;
\item $\mmm$ is principal;
\item $\dim(\mmm/\mmm^{2}=1)$;
\item Every nonzero ideal is a power of $\mmm$;
\item There exists an $x\in A$ such that every ideal of $A$ equals
  $(x^{k})$ for some $k$.
\end{enumerate}
\end{proposition}

\begin{example}
One canonical example is $\CC[x]_{(x)}$ so all ideals look like $(x^{n})$,
and $x$ is called a \define{Uniformizer} (a special case of a system
of parameters).
\end{example}

\begin{example}
Another canonical example is $\ZZ_{(p)}$ for some prime number $p\in\ZZ$,
which is the $p$-adic numbers (i.e., the completion of $\ZZ$ at $I$).
\end{example}

\begin{example}
Dedekind domains: $\mathcal{O}_{K}$, and coordinate rings of
nonsingular curves like $\CC[x]$ and $\CC[x,y]/(x^{2}+y^{2}-1)$.
\end{example}

\subsection{Fractional Ideals}

(Atiyah and Macdonald, pp.~96 \textit{et seq.})

\begin{definition}
Let $A$ be an integral domain, $\kk$ its field of fractions. We call
an $A$-submodule $M$ of $\kk$ (so $M\subset\kk$)
a \define{Fractional Ideal} of $A$ if for all $x\in A$, $xM\subset A$.

Furthermore, we call $M$ \define{Invertible} if there exists a module
$N$ such that $MN=A$.
\end{definition}

\begin{example}
We see $A=\ZZ\subset\kk=\QQ$ and $M=\frac{1}{2}\ZZ$ is a fractional ideal.
\end{example}

\begin{example}
Every non-zero principal fractional domain $(u)$ is invertible, its
inverse being $((u)^{-1})$.
\end{example}

\begin{theorem}[Atiyah--Macdonald 9.8]
Let $A$ be an integral domain. Then $A$ is Dedekind if and only if
every nonzero fractional ideal of $A$ is invertible.
\end{theorem}

\begin{definition}
The invertible ideals of a Dedekind domain $\mathcal{O}_{K}$ form a
group, but principal ideals are boring, so we quotient them out. This
gives us the \define{Class Group} of $\mathcal{O}_{K}$ defined as
\begin{equation*}
\Cl(\mathcal{O}_{K})=\{\mbox{fractional ideals}\}/\{\mbox{principal ideals}\}.
\end{equation*}
The \define{Class Number} is the size of the class group.
\end{definition}

